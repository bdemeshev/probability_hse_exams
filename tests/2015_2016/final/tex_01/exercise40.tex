
\begin{question}
Экзамен принимают два преподавателя: Злой и Добрый. Они поставили
следующие оценки:

\begin{center}
\begin{tabular}{lrrrrr} \toprule
Злой   & 2 & 3 & 10 & 8 & 3 \\
Добрый & 6 & 4 & 7  & 8 & \\
\bottomrule
\end{tabular}
\end{center}

Значение статистики критерия Вилкоксона о совпадении распределений
оценок равно
\begin{answerlist}
  \item 20
  \item 20.5
  \item 22.5
  \item 7.5
  \item 19
\end{answerlist}
\end{question}

\begin{solution}
\begin{answerlist}
  \item Неверно
  \item Неверно
  \item Отлично
  \item Неверно
  \item Неверно
\end{answerlist}
\end{solution}

