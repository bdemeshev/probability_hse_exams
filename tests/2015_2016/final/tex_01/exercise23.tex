
\begin{question}
\begin{verbatim}
Случайные величины $X_1$, $X_2$ и $X_3$ независимы и одинаково распределены,
\end{verbatim}

\begin{center}
    \begin{tabular}{lrr} \toprule
    $X_i$ & 3 & 5 \\
    \midrule
    $\P(\cdot)$ & $p$ & $1-p$ \\
    \bottomrule
    \end{tabular}
\end{center}

\begin{verbatim}
По выборке оказалось, что $\bar X = 4.5$. Оценка неизвестного $p$, полученная методом моментов, равна:
\end{verbatim}
\begin{answerlist}
  \item \(1/2\)
  \item \(1/3\)
  \item \(2/3\)
  \item Метод неприменим
  \item \(1/4\)
\end{answerlist}
\end{question}

\begin{solution}
\begin{answerlist}
  \item Неверно
  \item Неверно
  \item Неверно
  \item Неверно
  \item Отлично
\end{answerlist}
\end{solution}

