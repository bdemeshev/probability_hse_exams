\documentclass[12pt]{article}

\usepackage{tikz} % картинки в tikz
\usepackage{microtype} % свешивание пунктуации
\usepackage{array} % для столбцов фиксированной ширины
\usepackage{comment} % для комментирования целых окружений
\usepackage{indentfirst} % отступ в первом параграфе

\usepackage{sectsty} % для центрирования названий частей
\allsectionsfont{\centering}

\usepackage{amsmath, amssymb, amsthm, amsfonts} % куча стандартных математических плюшек

\usepackage[top=2cm, left=1cm, right=1cm, bottom=2cm]{geometry} % размер текста на странице
\usepackage{lastpage} % чтобы узнать номер последней страницы
 
\usepackage{enumitem} % дополнительные плюшки для списков
%  например \begin{enumerate}[resume] позволяет продолжить нумерацию в новом списке

\usepackage{caption} % подписи к рисункам
\usepackage{hyperref} % гиперссылки
\usepackage{multicol} % текст в несколько столбцов


\usepackage{fancyhdr} % весёлые колонтитулы
\pagestyle{fancy}
\lhead{Теория вероятностей и математическая статистика, ВШЭ}
\chead{}
\rhead{2019-09-27}
\lfoot{Вариант $\mu$}
\cfoot{Паниковать запрещается!}
% \rfoot{Тест}
\renewcommand{\headrulewidth}{0.4pt}
\renewcommand{\footrulewidth}{0.4pt}

\usepackage{ifthen} % для написания условий

\usepackage{todonotes} % для вставки в документ заметок о том, что осталось сделать
% \todo{Здесь надо коэффициенты исправить}
% \missingfigure{Здесь будет Последний день Помпеи}
% \listoftodos --- печатает все поставленные \todo'шки


% более красивые таблицы
\usepackage{booktabs}
% заповеди из докупентации:
% 1. Не используйте вертикальные линни
% 2. Не используйте двойные линии
% 3. Единицы измерения - в шапку таблицы
% 4. Не сокращайте .1 вместо 0.1
% 5. Повторяющееся значение повторяйте, а не говорите "то же"


\usepackage{fontspec}
\usepackage{polyglossia}

\setmainlanguage{russian}
\setotherlanguages{english}

% download "Linux Libertine" fonts:
% http://www.linuxlibertine.org/index.php?id=91&L=1
\setmainfont{Linux Libertine O} % or Helvetica, Arial, Cambria
% why do we need \newfontfamily:
% http://tex.stackexchange.com/questions/91507/
\newfontfamily{\cyrillicfonttt}{Linux Libertine O}

\AddEnumerateCounter{\asbuk}{\russian@alph}{щ} % для списков с русскими буквами
\setlist[enumerate, 2]{label=\asbuk*),ref=\asbuk*}

%% эконометрические сокращения
\DeclareMathOperator{\Cov}{Cov}
\DeclareMathOperator{\Corr}{Corr}
\DeclareMathOperator{\Var}{Var}
\DeclareMathOperator{\E}{E}
\def \hb{\hat{\beta}}
\def \hs{\hat{\sigma}}
\def \htheta{\hat{\theta}}
\def \s{\sigma}
\def \hy{\hat{y}}
\def \hY{\hat{Y}}
\def \v1{\vec{1}}
\def \e{\varepsilon}
\def \he{\hat{\e}}
\def \z{z}
\def \hVar{\widehat{\Var}}
\def \hCorr{\widehat{\Corr}}
\def \hCov{\widehat{\Cov}}
\def \cN{\mathcal{N}}
\def \P{\mathbb{P}}


\def \putyourname{\fbox{
    \begin{minipage}{42em}
      Фамилия, имя, номер группы:\vspace*{3ex}\par
      \noindent\dotfill\vspace{2mm}
    \end{minipage}
  }
}

\def \checktable{

	\vspace{5pt}
	Табличка для проверяющих работу:

\vspace{5pt}

	\begin{tabular}{|m{2cm}|m{1cm}|m{1cm}|m{1cm}|m{1cm}|m{1cm}|m{2cm}|}
\toprule
		Тест & 1 &  2 & 3 & 4 & 5 & Итого \\
\midrule
		&  &  & & & & \\
		&  &  & & & & \\
 \bottomrule
\end{tabular}
}



\def \testtable{

	\vspace{5pt}
	Внесите сюда ответы на тест:

\vspace{5pt}

\begin{tabular}{|m{2cm}|m{0.6cm}|m{0.6cm}|m{0.6cm}|m{0.6cm}|m{0.6cm}|m{0.6cm}|m{0.6cm}|m{0.6cm}|m{0.6cm}|m{0.6cm}|}
\toprule
		Вопрос & 1 &  2 & 3 & 4 & 5 & 6 & 7 & 8 & 9 & 10 \\
\midrule
		Ответ &  &  & & & & & & & & \\
 \bottomrule
\end{tabular}
}




% [1][3] 1 = one argument, 3 = value if missing
% эта магия создаёт окружение answerlist
% именно в окружении answerlist записаны варианты ответов в подключаемых exerciseXX
% просто \begin{answerlist} сделает ответы в три столбца
% если ответы длинные, то надо в них руками сделать
% \begin{answerlist}[1] чтобы они шли в один столбец
\newenvironment{answerlist}[1][3]{
\begin{multicols}{#1}

\begin{enumerate}[label=\fbox{\emph{\Alph*}},ref=\emph{\alph*}]
}
{
\item Нет верного ответа.
\end{enumerate}
\end{multicols}
}

% BB: unicol version. don't know why \ifthenelse fails in second part of new-env
\newenvironment{answerlistu}{
\begin{enumerate}[label=\fbox{\emph{\Alph*}},ref=\emph{\alph*}]
}
{
\item Нет верного ответа.
\end{enumerate}
}



\excludecomment{solution} % without solutions

\theoremstyle{definition}
\newtheorem{question}{Вопрос}



\begin{document}

\putyourname


%\testtable

%\checktable


\begin{question}
Если \(F_X(x)\) — функция распределения случайной величины \(X\), то
для любых \(a\) и \(b\)
\begin{answerlist}
  \item \(\P (X = a) = F_X(a)\)
  \item \(\P (X > a) = 1 - F_X(a)\)
  \item \(\P (X < a) = 1 - F_X(a)\)
  \item \(\P (X > a) = F_X(a)\)
  \item \(\P (a<X\le b) = F_X(a)-F_X(b)\)
\end{answerlist}
\end{question}

\begin{solution}
\begin{answerlist}
  \item Bad answer :(
  \item Good answer :)
  \item Bad answer :(
  \item Bad answer :(
  \item Bad answer :(
\end{answerlist}
\end{solution}



\begin{question}
Максимальная ширина 90\%-го симметричного по вероятности доверительного
интервала для доли, построенного по выборке из 64 наблюдений,
приблизительно равна
\begin{answerlist}
  \item 0.206
  \item 0.234
  \item 0.156
  \item 0.102
  \item 0.368
\end{answerlist}
\end{question}

\begin{solution}
\begin{answerlist}
  \item Good answer :)
  \item Bad answer :(
  \item Bad answer :(
  \item Bad answer :(
  \item Bad answer :(
\end{answerlist}
\end{solution}



\begin{question}
По случайной выборке из 100 наблюдений было оценено выборочное среднее
\(\bar{X}=20\)\\
и несмещенная оценка дисперсии \(\hat{\sigma}^2=25\). В рамках проверки
гипотезы \(H_0: \; \mu=15\) против альтернативной гипотезы
\(H_a: \; \mu>15\)\\
можно сделать следующее заключение
\begin{answerlist}
  \item Гипотеза \(H_0\) не отвергается на любом разумном уровне значимости
  \item Гипотеза \(H_0\) отвергается на уровне значимости 10\%, но не на уровне
значимости 5\%
  \item Гипотеза \(H_0\) отвергается на уровне значимости 5\%, но не на уровне
значимости 1\%
  \item Гипотеза \(H_0\) отвергается на любом разумном уровне значимости
  \item Гипотеза \(H_0\) отвергается на уровне значимости 20\%, но не на уровне
значимости 10\%
\end{answerlist}
\end{question}

\begin{solution}
\begin{answerlist}
  \item Неверно
  \item Неверно
  \item Неверно
  \item Отлично
  \item Неверно
\end{answerlist}
\end{solution}



\begin{question}
Ковариационная матрица вектора \(X=(X_1, X_2)\) имеет вид \[
\begin{pmatrix}
10 & 3 \\
3 & 8
\end{pmatrix}.
\] Дисперсия разности элементов вектора, \(\Var(X_1-X_2)\), равняется
\begin{answerlist}
  \item 18
  \item 6
  \item 2
  \item 15
  \item 12
\end{answerlist}
\end{question}

\begin{solution}
\begin{answerlist}
  \item Неверно
  \item Неверно
  \item Неверно
  \item Неверно
  \item Отлично
\end{answerlist}
\end{solution}


\newpage

\begin{question}
Дисперсию случайной величины \(X\) можно найти, зная
\begin{answerlist}
  \item \(F_Y(x)\)
  \item \((\E(X))^2\) и \(\E(X)\)
  \item \(\E(XY)\) и \(\E(Y)\)
  \item \(\Cov(X,Y)\) и \(\Var(Y)\)
  \item \(\E(X^2)\) и \(\E(X)\)
\end{answerlist}
\end{question}

\begin{solution}
\begin{answerlist}
  \item Bad answer :(
  \item Bad answer :(
  \item Bad answer :(
  \item Bad answer :(
  \item Good answer :)
\end{answerlist}
\end{solution}



\begin{question}
При каждом ударе по воротам Месси забивает гол с вероятностью \(0.9\),
независимо от прошлых ударов. Вероятность того, что Месси забьет ровно
три мяча за пять ударов, равна
\begin{answerlist}
  \item \(C_{5}^{3}0.9^{3}0.1^2\)
  \item \(0.9^{4}0.1\)
  \item \(C_{5}^{2}0.1^30.9^5\)
  \item \(0.9^5\)
  \item \(\int_{3}^{5}\frac{1}{10}dx\)
\end{answerlist}
\end{question}

\begin{solution}
\begin{answerlist}
  \item Good answer :)
  \item Bad answer :(
  \item Bad answer :(
  \item Bad answer :(
  \item Bad answer :(
\end{answerlist}
\end{solution}



\begin{question}
Случайным образом выбирается семья с двумя детьми. Событие \(A\) --- в
семье старший ребенок --- мальчик, событие \(B\) --- в семье только один
из детей --- мальчик, событие \(C\) --- в семье хотя бы один из детей
--- мальчик.
\begin{answerlist}
  \item Любые два события из \(A\), \(B\), \(C\) --- зависимы
  \item События \(A\), \(B\), \(C\) --- независимы в совокупности
  \item \(A\) и \(B\) --- независимы, \(A\) и \(C\) --- зависимы, \(B\) и \(C\)
--- зависимы
  \item \(\P(A\cap B\cap C)=\P(A)\P(B)\P(C)\)
  \item События \(A\), \(B\), \(C\) --- независимы попарно, но зависимы в
совокупности
\end{answerlist}
\end{question}

\begin{solution}
\begin{answerlist}
  \item Неверно
  \item Неверно
  \item Отлично
  \item Неверно
  \item Неверно
\end{answerlist}
\end{solution}



\begin{question}
Рассмотрим алгоритм Метрополиса-Гастингса для получения выборки
параметра с апостериорной плотностью пропорциональной \(t^2\).
Предлагаемый переход из \(a\) в \(b\) задаётся правилом, \(b = a + Z\),
где \(Z \sim \cN(0;4)\). Вероятность одобрения перехода из точки \(0.5\)
в точку \(0.3\) равна
\begin{answerlist}
  \item \(0.5\)
  \item \(0.6\)
  \item \(0.64\)
  \item \(1\)
  \item \(0.36\)
\end{answerlist}
\end{question}

\begin{solution}
\begin{answerlist}
  \item Bad answer :(
  \item Bad answer :(
  \item Bad answer :(
  \item Bad answer :(
  \item Good answer :)
\end{answerlist}
\end{solution}



\begin{question}
Cлучайные величины \(\xi_1, \, \ldots, \, \xi_n, \, \ldots\) независимы
и имеют таблицы распределения \[
\begin{tabular}{c|c|c}
$\xi_i$                     & $-1$   & $1$   \\ \cline{1-3}
$\P_{\xi_i}$        & $1/2$       & $1/2$   \\
\end{tabular}
\] Если \(S_n = \xi_1 + \ldots + \xi_n\), то предел
\(\lim\limits_{n \rightarrow \infty}\P\Bigl(\frac{S_n - \E[S_n]}{\sqrt{\Var(S_n)}} > 1\Bigr)\)
равен
\begin{answerlist}
  \item \(\int_{-\infty}^{1}\frac{1}{\sqrt{2\pi}}\,e^{-t^2/2}\,dt\)
  \item \(\int_{1}^{+\infty}\frac{1}{\sqrt{2\pi}}\,e^{-t^2/2}\,dt\)
  \item \(\int_{1}^{+\infty}\frac{1}{2}\,e^{-t/2}\,dt\)
  \item \(\int_{-1}^{1}\frac{1}{\sqrt{2\pi}}\,e^{-t^2/2}\,dt\)
  \item \(0.5\)
\end{answerlist}
\end{question}

\begin{solution}
\begin{answerlist}
  \item Bad answer :(
  \item Good answer :)
  \item Bad answer :(
  \item Bad answer :(
  \item Bad answer :(
\end{answerlist}
\end{solution}



\begin{question}
Требуется проверить гипотезу о равенстве математических ожиданий по двум
нормальным независимым выборкам размером 33 и 16 наблюдений. Истинные
дисперсии по обеим выборкам известны, совпадают и равны 196. Разница
выборочных средних равна 1. Тестовая статистика может быть равна
\begin{answerlist}
  \item \(-1/4\)
  \item \(-1/14\)
  \item \(-1/7\)
  \item \(-1/49\)
  \item \(-1/2\)
\end{answerlist}
\end{question}

\begin{solution}
\begin{answerlist}
  \item Bad answer :(
  \item Bad answer :(
  \item Bad answer :(
  \item Bad answer :(
  \item Good answer :)
\end{answerlist}
\end{solution}


\newpage

\begin{question}
Размер выплаты страховой компанией является неотрицательной случайной
величиной с математическим ожиданием \(10\,000\) рублей. Согласно
неравенству Маркова, вероятность того, что очередная выплата превысит
\(50\,000\) рублей, ограничена сверху числом
\begin{answerlist}
  \item \(0.1359\)
  \item \(0.2\)
  \item \(0.5\)
  \item неравенство Маркова здесь неприменимо
  \item \(0.3413\)
\end{answerlist}
\end{question}

\begin{solution}
\begin{answerlist}
  \item Bad answer :(
  \item Good answer :)
  \item Bad answer :(
  \item Bad answer :(
  \item Bad answer :(
\end{answerlist}
\end{solution}



\begin{question}
Дана реализация выборки: 1, 2, 0. Выборочный начальный момент второго
порядка равен
\begin{answerlist}
  \item \(1\)
  \item \(5/3\)
  \item \(2.5\)
  \item \(1/3\)
  \item \(3\)
\end{answerlist}
\end{question}

\begin{solution}
\begin{answerlist}
  \item Неверно
  \item Ураа!!!
  \item Тоже ересь
  \item Не угадал
  \item Не туда!
\end{answerlist}
\end{solution}



\begin{question}
Величины \(X_1, \, \ldots, \, X_n\) — случайная выборка из
распределения Бернулли с параметром \(p \in (0;\,1)\). Оценка
максимального правдоподобия параметра \(p\) равна \(\bar X\). Оценка
максимального правдоподобия для \(\sqrt{p}\) равна
\begin{answerlist}
  \item \(\frac{\sqrt{\sum_{i=1}^{n}X_i}}{n}\)
  \item \(\sqrt{\sum_{i=1}^{n}X_i}\)
  \item \(\frac{1}{n}\sum_{i=1}^{n}X_i\)
  \item \(\frac{1}{n}\sum_{i=1}^{n}\sqrt{X_i}\)
  \item \(\sqrt{\frac{1}{n}\sum_{i=1}^{n}X_i}\)
\end{answerlist}
\end{question}

\begin{solution}
\begin{answerlist}
  \item Тоже ересь
  \item Не туда!
  \item Неверно
  \item Не угадал
  \item Ураа!!!
\end{answerlist}
\end{solution}



\begin{question}
Совместное распределение пары величин \(X\) и \(Y\) задано таблицей:

\begin{tabular}{@{}c|ccc@{}}
\toprule
       & $Y=-1$ & $Y=0$ & $Y=1$ \\ \midrule
$X=-1$ & $1/4$  & $0$   & $1/4$ \\
$X=1$  & $1/6$  & $1/6$ & $1/6$ \\ \bottomrule
\end{tabular}

\vspace{0.5cm}

Ковариация, \(\Cov(X,Y)\), равна
\begin{answerlist}
  \item \(-0.5\)
  \item \(0\)
  \item \(1\)
  \item \(0.5\)
  \item \(-1\)
\end{answerlist}
\end{question}

\begin{solution}
\begin{answerlist}
  \item Неверно
  \item Отлично
  \item Неверно
  \item Неверно
  \item Неверно
\end{answerlist}
\end{solution}



\begin{question}
Плотность величины \(X\) имеет вид \(f(x)=2x\) при \(0<x<1\) и
\(f(x)=0\) при остальных \(x\). Условная плотность величины \(Y\)
задаётся формулой
\(f_{Y|X}(y|x)=\begin{cases} \frac{1}{x}, \text{ если } 0<y\le x; \\ 0, \text{ иначе } \end{cases}.\)
Совместная плотность величин \(X\) и \(Y\) равна
\begin{answerlist}
  \item \(f(x,y)=\begin{cases} 2, \text{ если } 0<y\le x<1; \\ 0, \text{ иначе} \end{cases}\)
  \item \(f(x,y)=\begin{cases} 2, \text{ если } 0<y<1, 0 < x<1; \\ 0, \text{ иначе} \end{cases}\)
  \item \(f(x,y)=\begin{cases} 1, \text{ если } 0<y\le x<1; \\ 0, \text{ иначе} \end{cases}\)
  \item \(f(x,y)=\begin{cases} 1/x, \text{ если } 0<y<1, 0 < x<1; \\ 0, \text{ иначе} \end{cases}\)
  \item \(f(x,y)=\begin{cases} 1/x, \text{ если } 0<y\le x<1; \\ 0, \text{ иначе} \end{cases}\)
\end{answerlist}
\end{question}

\begin{solution}
\begin{answerlist}
  \item Good answer :)
  \item Bad answer :(
  \item Bad answer :(
  \item Bad answer :(
  \item Bad answer :(
\end{answerlist}
\end{solution}


\newpage

\begin{question}
Пусть \(t_n\) --- случайная величина, распределенная по Стьюденту с
\(n\) степенями свободы. Предел
\(\lim\limits_{n\to\infty}\P\left(t_{n}^2>1\right)\) равен
\begin{answerlist}
  \item \(0.317\)
  \item \(0.841\)
  \item \(0.253\)
  \item \(0.788\)
  \item \(0.102\)
\end{answerlist}
\end{question}

\begin{solution}
\begin{answerlist}
  \item Good answer :)
  \item Bad answer :(
  \item Bad answer :(
  \item Bad answer :(
  \item Bad answer :(
\end{answerlist}
\end{solution}



\begin{question}
Известно, что \(\E(X)=-1\), \(\E(Y)=2\), \(\Var(X)=4\), \(\Var(Y)=9\),
\(\Cov(X,Y)=-3\). Дисперсия \(\Var(2X-Y+1)\) равна
\begin{answerlist}
  \item 31
  \item \(37\)
  \item \(24\)
  \item \(-31\)
  \item 34
\end{answerlist}
\end{question}

\begin{solution}
\begin{answerlist}
  \item Bad answer :(
  \item Good answer :)
  \item Bad answer :(
  \item Bad answer :(
  \item Bad answer :(
\end{answerlist}
\end{solution}



\begin{question}
Правильный кубик подбрасывается 5 раз. Математическое ожидание суммы
выпавших очков равно
\begin{answerlist}
  \item \(21\)
  \item \(18\)
  \item \(3.5\)
  \item \(18.5\)
  \item \(17.5\)
\end{answerlist}
\end{question}

\begin{solution}
\begin{answerlist}
  \item Bad answer :(
  \item Bad answer :(
  \item Bad answer :(
  \item Bad answer :(
  \item Good answer :)
\end{answerlist}
\end{solution}



\begin{question}
Экзамен принимают два преподавателя: Злой и Добрый. Злой поставил оценки
2, 3, 10, 8, 1. А Добрый --- оценки 6, 4, 7, 9. Значение статистики
критерия Вилкоксона о совпадении распределений оценок может быть равно
\begin{answerlist}
  \item \(22\)
  \item \(26\)
  \item \(25\)
  \item \(24\)
  \item \(23\)
\end{answerlist}
\end{question}

\begin{solution}
\begin{answerlist}
  \item Не угадал
  \item Не туда!
  \item Тоже ересь
  \item Неверно
  \item Ураа!!!
\end{answerlist}
\end{solution}



\begin{question}
Величина \(X\) принимает три значения \(1\), \(2\) и \(3\). По случайной
выборке из ста наблюдений оказалось, что \(1\) выпало 40 раз, \(2\) ---
40 раз и \(3\) --- 20 раз. Карл хочет проверить гипотезу о том, что все
три вероятности одинаковые. При верной \(H_0\) критерий Пирсона имеет
распределение
\begin{answerlist}
  \item \(\cN(0;1)\)
  \item \(\chi^2_3\)
  \item \(\chi^2_2\)
  \item \(\chi^2_1\)
  \item \(\chi^2_{99}\)
\end{answerlist}
\end{question}

\begin{solution}
\begin{answerlist}
  \item Bad answer :(
  \item Bad answer :(
  \item Good answer :)
  \item Bad answer :(
  \item Bad answer :(
\end{answerlist}
\end{solution}


\newpage

\begin{question}
Совместное распределение пары величин \(X\) и \(Y\) задано таблицей:

\begin{tabular}{@{}c|ccc@{}}
\toprule
       & $Y=-1$ & $Y=0$ & $Y=1$ \\ \midrule
$X=-1$ & $1/4$  & $0$   & $1/4$ \\
$X=1$  & $1/6$  & $1/6$ & $1/6$ \\ \bottomrule
\end{tabular}

\vspace{0.5cm}

Дисперсия случайной величины \(Y\) равна
\begin{answerlist}
  \item \(5/12\)
  \item \(1/3\)
  \item \(1/2\)
  \item \(12/5\)
  \item \(5/6\)
\end{answerlist}
\end{question}

\begin{solution}
\begin{answerlist}
  \item Неверно
  \item Неверно
  \item Неверно
  \item Неверно
  \item Отлично
\end{answerlist}
\end{solution}



\begin{question}
Математическое ожидание случайной величины \(X\) при условии \(Y=0\)
равно
\begin{answerlist}
  \item \(1/6\)
  \item \(0\)
  \item \(-1\)
  \item \(1\)
  \item \(1/3\)
\end{answerlist}
\end{question}

\begin{solution}
\begin{answerlist}
  \item Bad answer :(
  \item Bad answer :(
  \item Bad answer :(
  \item Good answer :)
  \item Bad answer :(
\end{answerlist}
\end{solution}



\begin{question}
Выборочная функция распределения, построенная по выборке объёма \(n\) из
равномерного распределения на отрезке \([0,2]\), в точке \(х=0.2\) при
\(n\) стремящимся к бесконечности стремится по вероятности к
\begin{answerlist}
  \item \(1\)
  \item \(0.5\)
  \item \(0.1\)
  \item \(0.2\)
  \item \(0\)
\end{answerlist}
\end{question}

\begin{solution}
\begin{answerlist}
  \item Не туда!
  \item Тоже ересь
  \item Ураа!!!
  \item Неверно
  \item Не угадал
\end{answerlist}
\end{solution}



\begin{question}
По случайной выборке из 200 наблюдений было оценено выборочное среднее
\(\bar{X} = 25\) и несмещённая оценка дисперсии \(\hat{\sigma}^2 = 25\).
В рамках проверки гипотезы \(H_0: \mu = 20\) против \(H_a: \mu > 20\)
можно сделать вывод, что гипотеза \(H_0\)
\begin{answerlist}
  \item отвергается при \(\alpha = 0.01\), не отвергается при \(\alpha = 0.05\)
  \item отвергается при любом разумном значении \(\alpha\)
  \item не отвергается при любом разумном значении \(\alpha\)
  \item отвергается при \(\alpha = 0.05\), не отвергается при \(\alpha = 0.01\)
  \item Гипотезу невозможно проверить
\end{answerlist}
\end{question}

\begin{solution}
\begin{answerlist}
  \item Неверно
  \item Отлично
  \item Неверно
  \item Неверно
  \item Неверно
\end{answerlist}
\end{solution}



\begin{question}
Если величина \(\hat\theta\) имеет нормальное распределение
\(\cN(2;0.01^2)\), то, согласно дельта-методу, \(\hat\theta^2\) имеет
примерно нормальное распределение
\begin{answerlist}
  \item \(\cN(2;4\cdot 0.01^2)\)
  \item \(\cN(4;8\cdot 0.01^2)\)
  \item \(\cN(4;16\cdot 0.01^2)\)
  \item \(\cN(4;2\cdot 0.01^2)\)
  \item \(\cN(4;4\cdot 0.01^2)\)
\end{answerlist}
\end{question}

\begin{solution}
\begin{answerlist}
  \item Неверно
  \item Неверно
  \item Отлично
  \item Неверно
  \item Неверно
\end{answerlist}
\end{solution}


\newpage

\begin{question}
Ковариация случайных величин \(X\) и \(Y\) равна:
\begin{answerlist}
  \item \(-2/3\)
  \item \(-1/3\)
  \item \(0\)
  \item \(2/3\)
  \item \(1/3\)
\end{answerlist}
\end{question}

\begin{solution}
\begin{answerlist}
  \item Bad answer :(
  \item Good answer :)
  \item Bad answer :(
  \item Bad answer :(
  \item Bad answer :(
\end{answerlist}
\end{solution}



\begin{question}
Сумма независимых абсолютно непрерывной и дискретной случайных величин
имеет распределение
\begin{answerlist}
  \item вырожденное
  \item абсолютно непрерывное
  \item дискретное
  \item сингулярное
  \item нормальное
\end{answerlist}
\end{question}

\begin{solution}
\begin{answerlist}
  \item Bad answer :(
  \item Good answer :)
  \item Bad answer :(
  \item Bad answer :(
  \item Bad answer :(
\end{answerlist}
\end{solution}



\begin{question}
Случайные величины \(X\) и \(Y\) имеют совместное нормальное
распределение, а \(x\in[1,2]\) --- константа. При любом \(x\) верно
неравенство
\begin{answerlist}
  \item \(\Corr(X,Y)\ne0\)
  \item \(\Var(Y|X=x)\geq \Var(Y)\)
  \item \(\E(Y|X=x)\geq \E(Y)\)
  \item \(\E(Y|X=x)\leq \E(Y)\)
  \item \(\Var(Y|X=x)\leq \Var(Y)\)
\end{answerlist}
\end{question}

\begin{solution}
\begin{answerlist}
  \item Bad answer :(
  \item Bad answer :(
  \item Bad answer :(
  \item Bad answer :(
  \item Good answer :)
\end{answerlist}
\end{solution}



\begin{question}
Величина \(X\) принимает три значения \(1\), \(2\) и \(3\). По случайной
выборке из ста наблюдений оказалось, что \(1\) выпало 40 раз, \(2\) ---
40 раз и \(3\) --- 20 раз. Андрей Николаевич хочет проверить гипотезу о
том, что все три вероятности одинаковые. Значение критерия согласия
Колмогорова равно
\begin{answerlist}
  \item \(3/5\)
  \item \(2/15\)
  \item \(3/4\)
  \item \(2/5\)
  \item \(1/4\)
\end{answerlist}
\end{question}

\begin{solution}
\begin{answerlist}
  \item Bad answer :(
  \item Good answer :)
  \item Bad answer :(
  \item Bad answer :(
  \item Bad answer :(
\end{answerlist}
\end{solution}



\begin{question}
Истинное значение параметра \(\theta\) равно \(2\), в случайной выборке
100 наблюдений, а информация Фишера о параметре \(\theta\), заключенная
в одном наблюдении равна \(I_1(\theta) = 9\). Распределение оценки
максимального правдоподобия \(\hat{\theta}\) похоже на
\begin{answerlist}
  \item \(\cN(2, \, 9)\)
  \item \(\cN(2, \, 1/30)\)
  \item \(\cN(2, \, 1/3)\)
  \item \(\cN(2, \, 1/9)\)
  \item \(\cN(2, \, 1/900)\)
\end{answerlist}
\end{question}

\begin{solution}
\begin{answerlist}
  \item Bad answer :(
  \item Bad answer :(
  \item Bad answer :(
  \item Bad answer :(
  \item Good answer :)
\end{answerlist}
\end{solution}


 






\newpage
\lfoot{Вариант $\kappa$}
\setcounter{question}{0}


\putyourname

%\testtable

%\checktable


\begin{question}
Если \(F_X(x)\) — функция распределения случайной величины \(X\), то
для любых \(a\) и \(b\)
\begin{answerlist}
  \item \(\P (X = a) = F_X(a)\)
  \item \(\P (X > a) = 1 - F_X(a)\)
  \item \(\P (X < a) = 1 - F_X(a)\)
  \item \(\P (X > a) = F_X(a)\)
  \item \(\P (a<X\le b) = F_X(a)-F_X(b)\)
\end{answerlist}
\end{question}

\begin{solution}
\begin{answerlist}
  \item Bad answer :(
  \item Good answer :)
  \item Bad answer :(
  \item Bad answer :(
  \item Bad answer :(
\end{answerlist}
\end{solution}



\begin{question}
Максимальная ширина 90\%-го симметричного по вероятности доверительного
интервала для доли, построенного по выборке из 64 наблюдений,
приблизительно равна
\begin{answerlist}
  \item 0.206
  \item 0.234
  \item 0.156
  \item 0.102
  \item 0.368
\end{answerlist}
\end{question}

\begin{solution}
\begin{answerlist}
  \item Good answer :)
  \item Bad answer :(
  \item Bad answer :(
  \item Bad answer :(
  \item Bad answer :(
\end{answerlist}
\end{solution}



\begin{question}
По случайной выборке из 100 наблюдений было оценено выборочное среднее
\(\bar{X}=20\)\\
и несмещенная оценка дисперсии \(\hat{\sigma}^2=25\). В рамках проверки
гипотезы \(H_0: \; \mu=15\) против альтернативной гипотезы
\(H_a: \; \mu>15\)\\
можно сделать следующее заключение
\begin{answerlist}
  \item Гипотеза \(H_0\) не отвергается на любом разумном уровне значимости
  \item Гипотеза \(H_0\) отвергается на уровне значимости 10\%, но не на уровне
значимости 5\%
  \item Гипотеза \(H_0\) отвергается на уровне значимости 5\%, но не на уровне
значимости 1\%
  \item Гипотеза \(H_0\) отвергается на любом разумном уровне значимости
  \item Гипотеза \(H_0\) отвергается на уровне значимости 20\%, но не на уровне
значимости 10\%
\end{answerlist}
\end{question}

\begin{solution}
\begin{answerlist}
  \item Неверно
  \item Неверно
  \item Неверно
  \item Отлично
  \item Неверно
\end{answerlist}
\end{solution}



\begin{question}
Ковариационная матрица вектора \(X=(X_1, X_2)\) имеет вид \[
\begin{pmatrix}
10 & 3 \\
3 & 8
\end{pmatrix}.
\] Дисперсия разности элементов вектора, \(\Var(X_1-X_2)\), равняется
\begin{answerlist}
  \item 18
  \item 6
  \item 2
  \item 15
  \item 12
\end{answerlist}
\end{question}

\begin{solution}
\begin{answerlist}
  \item Неверно
  \item Неверно
  \item Неверно
  \item Неверно
  \item Отлично
\end{answerlist}
\end{solution}



\begin{question}
Дисперсию случайной величины \(X\) можно найти, зная
\begin{answerlist}
  \item \(F_Y(x)\)
  \item \((\E(X))^2\) и \(\E(X)\)
  \item \(\E(XY)\) и \(\E(Y)\)
  \item \(\Cov(X,Y)\) и \(\Var(Y)\)
  \item \(\E(X^2)\) и \(\E(X)\)
\end{answerlist}
\end{question}

\begin{solution}
\begin{answerlist}
  \item Bad answer :(
  \item Bad answer :(
  \item Bad answer :(
  \item Bad answer :(
  \item Good answer :)
\end{answerlist}
\end{solution}


\newpage

\begin{question}
При каждом ударе по воротам Месси забивает гол с вероятностью \(0.9\),
независимо от прошлых ударов. Вероятность того, что Месси забьет ровно
три мяча за пять ударов, равна
\begin{answerlist}
  \item \(C_{5}^{3}0.9^{3}0.1^2\)
  \item \(0.9^{4}0.1\)
  \item \(C_{5}^{2}0.1^30.9^5\)
  \item \(0.9^5\)
  \item \(\int_{3}^{5}\frac{1}{10}dx\)
\end{answerlist}
\end{question}

\begin{solution}
\begin{answerlist}
  \item Good answer :)
  \item Bad answer :(
  \item Bad answer :(
  \item Bad answer :(
  \item Bad answer :(
\end{answerlist}
\end{solution}



\begin{question}
Случайным образом выбирается семья с двумя детьми. Событие \(A\) --- в
семье старший ребенок --- мальчик, событие \(B\) --- в семье только один
из детей --- мальчик, событие \(C\) --- в семье хотя бы один из детей
--- мальчик.
\begin{answerlist}
  \item Любые два события из \(A\), \(B\), \(C\) --- зависимы
  \item События \(A\), \(B\), \(C\) --- независимы в совокупности
  \item \(A\) и \(B\) --- независимы, \(A\) и \(C\) --- зависимы, \(B\) и \(C\)
--- зависимы
  \item \(\P(A\cap B\cap C)=\P(A)\P(B)\P(C)\)
  \item События \(A\), \(B\), \(C\) --- независимы попарно, но зависимы в
совокупности
\end{answerlist}
\end{question}

\begin{solution}
\begin{answerlist}
  \item Неверно
  \item Неверно
  \item Отлично
  \item Неверно
  \item Неверно
\end{answerlist}
\end{solution}



\begin{question}
Рассмотрим алгоритм Метрополиса-Гастингса для получения выборки
параметра с апостериорной плотностью пропорциональной \(t^2\).
Предлагаемый переход из \(a\) в \(b\) задаётся правилом, \(b = a + Z\),
где \(Z \sim \cN(0;4)\). Вероятность одобрения перехода из точки \(0.5\)
в точку \(0.3\) равна
\begin{answerlist}
  \item \(0.5\)
  \item \(0.6\)
  \item \(0.64\)
  \item \(1\)
  \item \(0.36\)
\end{answerlist}
\end{question}

\begin{solution}
\begin{answerlist}
  \item Bad answer :(
  \item Bad answer :(
  \item Bad answer :(
  \item Bad answer :(
  \item Good answer :)
\end{answerlist}
\end{solution}



\begin{question}
Cлучайные величины \(\xi_1, \, \ldots, \, \xi_n, \, \ldots\) независимы
и имеют таблицы распределения \[
\begin{tabular}{c|c|c}
$\xi_i$                     & $-1$   & $1$   \\ \cline{1-3}
$\P_{\xi_i}$        & $1/2$       & $1/2$   \\
\end{tabular}
\] Если \(S_n = \xi_1 + \ldots + \xi_n\), то предел
\(\lim\limits_{n \rightarrow \infty}\P\Bigl(\frac{S_n - \E[S_n]}{\sqrt{\Var(S_n)}} > 1\Bigr)\)
равен
\begin{answerlist}
  \item \(\int_{-\infty}^{1}\frac{1}{\sqrt{2\pi}}\,e^{-t^2/2}\,dt\)
  \item \(\int_{1}^{+\infty}\frac{1}{\sqrt{2\pi}}\,e^{-t^2/2}\,dt\)
  \item \(\int_{1}^{+\infty}\frac{1}{2}\,e^{-t/2}\,dt\)
  \item \(\int_{-1}^{1}\frac{1}{\sqrt{2\pi}}\,e^{-t^2/2}\,dt\)
  \item \(0.5\)
\end{answerlist}
\end{question}

\begin{solution}
\begin{answerlist}
  \item Bad answer :(
  \item Good answer :)
  \item Bad answer :(
  \item Bad answer :(
  \item Bad answer :(
\end{answerlist}
\end{solution}



\begin{question}
Требуется проверить гипотезу о равенстве математических ожиданий по двум
нормальным независимым выборкам размером 33 и 16 наблюдений. Истинные
дисперсии по обеим выборкам известны, совпадают и равны 196. Разница
выборочных средних равна 1. Тестовая статистика может быть равна
\begin{answerlist}
  \item \(-1/4\)
  \item \(-1/14\)
  \item \(-1/7\)
  \item \(-1/49\)
  \item \(-1/2\)
\end{answerlist}
\end{question}

\begin{solution}
\begin{answerlist}
  \item Bad answer :(
  \item Bad answer :(
  \item Bad answer :(
  \item Bad answer :(
  \item Good answer :)
\end{answerlist}
\end{solution}


\newpage

\begin{question}
Размер выплаты страховой компанией является неотрицательной случайной
величиной с математическим ожиданием \(10\,000\) рублей. Согласно
неравенству Маркова, вероятность того, что очередная выплата превысит
\(50\,000\) рублей, ограничена сверху числом
\begin{answerlist}
  \item \(0.1359\)
  \item \(0.2\)
  \item \(0.5\)
  \item неравенство Маркова здесь неприменимо
  \item \(0.3413\)
\end{answerlist}
\end{question}

\begin{solution}
\begin{answerlist}
  \item Bad answer :(
  \item Good answer :)
  \item Bad answer :(
  \item Bad answer :(
  \item Bad answer :(
\end{answerlist}
\end{solution}



\begin{question}
Дана реализация выборки: 1, 2, 0. Выборочный начальный момент второго
порядка равен
\begin{answerlist}
  \item \(1\)
  \item \(5/3\)
  \item \(2.5\)
  \item \(1/3\)
  \item \(3\)
\end{answerlist}
\end{question}

\begin{solution}
\begin{answerlist}
  \item Неверно
  \item Ураа!!!
  \item Тоже ересь
  \item Не угадал
  \item Не туда!
\end{answerlist}
\end{solution}



\begin{question}
Величины \(X_1, \, \ldots, \, X_n\) — случайная выборка из
распределения Бернулли с параметром \(p \in (0;\,1)\). Оценка
максимального правдоподобия параметра \(p\) равна \(\bar X\). Оценка
максимального правдоподобия для \(\sqrt{p}\) равна
\begin{answerlist}
  \item \(\frac{\sqrt{\sum_{i=1}^{n}X_i}}{n}\)
  \item \(\sqrt{\sum_{i=1}^{n}X_i}\)
  \item \(\frac{1}{n}\sum_{i=1}^{n}X_i\)
  \item \(\frac{1}{n}\sum_{i=1}^{n}\sqrt{X_i}\)
  \item \(\sqrt{\frac{1}{n}\sum_{i=1}^{n}X_i}\)
\end{answerlist}
\end{question}

\begin{solution}
\begin{answerlist}
  \item Тоже ересь
  \item Не туда!
  \item Неверно
  \item Не угадал
  \item Ураа!!!
\end{answerlist}
\end{solution}



\begin{question}
Совместное распределение пары величин \(X\) и \(Y\) задано таблицей:

\begin{tabular}{@{}c|ccc@{}}
\toprule
       & $Y=-1$ & $Y=0$ & $Y=1$ \\ \midrule
$X=-1$ & $1/4$  & $0$   & $1/4$ \\
$X=1$  & $1/6$  & $1/6$ & $1/6$ \\ \bottomrule
\end{tabular}

\vspace{0.5cm}

Ковариация, \(\Cov(X,Y)\), равна
\begin{answerlist}
  \item \(-0.5\)
  \item \(0\)
  \item \(1\)
  \item \(0.5\)
  \item \(-1\)
\end{answerlist}
\end{question}

\begin{solution}
\begin{answerlist}
  \item Неверно
  \item Отлично
  \item Неверно
  \item Неверно
  \item Неверно
\end{answerlist}
\end{solution}



\begin{question}
Плотность величины \(X\) имеет вид \(f(x)=2x\) при \(0<x<1\) и
\(f(x)=0\) при остальных \(x\). Условная плотность величины \(Y\)
задаётся формулой
\(f_{Y|X}(y|x)=\begin{cases} \frac{1}{x}, \text{ если } 0<y\le x; \\ 0, \text{ иначе } \end{cases}.\)
Совместная плотность величин \(X\) и \(Y\) равна
\begin{answerlist}
  \item \(f(x,y)=\begin{cases} 2, \text{ если } 0<y\le x<1; \\ 0, \text{ иначе} \end{cases}\)
  \item \(f(x,y)=\begin{cases} 2, \text{ если } 0<y<1, 0 < x<1; \\ 0, \text{ иначе} \end{cases}\)
  \item \(f(x,y)=\begin{cases} 1, \text{ если } 0<y\le x<1; \\ 0, \text{ иначе} \end{cases}\)
  \item \(f(x,y)=\begin{cases} 1/x, \text{ если } 0<y<1, 0 < x<1; \\ 0, \text{ иначе} \end{cases}\)
  \item \(f(x,y)=\begin{cases} 1/x, \text{ если } 0<y\le x<1; \\ 0, \text{ иначе} \end{cases}\)
\end{answerlist}
\end{question}

\begin{solution}
\begin{answerlist}
  \item Good answer :)
  \item Bad answer :(
  \item Bad answer :(
  \item Bad answer :(
  \item Bad answer :(
\end{answerlist}
\end{solution}


\newpage

\begin{question}
Пусть \(t_n\) --- случайная величина, распределенная по Стьюденту с
\(n\) степенями свободы. Предел
\(\lim\limits_{n\to\infty}\P\left(t_{n}^2>1\right)\) равен
\begin{answerlist}
  \item \(0.317\)
  \item \(0.841\)
  \item \(0.253\)
  \item \(0.788\)
  \item \(0.102\)
\end{answerlist}
\end{question}

\begin{solution}
\begin{answerlist}
  \item Good answer :)
  \item Bad answer :(
  \item Bad answer :(
  \item Bad answer :(
  \item Bad answer :(
\end{answerlist}
\end{solution}



\begin{question}
Известно, что \(\E(X)=-1\), \(\E(Y)=2\), \(\Var(X)=4\), \(\Var(Y)=9\),
\(\Cov(X,Y)=-3\). Дисперсия \(\Var(2X-Y+1)\) равна
\begin{answerlist}
  \item 31
  \item \(37\)
  \item \(24\)
  \item \(-31\)
  \item 34
\end{answerlist}
\end{question}

\begin{solution}
\begin{answerlist}
  \item Bad answer :(
  \item Good answer :)
  \item Bad answer :(
  \item Bad answer :(
  \item Bad answer :(
\end{answerlist}
\end{solution}



\begin{question}
Правильный кубик подбрасывается 5 раз. Математическое ожидание суммы
выпавших очков равно
\begin{answerlist}
  \item \(21\)
  \item \(18\)
  \item \(3.5\)
  \item \(18.5\)
  \item \(17.5\)
\end{answerlist}
\end{question}

\begin{solution}
\begin{answerlist}
  \item Bad answer :(
  \item Bad answer :(
  \item Bad answer :(
  \item Bad answer :(
  \item Good answer :)
\end{answerlist}
\end{solution}



\begin{question}
Экзамен принимают два преподавателя: Злой и Добрый. Злой поставил оценки
2, 3, 10, 8, 1. А Добрый --- оценки 6, 4, 7, 9. Значение статистики
критерия Вилкоксона о совпадении распределений оценок может быть равно
\begin{answerlist}
  \item \(22\)
  \item \(26\)
  \item \(25\)
  \item \(24\)
  \item \(23\)
\end{answerlist}
\end{question}

\begin{solution}
\begin{answerlist}
  \item Не угадал
  \item Не туда!
  \item Тоже ересь
  \item Неверно
  \item Ураа!!!
\end{answerlist}
\end{solution}



\begin{question}
Величина \(X\) принимает три значения \(1\), \(2\) и \(3\). По случайной
выборке из ста наблюдений оказалось, что \(1\) выпало 40 раз, \(2\) ---
40 раз и \(3\) --- 20 раз. Карл хочет проверить гипотезу о том, что все
три вероятности одинаковые. При верной \(H_0\) критерий Пирсона имеет
распределение
\begin{answerlist}
  \item \(\cN(0;1)\)
  \item \(\chi^2_3\)
  \item \(\chi^2_2\)
  \item \(\chi^2_1\)
  \item \(\chi^2_{99}\)
\end{answerlist}
\end{question}

\begin{solution}
\begin{answerlist}
  \item Bad answer :(
  \item Bad answer :(
  \item Good answer :)
  \item Bad answer :(
  \item Bad answer :(
\end{answerlist}
\end{solution}


\newpage

\begin{question}
Совместное распределение пары величин \(X\) и \(Y\) задано таблицей:

\begin{tabular}{@{}c|ccc@{}}
\toprule
       & $Y=-1$ & $Y=0$ & $Y=1$ \\ \midrule
$X=-1$ & $1/4$  & $0$   & $1/4$ \\
$X=1$  & $1/6$  & $1/6$ & $1/6$ \\ \bottomrule
\end{tabular}

\vspace{0.5cm}

Дисперсия случайной величины \(Y\) равна
\begin{answerlist}
  \item \(5/12\)
  \item \(1/3\)
  \item \(1/2\)
  \item \(12/5\)
  \item \(5/6\)
\end{answerlist}
\end{question}

\begin{solution}
\begin{answerlist}
  \item Неверно
  \item Неверно
  \item Неверно
  \item Неверно
  \item Отлично
\end{answerlist}
\end{solution}



\begin{question}
Математическое ожидание случайной величины \(X\) при условии \(Y=0\)
равно
\begin{answerlist}
  \item \(1/6\)
  \item \(0\)
  \item \(-1\)
  \item \(1\)
  \item \(1/3\)
\end{answerlist}
\end{question}

\begin{solution}
\begin{answerlist}
  \item Bad answer :(
  \item Bad answer :(
  \item Bad answer :(
  \item Good answer :)
  \item Bad answer :(
\end{answerlist}
\end{solution}



\begin{question}
Выборочная функция распределения, построенная по выборке объёма \(n\) из
равномерного распределения на отрезке \([0,2]\), в точке \(х=0.2\) при
\(n\) стремящимся к бесконечности стремится по вероятности к
\begin{answerlist}
  \item \(1\)
  \item \(0.5\)
  \item \(0.1\)
  \item \(0.2\)
  \item \(0\)
\end{answerlist}
\end{question}

\begin{solution}
\begin{answerlist}
  \item Не туда!
  \item Тоже ересь
  \item Ураа!!!
  \item Неверно
  \item Не угадал
\end{answerlist}
\end{solution}



\begin{question}
По случайной выборке из 200 наблюдений было оценено выборочное среднее
\(\bar{X} = 25\) и несмещённая оценка дисперсии \(\hat{\sigma}^2 = 25\).
В рамках проверки гипотезы \(H_0: \mu = 20\) против \(H_a: \mu > 20\)
можно сделать вывод, что гипотеза \(H_0\)
\begin{answerlist}
  \item отвергается при \(\alpha = 0.01\), не отвергается при \(\alpha = 0.05\)
  \item отвергается при любом разумном значении \(\alpha\)
  \item не отвергается при любом разумном значении \(\alpha\)
  \item отвергается при \(\alpha = 0.05\), не отвергается при \(\alpha = 0.01\)
  \item Гипотезу невозможно проверить
\end{answerlist}
\end{question}

\begin{solution}
\begin{answerlist}
  \item Неверно
  \item Отлично
  \item Неверно
  \item Неверно
  \item Неверно
\end{answerlist}
\end{solution}



\begin{question}
Если величина \(\hat\theta\) имеет нормальное распределение
\(\cN(2;0.01^2)\), то, согласно дельта-методу, \(\hat\theta^2\) имеет
примерно нормальное распределение
\begin{answerlist}
  \item \(\cN(2;4\cdot 0.01^2)\)
  \item \(\cN(4;8\cdot 0.01^2)\)
  \item \(\cN(4;16\cdot 0.01^2)\)
  \item \(\cN(4;2\cdot 0.01^2)\)
  \item \(\cN(4;4\cdot 0.01^2)\)
\end{answerlist}
\end{question}

\begin{solution}
\begin{answerlist}
  \item Неверно
  \item Неверно
  \item Отлично
  \item Неверно
  \item Неверно
\end{answerlist}
\end{solution}


\newpage

\begin{question}
Ковариация случайных величин \(X\) и \(Y\) равна:
\begin{answerlist}
  \item \(-2/3\)
  \item \(-1/3\)
  \item \(0\)
  \item \(2/3\)
  \item \(1/3\)
\end{answerlist}
\end{question}

\begin{solution}
\begin{answerlist}
  \item Bad answer :(
  \item Good answer :)
  \item Bad answer :(
  \item Bad answer :(
  \item Bad answer :(
\end{answerlist}
\end{solution}



\begin{question}
Сумма независимых абсолютно непрерывной и дискретной случайных величин
имеет распределение
\begin{answerlist}
  \item вырожденное
  \item абсолютно непрерывное
  \item дискретное
  \item сингулярное
  \item нормальное
\end{answerlist}
\end{question}

\begin{solution}
\begin{answerlist}
  \item Bad answer :(
  \item Good answer :)
  \item Bad answer :(
  \item Bad answer :(
  \item Bad answer :(
\end{answerlist}
\end{solution}



\begin{question}
Случайные величины \(X\) и \(Y\) имеют совместное нормальное
распределение, а \(x\in[1,2]\) --- константа. При любом \(x\) верно
неравенство
\begin{answerlist}
  \item \(\Corr(X,Y)\ne0\)
  \item \(\Var(Y|X=x)\geq \Var(Y)\)
  \item \(\E(Y|X=x)\geq \E(Y)\)
  \item \(\E(Y|X=x)\leq \E(Y)\)
  \item \(\Var(Y|X=x)\leq \Var(Y)\)
\end{answerlist}
\end{question}

\begin{solution}
\begin{answerlist}
  \item Bad answer :(
  \item Bad answer :(
  \item Bad answer :(
  \item Bad answer :(
  \item Good answer :)
\end{answerlist}
\end{solution}



\begin{question}
Величина \(X\) принимает три значения \(1\), \(2\) и \(3\). По случайной
выборке из ста наблюдений оказалось, что \(1\) выпало 40 раз, \(2\) ---
40 раз и \(3\) --- 20 раз. Андрей Николаевич хочет проверить гипотезу о
том, что все три вероятности одинаковые. Значение критерия согласия
Колмогорова равно
\begin{answerlist}
  \item \(3/5\)
  \item \(2/15\)
  \item \(3/4\)
  \item \(2/5\)
  \item \(1/4\)
\end{answerlist}
\end{question}

\begin{solution}
\begin{answerlist}
  \item Bad answer :(
  \item Good answer :)
  \item Bad answer :(
  \item Bad answer :(
  \item Bad answer :(
\end{answerlist}
\end{solution}



\begin{question}
Истинное значение параметра \(\theta\) равно \(2\), в случайной выборке
100 наблюдений, а информация Фишера о параметре \(\theta\), заключенная
в одном наблюдении равна \(I_1(\theta) = 9\). Распределение оценки
максимального правдоподобия \(\hat{\theta}\) похоже на
\begin{answerlist}
  \item \(\cN(2, \, 9)\)
  \item \(\cN(2, \, 1/30)\)
  \item \(\cN(2, \, 1/3)\)
  \item \(\cN(2, \, 1/9)\)
  \item \(\cN(2, \, 1/900)\)
\end{answerlist}
\end{question}

\begin{solution}
\begin{answerlist}
  \item Bad answer :(
  \item Bad answer :(
  \item Bad answer :(
  \item Bad answer :(
  \item Good answer :)
\end{answerlist}
\end{solution}








 

\end{document}
