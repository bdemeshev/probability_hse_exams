
\begin{question}
Случайные величины \(X_1\), \ldots, \(X_m\) — случайная выборка из
нормального распределения. Величины \(Y_1\), \ldots, \(Y_n\) —
независимая случайная выборка из нормального распределения. Для
построения доверительного интервала для отношения дисперсий можно
использовать статистику с распределением
\begin{answerlist}
  \item \(F_{m,n-2}\)
  \item \(F_{m-1, n-1}\)
  \item \(\chi^2_{m+n-2}\)
  \item \(F_{m+1,n+1}\)
  \item \(t_{m+n-2}\)
\end{answerlist}
\end{question}

\begin{solution}
\begin{answerlist}
  \item Тоже ересь
  \item Ураа!!!
  \item Не туда!
  \item Неверно
  \item Не угадал
\end{answerlist}
\end{solution}

