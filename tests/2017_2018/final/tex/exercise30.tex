
\begin{question}
Исследовательница Глафира считает, что любовь к энергетическим напиткам
и успешность сдачи экзамена по математической статистике должны быть
как-то связаны. Опросив 200 своих однокурсников, она получила следующие
результаты:

\begin{tabular}{ccc}
\toprule
 & пьёт энергетик & не пьёт энергетик \\
Успешно сдал & 20 & 120 \\
Завалил  & 40 & 20 \\
\bottomrule
\end{tabular}

Статистика \(\chi^2\) Пирсона для проверки независимости признаков с
округлением до целых равна
\begin{answerlist}
  \item \(70\)
  \item \(55\)
  \item \(65\)
  \item \(35\)
  \item \(45\)
\end{answerlist}
\end{question}

\begin{solution}
\begin{answerlist}
  \item Не туда!
  \item Ураа!!!
  \item Неверно
  \item Тоже ересь
  \item Не угадал
\end{answerlist}
\end{solution}

