
\begin{question}
У пяти случайно выбранных студентов первого потока результаты за
контрольную по статистике оказались равны 82, 47, 20, 43 и 73. У четырёх
случайно выбранных студентов второго потока --- 68, 83, 60 и 52.
Вычислите статистику Вилкоксона для меньшей выборки и проверьте гипотезу
\(H_0\) об однородности результатов двух потоков. Критические значения
статистики Вилкоксона равны \(T_L=12\) и \(T_R=28\).
\begin{answerlist}
  \item 65.75, \(H_0\) отвергается
  \item 12.75, \(H_0\) не отвергается
  \item 53, \(H_0\) отвергается
  \item 20, \(H_0\) не отвергается
  \item 24, \(H_0\) не отвергается
\end{answerlist}
\end{question}

\begin{solution}
\begin{answerlist}
  \item Неверно
  \item Неверно
  \item Неверно
  \item Неверно
  \item Отлично
\end{answerlist}
\end{solution}

