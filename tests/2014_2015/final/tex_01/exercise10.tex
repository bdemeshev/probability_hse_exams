
\begin{question}
Критерий знаков проверяет нулевую гипотезу
\begin{answerlist}
  \item о равенстве математических ожиданий двух нормально распределенных
случайных величин
  \item о равенстве \(1/2\) вероятности того, что случайная величина \(X\)
окажется больше случайной величины \(Y\), если альтернативная гипотеза
записана как \(\mu_X>\mu_Y\)
  \item о равенстве нулю вероятности того, что случайная величина \(X\) окажется
больше случайной величины \(Y\), если альтернативная гипотеза записана
как \(\mu_X>\mu_Y\)
  \item о совпадении функции распределения случайной величины с заданной
теоретической функцией распределения
  \item о равенстве нулю вероятности того, что случайная величина \(X\) окажется
больше случайной величины \(Y\), если альтернативная гипотеза записана
как \(\mu_X>\mu_Y\)
\end{answerlist}
\end{question}

\begin{solution}
\begin{answerlist}
  \item Неверно
  \item Неверно
  \item Отлично
  \item Неверно
  \item Неверно
\end{answerlist}
\end{solution}

