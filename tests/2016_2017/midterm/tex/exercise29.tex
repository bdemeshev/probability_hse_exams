
\begin{question}
В школе три девятых класса: 9А, 9Б и 9В. В 9А классе — 50\% отличники,
в 9Б — 30\%, в 9В — 40\%. Если сначала равновероятно выбрать один из
трёх классов, а затем внутри класса равновероятно выбрать школьника, то
вероятность выбрать отличника равна
\begin{answerlist}
  \item \((3+4+5)/3\)
  \item \(0.3\)
  \item \(0.5\)
  \item \(0.27\)
  \item \(0.4\)
\end{answerlist}
\end{question}

\begin{solution}
\begin{answerlist}
  \item Bad answer :(
  \item Bad answer :(
  \item Bad answer :(
  \item Bad answer :(
  \item Good answer :)
\end{answerlist}
\end{solution}

