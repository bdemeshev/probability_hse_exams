
\begin{question}
Трёх случайно выбранных студентов 2-го курса попросили оценить сложность
Теории вероятностей и Статистики по 100 балльной шкале

\vspace{5mm}
\begin{tabular}{lrrr}
\toprule
& Аким & Ариадна & Темуужин \\
\midrule
Теория вероятностей & 70 & 75 & 82 \\
Статистика & 64 & 69 & 100 \\
\bottomrule
\end{tabular}
\vspace{5mm}

Тест знаков отвергает гипотезу о том, что Статистика и Теории
вероятностей одинаково сложны в пользу альтернативы, что Статистика
проще при уровне значимости
\begin{answerlist}
  \item \(1/3\)
  \item \(0.51\)
  \item \(0.05\)
  \item \(3/8\)
  \item \(0.1\)
\end{answerlist}
\end{question}

\begin{solution}
\begin{answerlist}
  \item Bad answer :(
  \item Good answer :)
  \item Bad answer :(
  \item Bad answer :(
  \item Bad answer :(
\end{answerlist}
\end{solution}

