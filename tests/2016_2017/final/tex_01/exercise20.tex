
\begin{question}
Величины \(X_1\), \(X_2\), \ldots, \(X_n\) представляют собой случайную
выборку из \(\cN(\mu; \sigma^2)\). Вася оценивает оба параметра с
помощью максимального правдоподобия. При этом
\begin{answerlist}
  \item \(\E(\hat \mu)=\mu\), \(\E(\hat\sigma^2) = \sigma^2\)
  \item \(\E(\hat \mu)=\mu\), \(\E(\hat\sigma^2) < \sigma^2\)
  \item \(\E(\hat \mu)=\mu\), \(\E(\hat\sigma^2) > \sigma^2\)
  \item \(\E(\hat \mu)>\mu\), \(\E(\hat\sigma^2) = \sigma^2\)
  \item \(\E(\hat \mu)<\mu\), \(\E(\hat\sigma^2) = \sigma^2\)
\end{answerlist}
\end{question}

\begin{solution}
\begin{answerlist}
  \item Bad answer :(
  \item Good answer :)
  \item Bad answer :(
  \item Bad answer :(
  \item Bad answer :(
\end{answerlist}
\end{solution}

