
\begin{question}
Преподаватель в течение 10 лет ведет статистику о посещаемости лекций.
Он заметил, что перед контрольной работой посещаемость улучшается.
Преподаватель составил следующую таблицу сопряженности

\vspace{5mm}
\begin{tabular}{lrr}
\toprule
& Контрольная будет & Контрольной не будет \\
\midrule
Пришло больше половины курса & 35 & 80 \\
Пришло меньше половины курса & 5 & 200 \\
\bottomrule
\end{tabular}
\vspace{5mm}

Если \(T\) --- статистика Пирсона, а \(k\) --- число степеней свободы её
распределения, то
\begin{answerlist}
  \item \(T>52\), \(k=2\)
  \item \(T>52\), \(k=3\)
  \item \(T<52\), \(k=4\)
  \item \(T>52\), \(k=1\)
  \item \(T<52\), \(k=1\)
\end{answerlist}
\end{question}

\begin{solution}
\begin{answerlist}
  \item Bad answer :(
  \item Bad answer :(
  \item Bad answer :(
  \item Good answer :)
  \item Bad answer :(
\end{answerlist}
\end{solution}

