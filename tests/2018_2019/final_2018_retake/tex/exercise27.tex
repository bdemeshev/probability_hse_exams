
\begin{question}
В алгоритме Метрополиса-Гастингса с предложением $\theta_{prop} = \theta + \mathcal N(0;1)$ 
был предложен переход из точки
\(\theta=5\) в точку \(\theta_{prop}=4\). Априорное
распределение \(\theta\) равномерное. Известны значения функций
правдоподобия, \(f(data|\theta=4)=0.7\), \(f(data|\theta=5)=0.8\).
Вероятность одобрения перехода равна
\begin{answerlist}
  \item \(28/40\)
  \item \(7/8\)
  \item \(4/5\)
  \item \(1\)
  \item \(0.8/5\)
\end{answerlist}
\end{question}

\begin{solution}
\[
\alpha(x \to y) = \begin{cases}
1, \text{ если } f(y|data) > f(x|data) \\
f(y|data) / f(x|data), \text{ если } f(y|data) < f(x|data) \\
\end{cases}
\]
\begin{answerlist}
  \item Неверно
  \item Отлично
  \item Неверно
  \item Не угадал
  \item Тоже ересь
\end{answerlist}
\end{solution}

