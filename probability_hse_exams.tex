\documentclass[12pt, a4paper]{article}

\usepackage{fontspec}
\usepackage{polyglossia}

\setmainlanguage{russian}
\setotherlanguages{english}

% download "Linux Libertine" OTF-fonts:
% http://www.linuxlibertine.org/index.php?id=91&L=1
\setmainfont{Linux Libertine O} % or Helvetica, Arial, Cambria
% why do we need \newfontfamily:
% http://tex.stackexchange.com/questions/91507/
\newfontfamily{\cyrillicfonttt}{Linux Libertine O}
\newfontfamily{\cyrillicfont}{Linux Libertine O}
\newfontfamily{\cyrillicfontsf}{Linux Libertine O}

\usepackage{etoolbox} % to use ifdef, must be after babel

\usepackage[paper=a4paper,
top=13.5mm, bottom=13.5mm,
left=16.5mm, right=13.5mm, includefoot]{geometry}

\usepackage{etex} % расширение классического tex
% в частности позволяет подгружать гораздо больше пакетов, чем мы и займёмся далее




\usepackage{makeidx} % для создания предметных указателей
\usepackage{verbatim} % для многострочных комментариев
%\usepackage[pdftex]{graphicx} % для вставки графики
% omit pdftex option if not using pdflatex


%\usepackage{dsfont} % шрифт для единички с двойной палочкой (для индикатора события)
\usepackage{bbm} % шрифт - двойные буквы


\usepackage[usenames, dvipsnames, svgnames, table, rgb]{xcolor}

\usepackage{colortbl}


% пакет для тестов:
\usepackage[box, % запрет на перенос вопросов
nopage, % убираем колонтитулы страницы
insidebox, % ставим буквы в квадратики
separateanswersheet, % добавляем бланк ответов
nowatermark, % отсутствие надписи "Черновик"
%indivanswers,  % показываем верные ответы
%answers,
lang=RU, % локализация слов "нет верных ответов", "вопрос" и тд
completemulti % добавлять "нет правильного ответа" во всех вопросах множественного выбора
]{automultiplechoice}


\usepackage{multicol}

\usepackage[colorlinks, hyperindex, unicode, breaklinks]{hyperref} % гиперссылки в pdf


\usepackage{amssymb}
\usepackage{amsmath}
\usepackage{amsthm}
\usepackage{epsfig}
\usepackage{bm}
\usepackage{color}



\usepackage{multirow} % Слияние строк в таблице

\usepackage{textcomp}  % Чтобы в формулах можно было русские буквы писать через \text{}

%\usepackage{embedfile} % Чтобы код LaTeXа включился как приложение в PDF-файл

\usepackage{subfigure} % для создания нескольких рисунков внутри одного

\usepackage{tikz, pgfplots} % язык для рисования графики из latex'a
\usetikzlibrary{trees} % прибамбас в нем для рисовки деревьев
\usetikzlibrary{arrows} % прибамбас в нем для рисовки стрелочек подлиннее
\usepackage{tikz-qtree} % прибамбас в нем для рисовки деревьев






\usepackage{enumitem}


%\embedfile[desc={Исходный LaTeX файл}]{\jobname.tex} % Включение кода в выходной файл
%\embedfile[desc={Стилевой файл}]{title_bor_utf8.tex}



% вместо горизонтальной делаем косую черточку в нестрогих неравенствах
\renewcommand{\le}{\leqslant}
\renewcommand{\ge}{\geqslant}
\renewcommand{\leq}{\leqslant}
\renewcommand{\geq}{\geqslant}

% делаем короче интервал в списках
\setlength{\itemsep}{0pt}
\setlength{\parskip}{0pt}
\setlength{\parsep}{0pt}

% свешиваем пунктуацию (т.е. знаки пунктуации могут вылезать за правую границу текста, при этом текст выглядит ровнее)
\usepackage{microtype}

% более красивые таблицы
\usepackage{booktabs}
% заповеди из докупентации:
% 1. Не используйте вертикальные линни
% 2. Не используйте двойные линии
% 3. Единицы измерения - в шапку таблицы
% 4. Не сокращайте .1 вместо 0.1
% 5. Повторяющееся значение повторяйте, а не говорите "то же"

\usepackage{minted} % вставка кода, нужен питон :)


\DeclareMathOperator{\grad}{grad}
\DeclareMathOperator{\card}{card}
\DeclareMathOperator{\sgn}{sign}
\DeclareMathOperator{\sign}{sign}

\DeclareMathOperator*{\argmin}{arg\,min}
\DeclareMathOperator*{\argmax}{arg\,max}
\DeclareMathOperator*{\amn}{arg\,min}
\DeclareMathOperator*{\amx}{arg\,max}
\DeclareMathOperator{\cov}{Cov}
\DeclareMathOperator{\Var}{Var}
\DeclareMathOperator{\Cov}{Cov}
\DeclareMathOperator{\Corr}{Corr}
\DeclareMathOperator{\E}{\mathbb{E}}
\let\P\relax
\DeclareMathOperator{\P}{\mathbb{P}}




\def \R{\mathbb{R}}
\def \N{\mathbb{N}}
\def \Z{\mathbb{Z}}





\newcommand{\dx}[1]{\,\mathrm{d}#1} % для интеграла: маленький отступ и прямая d
\newcommand{\ind}[1]{\mathbbm{1}_{\{#1\}}} % Индикатор события
%\renewcommand{\to}{\rightarrow}
\newcommand{\eqdef}{\mathrel{\stackrel{\rm def}=}}
\newcommand{\iid}{\mathrel{\stackrel{\rm i.\,i.\,d.}\sim}}
\newcommand{\const}{\mathrm{const}}


\usepackage{epigraph}

\AddEnumerateCounter{\asbuk}{\russian@alph}{щ} % для списков с русскими буквами
\setlist[enumerate, 2]{label=\asbuk*),ref=\asbuk*}




% \newenvironment{problem}{}{}
% тут перещёлкиваем комментарий, чтобы убрать или показать решения:
% \newenvironment{sol}{}{} % with solutions
% \excludecomment{sol} % without solutions



\unitlength=0.6mm

\title{Подборка экзаменов по теории вероятностей. \\Факультет экономики, НИУ ВШЭ}
\date{\today}
\author{Коллектив кафедры \\
математической экономики и эконометрики,
талантливые студенты,\\
 фольклор}


%%%%%%%%%%%%%%%%%% вставки
%%%%%%%%%%%%%%%%%%%%%%%%%%%%%%%%%%%%%%% Списки без уродских отступов
\newenvironment{enumerate*}{
\begin{enumerate}
  \setlength{\itemsep}{0pt}
  \setlength{\parskip}{0pt}
  \setlength{\parsep}{0pt}
}{\end{enumerate}}

\newenvironment{itemize*}{
\begin{itemize}
  \setlength{\itemsep}{0pt}
  \setlength{\parskip}{0pt}
  \setlength{\parsep}{0pt}
}{\end{itemize}}

\abovedisplayskip=0mm
\abovedisplayshortskip=0mm
\belowdisplayskip=0mm
\belowdisplayshortskip=0mm

% \newcommand{\MIN}{\textbf{(MIN)}{}}


\DeclareMathOperator{\Lin}{\mathrm{Lin}}
\DeclareMathOperator{\Linp}{\Lin^{\perp}}
\DeclareMathOperator*\plim{plim}
\newcommand{\cN}{\mathcal{N}}


\setcounter{secnumdepth}{0} % убираем нумерацию секций, подсекций и т.д.

\begin{document}
\maketitle

\tableofcontents{}


\parindent=0 pt % no indent

\section{Описание}

Свежую версию можно скачать с github-репозитория \url{https://github.com/bdemeshev/probability_hse_exams}.

Красные ссылки внутри pdf-файла кликабельны, и ведут на ответы и обратно.


Уникальное предложение для студентов факультета экономики НИУ-ВШЭ:

Найдите ошибки в этом документе или пришлите отсутствующие решения в техе и получите дополнительные бонусы!
Найденные смысловые ошибки поощряются сильнее, чем просто опечатки.
Замеченные ошибки и новые решения оформляйте в виде запросов на
\url{https://github.com/bdemeshev/probability_hse_exams/issues/}.
Перед публикацией запроса, пожалуйста, свертесь со свежей версией подборки.

В создании подборки храбро участвовали
Андрей Зубанов, Кирилл Пономарёв, Александр Левкун, Оля Гнилова,
Настя Жаркова, Гарик Варданян и другие :)


\subsection*{Доброе напутствие пишущим эту подборку :)}

Здесь перечислены стилевые особенности коллекции и самые популярные ошибки.
Узнать технические подробности по теху можно, например, \href{http://www.ccas.ru/voron/download/voron05latex.pdf}{в учебнике} К.В. Воронцова.

\begin{enumerate}

\item Дробную часть числа отделяй от целой точкой: $3.14$ — хорошо, $3{,}14$ — плохо.
Это нарушает русскую традицию, но облегчает копирование-вставку в любой программный пакет.
\item Существует длинное тире, —, которое отличается от просто дефиса - и нужно,
чтобы разделять части предложения, \href{https://ru.wikihow.com/напечатать-тире}{Инструкция в картинках по набору тире :)}
\item Выключные формулы следует окружать \verb|\[|\ldots\verb|\]|. Никаких \$\$\ldots\$\$!
\item Про остальные окружения: для системы уравнений подойдёт \verb|cases|, для формул на несколько строк – \verb|multline*|, для нумерации – \verb|enumerate|.
\item Русский текст внутри формулы нужно писать в \verb|\text{|\ldots\}.
\item Для многоточий существует команда \verb|\ldots|.
\item В преамбуле определены сокращения! Самые популярные: \verb|\P, \E, \Var, \Cov, \Corr, \cN|.
\item Названия функций тоже идут со слэшем: \verb|\ln, \exp, \cos|\ldots
\item Таблицы нужно оформлять по стандарту booktabs. Самый удобный способ сделать это – зайти на
\href{https://www.tablesgenerator.com}{tablesgenerator} и выбрать там опцию booktabs table style вместо default table style.
\item Уважай букву ё – ставь над ней точки! :)
\item Начинай каждой предложение внутри теховского файла с новой строки.
В готовом pdf предложения будут идти без разрыва, а читабельность теха повыситься.
\end{enumerate}

% стандарт имени файла:
% добавляется префикс sol_ в файле с решениями

\newpage
\thispagestyle{empty}
\section{Минимумы}

\subsection{Контрольная работа 1}

\subsubsection*{Теоретический минимум}


\begin{enumerate}
	\item Классическое определение вероятности.
	\item Определение условной вероятности.
	\item Определение независимости (попарной и в совокупности) $n$ случайных событий.
	\item Формула полной вероятности.
	\item Формула Байеса.
	\item Функция распределения случайной величины. Определение и свойства.
	\item Функция плотности. Определение и свойства.
	\item Математическое ожидание. Определения для дискретного и абсолютно непрерывного случаев. Свойства.
	\item Дисперсия. Определение и свойства.
	\item Законы распределений. Определение, $\E(X)$, $\Var(X)$.
	\begin{enumerate}
	\item Биномиальное распределение.
	\item Распределение Пуассона.
	\item Геометрическое распределение.
	\item Равномерное распределение.
	\item Экспоненциальное распределение.
	\end{enumerate}
\end{enumerate}

\newpage
\subsubsection*{\hyperref[sec:sol_minimum_kr_01]{Задачный минимум}}\label{sec:minimum_kr_01}

\begin{enumerate}
\item  Пусть $\P(A) = 0.3, \P(B) = 0.4, \P(A\cap B) = 0.1 $.
	\begin{enumerate}
		\item  Найдите $\P(A|B)$;
		\item  Найдите $\P(A\cup B)$;
		\item  Являются ли события $A$ и $B$ независимыми?
	\end{enumerate}

\item  Пусть $\P(A) = 0.5, \P(B) = 0.5, \P(A\cap B) = 0.25 $.
\begin{enumerate}
	\item  Найдите $\P(A|B)$;
	\item  Найдите $\P(A\cup B)$;
	\item  Являются ли события $A$ и $B$ независимыми?
\end{enumerate}

\item  Карлсон выложил кубиками слово КОМБИНАТОРИКА.
Малыш выбирает наугад четыре кубика и выкладывает их в случайном порядке.
Найдите вероятность того, что при этом получится слово КОРТ.

\item  Карлсон выложил кубиками слово КОМБИНАТОРИКА.
Малыш выбирает наугад четыре кубика и выкладывает их в случайном порядке.
Найдите вероятность того, что при этом получится слово РОТА.

\item  В первой урне 7 белых и 3 черных шара, во второй урне 8 белых и 4 черных
шара, в третьей урне 2 белых и 13 черных шаров.

Из этих урн наугад выбирается одна урна.
Какова вероятность того, что шар, взятый наугад из выбранной урны, окажется белым?

\item  В первой урне 7 белых и 3 черных шара, во второй урне 8 белых и 4 черных
шара, в третьей урне 2 белых и 13 черных шаров.

Из этих урн наугад выбирается одна урна.
Какова вероятность того, что была выбрана первая урна,
если шар, взятый наугад из выбранной урны, оказался белым?

\item  В операционном отделе банка работает 80\% опытных сотрудников и 20\%
неопытных. Вероятность совершения ошибки при очередной банковской операции
опытным сотрудником равна 0.01, а неопытным — 0.1.

Найдите вероятность совершения ошибки при очередной банковской операции в этом отделе.

\item  В операционном отделе банка работает 80\% опытных сотрудников и 20\%
неопытных. Вероятность совершения ошибки при очередной банковской операции
опытным сотрудником равна 0.01, а неопытным — 0.1.
Известно, что при очередной банковской операции была допущена ошибка.

Найдите вероятность того, что ошибку допустил неопытный сотрудник.

\item  Пусть случайная величина $X$ имеет таблицу распределения:

\begin{tabular}{lccc}
	\toprule
	$x$ & $-1$  & $0$  & $1$ \\
	$\P(X = x)$ & $0.25$  & $c$  & $0.25$ \\
  \bottomrule
\end{tabular}

Найдите
	\begin{enumerate}
	\item константу $c$
	\item $\P(X \geq 0)$
	\item $\P(X < -3)$
	\item $\P\left(X \in \left[-\frac{1}{2}; \frac{1}{2}\right]\right)$
	\item Функцию распределения случайной величины $X$
	\item Имеет ли случайная величина $X$ плотность распределения?
	\end{enumerate}

\item  Пусть случайная величина $X$ имеет таблицу распределения: % задача 10

\begin{tabular}{lccc}
\toprule
$x$ & $-1$  & $0$  & $1$ \\
$\P(X = x)$ & $0.25$ & $c$  & $0.25$ \\
\bottomrule
\end{tabular}

Найдите
\begin{enumerate}
	\item константу $c$
	\item $\E(X)$
	\item $\E\left(X^2\right)$
	\item $\Var(X)$
	\item $\E(|X|)$
\end{enumerate}

\item  Пусть случайная величина $X$ имеет таблицу распределения:

\begin{tabular}{lccc}
\toprule
$x$ & $-1$  & $0$  & $1$ \\
$\P(X = x)$ & $0.25$  & $c$  & $0.5$ \\
\bottomrule
\end{tabular}

Найдите
	\begin{enumerate}
	\item константу $c$
	\item $\P(X \geq 0)$
	\item $\P(X < -3)$
	\item $\P\left(X \in \left[-\frac{1}{2}; \frac{1}{2}\right]\right)$
	\item Функцию распределения случайной величины $X$
	\item Имеет ли случайная величина $X$ плотность распределения?
\end{enumerate}

\item  Пусть случайная величина $X$ имеет таблицу распределения:

\begin{tabular}{lccc}
\toprule
$x$ & $-1$  & $0$  & $1$ \\
$\P(X = x)$ & $0.25$  & $c$  & $0.5$ \\
\bottomrule
\end{tabular}

Найдите
\begin{enumerate}
	\item константу $c$
	\item $\E(X)$
	\item $\E\left(X^2\right)$
	\item $\Var(X)$
	\item $\E(|X|)$
\end{enumerate}

\item Пусть случайная величина $X$ имеет биномиальное распределение с
параметрами $n = 4$ и $p = 0.75$.
 Найдите
\begin{enumerate}
	\item $\P(X = 0)$
	\item $\P(X > 0)$
	\item $\P(X < 0)$
	\item $\E(X)$
	\item $\Var(X)$
	\item Наиболее вероятное значение, которое принимает случайная величина $X$
\end{enumerate}

\item Пусть случайная величина $X$ имеет биномиальное распределение с
параметрами $n = 5$ и $p = 0.4$.
Найдите
\begin{enumerate}
	\item $\P(X = 0)$
	\item $\P(X > 0)$
	\item $\P(X < 0)$
	\item $\E(X)$
	\item $\Var(X)$
	\item Наиболее вероятное значение, которое принимает случайная величина $X$
\end{enumerate}

\item  Пусть случайная величина $X$ имеет распределение Пуассона с параметром $\lambda = 100$.
Найдите
\begin{enumerate}
	\item $\P(X = 0)$
	\item $\P(X > 0)$
	\item $\P(X < 0)$
	\item $\E(X)$
	\item $\Var(X)$
	\item Наиболее вероятное значение, которое принимает случайная величина $X$
\end{enumerate}

\item  Пусть случайная величина $X$ имеет распределение Пуассона с параметром $\lambda = 101$.
Найдите
\begin{enumerate}
	\item $\P(X = 0)$
	\item $\P(X > 0)$
	\item $\P(X < 0)$
	\item $\E(X)$
	\item $\Var(X)$
	\item Наиболее вероятное значение, которое принимает случайная величина $X$
\end{enumerate}

\item В лифт 10-этажного дома на первом этаже вошли 5 человек.
Они выходят на каждом этаже начиная со второго равновероятно и независимо
друг от друга.
Вычислите вероятность того, что на 6-м этаже выйдет хотя бы один человек.

\item В лифт 10-этажного дома на первом этаже вошли 5 человек.
Они выходят на каждом этаже начиная со второго равновероятно и независимо
друг от друга.
Вычислите вероятность того, что на 6-м этаже не выйдет ни один человек.

\item При работе некоторого устройства время от времени возникают сбои.
Количество сбоев за сутки имеет распределение Пуассона. Среднее количество сбоев за сутки равно 3. Найти вероятность того, что в течение суток произойдет хотя бы один сбой.

\item При работе некоторого устройства время от времени возникают сбои.
Количество сбоев за сутки имеет распределение Пуассона.
Среднее количество сбоев за сутки равно 3.

Найти вероятность того, что за двое суток не произойдет ни одного сбоя.

\item Пусть случайная величина $X$ имеет плотность распределения

\[
f_X(x) =
	\begin{cases}
	c,\text{ при }  x \in [-1; 1] \\
	0,\text{ при } x \notin  [-1; 1] \\
	\end{cases}
\]

Найдите
\begin{enumerate}
	\item константу $c$
	\item $\P(X \leq 0)$
	\item $\P\left(X \in \left[\frac{1}{2}; \frac{3}{2}\right]\right)$
	\item $\P(X \in [2;3])$
	\item $F_X(x)$
\end{enumerate}

\item Пусть случайная величина $X$ имеет плотность распределения

\[
f_X(x) =
	\begin{cases}
	c,\text{ при }  x \in [-1; 1] \\
	0,\text{ при } x \notin  [-1; 1] \\
	\end{cases}
\]

Найдите
\begin{enumerate}
	\item константу $c$
	\item $\E(X)$
	\item $\E\left(X^2\right)$
	\item $\Var(X)$
	\item $\E(|X|)$
\end{enumerate}

\item Пусть случайная величина $X$ имеет плотность распределения

\[
f_X(x) =
	\begin{cases}
	cx,\text{ при }  x \in [0; 1] \\
	0,\text{ при } x \notin  [0; 1] \\
	\end{cases}
\]

Найдите
\begin{enumerate}
	\item константу $c$
	\item $\P\left(X \leq \frac{1}{2}\right)$
	\item $\P\left(X \in \left[\frac{1}{2}; \frac{3}{2}\right]\right)$
	\item $\P(X \in [2;3])$
	\item $F_X(x)$
\end{enumerate}

\item Пусть случайная величина $X$ имеет плотность распределения

\[
f_X(x) =
	\begin{cases}
	cx,\text{ при }  x \in [0; 1] \\
	0,\text{ при } x \notin  [0; 1] \\
	\end{cases}
\]

Найдите
\begin{enumerate}
	\item константу $c$
	\item $\E(X)$
	\item $\E\left(X^2\right)$
	\item $\Var(X)$
	\item $\E(\sqrt{X})$
\end{enumerate}
\end{enumerate}

\newpage
\subsection{Контрольная работа 2}
\label{sec:minimum_kr_02}

\subsubsection*{Теоретический минимум}

\begin{enumerate}
\item Сформулируйте определение независимости событий, формулу полной вероятности.
\item Приведите определение условной вероятности случайного события, формулу Байеса.
\item Сформулируйте определение и свойства функции распределения случайной величины.
\item Сформулируйте определение и свойства функции плотности случайной величины.
\item Сформулируйте определение и свойства математического ожидания для абсолютно непрерывной случайной величины.
\item Сформулируйте определение и свойства математического ожидания для дискретной случайной величины.
\item Сформулируйте определение и свойства дисперсии случайной величины.
\item Сформулируйте определения следующих законов распределений: биномиального, Пуассона, шеометрического, равномерного, экспоненциального, нормального. Укажите математическое ожидание и дисперсию.
\item Сформулируйте определение функции совместного распределения двух случайных величин, независимости случайных величин. Укажите, как связаны совместное распределение и частные распределения компонент случайного вектора.
\item	Сформулируйте определение и свойства совместной функции плотности двух случайных величин, сформулируйте определение независимости случайных величин.
\item Сформулируйте определение и свойства ковариации случайных величин.
\item Сформулируйте определение и свойства корреляции случайных величин.
\item Сформулируйте определение и свойства условной функции плотности.
\item Сформулируйте определение  условного математического ожидания $\E(Y|X=x)$ для совместного дискретного и совместного абсолютно непрерывного распределений.
\item Сформулируйте определение математического ожидания и ковариационной матрицы случайного вектора и их свойства.
\item Сформулируйте неравенство Чебышёва и неравенство Маркова.
\item Сформулируйте закон больших чисел в слабой форме.
\item Сформулируйте центральную предельную теорему.
\item Сформулируйте теорему Муавра—Лапласа.
\item Сформулируйте определение сходимости по вероятности для последовательности случайных величин.
\end{enumerate}

\newpage
\subsubsection*{\hyperref[sec:sol_minimum_kr_02]{Задачный минимум}}\label{sec:minimum_kr_02}

\begin{enumerate}
\item Пусть задана таблица совместного распределения случайных величин $X$ и $Y$.

\begin{center}
\begin{tabular}{lccc}
\toprule
                       & $Y=-1$  & $Y=0$   & $Y=1$   \\ \midrule
$X=-1$                 & $0.2$ & $0.1$ & $0.2$ \\
 $X=1$                 & $0.1$ & $0.3$ & $0.1$ \\ \bottomrule
\end{tabular}
\end{center}

Найдите
\begin{enumerate}
\item $\P(X = -1)$
\item $\P(Y = -1)$
\item $\P(X = -1 \cap Y = -1 )$
\item Являются ли случайные величины $X$ и $Y$ независимыми?
\item $F_{X,Y}(-1,0)$
\item Таблицу распределения случайной величины $X$
\item Функцию $F_{X}(x)$ распределения случайной величины $X$
\item Постройте график функции $F_{X}(x)$ распределения случайной величины $X$
\end{enumerate}

\item Пусть задана таблица совместного распределения случайных величин $X$ и $Y$.

\begin{center}
\begin{tabular}{lccc}
\toprule
                       & $Y=-1$  & $Y=0$  & $Y=1$   \\ \midrule
$X=-1$                 & $0.2$ & $0.1$ & $0.2$ \\
 $X=1$                 & $0.2$ & $0.1$ & $0.2$ \\ \bottomrule
\end{tabular}
\end{center}

Найдите
\begin{enumerate}
\item $\P(X = 1)$
\item $\P(Y = 1)$
\item $\P(X = 1 \cap Y = 1)$
\item Являются ли случайные величины $X$ и $Y$ независимыми?
\item $F_{X,Y}(1,0)$
\item Таблицу распределения случайной величины $Y$
\item Функцию $F_{Y}(y)$ распределения случайной величины $Y$
\item Постройте график функции $F_{Y}(y)$ распределения случайной величины $Y$
\end{enumerate}

\item Пусть задана таблица совместного распределения случайных величин $X$ и $Y$.

\begin{center}
\begin{tabular}{lccc}
\toprule
                       & $Y=-1$  & $Y=0$   & $Y=1$   \\ \midrule
$X=-1$                 & $0.2$ & $0.1$ & $0.2$ \\
 $X=1$                 & $0.1$ & $0.3$ & $0.1$ \\ \bottomrule
\end{tabular}
\end{center}

Найдите
\begin{enumerate}
\item $\E(X)$
\item $\E(X^{2})$
\item $\Var(X)$
\item $\E(Y)$
\item $\E(Y^{2})$
\item $\Var(Y)$
\item $\E(XY)$
\item $\Cov(X,Y)$
\item $\Corr(X,Y)$
\item Являются ли случайные величины $X$ и $Y$ некоррелированными?
\end{enumerate}

\item Пусть задана таблица совместного распределения случайных величин $X$ и $Y$.

\begin{center}
\begin{tabular}{lccc}
\toprule
                       & $Y=-1$  & $Y=0$  & $Y=1$  \\ \midrule
$X=-1$                 & $0.2$ & $0.1$ & $0.2$ \\
 $X=1$                 & $0.2$ & $0.1$ & $0.2$ \\ \bottomrule
\end{tabular}
\end{center}

Найдите
\begin{enumerate}
\item $\E(X)$
\item $\E(X^{2})$
\item $\Var(X)$
\item $\E(Y)$
\item $\E(Y^{2})$
\item $\Var(Y)$
\item $\E(XY)$
\item $\Cov(X,Y)$
\item $\Corr(X,Y)$
\item Являются ли случайные величины $X$ и $Y$ некоррелированными?
\end{enumerate}

\item
Пусть задана таблица совместного распределения случайных величин $X$ и $Y$.

\begin{center}
\begin{tabular}{lccc}
\toprule
                       & $Y=-1$  & $Y=0$   & $Y=1$   \\ \midrule
$X=-1$                 & $0.2$ & $0.1$ & $0.2$ \\
 $X=1$                 & $0.1$ & $0.3$ & $0.1$ \\ \bottomrule
\end{tabular}
\end{center}

Найдите
\begin{enumerate}
\item $\P(X = -1 | Y = 0)$
\item $\P(Y = 0 | X = -1)$
\item Таблицу условного распределения случайной величины $Y$ при условии $X = -1$
\item Условное математическое ожидание случайной величины $Y$ при $X = -1$
\item Условную дисперсию случайной величины $Y$ при условии $X = -1$
\end{enumerate}

\item Пусть задана таблица совместного распределения случайных величин $X$ и $Y$.

\begin{center}
\begin{tabular}{lccc}
\toprule
                   & $Y=-1$  & $Y=0$   & $Y=1$   \\ \midrule
$X=-1$                 & $0.2$ & $0.1$ & $0.2$ \\
 $X=1$                 & $0.2$ & $0.1$ & $0.2$ \\ \bottomrule
\end{tabular}
\end{center}

Найдите
\begin{enumerate}
\item $\P(X = 1 | Y = 0)$
\item $\P(Y = 0 | X = 1)$
\item Таблицу условного распределения случайной величины $Y$ при условии $X = 1$
\item Условное математическое ожидание случайной величины $Y$ при $X = 1$
\item Условную дисперсию случайной величины $Y$ при условии $X = 1$
\end{enumerate}

\item Пусть $\E(X)=1$, $\E(Y)=2$, $\Var(X) = 3$, $\Var(Y) = 4$, $\Cov(X,Y) = -1$. Найдите
\begin{enumerate}
\item $\E(2X + Y - 4)$
\item $\Var(3Y + 3)$
\item $\Var(X - Y)$
\item $\Var(2X - 3Y +1)$
\item $\Cov(X+ 2Y + 1,3X - Y -1)$
\item $\Corr(X + Y, X - Y)$
\item Ковариационную матрицу случайного вектора $Z = (X \qquad Y)$
\end{enumerate}


\item Пусть $\E(X)=-1$, $\E(Y)=2$, $\Var(X) = 1$, $\Var(Y) = 2$, $\Cov(X,Y) = 1$. Найдите
\begin{enumerate}
\item $\E(2X + Y - 4)$
\item $\Var(2Y + 3)$
\item $\Var(X - Y)$
\item $\Var(2X - 3Y +1)$
\item $\Cov(3X+ Y + 1,X - 2Y -1)$
\item $\Corr(X + Y, X - Y)$
\item Ковариационную матрицу случайного вектора $Z = (X \qquad Y)$
\end{enumerate}

\item Пусть случайная величина $X$ имеет стандартное нормальное распределение.
Найдите
\begin{enumerate}
\item $\P(0 < X < 1)$
\item $\P(X > 2)$
\item $\P(0 < 1 - 2X \leq 1)$
\end{enumerate}

\item Пусть случайная величина $X$ имеет стандартное нормальное распределение.
Найдите
\begin{enumerate}
\item $\P(-1 < X < 1)$
\item $\P(X < -2)$
\item $\P(-2 < -X + 1 \leq 0)$
\end{enumerate}

\item Пусть случайная величина $X \sim \cN(1,4)$. Найдите $\P(1<X<4)$.

\item Пусть случайная величина $X \sim \cN(2,4)$. Найдите $\P(-2<X<4)$.

\item Случайные величины $X$ и $Y$ независимы и  имеют нормальное распределение,
$\E(X) = 0 $, $\Var(X) = 1$, $\E(Y) = 2$, $\Var(Y) = 6$. Найдите $\P(1 < X + 2Y < 7)$.

\item Случайные величины $X$ и $Y$ независимы и  имеют нормальное распределение,
$\E(X) = 0 $, $\Var(X) = 1$, $\E(Y) = 3$, $\Var(Y) = 7$. Найдите $\P(1 < 3X + Y < 7)$.

\item Игральная кость подбрасывается $420$ раз.
При помощи центральной предельной теоремы приближенно найти вероятность того,
что суммарное число очков будет находиться в пределах от $1400$ до $1505$?

\item При выстреле по мишени стрелок попадает в десятку с вероятностью $0.5$,
в девятку – $0.3$, в восьмерку – $0.1$, в семерку – $0.05$, в шестерку – $0.05$.
Стрелок сделал $100$ выстрелов.
При помощи центральной предельной теоремы приближенно найти вероятность того,
что он набрал не менее $900$ очков?

\item Предположим, что на станцию скорой помощи поступают вызовы,
число которых распределено по закону Пуассона с параметром $\lambda = 73$,
и в разные сутки их количество не зависит друг от друга.
При помощи центральной предельной теоремы приближенно найти вероятность того,
что в течение года (365 дней) общее число вызовов будет в пределах от $26500$ до $26800$.

\item Число посетителей магазина (в день) имеет распределение Пуассона
с математическим ожиданием $289$.
При помощи центральной предельной теоремы приближенно найти вероятность того,
что за $100$ рабочих дней суммарное число посетителей составит
от $28550$ до $29250$ человек.

\item Пусть плотность распределения случайного вектора $(X,Y)$ имеет вид
\[
f_{X,Y}(x,y) =
\begin{cases}
x+y, & \text{при } (x,y) \in [0;1] \times [0;1] \\
0 , & \text{при } (x,y) \not\in [0;1] \times [0;1]
\end{cases}
\]
Найдите
\begin{enumerate}
\item $\P(X \leq \frac{1}{2} \cap Y \leq \frac{1}{2})$,
\item $\P(X\leq Y)$,
\item $f_{X}(x)$,
\item $f_{Y}(y)$,
\item Являются ли случайные величины $X$ и $Y$ независимыми?
\end{enumerate}

\item Пусть плотность распределения случайного вектора $(X,Y)$ имеет вид
\[
f_{X,Y}(x,y) =
\begin{cases} 4xy, & \text{при } (x,y) \in [0;1] \times [0;1] \\
0 , & \text{при } (x,y) \not\in [0;1] \times [0;1]
\end{cases}
\]
Найдите
\begin{enumerate}
\item $\P(X \leq \frac{1}{2} \cap Y \leq \frac{1}{2})$,
\item $\P(X\leq Y)$,
\item $f_{X}(x)$,
\item $f_{Y}(y)$,
\item Являются ли случайные величины $X$ и $Y$ независимыми?
\end{enumerate}

\item Пусть плотность распределения случайного вектора $(X,Y)$ имеет вид
\[
f_{X,Y}(x,y) =
\begin{cases} x+y, & \text{при } (x,y) \in [0;1] \times [0;1] \\
0 , & \text{при } (x,y) \not\in [0;1] \times [0;1]
\end{cases}
\]
Найдите
\begin{enumerate}
\item $\E(X)$,
\item $\E(Y)$,
\item $\E(XY)$,
\item $\Cov(X,Y)$,
\item $\Corr(X,Y)$.
\end{enumerate}

\item Пусть плотность распределения случайного вектора $(X,Y)$ имеет вид
\[
f_{X,Y}(x,y) =
\begin{cases}
4xy, & \text{при } (x,y) \in [0;1] \times [0;1] \\
0 , & \text{при } (x,y) \not\in [0;1] \times [0;1]
\end{cases}
\]
Найдите
\begin{enumerate}
\item $\E(X)$,
\item $\E(Y)$,
\item $\E(XY)$,
\item $\Cov(X,Y)$,
\item $\Corr(X,Y)$.
\end{enumerate}

\item Пусть плотность распределения случайного вектора $(X,Y)$ имеет вид
\[
f_{X,Y}(x,y) =
\begin{cases}
x+y, & \text{при } (x,y) \in [0;1] \times [0;1] \\
0 , & \text{при } (x,y) \not\in [0;1] \times [0;1]
\end{cases}
\]
Найдите
\begin{enumerate}
\item $f_{Y}(y)$,
\item $f_{X|Y}\left(x|\frac{1}{2}\right)$
\item $\E\left(X|Y = \frac{1}{2}\right)$
\item $\Var\left(X|Y = \frac{1}{2}\right)$
\end{enumerate}

\item Пусть плотность распределения случайного вектора $(X,Y)$ имеет вид
\[
f_{X,Y}(x,y) =
\begin{cases}
4xy, & \text{при } (x,y) \in [0;1] \times [0;1] \\
0 , & \text{при } (x,y) \not\in [0;1] \times [0;1]
\end{cases}
\]
Найдите
\begin{enumerate}
\item $f_{Y}(y)$,
\item $f_{X|Y}\left(x|\frac{1}{2}\right)$
\item $\E\left(X|Y = \frac{1}{2}\right)$
\item $\Var\left(X|Y = \frac{1}{2}\right)$
\end{enumerate}
\end{enumerate}

\newpage
\subsection{Контрольная работа 3}

\subsubsection*{Теоретический минимум}

\begin{enumerate}
  \item Дайте определение нормально распределённой случайной величины.
	Укажите диапазон возможных значений, функцию плотности, ожидание, дисперсию.
	Нарисуйте функцию плотности.
  \item Дайте определение хи-квадрат распределения.
	Укажите диапазон возможных значений, выражение через нормальные распределения,
	математическое ожидание.
	Нарисуйте функцию плотности при разных степенях свободы.
  \item Дайте определение распределения Стьюдента.
	Укажите диапазон возможных значений, выражение через нормальные распределения.
	Нарисуйте функцию плотности распределения Стьюдента при разных степенях свободы
	на фоне нормальной стандартной функции плотности.
  \item Дайте определение распределения Фишера.
	Укажите диапазон возможных значений, выражение через нормальные распределенеия.
	Нарисуйте возможную функцию плотности.
\end{enumerate}

Для следующего блока вопросов предполагается, что
имеется случайная выборка $X_1$, $X_2$, \ldots, $X_n$ из распределения
с функцией плотности $f(x, \theta)$, зависящей от от параметра $\theta$.
Дайте определение каждого понятия из списка или сформулируйте соответствующую теорему:

\begin{enumerate}[resume]
  \item Выборочное среднее и выборочная дисперсия;
  \item Формула несмещённой оценки дисперсии;
  \item Выборочный начальный момент порядка $k$;
  \item Выборочный центральный момент порядка $k$;
  \item Выборочная функция распределения;
  \item Несмещённая оценка $\hat \theta$ параметра $\theta$;
  \item Состоятельная последовательность оценок $\hat \theta_n$;
  \item Эффективность оценки $\hat \theta$ среди множества оценок $\hat \Theta$;
  \item Неравенство Крамера–Рао для несмещённых оценок;
  \item Функция правдоподобия и логарифмическая функция правдоподобия;
  \item Информация Фишера о параметре $\theta$, содержащаяся в одном наблюдении;
  \item Оценка метода моментов параметра $\theta$ при использовании первого момента,
	если $\E(X_i)=g(\theta)$ и существует обратная функция $g^{-1}$;
  \item Оценка метода максимального правдоподобия параметра $\theta$;
\end{enumerate}

Для следующего блока вопросов предполагается, что величины $X_1$, $X_2$, \ldots, $X_n$ независимы и нормальны $\cN(\mu;\sigma^2)$.

\begin{enumerate}[resume]
  \item Укажите закон распределения выборочного среднего,
	величины $\frac{\bar X - \mu}{\sigma/\sqrt{n}}$,
	величины $\frac{\bar X - \mu}{\hat\sigma/\sqrt{n}}$,
	величины $\frac{\hat\sigma^2(n-1)}{\sigma^2}$;
  \item Укажите формулу доверительного интервала с уровнем доверия
	$(1-\alpha)$ для $\mu$ при известной дисперсии,
	для $\mu$ при неизвестной дисперсии, для $\sigma^2$;
\end{enumerate}


\newpage
\subsubsection*{\hyperref[sec:sol_minimum_kr_03]{Задачный миннимум}}
\label{sec:minimum_kr_03}

\begin{enumerate}
\item Рост в сантиметрах (случайная величина $X$) и вес в килограммах
(случайная величина $Y$) взрослого мужчины является нормальным
случайным вектором $Z = (X, Y)$ с математическим ожиданием $\E(Z) = (175, 74)$
и ковариационной матрицей

\[
\Var(Z) =
\begin{pmatrix}
 49 & 28 \\
28 & 36
\end{pmatrix}
\]

Лишний вес характеризуется случайной величиной $U = X - Y$.
Считается, что человек страдает избыточным весом, если $U < 90$.

\begin{enumerate}
\item Определите вероятность того, что рост мужчины отклоняется от среднего более, чем на $10$ см.
\item Укажите распределение случайной величины $U$. Выпишите её плотность распределения.
\item Найдите вероятность того, что случайно выбранный мужчина страдает избыточным весом.
\end{enumerate}

\item Рост в сантиметрах, случайная величина $X$, и вес в килограммах,
случайная величина $Y$, взрослого мужчины является нормальным случайным вектором
$Z = (X, Y)$ с математическим ожиданием $\E(Z) = (175, 74)$ и ковариационной матрицей

\[
\Var(Z) =
\begin{pmatrix}
 49 & 28 \\
28 & 36
\end{pmatrix}
\]

\begin{enumerate}
\item Найдите средний вес мужчины при условии, что его рост составляет $170$ см.
\item Выпишите условную плотность распределения веса мужчины при условии, что его рост составляет $170$ см.
\item Найдите условную вероятность того, что человек будет иметь вес, больший $90$ кг, при условии, что его рост составляет $170$ см.
\end{enumerate}

\item Для реализации случайной выборки $x=(1,0,-1,1)$ найдите:

\begin{enumerate}
\item выборочное среднее,
\item неисправленную выборочную дисперсию,
\item исправленную выборочную дисперсию,
\item выборочный второй начальный момент,
\item выборочный третий центральный момент.
\end{enumerate}

\item Для реализации случайной выборки $x=(1,0,-1,1)$ найдите:

\begin{enumerate}
\item вариационный ряд,
\item первый член вариационного ряда,
\item последний член вариационного ряда,
\item график выборочной функции распределения.
\end{enumerate}

\item Пусть $X_1, \ldots,X_n$ — случайная выборка из дискретного распределения, заданного с помощью таблицы

\begin{center}
\begin{tabular}{cccc}
\toprule
 $x$ & $-3$  &$ 0 $  & $2 $  \\
 \midrule
 $\P(X_i = x)$ & $2/3 - \theta$ & $1/3$ & $\theta$ \\
 \bottomrule
\end{tabular}
\end{center}

Рассмотрите оценку $\hat{\theta} = \dfrac{\bar{X}+2}{5}$.

\begin{enumerate}
    \item Найдите $\E\left(\hat{\theta}\right)$.
    \item Является ли оценка $\hat{\theta}$ несмещенной оценкой неизвестного
		параметра $\theta$?
\end{enumerate}

\item Пусть $X_1, \ldots ,X_n$ — случайная выборка из распределения
с плотностью распределения

\[
f(x,\theta) = \begin{cases}
\dfrac{6x(\theta - x)}{\theta^3} & \text{при } x \in [0;\theta], \\
0 & \text{при } x \not\in [0;\theta],
\end{cases}
\]

где $\theta > 0$ — неизвестный параметр распределения и $\hat{\theta} = \bar{X}$.

\begin{enumerate}
\item Является ли оценка $\hat{\theta} = \bar{X}$ несмещенной оценкой
неизвестного параметра $\theta$?
\item Подберите константу $c$ так, чтобы оценка $\tilde{\theta} = c\bar{X}$
оказалась несмещенной оценкой неизвестного параметра $\theta$.
\end{enumerate}

\item Пусть $X_1,X_2,X_3$ — случайная выборка из распределения
Бернулли с неизвестным параметром $p \in (0,1)$.
Какие из следующих ниже оценкой являются несмещенными?
Среди перечисленных ниже оценок найдите наиболее эффективную оценку:

\begin{itemize}
  \item $\hat{p}_1 = \dfrac{X_1+X_3}{2}$,
  \item $\hat{p}_2 = \frac{1}{4}X_1+\frac{1}{2}X_2+\frac{1}{4}X_3$,
  \item $\hat{p}_3 = \frac{1}{3}X_1+\frac{1}{3}X_2+\frac{1}{3}X_3$.
\end{itemize}

\item Пусть $X_1, \ldots,X_n$ — случайная выборка из распределения с плотностью

\[
f(x,\theta) =
\begin{cases}
\frac{1}{\theta} \ e^{-\frac{x}{\theta}} & \text{при } x \geq 0, \\
0 & \text{при } x < 0,
\end{cases}
\]

где $\theta > 0$ — неизвестный параметр.
Является ли оценка  $\hat{\theta}_n = \frac{X_1+...+X_n}{n+1}$ состоятельной?

\item Пусть $X_1, \ldots ,X_n$ — случайная выборка из распределения
с плотностью распределения

\[
f(x,\theta) = \begin{cases}
\dfrac{6x(\theta-x)}{\theta^3} & \text{при } x \in [0;\theta], \\
0 & \text{при } x \not\in [0;\theta], \end{cases}
\]

где $\theta > 0$ — неизвестный параметр распределения.
Является ли оценка $\hat{\theta}_n = \frac{2n+1}{n}\bar{X}_n$ состоятельной оценкой
неизвестного параметра $\theta$?

\item Пусть $X_1, \ldots ,X_n$ — случайная выборка из распределения
с плотностью распределения

\[
f(x,\theta) =
\begin{cases}
\dfrac{6x(\theta-x)}{\theta^3} & \text{при } x \in [0;\theta], \\
0 & \text{при } x \not\in [0;\theta],
\end{cases}
\]

где $\theta > 0$ — неизвестный параметр распределения.
Используя центральный момент второго порядка, при помощи метода моментов
найдите оценку для неизвестного параметра $\theta$.

\item Пусть $X_1, \ldots,X_n$ — случайная выборка. Случайные величины $X_1, \ldots, X_n$ имеют дискретное распределение, которое задано при помощи таблицы

\begin{center}
\begin{tabular}{cccc}
\toprule
 $x$ & $-3$  &$ 0 $  & $2 $  \\
 \midrule
 $\P(X_i = x)$ & $2/3 - \theta$ & $1/3$ & $\theta$ \\
 \bottomrule
\end{tabular}
\end{center}

Используя второй начальный момент, при помощи метода моментов
найдите оценку неизвестного параметра $\theta$.
Для реализации случайной выборки $x=(0,0,-3,0,2)$
найдите числовое значение найденной оценки параметра $\theta$.

\item Пусть $X_1, \ldots,X_n$ — случайная выборка из распределения
с плотностью распределения

\[
f(x,\theta) =
\begin{cases}
\frac{2x}{\theta} \ e^{-\frac{x^2}{\theta}} & \text{при } x>0, \\
0 & \text{при } x \leq 0,
\end{cases}
\]

где $\theta > 0$. При помощи метода максимального правдоподобия найдите оценку неизвестного параметра $\theta$.

\item Пусть $X_1, \ldots, X_n$ – случайная выборка из распределения Бернулли
с параметром $p \in (0;1)$.
При помощи метода максимального правдоподобия найдите оценку
неизвестного параметра $p$.

\item Пусть $X=(X_1, \ldots, X_n)$ — случайная выборка из распределения с плотностью

\[
f(x,\theta) =
\begin{cases}
\frac{1}{\theta} \ e^{-\frac{x}{\theta}} & \text{при } x \geq 0, \\
0 & \text{при } x < 0, \end{cases}
\]

где $\theta > 0$ — неизвестный параметр. Является ли оценка  $\hat{\theta} = \bar{X}$ эффективной?

\item Стоимость выборочного исследования генеральной совокупности,
состоящей из трех страт, определяется по формуле
$TC = c_1n_1 + c_2n_2 + c_3n_3$, где $c_i$ — цена одного наблюдения в $i$-ой страте,
a $n_i$ — число наблюдений, которые приходятся на $i$-ую страту.
Найдите $n_1$, $n_2$ и $n_3$, при которых дисперсия стратифицированного среднего
достигает наименьшего значения,
если бюджет исследования 8000 и имеется следующая информация:

\begin{center}
\begin{tabular}{cccc}
\toprule
 Страта & $1$ & $2$ & $3$  \\
 \midrule
 Среднее значение & $30$ & $40$ & $50$ \\
 Стандартная ошибка  & $5$ & $10$ & $20$ \\
 Вес & $25\%$ & $25\%$ & $50\%$ \\
 Цена наблюдения & $1$ & $5$ & $10$ \\
 \bottomrule
\end{tabular}
\end{center}
\end{enumerate}


\newpage
\subsection{Контрольная работа 4}

\subsubsection*{Теоретический минимум}

\begin{enumerate}
  \item Дайте определение ошибки первого и второго рода, критической области.
  \item Укажите формулу доверительного интервала с уровнем доверия $(1-\alpha)$ для вероятности успеха,
	построенного по случайной выборке большого размера из распределения Бернулли $Bin(1, p)$.
\end{enumerate}

Для следующего блока вопросов предполагается, что величины $X_1$, $X_2$, \ldots, $X_n$ независимы и нормальны $\cN(\mu;\sigma^2)$.
Укажите формулу для статистики:

\begin{enumerate}[resume]
  \item Статистика, проверяющая гипотезу о математическом ожидании при известной дисперсии $\sigma^2$,
    и её распределение при справедливости основной гипотезы  $H_0$: $\mu = \mu_0$.
  \item Статистика, проверяющая гипотезу о математическом ожидании при неизвестной дисперсии $\sigma^2$,
    и её распределение при справедливости основной гипотезы  $H_0$: $\mu = \mu_0$.
\end{enumerate}


Для следующего блока вопросов предполагается, что есть две независимые случайные выборки:
выборка $X_1$, $X_2$, \ldots{ }размера $n_x$ из нормального распределения $\cN(\mu_x;\sigma^2_x)$
и выборка $Y_1$, $Y_2$, \ldots{ }размера $n_y$ из нормального распределения $\cN(\mu_y;\sigma^2_y)$.

Укажите формулу для статистики или границ доверительного интервала:

\begin{enumerate}[resume]
  \item Доверительный интервал для разницы математических ожиданий, когда дисперсии известны;
  \item Доверительный интервал для разницы математических ожиданий, когда дисперсии не известны, но равны;
  \item Статистика, проверяющая гипотезу о разнице математических ожиданий при известных дисперсиях,
    и её распределение при справедливости основной гипотезы $H_0$: $\mu_x - \mu_y = \Delta_0$;
  \item Статистика, проверяющая гипотезу о разнице математических ожиданий при неизвестных, но равных дисперсиях,
    и её распределение при справедливости основной гипотезы $H_0$: $\mu_x - \mu_y = \Delta_0$;
  \item Статистика, проверяющая гипотезу о равенстве дисперсий,
    и её распределение при справедливости основной гипотезы $H_0$: $\sigma^2_x = \sigma^2_y$.
\end{enumerate}


\newpage
\subsubsection*{\hyperref[sec:sol_minimum_kr_04]{Задачный минимум}}
\label{sec:minimum_kr_04}

\begin{enumerate}

\item Пусть $X_{1}, \ldots, X_{n}$  — случайная выборка из нормального
распределения с параметрами $\mu$ и ${\sigma}^2 = 4$.
Используя реализацию случайной выборки,
\[
x_{1} = -1.11, \quad x_{2} = -6.10, \quad x_{3} =  2.42,
\]
постройте 90\%-ый доверительный интервал для неизвестного параметра $\mu$.

\item Пусть $X_{1}, \ldots, X_{n}$ — случайная выборка
из нормального распределения с неизвестными параметрами $\mu$ и ${\sigma}^2$.
Используя реализацию случайной выборки,
\[
x_{1} = -1.11, \quad x_{2} = -6.10, \quad x_{3} = 2.42,
\]
постройте 90\%-ый доверительный интервал для неизвестного параметра $\mu$.

\item Пусть $X_{1}, \ldots, X_{n}$ — случайная выборка из нормального распределения
с неизвестными параметрами $\mu$ и ${\sigma}^2$.
Используя реализацию случайной выборки,
\[
x_{1} = 1.07, \quad x_{2} = 3.66, \quad x_{3} = -4.51,
\]
постройте 80\%-ый доверительный интервал для неизвестного параметра ${\sigma}^2$.

\item Пусть $X_{1}, \ldots, X_{n}$ и $Y_{1}, \ldots, Y_{m}$ —
независимые случайные выборки из нормального распределения с параметрами
$(\mu_{X},{\sigma^2_{X}})$ и $(\mu_{Y},{\sigma^2_{Y}})$ соответственно,
причем $\sigma^2_{X} = 2$ и $\sigma^2_{Y} = 1$.
Используя реализации случайных выборок
\begin{align*}
x_{1} &= -1.11, \quad x_{2} = -6.10, \quad x_{3} = 2.42, \\
y_{1} &= -2.29, \quad y_{2} = -2.91,
\end{align*}
постройте 95\%-ый доверительный интервал для разности математических ожиданий
$\mu_{X} - \mu_{Y}$.

\item Пусть $X_{1}, \ldots, X_{n}$ и $Y_{1}, \ldots, Y_{m}$ —
независимые случайные выборки из нормального распределения с параметрами
$(\mu_{X},{\sigma^2_{X}})$ и $(\mu_{Y},{\sigma^2_{Y}})$ соответственно.
Известно, что $\sigma^2_{X} = \sigma^2_{Y}$.
Используя реализации случайных выборок
\begin{align*}
x_{1} &= 1.53, \quad x_{2} = 2.83, \quad x_{3} = -1.25 \\
y_{1} &= -0.8, \quad y_{2} = 0.06
\end{align*}
постройте 95\%-ый доверительный интервал для разности математических ожиданий
$\mu_{X} - \mu_{Y}$.

\item Пусть $X_{1}, \ldots, X_{n}$ — случайная выборка из распределения Бернулли
с параметром $p$.
Используя реализацию случайной выборки $X_{1}, \ldots, X_{n}$,
в которой 55 нулей и 45 единиц,
постройте приближенный 95\%-ый доверительный интервал для неизвестного параметра $p$.

\item Пусть $X_{1}, \ldots, X_{n}$ и $Y_{1}, \ldots, Y_{m}$ — независимые случайные
выборки из распределения Бернулли с параметрами $p_{X} \in (0;1)$ и $p_{Y} \in (0;1)$ соответственно.
Известно, что $n = 100$, $\bar{x}_{n} = 0.6$, $m = 200$, $\bar{y}_{m} = 0.4$.
Постройте приближенный 95\%-ый доверительный интервал для отношения разности
вероятностей успеха $p_{X} - p_{Y}$.

\item Дядя Вова (Владимир Николаевич) и Скрипач (Гедеван) зарабатывают на Плюке чатлы,
чтобы купить гравицапу.
Число заработанных за $i$-ый день чатлов имеет распределение Пуассона с неизвестным параметром $\lambda$.
Заработки в различные дни независимы. За прошедшие 100 дней они заработали 250 чатлов.

С помощью метода максимального правдоподобия постройте приближенный
95\%-ый доверительный интервал для неизвестного параметра $\lambda$.

\item Пусть $X_{1}, \ldots, X_{n}$ — случайная выборка из показательного
(экспоненциального) распределения с плотностью распределения
\[
f(x,\lambda)=
\begin{cases}
\lambda e^{-\lambda x}\text{ при } x\geq 0 \\
0 \text{ при } x < 0 \\
\end{cases}
\]
где $\lambda > 0$ — неизвестный параметр распределения.
Известно, что $n = 100$ и $\bar{x}_n = 0.52$.

С помощью метода максимального правдоподобия постройте приближенный
95\%-ый доверительный интервал для параметра $\lambda$.

% \item Пусть $X_{1}, \ldots, X_{n}$  — случайная выборка из равномерного распределения на отрезке $[0; \theta]$, где  $\theta > 0$ — неизвестный параметр распределения. Известно, что $n = 100$ и $\bar{x}_n = 0.57$.

% С помощью метода максимального правдоподобия постройте приближенный 95\%-ый доверительный интервал для параметра $\theta$.
% тут брутальный дельта-метод :) это не минимум :)


\item Пусть $X_{1}, \ldots, X_{n}$ — случайная выборка из нормального распределения
с неизвестным математическим ожиданием $\mu$ и известной дисперсией $\sigma^2 = 4$.
Объем выборки $n = 16$. Для тестирования основной гипотезы $H_{0}:\mu = 0$ против
альтернативной гипотезы $H_{1}:\mu = 2$ вы используете критерий: если $\bar{X} \leq 1$,
то вы не отвергаете гипотезу $H_{0}$,
в противном случае вы отвергаете гипотезу $H_{0}$ в пользу гипотезы $H_{1}$. Найдите

\begin{enumerate}
   \item  вероятность ошибки 1-го рода;
   \item вероятность ошибки 2-го рода;
   \item мощность критерия.
\end{enumerate}


\item На основе случайной выборки, содержащей одно наблюдение $X_{1}$,
тестируется гипотеза $H_{0} : X_{1} \sim U[-0.7;0.3]$ против альтернативной гипотезы
$H_{1}: X_{1} \sim U[-0.3;0.7]$.
Рассматривается критерий вида: если $X_{1} > c$,
то гипотеза $H_{0}$ отвергается в пользу гипотезы $H_{1}$.
Выберите константу $c$ так, чтобы уровень значимости этого критерия составлял $0.1$.

\item Пусть $X_{1}, \ldots, X_{n}$ — случайная выборка из нормального распределения
с параметрами $\mu$ и $\sigma^2 = 4$.
Уровень значимости  $\alpha = 0.1$.
Используя реализацию случайной выборки $x_{1} = -1.11, x_{2} = -6.10, x_{3} = 2.42$,
проверьте следующую гипотезу:
\[
\begin{cases}
H_{0}:\mu = 0, \\
H_{1}:\mu > 0 \\
\end{cases}
\]

\item Пусть $X_{1}, \ldots, X_{n}$ — случайная выборка из нормального
распределения с параметрами $\mu$ и $\sigma^2$.
Уровень значимости  $\alpha = 0.1$.
Используя реализацию случайной выборки $x_{1} = -1.11, x_{2} = -6.10, x_{3} = 2.42$,
проверьте следующую гипотезу:
\[
\begin{cases}
H_{0}:\mu = 0, \\
H_{1}:\mu > 0 \\
\end{cases}
\]

\item Пусть $X_{1}, \ldots, X_{n}$ и $Y_{1}, \ldots, Y_{m}$ —
независимые случайные выборки из нормального распределения
с параметрами $(\mu_{X},\sigma^2_{X})$ и $(\mu_{Y},\sigma^2_{Y})$ соответственно,
причем  $\sigma^2_{X} = 2$ и $\sigma^2_{Y} = 1$. Уровень значимости $\alpha = 0.05$.
Используя реализации случайных выборок

\begin{align*}
x_{1} &= -1.11, \quad x_{2} = -6.10, \quad x_{3} = 2.42, \\
y_{1} &= -2.29, \quad y_{2} = -2.91,
\end{align*}

проверьте следующую гипотезу:
\[
\begin{cases}
H_{0}:\mu_{X} = \mu_{Y}, \\
H_{1}:\mu_{X} < \mu_{Y} \\
\end{cases}
\]

\item Пусть $X_{1}, \ldots, X_{n}$ и $Y_{1}, \ldots, Y_{m}$ — независимые случайные
выборки из нормального распределения с параметрами $(\mu_{X},\sigma^2_{X})$ и
$(\mu_{Y},\sigma^2_{Y})$ соответственно.
Известно, что $\sigma^2_{X} = \sigma^2_{Y}$.
Уровень значимости $\alpha = 0.05$. Используя реализации случайных выборок

\begin{align*}
x_{1} &= 1.53, \quad x_{2} = 2.83, \quad x_{3} = -1.25 \\
y_{1} &= -0.8, \quad y_{2} = 0.06
\end{align*}

проверьте следующую гипотезу:
\[
\begin{cases}
H_{0}:\mu_{X} = \mu_{Y}, \\
H_{1}:\mu_{X} < \mu_{Y} \\
\end{cases}
\]

\item Пусть $X_{1}, \ldots, X_{n}$ и $Y_{1}, \ldots, Y_{m}$ — независимые случайные
выборки из нормального распределения с параметрами $(\mu_{X},\sigma^2_{X})$ и
$(\mu_{Y},\sigma^2_{Y})$ соответственно.
Уровень значимости $\alpha = 0.05$.
Используя реализации случайных выборок\newline
\begin{align*}
x_{1} &= -1.11, \quad x_{2} = -6.10, \quad x_{3} = 2.42, \\
y_{1} &= -2.29, \quad y_{2} = -2.91,
\end{align*}
проверьте следующую гипотезу:
\[
\begin{cases}
H_{0}:\sigma^2_{X} = \sigma^2_{Y}, \\
H_{1}:\sigma^2_{X} > \sigma^2_{Y} \\
\end{cases}
\]

\item Пусть  $X_{1}, \ldots, X_{n}$ — случайная выборка из распределения Бернулли с
неизвестным параметром $p \in (0;1)$.Имеется следующая информация о реализации
случайной выборки, содержащей $n = 100$ наблюдений: $\sum_{i=0}^{n} x_{i} = 60$.
На уровне значимости $\alpha = 0.05$ требуется протестировать следующую гипотезу:
\[
\begin{cases}
H_{0}:p = 0.5, \\
H_{1}:p > 0.5 \\
\end{cases}
\]

\item Пусть $X_{1}, \ldots, X_{n}$ и $Y_{1}, \ldots, Y_{m}$ —
две независимые случайные выборки из распределения Бернулли с неизвестными параметрами
$p_{X} \in (0; 1)$ и $p_{Y} \in (0; 1)$.
Имеется следующая информация о реализациях этих случайных выборок: $n = 100$,
$\sum_{i=1}^{n} x_{i} = 60$, $m = 150$,$\sum_{j=1}^{m} y_{j} = 50$.
На уровне значимости $\alpha = 0.05$ требуется протестировать следующую гипотезу:
\[\begin{cases}
H_{0}:p_{X} = p_{Y}, \\
H_{1}:p_{X} \neq p_{Y} \\
\end{cases}\]

\item Вася Сидоров утверждает, что ходит в кино в два раза чаще, чем в спортзал,
а в спортзал в два раза чаще, чем в театр.
За последние полгода он 10 раз был в театре, 17 раз – в спортзале и 39 раз в кино.
На уровне значимости 5\% проверьте утверждение Васи.

\item Вася очень любит тестировать статистические гипотезы.
В этот раз Вася собирается проверить утверждение о том,
что его друг Пётр звонит Васе исключительно в то время, когда Вася ест.
Для этого Вася трудился целый год и провел серию из 365 испытаний.
Результаты приведены в таблице ниже.

\begin{center}\begin{tabular}{r|rr}
\toprule
   & Пётр звонит   & Пётр не звонит  \\ \midrule
Вася ест           & $200$ & $40$ \\
 Вася не ест       & $25$ & $100$  \\ \bottomrule
\end{tabular}\end{center}

На уровне значимости 5\% протестируйте гипотезу о том, что Пётр звонит Васе
независимо от момента приема пищи Васей.

\item Пусть $X_{1}, \ldots, X_{n}$ — случайная выборка из нормального распределения
с математическим ожиданием $\mu \in \mathbb{R}$ и дисперсией $v > 0$,
где $\mu$ и $v$ — неизвестные параметры.
Известно, что выборка состоит из $n = 100$ наблюдений,
$\sum_{i=1}^{n} x_{i} = 30$, $\sum_{i=1}^{n} x^2_{i} = 146$.
При помощи теста отношения правдоподобия протестируйте гипотезу $H_{0}:v = 1$
на уровне значимости 5\%.

\end{enumerate}

\section{Контрольная работа 1}


% \subsection[что идет в оглавление]{\hyperref[на что ссылка]{текст ссылки}}
\subsection[2017-2018]{\hyperref[sec:sol_kr_01_2017_2018]{2017-2018}}
\label{sec:kr_01_2017_2018} % \label{ссылка сюда}

% * — не идёт в оглавление
\subsubsection*{Минимум}

\begin{enumerate}
\item Функция распределения случайной величины: определения и свойства.
\item Экспоненциальное распределение: определение, математическое ожидание и дисперсия.
\item В операционном отделе банка работает 80\% опытных сотрудников и 20\% неопытных. Вероятность совершения ошибки при очередной банковской операции опытным сотрудником равна $0.01$, а неопытным — $0.1$. Известно, что при очередной банковской операции была допущена ошибка. Найдите вероятность того, что ошибку допустил неопытный сотрудник.
\item При работе некоторого устройства время от времени возникают сбои. Количество сбоев за сутки имеет распределение Пуассона. Среднее количество сбоев за сутки равно 3. Найдите вероятность того, что за двое суток не произойдет ни одного сбоя.

\end{enumerate}

\subsubsection*{Задачи}

\begin{enumerate}

\item Правильный кубик подбрасывают один раз. Событие $A$ — выпало чётное число, событие $B$ — выпало число кратное трём, событие $C$ — выпало число, большее трёх.

\begin{enumerate}
\item Сформулируйте определение независимости двух событий;
\item Определите, какие из пар событий $A$, $B$ и $C$ будут независимыми.
\end{enumerate}


\item Теоретический минимум (ТМ) состоит из 10 вопросов, задачный (ЗМ) — из 24 задач.
Каждый вариант контрольной содержит два вопроса из ТМ и две задачи из ЗМ.
Чтобы получить за контрольную работу оценку 4 и выше, необходимо и достаточно правильно ответить на каждый вопрос ТМ и задачу ЗМ доставшегося варианта. Студент Вася принципиально выучил только $k$ вопросов ТМ и две трети ЗМ.
\begin{enumerate}
\item Сколько всего можно составить вариантов, отличающихся хотя бы одним заданием в ТМ или ЗМ части? Порядок заданий внутри варианта не важен.
\item Найдите вероятность того, что Вася правильно решит задачи ЗМ;
\item Дополнительно известно, что Васина вероятность правильно ответить на вопросы ТМ, составляет $1/15$. Сколько вопросов ТМ выучил Вася?
\end{enumerate}

\item Производитель молочных продуктов выпустил новый низкокалорийный йогурт Fit и утверждает, что он вкуснее его более калорийного аналога Fat.
Четырем независимым экспертам предлагают выбрать наиболее вкусный йогурт из трёх, предлагая им в одинаковых стаканчиках в случайном порядке два Fat и один Fit.
Предположим, что йогурты одинаково привлекательны.
Величина $\xi$ — число экспертов, отдавших предпочтение Fit.
\begin{enumerate}
\item Какова вероятность, что большинство экспертов выберут Fit?
\item Постройте функцию распределения величины $\xi$;
\item Каково наиболее вероятное число экспертов, отдавших предпочтение йогорту Fit?
\item Вычислите математическое ожидание и дисперсию $\xi$.
\end{enumerate}

\item Дядя Фёдор каждую субботу закупает в магазине продукты по списку, составленному котом Матроскином. Список не изменяется, и в него всегда входит 1 кг сметаны, цена которого является равномерно распределённой величиной $\alpha$, принимающей значения от 250 до 1000 рублей. Стоимость остальных продуктов из списка в тысячах рублей является случайной величиной $\xi$ с функцией распределения

\[
F(x)=\begin{cases}
1-\exp(-x^2 ), \text{ если } x \geq 0 \\
0, \text{ иначе.}\\
\end{cases}
\]

\begin{enumerate}
\item Какую сумму должен выделить кот Матроскин дяде Фёдору, чтобы её достоверно хватало на покупку сметаны?
\item Какую сумму должен выделить кот Матроскин дяде Фёдору, чтобы Дядя Фёдор с вероятностью 0.9 мог оплатить продукты без сметаны?
\item Найдите математическое ожидание стоимости продуктов без сметаны;
\item Найдите математическое ожидание стоимости всего списка.
\item Какова вероятность того, что общие расходы будут в точности равны их математическому ожиданию?
\end{enumerate}

Подсказка: $\int_0^{\infty} \exp(-x^2) \, dx = \sqrt{\pi} / 2$.

\item Эксперт с помощью детектора лжи пытается определить, говорит ли подозреваемый правду. Если подозреваемый говорит правду, то эксперт ошибочно выявляет ложь с вероятностью 0.1. Если подозреваемый обманывает, то эксперт выявляет ложь с вероятностью 0.95.

В деле об одиночном нападении подозревают десять человек, один из которых виновен и будет лгать, остальные невиновны и говорят правду.

\begin{enumerate}
\item Какова вероятность того, что детектор покажет, что конкретный подозреваемый лжёт?
\item Какова вероятность того, что подозреваемый невиновен, если детектор показал, что он лжёт?
\item Какова вероятность того, что эксперт точно выявит преступника?
\item Какова вероятность того, что эксперт ошибочно выявит  преступника, то есть покажет, что лжёт невиновный, а все остальные говорят правду?
\end{enumerate}
\end{enumerate}


\newpage
\subsection[2016-2017]{\hyperref[sec:sol_kr_01_2016_2017]{2016-2017}}
\label{sec:kr_01_2016_2017}

\begin{enumerate}
\item Из семей, имеющих двоих разновозрастных детей, случайным образом выбирается одна семья.
Известно, что в семье есть девочка (событие $A$).

\begin{enumerate}
\item	Какова вероятность того, что в семье есть мальчик (событие $B$)?
\item	Сформулируйте определение независимости событий и проверьте,
являются ли события $A$ и $B$ независимыми?
\end{enumerate}

\item Система состоит из $N$ независимых узлов.
При выходе из строя хотя бы одного узла, система дает сбой.
Вероятность выхода из строя любого из узлов равна $0.000001$.
Вычислите максимально возможное число узлов системы,
при котором вероятность её сбоя не превышает $0.01$.

\item Исследование состояния здоровья населения в шахтерском регионе
«Велико-кротовск» за пятилетний период показало,
что из всех людей с диагностированным заболеванием легких, 22\% работало на шахтах.
Из тех, у кого не было диагностировано заболевание легких, только 14\% работало на шахтах.
Заболевание легких было диагностировано у 4\% населения региона.

\begin{enumerate}
\item	Какой процент людей среди тех, кто работал в шахте,
составляют люди с диагностированным заболеванием легких?
\item	Какой процент людей среди тех, кто НЕ работал в шахте,
составляют люди с диагностированным заболеванием легких?
\end{enumerate}

\item  Студент Петя выполняет тест (множественного выбора) проставлением ответов наугад.
В тесте 17 вопросов, в каждом из которых пять вариантов ответов и только один из них правильный.
Оценка по десятибалльной шкале формируется следующим образом:
\[
    \text{Оценка} = \left\{
                      \begin{array}{ll}
                        \text{ЧПО} - 7, & \text{если $\text{ЧПО}\in [8;\,17]$,} \\
                        1,              & \text{если $\text{ЧПО}\in [0;\,7]$,}
                      \end{array}
                    \right.
\]
где ЧПО означает число правильных ответов.

\begin{enumerate}
\item	Найдите наиболее вероятное число правильных ответов.
\item	Найдите математическое ожидание и дисперсию числа правильных ответов.
\item	Найдите вероятность того, что Петя получит «отлично»
(по десятибалльной шкале получит 8, 9 или 10 баллов).

Студент Вася также выполняет тест проставлением ответов наугад.

\item	Найдите вероятность того, что все ответы Пети и Васи совпадут.
\end{enumerate}

\item  Продавец высокотехнологичного оборудования контактирует с одним или двумя
потенциальными покупателями в день с вероятностями $1/3$ и $2/3$ соответственно.
Каждый контакт заканчивается «ничем» с вероятностью $0.9$ и покупкой оборудования
на сумму в 50\,000 у.\,е. с вероятностью $0.1$.
Пусть $\xi$ — случайная величина, означающая объем дневных продаж в у.\,е.

\begin{enumerate}
\item	Вычислите  $\P(\xi = 0)$.
\item	Сформулируйте определение функции распределения и постройте функцию распределения
случайной величины $\xi$.
\item	Вычислите математическое ожидание и дисперсию случайной величины $\xi$.
\end{enumerate}

\item Интервал движения поездов метро фиксирован и равен $b$ минут,
т.е. каждый следующий поезд появляется после предыдущего ровно через $b$ минут.
Пассажир приходит на станцию в случайный момент времени.
Пусть случайная величина $\xi$, означающая время ожидания поезда,
имеет равномерное распределение на отрезке $[0; b]$.

\begin{enumerate}
\item Запишите плотность распределения случайной величины $\xi$.
\item	Найдите константу $b$, если известно, что в среднем пассажиру приходится
ждать поезда одну минуту, т.\,е. $\E(\xi) = 1$.
\item	Вычислите дисперсию случайной величины $\xi$.
\item	Найдите вероятность того, что пассажир будет ждать поезд менее одной минуты.
\item	Найдите квантиль порядка $0.25$ распределения случайной величины $\xi$.
\item	Найдите центральный момент порядка 2017 случайной величины $\xi$.
\item	Постройте функцию распределения случайной величины $\xi$.

Марья Ивановна из суеверия всегда пропускает два поезда и садится в третий.

\item	Найдите математическое ожидание и дисперсию времени,
затрачиваемого Марьей Ивановной на ожидание «своего» поезда.

Глафира Петровна не садится в поезд, если видит в нем подозрительного человека.
Подозрительные люди встречаются в каждом поезде с вероятностью $3/4$.

\item	Найдите вероятность того, что Глафире Петровне придется ждать не менее пяти минут,
чтобы уехать со станции.
\item	Найдите математическое ожидание времени ожидания «своего» поезда для Глафиры Петровны.
\end{enumerate}

\item (Бонусная задача)
На первом этаже десятиэтажного дома в лифт заходят 9 человек.
Найдите математическое ожидание числа остановок лифта, если люди выходят из лифта независимо друг от друга.
\end{enumerate}

\newpage
\thispagestyle{empty}
\section{Контрольная работа 1. ИП}

\subsection[2018-2019]{\hyperref[sec:sol_kr_01_ip_2018_2019]{2018-2019}}
\label{sec:kr_01_ip_2018_2019}

\begin{enumerate}
\item Пират Злопамятный Джо очень любит неразбавленный ром. Из-за того,
что он много пьёт, у него проблемы с памятью, и он помнит не больше, чем три последних
пинты. Хозяин таверны «Огненная зебра» с вероятностью $1/8$ разбавляет каждую подаваемую пинту рома.
Если по ощущением Джо половина выпитых пинт или больше была разбавлена, то он
разносит таверну к чертям собачьим. Только что Джо вошёл в таверну и закал первую пинту.

Сколько в среднем пинт выпьет Джо, прежде чем разнесёт таверну?

\item В таверне «Крутой ковбой» разбавленный ром подают с вероятностью $1/2$.
Джо немного сменил свой характер и теперь устраивает скандал,
если две пинты рома подряд разбавлены.

Какова вероятность того, что Джо сможет выпить 100 пинт подряд без скандалов?

\item Али-Баба хочет проникнуть в пещеру с сокровищами. Вход в
пещеру закрыт и его охраняет Джин с квадратным подносом.
В каждой вершине подноса — непрозрачный стаканчик. Под
каждым стаканчиком — монетка.
Если все четыре монетки окажутся в одинаковом положении, все —
орлом вверх, или все — решкой вверх, то вход откроется.
За одно действие Али-Баба может открыть любые два стаканчика и
положить открывшиеся монетки любой стороной вверх.
После действия Али-Бабы Джин накрывает монетки стаканчиками,
быстро-быстро вращает поднос и снова предоставляет поднос
Али-Бабе.
Углядеть за Джином или сделать пометки на подносе невозможно.

\begin{enumerate}
  \item Как надо действовать Али-Бабе, чтобы гарантировать себе вход в
  пещеру за наименьшее количество действий?
  \item Сколько действий
  потребуется в худшем случае?
\end{enumerate}

\item Злопамятный Джо очень любит играть в картишки. Перед Джо хорошо перемешанная
стандартная колода в 52 карты. Джо извлекает карты по одной.

На каком месте в среднем появляется первая Дама?

\item Вероятность того, что Ученик достигнет Просветления за малый интервал времени,
прямо пропорциональна длине этого интервала, а именно,
\[
\P(\text{достигнуть Просветления за отрезок времени }[t;t+\Delta]) = 0.2018 \Delta + o(\Delta)
\]

Какова точная вероятность того, что Ученик, начавший искать Просветление, так
и не достигнет его к моменту времени $t$?

\item Исследователь Василий выбирает равномерно и независимо друг от друга 10 точек на
отрезке $[0;1]$. Затем Василий записывает их координаты в порядке возрастания,
$Y_1 \leq Y_2 \leq \ldots \leq Y_{10}$.

Не производя вычислений, \textit{по определению},
выпишите функции плотности случайной величины $Y_4$.
\end{enumerate}




\subsection[2017-2018]{\hyperref[sec:sol_kr_01_ip_2017_2018]{2017-2018}}
\label{sec:kr_01_ip_2017_2018}

Ровно 272 года назад императрица Елизавета повелела завезти во дворцы котов для ловли мышей.

\begin{enumerate}
\item В отсутствии кота Леопольда мыши Белый и Серый подкидывают по очереди
игральный додекаэдр.
%\footnote{Леопольд подсказывает по случаю праздника, что у додекаэдра 12 граней :)}
Сыр достаётся тому, кто первым выкинет число 6. Начинает подкидывать Белый.
\begin{enumerate}
  \item Какова вероятность того, что сыр достанется Белому?
  \item Сколько в среднем бросков продолжается игра?
  \item Какова дисперсия числа бросков?
\end{enumerate}

\item Микки Маус, Белый и Серый решили устроить труэль из любви к мышки Мии.
Сначала делает свой выстрел Микки, затем Белый, затем Серый, затем снова Микки и так до тех пор,
пока в живых не останется только один.

Прошлые данные говорят о том, что Микки попадает с вероятностью $1/3$,
Белый — с вероятностью $2/3$, а Серый стреляет без промаха.

Найдите оптимальную стратегию каждого мыша, куда кому следует целиться.

\item Микки Маус, Белый и Серый пойманый злобным котом Леопольдом до начала труэли.
И теперь Леопольд будет играть с ними в странную игру.

В комнате три закрытых внешне неотличимых коробки: с золотом, серебром и платиной.
Общаться после начала игры мыши не могут, но могут заранее договориться о стратегии.

Правила игры таковы. Кот Леопольд будет заводить мышей в комнату по очереди.
Каждый из мышей может открыть две коробки по своему выбору.
Перед следующим мышом коробки закрываются.

Если Микки откроет коробку с золотом, Белый — с серебром, а Серый — с платиной,
то они выигрывают. Если хотя бы один из мышей не найдёт свой металл, то Леопольд
их съест.
\begin{enumerate}
\item Какова оптимальная стратегия?
\item Какова вероятность выигрыша при использовании оптимальной стратегии?
\end{enumerate}

\item Накануне войны Жестокий Тиран Мышь очень большой страны издал указ.
Отныне за каждого новорождённого мыша-мальчика семья получает денежную премию,
но если в семье рождается вторая мышка-девочка, то всю семью убивают.
Бедные жители страны запуганы и остро нуждаются в деньгах, поэтому в каждой семье
мыши будут появляться до тех пор, пока не родится первая мышка-девочка.
\begin{enumerate}
  \item Каким будет среднее число детей в мышиной семье?
  \item Какой будет доля мышей-мальчиков в стране?
  \item Какой будет средняя доля мышей-мальчиков в случайной семье?
  \item Сколько в среднем мышей-мальчиков в случайно выбираемой семье?
\end{enumerate}

\item Вальяжный кот Василий положил на счёт в банке на Гаити один гурд.
Сумма на счету растёт непрерывно с постоянной ставкой в течение очень длительного
промежутка времени. В случайный момент этого промежутка кот Василий закрывает свой вклад.

Каков закон распределения первой цифры полученной Василием суммы?
\end{enumerate}



\newpage
\subsection[2016-2017]{\hyperref[sec:sol_kr_01_ip_2016_2017]{2016-2017}}
\label{sec:kr_01_ip_2016_2017}


\begin{enumerate}
\item Задача о макаронинах

В тарелке запутавшись лежат много-много макаронин.
Я по очереди связываю попарно все торчащие концы макаронин.
\begin{enumerate}
\item Какова примерно вероятность того, что я свяжу все макаронины в одно большое кольцо?
\item Сколько в среднем колец образуется?
\item Каково среднее число колец длиной в одну макаронину?
\end{enumerate}

\item Планета Плюк

На планету Плюк, окружность, в случайных точках садятся $n$ пепелацев.
Радиосвязь между двумя точками на планете Плюк возможна, если центральный угол
между этими двумя точками меньше $\pi/2$.
\begin{enumerate}
\item Какова вероятность того, что из любой точки планеты можно связаться хотя бы с одним пепелацем?
\item Какова вероятность того, что при $n=3$ все три пепелаца смогут поддерживать связь друг с другом (необязательно напрямую, возможно через посредника)?
\item Как изменятся ответы, если планета Плюк — это сфера?
\end{enumerate}

\item Чайник Рассела

Вокруг Солнца по эллиптической орбите вращается абсолютно плоский чайник Рассела
с площадью $42$ см$^2$. Летающий Макаронный Монстр проецирует чайник Рассела на
случайную плоскость.

Чему равна ожидаемая площадь проекции?

% \item Винни-Пух собирается играть в Пустяки и готовит для игры палочки. Он нашел
% палку длиной 1 м, а дальше поступает следующим образом. Разламывает палку равномерно
% в случайном месте, одну полученную часть использует для игры, а вторую снова случайным
% образом делит на две части. Далее одну новую часть Винни-Пух снова использует
% для игры, а вторую новую часть снова делит на две. И так далее. Обозначим $X_i$ —
% длину палочки, использованной Винни-Пухом в $i$-ых Пустяках.

% Найдите функцию плотности $X_i$, $\E(X_i)$, $\Var(X_i)$

\item Чак Норрис против Брюса Ли

Чак Норрис хватается за верёвку в форме окружности в произвольной точке.
Брюс Ли берёт мачете и с завязанными глазами разрубает верёвку в двух случайных
независимых местах. Чак Норрис забирает себе тот кусок, за который держится.
Брюс Ли забирает оставшийся кусок.  Вся верёвка имеет единичную длину.
\begin{enumerate}
\item Чему равна ожидаемая длина куска верёвки, доставшегося Брюсу Ли?
\item  Вероятность того, что у Брюса Ли верёвка длиннее?
\end{enumerate}

\item Истеричная певица

Начинающая певица дает концерты каждый день. Каждый её концерт приносит продюсеру
$0.75$ тысяч евро. После каждого концерта певица может впасть в депрессию
с вероятностью $0.5$. Самостоятельно выйти из депрессии певица не может.
В депрессии она не в состоянии проводить концерты. Помочь ей могут только хризантемы
от продюсера. Если подарить цветы на сумму $0\le x\le 1$ тысяч евро, то она выйдет
из депрессии с вероятностью $\sqrt{x}$.

Какова оптимальная стратегия продюсера, максимизирующего ожидаемую прибыль?

\item Гадалка

Джульетта пишет на бумажках два любых различных натуральных числа по своему выбору.
Одну бумажку она прячет в левую руку, а другую — в правую. Ромео выбирает любую руку
Джульетты. Джульетта показывает число, написанное на выбранной бумажке. Ромео высказывает
свою догадку о том, открыл ли он большее из двух чисел или меньшее. Ромео выигрывает,
если он угадал.

Приведите пример стратегии Ромео, дающей ему вероятность выигрыша строго больше
$0.5$ против любой стратегии Джульетты.

\item Мудрецы

В ряд друг за другом за бесконечным столом сидит счётное количество Мудрецов,
постигающих Истину. Первым сидит Абу Али Хусейн ибн Абдуллах ибн аль-Хасан ибн
Али ибн Сина:

\begin{figure}[h!]
  \begin{center}
\includegraphics[width=5cm]{images/abu_ali.jpg}
  \caption*{«Коль смолоду избрал к заветной правде путь, \\
 С невеждами не спорь, советы их забудь». }
 \end{center}
\end{figure}

Каждый Мудрец может постигнуть Истину самостоятельно с вероятностью $1/9$
или же от соседа\footnote{Студенты постигают Истину примерно также!}. Независимо
от способа постижения Истины, просветлённый Мудрец поделится Истиной с соседом
слева с вероятностью $2/9$ и с соседом справа также с вероятностью $2/9$ (независимо
от соседа слева).
\begin{enumerate}
\item Какова вероятность того, что Абу Али Хусейн ибн Абдуллах ибн аль-Хасан ибн
Али ибн Сина постигнет Истину?
\item Как изменится ответ, если ряд Мудрецов бесконечен в обе стороны?
\end{enumerate}
\end{enumerate}



\newpage
\subsection[2015-2016]{\hyperref[sec:sol_kr_01_ip_2015_2016]{2015-2016}}
\label{sec:kr_01_ip_2015_2016}

\subsubsection*{Индивидуальный тур}

\begin{enumerate}
\item Для разминки вспомним греческий алфавит!

\begin{enumerate}
\item По-гречески — Σωκρατης, а по-русски — \underline{\hspace{2cm}}
\item Изобразите прописные и строчные буквы: эта \underline{\hspace{2cm}},
дзета \underline{\hspace{2cm}}, вега \underline{\hspace{2cm}},
шо \underline{\hspace{2cm}}. Если такой буквы в греческом нет, то поставьте прочерк.
\item Назовите буквы: τ \underline{\hspace{2cm}}, θ \underline{\hspace{2cm}},
ξ \underline{\hspace{2cm}}.
%\item Если пересчитать все буквы в греческом алфавите, то их окажется ровно \underline{\hspace{2cm}} %24
\end{enumerate}

\item Подбрасываются 2 симметричные монеты. Событие $A$ — на первой монете выпал герб,
событие $B$ — на второй монете выпал герб, событие $C$ — монеты выпали разными сторонами.
\begin{enumerate}
\item Будут ли эти события попарно независимы?
\item Сформулируйте определение независимости в совокупности для трех событий
\item Являются ли события $A$, $B$, $C$ независимыми в совокупности?
\end{enumerate}


\item Имеются два игральных кубика: \textbf{красный} со смещенным центром тяжести,
так что вероятность выпадения «6» равняется $1/3$, а оставшиеся грани имеют равные
шансы на появление и
правильный \textbf{белый} кубик.  Петя случайным образом выбирает кубик и подбрасывает его.
\begin{enumerate}
\item Вероятность того, что выпадет «6», равна \underline{\hspace{2cm}}
\item Вероятность того, что Петя взял красный кубик, если известно, что выпала шестерка,
равна \underline{\hspace{2cm}}
\item Если бы в эксперименте Петя подбрасывал  бы кубик не один раз, а 60 раз,
то безусловное математическое ожидание количества выпавших шестёрок равнялось бы
\underline{\hspace{2cm}}
\end{enumerate}

\begin{comment}
\item Неразменный пятак всегда выпадает «орлом». У Александра Привалова в кармане
один неразменный пятак и два обычных, равновероятно выпадающих «орлом» и «решкой».
Привалов достаёт одну из монет наугад не глядя.
\begin{enumerate}
\item Вероятность того, что он достанет неразменный пятак равна \underline{\hspace{2cm}} % 1/3
\item Не глядя на монету, Привалов подкидывает её. Вероятность того, что она выпадет
«орлом», равна \underline{\hspace{2cm}} % 2/3
%\item Если бы эту случайную монету подкинуть не один раз, а 10, то математическое ожидание числа «орлов» равнялось бы \underline{\hspace{2cm}} % 20/3
\item Наконец Привалов глядит на упавшую монету и видит, что выпал «орёл».
Вероятность того, что монета — неразменный пятак, равна \underline{\hspace{2cm}}
\end{enumerate}
\end{comment}

\item Винни-Пуху снится сон, будто он спустился в погреб, а там бесконечное
количество горшков. Каждый из них независимо от других может оказаться либо
пустым с вероятностью $0.8$, либо с мёдом с вероятностью $0.2$. Винни-Пух начинает
перебирать горшки по очереди в поисках полного. Хотя у него в голове и опилки,
Винни-Пух два раза в один и тот же горшок заглядывать не будет.
\begin{enumerate}
\item Вероятность того, что все горшки окажутся пустыми равна \underline{\hspace{2cm}}
\item Вероятность того, что полный горшок будет найден ровно с шестой попытки, равна \underline{\hspace{2cm}}
\item Вероятность того, что полный горшок будет найден на шестой попытке или ранее, равна \underline{\hspace{2cm}}
%\item Математическое ожидание числа перебранных горшков равняется \underline{\hspace{2cm}} % 5
\end{enumerate}

\item На самом деле у Винни-Пуха в погребе стоит 10 горшков. Каждый из них независимо
от других может оказаться либо пустым с вероятностью $0.8$, либо с мёдом с вероятностью
$0.2$.
\begin{enumerate}
\item Все десять горшков окажутся пустыми с вероятностью \underline{\hspace{2cm}}
\item Ровно $7$ горшков из десяти окажутся пустыми с вероятностью \underline{\hspace{2cm}}
\item Математическое ожидание числа горшков с мёдом равно \underline{\hspace{2cm}}
\end{enumerate}


\begin{comment}
\item Внутри треугольника с вершинами $(0,0)$, $(2,5)$ и $(8,0)$ случайно
равномерно по площади выбирается точка. Пусть $X$ и $Y$ — абсцисса и ордината
этой случаной точки.
\begin{enumerate}
\item Вероятность того, что $X>5$ равна \underline{\hspace{2cm}}.
\item Вероятность того, что $X>5$ и одновременно $Y<3$ равна \underline{\hspace{2cm}}.
\item Вероятность того, что $X>5$ если известно, что $Y<3$ равна \underline{\hspace{2cm}}.
\item События $X>5$ и $Y<3$ являются \underline{\hspace{1cm}}висимыми.
\item Функция плотности величины $X$ равна \underline{\hspace{2cm}}
\end{enumerate}
\end{comment}

\item В галактике Флатландии все объекты двумерные. На планету Тау-Слона (окружность)
в случайных точках независимо друг от друга садятся три корабля. Любые два корабля
могут поддерживать прямую связь между собой, если центральный угол между ними меньше
прямого.
\begin{enumerate}
\item Вероятность того, что первый и второй корабли могут поддерживать прямую
связь равна \underline{\hspace{2cm}}
\item Вероятность того, что все корабли смогут поддерживать прямую связь друг
с другом равна \underline{\hspace{2cm}}
\item Вероятность того, что все корабли смогут поддерживать прямую связь друг
с другом, если первый и второй корабль могут поддерживать прямую связь, равна
\underline{\hspace{2cm}}
\end{enumerate}
Подсказка: во Флатландии хватит рисунка на плоскости, ведь координату третьего
корабля можно принять за\ldots

\item Время (в часах), за которое студенты выполняют экзаменационное задание
является случайной величиной $X$ с функцией плотности
\[
f(x)=\begin{cases}
3x^2, \, \text{ если } x \in [0;1] \\
0, \, \text{ иначе }
\end{cases}
\]

\begin{enumerate}
\item Функция распределения случайной величины $X$ равна \underline{\hspace{2cm}}
\item Вероятность того, что случайно выбранный студент закончит работу менее чем
за полчаса равна \underline{\hspace{2cm}}.
\item Медиана распределения равна \underline{\hspace{2cm}}
\item Вероятность того, что студент, которому требуется по меньшей мере 15 минут
для выполнения задания, справится с ним более, чем за 30 минут, равна \underline{\hspace{2cm}}
\item Функция распределения случайной величины $Y=1/X$ равна \underline{\hspace{2cm}}
\item Функция плотности случайной величины $Y=1/X$ равна \underline{\hspace{2cm}}
\end{enumerate}
\end{enumerate}

\subsubsection*{Командный тур}

\begin{enumerate}
\item Восьминогий Кракен. У Кракена 8 ног-шупалец. Если отрубить одно щупальце,
то в замен него с вероятностью $1/4$ вырастает новое; с вероятностью $1/4$ вырастает
два новых; с вероятностью $1/2$, слава Океану, не вырастает ничего.

Против Кракена бьётся сам Капитан! Он наносит точные удары и безупречно умело
уворачивается от ударов Кракена.
\begin{enumerate}
\item Какова вероятность того, что Капитан победит, отрубив ровно 10 щупалец?
\item Какова вероятность того, что бой Кракена и Капитана продлится вечно?
\item Сколько щупалец в среднем отрубит Капитан прежде чем победит?
\end{enumerate}

\item Разбавленный ром. Пират Злопамятный Джо очень любит неразбавленный ром. Из-за
того, что он много пьёт, у него проблемы с памятью, и он помнит не
больше, чем три последних пинты. Хозяин таверны с вероятностью $1/4$ разбавляет
каждую подаваемую пинту рома. Если по ощущением Джо половина выпитых
пинт или больше была разбавлена, то он разносит таверну к чертям собачьим.
\begin{enumerate}
\item Какова вероятность того, что хозяин таверны не успеет подать Джо третью пинту рома?
\item Сколько в среднем пинт выпьет Джо, прежде чем разнесёт таверну?
\end{enumerate}

\item $XY$ в степени $Z$. Чтобы поступить на службу Её Величества, пиратам предлагается
следующая задача. Случайные величины $X$, $Y$ и $Z$ равномерны на отрезке $[0;1]$
и независимы.
\begin{enumerate}
\item Найдите функцию распределения случайной величины $-\ln X$
\item Найдите функцию распределения случайной величины $-(\ln X + \ln Y)$
\item Найдите функцию распределения случайной величины $-Z(\ln X + \ln Y)$
\item Какое распределение имеет случайная величина $(XY)^Z$?
\end{enumerate}

\item Тортики. Пираты очень любят тортики и праздновать день рождения! Если хотя
бы у одного пирата на корабле день рождения, то все, включая капитана, празднуют
и кушают тортики. Корабль в праздничный день дрейфует под действием ветра и не факт,
что в нужном направлении.
\begin{enumerate}
\item Сколько пиратов нужно нанять капитану, чтобы ожидаемое количество праздничных
дней было равно 100?
\item Сколько пиратов нужно нанять капитану, чтобы максимизировать ожидаемое количество
рабочих пирато-дней (произведение числа пиратов на число рабочих дней)?
\end{enumerate}

\item Девятый вал. На побережье пиратского острова одна за одной набегают волны.
Высота каждой волны — равномерная на $[0;1]$ случайная величина. Высоты волн независимы.
Пираты называют волну «большой», если она больше предыдущей и больше следующей.
Пираты называют волну «рекордной», если она больше всех предыдущих волн от начала
наблюдения. Обозначим события $B_i= \{ i\text{-ая волна была большой} \}$ и
$R_i=\{ i\text{-ая волна была рекордной} \}$.
\begin{enumerate}
\item Найдите $\P(R_{100})$, $\P(B_{100})$
\item Капитан насчитал 100 волн. Сколько в среднем из них были «рекордными»?
\item Найдите $\P(R_{99} | R_{100})$, $\P(R_{100}|B_{100})$
\end{enumerate}

\item Три сундука. Три пирата, Генри Рубинов, Френсис Пиастров и Эдвард Золотов
играют одной командой в игру. В комнате в ряд, слева направо, стоят в случайном
порядке три закрытых внешне неотличимых сундука: с рубинами, пиастрами и золотом.
Общаться после начала игры они не могут, но могут заранее договориться о стратегии.
Они заходят в комнату по очереди. Каждый из них может открыть два сундука по своему
выбору. После каждого пирата комната возвращается уборщицей идеально точно в исходное
состояние. Если Рубинов откроет коробку с рубинами, Писатров — с пиастрами, а Золотов —
с золотом, то их команда выигрывает. Если хотя бы один из пиратов не найдёт свою цель,
то их команда проигрывает.
\begin{enumerate}
\item Какова вероятность выигрыша, если все пираты пробуют открыть первый и второй сундуки?
\item Какова оптимальная стратегия?
\item Какова вероятность выигрыша при использовании оптимальной стратегии?
\end{enumerate}
\end{enumerate}


\newpage
\subsection[2014-2015]{\hyperref[sec:sol_kr_01_ip_2014_2015]{2014-2015}}
\label{sec:kr_01_ip_2014_2015}

\subsubsection*{ Часть 1}

\begin{enumerate}
%\item Винни-Пух собирается играть в Пустяки и готовит для игры палочки. Он нашел палку длиной 1 м, а дальше поступает следующим образом. Разламывает палку равномерно в случайном месте, одну полученную часть использует для игры, а вторую снова случайным образом делит на две части. Далее одну новую часть Винни-Пух снова использует для игры, а вторую новую часть снова делит на две. И так далее. Обозначим $X_i$ — длину палочки, использованной Винни-Пухом в $i$-ых Пустяках.

%Найдите функцию плотности $X_i$, $\E(X_i)$, $\Var(X_i)$

\item Вася купил два арбуза у торговки тёти Маши и один арбуз у торговки тёти Оли.
Арбузы у тёти Маши спелые с вероятностью 90\% (независимо друг от друга), арбузы
у тёти Оли спелые с вероятностью 70\%.
\begin{enumerate}
\item Какова вероятность того, что все Васины арбузы спелые?
\item Придя домой Вася выбрал случайным образом один из трех арбузов и разрезал его.
Какова вероятность того, что это арбуз от тёти Маши, если он оказался спелым?
\item Какова вероятность того, что второй и третий съеденные Васей арбузы были от
тёти Маши, если все три арбуза оказались спелыми?
\end{enumerate}


\item В большой большой стране живет очень большое количество $n»0$ семей.
Количества детей в разных семьях независимы. Количество детей в каждой семье —
случайная величина с распределением заданным табличкой:

\begin{center}
\begin{tabular}{ccccc}
\toprule
$x$ & $0$ & $1$ & $2$ & $3$ \\ \midrule
$\P(X=x)$ & $0.1$ & $0.3$ & $0.2$ & $0.4$ \\ \bottomrule
\end{tabular}
\end{center}

\begin{enumerate}
\item Исследователь Афанасий выбирает одну семью из всех семей наугад, пусть $X$ —
число детей в этой семье. Найдите $\E(X)$ и $\Var(X)$.
\item Исследователь Бенедикт выбирает одного ребенка из всех детей наугад, пусть $Y$ —
число детей в семье этого ребёнка. Как распределена величина $Y$? Что больше,
$\E(Y)$ или $\E(X)$?
\end{enumerate}

\item Функция плотности случайной величины $X$ имеет вид
\[
f(x)=
\begin{cases}
\frac{3}{8} x^2, \text{ если } x\in [0;2] \\
0, \text{ иначе }
\end{cases}
\]
\begin{enumerate}
\item Не производя вычислений найдите $\int_{-\infty}^{+\infty}f(x)\,dx$
\item Найдите $\E(X)$, $\E(X^2)$ и дисперсию $\Var(X)$
\item Найдите $\P(X>1.5)$, $\P(X>1.5 \mid X>1)$
\item При каком $c$ функция $g(x)=c x f(x)$ будет функцией плотности некоторой
случайной величины?
\end{enumerate}

\item Известно, что  $\E\left(Z\right)=-3$, $\E\left(Z^{2} \right)=15$,
$\Var\left(X+Y\right)=20$  и  $\Var\left(X-Y\right)=10$.
\begin{enumerate}
\item Найдите $\Var\left(Z\right)$, $\Var\left(4-3Z\right)$ и
$\E\left(5+3Z-Z^{2} \right)$.
\item Найдите $\Cov\left(X,Y\right)$ и $\Cov\left(6-X,3Y\right)$.
\item Можно ли утверждать, что случайные величины $X$ и $Y$ независимы?
\end{enumerate}

\item Листая сборник задач по теории вероятностей Вася наткнулся на задачу:

\fbox{%
\parbox{15cm}{%
Какова вероятность того, что наугад выбранный ответ на этот вопрос окажется верным?

1) $0.25$		2) $0.5$		3) $0.6$		4) $0.25$ }
}

Чему же равна вероятность выбора верного ответа?

\item Книга в 500 страниц содержит 400 опечаток. Предположим, что каждая из них
независимо от остальных опечаток может с одинаковой вероятностью оказаться на любой
странице книги.
\begin{enumerate}
\item Определите вероятность того, что на 13-й странице будет не менее двух опечаток,
в явном виде и с помощью приближения Пуассона.
\item Определите наиболее вероятное число, математическое ожидание и дисперсию
числа опечаток на 13-ой странице.
\item Является ли 13-ая страница более «несчастливой», чем все остальные (в том
смысле, что на 13-ой странице ожидается большее количество очепяток, чем на любой другой)?
\end{enumerate}

\item Вася случайным образом посещает лекции по ОВП (Очень Важному Предмету).
С вероятностью $0.9$ произвольно выбранная лекция полезна, и с вероятностью $0.7$
она интересна. Полезность и интересность — независимые друг от друга и от номера
лекции свойства. Всего Вася прослушал 30 лекций.
\begin{enumerate}
\item Определите математическое ожидание и дисперсию числа полезных лекций,
прослушанных Васей
\item Определите математическое ожидание числа одновременно бесполезных и
неинтересных лекций, прослушанных Васей, и математическое ожидание числа лекций,
обладающих хотя бы одним из свойств (полезность, интересность).
\end{enumerate}
\item Функция распределения случайной величины X задана следующей формулой:
 \[
 F(x)=\frac{ae^x}{1+e^x}+b
 \]
Определите: константы $a$ и $b$, математическое ожидание и третий начальный
момент случайной величины $X$, медиану и моду распределения.

%\item Вы хотите приобрести некую фирму. Стоимость фирмы для ее нынешних владельцев — случайная величина, равномерно распределенная на отрезке $[0;1]$. Вы предлагаете владельцам продать ее за называемую Вами сумму. Владельцы либо соглашаются, либо нет. Если владельцы согласны, то Вы платите обещанную сумму и получаете фирму. Когда фирма переходит в Ваши руки, ее стоимость сразу возрастает на 20\%.

%\begin{enumerate}
%\item Чему равен Ваш ожидаемый выигрыш, если Вы предлагаете цену 0.5?
%\item Какова оптимальная предлагаемая цена?
%\end{enumerate}
\end{enumerate}

\subsubsection*{Часть 2}

\begin{enumerate}
\item Маша подкидывает кубик до тех пор, пока два последних броска в сумме не
дадут\footnote{Изначально вместо 12 задумывалось число 10, но  опечатка была
замечена поздно, поэтому решение приводится для 12.} 12. Обозначим случайные
величины: $N$ — количество бросков, а $S$ — сумма набранных за всю игру очков.
\begin{enumerate}
\item Найдите $\P(N=2)$, $\P(N=3)$
\item Найдите $\E(N)$, $\E(S)$, $\E(N^2)$
\item Пусть $X_N$ — результат последнего броска. Как распределена случайная
величина $X_N$?
\end{enumerate}


\item В столовую пришли 30 студентов и встали в очередь в случайном порядке.
Среди них есть Вовочка и Машенька. Пусть $V$ — это количество человек в очереди
перед Вовочкой, а $M\geq 0$ — количество человек между Вовочкой и Машенькой.
\begin{enumerate}
\item Найдите $\P(V=1)$, $\P(M=1)$, $\P(M=V)$
\item Найдите $\E(V)$, $\E(M)$, $\Var(M)$
\end{enumerate}

\item Польский математик Стефан Банах имел привычку носить в каждом из двух
карманов пальто по коробку спичек. Всякий раз, когда ему хотелось закурить трубку,
он выбирал наугад один из коробков и доставал из него спичку. Первоначально в каждом
коробке было по $n$ спичек. Но когда-то наступает момент, когда выбранный наугад
коробок оказывается пустым.
\begin{enumerate}
\item Какова вероятность того, что в другом коробке в этот момент осталось ровно
$k$ спичек?
\item Каково среднее количество спичек в другом коробке?
\end{enumerate}

\item Производитель чудо-юдо-йогуртов наклеивает на каждую упаковку одну из 50
случайно выбираемых наклеек. Покупатель собравший все виды наклеек получает приз
от производителя. Пусть $X$ — это количество упаковок йогурта, которое нужно купить,
чтобы собрать все наклейки.

Найдите $\P(X=50)$, $\E(X)$, $\Var(X)$

Hint: $\ln(50)\approx 3.91$, а $\sum_{i=1}^n \frac{1}{i} \approx \int_1^n \frac{1}{x}\, dx$ :)

\item В самолете $n$ мест и все билеты проданы. Первой в очереди на посадку стоит
Сумасшедшая Старушка. Сумасшедшая Старушка несмотря на билет садиться на случайно
выбираемое место. Каждый оставшийся пассажир садится на своё место, если оно свободно
и на случайное выбираемое место, если его место уже кем-то занято.
\begin{enumerate}
\item Какова вероятность того, что все пассажиры сядут на свои места?
%\item Какова вероятность того, что второй пассажир в очереди сядет на своё место?
\item Какова вероятность того, что последний пассажир сядет на своё место?
\item Чему примерно равно среднее количество пассажиров севших на свои места?
\end{enumerate}
\end{enumerate}



\newpage
\subsection[2013-2014]{\hyperref[sec:sol_kr_01_ip_2013_2014]{2013-2014}}
\label{sec:kr_01_ip_2013_2014}

\subsubsection*{Часть 1}

\begin{enumerate}

\item В жюри три человека, они должны одобрить или не одобрить конкурсанта.
Два члена жюри независимо друг от друга одобряют конкурсанта с одинаковой
вероятностью $p$. Третий член жюри  для вынесения решения бросает правильную монету.
Окончательное решение выносится большинством голосов.
\begin{enumerate}
\item С какой вероятностью жюри одобрит конкурсанта?
\item Что выгоднее для  конкурсанта: чтобы решение принимало данное жюри, или
чтобы решение принимал один человек, одобряющий с вероятностью $p$?
\end{enumerate}

\item Вероятность застать Васю на лекции зависит от того, пришли ли на лекцию
Маша и Алена. Данная вероятность равна $p$, если девушек нет; $5p$ — если обе
девушки пришли на лекцию; $3p$ — если пришла только Маша и $2p$ — если пришла
только Алена. Маша и Алена посещают лекции независимо друг от друга с вероятностями
$0.6$ и $0.3$ соответственно.
\begin{enumerate}
\item Определите вероятность того, что на лекции присутствует Алёна,
если в аудитории есть Вася.
\item Кого чаще можно застать на тех лекциях, на которых присутствует Вася:
Машу или Алёну?
\item При каком значении $p$ Вася посещает половину всех лекций?
\end{enumerate}

\item Страховая компания страхует туристов, выезжающих за границу, от невыезда
и наступления страхового медицинского случая за границей. Застраховано 100 туристов.
Вероятность «невыезда» за границу случайно выбранного туриста — $0.002$,
а страховые выплаты в этом случае — 2000 у.е.; вероятность обращения за медицинской
помощью за границей — $0.01$, а страховые выплаты — 3000 у.е.
\begin{enumerate}
\item Определите вероятность того, что ровно пятеро туристов не смогут выехать за границу.
\item Найдите математическое ожидание, дисперсию и наиболее вероятное число не выехавших туристов.
\item Вычислите математическое ожидание и дисперсию величины совокупных страховых выплат
\item Вычислите ковариацию между выплатами по двум видам страхования.
\end{enumerate}

\item Известно, что  $\E(X)=-1$, $\E(Y)=1$, $\Var(X)=9$, $\Var(Y)=4$, $\Corr(X,Y)=1$.
Найдите
\begin{enumerate}
\item $\E(Y-2X-3)$, $\Var(Y-2X-3)$
\item  $\Corr(Y-2X-3,X)$
\item Можно ли выразить $Y$ через $X$? Если да, то запишите уравнение связи.
\end{enumerate}

\item Совместное распределение доходов акций двух компаний $Y$ и $X$ задано в виде
таблицы
\begin{center}
\begin{tabular}{@{}cccc@{}}
\toprule
    & $X=-1$ & $X=0$ & $X=1$ \\ \midrule
$Y=-1$ & $0.1$  & $0.2$   & $0.2$ \\
$Y=1$ & $0.2$  & $0.1$ & $0.2$ \\ \bottomrule
\end{tabular}
\end{center}

Найдите:
\begin{enumerate}
\item Частные распределения случайных величин $X$ и $Y$
\item $\Cov(X,Y)$
\item Можно ли утверждать, что случайные величины $X$ и $Y$ зависимы?
\item У инвестора портфель, в котором доля акций $X$ составляет $\alpha$,
а доля акций $Y$ — $(1-\alpha)$. Каковы должны быть доли, чтобы риск портфеля
(дисперсия дохода) был бы минимальным?
\item Условное распределение случайной величины $X$ при условии $Y=-1$.
\item Условное математическое ожидание $\E(X\mid Y=-1)$
\end{enumerate}
\item Докажите, что из сходимости в среднем порядка $s>0$ следует сходимость
по вероятности.
\end{enumerate}


\subsubsection*{Часть 2}

\begin{enumerate}
\item Муравей находится внутри спичечного коробка, в вершине $A$. В противоположной
вершине $B$ есть маленькая дырочка, через которую муравей сможет выбраться на поверхность.
В вершине $C$, соседней с вершиной $A$, лежит крупинка сахара. Муравей ползает
только по рёбрам коробка, выбирая каждый раз равновероятно одно из доступных в
вершине рёбер наугад. Например, он может поползти обратно.
\begin{enumerate}
\item Какова вероятность того, что муравей найдет крупинку сахара до того, как выберется?
\item Сколько в среднем перемещений понадобится муравью, чтобы выбраться?
\item Какова дисперсия количества перемещений, которые понадобятся муравью, чтобы
выбраться?
\end{enumerate}

\item В очереди стояло $20$ человек, когда касса внезапно закрылась. Поэтому $10$
случайных людей из очереди решили покинуть очередь. В результате этого очередь
оказалась разбита на случайное число кусков $X$. Найдите $\E(X)$, $\Var(X)$.

\item Предположим, что три возможных генотипа \verb|aa|, \verb|Aa| и \verb|AA|
изначально встречаются с частотами $p_1$, $p_2$ и $p_3$, где $p_1 + p_2 + p_3 = 1$.
Ген не сцеплен с полом, поэтому частоты $p_1$, $p_2$ и $p_3$ одинаковы для мужчин
и для женщин.
\begin{enumerate}
\item У семейных пар из этой популяции рождаются дети. Назовём этих детей первым
поколением. Каковы частоты для трёх возможных генотипов в первом поколении?
\item У семейных пар первого поколения тоже рождаются дети. Назовём этих детей
вторым поколением. Каковы частоты для трёх возможных генотипов во втором поколении?
\item Каковы частоты для трёх возможных генотипов в $n$-ном поколении?
\item Заметив явную особенность предыдущего ответа сформулируйте теорему о равновесии
Харди-Вайнберга. Прокомментируйте утверждение: «Любой рецессивный ген со временем
исчезнет».
\end{enumerate}

\item Световая волна может быть разложена на две поляризованные составляющие,
вертикальную и горизонтальную. Поэтому состояние отдельного поляризованного фотона
может быть описано\footnote{На самом деле внутренний мир фотона гораздо разнообразнее.}
углом $\alpha$. Поляризационный фильтр описывается углом поворота $\theta$. Фотон
в состоянии $\alpha$ задерживается поляризационным фильтром с параметром $\theta$
с вероятностью $p=\sin^2(\alpha-\theta)$ или проходит сквозь фильтр с вероятностью
$1 - p$, переходя при этом в состояние $\theta$.

\begin{enumerate}
\item Какова вероятность того, что поляризованный фотон в состоянии $\alpha$ пройдёт
сквозь фильтр с параметром $\theta=0$?
\item Имеется два фильтра и поляризованный фотон в состоянии $\alpha$. Первый
фильтр — с $\theta=0$, второй — c $\theta=\pi/2$. Какова вероятность того, что
фотон пройдет через оба фильтра?
\item Имеется три фильтра и поляризованный фотон в состоянии $\alpha$. Первый
фильтр — с $\theta=0$, второй — c $\theta=\beta$, третий — с $\theta=\pi/2$.
Какова вероятность того, что фотон пройдет через все три фильтра? При каких $\alpha$
и $\beta$ она будет максимальной и чему при этом она будет равна?
\item Объясните следующий фокус. Фокусник берет два специальных стекла и видно,
что свет сквозь них не проходит. Фокусник ставит между двумя стёклами третье, и
свет начинает проходить через три стекла.
\end{enumerate}
\end{enumerate}

\newpage
\thispagestyle{empty}
\section{Контрольная работа 2}



\subsection[2017-2018]{\hyperref[sec:sol_kr_02_2017_2018]{2017-2018}}
\label{sec:kr_02_2017_2018}

\subsubsection*{Минимум}
% 2 + 4 + 14 + 16

\begin{enumerate}
\item Приведите определение условной вероятности случайного события, формулу Байеса.
\item Сформулируйте определение и свойства функции плотности случайной величины.
\item Сформулируйте определение  условного математического ожидания $\E(Y|X=x)$ для совместного дискретного и совместного абсолютно непрерывного распределений.
\item Сформулируйте неравенство Чебышёва и неравенство Маркова.

\item Задана таблица совместного распределения случайных величин $X$ и $Y$.
\begin{center}
\begin{tabular}{lccc}
\toprule
                       & $Y=-1$  & $Y=0$   & $Y=1$   \\
 \midrule
$X=0$                 & $0.2$ & $0.1$ & $0.3$ \\
 $X=1$                 & $0.2$ & $0.1$ & $0.1$ \\
 \bottomrule
\end{tabular}
\end{center}


\begin{enumerate}
    \item Найдите $F_{X,Y}(0, 0)$;
    \item Найдите $\E(X)$, $\E(X^2)$, $\E(Y)$, $\E(Y^2)$;
    \item Найдите $\Var(X)$, $\Var(Y)$;
    \item Найдите $\Cov(X, Y)$, $\Corr(X, Y)$
\end{enumerate}
\item Плотность распределения случайного вектора $(X,Y)$ имеет вид
\[
f_{X,Y}(x,y) =
\begin{cases}
\frac{4x+10y}{7}, & \text{при } (x,y) \in [0;1] \times [0;1] \\
0 , & \text{при } (x,y) \not\in [0;1] \times [0;1] \\
\end{cases}
\]

\begin{enumerate}
\item Найдите $\P(X \leq Y)$;
\item Найдите функцию плотности $f_X(x)$;
\item Найдите $\E(X)$, $\E(Y)$ и $\Cov(X, Y)$;
\item Являются ли случайные величины $X$ и $Y$ независимыми?
\end{enumerate}


\end{enumerate}

\subsubsection*{Задачи}

\begin{enumerate}[resume]

\item Статистика авиакомпании «А» за много лет свидетельствует о том, что 10\% людей, купивших билет на самолет, не являются на рейс. Авиакомпания продала 330 билетов на 300 мест.
\begin{enumerate}
\item Какова вероятность, что всем явившимся на рейс пассажирам хватит места?
\item Укажите наибольшее число билетов, которое можно продавать на 300 мест, чтобы случаи переполнения случались не чаще, чем на одном из десяти рейсов.
\end{enumerate}

\item Сегодня акция компании «Ух» стоит 1 рубль. Каждый день акция может с вероятностью 0.7 вырасти на 1\%, с вероятностью 0.2999 упасть на 1\% и с вероятностью 0.0001 обесцениться (упасть на 100\%).
\begin{enumerate}
\item Считая изменение цены акции независимыми, найдите математическое ожидание её стоимости через 20 торговых дней.
\item Найдите предел по вероятности среднего изменения цены акции в процентах на бесконечном промежутке времени (Ответ обоснуйте).
\item Найдите математическое ожидание цены акции на бесконечном промежутке времени.
\item Инвестор вложил все свои средства в акции компании «Ух». Найдите вероятность его разорения на бесконечном промежутке времени.
\end{enumerate}
\end{enumerate}


\newpage
\subsection[2016-2017]{\hyperref[sec:sol_kr_02_2016_2017]{2016-2017}}
\label{sec:kr_02_2016_2017}


\textbf{Неравенства Берри–Эссеена:} Для любых $n \in \mathbb{N}$ и всех $x \in \mathbb{R}$ имеет место оценка:
\[
    \bigl|F_{S_n^{*}}(x) - \Phi(x)\bigr| \leq 0.48 \cdot \frac{\E(|\xi_i - \E\xi_i|^3)}{\Var^{3/2}(\xi_i)\cdot\sqrt{n}} \text{,}
\]
где $\Phi(x) = \int_{-\infty}^{x}\frac{1}{\sqrt{2\pi}}e^{-\frac{t^2}{2}}\,dt$, \; $S_n^* = \frac{S_n - \E(S_n)}{\sqrt{\Var(S_n)}}$, \; $S_n = \xi_1 + \ldots + \xi_n$

\textbf{Распределение Пуассона:} Случайная величина $\xi$ имеет распределение Пуассона с параметром $\lambda > 0$,  если она принимает целые неотрицательные значения с вероятностями $\P(\{\xi = k\}) = \frac{\lambda^k}{k!}e^{-\lambda}$. Приличным студентам должно быть известно, что в этом случае $\E(\xi) = \Var(\xi) = \lambda$.

\begin{enumerate}
\item Пусть $\E(\xi) = 1$, $\E(\eta) = -2$, $\Var(\xi) = 1$, $\E(\eta^2) = 8$, $\E(\xi \eta) = -1$. Найдите
\begin{enumerate}
\item $\E(2\xi-\eta+1)$, $\Cov(\xi, \,\eta)$, $\Corr(\xi, \,\eta)$,  $\Var(2\xi-\eta+1)$;
\item $\Cov(\xi+\eta, \,\xi+1)$, $\Corr(\xi+\eta, \,\xi+1)$, $\Corr(\xi+\eta-24, \,365 - \xi - \eta)$, $\Cov(2016\cdot\xi, \, 2017)$.
\end{enumerate}

\item
Совместное распределение доходностей акций двух компаний задано с помощью таблицы:

\begin{center}
\begin{tabular}{ccc}
\toprule
         & $\eta=-1$ & $\eta=1$ \\
\midrule
$\xi=-1$  & $0.1$       & $0.2$   \\
$\xi=0$   & $0.2$       & $0.2$   \\
$\xi=2$   & $0.2$       & $0.1$   \\
\bottomrule
\end{tabular}
\end{center}

\begin{enumerate}
  \item Найдите частные распределения случайных величин $\xi$ и $\eta$.
  \item Найдите $\Cov(\xi,\,\eta)$.
  \item Сформулируйте определение независимости дискретных случайных величин.
  \item Являются ли случайные величины $\xi$ и $\eta$ независимыми?
  \item Найдите условное распределение случайной величины $\xi$, если $\eta = 1$.
  \item Найдите условное математическое ожидание случайной величины $\xi$, если $\eta = 1$.
  \item Найдите математическое ожидание и дисперсию величины $\pi = 0.5\, \xi + 0.5\, \eta$.
  \item Рассмотрим портфель, в котором $\alpha$ — доля акций с доходностью $\xi$ и $(1 - \alpha)$ — доля акций с доходностью $\eta$. Доходность этого портфеля есть случайная величина
  \[\pi(\alpha) = \alpha \xi + (1-\alpha)\eta.\]
  Найдите такую долю $\alpha \in [0;\,1]$, при которой доходность портфеля $\pi(\alpha)$ имеет наименьшую дисперсию.
\end{enumerate}

\item Число посетителей сайта \url{pokrovka11.wordpress.com} за один день имеет распределение Пуассона с математическим ожиданием 250.
\begin{enumerate}
  \item Сформулируйте неравенство Маркова. При помощи данного неравенства оцените вероятность того, что за один день сайт посетят более 500 человек.
  \item Сформулируйте неравенство Чебышева. Используя данное неравенство, определите наименьшее число дней, при котором с вероятностью не менее 99\% среднее за день число посетителей будет отличаться от 250 не~более чем на 10.
  \item Решите предыдущий пункт с помощью центральной предельной теоремы.
  \item Сформулируйте закон больших чисел. Обозначим через $\xi_i$ число посетителей сайта за $i$-ый день. Найдите предел по вероятности последовательности $\frac{\xi_1^2 + \ldots + \xi_n^2}{n}$ при $n \rightarrow \infty$.
\end{enumerate}

\item Отведав медовухи, Винни–Пух совершает случайное блуждание на прямой. Он стартует из начала координат и в каждую следующую минуту равновероятно совершает шаг единичной длины налево или направо. Передвижения Винни-Пуха схематично изображены на следующем рисунке.
\begin{figure}[h]
    \noindent\centering{
    \includegraphics[width=80mm]{images/Winnie_the_Pooh_and_Medovuh.jpg}
    }
    \caption{Случайные бродилки.}
    \label{wun762hkej}
\end{figure}
\begin{enumerate}
  \item Сформулируйте центральную предельную теорему.
  \item При помощи центральной предельной теоремы оцените вероятность того, что ровно через час блужданий Винни-Пух окажется в области $(-\infty; \, -5]$.
  \item Используя неравенство Берри–Эссеена оцените погрешность вычислений предыдущего пункта.
\end{enumerate}


\item
Cлучайные величины $\xi$ и $\eta$ означают время безотказной работы рулевого управления и двигателя автомобиля соответственно. Время измеряется в годах. Совместная плотность имеет вид:
\[
f_{\xi, \,\eta}(x,\,y) =
\begin{cases}
0.005\,e^{-0.05\,x-0.1\,y} & \text{ при } x > 0, y > 0, \\
0                    & \text{ иначе.}
\end{cases}
\]

\begin{enumerate}
  \item Найдите частные плотности распределения случайных величин $\xi$ и $\eta$.
  \item Являются ли случайные величины $\xi$ и $\eta$ независимыми?
  \item Найдите вероятность того, что двигатель прослужит без сбоев более пяти лет.
  \item Найдите вероятность того, что двигатель прослужит без сбоев более восьми лет, если он уже проработал без сбоев три года.
  \item Найдите условное математическое ожидание безотказной работы рулевого управления, если двигатель проработал без сбоев пять лет,  $\E(\xi | \eta = 5)$.
  \item Найдите вероятность того, что рулевое управление проработает без сбоев на два года больше двигателя,  $\P(\{\xi - \eta > 2\})$.
\end{enumerate}

\item Бонусная задача

Случайная величина $\xi$ имеет плотность распределения
\[
    f_{\xi}(x) = \frac{1}{2} \cdot \frac{1}{\sqrt{2\pi}}e^{-\frac{(x-1)^2}{2}} + \frac{1}{2} \cdot \frac{1}{\sqrt{2\pi}}e^{-\frac{(x+1)^2}{2}} \text{.}
\]

\begin{enumerate}
\item Найдите $\E(\xi)$, $\E\left(\xi^2\right)$, $\Var(\xi)$.
\item Покажите, что функция $f_{\xi}(x)$, действительно, является плотностью распределения.
\end{enumerate}
\end{enumerate}


\newpage
\subsection[2015-2016]{\hyperref[sec:sol_kr_02_2015_2016]{2015-2016}}
\label{sec:kr_02_2015_2016}



\begin{enumerate}
\item Функция плотности случайного вектора $\xi=(\xi_1, \xi_2)^T$ имеет вид
\[
f(x,y)=\begin{cases}
0.5x + 1.5y, \text{ если } 0<x<1, \; 0<y<1 \\
0, \text{ иначе }
\end{cases}
\]
Найдите:
\begin{enumerate}
\item Математическое ожидание $\E(\xi_1 \cdot \xi_2)$
\item Условную плотность распределения $f_{\xi_1|\xi_2} (x|y)$
\item Условное математическое ожидание $\E(\xi_1| \xi_2=y)$
\item Константу $k$, такую, что функция $h(x,y)=kx\cdot f(x,y)$ будет являться совместной функцией плотности некоторой пары случайных величин
\end{enumerate}

\item На курсе учится очень много студентов. Вероятность того, что случайно выбранный студент по результатам рубежного контроля имеет хотя бы один незачет равна $0.2$. Пусть $\xi$ и $\eta$ — число студентов с незачетами и без незачетов в случайной группе из $10$ студентов. Найдите $\Cov(\xi,\eta)$, $\Corr(\xi,\eta)$, $\Cov(\xi-\eta,\xi)$. Являются ли случайные величины $\xi-\eta$ и $\xi$ независимыми?

\item Доходности акций компаний А и В – случайные величины $\xi$ и $\eta$. Известно, что $\E(\xi)=1$, $E(\eta)=1$, $\Var(\xi)=4$, $\Var(\eta)=9$, $\Corr(\xi,\eta)=-0.5$. Петя принимает решение потратить свой рубль на акции компании А, Вася — 50 копеек на акции компании А и 50 копеек на акции компании В, а Маша  принимает решение вложить свой рубль в портфель $R=\alpha\xi+(1-\alpha)\eta$, $(0 \leq \alpha \leq 1)$, обладающий минимальным риском. Найдите $\alpha$, ожидаемые доходности и риски портфелей Пети, Васи и Маши.

\item Будем считать, что рождение мальчика и девочки равновероятны.
\begin{enumerate}
\item Оцените с помощью неравенства Маркова вероятность того, что среди тысячи новорожденных младенцев, мальчиков будет более 75\%.
\item Оцените с помощью неравенства Чебышёва вероятность того, что доля мальчиков среди тысячи новорожденных младенцев будет отличаться от 0.5 более, чем на 0.25
\item С помощью теоремы Муавра-Лапласа вычислите вероятность из предыдущего пункта.
\end{enumerate}

\item Сейчас валютный курс племени «Мумба» составляет 100 оболов за один рубль. Изменение курса за один день — случайная величина $\delta_i$ с законом распределения:

\begin{center}
\begin{tabular}{lrrr}
\toprule
$x$ & $-1$ & $0$ & $2$ \\ \midrule
$\P(\delta_i = x)$ & $0.25$ & $0.5$ & $0.25$ \\
\bottomrule
\end{tabular}
\end{center}

Найдите вероятность того, что через полгода (171 день) рубль будет стоить более 250 оболов, если ежедневные изменения курса происходят независимо друг от друга.

\item \textbf{Бонусная задача}

Число посетителей, зашедших в магазин в течении дня — пуассоновская случайная величина с параметром $\lambda$. Каждый из посетителей совершает покупку с вероятностью $p$, не зависимо от других посетителей. Найдите математическое ожидание числа человек, совершивших покупку.

\end{enumerate}



\newpage
\subsection[2014-2015]{\hyperref[sec:sol_kr_02_2014_2015]{2014-2015}}
\label{sec:kr_02_2014_2015}


\begin{enumerate}
\item Ежемесячные расходы студенческой семьи Маши и Васи хорошо описываются случайным
вектором $(X,Y)$, ($X$ — расходы Маши, $Y$ — расходы Васи), имеющим равномерное
распределение в треугольнике, задаваемом ограничениями $\{0 \leq X, \; 0\leq Y,
\; X+Y \leq 1 \}$.

Найдите:

\begin{enumerate}
\item Вероятность того, что совокупные расходы превысят половину бюджета, $\P(X+Y>1/2)$
\item Плотность распределения расходов Васи.
\item Вероятность того, что Машины расходы составили менее трети бюджета, если
известно, что Вася израсходовал более половины семейного бюджета.
\item Условную плотность распределения и условное математическое ожидание расходов Маши,
при условии, что Вася израсходовал половину бюджета.
\item Математическое ожидание условного математического ожидания расходов Маши,
$\E(\E(X|Y))$
\item Коэффициент корреляции расходов Маши и Васи
\end{enumerate}

\item Задана последовательность независимых случайных величин $X_1$, $X_2$, \ldots

\begin{center}
\begin{tabular}{llll}
\toprule
$x_n$ & $-\sqrt{n}$ & $0$ & $\sqrt{n}$ \\
$\P(X_n = x_n)$ & $1/2n$ & $1-1/n$ & $1/2n$ \\
\bottomrule
\end{tabular}
\end{center}

\begin{enumerate}
\item Сформулируйте закон больших чисел. Выполняется ли для данной последовательности
закон больших чисел?
\item Запишите неравенство Чебышёва. Оцените вероятность того, что модуль среднего
значения по $n$ наблюдениям не превысит $1$, $\P\left(|\bar X_n| \leq 1\right)$
\item Сколько членов последовательности необходимо взять, чтобы вероятность того, что
модуль среднего значения не превысит $1$, была не менее $0.9$, $\P\left(|\bar X_n| \leq 1\right)\geq 0.9$
\end{enumerate}

\item Размер выплат каждому клиенту банка — случайная величина с математическим
ожиданием, равным 5000 ед. и среднеквадратическим отклонением, равным 2000 ед.
Выплаты отдельным клиентам независимы. Сколько должно быть наличных денег в банке,
чтобы с вероятностью 0.95 денег хватило на обслуживание 60 клиентов?

\item Рекламная компания хочет оценить вероятность $p$, с которой адресная реклама
приводит к заявке. С этой целью она  рассылает $n$ рекламных проспектов. Обозначим за
$\hat p$ отношение числа поданных заявок к числу разосланных проспектов $n$.
С помощью теоремы Муавра–Лапласа и неравенства Чебышёва определите:
\begin{enumerate}
\item  Сколько нужно разослать рекламных проспектов, для того чтобы $\hat p$ отличалось от
истинной вероятности $p$ не более, чем на $0.1$ с вероятностью не меньшей $0.99$
\item С какой точностью $\varepsilon$ удастся оценить $p$ с вероятностью $0.99$,
если разослана 1000 проспектов, то есть $\P(|\hat p - p|\leq \varepsilon)\geq 0.99$?
\end{enumerate}
\end{enumerate}



\newpage
\subsection[2013-2014]{\hyperref[sec:sol_kr_02_2013_2014]{2013-2014}}
\label{sec:kr_02_2013_2014}

\noindent Самая важная формула:
\[ \frac{1}{(\sqrt{2\pi})^n \sqrt{det(C)}} \cdot e^{-\frac{1}{2}\left(x-\mu\right)^T C^{-1}\left(x-\mu\right)} \]

\noindent Неравенство Берри-Эссеена:
\[ | \hat{F}_n (x) - \text{Ф}(x)| \leqslant \frac{C_0 \E|X_n - \mu|^3}{\sigma^3\sqrt{n}}, \;\;\; 0.4 <C_0<0.48\]

\begin{enumerate}
\item  Совместная функция плотности случайной величины $(X,Y)$ имеет вид:
\begin{equation*}
f(x,y) =
 \begin{cases}
   x+y &\text{ при }x \in (0,1),\;y \in (0,1) \\
   0 &\text{иначе};
 \end{cases}
\end{equation*}

Найдите:
\begin{enumerate}
\item $\P(Y<X^2)$
\item функцию плотности и математическое ожидание случайной величины $X$
\item условную функцию плотности и условное математическое ожидание случайной
величины $X$ при условии, что $Y=2$
\end{enumerate}

\item Случайный вектор $(X,Y)^T$ имеет двумерное нормальное распределение
с математическим ожиданием $(0,0)^T$ и ковариационной матрицей

\[
C = \begin{pmatrix}
9 & -1 \\
-1 & 4 \\
\end{pmatrix}
\];

Найдите:
\begin{enumerate}
\item $\P(X>1)$
\item $\P(2X+Y>3)$
\item $\P(2X+Y>3|X=1)$
\item $\P\left(\frac{X^2}{9}+\frac{Y^2}{4} >12\right)$
\item Запишите совместную функцию плотности  $(X,Y)^T$
\end{enumerate}

\item Вычислите:
\begin{enumerate}
\item $\P\left(\frac{X_1}{\sqrt{X_3^2+X_4^2+X_5^2}}>\frac{5}{4\sqrt{3}}\right)$
\item $\P\left(\frac{X_1+2X_2}{\sqrt{X_3^2+X_4^2+X_5^2}}<4.5\right)$
\item $\P\left(\frac{X_1^2}{X_2^2+X_3^2}>17\right)$
\end{enumerate}

\item Оценка за зачет по теории вероятности $i$-го студента — неотрицательная
случайная величина $X_i$ с $\E(X_i)=1/2$ и $\Var(X_i)=1/12$.
Для случайной выборки из $36$ студентов оцените или вычислите следующие вероятности
$\left(\bar{X} = \frac{1}{n} \sum \limits_{i=1}^n X_i \right)$:
\begin{enumerate}
\item $\P(|X_i-0.5|\geqslant 0.3)$
\item $\P(X_i\geqslant 0.8)$
\item $\P(\bar{X}\geqslant 0.8)$

Пусть дополнительно известно, что $X_i \sim U(0,1)$:
\item Вычислите вероятность $\P(|X_i-0.5|\geqslant 0.3)$
\item Оцените погрешность вычисленной вероятности $\P(\bar{X}\geqslant 0.8)$
\item Покажите, что средняя оценка за экзамен сходится по вероятности к $0.5$

\end{enumerate}

\item При проведении социологических опросов в среднем $20\,\%$ респондентов
отказываются отвечать на вопрос о личном доходе. Сколько нужно опросить человек,
чтобы с вероятностью $0.99$ выборочная доля отказавшихся отвечать на вопрос о доходе
не превышала $0.25$? Насколько изменится ответ на предыдущий вопрос, если средний
процент отказывающихся отвечать неизвестен?

\item Оценки за контрольную работу по теории вероятностей $6$ случайно выбранных
студентов оказались равны: $8$, $4$, $5$, $7$, $3$, $9$.
\begin{enumerate}
\item Выпишите вариационный ряд;
\item Постройте выборочную функцию распределения;
\item Вычислите значение выборочного среднего и выборочной дисперсии.
\end{enumerate}
\end{enumerate}



\newpage
\subsection[2012-2013]{\hyperref[sec:sol_kr_02_2012_2013]{2012-2013}}
\label{sec:kr_02_2012_2013}


\begin{enumerate}
\item Купчиха Сосипатра Титовна очень любит чаёвничать. Её чаепитие продолжается
случайное время $S$, имеющее равномерное распределение от $0$ до $3$ часов.
Встретив Сосипатру Титовну в пассаже на Петровке, её подруга Олимпиада Карповна
узнала, сколько длилось вчерашнее чаепитие Сосипатры Титовны. Решив, что такая
продолжительность чаепития является максимально возможной, Олимпиада Карповна
устраивает чаепитие, продолжающееся случайное время $T$, имеющее равномерное
распределение от $0$ до $S$ часов.
\begin{enumerate}
\item Найдите совместную функцию плотности величин $S$ и $T$
\item Найдите вероятность $\P(S>T)$
\item Найдите $\E\left(T^2\right)$
\end{enumerate}

\item Для случайно выбранного домохозяйства случайные величины $X$ и $Y$ принимают
значения, равные доле расходов на продукты питания и алкоголь плюс табак соответственно.
Случайный вектор $(X,Y)^T$  хорошо описывается двумерным нормальным законом распределения
с математическим ожиданием $(0.45, 0.16)^T$ и ковариационной матрицей
\[
C=0.144\cdot
\left(\begin{array}{cc}
1 & -0.9 \\
-0.9 & 1
\end{array}\right)
\]
Найдите:
\begin{enumerate}
\item Вероятность того, что домохозяйство тратит более половины своих доходов на питание.
\item Вероятность того, что домохозяйство тратит более половины своих доходов
на алкогольную и табачную продукцию и продукты питания.
\item Ожидаемую долю расходов на алкоголь и табак для домохозяйства, которое тратит
на питание четверть своих доходов.
\item Вероятность того, что домохозяйство из предыдущего пункта тратит более трети с
воих доходов на алкогольную и табачную продукцию.

% из-за коррелированности этот пункт не решается :(
\begin{comment}
\item  Характеристикой отклонения от типичного потребления выступает величина
\[
U=\sqrt{\frac{(X-\E(X))^2}{\Var(X)}+\frac{(Y-\E(Y))^2}{\Var(Y)}}
\]
Найдите $\P(U>3)$.
\end{comment}
\item Для доли расходов на питание вычислите центральный момент $2013$-го порядка.
\end{enumerate}

\item Вычислите (или оцените) вероятность того, что по результатам $4000$ бросаний
симметричной монеты, частота выпадения герба будет отличаться от $0.5$ не более,
чем на $0.01$. Решите задачу с помощью неравенства Чебышёва и с помощью ЦПТ.

\item Компания кабельного телевидения НВТ, Новая Вершина Телевидения, анализирует
возможность присоединения к своей сети пригородов N-ска. Опросы показали, что в
среднем каждые 3 из 10 семей жителей пригородов хотели бы стать абонентами сети.
Стоимость работ, необходимых для организации сети в любом пригороде оценивается
величиной 2\,080\,000 у.е. При подключении каждого пригорода НВТ надеется получить
1\,000\,000 у.е. в год от рекламодателей. Планируемая чистая прибыль от оплаты за
кабельное телевидение одной семьей в год равна 120 у.е.

Каким должно быть минимальное количество семей в пригороде для того, чтобы
с вероятностью $0.99$ расходы на организацию сети в этом пригороде окупились за год?

\item Оценки за контрольную работу по теории вероятностей 6 случайно выбранных
студентов оказались равны $8$, $5$, $6$, $7$, $3$, $9$.

\begin{enumerate}
\item Выпишите вариационный ряд
\item Постройте график выборочной функции распределения
\item Вычислите значение выборочного среднего и выборочной дисперсии.
\end{enumerate}
\end{enumerate}

% % !TEX root = ../probability_hse_exams.tex
\section{Промежуточные экзамены}



\subsection[2018-2019]{\hyperref[sec:sol_midterm_exam_2018_2019]{2018-2019}}
\label{sec:midterm_exam_2018_2019}


\begin{question}
Если \(F_X(x)\) — функция распределения случайной величины \(X\), то
для любых \(a\) и \(b\)
\begin{answerlist}
  \item \(\P (X = a) = F_X(a)\)
  \item \(\P (X > a) = 1 - F_X(a)\)
  \item \(\P (X < a) = 1 - F_X(a)\)
  \item \(\P (X > a) = F_X(a)\)
  \item \(\P (a<X\le b) = F_X(a)-F_X(b)\)
\end{answerlist}
\end{question}

\begin{solution}
\begin{answerlist}
  \item Bad answer :(
  \item Good answer :)
  \item Bad answer :(
  \item Bad answer :(
  \item Bad answer :(
\end{answerlist}
\end{solution}



\begin{question}
Для любой функции распределения \(F_X(x)\) верно, что
\begin{answerlist}
  \item она не убывает
  \item она возрастает
  \item \(F_{X}(x)\) принимает любые значения на \([0, +\infty)\)
  \item она не возрастает
  \item \(F_X(x) > 0\)
\end{answerlist}
\end{question}

\begin{solution}
\begin{answerlist}
  \item Good answer :)
  \item Bad answer :(
  \item Bad answer :(
  \item Bad answer :(
  \item Bad answer :(
\end{answerlist}
\end{solution}



\begin{question}
По случайной выборке из 100 наблюдений было оценено выборочное среднее
\(\bar{X}=20\)\\
и несмещенная оценка дисперсии \(\hat{\sigma}^2=25\). В рамках проверки
гипотезы \(H_0: \; \mu=15\) против альтернативной гипотезы
\(H_a: \; \mu>15\)\\
можно сделать следующее заключение
\begin{answerlist}
  \item Гипотеза \(H_0\) не отвергается на любом разумном уровне значимости
  \item Гипотеза \(H_0\) отвергается на уровне значимости 10\%, но не на уровне
значимости 5\%
  \item Гипотеза \(H_0\) отвергается на уровне значимости 5\%, но не на уровне
значимости 1\%
  \item Гипотеза \(H_0\) отвергается на любом разумном уровне значимости
  \item Гипотеза \(H_0\) отвергается на уровне значимости 20\%, но не на уровне
значимости 10\%
\end{answerlist}
\end{question}

\begin{solution}
\begin{answerlist}
  \item Неверно
  \item Неверно
  \item Неверно
  \item Отлично
  \item Неверно
\end{answerlist}
\end{solution}



\begin{question}
Ковариационная матрица вектора \(X=(X_1, X_2)\) имеет вид \[
\begin{pmatrix}
10 & 3 \\
3 & 8
\end{pmatrix}.
\] Дисперсия разности элементов вектора, \(\Var(X_1-X_2)\), равняется
\begin{answerlist}
  \item 18
  \item 6
  \item 2
  \item 15
  \item 12
\end{answerlist}
\end{question}

\begin{solution}
\begin{answerlist}
  \item Неверно
  \item Неверно
  \item Неверно
  \item Неверно
  \item Отлично
\end{answerlist}
\end{solution}



\begin{question}
Дисперсию случайной величины \(X\) можно найти, зная
\begin{answerlist}
  \item \(F_Y(x)\)
  \item \((\E(X))^2\) и \(\E(X)\)
  \item \(\E(XY)\) и \(\E(Y)\)
  \item \(\Cov(X,Y)\) и \(\Var(Y)\)
  \item \(\E(X^2)\) и \(\E(X)\)
\end{answerlist}
\end{question}

\begin{solution}
\begin{answerlist}
  \item Bad answer :(
  \item Bad answer :(
  \item Bad answer :(
  \item Bad answer :(
  \item Good answer :)
\end{answerlist}
\end{solution}



\begin{question}
При каждом ударе по воротам Месси забивает гол с вероятностью \(0.9\),
независимо от прошлых ударов. Вероятность того, что Месси забьет ровно
три мяча за пять ударов, равна
\begin{answerlist}
  \item \(C_{5}^{3}0.9^{3}0.1^2\)
  \item \(0.9^{4}0.1\)
  \item \(C_{5}^{2}0.1^30.9^5\)
  \item \(0.9^5\)
  \item \(\int_{3}^{5}\frac{1}{10}dx\)
\end{answerlist}
\end{question}

\begin{solution}
\begin{answerlist}
  \item Good answer :)
  \item Bad answer :(
  \item Bad answer :(
  \item Bad answer :(
  \item Bad answer :(
\end{answerlist}
\end{solution}



\begin{question}
Дисперсию случайной величины \(X\) можно найти, зная
\begin{answerlist}
  \item \(\Cov(X,Y)\) и \(\Var(Y)\)
  \item \(\E(XY)\) и \(\E(Y)\)
  \item \(\E(X^2)\) и \(\E(X)\)
  \item \(F_Y(x)\)
  \item \((\E(X))^2\) и \(\E(X)\)
\end{answerlist}
\end{question}

\begin{solution}
\begin{answerlist}
  \item Bad answer :(
  \item Bad answer :(
  \item Good answer :)
  \item Bad answer :(
  \item Bad answer :(
\end{answerlist}
\end{solution}



\begin{question}
Для энтропий пары случайных величин выполнено соотношение
\begin{answerlist}
  \item \(H(X) \cdot H(Y) = H(X, Y)\)
  \item \(H(Y|X) + H(X|Y) = H(X, Y)\)
  \item \(H(Y|X) + H(X) = H(X, Y)\)
  \item \(H(X) + H(Y) = H(X, Y)\)
  \item \(H(X\cdot Y) / H(X) = H(Y|X)\)
\end{answerlist}
\end{question}

\begin{solution}
\begin{answerlist}
  \item Bad answer :(
  \item Bad answer :(
  \item Good answer :)
  \item Bad answer :(
  \item Bad answer :(
\end{answerlist}
\end{solution}



\begin{question}
Величины \(\xi_1\), \(\xi_2\), \ldots~независимы и имеют таблицы
распределения

\begin{tabular}{ccc}
\toprule
  $x$                     & $-1$   & $1$   \\ 
  $\P(\xi_i = x)$        & $1/2$       & $1/2$   \\
\bottomrule
\end{tabular}

Рассмотрим их сумму \(S_n = \xi_1 + \ldots + \xi_n\). Предел
\(\lim_{n \to \infty}\P\Bigl(\frac{S_n - \E(S_n)}{\sqrt{\Var(S_n)}} > 2\Bigr)\)
равен
\begin{answerlist}
  \item \(1\)
  \item \(\int_{2}^{+\infty}\frac{1}{\sqrt{2\pi}}\,e^{-t^2/2}\,dt\)
  \item \(\int_{-\infty}^{2}\frac{1}{\sqrt{2\pi}}\,e^{-t^2/2}\,dt\)
  \item \(\int_{2}^{+\infty} e^{-t^2/2}\,dt\)
  \item \(0.5\)
\end{answerlist}
\end{question}

\begin{solution}
\begin{answerlist}
  \item Bad answer :(
  \item Good answer :)
  \item Bad answer :(
  \item Bad answer :(
  \item Bad answer :(
\end{answerlist}
\end{solution}



\begin{question}
Требуется проверить гипотезу о равенстве математических ожиданий по двум
нормальным независимым выборкам размером 33 и 16 наблюдений. Истинные
дисперсии по обеим выборкам известны, совпадают и равны 196. Разница
выборочных средних равна 1. Тестовая статистика может быть равна
\begin{answerlist}
  \item \(-1/4\)
  \item \(-1/14\)
  \item \(-1/7\)
  \item \(-1/49\)
  \item \(-1/2\)
\end{answerlist}
\end{question}

\begin{solution}
\begin{answerlist}
  \item Bad answer :(
  \item Bad answer :(
  \item Bad answer :(
  \item Bad answer :(
  \item Good answer :)
\end{answerlist}
\end{solution}




\begin{question}
Cовместная функция плотности случайных величин \(X\) и \(Y\) имеет вид
\[
f(x,y)=\begin{cases}
\frac{1}{4}xy, \; \text{ если } x\in[0;2], y\in [0;2] \\
0, \; \text{ иначе}
\end{cases}
\]

Найдите вероятность \(\P(Y = X)\)
\begin{answerlist}
  \item \(3/4\)
  \item \(0\)
  \item \(1/2\)
  \item \(1/4\)
  \item невозможно вычислить на основе имеющихся данных
\end{answerlist}
\end{question}

\begin{solution}
\begin{answerlist}
  \item Bad answer :(
  \item Good answer :)
  \item Bad answer :(
  \item Bad answer :(
  \item Bad answer :(
\end{answerlist}
\end{solution}



\begin{question}
Для дискретной случайной величины функция распределения
\begin{answerlist}
  \item не определена
  \item вырождена
  \item непрерывна
  \item имеет разрывы
  \item строго возрастает
\end{answerlist}
\end{question}

\begin{solution}
\begin{answerlist}
  \item Bad answer :(
  \item Bad answer :(
  \item Bad answer :(
  \item Good answer :)
  \item Bad answer :(
\end{answerlist}
\end{solution}



\begin{question}
Величины \(X_1, \, \ldots, \, X_n\) — случайная выборка из
распределения Бернулли с параметром \(p \in (0;\,1)\). Оценка
максимального правдоподобия параметра \(p\) равна \(\bar X\). Оценка
максимального правдоподобия для \(\sqrt{p}\) равна
\begin{answerlist}
  \item \(\frac{\sqrt{\sum_{i=1}^{n}X_i}}{n}\)
  \item \(\sqrt{\sum_{i=1}^{n}X_i}\)
  \item \(\frac{1}{n}\sum_{i=1}^{n}X_i\)
  \item \(\frac{1}{n}\sum_{i=1}^{n}\sqrt{X_i}\)
  \item \(\sqrt{\frac{1}{n}\sum_{i=1}^{n}X_i}\)
\end{answerlist}
\end{question}

\begin{solution}
\begin{answerlist}
  \item Тоже ересь
  \item Не туда!
  \item Неверно
  \item Не угадал
  \item Ураа!!!
\end{answerlist}
\end{solution}



\begin{question}
Совместное распределение пары величин \(X\) и \(Y\) задано таблицей:

\begin{tabular}{@{}c|ccc@{}}
\toprule
       & $Y=-1$ & $Y=0$ & $Y=1$ \\ \midrule
$X=-1$ & $1/4$  & $0$   & $1/4$ \\
$X=1$  & $1/6$  & $1/6$ & $1/6$ \\ \bottomrule
\end{tabular}

\vspace{0.5cm}

Ковариация, \(\Cov(X,Y)\), равна
\begin{answerlist}
  \item \(-0.5\)
  \item \(0\)
  \item \(1\)
  \item \(0.5\)
  \item \(-1\)
\end{answerlist}
\end{question}

\begin{solution}
\begin{answerlist}
  \item Неверно
  \item Отлично
  \item Неверно
  \item Неверно
  \item Неверно
\end{answerlist}
\end{solution}



\begin{question}
Плотность величины \(X\) имеет вид \(f(x)=2x\) при \(0<x<1\) и
\(f(x)=0\) при остальных \(x\). Условная плотность величины \(Y\)
задаётся формулой
\(f_{Y|X}(y|x)=\begin{cases} \frac{1}{x}, \text{ если } 0<y\le x; \\ 0, \text{ иначе } \end{cases}.\)
Совместная плотность величин \(X\) и \(Y\) равна
\begin{answerlist}
  \item \(f(x,y)=\begin{cases} 2, \text{ если } 0<y\le x<1; \\ 0, \text{ иначе} \end{cases}\)
  \item \(f(x,y)=\begin{cases} 2, \text{ если } 0<y<1, 0 < x<1; \\ 0, \text{ иначе} \end{cases}\)
  \item \(f(x,y)=\begin{cases} 1, \text{ если } 0<y\le x<1; \\ 0, \text{ иначе} \end{cases}\)
  \item \(f(x,y)=\begin{cases} 1/x, \text{ если } 0<y<1, 0 < x<1; \\ 0, \text{ иначе} \end{cases}\)
  \item \(f(x,y)=\begin{cases} 1/x, \text{ если } 0<y\le x<1; \\ 0, \text{ иначе} \end{cases}\)
\end{answerlist}
\end{question}

\begin{solution}
\begin{answerlist}
  \item Good answer :)
  \item Bad answer :(
  \item Bad answer :(
  \item Bad answer :(
  \item Bad answer :(
\end{answerlist}
\end{solution}



\begin{question}
Пусть \(t_n\) --- случайная величина, распределенная по Стьюденту с
\(n\) степенями свободы. Предел
\(\lim\limits_{n\to\infty}\P\left(t_{n}^2>1\right)\) равен
\begin{answerlist}
  \item \(0.317\)
  \item \(0.841\)
  \item \(0.253\)
  \item \(0.788\)
  \item \(0.102\)
\end{answerlist}
\end{question}

\begin{solution}
\begin{answerlist}
  \item Good answer :)
  \item Bad answer :(
  \item Bad answer :(
  \item Bad answer :(
  \item Bad answer :(
\end{answerlist}
\end{solution}



\begin{question}
Известно, что \(\E(X)=-1\), \(\E(Y)=2\), \(\Var(X)=4\), \(\Var(Y)=9\),
\(\Cov(X,Y)=-3\). Дисперсия \(\Var(2X-Y+1)\) равна
\begin{answerlist}
  \item 31
  \item \(37\)
  \item \(24\)
  \item \(-31\)
  \item 34
\end{answerlist}
\end{question}

\begin{solution}
\begin{answerlist}
  \item Bad answer :(
  \item Good answer :)
  \item Bad answer :(
  \item Bad answer :(
  \item Bad answer :(
\end{answerlist}
\end{solution}



\begin{question}
Совместная функция плотности пары случайных величин \(X\) и \(Y\) имеет
вид \[
f(x,y)=\begin{cases}
c (2x+y), \; \text{ если } x\in[0;2], y\in [0;2] \\
0, \; \text{ иначе}
\end{cases}
\]

Константа \(c\) равна
\begin{answerlist}
  \item \(1/8\)
  \item \(1/12\)
  \item \(12\)
  \item \(8\)
  \item \(1/6\)
\end{answerlist}
\end{question}

\begin{solution}
\begin{answerlist}
  \item Bad answer :(
  \item Good answer :)
  \item Bad answer :(
  \item Bad answer :(
  \item Bad answer :(
\end{answerlist}
\end{solution}



\begin{question}
Количество скачиваний за день мобильного приложения распределено по
Пуассону. В среднем приложение скачивают \(12\) раз за день. Вероятность
того, что приложение будет скачено за день ровно \(5\) раз, равна
\begin{answerlist}
  \item \(e^{-12}\frac{5^{12}}{12!}\)
  \item \(\frac{5}{12}\)
  \item \(e^{-5}\)
  \item \(e^{-5}\frac{5^{12}}{12!}\)
  \item \({e}^{-12}\frac{12^5}{5!}\)
\end{answerlist}
\end{question}

\begin{solution}
\begin{answerlist}
  \item Bad answer :(
  \item Bad answer :(
  \item Bad answer :(
  \item Bad answer :(
  \item Good answer :)
\end{answerlist}
\end{solution}



\begin{question}
Сумма независимых абсолютно непрерывной и дискретной случайных величин
имеет распределение
\begin{answerlist}
  \item нормальное
  \item абсолютно непрерывное
  \item вырожденное
  \item дискретное
  \item сингулярное
\end{answerlist}
\end{question}

\begin{solution}
\begin{answerlist}
  \item Bad answer :(
  \item Good answer :)
  \item Bad answer :(
  \item Bad answer :(
  \item Bad answer :(
\end{answerlist}
\end{solution}




\begin{question}
Совместное распределение пары величин \(X\) и \(Y\) задано таблицей:

\begin{tabular}{@{}c|ccc@{}}
\toprule
       & $Y=-1$ & $Y=0$ & $Y=1$ \\ \midrule
$X=-1$ & $1/4$  & $0$   & $1/4$ \\
$X=1$  & $1/6$  & $1/6$ & $1/6$ \\ \bottomrule
\end{tabular}

\vspace{0.5cm}

Дисперсия случайной величины \(Y\) равна
\begin{answerlist}
  \item \(5/12\)
  \item \(1/3\)
  \item \(1/2\)
  \item \(12/5\)
  \item \(5/6\)
\end{answerlist}
\end{question}

\begin{solution}
\begin{answerlist}
  \item Неверно
  \item Неверно
  \item Неверно
  \item Неверно
  \item Отлично
\end{answerlist}
\end{solution}



\begin{question}
Математическое ожидание случайной величины \(X\) при условии \(Y=0\)
равно
\begin{answerlist}
  \item \(1/6\)
  \item \(0\)
  \item \(-1\)
  \item \(1\)
  \item \(1/3\)
\end{answerlist}
\end{question}

\begin{solution}
\begin{answerlist}
  \item Bad answer :(
  \item Bad answer :(
  \item Bad answer :(
  \item Good answer :)
  \item Bad answer :(
\end{answerlist}
\end{solution}



\begin{question}
Выборочная функция распределения, построенная по выборке объёма \(n\) из
равномерного распределения на отрезке \([0,2]\), в точке \(х=0.2\) при
\(n\) стремящимся к бесконечности стремится по вероятности к
\begin{answerlist}
  \item \(1\)
  \item \(0.5\)
  \item \(0.1\)
  \item \(0.2\)
  \item \(0\)
\end{answerlist}
\end{question}

\begin{solution}
\begin{answerlist}
  \item Не туда!
  \item Тоже ересь
  \item Ураа!!!
  \item Неверно
  \item Не угадал
\end{answerlist}
\end{solution}



\begin{question}
По случайной выборке из 200 наблюдений было оценено выборочное среднее
\(\bar{X} = 25\) и несмещённая оценка дисперсии \(\hat{\sigma}^2 = 25\).
В рамках проверки гипотезы \(H_0: \mu = 20\) против \(H_a: \mu > 20\)
можно сделать вывод, что гипотеза \(H_0\)
\begin{answerlist}
  \item отвергается при \(\alpha = 0.01\), не отвергается при \(\alpha = 0.05\)
  \item отвергается при любом разумном значении \(\alpha\)
  \item не отвергается при любом разумном значении \(\alpha\)
  \item отвергается при \(\alpha = 0.05\), не отвергается при \(\alpha = 0.01\)
  \item Гипотезу невозможно проверить
\end{answerlist}
\end{question}

\begin{solution}
\begin{answerlist}
  \item Неверно
  \item Отлично
  \item Неверно
  \item Неверно
  \item Неверно
\end{answerlist}
\end{solution}



\begin{question}
Про независимые случайные величины \(X\) и \(Y\) известно, что
\(\Var(X)=8\), \(\Var(Y)=1\). Корреляция \(\Corr(X,-2Y)\) равна
\begin{answerlist}
  \item \(-0.5\)
  \item \(0.25\)
  \item \(0\)
  \item \(-0.025\)
  \item \(0.5\)
\end{answerlist}
\end{question}

\begin{solution}
\begin{answerlist}
  \item Bad answer :(
  \item Bad answer :(
  \item Good answer :)
  \item Bad answer :(
  \item Bad answer :(
\end{answerlist}
\end{solution}



\begin{question}
Величина \(Y\) имеет экспоненциальное (показательное) распределение с
параметром \(\lambda=0.5\). Величины \(X\) и \(Y\) независимы. Ожидание
\(\E(Y|X=3/4)\) равно
\begin{answerlist}
  \item \(\frac{1}{2}\)
  \item \(1\)
  \item \(\frac{1}{8}\)
  \item \(\frac{3}{4}\)
  \item \(2\)
\end{answerlist}
\end{question}

\begin{solution}
\begin{answerlist}
  \item Bad answer :(
  \item Bad answer :(
  \item Bad answer :(
  \item Bad answer :(
  \item Good answer :)
\end{answerlist}
\end{solution}



\begin{question}
Сумма независимых абсолютно непрерывной и дискретной случайных величин
имеет распределение
\begin{answerlist}
  \item вырожденное
  \item абсолютно непрерывное
  \item дискретное
  \item сингулярное
  \item нормальное
\end{answerlist}
\end{question}

\begin{solution}
\begin{answerlist}
  \item Bad answer :(
  \item Good answer :)
  \item Bad answer :(
  \item Bad answer :(
  \item Bad answer :(
\end{answerlist}
\end{solution}



\begin{question}
Про случайные величины \(X, Y, Z\) известно, что \(\E(X)=1\),
\(\E(Y)=2\), \(\E(Z)=3\). Ожидание \(\E(X-Y+2Z)\) равно
\begin{answerlist}
  \item \(4\)
  \item \(1\)
  \item \(3\)
  \item \(2\)
  \item \(5\)
\end{answerlist}
\end{question}

\begin{solution}
\begin{answerlist}
  \item Bad answer :(
  \item Bad answer :(
  \item Bad answer :(
  \item Bad answer :(
  \item Good answer :)
\end{answerlist}
\end{solution}



\begin{question}
Величины \(X\) и \(Y\) одинаково распределены и равновероятно принимают
только два значения, \(-1\) и \(1\), при этом \(\P(Y=1|X=1)=0.4\).
Вероятность \(\P(Y=-1,X=1)\) равна
\begin{answerlist}
  \item \(0.3\)
  \item \(0.4\)
  \item \(0.6\)
  \item \(0.5\)
  \item \(1\)
\end{answerlist}
\end{question}

\begin{solution}
\begin{answerlist}
  \item Good answer :)
  \item Bad answer :(
  \item Bad answer :(
  \item Bad answer :(
  \item Bad answer :(
\end{answerlist}
\end{solution}



\begin{question}
Cовместная функция плотности случайных величин \(X\) и \(Y\) имеет вид
\[
f(x,y)=\begin{cases}
\frac{1}{4}xy, \; \text{ если } x\in[0;2], y\in [0;2] \\
0, \; \text{ иначе}
\end{cases}
\]

Найдите функцию распределения \(F_Y(y)\)
\begin{answerlist}
  \item \[F_Y(y)=\begin{cases} 0, \;  y < 0 \\ \frac{1}{2}y^2, \; y\in [0;2] \\ 0, \; y > 2 \end{cases}\]
  \item \[F_Y(y)=\begin{cases} \frac{1}{4}y^2, \; \text{ если } y\in [0;2] \\ 0, \; \text{ иначе} \end{cases}\]
  \item \[F_Y(y)=\begin{cases} 0, \;  y < 0 \\ y^2, \; y\in [0;2] \\ 0, \; y > 2 \end{cases}\]
  \item \[F_Y(y)=\begin{cases} 0, \;  y < 0 \\ \frac{1}{4}y^2, \; y\in [0;2] \\ 1, \; y > 2 \end{cases}\]
  \item \[F_Y(y)=\begin{cases} y^2, \; \text{ если } y\in [0;2] \\ 0, \; \text{ иначе} \end{cases}\]
\end{answerlist}
\end{question}

\begin{solution}
\begin{answerlist}
  \item Bad answer :(
  \item Bad answer :(
  \item Bad answer :(
  \item Good answer :)
  \item Bad answer :(
\end{answerlist}
\end{solution}




\subsection[2017-2018]{\hyperref[sec:sol_midterm_exam_2017_2018]{2017-2018}}
\label{sec:midterm_exam_2017_2018}



\begin{question}
Математическое ожидание величины \(X\) равно 2, а дисперсия равна 6.
Вероятность \(\P(X^2 \geq 100)\) лежит в диапазоне
\begin{answerlist}
  \item \([0;0.1]\)
  \item \([0.1;0.2]\)
  \item \([0.9;1]\)
  \item \([0.99;1]\)
  \item \([0;0.01]\)
\end{answerlist}
\end{question}

\begin{solution}
\begin{answerlist}
  \item Good answer :)
  \item Bad answer :(
  \item Bad answer :(
  \item Bad answer :(
  \item Bad answer :(
\end{answerlist}
\end{solution}



\begin{question}
Случайная величина \(\xi\) имеет распределение Пуассона с параметром
\(\lambda\). Математическое ожидание \(\E[\xi^2]\) равно
\begin{answerlist}
  \item \(\lambda\)
  \item \(e^{-\lambda}\)
  \item \(\lambda(1 - \lambda)\)
  \item \(\lambda^2\)
  \item \(\lambda(\lambda+1)\)
\end{answerlist}
\end{question}

\begin{solution}
\begin{answerlist}
  \item Bad answer :(
  \item Bad answer :(
  \item Bad answer :(
  \item Bad answer :(
  \item Good answer :)
\end{answerlist}
\end{solution}



\begin{question}
Известно, что \(\E(X)=-1\), \(\E(Y)=2\), \(\Var(X)=4\), \(\Var(Y)=9\),
\(\Cov(X,Y)=-3\). Корреляция \(\Corr(X+Y, Y)\) равна
\begin{answerlist}
  \item \(1/\sqrt{6}\)
  \item \(-1/\sqrt{7}\)
  \item \(2/\sqrt{7}\)
  \item \(-2/\sqrt{6}\)
  \item \(-3/\sqrt{6}\)
\end{answerlist}
\end{question}

\begin{solution}
\begin{answerlist}
  \item Bad answer :(
  \item Bad answer :(
  \item Good answer :)
  \item Bad answer :(
  \item Bad answer :(
\end{answerlist}
\end{solution}


% 
\begin{question}
Совместное распределение дискретных случайных величин \(X\) и \(Y\)
задано таблицей:

\begin{tabular}{cccc}
\toprule
 & $Y=-2$ & $Y=0$ & $Y=1$ \\
\midrule
$X=3$ & $0.3$ & $0.1$ & $0.2$  \\
$X=6$ & $0.1$ & $0.2$ & $0.1$ \\
\bottomrule
\end{tabular}

Условное ожидание \(\E(X|Y=-2)\) равно
\begin{answerlist}
  \item \(3.75\)
  \item \(3.(3)\)
  \item \(3.25\)
  \item \(4.2\)
  \item \(3.5\)
\end{answerlist}
\end{question}

\begin{solution}
\begin{answerlist}
  \item Good answer :)
  \item Bad answer :(
  \item Bad answer :(
  \item Bad answer :(
  \item Bad answer :(
\end{answerlist}
\end{solution}



\begin{question}
Случайная величина \(\xi\) имеет стандартное нормальное распределение.
Вероятность \(\P(\{\xi \in [-1; \, 2]\})\) равна
\begin{answerlist}
  \item \(\int_{-1}^{2}\tfrac{1}{\sqrt{2\pi}}e^{x^2 / 2}\,dx\)
  \item \(\int_{-1}^{2}\tfrac{1}{\sqrt{2\pi}}e^{-x^2 / 2}\,dx\)
  \item \(\int_{-1}^{2}\tfrac{1}{\sqrt{2\pi}}e^{-x^2}\,dx\)
  \item \(\int_{-1}^{2}\tfrac{1}{2\pi}e^{-x^2 / 2}\,dx\)
  \item \(\int_{-1}^{2}\tfrac{1}{\sqrt{2\pi}}e^{x^2}\,dx\)
\end{answerlist}
\end{question}

\begin{solution}
\begin{answerlist}
  \item Bad answer :(
  \item Good answer :)
  \item Bad answer :(
  \item Bad answer :(
  \item Bad answer :(
\end{answerlist}
\end{solution}


% 
\begin{question}
Для случайной величины \(X \sim \cN(\mu_X, \sigma^2_X)\) вероятность
\(\P(X - \mu_x > 5\sigma_X)\) примерно равна
\begin{answerlist}
  \item \(0\)
  \item \(0.5\)
  \item \(0.95\)
  \item \(1/5\)
  \item \(0.05\)
\end{answerlist}
\end{question}

\begin{solution}
\begin{answerlist}
  \item Good answer :)
  \item Bad answer :(
  \item Bad answer :(
  \item Bad answer :(
  \item Bad answer :(
\end{answerlist}
\end{solution}



\begin{question}
Двумерная случайная величина \((X, Y)\) равномерно распределена в
треугольнике ограниченном линиями \(x=0\), \(y=0\) и \(y+2x=4\).
Значение функции плотности \(f_{X,Y}(1,1)\) равно
\begin{answerlist}
  \item \(0.25\)
  \item \(0.5\)
  \item \(1\)
  \item \(\frac{1}{\sqrt{2\pi}}\exp(-0.5)\)
  \item \(0.125\)
\end{answerlist}
\end{question}

\begin{solution}
\begin{answerlist}
  \item Good answer :)
  \item Bad answer :(
  \item Bad answer :(
  \item Bad answer :(
  \item Bad answer :(
\end{answerlist}
\end{solution}


% 
\begin{question}
У Васи есть пять кнопок, генерирующих целые числа от 1 до 6. Три
работают как честные кубики, одна --- с увеличенной вероятностью
выпадения 6 (она выпадает с веростностью 0.5, остальные ---
равновероятно), одна --- с увеличенной вероятностью выпадения 1 (она
выпадает с вероятностью 0.5, остальные --- равновероятно). Вася нажимает
на случайную кнопку. Число 6 выпадет с вероятностью
\begin{answerlist}
  \item 0.11
  \item 0.12
  \item 1/4
  \item 1/6
  \item 0.22
\end{answerlist}
\end{question}

\begin{solution}
\begin{answerlist}
  \item Bad answer :(
  \item Bad answer :(
  \item Bad answer :(
  \item Bad answer :(
  \item Good answer :)
\end{answerlist}
\end{solution}


% 
\begin{question}
Величины \(X_1\), \(X_2\), \ldots, независимы и одинаково распределены с
\(\E(X_i) = 4\) и \(\Var(X_i) = 100\), а
\(S_n = X_1 + X_2 + \ldots + X_n\). К нормальному стандартному
распределению сходится последовательность
\begin{answerlist}
  \item \(\sqrt{n}\frac{S_n - 4n}{10/\sqrt{n}}\)
  \item \(\sqrt{n}\frac{S_n - 4}{10}\sqrt{n}\)
  \item \(\sqrt{n}\frac{S_n - 4}{10}\)
  \item \(\frac{S_n - 4n}{10\sqrt{n}}\)
  \item \(\sqrt{n}\frac{S_n - 4}{10/\sqrt{n}}\)
\end{answerlist}
\end{question}

\begin{solution}
\begin{answerlist}
  \item Bad answer :(
  \item Bad answer :(
  \item Bad answer :(
  \item Good answer :)
  \item Bad answer :(
\end{answerlist}
\end{solution}



\begin{question}
События A, B и C независимы в совокупности, если
\begin{answerlist}
  \item \(\P(A|B) = \P(A), \P(A|C) = \P(A)\)
  \item \(\P(ABC) = \P(A) \P(B) \P(C)\)
  \item \(\P(A|B) = \P(A), \P(A|C) = \P(A), \P(B|C) = \P(B)\)
  \item \(\P(A\cap B) = \P(A)\P(B), \P(A\cap C) = \P(A)\P(C), \P(B\cap C) = \P(B)\P(C)\)
  \item \(\P(A \cap B \cap C) = 0\)
\end{answerlist}
\end{question}

\begin{solution}
\begin{answerlist}
  \item Bad answer :(
  \item Bad answer :(
  \item Bad answer :(
  \item Bad answer :(
  \item Bad answer :(
\end{answerlist}
\end{solution}



\begin{question}
Случайная величина \(\xi\) имеет равномерное распределение на отрезке
\([0;\,4]\). Вероятность \(\P(\{\xi \in [3;\,6]\})\) равна
\begin{answerlist}
  \item \(\Phi(4) - \Phi(3)\)
  \item \(1/4\)
  \item \(3/6\)
  \item \(1/2\)
  \item \(3/4\)
\end{answerlist}
\end{question}

\begin{solution}
\begin{answerlist}
  \item Bad answer :(
  \item Good answer :)
  \item Bad answer :(
  \item Bad answer :(
  \item Bad answer :(
\end{answerlist}
\end{solution}



\begin{question}
Случайная величина \(X\) принимает равновероятно целые значение от
\(-5\) до \(5\) включительно. Случайная величина \(Y\) принимает
равновероятно целые значение от \(-1\) до \(1\) включительно. Величины
\(X\) и \(Y\) независимы. Вероятность \(\P(X+Y^2=2)\) равна
\begin{answerlist}
  \item \(1/5\)
  \item \(1/33\)
  \item \(1/11\)
  \item \(5/33\)
  \item \(2/33\)
\end{answerlist}
\end{question}

\begin{solution}
\begin{answerlist}
  \item Bad answer :(
  \item Bad answer :(
  \item Good answer :)
  \item Bad answer :(
  \item Bad answer :(
\end{answerlist}
\end{solution}


% 
\begin{question}
Известно, что \(\E(X)=-1\), \(\E(Y)=2\), \(\Var(X)=4\), \(\Var(Y)=9\),
\(\Cov(X,Y)=-3\). Ковариация \(\Cov(aX, (1-a)Y)\) минимальна при \(a\)
равном
\begin{answerlist}
  \item \(3/12\)
  \item \(0\)
  \item \(1/2\)
  \item \(-1/4\)
  \item \(2/3\)
\end{answerlist}
\end{question}

\begin{solution}
\begin{answerlist}
  \item Bad answer :(
  \item Bad answer :(
  \item Good answer :)
  \item Bad answer :(
  \item Bad answer :(
\end{answerlist}
\end{solution}



\begin{question}
Круг разделён на секторы с углом \(\frac{\pi}{3}\). Один из них закрашен
красным, один сектор --- синим, остальные сектора --- белым. Вася кидает
дротики и всегда попадает в круг, все точки круга равновероятны.
Вероятность того, что Вася попадёт в красный сектор, равна
\begin{answerlist}
  \item не хватает данных
  \item 1/4
  \item \(\pi / 3\)
  \item \(\pi / 6\)
  \item 1/6
\end{answerlist}
\end{question}

\begin{solution}
\begin{answerlist}
  \item Bad answer :(
  \item Bad answer :(
  \item Bad answer :(
  \item Bad answer :(
  \item Good answer :)
\end{answerlist}
\end{solution}


%
\begin{question}
Двумерная функция распределения \(F_{X,Y}(x,y)\) может \textbf{НЕ}
удовлетворять свойству
\begin{answerlist}
  \item функция \(F_{X,Y}(x, y)\) непрерывна
  \item \(\lim_{y \to +\infty} F_{X,Y}(x,y) = F_X(x)\)
  \item \(F_{X,Y}(x,y)\) не убывает по \(x\)
  \item \(0 \leq F_{X,Y}(x, y)\leq 1\)
  \item \(\lim_{x,y \to +\infty} F_{X,Y}(x,y) = 1\)
\end{answerlist}
\end{question}

\begin{solution}
\begin{answerlist}
  \item Bad answer :(
  \item Bad answer :(
  \item Good answer :)
  \item Bad answer :(
  \item Bad answer :(
\end{answerlist}
\end{solution}



\begin{question}
Известно, что \(\P(A \cap B) = 0.2\), \(\P(A \cup B) = 0.6\),
\(\P(A) = 0.3\). Вероятность \(\P(B)\) равна
\begin{answerlist}
  \item 0.3
  \item 0.5
  \item 0.1
  \item не хватает данных
  \item 0.6
\end{answerlist}
\end{question}

\begin{solution}
\begin{answerlist}
  \item Bad answer :(
  \item Good answer :)
  \item Bad answer :(
  \item Bad answer :(
  \item Bad answer :(
\end{answerlist}
\end{solution}



\begin{question}
Известно, что \(\E(X)=-1\), \(\E(Y)=2\), \(\Var(X)=4\), \(\Var(Y)=9\),
\(\Cov(X,Y)=-3\). Дисперсия \(\Var(2X-Y+1)\) равна
\begin{answerlist}
  \item 31
  \item \(37\)
  \item \(24\)
  \item \(-31\)
  \item 34
\end{answerlist}
\end{question}

\begin{solution}
\begin{answerlist}
  \item Bad answer :(
  \item Good answer :)
  \item Bad answer :(
  \item Bad answer :(
  \item Bad answer :(
\end{answerlist}
\end{solution}



\begin{question}
Величины \(X_1\), \(X_2\), \ldots, независимы и одинаково распределены
\(\cN(0;1)\). Предел по вероятности
\(\plim_{n\to\infty} \frac{X_1^2+ X_2^2 + \ldots + X_n^2}{n}\) равен
\begin{answerlist}
  \item \(3\)
  \item \(1\)
  \item \(2\)
  \item \(0\)
  \item \(1/2\)
\end{answerlist}
\end{question}

\begin{solution}
\begin{answerlist}
  \item Bad answer :(
  \item Good answer :)
  \item Bad answer :(
  \item Bad answer :(
  \item Bad answer :(
\end{answerlist}
\end{solution}



\begin{question}
Совместная функция плотности величин \(X\) и \(Y\) имеет вид \[
f(x,y) =
\begin{cases}
6xy^2, \text{ при } x, y \in [0;1] \\
0, \text{ иначе } \\
\end{cases}.
\] При \(Y=1/2\) величина \(X\) имеет условное распределение
\begin{answerlist}
  \item с плотностью \(f(x)=1.5x\) при \(x\in[0;1]\)
  \item равномерное, \(U[0;1]\)
  \item с плотностью \(f(x)=2x\) при \(x\in[0;1]\)
  \item нормальное, \(\cN(0;1)\)
  \item с плотностью \(f(x)=3x^2\) при \(x\in[0;1]\)
\end{answerlist}
\end{question}

\begin{solution}
\begin{answerlist}
  \item Bad answer :(
  \item Bad answer :(
  \item Good answer :)
  \item Bad answer :(
  \item Bad answer :(
\end{answerlist}
\end{solution}


% 
\begin{question}
Известно, что \(\E(X)=-1\), \(\E(Y)=2\), \(\Var(X)=4\), \(\Var(Y)=9\),
\(\Cov(X,Y)=-3\). Ожидание \(\E(X^2-Y^2)\) равно
\begin{answerlist}
  \item \(-4\)
  \item 8
  \item 4
  \item \(-8\)
  \item 0
\end{answerlist}
\end{question}

\begin{solution}
\begin{answerlist}
  \item Bad answer :(
  \item Bad answer :(
  \item Bad answer :(
  \item Good answer :)
  \item Bad answer :(
\end{answerlist}
\end{solution}



\begin{question}
Величины \(X_1\), \(X_2\), \ldots, независимы и одинаково распределены с
\(\E(X_i) = 4\) и \(\Var(X_i) = 100\). Вероятность
\(\P(\bar X_n \leq 5)\) примерно равна
\begin{answerlist}
  \item \(0.50\)
  \item \(0.28\)
  \item \(0.84\)
  \item \(0.95\)
  \item \(0.67\)
\end{answerlist}
\end{question}

\begin{solution}
\begin{answerlist}
  \item Bad answer :(
  \item Bad answer :(
  \item Good answer :)
  \item Bad answer :(
  \item Bad answer :(
\end{answerlist}
\end{solution}



\begin{question}
Известно, что \(\E(X)=-1\), \(\E(Y)=2\), \(\Var(X)=4\), \(\Var(Y)=9\),
\(\Cov(X,Y)=-3\). Ковариация \(\Cov(X+2Y, 2X+3)\) равна
\begin{answerlist}
  \item \(-4\)
  \item \(1\)
  \item \(4\)
  \item \(0\)
  \item \(-1\)
\end{answerlist}
\end{question}

\begin{solution}
\begin{answerlist}
  \item Good answer :)
  \item Bad answer :(
  \item Bad answer :(
  \item Bad answer :(
  \item Bad answer :(
\end{answerlist}
\end{solution}



\begin{question}
Известно, что \(\E(X)=-1\), \(\E(Y)=2\), \(\Var(X)=4\), \(\Var(Y)=9\),
\(\Cov(X,Y)=-3\). Ожидание \(\E((X-1)Y)\) равно
\begin{answerlist}
  \item \(-6\)
  \item \(-7\)
  \item \(-5\)
  \item \(-8\)
  \item \(-9\)
\end{answerlist}
\end{question}

\begin{solution}
\begin{answerlist}
  \item Bad answer :(
  \item Good answer :)
  \item Bad answer :(
  \item Bad answer :(
  \item Bad answer :(
\end{answerlist}
\end{solution}


% 
\begin{question}
В каком из этих случаев события \(A\) и \(B\) будут независимы?
\begin{answerlist}
  \item \(\P(A \cup B) = 0.6\), \(\P (A) = 0.5\), \(\P(B) = 0.2\)
  \item \(\P(A \cap B) = 0.1\), \(\P (A) = 0.5\), \(\P(B) = 0.2\)
  \item \(\P(A \cap B) = 0\), \(\P (A) = 0.8\), \(\P(B) = 0.1\)
  \item \(\P(A \cup B) = 0.2\), \(\P (A) = 0.5\), \(\P(B) = 0.4\)
  \item \(\P(A \cap B) = 0.1\), \(\P (A) = 0.5\), \(\P(B) = 0.9\)
\end{answerlist}
\end{question}

\begin{solution}
\begin{answerlist}
  \item Bad answer :(
  \item Good answer :)
  \item Bad answer :(
  \item Bad answer :(
  \item Bad answer :(
\end{answerlist}
\end{solution}



\begin{question}
Правильный кубик подбрасывается два раза, величина \(X_i\) равна 1, если
в \(i\)-ый раз выпала шестёрка, и нулю иначе. Условный закон
распределения \(X_1\) при условии \(X_1+X_2=1\) совпадает с
распределением
\begin{answerlist}
  \item Биномиальным \(Bin(n=2, p=1/2)\)
  \item Бернулли с \(p=1/2\)
  \item Бернулли с \(p=1/6\)
  \item Биномиальным \(Bin(n=2, p=1/6)\)
  \item нормальным \(\cN(0;1)\)
\end{answerlist}
\end{question}

\begin{solution}
\begin{answerlist}
  \item Bad answer :(
  \item Good answer :)
  \item Bad answer :(
  \item Bad answer :(
  \item Bad answer :(
\end{answerlist}
\end{solution}



\begin{question}
Случайные величины \(X\) и \(Y\) независимы и нормально распределены с
параметрами \(\E(X)=2\), \(\Var(X)=3\), \(\E(Y)=1\), \(\Var(Y)=4\).
Вероятность \(\P(X+Y<3)\) равна
\begin{answerlist}
  \item \(3/7\)
  \item \(0.05\)
  \item \(0.5\)
  \item \(0.995\)
  \item \(2/7\)
\end{answerlist}
\end{question}

\begin{solution}
\begin{answerlist}
  \item Bad answer :(
  \item Bad answer :(
  \item Good answer :)
  \item Bad answer :(
  \item Bad answer :(
\end{answerlist}
\end{solution}



\begin{question}
У Васи есть пять кнопок, генерирующих целые числа от 1 до 6. Три
работают как честные кубики, одна --- с увеличенной вероятностью
выпадения 6 (она выпадает с веростностью 0.5, остальные ---
равновероятно), одна --- с увеличенной вероятностью выпадения 1 (она
выпадает с вероятностью 0.5, остальные --- равновероятно). Вася нажимает
на случайную кнопку. После нажатия на случайную кнопку выпала 6.
Условная вероятность того, что это была кнопка «честный кубик» равна
\begin{answerlist}
  \item 1/2
  \item 4/11
  \item 5/11
  \item 8/11
  \item 6/11
\end{answerlist}
\end{question}

\begin{solution}
\begin{answerlist}
  \item Bad answer :(
  \item Bad answer :(
  \item Good answer :)
  \item Bad answer :(
  \item Bad answer :(
\end{answerlist}
\end{solution}



\begin{question}
Ковариационной матрицей может являться матрица
\begin{answerlist}
  \item \(\begin{pmatrix} 1 & 2 \\ 2 & 1 \\ \end{pmatrix}\)
  \item \(\begin{pmatrix} 1 & 4 \\ 4 & 9 \\ \end{pmatrix}\)
  \item \(\begin{pmatrix} 9 & 7 \\ 7 & 6 \\ \end{pmatrix}\)
  \item \(\begin{pmatrix} -1 & 2 \\ 2 & 10 \\ \end{pmatrix}\)
  \item \(\begin{pmatrix} 1 & 2 \\ 1 & 2 \\ \end{pmatrix}\)
\end{answerlist}
\end{question}

\begin{solution}
\begin{answerlist}
  \item Bad answer :(
  \item Bad answer :(
  \item Good answer :)
  \item Bad answer :(
  \item Bad answer :(
\end{answerlist}
\end{solution}



\begin{question}
Известно, что \(\E(X)=-1\), \(\E(Y)=2\), \(\Var(X)=4\), \(\Var(Y)=9\),
\(\Cov(X,Y)=-3\). Из условия \(\E(aX+(1-a)Y)=0\) следует, что \(a\)
равно
\begin{answerlist}
  \item 2/3
  \item 1/2
  \item 1
  \item 0
  \item 1/3
\end{answerlist}
\end{question}

\begin{solution}
\begin{answerlist}
  \item Good answer :)
  \item Bad answer :(
  \item Bad answer :(
  \item Bad answer :(
  \item Bad answer :(
\end{answerlist}
\end{solution}



\begin{question}
Случайная величина \(\xi\) имеет биномиальное распределение с
параметрами \(n = 2\) и \(p = 3/4\). Вероятность \(\P(\xi = 0)\) равна
\begin{answerlist}
  \item \(1/2\)
  \item \(1/16\)
  \item \(9/16\)
  \item \(3/4\)
  \item \(3/4\)
\end{answerlist}
\end{question}

\begin{solution}
\begin{answerlist}
  \item Bad answer :(
  \item Good answer :)
  \item Bad answer :(
  \item Bad answer :(
  \item Bad answer :(
\end{answerlist}
\end{solution}



\begin{question}
Количество сбоев системы SkyNet за сутки имеет распределение Пуассона.
Среднее количество сбоев за сутки равно 4. Вероятность того, что за
сутки произойдет не менее одного сбоя, равна
\begin{answerlist}
  \item \(e^4\)
  \item \(1-e^4\)
  \item \(1- e^{-4}\)
  \item \(\tfrac{1}{4!}e^{-4}\)
  \item \(e^{-4}\)
\end{answerlist}
\end{question}

\begin{solution}
\begin{answerlist}
  \item Bad answer :(
  \item Bad answer :(
  \item Good answer :)
  \item Bad answer :(
  \item Bad answer :(
\end{answerlist}
\end{solution}


% 
\begin{question}
У пары случайных величин \(X\), \(Y\) существует совместная функция
плотности \(f(x,y)\) и условная функция плотности \(f(x|y)\). Условную
дисперсию \(\Var(X|Y)\) можно найти по формуле
\begin{answerlist}
  \item \(\int_{-\infty}^{+\infty} (x - \E(X|Y))^2 \, dx\)
  \item \(\int_{-\infty}^{+\infty} (x - \E(X))^2 f(x|Y) \, dx\)
  \item \(\int_{-\infty}^{+\infty} x^2 f(x|Y) \, dx - (\E(X|Y))^2\)
  \item \(\left(\int_{-\infty}^{+\infty} x f(x|Y) \, dx\right)^2 - (\E(X|Y))^2\)
  \item \(\int_{-\infty}^{+\infty} x^2 f(x|Y) \, dx\)
\end{answerlist}
\end{question}

\begin{solution}
\begin{answerlist}
  \item Bad answer :(
  \item Bad answer :(
  \item Good answer :)
  \item Bad answer :(
  \item Bad answer :(
\end{answerlist}
\end{solution}



\begin{question}
Случайная величина \(\xi\) имеет распределение Бернулли с параметром
\(p\). Математическое ожидание \(\E[\xi^2]\) равно
\begin{answerlist}
  \item \(p(1-p)\)
  \item \(p\)
  \item \(p^2\)
  \item \(0\)
  \item \(1-p\)
\end{answerlist}
\end{question}

\begin{solution}
\begin{answerlist}
  \item Bad answer :(
  \item Good answer :)
  \item Bad answer :(
  \item Bad answer :(
  \item Bad answer :(
\end{answerlist}
\end{solution}



\begin{question}
Случайная величина \(\xi\) имеет показательное (экспоненциальное)
распределение с параметром \(\lambda\). Математическое ожидание
\(\E[\xi^2]\) равно
\begin{answerlist}
  \item \(2/\lambda^2\)
  \item \(1/\lambda^2 - 1/ \lambda\)
  \item \(\lambda^2\)
  \item \(1/\lambda\)
  \item \(1/\lambda^2\)
\end{answerlist}
\end{question}

\begin{solution}
\begin{answerlist}
  \item Good answer :)
  \item Bad answer :(
  \item Bad answer :(
  \item Bad answer :(
  \item Bad answer :(
\end{answerlist}
\end{solution}



\begin{question}
Круг разделён на секторы с углом \(\frac{\pi}{3}\). Один из них закрашен
красным, один --- синим, остальные --- белым. Вася кидает дротики и
всегда попадает в круг, все точки круга равновероятны. Пусть событие
\(A\) --- попадание в красный сектор, \(B\) --- попадание в синий
сектор. Эти события
\begin{answerlist}
  \item образуют полную группу событий
  \item случаются с вероятностями 1/4
  \item независимы
  \item случаются с разными вероятностями
  \item несовместны
\end{answerlist}
\end{question}

\begin{solution}
\begin{answerlist}
  \item Bad answer :(
  \item Bad answer :(
  \item Bad answer :(
  \item Bad answer :(
  \item Good answer :)
\end{answerlist}
\end{solution}



\begin{question}
Совместная функция плотности величин \(X\) и \(Y\) имеет вид \[
f(x,y) =
\begin{cases}
6xy^2, \text{ при } x, y \in [0;1] \\
0, \text{ иначе } \\
\end{cases}.
\]

Математическое ожидание \(\E(XY)\) равно
\begin{answerlist}
  \item \(1/2\)
  \item \(1\)
  \item \(4/5\)
  \item \(3/4\)
  \item \(2/3\)
\end{answerlist}
\end{question}

\begin{solution}
\begin{answerlist}
  \item Good answer :)
  \item Bad answer :(
  \item Bad answer :(
  \item Bad answer :(
  \item Bad answer :(
\end{answerlist}
\end{solution}


% 
\begin{question}
Случайная величина \(\xi\) имеет равномерное распределение на отрезке
\([0;\,4]\). Математическое ожидание \(\E[\xi^2]\) равно
\begin{answerlist}
  \item \(52/12\)
  \item \(2\)
  \item \(4\)
  \item \(16/12\)
  \item \(64/12\)
\end{answerlist}
\end{question}

\begin{solution}
\begin{answerlist}
  \item Bad answer :(
  \item Bad answer :(
  \item Bad answer :(
  \item Bad answer :(
  \item Good answer :)
\end{answerlist}
\end{solution}



\begin{question}
Известно, что \(\E(X)=-1\), \(\E(Y)=2\), \(\Var(X)=4\), \(\Var(Y)=9\),
\(\Cov(X,Y)=-3\). Дисперсия \(\Var(aX+(1-a)Y)\) минимальна при \(a\)
равном
\begin{answerlist}
  \item 11/12
  \item 7/12
  \item \(3/12\)
  \item \(-1/4\)
  \item 3/24
\end{answerlist}
\end{question}

\begin{solution}
\begin{answerlist}
  \item Good answer :)
  \item Bad answer :(
  \item Bad answer :(
  \item Bad answer :(
  \item Bad answer :(
\end{answerlist}
\end{solution}



\begin{question}
В самолёте 200 пассажиров. Четверть пассажиров летит без багажа,
половина из них — с рюкзаками. Среди пассажиров с багажом 55 человек
летит с рюкзаками. Вероятность того, что случайно выбранный человек
летит без рюкзака, равна
\begin{answerlist}
  \item 0.6
  \item 0.4
  \item 0.65
  \item 0.45
  \item 0.5
\end{answerlist}
\end{question}

\begin{solution}
\begin{answerlist}
  \item Good answer :)
  \item Bad answer :(
  \item Bad answer :(
  \item Bad answer :(
  \item Bad answer :(
\end{answerlist}
\end{solution}



\begin{question}
Математическое ожидание величины \(X\) равно 2, а дисперсия равна 6.
Вероятность \(\P(|2-X|\leq 10)\) принадлежит диапазону
\begin{answerlist}
  \item \([0.2;0.4]\)
  \item \([0; 0.06]\)
  \item \([0.94; 1]\)
  \item \([0.6; 0.8]\)
  \item \([0.99;1]\)
\end{answerlist}
\end{question}

\begin{solution}
\begin{answerlist}
  \item Bad answer :(
  \item Bad answer :(
  \item Good answer :)
  \item Bad answer :(
  \item Bad answer :(
\end{answerlist}
\end{solution}






\subsection[2016-2017]{\hyperref[sec:sol_midterm_exam_2016_2017]{2016-2017}}
\label{sec:midterm_exam_2016_2017}



\begin{question}
Граф Сен-Жермен извлекает карты в случайном порядке из стандартной
колоды в 52 карты без возвращения. Рассмотрим три события: \(A\) ---
«первая карта — тройка»; \(B\) — «вторая карта — семёрка»; \(C\)
— «третья карта — дама пик».
\begin{answerlist}
  \item События \(A\) и \(B\) зависимы, события \(B\) и \(C\) независимы.
  \item События \(A\) и \(B\) зависимы, события \(B\) и \(C\) зависимы.
  \item События \(A\) и \(B\) независимы, события \(B\) и \(C\) независимы.
  \item События \(A\) и \(С\) независимы, события \(B\) и \(C\) зависимы.
  \item События \(A\) и \(B\) независимы, события \(B\) и \(C\) зависимы.
\end{answerlist}
\end{question}

\begin{solution}
\begin{answerlist}
  \item Bad answer :(
  \item Good answer :)
  \item Bad answer :(
  \item Bad answer :(
  \item Bad answer :(
\end{answerlist}
\end{solution}



\begin{question}
Функцией плотности случайной величины может являться функция
\begin{answerlist}
  \item \(f(x) = \begin{cases} -1, x \in [-1, 0] \\ 0,\text{ иначе} \end{cases}\)
  \item \(f(x) = \begin{cases} x - 1, x \in [0,1+\sqrt{3}] \\ 0,\text{ иначе} \end{cases}\)
  \item \(f(x) = \begin{cases} \frac{1}{x^2}, x \in [1,+ \infty) \\ 0,\text{ иначе} \end{cases}\)
  \item \(f(x) = \frac{1}{\sqrt{2\pi}} e^{-x^2}\)
  \item \(f(x) = \begin{cases} x^2, x \in [0,2] \\ 0,\text{ иначе} \end{cases}\)
\end{answerlist}
\end{question}

\begin{solution}
\begin{answerlist}
  \item Bad answer :(
  \item Bad answer :(
  \item Good answer :)
  \item Bad answer :(
  \item Bad answer :(
\end{answerlist}
\end{solution}



\begin{question}
Известно, что \(\E(X)=3\), \(\E(Y)=2\), \(\Var(X)=12\), \(\Var(Y)=1\),
\(\Cov(X,Y)=2\). Ожидание \(\E(XY)\) равно
\begin{answerlist}
  \item 5
  \item 2
  \item 8
  \item 6
  \item 0
\end{answerlist}
\end{question}

\begin{solution}
\begin{answerlist}
  \item Bad answer :(
  \item Bad answer :(
  \item Good answer :)
  \item Bad answer :(
  \item Bad answer :(
\end{answerlist}
\end{solution}



\begin{question}
Известно, что \(\E(X)=3\), \(\E(Y)=2\), \(\Var(X)=12\), \(\Var(Y)=1\),
\(\Cov(X,Y)=2\). Корреляция \(\Corr(X,Y)\) равна
\begin{answerlist}
  \item \(\frac{1}{\sqrt{3}}\)
  \item \(\frac{1}{\sqrt{12}}\)
  \item \(\frac{2}{\sqrt{13}}\)
  \item \(\frac{2}{12}\)
  \item \(\frac{1}{12}\)
\end{answerlist}
\end{question}

\begin{solution}
\begin{answerlist}
  \item Good answer :)
  \item Bad answer :(
  \item Bad answer :(
  \item Bad answer :(
  \item Bad answer :(
\end{answerlist}
\end{solution}



\begin{question}
Известно, что \(\E(X)=3\), \(\E(Y)=2\), \(\Var(X)=12\), \(\Var(Y)=1\),
\(\Cov(X,Y)=2\). Дисперсия \(\Var(2X-Y+4)\) равна
\begin{answerlist}
  \item 41
  \item 45
  \item 49
  \item 57
  \item 53
\end{answerlist}
\end{question}

\begin{solution}
\begin{answerlist}
  \item Good answer :)
  \item Bad answer :(
  \item Bad answer :(
  \item Bad answer :(
  \item Bad answer :(
\end{answerlist}
\end{solution}



\begin{question}
Если случайные величины \(X\) и \(Y\) имеют совместное нормальное
распределение с нулевыми математическими ожиданиями и единичной
ковариационной матрицей, то
\begin{answerlist}
  \item \(\Corr(X,Y)>0\)
  \item \(\Corr(X,Y)<0\)
  \item \(\forall \alpha \in [0,1]: \Var(\alpha X + (1-\alpha)Y) = 0\)
  \item \(X\) и \(Y\) независимы
  \item распределение \(X\) может быть дискретным
\end{answerlist}
\end{question}

\begin{solution}
\begin{answerlist}
  \item Bad answer :(
  \item Bad answer :(
  \item Bad answer :(
  \item Good answer :)
  \item Bad answer :(
\end{answerlist}
\end{solution}



\begin{question}
Если \(\Corr(X, Y)= 0.5\) и \(\Var(X)=\Var(Y)\), то
\(\Corr(X + Y, 2Y - 7)\) равна
\begin{answerlist}
  \item \(0\)
  \item \(\sqrt{3}/2\)
  \item \(\sqrt{2}/3\)
  \item \(1\)
  \item \(1/2\)
\end{answerlist}
\end{question}

\begin{solution}
\begin{answerlist}
  \item Bad answer :(
  \item Good answer :)
  \item Bad answer :(
  \item Bad answer :(
  \item Bad answer :(
\end{answerlist}
\end{solution}



\begin{question}
Известно, что \(\xi \sim U[0;\,1]\). Вероятность \(\P(0.2<\xi<0.7)\)
равна
\begin{answerlist}
  \item \(0.17\)
  \item \(1/4\)
  \item \(\int_{0}^{1}\frac{1}{\sqrt{2\pi}}\,e^{-t^2/2}\,dt\)
  \item \(1/2\)
  \item \(\int_{0.2}^{0.7}\frac{1}{\sqrt{2\pi}}\,e^{-t^2/2}\,dt\)
\end{answerlist}
\end{question}

\begin{solution}
\begin{answerlist}
  \item Bad answer :(
  \item Bad answer :(
  \item Bad answer :(
  \item Good answer :)
  \item Bad answer :(
\end{answerlist}
\end{solution}



\begin{question}
Cлучайные величины \(\xi_1, \, \ldots, \, \xi_n, \, \ldots\) независимы
и имеют таблицы распределения \[
\begin{tabular}{c|c|c}
$\xi_i$                     & $-1$   & $1$   \\ \cline{1-3}
$\P_{\xi_i}$        & $1/2$       & $1/2$   \\
\end{tabular}
\] Если \(S_n = \xi_1 + \ldots + \xi_n\), то предел
\(\lim\limits_{n \rightarrow \infty}\P\Bigl(\frac{S_n - \E[S_n]}{\sqrt{\Var(S_n)}} > 1\Bigr)\)
равен
\begin{answerlist}
  \item \(\int_{-\infty}^{1}\frac{1}{\sqrt{2\pi}}\,e^{-t^2/2}\,dt\)
  \item \(\int_{1}^{+\infty}\frac{1}{\sqrt{2\pi}}\,e^{-t^2/2}\,dt\)
  \item \(\int_{1}^{+\infty}\frac{1}{2}\,e^{-t/2}\,dt\)
  \item \(\int_{-1}^{1}\frac{1}{\sqrt{2\pi}}\,e^{-t^2/2}\,dt\)
  \item \(0.5\)
\end{answerlist}
\end{question}

\begin{solution}
\begin{answerlist}
  \item Bad answer :(
  \item Good answer :)
  \item Bad answer :(
  \item Bad answer :(
  \item Bad answer :(
\end{answerlist}
\end{solution}



\begin{question}
Число посетителей сайта за один день является неотрицательной случайной
величиной с математическим ожиданием 400 и дисперсией 400. Вероятность
того, что за 100 дней общее число посетителей сайта превысит
\(40\,400\), приближённо равна
\begin{answerlist}
  \item \(0.0553\)
  \item \(0.3413\)
  \item \(0.0227\)
  \item \(0.1359\)
  \item \(0.9772\)
\end{answerlist}
\end{question}

\begin{solution}
\begin{answerlist}
  \item Bad answer :(
  \item Bad answer :(
  \item Good answer :)
  \item Bad answer :(
  \item Bad answer :(
\end{answerlist}
\end{solution}



\begin{question}
Размер выплаты страховой компанией является неотрицательной случайной
величиной с математическим ожиданием \(10\,000\) рублей. Согласно
неравенству Маркова, вероятность того, что очередная выплата превысит
\(50\,000\) рублей, ограничена сверху числом
\begin{answerlist}
  \item \(0.1359\)
  \item \(0.2\)
  \item \(0.5\)
  \item неравенство Маркова здесь неприменимо
  \item \(0.3413\)
\end{answerlist}
\end{question}

\begin{solution}
\begin{answerlist}
  \item Bad answer :(
  \item Good answer :)
  \item Bad answer :(
  \item Bad answer :(
  \item Bad answer :(
\end{answerlist}
\end{solution}



\begin{question}
Монетку подбрасывают три раза. Рассмотрим три события: \(A\) --- «хотя
бы один раз выпала решка»; \(B\) --- «хотя бы один раз выпал орёл»;
\(C\) --- «все три раза выпал орёл».
\begin{answerlist}
  \item События \(A\) и \(B\) несовместны, события \(B\) и \(C\) совместны.
  \item События \(A\) и \(B\) совместны, события \(A\) и \(C\) совместны.
  \item События \(A\) и \(B\) несовместны, события \(B\) и \(C\) несовместны.
  \item События \(A\) и \(B\) совместны, события \(A\) и \(C\) несовместны.
  \item События \(A\) и \(B\) несовместны, события \(A\) и \(C\) совместны.
\end{answerlist}
\end{question}

\begin{solution}
\begin{answerlist}
  \item Bad answer :(
  \item Bad answer :(
  \item Bad answer :(
  \item Good answer :)
  \item Bad answer :(
\end{answerlist}
\end{solution}



\begin{question}
Размер выплаты страховой компанией является неотрицательной случайной
величиной с математическим ожиданием \(50\,000\) рублей и стандартным
отклонением \(10\,000\) рублей. Согласно неравенству Чебышёва,
вероятность того, что очередная выплата будет отличаться от своего
математического ожидания не более чем на 20,000 рублей, ограничена снизу
числом
\begin{answerlist}
  \item \(3/5\)
  \item \(1/2\)
  \item \(3/4\)
  \item \(1/4\)
  \item неравенство Чебышёва здесь неприменимо
\end{answerlist}
\end{question}

\begin{solution}
\begin{answerlist}
  \item Bad answer :(
  \item Bad answer :(
  \item Good answer :)
  \item Bad answer :(
  \item Bad answer :(
\end{answerlist}
\end{solution}



\begin{question}
Вероятность поражения мишени при одном выстреле равна \(0.6\). Случайная
величина \(\xi_i\) равна \(1\), если при \(i\)-ом выстреле было
попадание, и равна \(0\) в противном случае. Предел по вероятности
последовательности \(\frac{\xi_1^{2016} + \ldots + \xi_n^{2016}}{n}\)
при \(n \rightarrow \infty\) равен
\begin{answerlist}
  \item \(2/5\)
  \item \(3/4\)
  \item \(0.6^{2016}\)
  \item \(3/5\)
  \item \(1/2\)
\end{answerlist}
\end{question}

\begin{solution}
\begin{answerlist}
  \item Bad answer :(
  \item Bad answer :(
  \item Bad answer :(
  \item Good answer :)
  \item Bad answer :(
\end{answerlist}
\end{solution}



\begin{question}
Правильный кубик подбрасывается 5 раз. Вероятность того, что ровно два
раза выпадет шестерка равна
\begin{answerlist}
  \item \(1/36\)
  \item \(2/5\)
  \item \(25/(2^5 3^5)\)
  \item \(1/(2^5 3^5)\)
  \item \(125/(2^4 3^5)\)
\end{answerlist}
\end{question}

\begin{solution}
\begin{answerlist}
  \item Bad answer :(
  \item Bad answer :(
  \item Bad answer :(
  \item Bad answer :(
  \item Bad answer :(
\end{answerlist}
\end{solution}



\begin{question}
Правильный кубик подбрасывается 5 раз. Математическое ожидание и
дисперсия числа выпавших шестерок равны соответственно
\begin{answerlist}
  \item \(0\) и \(1\)
  \item \(0\) и \(5/6\)
  \item \(5/6\) и \(5/36\)
  \item \(5/6\) и \(1/36\)
  \item \(1\) и \(5/6\)
\end{answerlist}
\end{question}

\begin{solution}
\begin{answerlist}
  \item Bad answer :(
  \item Bad answer :(
  \item Bad answer :(
  \item Bad answer :(
  \item Bad answer :(
\end{answerlist}
\end{solution}



\begin{question}
Правильный кубик подбрасывается 5 раз. Наиболее вероятное число шестерок
равняется
\begin{answerlist}
  \item только \(1\)
  \item только \(0\)
  \item \(5/6\)
  \item \(5\)
  \item \(0\) и \(1\)
\end{answerlist}
\end{question}

\begin{solution}
\begin{answerlist}
  \item Bad answer :(
  \item Bad answer :(
  \item Bad answer :(
  \item Bad answer :(
  \item Good answer :)
\end{answerlist}
\end{solution}



\begin{question}
Правильный кубик подбрасывается 5 раз. Математическое ожидание суммы
выпавших очков равно
\begin{answerlist}
  \item \(21\)
  \item \(18\)
  \item \(3.5\)
  \item \(18.5\)
  \item \(17.5\)
\end{answerlist}
\end{question}

\begin{solution}
\begin{answerlist}
  \item Bad answer :(
  \item Bad answer :(
  \item Bad answer :(
  \item Bad answer :(
  \item Good answer :)
\end{answerlist}
\end{solution}



\begin{question}
Случайный вектор \((\xi, \eta)^T\) имеет нормальное распределение
\(\cN \left( \begin{pmatrix} 0 \\ 0 \end{pmatrix}; \begin{pmatrix} 1 & 1/2 \\ 1/2 & 1 \end{pmatrix} \right)\)
и функцию плотности
\(f_{\xi, \eta}(x, y) = \frac{1}{2\pi a} \exp\left(-\frac{1}{2a^2}(x^2-bxy+y^2) \right)\).
При этом
\begin{answerlist}
  \item \(a=\sqrt{3/4}\), \(b=0\)
  \item \(a=1\), \(b=0\)
  \item \(a=\sqrt{3}/2\), \(b=1\)
  \item \(a=1\), \(b=1\)
  \item \(a=1/2\), \(b=1\)
\end{answerlist}
\end{question}

\begin{solution}
\begin{answerlist}
  \item Bad answer :(
  \item Bad answer :(
  \item Good answer :)
  \item Bad answer :(
  \item Bad answer :(
\end{answerlist}
\end{solution}



\begin{question}
Случайный вектор \((\xi, \eta)^T\) имеет нормальное распределение
\(\cN \left( \begin{pmatrix} 0 \\ 0 \end{pmatrix}; \begin{pmatrix} 1 & 1/2 \\ 1/2 & 1 \end{pmatrix} \right)\).
Если случайный вектор \(z\) определён как \(z=(\xi - 0.5\eta, \eta)^T\),
то
\begin{answerlist}
  \item \((\xi - 0.5\eta)^2 + 2\eta^2 \sim \chi_2^2\)
  \item \(z\) является двумерным нормальным вектором
  \item компоненты вектора \(z\) зависимы
  \item компоненты вектора \(z\) коррелированы
  \item \(\xi - 0.5\eta \sim \cN(0;1)\)
\end{answerlist}
\end{question}

\begin{solution}
\begin{answerlist}
  \item Bad answer :(
  \item Good answer :)
  \item Bad answer :(
  \item Bad answer :(
  \item Bad answer :(
\end{answerlist}
\end{solution}



\begin{question}
Случайный вектор \((\xi, \eta)^T\) имеет нормальное распределение
\(\cN \left( \begin{pmatrix} 0 \\ 0 \end{pmatrix}; \begin{pmatrix} 1 & 1/2 \\ 1/2 & 1 \end{pmatrix} \right)\).
Условное математическое ожидание и условная дисперсия равны
\begin{answerlist}
  \item \(\E(\xi | \eta=1)=1\), \(\Var(\xi | \eta=1)=1\)
  \item \(\E(\xi | \eta=1)=0\), \(\Var(\xi | \eta=1)=1\)
  \item \(\E(\xi | \eta=1)=1/2\), \(\Var(\xi | \eta=1)=3/4\)
  \item \(\E(\xi | \eta=1)=1\), \(\Var(\xi | \eta=1)=1/2\)
  \item \(\E(\xi | \eta=1)=1/2\), \(\Var(\xi | \eta=1)=1/4\)
\end{answerlist}
\end{question}

\begin{solution}
\begin{answerlist}
  \item Bad answer :(
  \item Bad answer :(
  \item Good answer :)
  \item Bad answer :(
  \item Bad answer :(
\end{answerlist}
\end{solution}


\rule{\textwidth}{1pt}
\textbf{В вопросах 22-25} совместное распределение пары величин $X$ и $Y$ задано таблицей:

\begin{tabular}{c|ccc}
 & $Y=-1$ & $Y=0$ & $Y=1$ \\
\hline
$X=0$ & $0$ & $1/6$  &  $1/6$\\
$X=2$ & $1/3$ & $1/6$ &  $1/6$ \\
\end{tabular}


\vspace{0.2cm}

\begin{question}
Математическое ожидание случайной величины \(X\) при условии \(Y=0\)
равно
\begin{answerlist}
  \item \(1/6\)
  \item \(0\)
  \item \(-1\)
  \item \(1\)
  \item \(1/3\)
\end{answerlist}
\end{question}

\begin{solution}
\begin{answerlist}
  \item Bad answer :(
  \item Bad answer :(
  \item Bad answer :(
  \item Good answer :)
  \item Bad answer :(
\end{answerlist}
\end{solution}


% 
\begin{question}
На шахматной доске в клетке A1 стоит белая ладья. На одну из оставшихся
клеток случайным образом выставляется чёрная ладья. Вероятность того,
что ладьи «бьют» друг друга равна
\begin{answerlist}
  \item \(16/63\)
  \item \(15/64\)
  \item \(1/2\)
  \item \(14/64\)
  \item \(14/63\)
\end{answerlist}
\end{question}

\begin{solution}
\begin{answerlist}
  \item Bad answer :(
  \item Bad answer :(
  \item Bad answer :(
  \item Bad answer :(
  \item Good answer :)
\end{answerlist}
\end{solution}



\begin{question}
Вероятность того, что \(X=0\) при условии \(Y<1\) равна
\begin{answerlist}
  \item \(0\)
  \item \(1/4\)
  \item \(1/2\)
  \item \(3/4\)
  \item \(1/6\)
\end{answerlist}
\end{question}

\begin{solution}
\begin{answerlist}
  \item Bad answer :(
  \item Good answer :)
  \item Bad answer :(
  \item Bad answer :(
  \item Bad answer :(
\end{answerlist}
\end{solution}



\begin{question}
Дисперсия случайной величины \(Y\) равна
\begin{answerlist}
  \item \(-1\)
  \item \(2/3\)
  \item \(1\)
  \item \(1/3\)
  \item \(0\)
\end{answerlist}
\end{question}

\begin{solution}
\begin{answerlist}
  \item Bad answer :(
  \item Good answer :)
  \item Bad answer :(
  \item Bad answer :(
  \item Bad answer :(
\end{answerlist}
\end{solution}



\begin{question}
Ковариация случайных величин \(X\) и \(Y\) равна:
\begin{answerlist}
  \item \(-2/3\)
  \item \(-1/3\)
  \item \(0\)
  \item \(2/3\)
  \item \(1/3\)
\end{answerlist}
\end{question}

\begin{solution}
\begin{answerlist}
  \item Bad answer :(
  \item Good answer :)
  \item Bad answer :(
  \item Bad answer :(
  \item Bad answer :(
\end{answerlist}
\end{solution}


\rule{\textwidth}{1pt}

\textbf{В вопросах 26 и 27} совместное распределение пары величин $X$ и $Y$ задается функцией плотности
\[
f(x) = \begin{cases}
     				9 x^2 y^2, x \in [0,1], y \in [0,1] \\
     				0,\text{ иначе}
 				\end{cases}
\]
\vspace{0.2cm}



\begin{question}
Вероятность того, что \(X<0.5, Y<0.5\) равна:
\begin{answerlist}
  \item \(1/4\)
  \item \(1/96\)
  \item \(1/128\)
  \item \(1/64\)
  \item \(1/16\)
\end{answerlist}
\end{question}

\begin{solution}
\begin{answerlist}
  \item Bad answer :(
  \item Bad answer :(
  \item Bad answer :(
  \item Good answer :)
  \item Bad answer :(
\end{answerlist}
\end{solution}



\begin{question}
Условное распределение \(X\) при условии \(Y=1\) имеет вид
\begin{answerlist}
  \item Не определено
  \item \(f(x) = \begin{cases} 3 x^2 , x \in [0,1] \\ 0,\text{ иначе} \end{cases}\)
  \item \(f(x) = \begin{cases} 3 x , x \in [0,1] \\ 0,\text{ иначе} \end{cases}\)
  \item \(f(x) = \begin{cases} 9 x^2 , x \in [0,1] \\ 0,\text{ иначе} \end{cases}\)
  \item \(f(x) = \begin{cases} 9 x , x \in [0,1] \\ 0,\text{ иначе} \end{cases}\)
\end{answerlist}
\end{question}

\begin{solution}
\begin{answerlist}
  \item Bad answer :(
  \item Good answer :)
  \item Bad answer :(
  \item Bad answer :(
  \item Bad answer :(
\end{answerlist}
\end{solution}


\rule{\textwidth}{1pt}

\begin{question}
В школе три девятых класса: 9А, 9Б и 9В. В 9А классе — 50\% отличники,
в 9Б — 30\%, в 9В — 40\%. Если сначала равновероятно выбрать один из
трёх классов, а затем внутри класса равновероятно выбрать школьника, то
вероятность выбрать отличника равна
\begin{answerlist}
  \item \((3+4+5)/3\)
  \item \(0.3\)
  \item \(0.5\)
  \item \(0.27\)
  \item \(0.4\)
\end{answerlist}
\end{question}

\begin{solution}
\begin{answerlist}
  \item Bad answer :(
  \item Bad answer :(
  \item Bad answer :(
  \item Bad answer :(
  \item Good answer :)
\end{answerlist}
\end{solution}



\begin{question}
Если \(\P(A)=0.2\), \(\P(B)=0.5\), \(\P(A | B) = 0.3\), то
\begin{answerlist}
  \item \(\P(A \cup B) = 0.8\)
  \item \(\P(A \cap B) = 0.15\)
  \item \(\P(B \cup A) = 0.3\)
  \item \(\P(A \cup B) = 0.7\)
  \item \(\P(A \cap B) = 0.05\)
\end{answerlist}
\end{question}

\begin{solution}
\begin{answerlist}
  \item Bad answer :(
  \item Good answer :)
  \item Bad answer :(
  \item Bad answer :(
  \item Bad answer :(
\end{answerlist}
\end{solution}


% 
\begin{question}
Традиционно себя называют Стрельцами люди, родившиеся с 22 ноября по 21
декабря. Из-за прецессии земной оси линия Солнце--Земля указывает в
созведие Стрельца в наше время с 17 декабря по 20 января. Предположим,
что все даты рождения равновероятны. Вероятность того, что человек,
называющий себя Стрельцом, родился в день, когда линия Солнце--Земля
указывала в созвездие Стрельца, равна
\begin{answerlist}
  \item \(4/31\)
  \item \(4/30\)
  \item \(4/35\)
  \item \(1/2\)
  \item \(5/30\)
\end{answerlist}
\end{question}

\begin{solution}
\begin{answerlist}
  \item Bad answer :(
  \item Bad answer :(
  \item Bad answer :(
  \item Bad answer :(
  \item Good answer :)
\end{answerlist}
\end{solution}



\begin{question}
Монетка выпадает орлом с вероятностью \(0.2\). Вероятность того, что при
10 подбрасываниях монетка выпадет орлом хотя бы один раз, равна
\begin{answerlist}
  \item \(2/10\)
  \item \(1 - 0.8^{10}\)
  \item \(0.2^{10}\)
  \item \(1/2\)
  \item \(C_{10}^1 0.8^{1}0.2^9\)
\end{answerlist}
\end{question}

\begin{solution}
\begin{answerlist}
  \item Bad answer :(
  \item Good answer :)
  \item Bad answer :(
  \item Bad answer :(
  \item Bad answer :(
\end{answerlist}
\end{solution}



\begin{question}
Среди покупателей магазина мужчин и женщин поровну. Женщины тратят
больше 1000 рублей с вероятностью 60\%, а мужчины --- с вероятностью
30\%. Только что был пробит чек на сумму 1234 рубля. Вероятность того,
что покупателем была женщина равна
\begin{answerlist}
  \item \(1/3\)
  \item \(0.18\)
  \item \(0.3\)
  \item \(2/3\)
  \item \(0.5\)
\end{answerlist}
\end{question}

\begin{solution}
\begin{answerlist}
  \item Bad answer :(
  \item Bad answer :(
  \item Bad answer :(
  \item Good answer :)
  \item Bad answer :(
\end{answerlist}
\end{solution}



\begin{question}
Если \(F_X(x)\) --- функция распределения случайной величины, то
\begin{answerlist}
  \item \(\P(X \in (a;b] = F_X(b) - F_X(a)\)
  \item величина \(X\) дискретна
  \item \(F_X(x)\) может принимать значение 2016
  \item \(F_X(x)\) может принимать отрицательные значения
  \item \(\lim\limits_{x \rightarrow -\infty} F_X(x) = 1\)
\end{answerlist}
\end{question}

\begin{solution}
\begin{answerlist}
  \item Good answer :)
  \item Bad answer :(
  \item Bad answer :(
  \item Bad answer :(
  \item Bad answer :(
\end{answerlist}
\end{solution}









\subsection[2014-2015]{\hyperref[sec:sol_midterm_exam_2014_2015]{2014-2015}}
\label{sec:midterm_exam_2014_2015}


\begin{question}
Совместная функция плотности пары \(X\) и \(Y\) имеет вид \[
f(x,y)=\begin{cases}
cx^2y^2, \; \text{ если } x\in[0;1], y\in [0;1] \\
0, \; \text{ иначе}
\end{cases}
\]

\vspace{0.5cm}

Константа \(c\) равна
\begin{answerlist}
  \item \(2\)
  \item \(1/4\)
  \item \(9\)
  \item \(1\)
  \item \(1/2\)
\end{answerlist}
\end{question}

\begin{solution}
\begin{answerlist}
  \item Неверно
  \item Неверно
  \item Отлично
  \item Неверно
  \item Неверно
\end{answerlist}
\end{solution}



\begin{question}
Если \(\E(X)=0\), то, согласно неравенству Чебышева,
\(\P(|X| \leq 5 \sqrt{\Var(X)})\) лежит в интервале
\begin{answerlist}
  \item \([0.96;1]\)
  \item \([0;0.2]\)
  \item \([0.8;1]\)
  \item \([0.5;1]\)
  \item \([0;0.04]\)
\end{answerlist}
\end{question}

\begin{solution}
\begin{answerlist}
  \item Отлично
  \item Неверно
  \item Неверно
  \item Неверно
  \item Неверно
\end{answerlist}
\end{solution}



\begin{question}
По случайной выборке из 100 наблюдений было оценено выборочное среднее
\(\bar{X}=20\)\\
и несмещенная оценка дисперсии \(\hat{\sigma}^2=25\). В рамках проверки
гипотезы \(H_0: \; \mu=15\) против альтернативной гипотезы
\(H_a: \; \mu>15\)\\
можно сделать следующее заключение
\begin{answerlist}
  \item Гипотеза \(H_0\) не отвергается на любом разумном уровне значимости
  \item Гипотеза \(H_0\) отвергается на уровне значимости 10\%, но не на уровне
значимости 5\%
  \item Гипотеза \(H_0\) отвергается на уровне значимости 5\%, но не на уровне
значимости 1\%
  \item Гипотеза \(H_0\) отвергается на любом разумном уровне значимости
  \item Гипотеза \(H_0\) отвергается на уровне значимости 20\%, но не на уровне
значимости 10\%
\end{answerlist}
\end{question}

\begin{solution}
\begin{answerlist}
  \item Неверно
  \item Неверно
  \item Неверно
  \item Отлично
  \item Неверно
\end{answerlist}
\end{solution}



\begin{question}
Число изюминок в булочке --- случайная величина, имеющая распределение
Пуассона. Известно, что в среднем каждая булочка содержит 13 изюминок.
Вероятность того, что в случайно выбранной булочке окажется только одна
изюминка равна:
\begin{answerlist}
  \item \(13e^-13\)
  \item \(e^-13/13\)
  \item \(e^-13\)
  \item \(1/13\)
  \item \(e^13/13!\)
\end{answerlist}
\end{question}

\begin{solution}
\begin{answerlist}
  \item Отлично
  \item Неверно
  \item Неверно
  \item Неверно
  \item Неверно
\end{answerlist}
\end{solution}



\begin{question}
Функция распределения случайной величины \(X\) имеет вид \[
F(x)=\begin{cases}
0, \; \text{ если } x<0 \\
cx^2, \; \text{ если } x\in [0;1] \\
1, \; \text{ если } x>1
\end{cases}
\]

\vspace{0.5cm}

Константа \(c\) равна
\begin{answerlist}
  \item \(2/3\)
  \item \(0.5\)
  \item \(1.5\)
  \item \(2\)
  \item \(1\)
\end{answerlist}
\end{question}

\begin{solution}
\begin{answerlist}
  \item Неверно
  \item Неверно
  \item Неверно
  \item Неверно
  \item Отлично
\end{answerlist}
\end{solution}



\begin{question}
Совместная функция плотности пары \(X\) и \(Y\) имеет вид \[
f(x,y)=\begin{cases}
cx^2y^2, \; \text{ если } x\in[0;1], y\in [0;1] \\
0, \; \text{ иначе}
\end{cases}
\]

\vspace{0.5cm}

Вероятность \(\P(X<0.5, Y<0.5)\) равна
\begin{answerlist}
  \item \(1/64\)
  \item \(1/8\)
  \item \(1/4\)
  \item \(9/16\)
  \item \(1/16\)
\end{answerlist}
\end{question}

\begin{solution}
\begin{answerlist}
  \item Отлично
  \item Неверно
  \item Неверно
  \item Неверно
  \item Неверно
\end{answerlist}
\end{solution}



\begin{question}
Случайным образом выбирается семья с двумя детьми. Событие \(A\) --- в
семье старший ребенок --- мальчик, событие \(B\) --- в семье только один
из детей --- мальчик, событие \(C\) --- в семье хотя бы один из детей
--- мальчик.
\begin{answerlist}
  \item Любые два события из \(A\), \(B\), \(C\) --- зависимы
  \item События \(A\), \(B\), \(C\) --- независимы в совокупности
  \item \(A\) и \(B\) --- независимы, \(A\) и \(C\) --- зависимы, \(B\) и \(C\)
--- зависимы
  \item \(\P(A\cap B\cap C)=\P(A)\P(B)\P(C)\)
  \item События \(A\), \(B\), \(C\) --- независимы попарно, но зависимы в
совокупности
\end{answerlist}
\end{question}

\begin{solution}
\begin{answerlist}
  \item Неверно
  \item Неверно
  \item Отлично
  \item Неверно
  \item Неверно
\end{answerlist}
\end{solution}



\begin{question}
Рассмотрим алгоритм Метрополиса-Гастингса для получения выборки
параметра с апостериорной плотностью пропорциональной \(t^2\).
Предлагаемый переход из \(a\) в \(b\) задаётся правилом, \(b = a + Z\),
где \(Z \sim \cN(0;4)\). Вероятность одобрения перехода из точки \(0.5\)
в точку \(0.3\) равна
\begin{answerlist}
  \item \(0.5\)
  \item \(0.6\)
  \item \(0.64\)
  \item \(1\)
  \item \(0.36\)
\end{answerlist}
\end{question}

\begin{solution}
\begin{answerlist}
  \item Bad answer :(
  \item Bad answer :(
  \item Bad answer :(
  \item Bad answer :(
  \item Good answer :)
\end{answerlist}
\end{solution}



\begin{question}
Функция распределения случайной величины \(X\) имеет вид \[
F(x)=\begin{cases}
0, \; \text{ если } x<0 \\
cx^2, \; \text{ если } x\in [0;1] \\
1, \; \text{ если } x>1
\end{cases}
\]

\vspace{0.5cm}

Математическое ожидание \(\E(X)\) равно
\begin{answerlist}
  \item \(3/4\)
  \item \(2/3\)
  \item \(1/4\)
  \item \(2\)
  \item \(1/2\)
\end{answerlist}
\end{question}

\begin{solution}
\begin{answerlist}
  \item Неверно
  \item Отлично
  \item Неверно
  \item Неверно
  \item Неверно
\end{answerlist}
\end{solution}



\begin{question}
Вася бросает 7 правильных игральных кубиков. Пусть величина \(X\) ---
сумма очков, выпавших на первых двух кубиках, а величина \(Y\) --- сумма
очков, выпавших на следующих пяти кубиках. Ковариация \(\Cov(X,Y)\)
равна
\begin{answerlist}
  \item \(0\)
  \item \(-2/5\)
  \item \(2/5\)
  \item \(1\)
  \item \(0.5\)
\end{answerlist}
\end{question}

\begin{solution}
\begin{answerlist}
  \item Отлично
  \item Неверно
  \item Неверно
  \item Неверно
  \item Неверно
\end{answerlist}
\end{solution}




\begin{question}
Случайным образом выбирается семья с двумя детьми. Событие \(A\) — в
семье старший ребенок — мальчик, событие \(B\) — в семье только один
из детей — мальчик, событие \(C\) — в семье хотя бы один из детей
--- мальчик. Вероятность \(\P(A | C)\) равна
\begin{answerlist}
  \item \(1/2\)
  \item \(2/3\)
  \item \(1/4\)
  \item \(1\)
  \item \(3/4\)
\end{answerlist}
\end{question}

\begin{solution}
\begin{answerlist}
  \item Неверно
  \item Отлично
  \item Неверно
  \item Неверно
  \item Неверно
\end{answerlist}
\end{solution}



\begin{question}
Пусть случайные величины \(X\) и \(Y\) — независимы, тогда
\textbf{НЕ ВЕРНЫМ} является утверждение
\begin{answerlist}
  \item \(\Cov(X,Y) = 0\)
  \item \(\E(X|Y)=\E(X)\)
  \item \(\P(X<a, Y<b)=\P(X<a)\P(Y<b)\)
  \item \(\E(XY)=\E(X)\E(Y)\)
  \item \(\Var(X-Y)<\Var(X)+\Var(Y)\)
\end{answerlist}
\end{question}

\begin{solution}
\begin{answerlist}
  \item Неверно
  \item Неверно
  \item Неверно
  \item Неверно
  \item Отлично
\end{answerlist}
\end{solution}



\begin{question}
Совместное распределение пары величин \(X\) и \(Y\) задано таблицей:

\begin{tabular}{@{}c|ccc@{}}
\toprule
       & $Y=-1$ & $Y=0$ & $Y=1$ \\ \midrule
$X=-1$ & $1/4$  & $0$   & $1/4$ \\
$X=1$  & $1/6$  & $1/6$ & $1/6$ \\ \bottomrule
\end{tabular}

\vspace{0.5cm}

Вероятность того, что \(X=1\) при условии, что \(Y<0\) равна
\begin{answerlist}
  \item \(5/12\)
  \item \(1/3\)
  \item \(1/6\)
  \item \(1/12\)
  \item \(2/5\)
\end{answerlist}
\end{question}

\begin{solution}
\begin{answerlist}
  \item Неверно
  \item Неверно
  \item Неверно
  \item Неверно
  \item Отлично
\end{answerlist}
\end{solution}



\begin{question}
Совместное распределение пары величин \(X\) и \(Y\) задано таблицей:

\begin{tabular}{@{}c|ccc@{}}
\toprule
       & $Y=-1$ & $Y=0$ & $Y=1$ \\ \midrule
$X=-1$ & $1/4$  & $0$   & $1/4$ \\
$X=1$  & $1/6$  & $1/6$ & $1/6$ \\ \bottomrule
\end{tabular}

\vspace{0.5cm}

Ковариация, \(\Cov(X,Y)\), равна
\begin{answerlist}
  \item \(-0.5\)
  \item \(0\)
  \item \(1\)
  \item \(0.5\)
  \item \(-1\)
\end{answerlist}
\end{question}

\begin{solution}
\begin{answerlist}
  \item Неверно
  \item Отлично
  \item Неверно
  \item Неверно
  \item Неверно
\end{answerlist}
\end{solution}



\begin{question}
Совместная функция плотности пары \(X\) и \(Y\) имеет вид \[
f(x,y)=\begin{cases}
cx^2y^2, \; \text{ если } x\in[0;1], y\in [0;1] \\
0, \; \text{ иначе}
\end{cases}
\]

\vspace{0.5cm}

Математическое ожидание \(\E(X/Y)\) равно
\begin{answerlist}
  \item \(9/8\)
  \item \(3\)
  \item \(1\)
  \item \(2\)
  \item \(1/2\)
\end{answerlist}
\end{question}

\begin{solution}
\begin{answerlist}
  \item Отлично
  \item Неверно
  \item Неверно
  \item Неверно
  \item Неверно
\end{answerlist}
\end{solution}



\begin{question}
Вася бросает 7 правильных игральных кубиков. Вероятность того, что ровно
на пяти из кубиков выпадет шестёрка равна
\begin{answerlist}
  \item \(\frac{525}{12}\left(\frac{1}{6}\right)^7\)
  \item \(\left(\frac{1}{6}\right)^5\)
  \item \(\left(\frac{1}{6}\right)^7\)
  \item \(525\left(\frac{1}{6}\right)^7\)
  \item \(\frac{7}{12}\left(\frac{1}{6}\right)^5\)
\end{answerlist}
\end{question}

\begin{solution}
\begin{answerlist}
  \item Неверно
  \item Неверно
  \item Неверно
  \item Отлично
  \item Неверно
\end{answerlist}
\end{solution}



\begin{question}
Имеется три монетки. Две «правильных» и одна --- с «орлами» по обеим
сторонам. Вася выбирает одну монетку наугад и подкидывает ее один раз.
Вероятность того, что была выбрана неправильная монетка, если выпал
орел, равна
\begin{answerlist}
  \item \(3/5\)
  \item \(1/3\)
  \item \(2/3\)
  \item \(1/2\)
  \item \(3/2\)
\end{answerlist}
\end{question}

\begin{solution}
\begin{answerlist}
  \item Неверно
  \item Неверно
  \item Неверно
  \item Отлично
  \item Неверно
\end{answerlist}
\end{solution}



\begin{question}
Пусть \(X_1\), \(X_2\), \ldots, \(X_n\) — последовательность
независимых одинаково распределенных случайных величин, \(\E(X_i)=3\) и
\(\Var(X_i)=9\). Следующая величина имеет асимптотически стандартное
нормальное распределение
\begin{answerlist}
  \item \(\frac{X_n-3}{3}\)
  \item \(\sqrt{n}\frac{\bar{X}-3}{3}\)
  \item \(\frac{\bar{X}_n-3}{3}\)
  \item \(\frac{\bar{X}_n-3}{3\sqrt{n}}\)
  \item \(\sqrt{n}(\bar{X}-3)\)
\end{answerlist}
\end{question}

\begin{solution}
\begin{answerlist}
  \item Неверно
  \item Отлично
  \item Неверно
  \item Неверно
  \item Неверно
\end{answerlist}
\end{solution}



\begin{question}
Функция распределения случайной величины \(X\) имеет вид \[
F(x)=\begin{cases}
0, \; \text{ если } x<0 \\
cx^2, \; \text{ если } x\in [0;1] \\
1, \; \text{ если } x>1
\end{cases}
\]

\vspace{0.5cm}

Вероятность того, что величина \(X\) примет значение из интервала
\([0.5, 1.5]\) равна
\begin{answerlist}
  \item \(1\)
  \item \(2/3\)
  \item \(3/2\)
  \item \(1/2\)
  \item \(3/4\)
\end{answerlist}
\end{question}

\begin{solution}
\begin{answerlist}
  \item Неверно
  \item Неверно
  \item Неверно
  \item Неверно
  \item Отлично
\end{answerlist}
\end{solution}



\begin{question}
Известно, что \(\E(X)=1\), \(\Var(X)=1\), \(\E(Y)=4\), \(\Var(Y)=9\),
\(\Cov(X,Y)=-3\)

\vspace{0.5cm}

Ковариация \(\Cov(2X-Y,X+3Y)\) равна
\begin{answerlist}
  \item \(-18\)
  \item \(-40\)
  \item \(22\)
  \item \(40\)
  \item \(18\)
\end{answerlist}
\end{question}

\begin{solution}
\begin{answerlist}
  \item Неверно
  \item Отлично
  \item Неверно
  \item Неверно
  \item Неверно
\end{answerlist}
\end{solution}




\begin{question}
Совместное распределение пары величин \(X\) и \(Y\) задано таблицей:

\begin{tabular}{@{}c|ccc@{}}
\toprule
       & $Y=-1$ & $Y=0$ & $Y=1$ \\ \midrule
$X=-1$ & $1/4$  & $0$   & $1/4$ \\
$X=1$  & $1/6$  & $1/6$ & $1/6$ \\ \bottomrule
\end{tabular}

\vspace{0.5cm}

Дисперсия случайной величины \(Y\) равна
\begin{answerlist}
  \item \(5/12\)
  \item \(1/3\)
  \item \(1/2\)
  \item \(12/5\)
  \item \(5/6\)
\end{answerlist}
\end{question}

\begin{solution}
\begin{answerlist}
  \item Неверно
  \item Неверно
  \item Неверно
  \item Неверно
  \item Отлично
\end{answerlist}
\end{solution}



\begin{question}
Математическое ожидание случайной величины \(X\) при условии \(Y=0\)
равно
\begin{answerlist}
  \item \(1/6\)
  \item \(0\)
  \item \(-1\)
  \item \(1\)
  \item \(1/3\)
\end{answerlist}
\end{question}

\begin{solution}
\begin{answerlist}
  \item Bad answer :(
  \item Bad answer :(
  \item Bad answer :(
  \item Good answer :)
  \item Bad answer :(
\end{answerlist}
\end{solution}



\begin{question}
Вася бросает 7 правильных игральных кубиков. Дисперсия суммы выпавших
очков равна
\begin{answerlist}
  \item \(7\)
  \item \(7\cdot \frac{35}{36}\)
  \item \(7/6\)
  \item \(35/36\)
  \item \(7\cdot\frac{35}{12}\)
\end{answerlist}
\end{question}

\begin{solution}
\begin{answerlist}
  \item Неверно
  \item Неверно
  \item Неверно
  \item Неверно
  \item Отлично
\end{answerlist}
\end{solution}



\begin{question}
Пусть \(X_1\), \(X_2\), \ldots, \(X_n\) --- последовательность
независимых одинаково распределенных случайных величин, \(\E(X_i)=\mu\)
и \(\Var(X_i)=\sigma^2\). Следующее утверждение в общем случае
\textbf{НЕ ВЕРНО}:
\begin{answerlist}
  \item \(\bar{X}_n \overset{P}{\to} \mu\) при \(n\to\infty\)
  \item \(\frac{X_n-\mu}{\sigma} \overset{F}{\to} \cN(0;1)\) при \(n\to\infty\)
  \item \(\bar{X}_n-\mu \overset{F}{\to } 0\) при \(n\to\infty\)
  \item \(\lim_{n\to\infty} \Var(\bar{X}_n)=0\)
  \item \(\frac{\bar{X}_n-\mu}{\sqrt{n} \sigma } \overset{P}{\to } 0\) при
\(n\to\infty\)
\end{answerlist}
\end{question}

\begin{solution}
\begin{answerlist}
  \item Неверно
  \item Отлично
  \item Неверно
  \item Неверно
  \item Неверно
\end{answerlist}
\end{solution}



\begin{question}
Вася бросает 7 правильных игральных кубиков. Наиболее вероятное
количество выпавших шестёрок равно
\begin{answerlist}
  \item \(2\)
  \item \(1\)
  \item \(7/6\)
  \item \(6/7\)
  \item \(0\)
\end{answerlist}
\end{question}

\begin{solution}
\begin{answerlist}
  \item Неверно
  \item Отлично
  \item Неверно
  \item Неверно
  \item Неверно
\end{answerlist}
\end{solution}



\begin{question}
Известно, что \(\E(X)=1\), \(\Var(X)=1\), \(\E(Y)=4\), \(\Var(Y)=9\),
\(\Cov(X,Y)=-3\)

\vspace{0.5cm}

Корреляция \(\Corr(2X+3,4Y-5)\) равна
\begin{answerlist}
  \item \(1/6\)
  \item \(-1\)
  \item \(1/3\)
  \item \(1\)
  \item \(-1/8\)
\end{answerlist}
\end{question}

\begin{solution}
\begin{answerlist}
  \item Неверно
  \item Отлично
  \item Неверно
  \item Неверно
  \item Неверно
\end{answerlist}
\end{solution}



\begin{question}
Вася бросает 7 правильных игральных кубиков. Математическое ожидание
суммы выпавших очков равно
\begin{answerlist}
  \item \(42\)
  \item \(7/6\)
  \item \(30\)
  \item \(21\)
  \item \(24.5\)
\end{answerlist}
\end{question}

\begin{solution}
\begin{answerlist}
  \item Неверно
  \item Неверно
  \item Неверно
  \item Неверно
  \item Отлично
\end{answerlist}
\end{solution}



\begin{question}
Имеется три монетки. Две «правильных» и одна --- с «орлами» по обеим
сторонам. Вася выбирает одну монетку наугад и подкидывает ее один раз.
Вероятность того, что выпадет орел равна
\begin{answerlist}
  \item \(2/3\)
  \item \(1/2\)
  \item \(1/3\)
  \item \(3/5\)
  \item \(2/5\)
\end{answerlist}
\end{question}

\begin{solution}
\begin{answerlist}
  \item Отлично
  \item Неверно
  \item Неверно
  \item Неверно
  \item Неверно
\end{answerlist}
\end{solution}



\begin{question}
Совместная функция плотности пары \(X\) и \(Y\) имеет вид \[
f(x,y)=\begin{cases}
cx^2y^2, \; \text{ если } x\in[0;1], y\in [0;1] \\
0, \; \text{ иначе}
\end{cases}
\]

\vspace{0.5cm}

Условная функция плотности \(f_{X|Y=2}(x)\) равна
\begin{answerlist}
  \item \(f_X|Y=2(x)=\begin{cases} 36x^2\, \text{если} x\in [0;1] \\ 0, \text{иначе} \\ \end{cases}\)
  \item \(f_X|Y=2(x)=\begin{cases} 3x^2\, \text{если} x\in [0;1] \\ 0, \text{иначе} \\ \end{cases}\)
  \item \(f_X|Y=2(x)=\begin{cases} 9x^2\, \text{если} x\in [0;1] \\ 0, \text{иначе} \\ \end{cases}\)
  \item не определена
  \item \(f_X|Y=2(x)=\begin{cases} x^2\, \text{если} x\in [0;1] \\ 0, \text{иначе} \\ \end{cases}\)
\end{answerlist}
\end{question}

\begin{solution}
\begin{answerlist}
  \item Неверно
  \item Неверно
  \item Неверно
  \item Отлично
  \item Неверно
\end{answerlist}
\end{solution}



\begin{question}
Случайная величина \(X\) имеет функцию плотности
\(f(x)=\frac{1}{3\sqrt{2\pi}} \exp\left(-\frac{(x-1)^2}{18} \right)\).
Следующее утверждение \textbf{НЕ ВЕРНО}
\begin{answerlist}
  \item \(\P(X<0)>0\)
  \item \(\P(X=0)=0\)
  \item Случайная величина \(X\) дискретна
  \item \(\P(X>1)=0.5\)
  \item \(\E(X)=1\)
\end{answerlist}
\end{question}

\begin{solution}
\begin{answerlist}
  \item Неверно
  \item Неверно
  \item Отлично
  \item Неверно
  \item Неверно
\end{answerlist}
\end{solution}




\newpage
\thispagestyle{empty}
\section{Контрольная работа 3}



\subsection[2018-2019]{\hyperref[sec:sol_kr_03_2018_2019]{2018-2019}}
\label{sec:kr_03_2018_2019}

Примечание: минимум писали 30 минут без чит-листа, потом перерыв 10 минут,
потом задачи писали час сорок с чит-листом.

\subsubsection*{Минимум}

\begin{enumerate}
\item У случайно выбираемого взрослого мужчины рост в сантиметрах, $X$, и вес
в килограммах, $Y$,
являются нормальным случайным вектором $Z=(X,Y)$ с математическим ожиданием $E(Z)=(180,90)$ и ковариационной матрицей:

\[
\Var(Z)=\begin{bmatrix}100 & 35\\35 & 25\end{bmatrix}
\]

Рассмотрим величину $U=X-Y$. Считается, что человек страдает избыточным весом, если $U<80$.
\begin{enumerate}
\item Укажите распределение случайной величины $U$. Выпишите её плотность распределения.
\item Найдите вероятность того, что случайно выбранный мужчина страдает избыточным весом.
\item Найдите условную вероятность того, что случайно выбранный мужчина страдает избыточным весом, при условии, что его рост равен 185 см.
\end{enumerate}
\item Стоимость выборочного исследования генеральной совокупности, состоящей из двух страт, определяется по формуле $TC=c_{1}n_{1}+c_{2}n_{2}$, где $c_{i}$ — цена одного наблюдения в $i$-ой страте, а $n_{i}$ — число наблюдений, которое приходится на $i$-ую страту. Найдите $n_{1}$ и $n_{2}$, при которых дисперсия стратифицированного среднего достигает наименьшего значения, если бюджет исследования 10000 и имеется следующая информация:

			\begin{tabular}{c|c|c|c}
			Страта & 1 & 2 \\
			\hline
			Среднее значение & 10 & 20  \\ \hline
			Стандартная ошибка & 20 & 10  \\ \hline
			Вес & 10\% & 90\% \\
			\hline
			Цена наблюдения & 1 & 4 \\
			\end{tabular}

\item Для независимых нормальных $\cN(\mu,\sigma^2)$ случайных величин $(X_{1}, \ldots, X_{n})$ укажите формулу доверительного интервала с уровнем доверия $(1-\alpha)$ для неизвестного математического ожидания $\mu$ при \textbf{известной} дисперсии $\sigma^2$.

\item Дайте определение распределения Стьюдента с помощью нормальных
случайных величин. Укажите диапазон возможных значений. Нарисуйте функцию плотности распределения Стьюдента при разных степенях свободы на фоне нормальной стандартной функции плотности.

Дополнительная задача для пропустивших мини-контрольную на лекции по уважительной причине

\item Для реализации случайной выборки $x=(2,2,-1,2,1)$:
\begin{enumerate}
\item Найдите вариационный ряд;
\item Найдите выборочный второй начальный момент;
\item Постройте график выборочной функции распределения.
\end{enumerate}

\end{enumerate}


\subsubsection*{Задачи}



\begin{enumerate}[resume]

%Задача №1
\item Василиса Премудрая в рамках борьбы с гендерным дресс-кодом в Тридевятом Царстве устроила распродажу всех своих пяти кокошников. Ожидаемая цена случайно выбираемого кокошника составляет $3500$ у.е., а стандартное отклонение – $500$ у.е.. За неделю было продано три кокошника.

Найдите математическое ожидание и дисперсию вырученных Василисой денег, если она продаёт кокошники по себестоимости и вероятность покупки любого из них одна и та же. \textbf{(5 баллов)}

%Задача №2
\item К Весеннему слёту по обмену премудростями
независимо готовятся $n$ Василис Премудрых.
Время подготовки каждой Василисы в часах, $X_{i}$,
имеет функцию плотности:


\[
f(x;\theta)=\begin{cases}\frac{2x}{\theta^2}\text{, при }x\in[0;\theta]\\0\text{, при }x\notin[0;\theta]\end{cases},
\]

где $\theta>0$ — неизвестный параметр.
Найдите оценку $\theta$ методом моментов. \textbf{(5 баллов)}

%Задача №3
\item Пусть $X_{1}, \ldots, X_{n}$ — случайная выборка из распределения с плотностью:

\[
f(x;\theta)=\begin{cases}\frac{2x}{\theta^2}\text{, при }x\in[0;\theta]\\0\text{, при }x\notin[0;\theta]\end{cases},
\]


где $\theta>0$ — неизвестный параметр. Для параметра $\theta$ предлагаются две оценки: $\hat{\theta}_{n}=\frac{3}{2}\overline{X}_{n}$ и $T_{n}=\max(X_{1}, \ldots, X_{n})$:

\begin{enumerate}
\item Является ли оценка $\hat{\theta}_{n}$ несмещенной оценкой неизвестного параметра $\theta$? \textbf{(3 балла)}
\item Найдите $D(\hat{\theta}_{n})$ \textbf{(3 балла)}
\item Проверьте состоятельность оценки $\hat{\theta}_{n}$
\textbf{(4 балла)}
\item Проверьте несмещенность оценки $T_{n}$ и вычислите величину смещения \textbf{(7 баллов)}
\item Какая из оценок $\hat{\theta}_{n}$ или $T_{n}$ является более эффективной согласно критерию MSE? \textbf{(8 баллов)}
\end{enumerate}

%Задача №4
\item Пусть $X_{1}, \ldots, X_{n}$ — случайная выборка из нормального распределения с нулевым математическим ожиданием и дисперсией $\theta$.

\begin{enumerate}
\item С помощью метода максимального правдоподобия найдите оценку $\hat{\theta}_{n}$ параметра $\theta$ \textbf{(6 баллов)}
\item Проверьте несмещенность найденной оценки \textbf{(3 балла)}
\item Вычислите информацию Фишера о параметре $\theta$, заключенную в выборке \textbf{(2 балла)}
\item Проверьте, является ли найденная оценка эффективной \textbf{(4 балла)}

\textbf{Подсказка}: четвёртый момент стандартной нормальной случайной величины равен 3.
\end{enumerate}

\item Пусть $X_{1}, \ldots, X_{n}$ — случайная выборка из равномерного распределения с плотностью


\[
f(x;\theta)=\begin{cases}\frac{1}{\theta}\text{, если }x\in[0,\theta]\\ 0\text{, если }x\notin[0,\theta]\end{cases},
\]

где $\theta>0$. Постройте оценку параметра $\theta$ методом максимального правдоподобия. \textbf{(10 баллов)}

\end{enumerate}



\subsection[2017-2018]{\hyperref[sec:sol_kr_03_2017_2018]{2017-2018}}
\label{sec:kr_03_2017_2018}

\subsubsection*{Минимум}

\begin{enumerate}
\item Дайте определение выборочной функции распределения.
\item Предположим, что величины $X_1$, $X_2$, \ldots, $X_n$ независимы и нормальны
$\cN(\mu;\sigma^2)$. Укажите закон распределения выборочного среднего, величины
$\frac{\bar X - \mu}{\sigma/\sqrt{n}}$, величины $\frac{\bar X - \mu}{\hat\sigma/\sqrt{n}}$,
величины $\frac{\hat\sigma^2(n-1)}{\sigma^2}$.
\item Рост в сантиметрах, случайная величина $X$, и вес в килограммах, случайная
величина $Y$, взрослого мужчины является нормальным случайным вектором $Z = (X, Y)$
с математическим ожиданием $\E(Z) = (175, 75)$ и ковариационной матрицей

\[
\Var(Z) =
\begin{pmatrix}
49 & 28 \\
28 & 36
\end{pmatrix}
\]

\begin{enumerate}
\item Найдите средний вес мужчины при условии, что его рост составляет $172$ см.
\item Выпишите условную плотность распределения веса мужчины при условии, что его
рост составляет $172$ см.
\item Найдите условную вероятность того, что человек будет иметь вес, больший $92$ кг,
при условии, что его рост составляет $172$ см.
\end{enumerate}

\item Стоимость выборочного исследования генеральной совокупности, состоящей из трёх
страт, определяется по формуле $TC = c_1n_1 + c_2n_2 + c_3n_3$, где $c_i$ — цена
одного наблюдения в $i$-ой страте, a $n_i$ — число наблюдений, которые приходятся
на $i$-ую страту. Найдите $n_1$, $n_2$ и $n_3$, при которых дисперсия стратифицированного
среднего достигает наименьшего значения, если бюджет исследования 8000 и имеется
следующая информация:

\begin{center}
\begin{tabular}{cccc}
\toprule
Страта & $1$ & $2$ & $3$  \\
\midrule
Среднее значение & $30$ & $40$ & $50$ \\
Стандартная ошибка  & $5$ & $10$ & $20$ \\
Вес & $25\%$ & $25\%$ & $50\%$ \\
Цена наблюдения & $2$ & $5$ & $8$ \\
\bottomrule
\end{tabular}
\end{center}
\end{enumerate}


\subsubsection*{Задачи}

\begin{enumerate}[resume]
	%Задача 1
\item Пусть $X_{1}, \ldots, X_{n}$ выборка из нормального распределения $N(\mu,1)$.
\begin{enumerate}
\item Выпишите функцию правдоподобия;
\item Методом максимального правдоподобия найдите оценку $\hat{\mu}$ математического
ожидания $\mu$;
\item Проверьте состоятельность и несмещённость оценки $\hat{\mu}$;
\item Вычислите информацию Фишера о параметре $\mu$, содержащуюся во всей выборке;
\item Для произвольной несмещённой оценки $\mu$ выпишите неравенство Рао-Крамера-Фреше;
\item Проверьте свойство эффективности оценки $\hat{\mu}$;
\item Найдите оценку максимального правдоподобия $\hat{\theta}$ для второго начального
момента;
\item Проверьте свойства несмещенности и асимптотической несмещенности оценки $\hat{\theta}$;
\item С помощью дельта-метода вычислите, примерно, дисперсию оценки $\hat{\theta}$;
\item Проверьте состоятельность оценки $\hat{\theta}$.
\end{enumerate}

 %Задача 2
 \item Пусть $X_{1}$, \ldots, $X_{n}$ выборка из распределения с функцией плотности:

\[
f(x)=\begin{cases}
\frac{2}{\theta^2}(\theta-x),&\text{при }x\in[0,\theta]\\
0,&\text{при }x\notin[0,\theta]
\end{cases}
\]

\begin{enumerate}
\item Методом моментов найдите оценку параметра $\theta$;
\item Приведите определение состоятельности оценки и проверьте, будет ли найденная
оценка состоятельной.
\end{enumerate}

%Задача 3
\item В прихожей лежат четыре карты «тройка». На двух из них нет денег, на двух
других 30 и 500 рублей. Вовочка не помнит, на какой из карт есть деньги, поэтому
берёт три карточки.

\begin{enumerate}
\item Найдите математическое ожидание и дисперсию средней по выбранным карточкам
суммы денег;
\item Определите, какова вероятность того, что Вовочке удастся войти в метро,
если стоимость проезда по тройке составляет 35 рублей.
\end{enumerate}

%Задача 4
\item По выборочному опросу студенческих семейных пар о расходах на ланч были
получены следующие результаты:

\begin{center}
\begin{tabular}{ccccc}
\toprule
Номер семьи & 1 & 2 & 3 & 4\\
Расходы мужа & 450 & 370 & 170 & 200\\
Расходы жены & 210 & 350 & 250 & 180\\
\bottomrule
\end{tabular}
\end{center}

Считая, что разница в расходах мужа и жены хорошо описываются нормальным распределением,
постройте 95\%-ый доверительный интервал для разницы математических ожиданий расходов
супругов. Есть ли основания утверждать, что расходы одинаковы?

	%Задание №5
\item Наблюдатель Алексей Недопускальный решил проверить честность выборов.
Ему удалось подглядеть, как проголосовали 60 избирателей. Из них 42 выбрали
действующего президента.

\begin{enumerate}
\item Постройте 95\%-ый доверительный интервал для истинной доли избирателей,
проголосовавших «за» действующего президента.
\item По результатам ЦентрИзберКома «за» действующего президента проголосовало
76.67\% населения. Согласуются ли эти данные с данными Алексея?
\item Сколько бюллетеней нужно подглядеть Алексею, чтобы с вероятностью
$0.95$ отклонение от выборочной доли проголосовавших «за» действующего
президента от истинной не превышало $0.01$?
\end{enumerate}
\end{enumerate}



\newpage
\subsection[2016-2017]{\hyperref[sec:sol_kr_03_2016_2017]{2016-2017}}
\label{sec:kr_03_2016_2017}

\begin{enumerate}

\item Дана реализация случайной выборки: $1$, $10$, $7$, $4$, $-2$. Выпишите
определения и найдите значения следующих характеристик:
\begin{enumerate}
  \item вариационного ряда,
  \item выборочного среднего,
  \item выборочной дисперсии,
  \item несмещенной оценки дисперсии,
  \item выборочного второго начального момента.
  \item Постройте выборочную функцию распределения.
\end{enumerate}


\item
Мама дяди Фёдора каждое лето приезжает в Простоквашино с тремя вечерними платьями.
Если выбирать одно платье из трёх случайно, то ожидание стоимости равно 11 тысяч рублей,
а дисперсия стоимости равна 3 тысячи квадратных рублей.
Рачительный кот Матроскин случайным образом выбирает одно из платьев и продаёт его как ненужное. Вычислите математическое ожидание и дисперсию стоимости двух оставшихся платьев.

\item
Ресторанный критик ходит по трём типам ресторанов (дешевых, бюджетных и дорогих)
города N для того, чтобы оценить среднюю стоимость бизнес-ланча. В городе 40\%
дешевых ресторанов, 50\% — бюджетных и 10\% — дорогих. Стандартное отклонение
цены бизнес-ланча составляет 10, 30 и 60 рублей соответственно. В ресторане
критик заказывает только кофе. Стоимость кофе в дешевых/бюджетных/дорогих ресторанах
составляет 150, 300 и 600 рублей соответственно, а бюджет исследования — 30\,000 рублей.
\begin{enumerate}
  \item Какое количество ресторанов каждого типа нужно посетить критику, чтобы
	как можно точнее оценить среднюю стоимость бизнес-ланча при заданном бюджетном
	ограничении (округлите полученные значения до ближайших целых)?
  \item Вычислите дисперсию соответствующего стратифицированного среднего.
\end{enumerate}

\item
В «акции протеста против коррупции» в Москве 26.03.2017 по данным МВД приняло
участие 8\,000 человек. Считая, что население Москвы составляет 12\,300\,000 человек,
постройте 95\% доверительный интервал для истинной доли желающих участвовать в
подобных акциях жителей России. Можно ли утверждать, что эта доля статистически
не отличается от нуля?

\item
Для некоторой отрасли проведено исследование об оплате труда мужчин и женщин. Их зарплаты (тыс. руб. в месяц) приведены ниже:
\begin{center}
\begin{tabular}{cccccc}
  \toprule
  \text{мужчины}         &$50$    &$40$    &$45$   &$45$   &$35$   \\
  \text{женщины}         &$60$    &$30$    &$30$   &$35$   &$30$   \\ \bottomrule
\end{tabular}
\end{center}

\begin{enumerate}
  \item Считая, что распределение заработных плат мужчин хорошо описывается
	нормальным распределением, постройте
  \begin{enumerate}
    \item 99\%-ый доверительный интервал для математического ожидания заработной
		платы мужчин,
    \item 90\%-ый доверительный интервал для стандартного отклонения заработной
		платы мужчин.
  \end{enumerate}
  \item
  \begin{enumerate}
    \item Сформулируйте предпосылки, необходимые для построения доверительно интервала
		для разности математических ожиданий заработных плат мужчин и женщин.
    \item Считая предпосылки выполненными, постройте 90\%-ый доверительный интервал
		для разности математических ожиданий заработных плат мужчин и женщин.
    \item Можно ли считать зарплаты мужчин и женщин одинаковыми?
  \end{enumerate}
\end{enumerate}

\item
Пусть $X = (X_1, \, \ldots, \, X_n)$ — случайная выборка из нормального распределения
с нулевым математическим ожиданием и дисперсией $\theta$.
\begin{enumerate}
  \item Используя второй начальный момент, найдите оценку параметра $\theta$
	методом моментов.
  \item Сформулируйте определение несмещённости оценки и проверьте выполнение
	данного свойства для оценки, найденной в пункте а).
  \item Сформулируйте определение состоятельности оценки и проверьте выполнение
	данного свойства для оценки, найденной в пункте а).
  \item Найдите оценку параметра $\theta$ методом максимального правдоподобия.
  \item Вычислите информацию Фишера о параметре $\theta$, заключенную в $n$
	наблюдениях случайной выборки.
  \item Сформулируйте неравенство Рао-Крамера-Фреше.
  \item Сформулируйте определение эффективности оценки и проверьте выполнение
	данного свойства для оценки, найденной в пункте г).
\end{enumerate}

\item
Аэрофлот утверждает, что 10\% пассажиров, купивших билет, не являются на рейс.
В случайной выборке из шести рейсов аэробуса А320, имеющего 180 посадочных мест,
число не явившихся оказалось: $5$, $10$, $25$, $0$, $17$, $30$. Пусть число пассажиров
$X$, не явившихся на рейс, хорошо описывается распределением Пуассона $\P(X = k)
= \tfrac{\lambda^{k}}{k!}e^{-\lambda}$, $k \in \{0,\, 1,\, 2,\, \ldots\}$. При
помощи метода максимального правдоподобия найдите:
\begin{enumerate}
  \item оценку $\E(X)$ и её числовое значение по выборке,
  \item оценку дисперсии $X$ и её числовое значение по выборке,
  \item оценку стандартного отклонения $X$ и её числовое значение по выборке,
  \item оценку вероятности того, что на рейс явятся все пассажиры, а также найдите
	её числовое значение по выборке.
  \item Используя асимптотические свойства оценок максимального правдоподобия,
	постройте 95\% доверительный интервал для $\E(X)$.
  \item С помощью дельта-метода найдите 95\% доверительный интервал для вероятности
	полной загруженности самолёта.
\end{enumerate}
\end{enumerate}


\newpage
\subsection[2015-2016]{\hyperref[sec:sol_kr_03_2015_2016]{2015-2016}}
\label{sec:kr_03_2015_2016}

\epigraph{Ищите и обрящете, толцыте и отверзется вам}{Лука 11:9}

\begin{enumerate}
\item В студенческом буфете осталось только три булочки одинаковой привлекательности
и цены, но разной калорийности: 250, 400 и 550 ккал. Голодные Маша и Саша, не глядя
на калорийность, покупают по булочке. Найдите математическое ожидание и дисперсию
суммы поглощенных студентами калорий.
\item Дана реализация случайной выборки  независимых одинаково распределенных
случайных величин: 11, 4, 6.
\begin{enumerate}
  \item Выпишите вариационный ряд;
  \item Постройте выборочную функцию распределения;
  \item Найдите выборочную медиану распределения;
  \item Вычислите выборочное среднее и несмещенную оценку дисперсии.
\end{enumerate}

\item Найдите математическое ожидание, дисперсию и коэффициент корреляции случайных
величин $X$ и $Y$, совместное распределение которых имеет функцию плотности
\[
f(x, y) = \frac{5}{4\pi \sqrt{6}} \exp\left(
-\frac{25}{48}\left( (x-1)^2 -0.4(x-1)y + y^2 \right)
\right)
\]

\item Рост и размер обуви $(X, Y)$ взрослого мужчины хорошо описывается двумерным
нормальным распределением с математическим ожиданием $(178, 42)$ и ковариационной
матрицей
\[
C = \begin{pmatrix}
49 & 5.6 \\
5.6 & 1 \\
\end{pmatrix}
\]
\begin{enumerate}
  \item Какой процент мужчин обладает ростом выше 185 см?
  \item Являются ли рост и размер обуви случайно выбранного мужчины независимыми?
	Обоснуйте ответ.
  \item Среди мужчин с ростом 185 см, каков процент тех, кто имеет размер обуви,
	меньший сорок второго  $\P(Y < 42 \mid X=185)$?
\end{enumerate}


\item Дана случайная выборка $X_1$, \ldots, $X_n$ из равномерного распределения
$U[0; 2\theta]$.
\begin{enumerate}
  \item С помощью первого момента найдите оценку параметра  $\theta$ методом моментов;
  \item Сформулируйте определения несмещенности, состоятельности и эффективности оценок;
  \item Проверьте, будет ли найденная в пункте (а) оценка несмещенной и состоятельной.
  \item С помощью статистики $X_{(n)}= \max\{ X_1,\ldots, X_n \}$ постройте несмещенную
	оценку параметра $\theta$  вида $cX_{(n)}$. Укажите значение $c$.
  \item Проверьте, будет ли данная оценка состоятельной;
  \item Какая из двух оценок является более эффективной? Обоснуйте ответ.
\end{enumerate}

\item Вовочка хочет проверить утверждение организаторов юбилейной лотереи
«Метро-80 лет в ритме столицы», что почти треть всех билетов выигрышные.
Для этого он попросил $n$ своих друзей купить по 10 лотерейных билетов.
Пусть  $X_i$ — число выигрышных билетов друга $i$ и $p$ — вероятность выигрыша
одного билета.
\begin{enumerate}
  \item  Какое распределение имеет величина $X_i$?
  \item Запишите функцию правдоподобия $L(p)$  для выборки $X_1$, \ldots, $X_n$;
  \item Методом максимального правдоподобия найдите оценку $p$;
  \item Найдите информацию Фишера для одного наблюдения $i(p)$;
  \item Для произвольной несмещенной оценки $T(X_1, \ldots, X_n)$ запишите
	неравенство Рао-Крамера-Фреше;
  \item Будет оценка $\hat p_{ML}$ эффективной?
  \item Найдите оценку максимального правдоподобия математического ожидания и
	дисперсии выигранных произвольным другом билетов;
  \item Дана реализация случайной выборки 5 Вовочкиных друзей. Число выигрышных
	билетов  оказалось равно (3, 4, 0, 2, 6). Найдите значение точечной оценки
	вероятности выигрыша $p$. Как Вы думаете, похоже ли утверждение организаторов на правду?
\end{enumerate}

\item  Дана выборка $X_1$, $X_2$, \ldots, $X_n$ независимых одинаково распределенных
величин из распределения с функцией плотности
\[
f(x)=\begin{cases}
(1+\theta)x^\theta, \text{ если } 0<y<1, \theta+1>0 \\
0, \text{ иначе}
\end{cases}.
\]

Методом максимального правдоподобия найдите оценку параметра $\theta$.

\item Пробег (в 1000 км) автомобиля «Лада Калина» до капитального ремонта двигателя
является нормальной случайной величиной с неизвестным математическим ожиданием $\mu$
и известной дисперсией 49. По выборке из 20 автомобилей найдите значение доверительного
интервала для математического ожидания пробега с уровнем доверия $0.95$.
\end{enumerate}


\newpage
\subsection[2014-2015]{\hyperref[sec:sol_kr_03_2014_2015]{2014-2015}}
\label{sec:kr_03_2014_2015}

\begin{enumerate}

\item В студенческом буфете осталось только три булочки одинаковой привлекательности
и цены, но разной калорийности: 250, 400 и 550 ккал. Голодные Маша и Саша, не глядя
на калорийность, покупают по булочке. Найдите математическое ожидание и дисперсию
суммы поглощённых студентами калорий.

\item Ресторанный критик ходит по трём типам ресторанов (дешёвых, бюджетных и дорогих)
города N для того, чтобы оценить среднюю стоимость бизнес-ланча. В городе N 30\%
дешёвых ресторанов, 60\% бюджетных  и 10\% дорогих. Стандартное отклонение цены
бизнес-ланча составляет 10, 30 и 60 рублей соответственно. В ресторане критик
заказывает только кофе.  Стоимость кофе в дешёвых/бюджетных/дорогих ресторанах
составляет 150, 300 и 600 рублей соответственно, а бюджет  исследования — 15 000
рублей. Какое количество ресторанов каждого типа нужно посетить критику, чтобы
как можно точнее оценить среднюю стоимость бизнес-ланча при заданном бюджетном
ограничении (округлите полученные значения до ближайших целых)? Вычислите дисперсию
соответствующего стратифицированного среднего.

\item Дана случайная выборка $X_1$, \ldots, $X_n$  из некоторого распределения
с математическим ожиданием $\mu$ и дисперсией $\sigma^2$. Даны три оценки $\mu$:
 \[
\hat\mu_1 = (X_1 + X_2)/2, \quad \hat\mu_2 = X_1/4 + (X_2 + \ldots + X_{n-1})/(2n-4)
+ X_n/4, \quad \hat\mu_3 = \bar X
 \]
\begin{enumerate}
\item Какая из оценок является несмещённой?
\item Какая из оценок является более эффективной, чем остальные?
\end{enumerate}

\item Случайный вектор $(X, Y)^T$ имеет двумерное нормальное распределение
с математическим ожиданием  $(1, 2)^T$ и ковариационной матрицей
\[
C=\begin{pmatrix}
1 & -1 \\
-1 & 4
\end{pmatrix}
\]
\begin{enumerate}
\item $\P(X>1)$
\item $\P(2X+Y>2)$
\item $\E(2X+Y|X=2)$, $\Var(2X+Y|X=2)$, $\P(2X+Y>2|X=2)$
\item Сравните вероятности двух предыдущих пунктов, объясните, почему они отличаются.
Являются ли компоненты случайного вектора независимыми?
\end{enumerate}


\item Величины $X_1$, $X_2$ и $X_3$  независимы и стандартно нормально распределены.
Вычислите:
\begin{enumerate}
\item $\P(X_1^2 + X_2^2 > 6)$
\item $\P(X_1^2 / (X_2^2 + X_3^2) > 9.25 )$
\end{enumerate}

\item Дана случайная выборка $X_1$, \ldots, $X_n$ из равномерного распределения
$U[0, \theta]$.
\begin{enumerate}
\item С помощью статистики $X_{(n)}=\max\{X_1, \ldots, X_n \}$ постройте несмещённую
оценку параметра $\theta$  вида $cX_{(n)}$ (укажите значение $c$).
\item Будет ли данная оценка состоятельной?
\item Найдите оценку параметра $\theta$ методом моментов.
\item Какая из двух оценок является более эффективной?
\end{enumerate}

\item Каждый из $n$ биатлонистов одинакового уровня подготовки стреляет по мишеням
до первого промаха.  Пусть $X_i$ — число выстрелов $i$-го биатлониста,
$\P(X_i = x_i)=p^{x_i-1}(1-p)$, где $p$ — вероятность попадания при одном выстреле.
\begin{enumerate}
\item Методом максимального правдоподобия найдите оценку $p$.
\item Методом максимального правдоподобия найдите оценку математического ожидания
числа выстрелов.
\item Сформулируйте определения несмещенности, состоятельности и эффективности
оценок, и проверьте выполнение данных свойств для найденной в предыдущем пункте
оценки математического ожидания.
\end{enumerate}
\end{enumerate}


\newpage
\subsection[2013-2014]{\hyperref[sec:sol_kr_03_2013_2014]{2013-2014}}
\label{sec:kr_03_2013_2014}


Вычислите константы $B_1=\{\text{Цифра, соответствующая первой букве}$
Вашей\\ фамилии$\}$ и $B_2=\{\text{Цифра, соответствующая первой букве}$
 Вашего имени$\}$.\\
Уровень значимости для всех проверяемых гипотез $0.0\alpha$, уровень доверия
для всех доверительных интервалов $(1-0.0\alpha)$, где  $\alpha = 1+
\{\text{остаток от деления } B_1 \text{ на }  5\}$.\\

\begin{center}
\begin{tabular}{|c|c|c|c|c|c|c|c|c|c|c|c|c|c|}
\hline  А & Б & В & Г & Д & Е & Ж & З & И & К & Л & М & Н & О \\
\hline 1 & 2 & 3 & 4 & 5 & 6 & 7 & 8 & 9 & 10 & 11 & 12 & 13 & 14 \\
\hline  П & Р & С & Т & У & Ф & Х & Ц & Ч & Ш & Щ & Э & Ю & Я \\
\hline 15& 16  &  17 &  18&  19&  20&  21& 22 & 23 &  24& 25 & 26  &  27 & 28 \\
\hline
\end{tabular}
\end{center}

\begin{enumerate}
\item Вес упаковки с лекарством является нормальной случайной величиной с
неизвестными математическим ожиданием  $\mu$ и дисперсией $\sigma^2$. Контрольное
взвешивание $(10+B_1)$ упаковок показало, что выборочное среднее  $\bar{X} =
(50+B_2)$, а  несмещенная оценка дисперсии равна $B_1\cdot B_2$. Постройте
доверительные интервалы для математического ожидания и дисперсии веса упаковки
(для дисперсии односторонний с нижней границей).

\item Экзамен принимают два преподавателя, случайным образом выбирая студентов.
По выборкам из 85 и 100 наблюдений, выборочные доли не сдавших экзамен студентов
составили соответственно $\frac{1}{B_1+1}$ и $\frac{1}{B_2+1}$ . Можно ли утверждать,
что преподаватели предъявляют к студентам одинаковый уровень требований? Вычислите
минимальный уровень значимости, при котором основная гипотеза (уровень требований одинаков)
отвергается (p-value).

\item Даны независимые выборки доходов выпускников двух ведущих экономических вузов
A и B, по $(10+B_1)$ и $(10+B_2)$ выпускников соответственно: $\bar{X}_A=45$,
$\hat{\sigma}_A=5$, $\bar{X}_B=50$, $\hat{\sigma}_B=6$.
Предполагая, что распределение доходов подчиняется нормальному закону, проверьте
гипотезу об отсутствии преимуществ выпускников вуза B.

\item 	По выборке независимых одинаково распределенных случайных величин
$X_1,\dots,X_n$ с функцией плотности $f(x)=\frac{1}{\theta} x^{-1+\frac{1}{\theta}}$,
$x\in(0, 1)$, найдите оценки максимального правдоподобия параметра $\theta$.
Сформулируйте определения свойств несмещенности, состоятельности и эффективности
и проверьте, выполняются ли эти свойства для найденной оценки.
\end{enumerate}
\underline{Примечание.} В помощь несчастным, забывшим формулу интегрирования по
частям и таблицу неопределенных интегралов, или просто ленивым студентам:
\[
\int\limits_{0}^1 t^\alpha \ln (t) dt = -\frac{1}{(\alpha+1)^2}
\]

% TODO: check typsetting below this line

\subsection[2011-2012]{\hyperref[sec:sol_kr_03_2011_2012]{2011-2012}}
\label{sec:kr_03_2011_2012}


Условия: 80 минут, без официальной шпаргалки.

\begin{enumerate}
\item Наблюдения $X_1$, $X_2$, \ldots, $X_n$ независимы и одинаково распределены с функцией плотности $f(x)=\lambda \exp(-\lambda x)$ при $x\geq 0$.
\begin{enumerate}
\item Методом максимального правдоподобия найдите оценку параметра  $\lambda$.
\item Найдите оценку максимального правдоподобия $\hat{a}$ для параметра $a=1/\lambda$.
\item Сформулируйте определение несмещенности оценки и проверьте выполнение данного свойства для оценки $\hat{a}$.
\item Сформулируйте определение состоятельности оценки и проверьте выполнение данного свойства для оценки $\hat{a}$.
\item Сформулируйте определение эффективности  оценки и проверьте выполнение данного свойства для оценки $\hat{a}$.
\item Оцените параметр $\lambda$ методом моментов.
\end{enumerate}
Подсказка: $\E(X_i^2)=2/\lambda^2$

\item В ходе анкетирования 100 сотрудников банка «Альфа» ответили на вопрос о том, сколько времени они проводят на работе ежедневно. Среднее выборочное оказалось равно $9.5$ часам при выборочном стандартном отклонении $0.5$ часа.
\begin{enumerate}
\item Постройте 95\% доверительный интервал для математического ожидания времени проводимого сотрудниками на работе.
\item Проверьте гипотезу о том, что в среднем люди проводят на работе 10 часов, против альтернативной гипотезы о том, что в среднем люди проводят на работе меньше 10 часов, укажите точное Р-значение.
\item Сформулируйте предпосылки, которые были использованы для проведения теста.
\end{enumerate}

\item В ходе анкетирования 20 сотрудников банка «Альфа» ответили на вопрос о том, сколько времени они проводят на работе ежедневно. Среднее выборочное оказалось равно 9.5 часам при стандартном отклонении 0.5 часа. Аналогичные показатели для 25 сотрудников банка «Бета» составили 9.8 и 0.6 часа соответственно.
\begin{enumerate}
\item Проверьте гипотезу о равенстве дисперсий времени, проводимого на работе, сотрудниками банков «Альфа» и «Бета». Укажите необходимые предпосылки относительно распределения наблюдаемых значений.
\item Проверьте гипотезу о том, что сотрудники банка «Альфа» проводят на работе столько же времени, что и сотрудники банка «Бета». Укажите необходимые предпосылки относительно распределения наблюдаемых значений.
\end{enumerate}
\end{enumerate}


\subsection[2010-2011]{\hyperref[sec:sol_kr_03_2010_2011]{2010-2011}}
\label{sec:kr_03_2010_2011}

Решение задач с обозначением «\verb|MIN|» необходимо и достаточно для получения удовлетворительной оценки за данную контрольную работу.\par\smallskip

\begin{enumerate}
\item Во время эпидемии гриппа среди привитых людей заболевают в среднем 15\,\%, среди непривитых — 20\,\%. Ежегодно прививаются 10\,\% всего населения (прививка действует один год).
\begin{enumerate*}
\item \verb|MIN| Какой процент населения заболевает во время эпидемии гриппа?
\item Каков процент привитых среди заболевших людей?
\end{enumerate*}

\item Известно, что случайная величина $X\sim\cN(3,25)$.
\begin{enumerate*}
\item \verb|MIN| Найдите вероятности $\P(X>4)$ и $\P(4<X\leqslant 5)$.
\item Дополнительно известно, что случайная величина $Y$ имеет распределение $\cN(1,16)$, что $X$ и $Y$ имеют совместное нормальное распределение и что $\Corr(X,Y)=0.4$. Найдите $\P(X-2Y<4)$.
\item Случайная величина $Z\sim \cN(6,49)$ обладает тем свойством, что $\Var\left(X-2Y+\frac{1}{\sqrt{7}}Z\right)=88$. Найдите условную вероятность $\P(X-2Y<4 | Z>8)$.
\end{enumerate*}

\item Опрос домохозяйств, проживающих в Южном (выборка $X$) и Юго-Западном (выборка $Y$)
административных округах города Москвы, выявил следующие результаты:

\begin{tabular}{@{}lccccccccccccccc@{}}
\toprule
$X$  & $8.4$ & $15.6$ & $21.2$ & $15.2$ & $38.2$ & $28.3$ & $19.1$ & $44.1$ & $68.2$ & $56.0$ & $34.5$ & $33.8$ & $84.2$ & $45.0$ & $28.2$  \\
$Y$ & $54.8$ & $26.6$ & $14.4$ & $22.0$ & $23.9$ & $43.3$ & $65.1$ & $18.0$ & $69.2$ & $32.0$ & $46.7$ & $64.0$  \\ \bottomrule
\end{tabular}

Вычислены следующие суммы: $\sum\limits_i X_i=540$, $\sum\limits_i Y_i=480$, $\sum\limits_i \frac{X_i^2}{15}=1\,706.264$, $\sum\limits_i \frac{Y_i^2}{12}=1\,958.3$, $\sum\limits_i \frac{(X_i-36)^2}{15}\hm=410.264$, $\sum\limits_i \frac{(Y_i-40)^2}{12}=358.3$, $\sum\limits_i \frac{(X_i-40)^2}{15}=426.264$, $\sum\limits_i \frac{(Y_i-36)^2}{12}=374.3$.
\begin{enumerate*}
\item \verb|MIN| Постройте 90\,\% доверительный интервал для математического ожидания дохода в Юго-Западном АО.
\item На 5\,\% уровне значимости проверьте гипотезу о том, что средний доход в Юго-Западном АО не превышает среднего дохода в Южном АО, предполагая, что распределения доходов нормальны.
\item Проверьте гипотезу о равенстве распределений доходов в двух округах, используя статистику Вилкоксона-Манна-Уитни, на 5\,\% уровне значимости. Разрешается использование нормальной аппроксимации.
\end{enumerate*}

\item Вася решил проверить известное утверждение о том, что бутерброд падает маслом вниз. Для этого он провёл серию из 200 испытаний. Ниже приведена таблица с результатами:

\begin{center}
  \begin{tabular}{@{}ccc@{}}
  \toprule
  Бутерброд        & Маслом вверх & Маслом вниз \\ \midrule
  Число наблюдений & $105$        & $95$        \\ \bottomrule
  \end{tabular}
\end{center}\par\smallskip
\verb|MIN| Можно ли утверждать, что бутерброд падает маслом вниз так же часто, как и маслом вверх? (Уровень значимости 0.05.)
\par\medskip
\item
\begin{enumerate*}
\item \verb|MIN| По случайной выборке $X_1, \ldots, X_n$ из нормального распределения
$\cN(\mu_1, \mu_2-\mu_1^2)$ методом моментов оценить параметры $\mu_1$, $\mu_2$.
Дайте определения несмещённости и состоятельности и проверьте выполнение этих свойств
для оценки $\mu_1$.
\item По случайной выборке $X_1, \ldots, X_n$ из нормального распределения
$\cN(\theta, 1)$ методом максимального правдоподобия оцените параметр $\theta$.
Будет ли найденная оценка эффективной?
Ответ обоснуйте.
\end{enumerate*}
\end{enumerate}




\subsection[2009-2010]{\hyperref[sec:sol_kr_03_2009_2010]{2009-2010}}
\label{sec:kr_03_2009_2010}


\begin{enumerate}
\item Имеются наблюдения $-1.5$, $2.6$, $1.2$, $-2.1$, $0.1$, $0.9$. Найдите выборочное среднее, выборочную дисперсию. Постройте эмпирическую функцию распределения.
\item Известно, что в урне всего $n_{t}$ шаров. Часть этих шаров — белые. Количество белых шаров, $n_{w}$, неизвестно. Мы извлекаем из урны $n$ шаров без возвращения. Количество белых шаров в выборке, $X$, — это случайная величина и $\nu=X/n$. Найдите $\E(\nu)$, $\\Var(\nu)$. Будет ли $\nu$ состоятельной оценкой неизвестной доли $p=n_{w}/n_{t}$ белых шаров в выборке? Будет ли оценка $\nu$ несмещенной? Дайте определение несмещенной оценки.
\item Стоимость выборочного исследования генеральной совокупности, состоящей из трёх страт определяется по формуле $TC=150n_1+40n_2+15n_3$, где $n_i$ — количество наблюдений в выборке, относящихся к $i$-ой страте. Стоимость исследования фиксирована. Цель исследования — получить несмещенную оценку среднего по генеральной совокупности с наименьшей дисперсией. Сколько наблюдений нужно взять из каждой страты, если:

\begin{tabular}{@{}lrrr@{}}
\toprule
Страта             & 1      & 2      & 3      \\ \midrule
Стандартная ошибка & $50$   & $20$   & $10$   \\
Вес                & $10\%$ & $30\%$ & $60\%$ \\
Цена наблюдения    & $150$  & $40$   & $15$   \\ \bottomrule
\end{tabular}

\item По выборке $X_1$, $X_2$, \ldots, $X_n$ найдите методом моментов оценку параметра $\theta$ равномерного распределения $\cU[0;\theta]$. Является ли она несмещенной? Является ли она состоятельной? Какая оценка эффективнее, оценка метода моментов или оценка $T=\frac{n+1}{n}\max\{X_1,\ldots,X_n\}$?
\item Неправильная монетка подбрасывается $n$ раз. Количество выпавших орлов — случайная величина $X$.  Найдите оценку вероятности выпадения орла. Проверьте несмещенность, состоятельность и эффективность этой оценки.
\item «Насяльника» отправил Равшана и Джамшуда измерить ширину и длину земельного участка. Равшан и Джамшуд для надежности измеряют длину и ширину 100 раз. Равшан меряет длину, результат каждого измерения — случайная величина $X_i=a+e_i$, где $a$ — истинная длина участка, а $e_i\sim \cN(0,1)$ — ошибка измерения. Джамшуд меряет ширину, результат каждого измерения — случайная величина $Y_i=b+u_i$, где $b$ — истинная ширина участка, а $u_i\sim \cN(0,1)$ — ошибка измерения. Все ошибки независимы. Думая, что «насяльника» хочет измерить площадь участка, Равшан и Джамшуд каждый раз сообщают «насяльнику» только величину $S_i = X_iY_i$.

Помогите «насяльнику» оценить параметры $a$ и $b$ по отдельности методом моментов. По выборке оказалось, что $\sum s_i=3600$ сотен метров, $\sum s_i^2 =162500$ квадратных сотен метров.
\end{enumerate}







\subsection[2008-2009]{\hyperref[sec:sol_kr_03_2008_2009]{2008-2009}}
\label{sec:kr_03_2008_2009}

\subsubsection*{Часть I}

\begin{enumerate}
\item Если $X\sim \cN(0;1)$, то $X^{2}\sim \chi^{2}_{1}$. Верно. Нет.
\item Если $X\sim t_{n}$ и $Y\sim t_{m}$, то $\frac{X/n}{Y/m}\sim F_{n,m}$. Верно. Нет.
\item Если основная гипотеза отвергается  при 1\% уровне значимости, то она будет отвергаться и при 5\% уровне значимости. Верно. Нет.
\item Неравенство Рао-Крамера справедливо только для оценок максимального правдоподобия. Верно. Нет.
\item Оценки метода максимального правдоподобия всегда несмещенные. Верно. Нет.
\item Ошибка второго рода происходит при отвержении основной гипотезы, когда она верна. Верно. Нет.
\item Из несмещенности оценки следует её состоятельность. Верно. Нет.
\item Длина доверительного интервала увеличивается при увеличении уровня доверия (доверительной вероятности) Верно. Нет.
\item Выборочное среднее независимых одинаково распределенных случайных величин с конечной дисперсией имеет асимптотически нормальное распределение. Верно. Нет.
\item Теорема Муавра-Лапласа  является частным случаем ЦПТ. Верно. Нет.
\item Оценка, получаемая за эту контрольную, является несмещенной. Верно. Нет.
\end{enumerate}

Любой ответ на 11 считается правильным.

\subsubsection*{Часть II-A.}

Стоимость задач 10 баллов. Теория вероятностей.

Нужно решить любые \textbf{\underbar{3 (три)}} задачи из части II-A.

\begin{enumerate}
% числа выверены
\item При контроле правдивости показаний подозреваемого на «детекторе лжи» вероятность признать ложью ответ, не соответствующий действительности, равна 0.99, вероятность ошибочно признать ложью правдивый ответ равна 0.01. Известно, что ответы, не соответствующие действительности, составляют 1\% всех ответов подозреваемого.
Какова вероятность того, что ответ, признанный ложью, и в самом деле не соответствует действительности?
\item Предположим, что вероятность того, что среднегодовой доход наугад выбранного жителя некоторого города не превосходит уровень $t$, равна $\P(I\le t)=a+be^{-t/300}$ при $t\ge 0$.
\begin{enumerate}
\item Найдите числа $a$ и $b$.
\item Найдите математическое ожидание, моду и медиану дохода жителей города. Какую из данных характеристик следует использовать для рапорта о высоком уровне жизни?
\end{enumerate}

\item Доходности акций двух компаний являются случайными величинами $X$ и $Y$ с одинаковым математическим ожиданием и ковариационной матрицей $\left( \begin{array}{cc}
   4 & -2  \\
   -2 & 9  \\
\end{array}\right)$
\begin{enumerate}
\item Найдите $\Corr(X,Y)$.
\item В какой пропорции нужно приобрести акции этих двух компаний, чтобы дисперсия доходности получившегося портфеля была наименьшей?

Подсказка: Если $R$ — доходность портфеля, то $R=\alpha X+(1-\alpha)Y$
\item Можно ли утверждать, что величины $X+Y$ и $7X-2Y$ независимы?
\end{enumerate}

\item Волшебный Сундук

Если присесть на Волшебный Сундук, то сумма денег, лежащих в нем, увеличится в два раза. Изначально в Сундуке был один рубль. Предположим, что «посадки» на Сундук — Пуассоновский процесс с интенсивностью $\lambda$. Каково ожидаемое количество денег в Сундуке к моменту времени $t$?

\item На окружности единичной длины случайным образом равномерно и независимо друг от друга выбирают две дуги: длины 0.3 и длины 0.4.
\begin{enumerate}
\item Найдите функцию распределения длины пересечения этих отрезков.
\item Найдите среднюю длину пересечения.
\end{enumerate}
\end{enumerate}

\subsubsection*{Часть II-B.}

Стоимость задач 10 баллов. Построение и свойства оценок.

Нужно решить любые \textbf{\underbar{2 (две)}} задачи из части II-B.

\begin{enumerate}
\item[6.] Асимметричная монета подбрасывается $n$ раз. При этом $X$ раз выпал «орел».
\begin{enumerate}
\item Методом максимального правдоподобия найдите оценку вероятности «орла».
\item Проверьте является ли полученная оценка состоятельной, несмещенной и эффективной.
\item Считая, что $n$ велико, укажите, в каких пределах с вероятностью 0,95 должно находиться значение оценки, если монета симметрична.
\end{enumerate}

\item[7.] Вася попадает по мишени с вероятностью $p$ при каждом выстреле независимо от других. Он стрелял до 3-х промахов (не обязательно подряд). При этом у него получилось $X$ попаданий.
\begin{enumerate}
\item Постройте оценку $p$ с помощью метода максимального правдоподобия.
\item Является ли полученная оценка несмещенной?
\end{enumerate}

\item[8.] Известно, что величины $X_{1}$, \ldots, $X_{n}$ независимы и равномерны на $[0;b]$. Пусть $Y$ — это минимум этих $n$ величин. Вася знает $n$ и $Y$.
\begin{enumerate}
\item Найдите оценку $b$ методом моментов.
\item Является ли полученная оценка несмещенной?
\end{enumerate}
\end{enumerate}

\subsubsection*{Часть II-C.}

Стоимость задач 10 баллов. Проверка гипотез и доверительные интервалы.

Нужно решить любые \textbf{\underbar{3 (три)}} задачи из части II-C.

\begin{enumerate}
\item[9.] Вес выпускаемого заводом кирпича распределен по нормальному закону. По выборке из 16 штук средний вес кирпича составил 2.9 кг, выборочное стандартное отклонение 0.3. Постройте 80\% доверительные интервалы для истинного значения веса кирпича и стандартного отклонения.

Примечание: можно строить односторонний интервал для стандартного отклонения, если таблицы не хватает, чтобы построить двусторонний.

\item[10.] В городе N за год родились 520 мальчиков и 500 девочек.
\begin{enumerate}
\item Проверьте гипотезу о том, что мальчики и девочки рождаются одинаково чаще против гипотезы о том, что мальчиков рождается больше, чем девочек.
\item Вычислите Р-значение (минимальный уровень значимости, при котором основная гипотеза отвергается).
\item Каким должен быть размер выборки, чтобы с вероятностью 0.95 можно было утверждать, что выборочная доля отличается от теоретической не более, чем на 0.02?
\end{enumerate}

\item[11.] Проверьте гипотезу о независимости пола респондента и предпочитаемого им сока.

\begin{tabular}{@{}cccc@{}}
\toprule
  & Апельсиновый & Томатный & Вишнёвый \\ \midrule
Мужчины & $70$         & $40$     & $25$     \\
Женщины & $75$         & $60$     & $35$     \\ \bottomrule
\end{tabular}

\item[12.] Даны независимые выборки доходов выпускников двух ведущих экономических вузов A и B, по 50 выпускников каждого вуза: $\bar{X}_{A}=650$, $\bar{X}_{B}=690$, $\hat{\sigma}_{A}=50$, $\hat{\sigma}_{B}=70$.

Предполагая, что распределение доходов подчиняется нормальному закону, проверьте гипотезу об отсутствии преимуществ выпускников вуза B на уровне значимости 0.05.

\item[13.] Величины $X_{1}$, $X_{2}$, \ldots, $X_{100}$ независимы и распределены $\cN(10,16)$. Вася знает дисперсию, но не знает среднего. Поэтому он строит 60\% доверительный интервал для истинного среднего значения.
\begin{enumerate}
\item Найдите вероятность того, что доверительный интервал накрывает настоящее среднее.
\item Найдите вероятность того, что доверительный интервал накрывает число 9.
\end{enumerate}
\end{enumerate}

\subsubsection*{Часть III.}

Стоимость задачи 20 баллов.

Нужно решить любую \textbf{\underbar{1 (одну)}} задачу из части III.

\begin{enumerate}
\item[14-A.] Набранную книгу независимо друг от друга вычитывают два корректора. Первый корректор обнаружил $m_{1}$ опечаток, второй заметил $m_{2}$ опечаток. При этом $m$ опечаток оказались обнаруженными и первым, и вторым корректорами.
\begin{enumerate}
\item Постройте любым методом состоятельную оценку для общего числа опечаток, замеченных и незамеченных.
\item Является ли построенная оценка несмещенной?
\end{enumerate}

\item[14-B.] Вася хочет купить чудо-швабру! Магазинов, где продается чудо-швабра, бесконечно много. Любое посещение магазина связано с издержками равными $c>0$. Цена чудо-швабры в каждом магазине имеет равномерное распределение на отрезке $[0;M]$. Цены в магазинах не меняются, то есть при желании Вася может вернуться в уже посещенный им магазин для совершения покупки.
\begin{enumerate}
\item Как выглядит оптимальная стратегия Васи? Вася нейтрален к риску.
\item Каковы ожидаемые Васины затраты при использовании оптимальной стратегии?
\item Сколько магазинов в среднем будет посещено?
\end{enumerate}
\emph{Подсказка}: Думайте!
\end{enumerate}


\subsection[2007-2008. Демо]{\hyperref[sec:sol_kr_03_2007_2008_demo]{2007-2008. Демо}}
\label{sec:kr_03_2007_2008_demo}

\subsubsection*{Часть II.}

Стоимость задач 10 баллов.

\begin{enumerate}
\item Вася и Петя метают дротики по мишени. Каждый из них сделал
по 100 попыток. Вася оказался метче Пети в 59 попытках.
\begin{enumerate}
\item На уровне
значимости 5\% проверьте гипотезу о том, что меткость Васи и Пети
одинаковая, против альтернативной гипотезы о том, что Вася метче
Пети.
\item Чему равно точное $P$-значение при проверке гипотезы в п. «а»?
\end{enumerate}

% числа выверены
\item Из 10 опрошенных студентов часть предпочитала готовиться по
синему учебнику, а часть - по зеленому. В таблице представлены их
итоговые баллы.

\begin{tabular}{@{}lcccccc@{}}
\toprule
Синий   & 76 & 45 & 57 & 65 &    &    \\
Зелёный & 49 & 59 & 66 & 81 & 38 & 88 \\ \bottomrule
\end{tabular}


С помощью теста Манна-Уитни (Mann-Whitney) проверьте гипотезу о
том, что выбор учебника не меняет закона распределения оценки.

% числа выверены
\item Имеется случайная выборка $X_{1}$, $X_{2}$, \ldots, $X_{n}$, где все $X_{i}$ имеют распределение, задаваемое табличкой:

\begin{tabular}{@{}lccc@{}}
\toprule
$x$         & $1$ & $2$   & $5$     \\ \midrule
$\P(X=x)$ & $a$ & $0.1$ & $0.9-1$ \\ \bottomrule
\end{tabular}
\begin{enumerate}
\item Постройте оценку неизвестного $a$ методом моментов.
\item Является ли построенная оценка состоятельной?
\end{enumerate}

% числа выверены
\item Имеется случайная выборка $X_{1}$, $X_{2}$, \ldots, $X_{n}$, где все $X_{i}$ имеют $\cN(27,a)$ распределение.
Найдите оценку неизвестного $a$ методом максимального правдоподобия.

Напоминалка: не забудьте проверить условия второго порядка

% числа выверены
\item На курсе два потока, на первом потоке учатся 40 человек, на втором
потоке 50 человек. Средний балл за контрольную на первом потоке
равен 78 при (выборочном) стандартном отклонении в 7 баллов. На
втором потоке средний балл равен 74 при (выборочном) стандартном
отклонении в 8 баллов.
\begin{enumerate}
\item Постройте 90\% доверительный интервал для разницы баллов между
двумя потоками.
\item На 10\%-ом уровне значимости проверьте гипотезу о том, что
результаты контрольной между потоками не отличаются.
\end{enumerate}


% числа выверены
\item Проверьте независимость пола респондента и предпочитаемого
им сока:

\begin{tabular}{@{}cccc@{}}
\toprule
  & Апельсиновый & Томатный & Вишнёвый \\ \midrule
М & $69$         & $40$     & $23$     \\
Ж & $74$         & $62$     & $34$     \\ \bottomrule
\end{tabular}

% числа выверены
\item На Древе познания Добра и Зла растет 6 плодов познания Добра и 5 плодов познания Зла. Адам и Ева съели по 2 плода. Какова вероятность того, что Ева познала Зло, если Адам познал Добро?

 % числа выверены
\item Пусть $X_{i}$ — независимы и имеют функцию плотности $p(t)=e^{a-t}$ при $t>a$, где $a$ - неизвестный параметр. В качестве оценки неизвестного $a$ используется $\hat{a}_{n}=\min\{X_{1},X_{2}, \ldots, X_{n}\}$.
\begin{enumerate}
\item Является ли предлагаемая оценка состоятельной?
\item Является ли предлагаемая оценка несмещенной?
\end{enumerate}


\end{enumerate}

\subsubsection*{Часть III.}

Стоимость задачи 20 баллов.

Требуется решить \textbf{\underbar{одну}} из двух 9-х задач по
выбору!

\begin{enumerate}
\item[9-A.] Имеются две монетки. Одна правильная, другая — выпадает орлом с
вероятностью $0.45$. Одну из них, неизвестно какую, подкинули $n$
раз и сообщили Вам, сколько раз выпал орел. Ваша задача проверить
гипотезу $H_{0}$: «подбрасывалась правильная монетка» против
$H_{a}$:
«подбрасывалась неправильная монетка».

Каким должно быть наименьшее $n$ и критерий выбора гипотезы, чтобы
вероятность ошибок первого рода не превышала 10\%, а вероятность
ошибки второго рода не превышала 15\%?

\item[9-B.] Пусть величины $X_{i}$ — независимы и равномерны $\cU[-b;b]$. Имеется выборка из 2-х наблюдений. Вася строит оценку для $b$ по формуле $\hat{b}=c\cdot (|X_{1}|+|X_{2}|)$.
\begin{enumerate}
\item При каком $c$ оценка будет несмещенной?
\item При каком $c$ оценка будет минимизировать средне-квадратичную ошибку, $MSE=\E((\hat{b}-b)^{2})$?
\end{enumerate}
\end{enumerate}



\subsection[2007-2008]{\hyperref[sec:sol_kr_03_2007_2008]{2007-2008}}
\label{sec:kr_03_2007_2008}



\subsubsection*{Часть I.}

Какие утверждения являются истинными?

\begin{enumerate}
\item Мощность теста можно рассчитать заранее, до проведения теста.
\item Точное $P$-значение можно рассчитать заранее, до проведения теста.
\item Если гипотеза отвергает при 5\%-ом уровне значимости, то
она обязательно будет отвергаться и при 10\%-ом уровне значимости.
\item Мощность больше у того теста, у которого вероятность ошибки
1-го рода меньше.
\item Функция плотности $F$-распределения $p(t)$ не определена при $t<0$.
\item При большом $k$ случайную величину, имеющую $\chi_{k}^{2}$ распределение, можно считать нормально распределенной.
\item Оценки метода моментов всегда несмещенные.
\item Оценки метода максимального правдоподобия асимптотически несмещенные.
\item Непараметрические тесты можно использовать, даже если закон распределения выборки неизвестен.
\item Неравенство Крамера-Рао применимо только к оценкам метода максимального правдоподобия.
\end{enumerate}

Тест не является блокирующим.

Обозначения: $\E(X)$ — математическое ожидание, $\Var(X)$ — дисперсия.


\subsubsection*{Часть II.}

Стоимость задач 10 баллов.

\begin{enumerate}
\item Школьник Вася аккуратно замерял время, которое ему требовалось, чтобы добраться от школы до дома. По результатам 90 наблюдений, среднее выборочное оказалось равным 14 мин, а несмещенная оценка дисперсии — 5 мин$^{2}$.
\begin{enumerate}
\item Постройте 90\% доверительный интервал для среднего времени на дорогу.
\item На уровне значимости 10\% проверьте гипотезу о том, что среднее время равно 14,5 мин, против альтернативной гипотезы о меньшем времени.
\item Чему равно точное $P$-значение при проверке гипотезы в пункте «б»?
\end{enumerate}

% числа выверены
\item Садовник осматривал розовые кусты и записывал число цветков. Всего в саду растет 25 розовых кустов. Предположим, что количество цветков на разных кустах независимы и одинаково распределены.
Вот заметки садовника:

12, 17, 21, 14, 15; 21, 16, 24, 11, 14; 22, 17, 21, 14, 15; 12, 26, 14, 21, 14; 11, 31, 18, 13, 18.

Проверьте гипотезу о том, что медиана количества цветков равна 19.

% числа выверены
\item Имеется случайная выборка $X_{1}$, $X_{2}$, \ldots, $X_{n}$, где все $X_{i}$ имеют распределение, задаваемое табличкой:

\begin{tabular}{@{}cccc@{}}
\toprule
$x$         & $1$ & $2$  & $5$    \\ \midrule
$\P(X=x)$ & $a$ & $2a$ & $1-3a$ \\ \bottomrule
\end{tabular}
\begin{enumerate}
\item Постройте оценку неизвестного $a$ методом моментов.
\item Является ли построенная оценка несмещенной?
\end{enumerate}

% числа выверены
\item Имеется случайная выборка $X_{1}$, $X_{2}$, \ldots, $X_{n}$, где все $X_{i}$ имеют $\cN(a,4a)$ распределение.
Найдите оценку неизвестного $a$ методом максимального правдоподобия.

Напоминалка: не забудьте проверить условия второго порядка.

 % числа выверены
\item Допустим, что логарифм дохода семьи имеет нормальное распределение. В городе А была проведена случайная выборка 40 семей, показавшая выборочную дисперсию 20 (тыс.р.)$^{2}$. В городе Б по 30 семьям выборочная дисперсия оказалась равной 32 (тыс.р.)$^{2}$.
На уровне значимости 5\% проверьте гипотезу о том, что дисперсия (логарифма дохода) одинакова, против альтернативной гипотезы о том, что город А более однородный.

 % числа выверены
\item Учебная часть утверждает, что все три факультатива («Вязание крючком для экономистов», «Экономика вышивания крестиком» и «Статистические методы в макраме») одинаково популярны. В этом году на эти факультативы соответственно записалось 35, 31 и 40 человек. Правдоподобно ли заявление учебной части?

% может изменить одно из 0.7 на 0.6?
% числа выверены
\item Снайпер попадает в «яблочко» с вероятностью 0.8, если в предыдущий раз он попал в «яблочко» и с вероятностью 0.7, если в предыдущий раз он не попал в «яблочко» или если это был первый выстрел. Снайпер стрелял по мишени 3 раза.
\begin{enumerate}
\item Какова вероятность попадания в «яблочко» при втором выстреле?
\item Какова вероятность попадания в «яблочко» при втором выстреле, если при первом снайпер попал, а при третьем — промазал?
\end{enumerate}

% числа выверены
\item Пусть $X_{i}$ — независимы и распределены равномерно на $[a-1;a]$, где $a$ — неизвестный параметр. В качестве оценки неизвестного $a$ используется $\hat{a}_{n}=\max\{X_{1},X_{2},...,X_{n}\}$.
\begin{enumerate}
\item Является ли предлагаемая оценка состоятельной?
\item Является ли предлагаемая оценка несмещенной?
\end{enumerate}
\end{enumerate}

\subsubsection*{Часть III.}

Стоимость задачи 20 баллов.

Требуется решить \textbf{\underbar{одну}} из двух 9-х задач по
выбору!

\begin{enumerate}
\item[9-A.] Два лекарства испытывали на мужчинах и женщинах. Каждый
человек принимал только одно лекарство. Общий процент людей,
почувствовавших улучшение, больше среди принимавших лекарство А.
Процент мужчин, почувствовавших улучшение, больше среди принимавших лекарство В. Процент женщин, почувствовавших улучшение, больше среди принимавших лекарство В. Возможно ли это?

\item[9-B.] Есть два золотых слитка, разных по весу. Сначала взвесили первый слиток и получили результат $X$. Затем взвесили второй слиток и получили результат $Y$. Затем взвесили оба слитка и получили результат $Z$. Допустим, что ошибка каждого взвешивания — это случайная величина с нулевым средним и дисперсией $\sigma^{2}$.
\begin{enumerate}
\item Придумайте наилучшую оценку веса первого слитка.
\item Сравните придуманную Вами оценку с оценкой, получаемой путем усреднения двух взвешиваний первого слитка.
\end{enumerate}
\end{enumerate}


\subsection[2006-2007]{\hyperref[sec:sol_kr_03_2006_2007]{2006-2007}}
\label{sec:kr_03_2006_2007}

\subsection[2005-2006]{\hyperref[sec:sol_kr_03_2005_2006]{2005-2006}}
\label{sec:kr_03_2005_2006}

\newpage
\section{Контрольная работа 3. ИП}



\subsection[2017-2018]{\hyperref[sec:sol_kr_03_ip_2017_2018]{2017-2018}}
\label{sec:kr_03_ip_2017_2018}

дата: 2018-03-24

24 марта 2018 года — Комоедица, день пробуждения медведя.


\begin{enumerate}
  \item Медведь Михайло-Потапыч уснул в берлоге и ему снится сон про $n$-мерное пространство.
    Особенно ярко ему снится вектор $X=(X_1, X_2, \ldots, X_n)$ и вектор $e=(1, 1, 1, \ldots, 1)$.
    \begin{enumerate}
      \item Изобразите векторы $X$ и $e$ в $n$-мерном пространстве;
      \item Изобразите проекцию $X$ на $\Lin \{e\}$, обозначим её $\hat X$;
      \item Изобразите проекцию $X$ на $\Linp \{e\}$, обозначим её $\hat X^{\perp}$;
      \item Выпишите явно вектора $\hat X$ и $\hat X^{\perp}$, и найдите их длины;
      \item Сформулируйте теорему Пифагора для нарисованного прямоугольного треугольника;
      \item Изобразите на рисунке такой угол $\alpha$, что обычная $t$-статистика, используемая при построении доверительного интервала для $\mu$, имела бы вид $t = \sqrt{n-1} \cdot \ctg \alpha$.
    \end{enumerate}

  \item Исследователь Михаил предполагает, что все виды медведепришельцев встречаются равновероятно.
    Отправившись на охоту в район Малой Медведицы Михаил поймал двух лиловых кальмаромедведей,
    одного двурога медведеспинного и одного медведезавра ящероголового.

      Помогите Михаилу оценитель общее количество видов медведепришельцев с помощью метода максимального правдоподобия.


    \item Помотавшись по просторам Вселенной Михаил изменил своё мнение.
      Никто кроме кальмаромедведей, двурогов и медведезавров не попадается, однако попадаются они явно с разной вероятностью.
      Из 300 отловленных пришельцев оказалось 150 кальмаромедведей, 100 двурогов и 50 медведезавров.
      Михаил считает, что медведепришельцы встречаются независимо, $p_1$ — вероятность встретить кальмаромедведя, $p_2$ — двурога.

      \begin{enumerate}
	\item Оцените вектор $p = (p_1, p_2)$ методом максимального правдоподобия;
	\item Оцените ковариационную матрицу $\Var(\hat p)$;
	\item Оцените дисперсию $\Var(\hat p_1 - \hat p_2)$;
	\item Постройте доверительный интервал для разницы долей $p_1 - p_2$.
      \end{enumerate}

  \item Винни-Пух лично измерил количество мёда (в кг) на 100 деревьях и обнаружил, что $\bar X = 10$ и $\hat\sigma^2 = 4$.
    По мнению Кролика, состоятельная оценка для параметра $\alpha$ правильности мёда имеет вид $\hat \alpha = \bar X + \sqrt{\bar X + 6}$.

    \begin{enumerate}
      \item «Халява, сэр!» Найдите точечную оценку параметра $\alpha$;
      \item Найдите 95\%-ый доверительный интервал для $\alpha$, симметричный относительно $\hat\alpha$.
    \end{enumerate}

  \item Фотографы Андрей и Белла независимо друг от друга пытаются фотографировать кадьяков.
    Андрею удаётся сфотографировать одного кадьяка в неделю с вероятностью $0.5$, а Белле — с вероятностью $p$,
    независимо друг от друга и от прошлого.
    За 100 недель они вместе сфотографировали 130 кадьяков.

    \begin{enumerate}
      \item Оцените $p$ и постройте 95\%-ый доверительный интервал для $p$;
      \item Оцените $p$ и постройте 95\%-ый доверительный интервал для $p$, если дополнительно известно, что один фотограф опередил другого на 10 фото.
    \end{enumerate}

\end{enumerate}


\textbf{Просто красивая задачка}. Эту задачу \textbf{не нужно} решать на кр :)

Медведю Мишутке никак не удаётся заснуть в берлоге, и потому он подбрасывает правильную монетку $n$ раз.
Обозначим вероятность того, что ни разу не идёт двух решек подряд буквой $q_n$.


        \begin{enumerate}[label=\asbuk*)]
	  \item Найдите $2^8q_8$ и \textbf{назовите} это число;
          \item Найдите $\lim 2q_{n+1}/q_n$ и \textbf{назовите} это число.
	\end{enumerate}

% !TEX root = ../probability_hse_exams.tex

\newpage
\thispagestyle{empty}
\section{Контрольная работа 4}


\subsection[2020-2021]{\hyperref[sec:sol_kr_04_2020_2021]{2020-2021}}
\label{sec:kr_04_2020_2021}

Врями написания 120 минут, можно было использовать 
чит-лист с любым материалом. Писали очно, 
за исключением тех, кто был отправлен в дистант. 
Первые четыре задачи по темам соответствуют минимуму, однако блокирующими не являлись. 


\begin{enumerate}
\item Длина хвоста взрослого бобра хорошо описывается нормальным распределением с математическим ожиданием $\mu$ и дисперсией $\sigma^2$. 
Вася поймал четырех бобров и произвел замеры их хвостов. 
Результаты (в сантиметрах) оказались следующими: 32, 26, 28, 24. 
Помогите Васе построить 95\%-й доверительный интервал для длины хвоста взрослого бобра.

% Задача 2
\item Длина хвоста взрослого бобра на запрудах «X» и «Y» хорошо описываются нормальными распределениями с параметрами $(\mu_{X},\sigma_{X}^2)$ и $(\mu_{Y},\sigma_{Y}^2)$ соответственно. 
Вася поймал четырех бобров на запруде «X» и три бобра на запруде «Y» и произвел замеры их хвостов. 
Результаты (в сантиметрах) оказались следующими: 32, 26, 28, 24 — для запруды «X» и 26, 24, 28 — для запруды «Y». 
На уровне значимости 5\% помогите Васе проверить гипотезу $H_{0}:\sigma_{X}^2=\sigma_{Y}^2$ против гипотезы $H_{0}:\sigma_{X}^2>\sigma_{Y}^2$.

% Задача 3
\item Вася очень любит ходить в тир. Он заявляет, что с вероятностью $0.5$  попадает в «десятку», с вероятностью $0.4$  — в «девятку» и с вероятностью $0.1$ — в «восьмерку». 
Ниже приведены результаты его ста последних выстрелов:

	\begin{center}
	\begin{tabular}{cccc}
	&в восьмерку& в девятку& в десятку\\
	\toprule
	Число попаданий & 15 & 55 & 30  \\
	\end{tabular}
\end{center}

С помощью хи-квадрат критерия Пирсона на уровне значимости 5\% проверьте гипотезу о том, что Вася говорит правду.

% Задача 4
\item На уроке по литературе Вася подсаживается за парту или к Маше, или к Свете или к Вовочке. 
Учительница литературы Марья Ивановна любит проводить внезапные сочинения на внезапные темы. 
Вася пишет такие сочинения либо на двойку, либо на тройку. Других результатов у Васи не бывает. 
Имеются следующие данные о 100 сочинениях, написанных Васей:
\begin{center}
	\begin{tabular}{cccc}
	& сидел с Машей & сидел со Светой & сидел с Вовочкой\\
	\toprule
	за сочинение двойка & 10 & 15 & 25\\
	за сочинение тройка & 25 & 15 & 10\\
	\end{tabular}
\end{center}
На уровне значимости 1\% протестируйте гипотезу о том, что Васина оценка за сочинение не зависит от того, с кем сидит Вася.

% Задача 5
\item Студенты Вася и Маша независимо друг от друга каждый учебный день пытаются вовремя добраться до университета. 
Вася и Маша опаздывают с вероятностями $p$ и $q$ соответственно. За $100$ дней Вася опоздал $30$ раз, а Маша — $20$ раз.

\begin{enumerate}
	\item Постройте $90$\%-й асимптотический доверительный интервал для разницы вероятности опоздания Васи и Маши.
	\item На уровне значимости $0.1$ проверьте гипотезу о том, что Вася и Маша опаздывают с равной вероятностью.
	\item Предполагая, что Вася и Маша опаздывают с равной вероятностью (то есть $p=q$), 
	постройте асимптотический $90$\%-й доверительный интервал для вероятности того, 
	что в конкретный день хотя бы один из студентов (Вася или Маша) опоздает.
\end{enumerate}

\newpage
% Задача 6
\item Имеется единственное наблюдение $X$ с функцией плотности:

\[
f_{X}(x;\theta)
=\begin{cases}
\frac{\theta}{x^{1+\theta}}\text{, при }x\geq 1\\
0\text{, при } x < 1
\end{cases}
\]

Тестируется основная гипотеза $H_{0}:\theta = 2$ против альтернативы $H_{1}:\theta = 1$. 
\begin{enumerate}
	\item С помощью леммы Неймана–Пирсона найдите наиболее мощный критерий, имеющий уровень значимости $\alpha = 0.05$.
	\item Рассчитайте мощность найденного вами критерия.
\end{enumerate}

% Задача 7
\item Рост (в дециметрах) и вес (в килограммах) случайно взятого большого русского кота хорошо описываются независимыми хи-квадрат случайными величинами с $10$-ю и $k$ степенями свободы соответственно. 
Нулевая гипотеза  $H_{0}:k=1$ отвергается, если отношение суммарного роста к суммарному весу у двух случайно взятых котов превышает $200$. 
Найдите уровень значимости данного теста.

\textbf{Подсказка}:  Если случайная величина $X$ имеет распределение Фишера с 20-ю и 2-мя степенями свободы, 
то $\P(X<5)\approx 0.82$, $\P(X<10)\approx 0.91$, $\P(X<15)\approx 0.93$, $\P(X<20)\approx 0.95$ и $\P(X<25)\approx 0.96$.

% Задача 8
\item Филин Карл подключает кротов к интернету. Каждую ночь он обыскивает норки кротов одну за одной до тех пор, 
пока не обнаружит крота. Вероятность $p$ обнаружить крота в очередной норке остается неизменной и не зависит от результатов предшествующих поисков. 
После обнаружения крота Карл подключает ему интернет и улетает. За последние 100 ночей Карл обыскал 500 норок.

\begin{enumerate}
	\item При помощи метода максимального правдоподобия оцените вероятность того, 
	что в очередную ночь Карл обнаружит крота в первой же норке.
	\item С помощью теста отношения правдоподобия на уровне значимости 5\% проверьте гипотезу о том, 
	что в очередную ночь Карл с равной вероятностью обыщет ровно одну или более одной норки.
	\item С помощью дельта-метода постройте асимптотический 95\%-й доверительный интервал для вероятности того, 
	что на протяжении трех ночей подряд Карл будет находить крота в первой же норке.
\end{enumerate}


\includegraphics[width=5cm]{images/filin_karl_01.png}
\hspace*{3cm}
\includegraphics[width=9cm]{images/filin_karl_02.png}

% Задача 9
\item Пусть $X_{1}, \ldots, X_{n}$  — случайная выборка из равномерного распределения на отрезке $[0,\theta]$, где $\theta>0$ — неизвестный параметр распределения. Известно, что $n=100$  и $\bar x=0.57$ . Используя центральную предельную теорему, постройте приближенный 95\%-ый доверительный интервал для параметра $\theta$.

% Задача 10
\item Вовочка любит генерировать случайные числа по поводу и без. 
В этот раз он сгенерировал выборку $x_{1}=-1$, $x_{2}=0$, $x_{3}=1$, и сообщил, 
что данная случайная выборка порождена логистическим распределением с функцией распределения $\Lambda(x)=\frac{e^{x}}{1+e^{x}}$. 

С помощью теста Колмогорова на уровне значимости 1\% проверьте гипотезу о том, 
что данная случайная выборка действительно была получена из логистического распределения. 

\textbf{Подсказка}: $e \approx 2.7$, критическое значение статистики Колмогорова примите равным $0.83$.

\end{enumerate}
	


\subsection[2019-2020]{\hyperref[sec:sol_kr_04_2019_2020]{2019-2020}}
\label{sec:kr_04_2019_2020}



\subsection[2018-2019]{\hyperref[sec:sol_kr_04_2018_2019]{2018-2019}}
\label{sec:kr_04_2018_2019}

Комментарий: минимум писали 30 минут без официальных шпаргалок, основную часть 90 минут — с листом А4. 

\subsubsection*{Минимум}


\begin{enumerate}

%Задача №1
\item Пусть $X_{1}, \ldots, X_{n}$ — случайная выборка из нормального распределения с неизвестным математическим ожиданием $\mu$ и известной дисперсией $\sigma^2=100$. Объем выборки $n=25$. Для тестирования основной гипотезы $H_{0}:\mu=(-1)$ против альтернативы $H_{1}:\mu=0$ Винни-Пух использует простой критерий. Если $\bar X \leq (-0.5)$, то Винни-Пух не отвергает гипотезу $H_{0}$, в противном случае Винни-Пух отвергает гипотезу $H_{0}$ в пользу гипотезы $H_{1}$. 

Найдите:

\begin{enumerate}
    \item вероятность ошибки 1-го рода;
    \item вероятность ошибки 2-го рода;
    \item мощность критерия.
\end{enumerate}

%Задача №2
\item Пусть $X_{1}, \ldots, X_{n}$ — случайная выборка из нормального распределения $\cN(\mu, \sigma^2)$ с неизвестными параметрами $\mu$ и ${\sigma}^2$.
Используя реализацию случайной выборки,
\[
x_{1} = -2, \quad x_{2} = -1, \quad x_{3} = 0, \quad x_{4} = 1, \quad x_{5} = 2
\]
постройте 90\%-ый доверительный интервал для неизвестного параметра $\sigma^2$.

%Задача №3
\item Пусть $X_{1}, \ldots, X_{n}$ и $Y_{1}, \ldots, Y_{m}$ — независимые случайные
выборки из нормальных распределений $\cN(\mu_{X},\sigma^2_{X})$ и
$\cN(\mu_{Y},\sigma^2_{Y})$ соответственно.
Уровень значимости $\alpha = 0.1$.

Используя реализации случайных выборок

\begin{align*}
x_{1} &= -2, \quad x_{2} = -1, \quad x_{3} = 0, \quad x_{4} = 1, \quad x_{5} = 2, \\
y_{1} &= -2, \quad y_{2} = 0, \quad y_{3} = 2,
\end{align*}
проверьте гипотезу о равенстве дисперсий $\sigma^2_X = \sigma^2_Y$ против гипотезы о неравенстве дисперсий.

%Задача №4
\item Пусть $X_{1}, \ldots, X_{n}$ и $Y_{1}, \ldots, Y_{m}$ —
независимые случайные выборки из нормальных распределений 
$\cN(\mu_{X},\sigma^2)$ и $\cN(\mu_{Y},\sigma^2)$ соответственно.

Используя реализации случайных выборок
\begin{align*}
x_{1} &= -2, \quad x_{2} = -1, \quad x_{3} = 0, \quad x_{4} = 1, \quad x_{5} = 2, \\
y_{1} &= -2, \quad y_{2} = 0, \quad y_{3} = 2,
\end{align*}
постройте 60\%-ый доверительный интервал для разности математических ожиданий
$\mu_{X} - \mu_{Y}$.

%Задача 5
\item Пусть $X_{1}, \ldots, X_{n}$ и $Y_{1}, \ldots, Y_{m}$ — две независимые случайные выборки из распределения Бернулли с неизвестными параметрами $p_{X}\in(0,1)$ и $p_{Y}\in(0,1)$. Известно, что $n=100$, $m=150$, $\sum_{i=1}^{n}x_{i}=60$, $\sum_{i=1}^{n}y_{i}=50$. На уровне значимости $\alpha=0.05$ протестируйте гипотезу $H_{0}:p_{X}=p_{Y}$ против альтернативной гипотезы $H_{1}:p_{X}>p_{Y}$.


\end{enumerate}

\subsubsection*{Основная часть}


\begin{enumerate}

%Задача №1

\item Пусть $X$ — случайная выборка, содержащая одно наблюдение, из распределения с функцией плотности

\[
f_X(x) =
	\begin{cases}
	\theta x^{\theta-1},\text{ при }  x \in [0; 1] \\
	0\quad \quad \text{, }\text{ при } x \notin  [0; 1] \\
	\end{cases},
\]

где $\theta>0$ — неизвестный параметр распределения. Тестируется основная гипотеза $H_{0}:\theta=1$ против альтернативной гипотезы $H_{1}:\theta=2$.

\begin{enumerate}
    \item (7) С помощью леммы Неймана–Пирсона найдите наиболее мощный критерий, имеющий уровень значимости $\alpha=0.05$.
    \item (3) Вычислите мощность полученного критерия.
\end{enumerate}

%Задача №2

\item На раскопках старинного замка Вася нашел древнюю монетку. Монетка ему показалась крайне необычной, и поэтому он решил провести серию из 100 подбрасываний. Результаты Вася записал в таблицу.

\begin{center}\begin{tabular}{rrrr}
\toprule
   & Орел   & Решка & Ребро  \\ \midrule
Количество раз &   $40$ & $50$ & $10$ \\ \bottomrule
\end{tabular}\end{center}
\begin{enumerate}
    \item (6) Постройте двусторонний 95\%-ый доверительный интервал для вероятности выпадения ребра.
    \item (4) На уровне значимости 5\% проверьте гипотезу о том, что ребро выпадает в одном случае из 100 против альтернативы, что чаще, и вычислите соответствующее $P$-значение.
    \item (10) С помощью теста отношения правдоподобия на уровне значимости 10\% проверьте гипотезу о том, что орёл выпадает так же часто, как решка, и в три раза чаще, чем ребро.
    \item (5) С помощью критерия согласия Пирсона на уровне значимости 10\% проверьте гипотезу из предыдущего пункта.
\end{enumerate}

Подсказка: $\ln 2 \approx 0.7$, $\ln 3 \approx 1.1$, $\ln 5 \approx 1.6$, $\ln 7 \approx 1.95$, $\ln 11 \approx 2.4$.

%Задача №3

\item Губка Боб и Патрик ловят медуз для коллекции. Число пойманных за $i$-ый день медуз имеет распределение Пуассона с неизвестным параметром $\lambda$. Уловы медуз в различные дни независимы. За прошедшие $100$ дней они поймали $300$ медуз. 

\begin{enumerate}
    \item (5) С помощью асимптотических свойств оценок максимального правдоподобия постройте 95\%-ый доверительный интервал для неизвестного параметра $\lambda$.
    \item (10) С помощью дельта-метода постройте приближенную 95\%-ую интервальную оценку вероятности того, что за 101-ый день Губка Боб и Патрик не поймают ни одной медузы.
\end{enumerate}

\end{enumerate}





\subsection[2017-2018]{\hyperref[sec:sol_kr_04_2017_2018]{2017-2018}}
\label{sec:kr_04_2017_2018}


\subsubsection*{Минимум}

\begin{enumerate}


	\item Пусть $X=(X_{1}, \ldots,X_{n})$ — случайная выборка из нормального распределения с неизвестным математическим ожиданием $\mu$ и неизвестной дисперсией $\sigma^2=9$. Объем выборки $n=20$. Для тестирования основной гипотезы $H_{0}:\mu=0$ против альтернативной гипотезы $H_{1}:\mu=5$ вы используете критерий: если $\overline{X}\leq2$, то вы не отвергаете гипотезу $H_{0}$, в противном случае вы отвергаете гипотезу $H_{0}$ в пользу гипотезы $H_{1}$. Найдите
	\begin{enumerate}
	\item Вероятность ошибки 1-го рода
	\item Вероятносто ошибки 2-го рода
	\end{enumerate}


		\item Пусть $X=(X_{1}, \ldots,X_{n})$ — случайная выборка из нормального распределения с неизвестными параметрами $\mu$ и $\sigma^2$. Используя реализацию случайной выборки: $x_{1}=2$, $x_{3}=8$, $x_{3}=5$, постройте 95\%-ый доверительный интервал для неизвестного параметра $\sigma^2$.


	\item Вася очень любит тестировать статистические гипотезы. В этот раз Вася собирается проверить утверждение о том, что его друг Пётр звонит Васе исключительно в то время, когда Вася ест. Для этого Вася трудится целый год и проводит серию из 365 испытаний. Результаты приведены в табилце ниже.

	\begin{center}
		\begin{tabular}{c|cc}
			\toprule
			& Пётр не звонит & Пётр звонит\\
			\midrule
			Вася ест & $100$ & $50$\\
			Вася не ест  & $125$ & $90$\\
			\bottomrule
		\end{tabular}
	\end{center}

	На уровне значимости 10\% протестируйте гипотезу о том, что Пётр звонит Васе независимо от момента приема пищи.


\item Вася Сидоров утверждает, что ходит в кино в четыре раза чаще, чем в спортзал, а в спортзал в четыре раза чаще, чем в театр.
За последние полгода он 105 раз был в театре, 63 раза — в спортазе и 42 раза в кино.
На уровне значимости 10\% проверьте утверждение Васи.

\end{enumerate}


	\begin{center}
	Квантили $\chi^2$ распределения c 1, 2 и 3 степенями свободы\\
		\begin{tabular}{c|cccccc}
			\toprule
			& 0.025 & 0.05 & 0.1 & 0.9 & 0.95 & 0.975\\
			\midrule
							1 & 0.001 & 0.004 & 0.016 & 2.706 & 3.841 & 5.024 \\
							2 & 0.051 & 0.103 & 0.211 & 4.605 & 5.991 & 7.378 \\
							3 & 0.216 & 0.352 & 0.584 & 6.251 & 7.815 & 9.348 \\
			\bottomrule
		\end{tabular}
	\end{center}

\newpage

\subsubsection*{Задачи}

При решении задач пять–семь используйте данные обследования Росстата за первый квартал 2018 года:

	\begin{center}
		\begin{tabular}{c|ccc}
			\toprule
			& Число наблюдений & Среднее (тыс. руб.) & Выборочное отклонение (тыс. руб.) \\
			\midrule
							Врачи & 40 & 136 & 55  \\
							Преподаватели & 60 & 139 & 60 \\
			\bottomrule
		\end{tabular}
	\end{center}

Распределение заработной платы работников любой отрасли хорошо описывается нормальным законом.

\begin{enumerate}[resume]

	%Задача 1
	\item На уровне значимости 5\% проверьте гипотезу о том, что средняя зарплата врача составляет 100 т.р., против альтернативы, что она больше 100 т.р.  Вычислите минимальный уровень значимости, при котором основная гипотеза отвергается (Р–значение).

	%Задача 2
	\item На уровне значимости 10\% проверьте гипотезу о том, что разброс в зарплатах врачей и преподавателей одинаков, против двухсторонней альтернативы.

	%Задача 3
	\item На уровне значимости 10\% проверьте гипотезу о том, что средняя зарплата врачей и преподавателей совпадают, против альтернативы, что у преподавателей зарплата выше:

	\begin{enumerate}
	\item Считая объемы выборок достаточно большими
			\item Считая дисперсии одинаковыми
	 \end{enumerate}

			%Задача 4
			\item Время в часах безотказной работы  микронаушника, величина $X$, подчиняется экспоненциальному (показательному) закону распределения с неизвестным параметром $\lambda$:
\[
f(x;\lambda)
			=\begin{cases}
			\lambda e^{-\lambda x},\text{ при } x\geq0\\
			0, \text{ при }  x<0\\
			\end{cases}
\]
По выборке из 100 независимых наблюдений $\bar x=0.52$. С помощью асимптотических свойств оценок максимального правдоподобия постройте приближенный 95\%-ый доверительный интервал:

\begin{enumerate}
	\item Для параметра $\lambda$
			\item Для вероятности того, что наушник проработает без сбоев весь тест — 45 минут
	 \end{enumerate}

	 %Задача 5
	 \item Приглашенный на Петербургский международный экономический форум Германом Грефом индийский мистик Садхгуру подарил Грефу древнюю шестигранную кость для принятия решений в сложных макроэкономических ситуациях. Служба безопасности Сбербанка провела серию из 100 испытаний и составила таблицу:

			 \begin{center}
		\begin{tabular}{c|cccccc}
			\toprule
			Грань & 1 & 2 & 3 & 4 & 5 & 6\\
			\midrule
							Число выпадений & 10 & 10 & 15 & 15 & 25 & 25 \\
			\bottomrule
		\end{tabular}
	\end{center}

С помощью теста отношения правдоподобия на уровне значимости 5\% проверьте гипотезу о том, что все грани равновероятны.

\[
\ln(1/6)=-1.79, \; \ln(0.15)=-1.90, \; \ln(0.25)=-1.39, \; \ln(0.1)=-2.30
\]

\end{enumerate}


\newpage
\subsection[2016-2017]{\hyperref[sec:sol_kr_04_2016_2017]{2016-2017}}
\label{sec:kr_04_2016_2017}


\subsubsection*{I. Теоретический минимум}


В пунктах 1, 3, 11 и 12 предполагается, что $X = (X_1, \, \ldots, \, X_n)$ и
$Y = (Y_1, \, \ldots, \, Y_m)$ — две независимые случайные выборки из нормальных
распределений $N(\mu_X, \sigma_X^2)$ и $N(\mu_Y, \sigma_Y^2)$ соответственно.

\begin{enumerate}
  \item Приведите формулу статистики, при помощи которой можно проверить гипотезу
	$H_0 \colon \sigma_X^2 = \sigma_Y^2$. Укажите распределение этой статистики при
	верной гипотезе $H_0$.
  \item Приведите формулу информации Фишера о параметре $\theta$, содержащейся в
	одном наблюдении случайной выборки.
  \item Приведите формулу статистики, при помощи которой можно проверить гипотезу
	$H_0 \colon \mu_X - \mu_Y = \Delta_0$ при условии, что дисперсии $\sigma_X^2$
	и $\sigma_Y^2$ неизвестны, но равны между собой. Укажите распределение этой
	статистики при верной гипотезе $H_0$.
  \item Дайте определение критической области.
  \item Приведите формулу плотности нормального распределения $\cN(\mu, \sigma^2)$.
  \item Приведите формулы границ доверительного интервала с уровнем доверия
	$(1 - \alpha)$, $\alpha \in (0;\,1)$, для вероятности появления успеха в
	случайной выборке $X = (X_1, \, \ldots, \, X_n)$ из распределения Бернулли с
	параметром $p \in (0;\,1)$.
  \item Дайте определение несмещенной оценки $\hat{\theta}$ для неизвестного
	параметра $\theta \in \Theta$.
  \item Дайте определение эффективной оценки $\hat{\theta}$ для неизвестного
	параметра $\theta \in \Theta$.
  \item Приведите формулу выборочной дисперсии.
  \item Приведите формулу выборочной функции распределения.
  \item Приведите формулы границ доверительного интервала с уровнем доверия
	$(1 - \alpha)$, $\alpha \in (0;\,1)$, для $\mu_X$ при условии, что дисперсия
	$\sigma_X^2$ известна.
  \item Укажите распределение статистики $\frac{\overline{X} - \mu_X}{\sigma / \sqrt{n}}$.
\end{enumerate}


\subsubsection*{II. Задачи}

\begin{enumerate}
\item В ходе анкетирования ста сотрудников банка «Альфа» были получены ответы на
вопрос о том, сколько времени они проводят на работе ежедневно. Среднее выборочное
оказалось равным $9.5$ часам, а выборочное стандартное отклонение $0.5$ часа.
\begin{enumerate}
  \item На уровне значимости 5\,\% проверьте гипотезу о том, что сотрудники банка
	«Альфа» в среднем проводят на работе $10$ часов, против альтернативной гипотезы о
	том, что сотрудники банка «Альфа» в среднем проводят на работе менее $10$ часов.
  \item Найдите точное $P$-значение для наблюдаемой статистики из пункта (a).
  \item Сформулируйте предпосылки, которые были использованы вами для выполнения
	пункта (a).
  \item На уровне значимости 5\,\% проверьте гипотезу о $H_0 \colon \sigma^2 = 0.3$.
\end{enumerate}


\item Проверка сорока случайно выбранных лекций показала, что студент Халявин
присутствовал только на 16 из них. На уровне значимости 5\,\% проверьте гипотезу
о том, что Халявин посещает в среднем половину лекций.

\item В ходе анкетирования двадцати сотрудников банка «Альфа» были получены ответы
на вопрос о том, сколько времени они проводят на работе ежедневно. Среднее выборочное
оказалось равным $9.5$ часам, а выборочное стандартное отклонение $0.5$ часа.
Аналогичные показатели для 25 сотрудников банка «Бета» составили $9.8$ и $0.6$
часа соответственно.
\begin{enumerate}
  \item На уровне значимости 5\,\% проверьте гипотезу о равенстве математических
	ожиданий времени, проводимого на работе сотрудниками банков «Альфа» и «Бета».
  \item Сформулируйте предпосылки, которые были использованы вами для выполнения
	пункта (a).
  \item На уровне значимости 5\,\% проверьте гипотезу о равенстве дисперсий времени,
	проводимого на работе сотрудниками банков «Альфа» и «Бета».
\end{enumerate}

\item Вася решил проверить известное утверждение о том, что бутерброд падает
маслом вниз. Для этого он провел серию из 200 испытаний. Ниже приведена таблица
с результатами:
\begin{center}
\begin{tabular}{ccc}
  \toprule
  \text{Бутерброд}                &\text{Маслом вниз}    &\text{Маслом вверх}       \\ \midrule
  \text{Число наблюдений}         &$105$    &$95$       \\ \bottomrule
\end{tabular}
\end{center}
Можно ли утверждать, что бутерброд падает маслом вниз так же часто, как и маслом
вверх? При ответе на вопрос используйте уровень значимости 5\,\%.

\item Пусть $X = (X_1, \, \ldots, \, X_{100})$ — случайная выборка из нормального
распределения с математическим ожиданием $\mu$ и дисперсией $\nu$. Оба параметра
$\mu$ и $\nu$ неизвестны. Используя следующие данные $\sum_{i=1}^{100}x_i = 30$,
$\sum_{i=1}^{100}x_i^2 = 146$ и $\sum_{i=1}^{100}x_i^3 = 122$ с помощью теста
отношения правдоподобия проверьте гипотезу $H_0 \colon \nu = 1$ на уровне значимости 5\,\%.
\end{enumerate}


\newpage
\subsection[2015-2016]{\hyperref[sec:sol_kr_04_2015_2016]{2015-2016}}
\label{sec:kr_04_2015_2016}

\begin{enumerate}
\item	Сформулируйте определения несмещённости, состоятельности и эффективности оценок.
\item На курсе учится 250 человек. Предположим, что число студентов, не явившихся
на экзамен, хорошо описывается законом Пуассона.
\begin{enumerate}
\item	Методом максимального правдоподобия найдите оценку параметра распределения Пуассона.
\item	Проверьте выполнение свойств несмещенности, эффективности и состоятельности
для данной оценки.
\item	Найдите оценку максимального правдоподобия для вероятности стопроцентной
явки студентов на экзамен.
\item	Используя дельта-метод, постройте для этой вероятности асимптотический
доверительный интервал.
\end{enumerate}

\item	Фармацевтическая компания выпустила новое лекарство от бессонницы, утверждая,
что оно помогает 80\% людей, страдающих бессонницей. Чтобы проверить утверждение
компании, случайным образом выбираются 20 человек, страдающих бессонницей. Обозначим
за $Y$ количество человек из выборки, которым лекарство помогло. Основная гипотеза,
$H_0$: $p=0.8$, альтернативная гипотеза $H_a$: $p=0.6$. Критическая область: $\{Y<12\}$.
\begin{enumerate}
\item	В терминах этой задачи сформулируйте, что является ошибкой первого рода.
Найдите уровень значимости, соответствующий заданной критической области.
\item	В терминах этой задачи сформулируйте, что является ошибкой второго рода.
Найдите вероятность ошибки второго рода.
\item	Найдите такое значение $c$, что вероятность ошибки первого рода $\alpha
\approx 0.1$ при критической области вида $\{Y<c\}$. Найдите соответствующее
значение вероятности ошибки второго рода.
\item	Каким должен быть размер выборки, чтобы выборочная доля страдающих бессонницей
отличалась от истинной вероятности не более, чем на $0.01$ с вероятностью не менее,
чем $0.95$?
\end{enumerate}

\item	Вася Сидоров утверждает, что ходит в кино в два раза чаще, чем на лекции
по статистике, на лекции по статистике в два раза чаще, чем в спортзал. За
последние полгода он 10 раз был в спортзале, 1 раз — на лекциях по статистике и
39 раз в кино.

При помощи критерия хи-квадрат Пирсона на уровне значимости $0.05$ проверьте,
правдоподобно ли Васино утверждение.

\item У Евдокла есть случайная выборка из экспоненциального распределения с
неизвестным параметром $\lambda$ в 50 наблюдений, $X_1$, $X_2$, \ldots, $X_{50}$.
Оказалось, что $\bar X = 1.1$. Евдокл хочет проверить гипотезу о равенстве
$\lambda = 1$ против альтернативной гипотезы о неравенстве $\lambda \neq 1$ на
уровне значимости $0.1$.

Помогите Евдоклу и проверьте гипотезу с помощью критерия отношения правдоподобия.

Пачка логарифмов: $\ln 50 \approx 3.9$, $\ln 55 \approx 4.0$, $\ln 11 \approx 2.4$,
$\ln 60 \approx 4.1$, $\ln 12 \approx 2.5$

\item Американский демографический журнал опубликовал исследование, в котором
утверждается, что посетители крупных торговых центров за одно посещение тратят
в выходные дни больше, чем в будние. Наибольшие расходы приходятся на воскресенье
в период с 4 до 6 часов вечера. Для двух независимых выборок посетителей средние
расходы и выборочные стандартные отклонения расходов составили
\begin{center}
\begin{tabular}{lrr}
\toprule
 & Выходные & Рабочие дни \\
\midrule
Число наблюдений & 21 & 19 \\
Средние расходы (\$) & 78 & 67 \\
Выборочное стандартное отклонение (\$) & 22 & 20 \\
\bottomrule
\end{tabular}
\end{center}

\begin{enumerate}
\item Проверьте гипотезу о равенстве дисперсий расходов
\item Предполагая, что дисперсии расходов одинаковы, проверьте гипотезу об отсутствии
разницы в расходах в выходные и будние дни.
\item Сформулируйте все необходимые для проверки гипотез предыдущих пунктов предпосылки.
\end{enumerate}

\item Винни Пух знает, что пчёлы и мёд бывают правильные и неправильные.
По результатам 100~попыток добыть мёд Винни Пух составил таблицу сопряженности признаков.

\begin{center}
\begin{tabular}{lrr}
\toprule
 & Мёд правильный & Мёд неправильный \\
\midrule
Пчёлы правильные & 12	& 36 \\
Пчёлы неправильные & 32	& 20 \\
\bottomrule
\end{tabular}
\end{center}

На уровне значимости $0.05$ проверьте гипотезу о независимости характеристик пчёл и мёда.

\begin{figure}[b]
\centering
\includegraphics[width=9cm]{images/winnie_kr_4}
\end{figure}
\end{enumerate}



\newpage
\subsection[2014-2015]{\hyperref[sec:sol_kr_04_2014_2015]{2014-2015}}
\label{sec:kr_04_2014_2015}

\begin{enumerate}

\item[1.] \textbf{Задача для первого потока.}

Проверка 40 случайно выбранных лекций показала, что студент Халявин присутствовал
только на 16 из них.
\begin{enumerate}
\item Найдите 95\% доверительный интервал для вероятности увидеть Халявина на лекции.
\item На уровне значимости 5\% проверьте гипотезу о том, что Халявин посещает
в среднем половину лекций.
\item Вычислите минимальный уровень значимости, при котором основная гипотеза
отвергается (P-значение).
\end{enumerate}

\item[1.] \textbf{Задача для второго потока.}

Вес упаковки с лекарством является нормальной случайной величиной.
Взвешивание 20~упаковок показало, что выборочное среднее равно 51 г., а
несмещенная оценка дисперсии равна~4.
\begin{enumerate}
\item На уровне значимости 10\% проверьте гипотезу, что в среднем вес упаковки
составляет~55 г.
\item Контрольное взвешивание 30 упаковок такого же лекарства другого производителя
показало, что несмещенная оценка дисперсии веса равна 6. На уровне значимости 10\%
проверьте гипотезу о равенстве дисперсий веса упаковки двух производителей.
\end{enumerate}

\item[2.] \textbf{Задача для первого потока.}

В ходе анкетирования 15 сотрудников банка «Альфа» ответили на вопрос о том,
сколько времени они проводят на работе ежедневно. Среднее выборочное оказалось
равно $9.5$ часам при выборочном стандартном отклонении $0.5$ часа. Аналогичные
показатели для 12 сотрудников банка «Бета» составили $9.8$ и $0.6$ часа соответственно.

Считая распределение времени нормальным, на уровне значимости 5\% проверьте
гипотезу о том, что сотрудники банка «Альфа» в среднем проводят на работе столько
же времени, сколько и сотрудники банка «Бета».

\item[2.] \textbf{Задача для второго потока.}

Экзамен принимают два преподавателя, случайным образом выбирая студентов.
По выборке из 85 и 100 наблюдений, выборочные доли не сдавших экзамен студентов составили
соответственно $0.2$ и $0.17$.
\begin{enumerate}
\item Можно ли при уровне значимости в 1\% утверждать, что преподаватели предъявляют
к студентам одинаковый уровень требований?
\item Вычислите минимальный уровень значимости, при котором основная гипотеза
отвергается (P-значение).
\end{enumerate}

\item[3.] Методом максимального правдоподобия найдите оценку параметра $\theta$
для выборки $X_1$, \ldots, $X_n$ из распределения с функцией плотности
\[
f(x)=\begin{cases}
\frac{1}{\theta^2}xe^{-\frac{x}{\theta}}, \; x>0 \\
0, \; x\leq 0
\end{cases}
\]

\item[4.]
Пусть $X_1$, \ldots, $X_{100}$ — случайная выборка из нормального распределения
с математическим ожиданием $\mu$ и дисперсией $\nu$, где $\mu$ и $\nu$ — неизвестные
параметры. По 100 наблюдениям $\sum x_i=30$, $\sum x_i^2=146$, $\sum x_i^3=122$.

При помощи теста отношения правдоподобия протестируйте гипотезу $H_0: \nu=1$
на уровне значимости 5\%.

\item[5.] \textbf{Исследовательская задача.}

Пусть $X_1$, \ldots, $X_{n}$ — случайная выборка из нормального распределения
с математическим ожиданием $\mu$ и дисперсией $\nu$, где $\mu$ и $\nu$ — неизвестные
параметры. Рассмотрим три классических теста, отношения правдоподобия, $LR$,
множителей Лагранжа, $LM$ и Вальда, $W$, для тестирования гипотезы $H_0: \; \mu=0$.

\begin{enumerate}
\item Сравните  статистики $LR$, $LM$ и $W$ между собой. Какая — наибольшая,
какая — наименьшая?
\item Изменится ли упорядоченность статистик, если проверять гипотезу $H_0: \; \mu=\mu_0$?
\end{enumerate}

Подсказка: $\frac{x}{1+x} \leq \ln(1+x) \leq x\, \; \text{ при } x>-1$

\item[6.] \textbf{Исследовательская задача.}

Величины $X_1$, \ldots, $X_n$ независимы и одинаково распределены с функцией плотности
\[
f(x)=\begin{cases}
a^2xe^{-ax}, \; x>0 \\
0, \; x\leq 0
\end{cases}
\]

По выборке из 100 наблюдений оказалось, что $\sum x_i =300$, $\sum x_i^2=1000$,
$\sum x_i^3=3700$.

\begin{enumerate}
\item Найдите оценку неизвестного параметра $a$ методом моментов
\item Используя дельта-метод или иначе оцените дисперсию полученной оценки $a$
\item Постройте 95\%-ый доверительный интервал используя оценку метода моментов
\end{enumerate}
\end{enumerate}



\subsection[2009-2010]{\hyperref[sec:sol_kr_04_2009_2010]{2009-2010}}
\label{sec:kr_04_2009_2010}



\begin{enumerate}

\item Сколько нужно бросить игральных костей, чтобы вероятность выпадения хотя бы одной шестерки была не меньше $0.9$?
\item Снайпер попадает в «яблочко» с вероятностью 0.8, если он в предыдущий выстрел попал в «яблочко» и с вероятностью 0.7, если в предыдущий раз не попал в  «яблочко». Вероятность попасть в «яблочко» при первом выстреле также 0.7. Снайпер стреляет 2 раза.
\begin{enumerate}
\item Определите вероятность попасть в «яблочко» при втором выстреле
\item Какова вероятность того, что снайпер попал в «яблочко» при первом выстреле, если известно, что он попал при втором?
\end{enumerate}

\item Случайная величина $X$ моделирует время, проходящее между двумя телефонными звонками в справочную службу. Известно, что $X$ распределена экспоненциально со стандартным отклонением равным 11 минутам. Со времени последнего звонка прошло 5 минут. Найдите функцию распределения и математическое ожидание времени, оставшегося до следующего звонка.

\item Известно, что для двух случайных величин $X$ и $Y$: $\E(X)=1$, $\E(Y)=2$, $\E(X^2)=2$, $\E(Y^2)=8$, $\E(XY)=1$.
\begin{enumerate}
\item Найдите ковариацию и коэффициент корреляции величин $X$ и $Y$.
\item Определите, зависимы ли величины $X$ и $Y$.
\item Вычислите дисперсию их суммы.
\end{enumerate}

\item Предположим, что время «жизни» $X$ энергосберегающей лампы распределено по нормальному закону. По 10 наблюдениям среднее время «жизни» составило 1200 часов, а выборочное стандартное отклонение 120 часов.
\begin{enumerate}
\item Постройте двусторонний доверительный интервал для математического ожидания величины $X$ с уровнем доверия 0.90.
\item Постройте двусторонний доверительный интервал для стандартного отклонения величины $X$ с уровнем доверия 0.80.
\item Какова вероятность, что несмещенная оценка для дисперсии, рассчитанная по 20 наблюдениям, отклонится от истинной дисперсии меньше, чем на 40\%?
\end{enumerate}

\item Учебная часть утверждает, что все три факультатива «Вязание крючком для экономистов», «Экономика вышивания крестиком» и «Статистические методы в макраме» одинаково популярны. В этом году на эти факультативы соответственно записалось 35, 31 и 40 человек. Правдоподобно ли заявление учебной части?
\item Имеются две конкурирующие гипотезы:
\begin{enumerate}
\item[$H_0$:] Случайная величина X распределена равномерно на (0,100);
\item[$H_a$:] Случайная величина X распределена равномерно на (50,150).
\end{enumerate}
Исследователь выбрал следующий критерий: если $X<c$, принимать гипотезу $H_0$, иначе  $H_a$.
\begin{enumerate}
\item Дайте определение «ошибки первого рода», «ошибки второго рода», «мощности критерия».
\item Постройте графики зависимости вероятностей ошибок первого и второго рода от $c$.
\item Вычислите $c$ и вероятность ошибки второго рода, если уровень значимости критерия равен 0.05.
\end{enumerate}
\item Из 10 опрошенных студентов часть предпочитала готовиться по синему учебнику, а часть по зеленому. В таблице представлены их итоговые баллы.

\begin{tabular}{@{}lcccccc@{}}
\toprule
Синий   & 76 & 45 & 57 & 65 &    &    \\
Зелёный & 49 & 59 & 66 & 81 & 38 & 88 \\ \bottomrule
\end{tabular}


С помощью теста Манна-Уитни (Вилкоксона) проверьте гипотезу о том, что выбор учебника не меняет закона распределения оценки.

\item Случайная величина $X$, характеризующая срок службы элементов электронной аппаратуры, имеет распределение Релея: $F(x)=1-e^{-x^2/\theta}$ при $x\geq 0$. По случайной выборке $X_1$, $X_2$, ..., $X_n$ найдите оценку максимального правдоподобия параметра $\theta$.

\item По случайной выборке $X_1$, $X_2$, \ldots, $X_n$ из равномерного на интервале $[\theta;\theta+10]$ распределения методом моментов найдите оценку параметра $\theta$. Дайте определение несмещенности и состоятельности оценки и определите, будет ли обладать этими свойствами найденная оценка.

\item При расчете страхового тарифа страховая компания предполагает, что вероятность наступления страхового случая 0.005. По итогам прошедшего года из 10000 случайно выбранных договоров страховых случаев наблюдалось 67.
\begin{enumerate}
\item Согласуются ли полученные данные с предположением страховой компании? Альтернативная гипотеза: вероятность страхового случая больше.
\item Определить минимальный уровень значимости, при котором основная гипотеза отвергается.
\end{enumerate}
\end{enumerate}

% !TEX root = ../probability_hse_exams.tex
\newpage
\thispagestyle{empty}
\section{Контрольная работа 4. ИП}


\subsection[2018-2019]{\hyperref[sec:sol_kr_04_ip_2018_2019]{2018-2019}}
\label{sec:kr_04_ip_2018_2019}

Ровно 189 лет назад, 1 июня 1830 британский учёный Джон Росс открыл северный магнитный полюс :)


\begin{enumerate}
\item Пусть $y$ — стандартный нормальный $n$-мерный вектор. 
Он случайный, просто Джону Россу лень писать заглавные буквы для векторов :) 
Вектор $a$ — неслучайный, но тоже гордый $n$-мерный.

Пусть $H$ — матрица, проецирующая любой вектор на $(n-1)$-мерное подпространство $a^{\perp}$, 
являющееся ортогональным дополнением к вектору $a$. 
То есть, для любого вектора $v$ вектор $Hv$ перпендикулярен вектору $a$.

\begin{enumerate}
    \item Найдите матрицу $H$, если $n=3$ и $a=(1,2,2)'$.
    \item Найдите матрицу $H$ для произвольного $n$ и $a$;
    \item Найдите $\E(y)$ и $\Var(y)$;
    \item Найдите $\E(Hy)$ и $\Var(Hy)$;
    \item Укажите закон распределения $y'Hy$, где $y'$ — это транспонированный вектор $y$.
\end{enumerate}

\item Рассмотрим формулу, здорово упрощающую подсчёт критерия Пирсона:
\[
 \sum_{j=1}^m \frac{(X_j - np_j)^2}{np_j} + n = \sum_{j=1}^m \frac{X_j^2}{np_j}
\]

\begin{enumerate}
    \item Докажите формулу.
    \item Нарисуйте картинку к этой формуле. На картинке подпишите прямой угол, катеты и гипотенузу. 
    Явно запишите каждый вектор. Объясните, почему треугольник, действительно, прямоугольный. 
\end{enumerate}


\item На Земле короля Уильяма Джон Росс нашёл странную монетку. 
Он подбрасывает её $n$ раз и обнаруживает, что она выпадает на орла, решку и ребро. 
Джон Росс проверяет гипотезу $H_0$ о том, что все три вероятности равны.

Пусть $y = (y_1, y_2, y_3)'$ — количество выпадений орла, решки и ребра. Рассмотрим так же вектор
$z = (z_1, z_2, z_3)'$, такой, что $z_i = (y_i - \E(y_i)) / \sqrt{\E(y_i)}$. 
Джон Росс сознательно перепутал ожидание и дисперсию в классической формуле!

Предположим, что гипотеза $H_0$ верна.
\begin{enumerate}
    \item Укажите закон распределения каждой величины $y_i$;
    \item Найдите вектор $\E(y)$ и матрицу $\Var(y)$;
    \item Найдите вектор $\E(z)$ и матрицу $\Var(z)$;
    \item Докажите, что матрица $H=\Var(z)$ является проектором на ортогональное дополнение к некоторому вектору $a$. 
  Явно выпишите вектор $a$.
  \item Объясните, почему критерий Пирсона имеет хи-квадрат распределение с нужным числом степеней свободы.
  % \item Обобщите решение на случай прозвольных вероятностей или произвольного количества граней у монетки.
\end{enumerate}

\newpage

\item На Земле короля Уильяма Джон Росс нашёл странную монетку. 
Он подбрасывает её $n$ раз и обнаруживает, что она выпадает на орла, решку и ребро. 
Джон Росс проверяет гипотезу о том, что все три вероятности равны с помощью двух статистики: 
$LR$, отношения правдоподобия, и $CP$, критерия Пирсона. 

\begin{enumerate}
\item Найдите $\plim_{n\to\infty} \frac{LR}{CP}$;
\item Обобщите решение на случай произвольного количества равновероятных граней у монетки.    
% \item Обобщите решение на случай произвольного количества граней и произвольных вероятностей в гипотезе $H_0$.
\end{enumerate}

\item Идея доказательства состоятельности ML оценки :)

Пусть наблюдения $y_1$, \ldots, $y_n$ независимы и одинаково распределены с функцией плотности, зависящей от параметра $a$.
Истинное значение параметра обозначим буквой $a_0$. Оценку максимального правдоподобия обозначим $\hat a$.

Рассмотрим отмасштабированную логарифмическую функцию правдоподобия $\ell_n(a)=\ell(a) / n$, и
ожидаемую логарифмическую функцию правдоподобия\footnote{Внимание:
ожидание считается с помощью истинного $a_0$ от функции, в которую входит константа $a$.},
$\tilde \ell(a)=\E(\ell(a))$.
\begin{enumerate}
\item Что больше, $\ln x$ или $x-1$? Докажите!
\item В какой точке находится максимум функции $\ell_n(a)$?
\item В какой точке находится максимум функции $\tilde \ell(a)$?

Подсказка: рассмотрите выражение $\tilde \ell(a) - \tilde \ell(a_0)$ и примените доказанное неравество :)
\item К чему сходится $\ell_n(a)$ по вероятности?
%\item К чему сходится $\ell^{\prime\prime}_n(a)$?

\end{enumerate}


\end{enumerate}



\subsection[2017-2018]{\hyperref[sec:sol_kr_04_ip_2017_2018]{2017-2018}}
\label{sec:kr_04_ip_2017_2018}


Напутствие в добрый путь:

\begin{enumerate}
\item Работа сдаётся только в виде запроса pull-request на гитхаб-репозиторий.

\item Имя файла должно быть вида \verb|ivanov_ivan_161_kr_4.Rmd|.

\item Также фамилию и имя нужно указать в шапке документа в поле \verb|author| :)

\item Если нужно, то установите пакеты \verb|tidyverse|, \verb|maxLik|, \verb|nycflights13|.
\end{enumerate}


\begin{enumerate}
\item Симулируем бурную деятельность!
В качестве параметра $k$ в задаче используй число букв в своей фамилии в именительном падеже :)

Каждый день Василий съедает случайное количество булочек, которое распределено по Пуассону с параметром $10$. Логарифм затрат в рублях на каждую булочку распределён нормально $N(2, 1)$.
Андрей каждый день съедает биномиальное количество булочек $Bin(2k, 0.5)$. Затраты Андрей на каждую булочку распределены равномерно на отрезке $[2;20]$.

\begin{enumerate}
\item Сколько в среднем тратит Василий на булочки за день?
\item Чему равна дисперсия дневных расходов Василия?
\item Какова вероятность того, что за один день Василий потратит больше денег,
чем Андрей?
\item Какова условная вероятность того,
что Василий за день съел больше булочек, чем Андрей,
если известно, что Василий потратил больше денег?
\end{enumerate}


\item Сражаемся с реальностью!
В пакете \verb|nycflights13| встроен набор данных \verb|weather| о погоде в разные дни в разных аэропортах.
\begin{enumerate}
	\item Постройте гистограмму переменной влажность, \verb|humid|.
	У графика подпишите оси!

\item Постройте диаграмму рассеяния переменных влажность и количество осадков,
precip. У графика подпишите оси!

Посчитайте выборочное среднее и выборочную дисперсию влажности и количества осадков.

\item С помощью максимального правдоподобия оцените параметр $\mu$,
предположив, что наблюдения за влажностью имеют нормальное $N\left(\mu, 370\right)$-распределение и независимы.
Постройте 95\%-ый доверительный интервал для $\mu$.

\item С помощью максимального правдоподобия оцените параметр $\sigma^2$,
предположив, что наблюдения за влажностью имеют нормальное $N\left(60, \sigma^2\right)$-распределение и независимы.
Постройте 95\%-ый доверительный интервал для $\sigma^2$.

Если при численной оптимизации параметр $\sigma^2$ становится отрицательным,
можно задать параметры по-другому, например, $\sigma^2 = \exp(\gamma)$.
\end{enumerate}
\end{enumerate}

% \section{Финальные экзамены}

% [1][3] 1 = one argument, 3 = value if missing
% эта магия создаёт окружение answerlist
% именно в окружении answerlist записаны варианты ответов в подключаемых exerciseXX
% просто \begin{answerlist} сделает ответы в три столбца
% если ответы длинные, то надо в них руками сделать
% \begin{answerlist}[1] чтобы они шли в один столбец
\newenvironment{answerlist}[1][3]{
\begin{multicols}{#1}
\begin{enumerate}[label=\fbox{\emph{\Alph*}},ref=\emph{\alph*}]
}
{
\end{enumerate}
\end{multicols}
}


\excludecomment{solution} % without solutions

\theoremstyle{definition}
\newtheorem{question}{Вопрос}


\subsection[2017-2018]{\hyperref[sec:sol_final_exam_2017_2018]{2017-2018}}
\label{sec:final_exam_2017_2018}



\begin{question}
Если \(F_X(x)\) — функция распределения случайной величины \(X\), то
для любых \(a\) и \(b\)
\begin{answerlist}
  \item \(\P (X = a) = F_X(a)\)
  \item \(\P (X > a) = 1 - F_X(a)\)
  \item \(\P (X < a) = 1 - F_X(a)\)
  \item \(\P (X > a) = F_X(a)\)
  \item \(\P (a<X\le b) = F_X(a)-F_X(b)\)
\end{answerlist}
\end{question}

\begin{solution}
\begin{answerlist}
  \item Bad answer :(
  \item Good answer :)
  \item Bad answer :(
  \item Bad answer :(
  \item Bad answer :(
\end{answerlist}
\end{solution}



\begin{question}
Величины \(X_1\) и \(X_2\) независимы и одинаково распределены. Оценка
\(\hat\mu = 3aX_1 + 4a^2X_2\) математического ожидания \(\mu = \E(X_i)\)
будет несмещённой при \(a\) равном
\begin{answerlist}
  \item \(1.2\)
  \item \(3\)
  \item \(-1\)
  \item \(0\)
  \item \(-3\)
\end{answerlist}
\end{question}

\begin{solution}
\begin{answerlist}
  \item Не угадал
  \item Тоже ересь
  \item Отлично
  \item Неверно
  \item Неверно
\end{answerlist}
\end{solution}



\begin{question}
По случайной выборке из 100 наблюдений было оценено выборочное среднее
\(\bar{X}=20\)\\
и несмещенная оценка дисперсии \(\hat{\sigma}^2=25\). В рамках проверки
гипотезы \(H_0: \; \mu=15\) против альтернативной гипотезы
\(H_a: \; \mu>15\)\\
можно сделать следующее заключение
\begin{answerlist}
  \item Гипотеза \(H_0\) не отвергается на любом разумном уровне значимости
  \item Гипотеза \(H_0\) отвергается на уровне значимости 10\%, но не на уровне
значимости 5\%
  \item Гипотеза \(H_0\) отвергается на уровне значимости 5\%, но не на уровне
значимости 1\%
  \item Гипотеза \(H_0\) отвергается на любом разумном уровне значимости
  \item Гипотеза \(H_0\) отвергается на уровне значимости 20\%, но не на уровне
значимости 10\%
\end{answerlist}
\end{question}

\begin{solution}
\begin{answerlist}
  \item Неверно
  \item Неверно
  \item Неверно
  \item Отлично
  \item Неверно
\end{answerlist}
\end{solution}



\begin{question}
Ковариационная матрица вектора \(X=(X_1, X_2)\) имеет вид \[
\begin{pmatrix}
10 & 3 \\
3 & 8
\end{pmatrix}.
\] Дисперсия разности элементов вектора, \(\Var(X_1-X_2)\), равняется
\begin{answerlist}
  \item 18
  \item 6
  \item 2
  \item 15
  \item 12
\end{answerlist}
\end{question}

\begin{solution}
\begin{answerlist}
  \item Неверно
  \item Неверно
  \item Неверно
  \item Неверно
  \item Отлично
\end{answerlist}
\end{solution}



\begin{question}
Оценка \(\hat a\) называется эффективной оценкой параметра \(a\) в
классе оценок \(K\), если
\begin{answerlist}
  \item \(\E(\hat a^2) \geq \E(\tilde a ^2)\) для всех \(\tilde a \in K\)
  \item \(\E((\hat a - a)^2) \leq \E((\tilde a - a)^2)\) для всех
\(\tilde a \in K\)
  \item \(\E((\hat a - \tilde a)^2) \leq \E((\tilde a - a)^2)\) для всех
\(\tilde a \in K\)
  \item \(\E((\hat a - a)^2) \geq \E((\tilde a - a)^2)\) для всех
\(\tilde a \in K\)
  \item \(\E((\hat a - \tilde a)^2) \geq \E((\tilde a - a)^2)\) для всех
\(\tilde a \in K\)
\end{answerlist}
\end{question}

\begin{solution}
\begin{answerlist}
  \item Тоже ересь
  \item Отлично
  \item Не угадал
  \item Неверно
  \item Неверно
\end{answerlist}
\end{solution}



\begin{question}
Апостериорная функция плотности пропорциональна
\begin{answerlist}
  \item Отношению функции правдоподобия к априорной плотности
  \item Отношению априорной плотности к функции правдоподобия
  \item Разности априорной плотности и правдоподобия
  \item Произведению априорной плотности и правдоподобия
  \item Сумме априорной плотности и правдоподобия
\end{answerlist}
\end{question}

\begin{solution}
\[
f(\theta|data) = \frac{f(\theta)f(data|\theta)}{f(data)}\propto f(\theta) \cdot f(data|\theta)
\]
\begin{answerlist}
  \item Не угадал
  \item Неверно
  \item Тоже ересь
  \item Отлично
  \item Неверно
\end{answerlist}
\end{solution}



\begin{question}
Случайным образом выбирается семья с двумя детьми. Событие \(A\) --- в
семье старший ребенок --- мальчик, событие \(B\) --- в семье только один
из детей --- мальчик, событие \(C\) --- в семье хотя бы один из детей
--- мальчик.
\begin{answerlist}
  \item Любые два события из \(A\), \(B\), \(C\) --- зависимы
  \item События \(A\), \(B\), \(C\) --- независимы в совокупности
  \item \(A\) и \(B\) --- независимы, \(A\) и \(C\) --- зависимы, \(B\) и \(C\)
--- зависимы
  \item \(\P(A\cap B\cap C)=\P(A)\P(B)\P(C)\)
  \item События \(A\), \(B\), \(C\) --- независимы попарно, но зависимы в
совокупности
\end{answerlist}
\end{question}

\begin{solution}
\begin{answerlist}
  \item Неверно
  \item Неверно
  \item Отлично
  \item Неверно
  \item Неверно
\end{answerlist}
\end{solution}



\begin{question}
В алгоритме Метрополиса-Гастингса был предложен переход из точки
\(\theta^{(0)}=4\) в точку \(\theta^{(1)}_{prop}=5\). Априорное
распределение \(\theta\) равномерное. Известны значения функций
правдоподобия, \(f(data|\theta=4)=0.7\), \(f(data|\theta=5)=0.8\).
Вероятность одобрения перехода равна
\begin{answerlist}
  \item \(0.8/5\)
  \item \(1\)
  \item \(28/40\)
  \item \(4/5\)
  \item \(7/8\)
\end{answerlist}
\end{question}

\begin{solution}
\[
\alpha(x \to y) = \begin{cases}
1, \text{ если } f(y|data) > f(x|data) \\
f(y|data) / f(x|data), \text{ если } f(y|data) < f(x|data) \\
\end{cases}
\]
\begin{answerlist}
  \item Тоже ересь
  \item Отлично
  \item Неверно
  \item Неверно
  \item Не угадал
\end{answerlist}
\end{solution}



\begin{question}
Есть выборка \(X_1\), \(X_2\), \ldots, \(X_5\) и выборка \(Y_1\),
\(Y_2\), \(Y_3\), \(Y_4\). Исследовательница Ирина проводит тест суммы
рангов Вилкоксона. У выборки \(X_i\) сумма рангов равна 7. Сумма рангов
для выборки \(Y_j\) равна
\begin{answerlist}
  \item \(2\)
  \item \(1\)
  \item \(38\)
  \item \(43\)
  \item \(45\)
\end{answerlist}
\end{question}

\begin{solution}
\begin{answerlist}
  \item Неверно
  \item Тоже ересь
  \item Отлично
  \item Неверно
  \item Не угадал
\end{answerlist}
\end{solution}



\begin{question}
Требуется проверить гипотезу о равенстве математических ожиданий по двум
нормальным независимым выборкам размером 33 и 16 наблюдений. Истинные
дисперсии по обеим выборкам известны, совпадают и равны 196. Разница
выборочных средних равна 1. Тестовая статистика может быть равна
\begin{answerlist}
  \item \(-1/4\)
  \item \(-1/14\)
  \item \(-1/7\)
  \item \(-1/49\)
  \item \(-1/2\)
\end{answerlist}
\end{question}

\begin{solution}
\begin{answerlist}
  \item Bad answer :(
  \item Bad answer :(
  \item Bad answer :(
  \item Bad answer :(
  \item Good answer :)
\end{answerlist}
\end{solution}



\begin{question}
Размер выплаты страховой компанией является неотрицательной случайной
величиной с математическим ожиданием \(10\,000\) рублей. Согласно
неравенству Маркова, вероятность того, что очередная выплата превысит
\(50\,000\) рублей, ограничена сверху числом
\begin{answerlist}
  \item \(0.1359\)
  \item \(0.2\)
  \item \(0.5\)
  \item неравенство Маркова здесь неприменимо
  \item \(0.3413\)
\end{answerlist}
\end{question}

\begin{solution}
\begin{answerlist}
  \item Bad answer :(
  \item Good answer :)
  \item Bad answer :(
  \item Bad answer :(
  \item Bad answer :(
\end{answerlist}
\end{solution}



\begin{question}
Дана реализация выборки: 1, 2, 0. Выборочный начальный момент второго
порядка равен
\begin{answerlist}
  \item \(1\)
  \item \(5/3\)
  \item \(2.5\)
  \item \(1/3\)
  \item \(3\)
\end{answerlist}
\end{question}

\begin{solution}
\begin{answerlist}
  \item Неверно
  \item Ураа!!!
  \item Тоже ересь
  \item Не угадал
  \item Не туда!
\end{answerlist}
\end{solution}



\begin{question}
Величины \(X_1, \, \ldots, \, X_n\) — случайная выборка из
распределения Бернулли с параметром \(p \in (0;\,1)\). Оценка
максимального правдоподобия параметра \(p\) равна \(\bar X\). Оценка
максимального правдоподобия для \(\sqrt{p}\) равна
\begin{answerlist}
  \item \(\frac{\sqrt{\sum_{i=1}^{n}X_i}}{n}\)
  \item \(\sqrt{\sum_{i=1}^{n}X_i}\)
  \item \(\frac{1}{n}\sum_{i=1}^{n}X_i\)
  \item \(\frac{1}{n}\sum_{i=1}^{n}\sqrt{X_i}\)
  \item \(\sqrt{\frac{1}{n}\sum_{i=1}^{n}X_i}\)
\end{answerlist}
\end{question}

\begin{solution}
\begin{answerlist}
  \item Тоже ересь
  \item Не туда!
  \item Неверно
  \item Не угадал
  \item Ураа!!!
\end{answerlist}
\end{solution}



\begin{question}
Величины \(X_1, \, \ldots, \, X_n\) — случайная выборка из
распределения Бернулли с параметром \(p \in (0;\,1)\). Информация Фишера
о параметре \(p\), заключенная в одном наблюдении, равна
\begin{answerlist}
  \item \(\frac{1}{p(1-p)}\)
  \item \(\frac{1}{p}\)
  \item \(p(1-p)\)
  \item \(1 - p\)
  \item \(p\)
\end{answerlist}
\end{question}

\begin{solution}
\begin{answerlist}
  \item Ураа!!!
  \item Не угадал
  \item Неверно
  \item Тоже ересь
  \item Не туда!
\end{answerlist}
\end{solution}



\begin{question}
Плотность величины \(X\) имеет вид \(f(x)=2x\) при \(0<x<1\) и
\(f(x)=0\) при остальных \(x\). Условная плотность величины \(Y\)
задаётся формулой
\(f_{Y|X}(y|x)=\begin{cases} \frac{1}{x}, \text{ если } 0<y\le x; \\ 0, \text{ иначе } \end{cases}.\)
Совместная плотность величин \(X\) и \(Y\) равна
\begin{answerlist}
  \item \(f(x,y)=\begin{cases} 2, \text{ если } 0<y\le x<1; \\ 0, \text{ иначе} \end{cases}\)
  \item \(f(x,y)=\begin{cases} 2, \text{ если } 0<y<1, 0 < x<1; \\ 0, \text{ иначе} \end{cases}\)
  \item \(f(x,y)=\begin{cases} 1, \text{ если } 0<y\le x<1; \\ 0, \text{ иначе} \end{cases}\)
  \item \(f(x,y)=\begin{cases} 1/x, \text{ если } 0<y<1, 0 < x<1; \\ 0, \text{ иначе} \end{cases}\)
  \item \(f(x,y)=\begin{cases} 1/x, \text{ если } 0<y\le x<1; \\ 0, \text{ иначе} \end{cases}\)
\end{answerlist}
\end{question}

\begin{solution}
\begin{answerlist}
  \item Good answer :)
  \item Bad answer :(
  \item Bad answer :(
  \item Bad answer :(
  \item Bad answer :(
\end{answerlist}
\end{solution}



\begin{question}
Пусть \(t_n\) --- случайная величина, распределенная по Стьюденту с
\(n\) степенями свободы. Предел
\(\lim\limits_{n\to\infty}\P\left(t_{n}^2>1\right)\) равен
\begin{answerlist}
  \item \(0.317\)
  \item \(0.841\)
  \item \(0.253\)
  \item \(0.788\)
  \item \(0.102\)
\end{answerlist}
\end{question}

\begin{solution}
\begin{answerlist}
  \item Good answer :)
  \item Bad answer :(
  \item Bad answer :(
  \item Bad answer :(
  \item Bad answer :(
\end{answerlist}
\end{solution}



\begin{question}
Известно, что \(\E(X)=-1\), \(\E(Y)=2\), \(\Var(X)=4\), \(\Var(Y)=9\),
\(\Cov(X,Y)=-3\). Дисперсия \(\Var(2X-Y+1)\) равна
\begin{answerlist}
  \item 31
  \item \(37\)
  \item \(24\)
  \item \(-31\)
  \item 34
\end{answerlist}
\end{question}

\begin{solution}
\begin{answerlist}
  \item Bad answer :(
  \item Good answer :)
  \item Bad answer :(
  \item Bad answer :(
  \item Bad answer :(
\end{answerlist}
\end{solution}



\begin{question}
Дана реализация выборки: 3, 2, 5, 4, 2. Выборочная функция распределения
в точке \(x=2.5\) принимает значение
\begin{answerlist}
  \item \(0.25\)
  \item \(0.4\)
  \item \(0.5\)
  \item \(0.6\)
  \item \(0.2\)
\end{answerlist}
\end{question}

\begin{solution}
\begin{answerlist}
  \item Неверно
  \item Ураа!!!
  \item Тоже ересь
  \item Не туда!
  \item Не угадал
\end{answerlist}
\end{solution}



\begin{question}
Экзамен принимают два преподавателя: Злой и Добрый. Злой поставил оценки
2, 3, 10, 8, 1. А Добрый — оценки 6, 4, 7, 9. Значение статистики
критерия Вилкоксона о совпадении распределений оценок может быть равно
\begin{answerlist}
  \item \(25\)
  \item \(23\)
  \item \(26\)
  \item \(22\)
  \item \(24\)
\end{answerlist}
\end{question}

\begin{solution}
\begin{answerlist}
  \item Тоже ересь
  \item Ураа!!!
  \item Не туда!
  \item Не угадал
  \item Неверно
\end{answerlist}
\end{solution}



\begin{question}
Величина \(X\) принимает три значения \(1\), \(2\) и \(3\). По случайной
выборке из ста наблюдений оказалось, что \(1\) выпало 40 раз, \(2\) ---
40 раз и \(3\) --- 20 раз. Карл хочет проверить гипотезу о том, что все
три вероятности одинаковые. При верной \(H_0\) критерий Пирсона имеет
распределение
\begin{answerlist}
  \item \(\cN(0;1)\)
  \item \(\chi^2_3\)
  \item \(\chi^2_2\)
  \item \(\chi^2_1\)
  \item \(\chi^2_{99}\)
\end{answerlist}
\end{question}

\begin{solution}
\begin{answerlist}
  \item Bad answer :(
  \item Bad answer :(
  \item Good answer :)
  \item Bad answer :(
  \item Bad answer :(
\end{answerlist}
\end{solution}



\begin{question}
Случайные величины \(X_1\), \ldots, \(X_m\) — случайная выборка из
нормального распределения. Величины \(Y_1\), \ldots, \(Y_n\) —
независимая случайная выборка из нормального распределения. Для
построения доверительного интервала для отношения дисперсий можно
использовать статистику с распределением
\begin{answerlist}
  \item \(F_{m,n-2}\)
  \item \(F_{m-1, n-1}\)
  \item \(\chi^2_{m+n-2}\)
  \item \(F_{m+1,n+1}\)
  \item \(t_{m+n-2}\)
\end{answerlist}
\end{question}

\begin{solution}
\begin{answerlist}
  \item Тоже ересь
  \item Ураа!!!
  \item Не туда!
  \item Неверно
  \item Не угадал
\end{answerlist}
\end{solution}



\begin{question}
Математическое ожидание случайной величины \(X\) при условии \(Y=0\)
равно
\begin{answerlist}
  \item \(1/6\)
  \item \(0\)
  \item \(-1\)
  \item \(1\)
  \item \(1/3\)
\end{answerlist}
\end{question}

\begin{solution}
\begin{answerlist}
  \item Bad answer :(
  \item Bad answer :(
  \item Bad answer :(
  \item Good answer :)
  \item Bad answer :(
\end{answerlist}
\end{solution}



\begin{question}
Выборочная функция распределения, построенная по выборке объёма \(n\) из
равномерного распределения на отрезке \([0,2]\), в точке \(х=0.2\) при
\(n\) стремящимся к бесконечности стремится по вероятности к
\begin{answerlist}
  \item \(1\)
  \item \(0.5\)
  \item \(0.1\)
  \item \(0.2\)
  \item \(0\)
\end{answerlist}
\end{question}

\begin{solution}
\begin{answerlist}
  \item Не туда!
  \item Тоже ересь
  \item Ураа!!!
  \item Неверно
  \item Не угадал
\end{answerlist}
\end{solution}



\begin{question}
При построении 90\%-доверительного интервала для вероятности
используется выборка из 25 наблюдений. Выборочная доля составляет 0.6. В
симметричный доверительный интервал попадают значения
\begin{answerlist}
  \item \(0.6\), \(0.7\), \(0.85\)
  \item \(0.5\), \(0.6\), \(0.65\)
  \item \(0.35\), \(0.5\), \(0.65\)
  \item \(0.8\), \(0.9\), \(1.0\)
  \item \(0.7\), \(0.8\), \(0.9\)
\end{answerlist}
\end{question}

\begin{solution}
\begin{answerlist}
  \item Не угадал
  \item Отлично
  \item Неверно
  \item Тоже ересь
  \item Неверно
\end{answerlist}
\end{solution}



\begin{question}
Если величина \(\hat\theta\) имеет нормальное распределение
\(\cN(2;0.01^2)\), то, согласно дельта-методу, \(\hat\theta^2\) имеет
примерно нормальное распределение
\begin{answerlist}
  \item \(\cN(2;4\cdot 0.01^2)\)
  \item \(\cN(4;8\cdot 0.01^2)\)
  \item \(\cN(4;16\cdot 0.01^2)\)
  \item \(\cN(4;2\cdot 0.01^2)\)
  \item \(\cN(4;4\cdot 0.01^2)\)
\end{answerlist}
\end{question}

\begin{solution}
\begin{answerlist}
  \item Неверно
  \item Неверно
  \item Отлично
  \item Неверно
  \item Неверно
\end{answerlist}
\end{solution}



\begin{question}
Ковариация случайных величин \(X\) и \(Y\) равна:
\begin{answerlist}
  \item \(-2/3\)
  \item \(-1/3\)
  \item \(0\)
  \item \(2/3\)
  \item \(1/3\)
\end{answerlist}
\end{question}

\begin{solution}
\begin{answerlist}
  \item Bad answer :(
  \item Good answer :)
  \item Bad answer :(
  \item Bad answer :(
  \item Bad answer :(
\end{answerlist}
\end{solution}



\begin{question}
Сумма независимых абсолютно непрерывной и дискретной случайных величин
имеет распределение
\begin{answerlist}
  \item вырожденное
  \item абсолютно непрерывное
  \item дискретное
  \item сингулярное
  \item нормальное
\end{answerlist}
\end{question}

\begin{solution}
\begin{answerlist}
  \item Bad answer :(
  \item Good answer :)
  \item Bad answer :(
  \item Bad answer :(
  \item Bad answer :(
\end{answerlist}
\end{solution}



\begin{question}
Случайные величины \(X\) и \(Y\) имеют совместное нормальное
распределение, а \(x\in[1,2]\) --- константа. При любом \(x\) верно
неравенство
\begin{answerlist}
  \item \(\Corr(X,Y)\ne0\)
  \item \(\Var(Y|X=x)\geq \Var(Y)\)
  \item \(\E(Y|X=x)\geq \E(Y)\)
  \item \(\E(Y|X=x)\leq \E(Y)\)
  \item \(\Var(Y|X=x)\leq \Var(Y)\)
\end{answerlist}
\end{question}

\begin{solution}
\begin{answerlist}
  \item Bad answer :(
  \item Bad answer :(
  \item Bad answer :(
  \item Bad answer :(
  \item Good answer :)
\end{answerlist}
\end{solution}



\begin{question}
Величина \(X\) принимает три значения \(1\), \(2\) и \(3\). По случайной
выборке из ста наблюдений оказалось, что \(1\) выпало 40 раз, \(2\) ---
40 раз и \(3\) --- 20 раз. Андрей Николаевич хочет проверить гипотезу о
том, что все три вероятности одинаковые. Значение критерия согласия
Колмогорова равно
\begin{answerlist}
  \item \(3/5\)
  \item \(2/15\)
  \item \(3/4\)
  \item \(2/5\)
  \item \(1/4\)
\end{answerlist}
\end{question}

\begin{solution}
\begin{answerlist}
  \item Bad answer :(
  \item Good answer :)
  \item Bad answer :(
  \item Bad answer :(
  \item Bad answer :(
\end{answerlist}
\end{solution}



\begin{question}
Исследовательница Глафира считает, что любовь к энергетическим напиткам
и успешность сдачи экзамена по математической статистике должны быть
как-то связаны. Опросив 200 своих однокурсников, она получила следующие
результаты:

\begin{tabular}{ccc}
\toprule
 & пьёт энергетик & не пьёт энергетик \\
Успешно сдал & 20 & 120 \\
Завалил  & 40 & 20 \\
\bottomrule
\end{tabular}

Статистика \(\chi^2\) Пирсона для проверки независимости признаков с
округлением до целых равна
\begin{answerlist}
  \item \(70\)
  \item \(55\)
  \item \(65\)
  \item \(35\)
  \item \(45\)
\end{answerlist}
\end{question}

\begin{solution}
\begin{answerlist}
  \item Не туда!
  \item Ураа!!!
  \item Неверно
  \item Тоже ересь
  \item Не угадал
\end{answerlist}
\end{solution}




\section{Ответы}

% !TEX root = ../probability_hse_exams.tex
\thispagestyle{empty}
\section{Ответы к минимумам}

\subsection[Кр 1]{\hyperref[sec:minimum_kr_01]{Контрольная работа 1 — Задачный минимум}}
\label{sec:sol_minimum_kr_01}



\begin{multicols}{2}
\begin{enumerate}
	\item % 1
	\begin{enumerate}
		\item $0.25$
		\item $0.6$
		\item зависимы
	\end{enumerate}
	\item % 2
	$\frac{4}{10 \cdot 11 \cdot 12 \cdot 13}$
	% \item $\frac{4}{10 \cdot 11 \cdot 12 \cdot 13}$
	\item % 3
	\begin{enumerate}
		\item $0.5$
		\item $7/15$
	\end{enumerate}	
	\item % 4
	\begin{enumerate}
		\item $0.028$
		\item $\frac{5}{7}$
	\end{enumerate}	
	\item % 5
	\begin{enumerate}
		\item $0.5$
		\item $0.75$
		\item $0$
		\item $0.5$
		\item
		$F_{X}(x) = \begin{cases}
		0, & \text{при } x < -1 \\
		0.25 , & \text{при } -1 \le x < 0 \\
		0.75 , & \text{при } 0 \le x < 1 \\
		1, & \text{при }  x \geq 1
		\end{cases}$
		\begin{tikzpicture} 
		\draw[scale=1.5,->,thin,Gray] (0,0) -- (4,0) node[right,black] {$x$}; 
		\draw[scale=1.5,->,thin,Gray] (2,0) -- (2,1.2) node[above,black] {$y$}; 
		\draw[scale=1.5,domain=0:1,variable=\x,thick] plot ({\x},{0}); 
		\draw[scale=1.5,domain=1:2,variable=\x,thick] plot ({\x},{0.25}); 
		\draw[scale=1.5,domain=2:3,variable=\x,thick] plot ({\x},{0.75}); 
		\draw[scale=1.5,domain=3:4,variable=\x,thick] plot ({\x},{1}); 
		\end{tikzpicture}
		\item функция плотности не существует
	\end{enumerate}
	\item % 6
	\begin{enumerate}
		\item $0.5$
		\item $0$
		\item $0.5$
		\item $0.5$
		\item $0.5$
	\end{enumerate}
	\item % 7
	\begin{enumerate}
		\item $\left( \frac{1}{4} \right) ^4$
		\item $1 - \left( \frac{1}{4} \right) ^4$
		\item $0$
		\item $3$
		\item $0.75$
		\item $2$, $3$
	\end{enumerate}
	\item % 8
	\begin{enumerate}
		\item $e^{-100}$
		\item $1 - e^{-100}$
		\item $0$
		\item $100$
		\item $100$
		\item $99, 100$
	\end{enumerate}
	\item % 9
	\begin{enumerate}
		\item $1 - \frac{8^5}{9^5}$
		\item $\frac{8^5}{9^5}$
		\item $\frac{5^5}{9^5}$
		\item $\frac{4^5}{9^5}$
	\end{enumerate}
	\item % 10
	\begin{enumerate}
		\item $1 - e^{-3}$
		\item $e^{-6}$
	\end{enumerate}
	\item % 11
	\begin{enumerate}
		\item $2$
		\item $0.25$
		\item $0.75$
		\item
		$F_{X}(x) = \begin{cases}
		0, & \text{при } x < 0 \\
		x^2 , & \text{при } 0 \le x < 1 \\
		1, & \text{при }  x \geq 1
		\end{cases}$
	\end{enumerate}
	\item % 12
	\begin{enumerate}
		\item $2$
		\item $\frac{2}{3}$
		\item $0.5$
		\item $\frac{1}{18}$
		\item $0.8$
	\end{enumerate}
\end{enumerate}
\end{multicols}


\subsection[Кр 2]{\hyperref[sec:minimum_kr_02]{Контрольная работа 2 — Задачный минимум}}
\label{sec:sol_minimum_kr_02}


\begin{multicols}{2}
\begin{enumerate}

\item
\begin{enumerate}
\item   $0.5 $
\item   $0.3$
\item   $0.2$
\item   нет
\item   $0.3$
\item
\begin{tabular}{lrr}
\toprule
$x$ & $-1$  & $1$   \\ \midrule
$\P(X=x)$ & $0.5$ & $0.5$ \\ \bottomrule
\end{tabular}
\item  $F_{X}(x) = \begin{cases}
0, & \text{при } x < -1 \\
0.5 , & \text{при } x \in [-1;1) \\
1, & \text{при }  x \geq 1
\end{cases}$
\end{enumerate}
\item
\begin{enumerate}
\item   $0.5$
\item   $0.4$
\item   $0.2$
\item   да
\item   $0.6$
\item
\begin{tabular}{lrrr}
\toprule
$y$ & $-1$  & $0$   & $1$   \\ \midrule
$\P(Y=y)$ & $0.4$ & $0.2$ & $0.4$ \\ \bottomrule
\end{tabular}
\item   $F_{Y}(y) = \begin{cases}
0, & \text{при } y < -1 \\
0.4 , & \text{при } y \in [-1;0) \\
0.6, & \text{при }  y \in [0;1)\\
1, & \text{при } y \geq 1
\end{cases}$
\end{enumerate}

\item
\begin{enumerate}
\item $0$
\item $1$
\item $1$
\item $0$
\item $0.6$
\item $0.6$
\item $0$
\item $0$
\item $0$
\item да, являются некоррелированными, но нельзя утверждать, что являются независимыми
\end{enumerate}

\item
\begin{enumerate}
\item $0$
\item $1$
\item $1$
\item $0$
\item $0.8$
\item $0.8$
\item $0$
\item $0$
\item $0$
\item да, являются некоррелированными, но нельзя утверждать, что являются независимыми
\end{enumerate}

\item
\begin{enumerate}
\item $0.25$
\item $0.2$
\item Обозначим $A = \{X = -1\}$

\begin{tabular}{lrrr}
\toprule
$y$           & $-1$  & $0$   & $1$   \\ \midrule
$\P(Y=y|A)$   & $0.4$ & $0.2$ & $0.4$ \\ \bottomrule
\end{tabular}
\item $0$
\item $0.8 $
\end{enumerate}
\item
\begin{enumerate}
\item $0.5$
\item $0.2$
\item Обозначим $A = \{X = 1\}$
\begin{tabular}{lrrr}
\toprule
$y$ & $-1$  & $0$   & $1$   \\ \midrule
$\P(Y=y|A)$             & $0.4$ & $0.2$ & $0.4$ \\ \bottomrule
\end{tabular}
\item $0$
\item $0.8$
\end{enumerate}

\item
\begin{enumerate}
\item $0 $
\item $36$
\item $9 $
\item $60 $
\item $-4$
\item $\frac{-1}{3\sqrt{5}}$
\item $\begin{pmatrix}
 3 & -1 \\
-1 & 4
\end{pmatrix}$
\end{enumerate}

\item
\begin{enumerate}
\item $-4$
\item $8 $
\item $1 $
\item $10$
\item $-6$
\item$ \frac{-1}{\sqrt{5}}$

\item $\begin{pmatrix}
 1 & 1 \\
 1 & 2
\end{pmatrix}$
\end{enumerate}
\item
\begin{enumerate}
\item $0.3413$
\item $0.0228$
\item $0.1915$
\end{enumerate}

\item
\begin{enumerate}
\item $0.6826$
\item $0.0228$
\item $0.1574$
\end{enumerate}

\item $0.4332$
\item $0.8185$
\item $0.4514$
\item $0.5328$
\item $\approx 0.8185$
\item $\approx 0.9115$
\item $\approx 0.6422$
\item $\approx 0.9606$

\item
\begin{enumerate}
\item $0.125$
\item $0.5$
\item $f_{X}(x) = \begin{cases} x+\frac{1}{2}, & \text{при } x \in [0;1] \\ 0 , & \text{при } x \not\in [0;1] \end{cases}$
\item $f_{Y}(y) = \begin{cases} y+\frac{1}{2}, & \text{при } y \in [0;1] \\ 0 , & \text{при } y \not\in [0;1] \end{cases}$
\item нет
\end{enumerate}

\item
\begin{enumerate}
\item $\frac{1}{16}$
\item $\frac{1}{2}$
\item $f_{X}(x) =
\begin{cases} 2x, & \text{при } x \in [0;1] \\
0 , & \text{при } x \not\in [0;1]
\end{cases}$
\item $f_{Y}(y) =
\begin{cases} 2y, & \text{при } y \in [0;1] \\
0 , & \text{при } y \not\in [0;1]
\end{cases}$
\item да
\end{enumerate}
\item
\begin{enumerate}
\item $\frac{7}{12}$
\item $\frac{7}{12}$
\item $\frac{1}{3}$
\item $-\frac{1}{144}$
\item $-\frac{1}{11}$
\end{enumerate}

\item
\begin{enumerate}
\item $\frac{2}{3}$
\item $\frac{2}{3}$
\item $\frac{4}{9}$
\item $0$
\item $0$
\end{enumerate}

\item
\begin{enumerate}
\item $f_{Y}(y) =
\begin{cases} y+\frac{1}{2}, & \text{при } y \in [0;1] \\
0 , & \text{при } y \not\in [0;1]
\end{cases}$
\item $f_{X|Y}(x|\frac{1}{2}) =
\begin{cases} x+\frac{1}{2}, & \text{при } x \in [0;1] \\
0 , & \text{при } x \not\in [0;1]
\end{cases}$
\item $\frac{7}{12}$
\item $\frac{11}{144}$
\end{enumerate}

\item
\begin{enumerate}
\item $f_{Y}(y) =
\begin{cases} 2y, & \text{при } y \in [0;1] \\
0 , & \text{при } y \not\in [0;1]
\end{cases}$
\item $f_{X|Y}(x|\frac{1}{2}) =
\begin{cases} 2x, & \text{при } x \in [0;1] \\
0 , & \text{при } x \not\in [0;1]
\end{cases}$
\item $\frac{2}{3}$
\item $\frac{1}{18}$
\end{enumerate}
\end{enumerate}
\end{multicols}



\subsection[Кр 3]{\hyperref[sec:minimum_kr_03]{Контрольная работа 3 — Задачный минимум}}
\label{sec:sol_minimum_kr_03}


\begin{multicols}{2}
\begin{enumerate}
\item
\begin{enumerate}
\item $\approx 0.15$
\item $U \sim \cN(101,29)$, $f(u) = \frac{1}{\sqrt{2\pi\cdot 29}}e^{-\frac{1}{2}\frac{(u-101)^2}{29}}$
\item $\approx 0.02$
\end{enumerate}
\item
\begin{enumerate}
\item $71.14$
\item $f(y|x=170) = \frac{1}{\sqrt{2\pi\cdot20}}e^{-\frac{1}{2}\frac{(y-71.14)^2}{20}}$
\item $\approx 0$
\end{enumerate}
\item
\begin{enumerate}
\item $0.25$
\item $0.6875$
\item $0.91(6)$
\item $0.75$
\item $-0.28125$
\end{enumerate}
\item
\begin{enumerate}
\item $-1, 0, 1, 1$
\item $-1$
\item $1$
\item $f(x) = \begin{cases}
0, & x < -1 \\
0.25, & -1 \leq x < 0 \\
0.5, & 0 \leq x < 1 \\
1, & x \geq 1
\end{cases}$
\end{enumerate}
\item
\begin{enumerate}
\item $\theta$
\item да
\end{enumerate}

\item
\begin{enumerate}
\item нет, оценка смещена
\item $c = 2$
\end{enumerate}
\item
\begin{enumerate}
\item все оценки несмещенные
\item $\hat{p}_3$ наиболее эффективная
\end{enumerate}
\item да
\item да
\item $\hat{\theta}_{MM} = \sqrt{\frac{\sum_{i=1}^n(X_i-\overline{X})^2\cdot20}{n}}$

\item $\hat{\theta}_{MM} = \frac{1}{5}\left(6 - \frac{1}{n}\sum_{i=1}^n X_i^2 \right)$, $\hat{\theta}_{MM} = 0.68$
\item $\hat{\theta}_{ML} = \frac{\sum_{i=1}^n x_i^2}{n}$
\item $\hat{p}_{ML} = \frac{\sum_{i=1}^n x_i}{n}$
\item да
\item $n_1 \approx 260$, $n_2 \approx 232$, $n_3 \approx 658$

\end{enumerate}
\end{multicols}






\subsection[Кр 4]{\hyperref[sec:minimum_kr_04]{Контрольная работа 4 — Задачный минимум}}
\label{sec:sol_minimum_kr_04}


\begin{enumerate}
\item $\left[-1.6 - 1.65 \cdot \frac{2}{\sqrt{3}}; -1.6 + 1.65 \cdot \frac{2}{\sqrt{3}} \right]$
\item $\left[-1.6 - 2.92 \cdot \sqrt{\frac{18.33}{3}}; -1.6 + 2.92 \cdot \sqrt{\frac{18.33}{3}} \right]$
\item $\left[\frac{17.43 \cdot 2}{4.61}; \frac{17.43 \cdot 2}{0.21} \right]$
\item $\left[-1.6 - (-2.6) - 1.96 \cdot \sqrt{\frac{2}{3} + \frac{1}{2}}; -1.6 - (-2.6) + 1.96 \cdot \sqrt{\frac{2}{3} + \frac{1}{2}} \right]$
\item $\left[1.04 - (-0.37) - 3.18 \cdot \sqrt{3.02} \sqrt{\frac{1}{3} + \frac{1}{2}}; 1.04 - (-0.37) + 3.18 \cdot \sqrt{3.02} \sqrt{\frac{1}{3} + \frac{1}{2}} \right]$
\item $\left[0.45 - 1.96 \cdot \sqrt{\frac{0.45 \cdot 0.55}{100}}; 0.45 + 1.96 \cdot \sqrt{\frac{0.45 \cdot 0.55}{100}} \right]$
\item $\left[0.6 - 0.4 - 1.96  \cdot \sqrt{\frac{0.6\cdot0.4}{100} + \frac{0.4 \cdot 0.6}{200}}; 0.6 - 0.4 + 1.96 \cdot \sqrt{\frac{0.6\cdot0.4}{100} + \frac{0.4 \cdot 0.6}{200}} \right]$
\item $\left[2.5 - 1.96 \cdot \sqrt{\frac{1}{40}}; 2.5 + 1.96 \cdot \sqrt{\frac{1}{40}} \right]$
\item $\left[\frac{1}{0.52} - 1.96 \cdot \sqrt{\frac{1}{100 \cdot 0.52^2}}; \frac{1}{0.52} + 1.96 \cdot \sqrt{\frac{1}{100 \cdot 0.52^2}} \right]$
\item
\begin{enumerate}
\item $\approx 0.02$
\item $\approx 0.02$
\item $\approx 0.98$
\end{enumerate}
\item $0.2$
\item $z_{obs} \approx -1.39 $, $z_{crit} = 1.28$, нет оснований отвергать $H_0$.
\item $t_{obs} \approx -0.65$, $t_{crit} = 1.89$, нет оснований отвергать $H_0$.
\item $z_{obs} \approx 0.93$, $z_{crit} = -1.65$, нет оснований отвергать $H_0$.
\item $t_{obs} \approx 0.89$, $t_{crit} = -2.35$, нет оснований отвергать $H_0$.
\item $F_{obs} \approx 95.37$, $F_{crit} = 199.5$, нет оснований отвергать $H_0$.
\item $z_{obs} \approx 2.04$, $z_{crit} = 1.65$, основная гипотеза отвергается.
\item $z_{obs} \approx 4.16$, $z_{crit} = 1.96$, основная гипотеза отвергается.
\item $\gamma_{obs} \approx 0.26$, $\gamma_{crit} = 5.99$, нет оснований отвергать $H_0$.
\item $\gamma_{obs} \approx 139.4$, $\gamma_{crit} = 3.84$, основная гипотеза отвергается.
\item $LR_{obs} \approx 5.5$, $LR_{crit} = 3.84$, основная гипотеза отвергается.
\end{enumerate}

\thispagestyle{empty}
\section{Решения контрольной номер 1}

\subsection[2017-2018]{\hyperref[sec:kr_01_2017_2018]{2017-2018}}
\label{sec:sol_kr_01_2017_2018}

\begin{enumerate}
\item
\begin{enumerate}
\item События называются независимыми, если  $ \P(A \cap B) = \P(A) \cdot \P(B)$
\item Запасёмся всеми нужными вероятностями:

$\P(A) = \frac{1}{2}$

$\P(B) = \frac{1}{3}$

$\P(C) = \frac{1}{2}$

$\P(A \cap C) = \frac{1}{3} $ — выпадет чётое число больше трёх

$\P(A \cap B)  = \frac{1}{6}$ — выпадет чётное число, кратное трём

$\P(A \cap C) = \frac{1}{6}$ — выпадет число, большее трёх и кратное трём

Теперь можно проверять независимость:

$\P(A \cap C) \neq \P(A) \cdot \P(C) \Rightarrow$  не являются независимыми

$ \P(A \cap B) = \P(A) \cdot \P(B) \Rightarrow$ являются независимыми

$ \P(B \cap C) = \P(B) \cdot \P(C) \Rightarrow$ являются независимыми

\end{enumerate}
\item
\begin{enumerate}
\item Количество возможных вариантов ТМ: $ C_{10}^2 $,  количество возможных
вариантов ЗМ: $ C_{24}^2 $. Количество их возможных сочетаний: $ C_{10}^2 \cdot C_{24}^2$,
где $ C_n^k = \frac{n!}{k!(n-k)!}$.
\item По классическому определению вероятностей, предполагая исходы равновероятными,
искомая вероятность равна $\frac{C_{16}^2}{C_{24}^2}$.
\item По тому же принципу:
\[
\frac{C_k^2}{C_{10}^2} = \frac{1}{15} \Rightarrow \frac{\frac{k!}{2!(k-2)!}}{\frac{10!}{2! \cdot 8!}} = \frac{1}{15} \Rightarrow \frac{(k-1)k}{2}\frac{ 2}{9 \cdot 10} = \frac{1}{15}
\]
Получаем квадратное уравнение вида $ k^2 - k - 6 = 0 $ с корнями $-2$ и $3$.
Так как $k$ не может быть отрицательным, ответ $3$.
\end{enumerate}
\item
\begin{enumerate}
\item Если эксперт отдаёт предпочтение Fit, то это можно интерпретировать как
«успех» в схеме Бернулли. Так как $\xi$ - количество успехов,
$ k \in [0;4]$, $p = \frac{1}{3} $, то
\[
\P(\xi = k) = C_n^k(p)^k(1-p)^{n-k}
\]

Большинство означает, что либо три, либо четыре эксперта выбрали Fit.
\[
\P(\xi = 3) = C_4^3\left(\frac{1}{3}\right)^3 \left(\frac{2}{3}\right)^{1} = \frac{8}{81}
\]
\[
\P(\xi = 4) = C_4^4\left(\frac{1}{3}\right)^4 \left(\frac{2}{3}\right)^{0} = \frac{1}{81}
\]
\[
\P( \xi > 2) =  \frac{9}{81}
\]
\item Аналогично:

\[ \P(\xi = 0) = C_4^0\left(\frac{1}{3}\right)^0 \left(\frac{2}{3}\right)^{4} = \frac{16}{81}\]

\[ \P(\xi = 1) = C_4^1\left(\frac{1}{3}\right)^1 \left(\frac{2}{3}\right)^{3} = \frac{32}{81}\]

\[ \P(\xi = 2) = C_4^2\left(\frac{1}{3}\right)^2 \left(\frac{2}{3}\right)^{2} = \frac{24}{81}\]

\begin{figure}[h!]
    \noindent\centering{
    \includegraphics[width=80mm]{images/kr1_2017_3.png}
    }
    \caption{Функция распределения}
    \label{cdf_kr2017}
\end{figure}

\item Все вероятности посчитаны, видим, что наибольшая достигается при $\xi=1$.
\item $\E(X) = np = \frac{4}{3} $, $ \Var(X) = npq = \frac{8}{9}$
\end{enumerate}
\item
\begin{enumerate}
\item Так как указано, что цена сметаны распределена равномерно на отерзке
$[250, 1000]$, максимальное значение цены — $1000$, это и есть необходимая сумма.
\item Вспомним, что функция распределения $F(x) = \P(X \leq x)$, нужно найти
такой $x$, что $ \P(X \leq x)=0.9$:
\[
0.9 = 1 - \exp({-x^{2}}) \Rightarrow \exp(-x^{2}) = 0.1 \Rightarrow -x^2 = \ln(0.1)  \Rightarrow x=  \sqrt{-\ln(0.1)}
\]
\item Взяв производную от функции распределения списка без сметаны, получим функцию
плотности:
\[
f_X(x) =
\begin{cases}
2x\exp(-x^2) & x \ge 0 \\
0 & \text{иначе}
\end{cases}
\]
Найдём математическое ожидание:
\[
\int_{0}^{+\infty}2x^2\exp({-x^2}) dx = -x \exp({-x^2})\big|_0^{+\infty} + \int_{0}^{+\infty}\exp({-x^2}) dx = \frac{\sqrt{\pi}}{2}
\]
\item Математическое ожидание суммы случайных величин равно сумме математических
ожиданий случайных влечин, если они существуют. Математическое ожидание от цены
сметаны равно: $ \frac{1000 + 250}{2} = 625$.
Математическое ожидание списка без сметаны было найдено в предыдущем пункте, его
осталось перевести в рубли. Получаем ответ: $ 625 + \frac{\sqrt{\pi}}{2} \cdot 1000 $.
\item Так как обе величины имеют абсолютно непрерывные распределения, вероятность
попасть в конкретную точку равна нулю.
\end{enumerate}
\item
\begin{enumerate}
\item $\P(\text{детектор показал ложь и подозреваемый лжёт}) = 0.9 \cdot 0.1 + 0.1 \cdot 0.95 = 0.185$
\item $\P(\text{невиновен}|\text{детектор показал ложь}) = \frac{0.9\cdot0.1}{0.185} = \frac{90}{185}$
\item $\P(\text{эксперт точно выявит преступника}) = (0.9)^9 \cdot 0.95$
\item $\P(\text{эксперт ошибочно выявит преступника}) = 9 \cdot 0.1 \cdot 0.9^8\cdot 0.05$
\end{enumerate}
\end{enumerate}


\subsection[2016-2017]{\hyperref[sec:kr_01_2016_2017]{2016-2017}}
\label{sec:sol_kr_01_2016_2017}

\begin{enumerate}
\item
\begin{enumerate}
\item Возможны четыре равновероятные ситуации:
\[
\P(\text{ММ}) = \P(\text{МД}) = \P(\text{ДМ}) = \P(\text{ДД}) = 1/4
\]

Посчитаем условную вероятность:
\[
\P(B \mid A) = \frac{\P(B \cap A)}{\P(A)} = \frac{\P(\text{МД, ДМ})}{\P(\text{ДМ, МД, ДД})} = \frac{2/4}{3/4} = \frac{2}{3}
\]

\item События $A$ и $B$ называются независимыми, если $\P(A \cap B) = \P(A) \cdot \P(B)$

В нашем случае: $\P(A \cap B) = \P (\text{МД, ДМ}) = 2/4$,
$\P(A) \cdot \P (B) = 3/4 \cdot 3/4$.

Следовательно, $\P(A \cap B) \neq \P(A) \cdot \P (B)$,
значит, события $A$ и $B$ не являются независимыми.
\end{enumerate}

\item Пусть событие $A_i$ означает, что $i$-ый узел системы дал сбой,
а событие $B_N$, что вся система дала сбой.

В условии сказано, что $\P(A_i) = 10^{-6}$,
а найти нужно такое максимальное $N \in \mathbb{N}$, при котором

\[
\P(B_N) \leq \frac{1}{10^2}
\]

\begin{align*}
\P(B_N) &= \P\left(\cup_{i=1}^n A_i\right) = 1 - \P (\left(\cup_{i=1}^n A_i\right)^c) \\
&\stackrel{\text{ф-ла де Моргана}}{=} 1 - \P \left(\cup_{i=1}^N A_i^c\right) \stackrel{A_1, \ldots, A_N \text{– независ.}}{=} 1 - \P(A_1^c) \cdot \ldots \cdot \P(A_N^c) \\
&= 1 - \left(1-\frac{1}{10^6}\right)^N
\end{align*}
Чтобы найти такое максимальное $N \in \mathbb{N}$, надо решить следующее неравенство
\begin{align*}
& 1 - \left(1-10^{-6}\right)^N \leq 10^{-2} \\
& 1 - 10^{-2} \leq \left(1-10^{-6}\right)^N \\
& \ln\left(1 - 10^{-2}\right) \leq N \ln \left(1 - 10^{-6}\right) \\
& N \leq \frac{\ln\left(1 - 10^{-2}\right)}{ \ln \left(1 - 10^{-6}\right)} \approx 10050.33
\end{align*}
Значит, максимальное $N$ равно $10050$.

\item Введём обозначения для событий.
Пусть $A$ означает, что человек имеет заболевание лёгких,
а $B$, что человек работал в шахте.

В условии сказано, что $\P(B \mid A) = 0.22$, $\P(B \mid A^c) = 0.14$, $\P(A) = 0.04$.
\begin{enumerate}
\item Нужно найти
\[
\P(A \mid B) = \frac{\P(A\cap B)}{\P (B)} = \frac{\P(B|A)\P(A)}{\P(B)}
\]
Для этого с помощью формулы полной вероятности посчитаем
\[
\P (B) = \P (B \mid A) \P(A) + \P (B \mid A^c) \P (A^c) = 0.22 \cdot 0.04 + 0.14 \cdot 0.96 = 0.1432
\]
Осталось подставить значения:
\[
\P(A \mid B) = \frac{0.22 \cdot 0.04}{0.1432} \approx 0.0615
\]

\item Все необходимые значения для второго пункта у нас есть,
осталось применить формулу условной вероятности:
\begin{align*}
\P  (A \mid B^c) &=  \frac{\P(A\cap B^c)}{\P (B^c)} =  \frac{\P (B^c \cap A)}{\P(A)} \cdot \frac{\P(A)}{\P (B^c)} = \P (B^c \mid A) \cdot \frac{\P(A)}{\P (B^c)} \\
&= (1-\P (B \mid A)) \cdot \frac{\P(A)}{1-\P (B)} = (1-0.22) \cdot \frac{0.04}{1-0.1432} \approx 0.0364
\end{align*}
\end{enumerate}
\item Введём индикатор события «Петя дал верный ответ на $i$-ый вопрос»:
\[
X_i =
\begin{cases}
1, & \text{если на } i \text{-ый вопрос теста Петя дал верный ответ} \\
0, & \text{иначе}
\end{cases}
\]

Заметим, что $X_i \sim Be\left(p = 1/5 \right)$, $X_1, \ldots, X_{17}$ – независимы,
$X = X_1 + \ldots + X_{17}$ – общее число верных ответов,
$X \sim Bin\left(n=17, p=1/5\right)$.

\begin{enumerate}
\item Наибольшее вероятное число правильных ответов $m_0$ может быть нвйдено по формуле:
\begin{enumerate}
\item[1)] если число $(n\cdot p - q)$ – не целое, где $q:=1-p$, то
\[
m_0 = [np-q] +1,
\]
\item[2)] если число  $(n\cdot p - q)$ – целое, то наиболее вероятных значений $m_0$ два:
\[
m_0' = np-q \text{ и } m_0'' = np-q+1
\]
\end{enumerate}
Итак, поскольку $np-q = 17\cdot\frac{1}{5} - \frac{4}{5} = 2.6$ – не целое, наиболее вероятное число верных ответов $m_0$ может быть найдено по формуле из пункта (1):
\[
m_0 = [np-q] +1 = [2.6] + 1 = 3
\]
\item \[\E(X) = np = 17 \cdot \frac{1}{5}=3.4\]

\[\Var(X) = npq = 17 \cdot \frac{1}{5} \cdot \frac{4}{5} = 2.72\]

\item
\begin{align*}
\P (\text{у Пети «отлично»}) &= \P (X\geq 15) = \P (X = 15) + \P (X= 16) + \P (X = 17) \\
& = C^{15}_{17} \cdot \left(\frac{1}{5}\right)^{15} \cdot \left(\frac{4}{5}\right)^2 + C^{16}_{17} \cdot \left(\frac{1}{5}\right)^{16} \cdot \left(\frac{4}{5}\right)^1 + C^{17}_{17} \cdot \left(\frac{1}{5}\right)^{17} \cdot \left(\frac{4}{5}\right)^0 \\
&= 136 \cdot \frac{16}{5^{17}} + 17 \cdot \frac{4}{5^{17}} + \frac{1}{5^{17}} \approx 2.94 \cdot 10^{-9}
\end{align*}
\item Рассмотрим первый вопрос теста. Петя может выбрать первый ответ с вероятностью $1/5$, и Вася
может выбрать первый ответ с вероятностью $1/5$. Тогда они оба выберут одинаковый ответ с вероятностью $1/25$.
Вариантов ответа в каждом вопросе $5$, значит, вероятность совпадения ответа в одном вопросе равна $1/5$.
Всего вопросов 17, тогда получаем
\[
\P(\text{все ответы Пети и Васи совпадают}) = \left(\frac{1}{5}\right)^{17}
\]

\end{enumerate}
\item Введём случайную велчину $\eta$, которая означает число потенциальных покупателей, с которыми контактировал продавец оборудования. По условию задачи, $\eta$ имеет таблицу распеределения:
\begin{center}
\begin{tabular}{ccc}
\toprule
$\eta$ & $ 1 $ & $2$ \\
$\P_{\eta}$ & $1/3$ & $2/3$ \\ \bottomrule
\end{tabular}
\end{center}
Случайная величина $\xi$ может принимать значения $0, 50000$ и $100000$
\begin{enumerate}

\item Найдём $\P (\xi = 0 )$. По формуле полной вероятности, имеем:
\begin{align*}
\P (\xi = 0) &= \P (\xi = 0 \mid \eta = 1 ) \cdot \P ( \eta = 1 ) + \P (\xi = 0 \mid \eta = 2 )  \cdot \P ( \eta = 2 ) \\
&= 0.9 \cdot \frac{1}{3} + 0.9\cdot0.9 \cdot \frac{2}{3} = 0.84
\end{align*}

\item Найдём $\P (\xi = 50000 )$ и $\P (\xi = 100000 )$ :
\begin{align*}
\P (\xi = 50000 ) &= \P (\xi = 50000 \mid \eta = 1 ) \cdot \P ( \eta = 1  ) +  \P (\xi = 50000 \mid \eta = 2 ) \cdot  \P ( \eta = 2 ) \\
&= 0.1 \cdot \frac{1}{3} + 2 \cdot 0.1 \cdot 0.9 \cdot \frac{2}{3} = 0.15(3)
\end{align*}
\begin{align*}
\P (\xi = 100000 ) &=  \P (\xi = 100000 \mid \eta = 1 ) \cdot \P ( \eta = 1 ) +  \P (\xi = 100000 \mid \eta = 2 ) \cdot  \P ( \eta = 2  ) \\
&= 0 \cdot \frac{1}{3} + 0.1\cdot 0.1  \cdot \frac{2}{3} = 0.00(6)
\end{align*}
Таблица распределения случайной величина $\xi$ имеет вид:

\begin{center}
\begin{tabular}{cccc}
\toprule
$\xi$ & $ 0 $ & $5000$ & $100000$ \\
$\P_{\xi}$ & $0.84$ & $0.15(3)$ & $0.00(6)$ \\ \bottomrule
\end{tabular}
\end{center}

Тогда функция распределения случайной величины $\xi$ имеет вид:
\[
F_{\xi} (X) =
\begin{cases}
0 & \text{при } x<0 \\
0.84 & \text{при } 0 \leq x < 50000 \\
0.84 + 0.15(3) & \text{при } 50000
\leq x < 100000 \\
1 & \text{при } x > 100000
\end{cases}
\]
Опр.: $F_{\xi} = \P (\xi \leq x ), x \in \mathbb{R}$
\item \[
\E (X) = 0 \cdot 0.84 + 50000 \cdot 0.15(3) + 100000 \cdot 0.00(6) = 8333.(3)
\]
\begin{align*}
\Var(X) &= (0 - 8333.(3))^2 \cdot 0.84 + (50000-8333.(3))^2 \cdot 0.15(3) \\
&+ (100000 - 8333.(3))^2 \cdot 0.00(6) = 380555555.(5)
\end{align*}
\end{enumerate}
\item
\begin{enumerate}
\item $ f_{\xi} (x)=
\begin{cases}
\frac{1}{b} & \text{при } x \in [0, b] \\
0 & \text{при } x \notin [0, b]
\end{cases}
$
\item  Известно, что если $\xi \sim U[a, b]$, то $\E (\xi) = \frac{a+b}{2}$. Стало быть, из уравнения $\E (\xi) = 1$ получаем $\frac{b}{2} = 1$,  то есть $b=2$.
\item Известно, что если $\xi \sim U[a, b]$, то $\Var (\xi) = \frac{(b-a)^2}{12}$. Значит, $\Var (\xi) = \frac{2^2}{12} = \frac{1}{3}$
\item Воспользуемся формулой $\P (\xi \in B ) = \int_B f_{\xi} (x) dx$. Имеем:
\[
\P (\xi > 1 ) = \P (\xi \in (1, + \infty) ) = \int_{1}^{+ \infty} f_{\xi} (x) dx = \int_{1}^{2} \frac{1}{2} dx = \frac{1}{2}
\]
\item Требуется найти такое минимальное число $q_{0.25}$, что $\int_{-\infty}^{q_{0.25}} f_{\xi} (x) dx = 0.25$. Итак:
\[
\int_{-\infty}^{q_{0.25}} f_{\xi} (x) dx = 0.25 \Leftrightarrow \int_{-\infty}^{q_{0.25}} \frac{1}{2} dx = 0.25 \Leftrightarrow \frac{1/2}{q_{0.25}} = 0.25 \Leftrightarrow
\]
\[
q_{0.25} = 2 \cdot 0.25 = 0.5
\]
\item
\begin{align*}
\E [ (\xi - \E(\xi))^{2017} ] &= \int_{-\infty}^{+\infty} (x- \E(\xi) )^{2017} \cdot f_{\xi} (x) dx = \int_{-\infty}^{+\infty} (x-1)^{2017} f_{\xi} (x) dx \\
&= \int_{0}^{2} (x-1)^{2017} \cdot \frac{1}{2} dx = \frac{(x-1)^{2018}}{2018} \cdot \frac{1}{2} \bigg\rvert_{x=0}^{x=2} =0
\end{align*}
\item $F_{\xi} (x) =
\begin{cases}
0 & \text{при } x < 0 \\
\frac{x}{2} & \text{при } 0 \leq x \leq 2 \\
1 & \text{при } x > 2
\end{cases}
$
\item Согласно условиям задачи, время до прихода 1-го поезда есть $\xi$; время до прихода 2-го поезда равно $\xi + b$; время до прихода 3-го (заветного) поезда есть $\xi + 2b$. Таким образом, Марья Ивановна в среднем ожидает «своего» поезда $\E (\xi + 2b) = 1 + 2b = 1 + 2 \cdot 2 = 5 $ минут. При этом $\Var (\xi + 2b) = \Var (\xi) = 1/3$
\item[к)] Пусть $\tau$ – наименьший номер поезда без «подозрительных лиц». По условию задачи, таблица распределения случайной величины $\tau$ имеет вид:

\begin{center}
\begin{tabular}{cccccc}
\toprule
$\tau$ & $ 1 $ & $2$ & $3$ & $4$ & \ldots \\
$\P_{\tau}$ & $1/4$ & $3/4\cdot1/4$ & $(3/4)^2 \cdot 1/4$ & $(3/4)^3 \cdot 1/4$ & \ldots\\ \bottomrule
\end{tabular}
\end{center}

То есть случайная величина $\tau$ имеет геометрическое распределение с параметром $p=1/4$ $(\tau \sim G(p=1/4))$.

Несложно сообразить, что время ожидания Глафирой Петровной «своего» поезда составляет: $\eta := \xi + b(\tau- 1)$. Стало быть, $\E (\eta) = \E (\xi) + b \cdot (\E(\tau)-1)  = 1 + 2 \cdot (4-1) = 7$ минут.

Здесь мы воспользовались тем фактом, что если $\eta \sim G(p)$, то $\E (\eta) = 1/p$
\item[и)] Найдём теперь вероятность $\P (\eta \geq 5 )$. Для нахождения искомой вероятности воспользуемся формулой полной вероятности:
\[
	\P (\eta \geq 5 ) = \P(\eta \geq 5, \tau < 3) +\P(\eta \geq 5, \tau = 3)+\P(\eta \geq 5, \tau > 3)
\]

Если Глафира уехала на первом или втором поезде,
то ждать больше 5 минут она не могла, то есть $\P(\eta \geq 5, \tau <3)=0$.

Если Глафира уехала на третьем поезде, то чтобы ждать больше пяти минут,
ей нужно ждать первый поезд больше минуты,
то есть $\P(\eta \geq 5, \tau = 3)=0.5 \P(\tau = 3)$.

Если Глафира уехала на четвертом поезде или позже, то она точно ждала больше 5 минут,
$\P(\eta \geq 5, \tau >3)=\P(\tau>3)$.

\[
\P(\eta \geq 5) = 0.5\P(\tau = 3) + \P(\tau > 3) = 0.5 \cdot (3/4)^2 \cdot (1/4) + (3/4)^3 = 63 / 128
\]

\end{enumerate}
\item Пусть $\xi$ — случайная величина, обозначающая число остановок лифта. Предствим её в виде суммы $\xi = \xi_2 + \ldots + \xi_{10}$, где $\xi_i$ — индикатор
того, что лифт остановился на $i$-ом этаже, то есть
\[
\xi_i = \begin{cases}
1 & \text{если лифт остановился} \\
0 & \text{иначе}
\end{cases}
\quad \forall i = 2, \ldots, 10
\]
Найдём соответсвующие вероятности:
\[
\P(\xi_i = 0) = \left(\frac{8}{9}\right)^9
\]
\[
\P(\xi_i = 1) = 1 - \P(\xi = 0) = 1 - \left(\frac{8}{9}\right)^9
\]
Тогда $\E(\xi_i) = \P(\xi_i = 0) \cdot 0 + \P(\xi_i = 1) \cdot 1 = 1 - \left(\frac{8}{9}\right)^9$, и в итоге получаем:
\[
\E(\xi) = 9 \cdot \E(\xi_i) = 9 \cdot \left(1 - \left(\frac{8}{9}\right)^9\right)
\]
\end{enumerate}

\subsection[2015-2016]{\hyperref[sec:kr_01_2015_2016]{2015-2016}}
\label{sec:sol_kr_01_2015_2016}

\begin{enumerate}
\item
\begin{enumerate}
\item[$\alpha$)] Найдём вероятности каждого события:
$\P(A) = 1/2$, $\P(B) = 1/2$, $\P(C) = 1/2$.

Проверим попарную независимость:
\begin{itemize}
\item $\P(A \cap B) = 1/4$, $\P(A) \cdot \P(B) = 1/2 \cdot 1/2 = 1/4$
\item $\P(A \cap C) = 1/4$, $\P(A) \cdot \P(C) = 1/2 \cdot 1/2 = 1/4$
\item $\P(B \cap C) = 1/4$, $\P(B) \cdot \P(C) = 1/2 \cdot 1/2 = 1/4$
\end{itemize}
Значит, события попарно независимы.
\item[$\beta$)] События $A_1, A_2, A_3$ называются независимыми в совокупности,
если $\P(A_1 \cap A_2 \cap A_3) = \P(A_1) \cdot \P(A_2) \cdot \P(A_3)$.

В нашем случае: $\P(A \cap B \cap C) = 0$, $ \P(A) \cdot \P(B) \cdot \P(C) = (1/2)^3$,
следовательно, события не являются независимыми в совокупности.
\end{enumerate}

\item
\begin{enumerate}
\item[$\alpha$)] Воспользуемся формулой полной вероятности:
\begin{align*}
\P(\text{выпала «6»}) &= \P(\text{выпала «6»} \mid \text{взят белый кубик}) \cdot \P(\text{взят белый кубик}) \\
&+ \P(\text{выпала «6»} \mid \text{взят красный кубик}) \cdot \P(\text{взят красный кубик}) \\
&= \frac{1}{6} \cdot \frac{1}{2} + \frac{1}{3} \cdot \frac{1}{2} = \frac{1}{4}
\end{align*}
\item[$\beta$)] Воспользуемся формулой условной вероятности и результатом предыдущего пункта:
\begin{align*}
\P(\text{взят красный кубик} \mid \text{выпала «6»}) &= \frac{\P(\text{взят красный кубик} \cap \text{выпала «6»})}{\P(\text{выпала «6»})}  \\
&= \frac{\frac{1}{2}\cdot \frac{1}{3}}{\frac{1}{4}} = \frac{2}{3}
\end{align*}
\end{enumerate}

\item
\begin{enumerate}
\item[$\alpha$)] Совместное распределение имеет вид:
\begin{center}
\begin{tabular}{@{}lllllll@{}}
\toprule
$\eta$ $\backslash$ $\xi$ & $1$                            & $2$                            & $3$                            & $4$                            & $5$                            & $6$                            \\ \midrule
$1$           & $\frac{2}{15}\cdot\frac{1}{6}$ & $\frac{2}{15}\cdot\frac{1}{6}\mbox{*}$  & $\frac{2}{15}\cdot\frac{1}{6}\mbox{*}$   & $\frac{2}{15}\cdot\frac{1}{6} \mbox{*}$   & $\frac{2}{15}\cdot\frac{1}{6} \mbox{*}$   & $\frac{1}{3}\cdot\frac{1}{6} \mbox{*}$   \\
$2$           & $\frac{2}{15}\cdot\frac{1}{6}$ & $\frac{2}{15}\cdot\frac{1}{6}$ & $\frac{2}{15}\cdot\frac{1}{6}\mbox{*}$   & $\frac{2}{15}\cdot\frac{1}{6}\mbox{*}$   & $\frac{2}{15}\cdot\frac{1}{6}\mbox{*}$   & $\frac{1}{3}\cdot\frac{1}{6} \mbox{*}$   \\
$3$           & $\frac{2}{15}\cdot\frac{1}{6}$ & $\frac{2}{15}\cdot\frac{1}{6}$ & $\frac{2}{15}\cdot\frac{1}{6}$ & $\frac{2}{15}\cdot\frac{1}{6} \mbox{*}$   & $\frac{2}{15}\cdot\frac{1}{6} \mbox{*}$   & $\frac{1}{3}\cdot\frac{1}{6} \mbox{*}$   \\
$4$           & $\frac{2}{15}\cdot\frac{1}{6}$ & $\frac{2}{15}\cdot\frac{1}{6}$ & $\frac{2}{15}\cdot\frac{1}{6}$ & $\frac{2}{15}\cdot\frac{1}{6}$ & $\frac{2}{15}\cdot\frac{1}{6} \mbox{*}$ & $\frac{1}{3}\cdot\frac{1}{6} \mbox{*}$   \\
$5$           & $\frac{2}{15}\cdot\frac{1}{6}$ & $\frac{2}{15}\cdot\frac{1}{6}$ & $\frac{2}{15}\cdot\frac{1}{6}$ & $\frac{2}{15}\cdot\frac{1}{6}$ & $\frac{2}{15}\cdot\frac{1}{6}$ & $\frac{1}{3}\cdot\frac{1}{6} \mbox{*}$   \\
$6$           & $\frac{2}{15}\cdot\frac{1}{6}$ & $\frac{2}{15}\cdot\frac{1}{6}$ & $\frac{2}{15}\cdot\frac{1}{6}$ & $\frac{2}{15}\cdot\frac{1}{6}$ & $\frac{2}{15}\cdot\frac{1}{6}$ & $\frac{1}{3}\cdot\frac{1}{6}$ \\ \bottomrule
\end{tabular}
\end{center}
\item[$\beta$)] $\P(\text{выиграет белый кубик}) = (6 + 5 + 4 + 3 + 2) \cdot \frac{2}{15}\cdot\frac{1}{6} + 1 \cdot \frac{1}{3}\cdot\frac{1}{6} = \frac{1}{2}$.

Значит, Пете безразлично, какой кубик брать.
\item[$\gamma)$] $F_{\zeta}(x) = \P(\zeta \leq x)$

Выпишем таблицу распределения случайной величины $\zeta$:

\begin{center}
\begin{tabular}{@{}lcccccc@{}}
\toprule
$\zeta$     & $1$                              & $2$                                      & $3$                                      & $4$                                      & $5$                                      & $6$                                                                              \\ \midrule
$\P(\cdot)$ & $\frac{2}{15} \cdot \frac{1}{6}$ & $\frac{2}{15} \cdot \frac{1}{6} \cdot 3$ & $\frac{2}{15} \cdot \frac{1}{6} \cdot 5$ & $\frac{2}{15} \cdot \frac{1}{6} \cdot 7$ & $\frac{2}{15} \cdot \frac{1}{6} \cdot 9$ & $\frac{1}{3} \cdot \frac{1}{6} \cdot 6 + \frac{2}{15} \cdot \frac{1}{6} \cdot 5$ \\ \bottomrule
\end{tabular}
\end{center}

Тогда функция распределения имеет вид:
\[
F_{\zeta}(x) =
\begin{cases}
0 & x \leq 1 \\
\frac{1}{45} & 1 < x \leq 2 \\
\frac{4}{45} & 2 < x \leq 3 \\
\frac{9}{45} & 3 < x \leq 4 \\
\frac{16}{45} & 4 < x \leq 5 \\
\frac{25}{45} & 5 < x \leq 6 \\
1 & x > 6
\end{cases}
\]
\item[$\delta$)] $\E(\zeta) = \frac{2}{15} \cdot \frac{1}{6} \cdot 1 + \frac{2}{15} \cdot \frac{1}{6} \cdot 3 \cdot 2 + \frac{2}{15} \cdot \frac{1}{6} \cdot 5 \cdot 3 + \frac{2}{15} \cdot \frac{1}{6} \cdot 7 \cdot 4 + \frac{2}{15} \cdot \frac{1}{6} \cdot 9 \cdot 5 + \frac{1}{3} \cdot \frac{1}{6} \cdot 6 + \frac{2}{15} \cdot \frac{1}{6} \cdot 6 = \frac{43}{9} \approx 4.8 $
\end{enumerate}
\item Пусть $x$ — вероятность того, что мужчина честно любит петь в душе.

Распишем по формуле полной вероятности вероятность получить ответ «да»:
\begin{align*}
P(\text{ответ «Да»}) &= 1 \cdot \P(\text{выпала «6»}) + x \cdot(\P(\text{выпала «2»}) + \P(\text{выпала «3»}) \\
&+ \P(\text{выпала «4»}) + \P(\text{выпала «5»})) = 1 \cdot \frac{1}{6} + x \cdot \frac{4}{6} \Rightarrow x = \frac{3}{4}
\end{align*}
Тогда истинный процент «певцов» составляет $75 \%$

\item Предположим, что ваше имя — Студент (7 букв), а фамилия — Идеальный (9 букв).
\begin{enumerate}
\item[$\alpha$)] $\P(\text{напишет фаимлию правильно}) = (0.9)^9$
\item[$\beta$)] $\P(\text{ровно 2 ошибки в имени}) = C_{7}^2 \cdot 0.1^2 \cdot 0.9^5$
\item[$\gamma$)] Наиболее вероятное число ошибок — 1
\item[$\delta$)] $\P(\text{допустит хотя бы одну ошибку}) = 1 - \P(\text{не допустит ни одной ошибки}) = 1 - (0.9)^{16}$
\end{enumerate}

\item
\begin{enumerate}
\item[$\alpha$)] Из условия $\int_{0}^{1} (cy^2 + y) dy = 1$ получаем, что $c=3/2$.
\item[$\beta$)]
$F_{Y} (y) =
\begin{cases}
1 & y > 1 \\
\frac{y^3 + y^2}{2} & 0 < y \leq 1 \\
0 & y < 0
\end{cases} $
\item[$\gamma$)] $\P(Y < 0.5) = \int_{0}^{0.5} \left(\frac{3}{2} y^2 + y   \right) dy = \frac{3}{16}$
\item[$\delta$)] $F_{Y} (y) = 0.5 \Rightarrow y \approx 0.75 $
\item[$\epsilon$)] $\P(Y > 0.5 \mid Y \geq 0.25) = \frac{\P(Y > 0.5)}{\P(Y \geq 0.25)} = \frac{1 - \frac{3}{16}}{\int_{0.25}^{1} \left(\frac{3}{2} y^2 + y   \right) dy} = \frac{104}{123}$
\end{enumerate}

\item
\begin{enumerate}
\item[$\alpha$)] $\P(\text{кисточка окажется на слоне}) = \frac{1}{1.5} = \frac{2}{3}$
\item[$\beta$)] $f_{\xi, \eta}(x, y) = \frac{1}{1.5}$
\item[$\gamma$)] $f_{\xi} (x) = \int_{0}^{1} \frac{1}{1.5} dy = 1.5$

$f_{\eta}(y) = \int_{0}^{1.5} \frac{1}{1.5} dx = 1$
\item[$\delta$)] Да, поскольку $ f_{\xi} (x) \cdot f_{\eta}(y) = f_{\xi, \eta}(x, y)$
\item[$\epsilon$)] $f_{\xi+\eta} (t) = \int_{-\infty}^{+\infty} f_{\xi}(u) f_{\eta}(t-u) du $
\end{enumerate}
\end{enumerate}



\subsection[2014-2015]{\hyperref[sec:kr_01_2014_2015]{2014-2015}}
\label{sec:sol_kr_01_2014_2015}



\begin{enumerate}
\item Внимательно читайте примечание! Всего 6 возможных ситуаций, только 1 — благоприятная.
Требуемая вероятность равна $1/6$.

\item
Два события $A$ и $B$ независимы, если: $\P(AB) = \P(A) \P(B)$.

Проверим, независимы ли события $A = \{ \xi < 1/2 \} $ и  $B = \{ \eta < 1/2 \} $:

$\P(AB)$ ищется как отношение площади квадрата с вершинами в $(0,\,0)$, $(0,\,1/2)$,
$(1/2,\,1/2)$, $(1/2,\,0)$ к площади данного треугольника, то есть:
\[
\P(AB) = \frac{(1/2)^2}{1/2}= \frac{1}{2}
\]

$\P(A)$ ищется как отношение площади трапеции с вершинами в $(0,\,0)$, $(0,\,1)$,
$(1/2,\,1/2)$, $(1/2,\,0)$ к площади данного треугольника, то есть:
\[
\P(A) = \frac{(1/2)\cdot (3/2) \cdot (1/2)}{1/2}= \frac{3}{4}
\]

$\P(B)$ ищется как отношение площади трапеции с вершинами в $(0,\,0)$, $(1,\,0)$,
$(1/2,\,1/2)$, $(0,\,1/2)$ к площади данного треугольника, то есть:
\[
\P(B) = \frac{(1/2)\cdot (3/2) \cdot (1/2)}{1/2}= \frac{3}{4}
\]

\[
\P(A)\cdot \P(B) = \frac{3}{4} \cdot  \frac{3}{4} = \frac{9}{16} \ne \frac{1}{2} = \P(AB)
\]

Получается, события $A$ и $B$ зависимы.

\item
Пусть событие $A$ = \{Цель была поражена первым самолетом\},
событие $B$ = \{Цель была поражена только одним самолетом\}.
Тогда событие $AB$ = \{Первый самолет поразил цель, второй и третий — промахнулись\}.
По формуле условной вероятности:

\[\P(A|B) = \frac{\P(AB)}{\P(B)} = \frac{0.6 \cdot 0.6 \cdot 0.7}{0.6\cdot 0.6 \cdot 0.7 + 0.4 \cdot 0.4 \cdot 0.7 + 0.4 \cdot 0.6 \cdot 0.3} = \frac{0.252}{0.436} \approx 0.578\]

\item
Удобно рассуждать следующим образом: предположим, что каждая опечатка наугад
(с равными вероятностями и независимо от других опечаток) выбирает, на какую
страницу ей попасть.

\begin{enumerate}
\item Пусть $X$ — число опечаток на 13 странице. \[\P(X \geqslant 2) = 1 - \P(X=0) - \P(X=1) \]
$\P(X=0) = \left( \frac{499}{500} \right)^{400}$ — каждая из 400 опечаток не должна попасть на 13 страницу.\\
$\P(X=1) = 400\cdot\frac{1}{500}\cdot\left( \frac{499}{500} \right)^{399}$ — ровно одна опечатка (а есть 400 вариантов) должна попасть на 13 страницу, а остальные — мимо. Соответственно:
\[
\P(X \geqslant 2) = 1 - \left( \frac{499}{500} \right)^{400} - 400\cdot\frac{1}{500}\cdot\left( \frac{499}{500} \right)^{399} \approx 0.19
\]
Это если считать в явном виде. А если пользоваться приближением Пуассона:
\[
p(k) = \P(X = k) = \frac{\lambda^k}{k!}e^{-\lambda}
\]
неплохо бы вспомнить, что параметр $\lambda$ это математическое ожидание $X$, поэтому расчеты здесь пока оставим до лучших времен.

\item Пусть $X$ — число опечаток на 13 странице. Введем случайную величину
\[X_i =
\begin{cases}
1, & \text{если } i\text{-ая опечатка попала на 13 страницу}\\
0, & \text{если нет}
\end{cases}
\]
Тогда $X = \sum\limits_{i=1}^{400}X_i$. Рассмотрим отдельно $X_i$:

\begin{center}
\begin{tabular}{@{}ccc@{}}
\toprule
$x$         & $1$             & $0$               \\ \midrule
$\P(X=x)$ & $\frac{1}{500}$ & $\frac{499}{500}$ \\ \bottomrule
\end{tabular}
\end{center}

Так как $i$-ая опечатка наугад выбирает одну страницу из 500 и это должна быть именно 13.

Тогда:
\begin{align*}
\E(X_i) &= \frac{1}{500} = \E(X^2_i)  \\
\Var(X_i) &= \E(X^2_i) - (\E(X_i))^2 = \frac{1}{500} - \left(\frac{1}{500}\right)^2 = \frac{499}{500^2}
\end{align*}
Значит
\begin{align*}
\E(X) &= \E\left(\sum\limits_{i=1}^{400}X_i\right) = \sum\limits_{i=1}^{400}\E(X_i)  = \frac{400}{500} = 0.8 \\
\Var(X) &= \Var\left(\sum\limits_{i=1}^{400}X_i\right) = \sum\limits_{i=1}^{400}\Var(X_i) = 400\cdot\frac{499}{500^2} = 0.8\cdot\frac{499}{500}
\end{align*}

Теперь мы знаем, что $\lambda = \E(X) = 0.8$ поэтому можем вернуться к пункту (а):
\[
\P(X \geqslant 2) = 1 - \P(X=0) - \P(X=1)  = 1 - \frac{0.8^0}{0!}e^{-0.8} - \frac{0.8^1}{1!}e^{-0.8} \approx 0.19
\]

Осталось найти наиболее вероятное число опечаток на 13 странице:
\[
\P(X=k) = \frac{0.8^k}{k!}e^{-0.8} \rightarrow \max \limits_k
\]
Очевидно, что эта функция убывает по $k$, ведь с ростом $k$:\\
 $k!$ растет, а $0.8^k$ убывает. Значит наиболее вероятное число ошибок — $X = 0$

\item \href{https://en.wikipedia.org/wiki/Triskaidekaphobia}{Ох уж эти предрассудки!}
13-я страница точно такая же как и все остальные, ведь везде в решении можно просто заменить номер 13 на любой другой и ничего не изменится.

\end{enumerate}

\item
Пусть событие $A$ означает, что медицинский тест показал наличие заболевания.
Событие $B$ — заболевание на самом деле есть.

Перепишем условие задачи:

Чувствительность теста $=\P(A |B)$

Специфичность теста $=\P(A^c | B^c)$

Прогностическая сила теста $=\P(B | A)$

$\P(B) = 0.01 \Rightarrow \P(B^c) = 0.99 $

По условию, чувствительность теста равна $0.9$, тогда из формулы условной вероятности:
\[
\P(A | B) = \frac{\P(A \cap B)}{\P(B)} \Rightarrow
\P(A \cap B) = 0.9 \cdot 0.01 = 0.009
\]

При этом очевидно, что:
\[
\P(B) = \P(A \cap B) +  \P(A^c \cap B) \Rightarrow
\P(A^c \cap B) = 0.01 - 0.009 = 0.001
\]

По условию специфичность теста равна 0.95, тогда из формулы условной вероятности:
\[
\P(A^c | B^c) = \frac{\P(A^c \cap B^c)}{\P(B^c)} \Rightarrow
\P(A^c \cap B^c) =0.95 \cdot 0.99 = 0.9405
\]

При этом очевидно, что:
\[
\P(B^c) = \P(A \cap B^c) + \P(A^c \cap B^c) \Rightarrow
\P(A \cap B^c) = 0.99 - 0.9405 = 0.0495
\]

Теперь мы готовы отвечать на заданные вопросы:

\begin{enumerate}
\item
\[
\P(A) = \P(A \cap B^c) + \P(A \cap B) = 0.009+0.0495 = 0.0585
\]

\item Прогностическая сила теста:

\[
\P(B | A) = \frac{\P(A \cap B)}{\P(A) } = \frac{0.009}{0.0585} \approx 0.154
\]

Для того, чтобы повысить прогностическую силу теста, необходимо понизить
$\P(A \cap B^c) $, а для этого необходимо повысить специфичность теста.
\end{enumerate}

\item
\begin{enumerate}
\item
Должно выполняться условие нормировки:

\begin{align*}
& \int \limits_{-a}^0 1.5(x+a)^2 dx + \int \limits_0^a 1.5(x- a)^2  dx = 1   \\
& \left. 0.5(x+a)^3 \right|_{-a}^0 + \left. 0.5(x- a)^3 \right|_0^a  = 1  \\
& 0.5a^3 + 0.5a^3 = 1 \Rightarrow a = 1
\end{align*}

Теперь легко понять, как выглядит функция распределения (смотри определение функции распределения):

\[
F(x) = \begin{cases}
0, & x < 1 \\
0.5 (x+1)^3, & -1 \leqslant x <0 \\
1 + 0.5 (x-1)^3, & 0 \leqslant x < 1 \\
1, & x \geqslant 1
\end{cases}
\]

И с её помощью всё посчитать:
\begin{align*}
& P\left(X \in \left[\frac{1}{2}, 2 \right]  \right) = F(2) - F\left(\frac{1}{2} \right) =
1 - 1 +0.5^4 = 0.5^4
\end{align*}
\begin{align*}
\E(X) &= \int \limits_{-1}^0 x \cdot 1.5 (x + 1)^2 dx +  \int \limits_0^1 x \cdot 1.5 (x - 1)^2 dx \\
& = 1.5 \int \limits_{-1}^0\left( x^3 + 2x^2 + x\right) dx + 1.5 \int \limits_0^1\left( x^3 -2x^2 + x\right) dx \\
& =  \frac{3}{8} x^4 |_{-1}^0 + x^3 |_{-1}^0 + \frac{3}{4} x^2|_{-1}^0+    \frac{3}{8} x^4 |_0^1   - x^3 |_0^1 + \frac{3}{4} x^2|_0^1  = - \frac{3}{8}  + 1- \frac{3}{4} + \frac{3}{8} - 1 +\frac{3}{4} = 0
\end{align*}
А можно было заметить, что функция плотности — четная функция, поэтому сразу $\E(X) = 0$

Вычислим $\E\left(X^2\right)$:

\begin{align*}
\E(X^2) &= \int \limits_{-1}^0 x^2 \cdot 1.5 (x + 1)^2 dx +  \int \limits_0^1 x^2 \cdot 1.5 (x - 1)^2 dx \\
&= 1.5 \int \limits_{-1}^0\left( x^4 + 2x^3 + x^2\right) dx + 1.5 \int \limits_0^1\left( x^4 -2x^3 + x^2\right) dx \\
& =  \frac{3}{10} x^5 |_{-1}^0 + \frac{3}{4} x^4|_{-1}^0 + \frac{1}{2} x^3 |_{-1}^0 +  \frac{3}{10} x^5 |_0^1 - \frac{3}{4} x^4|_0^1  + \frac{1}{2} x^3 |_0^1 =  \frac{1}{10} \\
\end{align*}
Тогда:
\begin{align*}
&\Var(X) = \E(X^2) - (\E(X))^2 = 0.1
\end{align*}

\item Верим, что график $F(x)$, выписанной выше, вы построить можете :)
\end{enumerate}
\item
Пусть $A = \{\text{«Лекция полезна»}\}$, $B = \{\text{«Лекция интересна»}\}$. Заметим, что лекции вообще независимы друг от друга.

\begin{enumerate}
\item Пусть $X_A$ — число полезных лекций, прослушанных Васей,  $X_B$ — число интересных лекций, прослушанных Васей. Введем случайную величину:
\[X_i =
\begin{cases}
1 & \text{если } i\text{-ая лекция была полезна}\\
0 & \text{если нет}
\end{cases}
\]

Тогда $X_A = \sum\limits_{i=1}^{30}X_i$. Рассмотрим отдельно $X_i$:

\begin{center}
\begin{tabular}{@{}ccc@{}}
\toprule
$x$         & $1$             & $0$               \\
$\P(X=x)$ & $0.9$ & $0.1$ \\ \bottomrule
\end{tabular}
\end{center}

Вероятность $0.9$ дана. Тогда:
\begin{align*}
\E(X_i) &= 0.9 = \E(X^2_i) \Rightarrow \\
\Var(X_i) &= \E(X^2_i) - (\E(X_i))^2 = 0.9 - 0.9^2 = 0.09
\end{align*}

Значит
\begin{align*}
\E(X_A) &= \E\left(\sum\limits_{i=1}^{30}X_i\right) = \sum\limits_{i=1}^{30}\E(X_i)  = 0.9\cdot30 = 27 \\
\Var(X_A) &= \Var\left(\sum\limits_{i=1}^{30}X_i\right) = \sum\limits_{i=1}^{30}\Var(X_i) = 0.09\cdot30 = 2.7
\end{align*}

Аналогично для числа интересных лекций можем получить:
\begin{align*}
\E(X_B) &= 0.7\cdot 30 = 21 \\
\Var(X_A) &= 0.21\cdot 30 = 6.3
\end{align*}


\item Так как интересность и полезность — независимые свойства лекций, то:
\[
\P(A^c \cap B^c) = \P(A^c)\cdot \P(B^c) = 0.3\cdot0.1 = 0.03,
\]
где $A^c$ значит «не $A$».
В свою очередь:
\[
\P(A\cup B) = \P(A\cap B^c) + \P(B\cap A^c) + \P(A\cap B) = 1 - \P(A^c)\cdot \P(B^c) = 0.97,
\]
где $(A\cup B)$ значит «$A$ или $B$», а $(A\cap B)$ — «$A$ и $B$».
Аналогично, путем введения бинарной случайной величины можем получить:
\begin{align*}
& \E(X_{A^c \cap B^c}) = 0.03 \cdot  30 = 0.9 \\
& \E(X_{A\cup B}) = 0.97\cdot30 = 29.1
\end{align*}
\end{enumerate}

\item
Дано: $\E(X) = 1$, $\E(Y) = 2$, $\E(X^2) = 5$, $\E(Y^2) = 8$, $\E(XY) = -1$.

Будем использовать только свойства математического ожидания, ковариации и дисперсии, и ничего больше. Ни-че-го.

\begin{itemize}
\item  $\E(2X + Y - 4) = 2\E(X) + \E(Y) + \E(-4) = 2 + 2 - 4 = 0 $
\item $\Var(X) = \E(X^2) - (\E(X))^2 = 5 - 1 = 4 $
\item $\Var(Y) = \E(Y^2) - (\E(Y))^2 = 8 - 4 = 4 $
\item $\Cov(X, Y) = \E(XY) - \E(X)\E(Y) = -1 - 2 = -3$
\item $\Corr(X, Y) = \frac{\Cov(X, Y)}{\sqrt{\Var(X)}\sqrt{\Var(Y)}} = -\frac{3}{2\cdot 2} = -0.75$
\item $\Var(X-Y-1) = \Var(X) + \Var(Y) - 2\Cov(X, Y) = 4+4 -2(-3) = 14$
\item $\Var(X+Y+1) = \Var(X) + \Var(Y) + 2\Cov(X, Y) = 4+4+2(-3) =2 $
\item \begin{align*}
\Cov(X-Y-1, X+Y+1)&=\E((X-Y)(X+Y))-\E(X-Y)\E(X+Y) \\
&= \E\left(X^2-Y^2\right) - (\E(X)-\E(Y))(\E(X) + \E(Y)) \\
&= \E\left(X^2\right) - \E\left(Y^2\right) - \left((\E(X))^2 -(\E(Y))^2\right)\\
&= \Var(X)-\Var(Y) = 0
\end{align*}
 \item $\Cov(X-Y-1, X+Y+1)=0 \Rightarrow \Corr(X-Y-1, X+Y+1) = 0 $
\end{itemize}

\item Найдём частные распределения $Y$ и $Y^2$:
\begin{center}
\begin{tabular}{cccc}
\toprule
 & $X=1$ & $X=2$ & $\sum$ \\ \midrule
$Y=-1$ & $0.1$ & $0.2$ & $0.3$ \\
$Y=0$ & $0.2$ & $0.3$ & $0.5$ \\
$Y=1$ & $0$ & $0.2$ & $0.2$ \\
$\sum$ & $0.3$ & $0.7$ & \\ \bottomrule
\end{tabular}
\end{center}

\begin{center}
\begin{tabular}{@{}cccc@{}}
\toprule
$y$         & $-1$             & $0$      & $1$         \\
$\P(Y=y)$ & $0.3$ & $0.5$  & $0.2$\\ \bottomrule
\end{tabular}
\end{center}

Так как $Y^2$ может принимать только значения 0 или 1:

\begin{center}
\begin{tabular}{@{}ccc@{}}
\toprule
$y^2$         & $0$             & $1$               \\
$\P(Y^2 = y^2)$ & $0.5$ & $0.5$ \\ \bottomrule
\end{tabular}
\end{center}
А ковариация:
\begin{align*}
\Cov(X, Y) &= \E(XY) - \E(X)\E(Y) =
((-1)\cdot 1\cdot0.1 + (-1)\cdot2 \cdot 0.2 + 1\cdot2\cdot 0.2) \\
&- (0.3\cdot1 + 0.7 \cdot 2)\cdot(0.3\cdot(-1) + 0.1\cdot 0.2) = 0.07
\end{align*}

Так как $\Cov(X, Y) \ne 0$ — величины зависимы

\item Бонусная задача

Предположим, что правильный ответ 0.25. Но это невозможно, потому что вариантов ответа 0.25 — два (1 и 4), значит ответ 0.5 тоже был бы правильный. Предположим, что правильный 0.5. Тогда 0.25 тоже правильный — таких вариантов два из четырех, значит вероятность попасть в 0.25, выбрав ответ наугад, равна 0.5. Ответ 0.6, очевидно, неверен, потому что вероятность попасть в него равна 0.25. \\
\textbf{Правильный ответ:} 0
\end{enumerate}



\subsection[2013-2014]{\hyperref[sec:kr_01_2013_2014]{2013-2014}}
\label{sec:sol_kr_01_2013_2014}

\begin{enumerate}
\item Введём обозначения:
\begin{itemize}
\item $\P(\text{В} | \text{A}^{c} \cap \text{М}^{c}) = 0.18$ — Вася пришёл, а девушки — нет
\item $\P(\text{В} | \text{A} \cap \text{М}) = 0.9$ — пришли и Вася, и девушки
\item $\P(\text{В} | \text{A}^{c} \cap \text{М}) = 0.54$ — Вася пришёл, если пришла только Маша
\item $\P(\text{В} | \text{A} \cap \text{М}^{c}) = 0.36$ — Вася пришёл, если пришла только Алёна
\item $\P(\text{М}) = 0.4$ — Маша пришла на лекцию
\item $\P(\text{А}) = 0.6$ — Алёна пришла на лекцию
\end{itemize}
\begin{enumerate}
\item Используя формулы Байеса и полной вероятности, получим:
\[
\P(\text{A} | \text{В} ) = \frac{\P(\text{A} \cap \text{В})}{\P(\text{В})}
\]
В числителе:
\begin{align*}
\P(\text{В} | \text{A}) \cdot \P(\text{А}) &= P(\text{В} | \text{A} \cap \text{М}) \cdot \P(\text{А}) \cdot \P(\text{М}) + \P(\text{В} | \text{A} \cap \text{М}^{c}) \cdot \P(\text{А}) = \cdot \P(\text{М}^{c}) \\
&= 0.9 \cdot 0.4 \cdot 0.6 + 0.36 \cdot 0.6 \cdot 0.6 = 0.3456
\end{align*}
А в знаменателе:
\begin{align*}
\P(\text{В}) &=
\P(\text{В} | \text{A}^{c} \cap \text{М}^{c}) \cdot \P(\text{A}^{c} \cap \text{М}^{c})+\P(\text{В} | \text{A} \cap \text{М}) \cdot \P(\text{A} \cap \text{М}) + \P(\text{В} | \text{A}^{c} \cap \text{М}) \cdot \P(\text{A}^{c} \cap \text{М}) \\
&+  \P(\text{В} | \text{A} \cap \text{М}^{c}) \cdot \P(\text{A} \cap \text{М}^{c}) \\
&= 0.18 \cdot 0.6 \cdot 0.4 + 0.9 \cdot 0.4 \cdot 0.6 + 0.54 \cdot 0.4 \cdot 0.4 + 0.36 \cdot 0.6 \cdot 0.6 = 0.4752
\end{align*}
Ответ:
\[
\P(\text{A} | \text{В} ) = \frac{\P(\text{A} \cap \text{В})}{\P(\text{В})} = \frac{0.3456}{0.4752}  =0.(72)
\]

\item Необходимо найти
\[
\P(\text{М} | \text{В}) = \frac{\P(\text{М} \cap \text{В})}{\P(\text{В})}
\]
Знаменатель этой дроби посчитан в предыдущем пункте, посчитаем числитель:
\begin{align*}
\P(\text{М} \cap \text{В}) &= \P(\text{В} | \text{М}) \cdot \P(\text{М}) \\
&= P(\text{В} | \text{М} \cap \text{А}) \cdot \P(\text{А}) \cdot \P(\text{М}) + \P(\text{В} | \text{A}^{c} \cap \text{М}) \cdot \P(\text{А}^{c})  \cdot \P(\text{М}) \\
&= 0.9 \cdot 0.4 \cdot 0.6 + 0.54 \cdot 0.4 \cdot 0.4 = 0.3024
\end{align*}
Ответ:
\[
\P(\text{М} | \text{В}) = \frac{\P(\text{М} \cap \text{В})}{\P(\text{В})} = \frac{0.3024}{0.4752} = 0.(63)
\]
Если Вася на лекции, вероятность застать на ней Алёну выше.
\end{enumerate}


\item $\P(X = 5) = C_{100}^5 0.002^5 0.998^{95}$,

$\E(X) = 0.2$,

$\Var(X) = 0.2\cdot 0.998$,

наиболее вероятно событие $X = 0$.
\item $c = 1/2$,
$\P(X \in [\ln 0.5,\ln 4]) = 5/8$,
$\E(X) = 0$,
$\Var(X)=2$,
$\E(X^{2k+1})=0$,
$\E(X^{2k})=(2k)!$
\item
\begin{enumerate}
\item $\E(Y - 2X - 3) = \E(Y) - 2 \E(X) - 3 = 0$

$\Var(Y - 2X - 3) = \Var(Y) + 4\Var(X) - 2\Cov(Y, 2X) = 16$

$\Cov(X, Y) = \Corr(X,Y) \cdot \sqrt{\Var(X) \cdot \Var(Y)} = 6$
\item $\Corr(Y - 2X - 3, X) = \frac{\Cov(Y, X) - 2 \Var(X)}{\sqrt{\Var(Y - 2X - 3) \cdot \Var(X)}} = -1$.
\item Корреляция равна $-1$, значит, есть линейная взаимосвязь между переменными.
Пусть $Y+ a X = b$, тогда $\Var(Y+ a X)=0$, $\E(Y) = -a + b =1 $.
Решая уравнения, находим, что $a=-2/3, b=1/3$.
\end{enumerate}

\item \begin{enumerate}
\item Таблицы распределения имеют вид:
\begin{center}
\begin{tabular}{@{}cccc@{}}
\toprule
$x$         & $-1$  & $0$   & $1$   \\ \midrule
$\P(X=x)$ & $0.3$ & $0.3$ & $0.4$ \\ \bottomrule
\end{tabular}
\hspace{1cm}
\begin{tabular}{@{}ccc@{}}
\toprule
$y$         & $-1$  & $1$   \\ \midrule
$\P(Y=y)$ & $0.5$ & $0.5$ \\ \bottomrule
\end{tabular}
\end{center}

\item
\begin{multline*}
\Cov(X, Y) = \E(XY) - \E(X) \E(Y)  = (-1)\cdot (-1) \cdot 0.1 + (-1) \cdot 0 \cdot 0.2 + \\
+ (-1) \cdot 1 \cdot 0.2 + 1 \cdot (-1) \cdot 0.2 + 1 \cdot 0 \cdot 0.1 + 1 \cdot 1 \cdot 0.1 -
0.1 \cdot 0 = -0.1
\end{multline*}
\item Да, поскольку если случайные величины независимы, то их ковариция равна нулю.
\item Условное распределение:
\begin{center}
\begin{tabular}{@{}cccc@{}}
\toprule
$X|Y=-1$    & $-1$  & $0$   & $1$   \\ \midrule
$\P(\cdot)$ & $0.2$ & $0.4$ & $0.4$ \\ \bottomrule
\end{tabular}
\end{center}
\item $\E(X | Y = -1) = -1 \cdot 0.2 + 0 \cdot 0.4 + 1 \cdot 0.4 = 0.2$
\end{enumerate}
\end{enumerate}




\subsection[2012-2013]{\hyperref[sec:kr_01_2012_2013]{2012-2013}}
\label{sec:sol_kr_01_2012_2013}

\begin{enumerate}
\item
\begin{enumerate}
\item $\P(A)=0.8\cdot 0.3+0.7\cdot 0.2=0.38$
\item $\P(B)=0.9$
\item $\P(C|A)=\frac{0.3\cdot 0.8}{0.38}=0.632$
\item $\P(C|D)=\frac{0.3\cdot (0.9\cdot 0.8+0.1\cdot 0.2)}{0.9\cdot 0.38+0.1\cdot (1-0.38)}=0.55$
\end{enumerate}
\item Это была задачка-неберучка!
\item
\begin{enumerate}
\item $1$
\item $\E(X)=45/28\approx 1.61$, $\E(X^2)=93/35\approx 2.66$, $\Var(X)=291/3920\approx 0.07$
\item $37/56\approx 0.66$
\item $F(x)=\begin{cases} 0,\, x<1 \\
\frac{x^3-1}{7},\, x\in [1;2] \\
1,\, x>1 \end{cases}$
\end{enumerate}
\item
\begin{enumerate}
\item $a=0.1$
\item $\P(X>-1)=0.7$, $\P(X>Y)=0.1$
\item $\E(X)=-0.2$, $\E(X^2)=2$
\item $\Corr(X,Y)=0.117$
\end{enumerate}
\item
\begin{enumerate}
\item Правильные: $\E(X)=10$, $\Var(X)=9$, неправильные: $\E(Y)=9$, $\Var(Y)=0.9$
\item Наиболее вероятное число укусов равно математическому ожиданию
\item Лучше идти к неправильным пчёлам, так как $\P(X\leq 2)<\P(Y\leq 2)$.
\end{enumerate}
\end{enumerate}

% !TEX root = ../probability_hse_exams.tex
\thispagestyle{empty}
\section{Решения контрольной номер 1. ИП}


\subsection[2019-2020]{\hyperref[sec:kr_01_ip_2019_2020]{2019-2020}}
\label{sec:sol_kr_01_ip_2019_2020}

\begin{enumerate}
  \item xxx
  \item xxx
  \item 
  
  \begin{minipage}{0.3\textwidth}
\begin{tikzpicture}[thick]
	\draw [color=gray, fill=gray!15] circle(2.5cm);
	\draw (0,0) coordinate [label= below:$O$] (D);
	\draw[color=teal] (0:-0.4) arc (140:90:0.68cm);
    \draw[color=teal] (-1.2cm:-0.8cm) node {$\pi - 2\alpha$};
    \draw[color=purple] (0:-2) arc (17:50:0.5cm);
    \draw[color=purple] (-8:-1.8) node {$\alpha$};
    \path[draw] (-2.5,0) coordinate [label= left:$I$] (A)
            -- node [above] {$x$} (1,2.29)  coordinate [label=above:$X$] (B)
            -- ( 2.5,0)  coordinate [label=right:$F$] (F);
    \path[draw] (0.72,2.12) -- (0.91, 1.84) -- (1.17, 2.01);
    \path[draw] (B) -- (D);
    \draw (-0.7,2.4) coordinate [label= above:$G$] (G);
    \path[draw, color=Salmon] (A) -- node [left] {$L$} (G);
    \path[draw] ( 2.5,0)  coordinate (F)
            -- (-2.5,0) coordinate [label= left:$I$] (A);
    \foreach \point in {A,B,D,F,G}
           \fill [black] (\point) circle (2pt);
\end{tikzpicture}
\end{minipage}
\begin{minipage}{0.64\textwidth}
	\begin{enumerate}
	\item Пусть $I$ -- точка, куда приземлился Илон Маск, \\$G$ -- точка, куда приземлилась Грета Тунберг.
	
	Так как точки, куда приземлились ребята выбираются судьбой независимо, мы можем зафиксировать $I$ и менять только~$G$.
	
	$L \le x \Leftrightarrow$ длина меньшей дуги~$\breve{IG} \le$ длина меньшей дуги~$\breve{IX}$
	
	Так как координата Греты распределена равномерно, то:
	\[
	F_L (x) = \P(L \le x) =\P(\breve{IG} \le \breve{IX}) = \frac{\breve{IX}}{\breve{IF}} = \frac{\pi r \cdot \frac{\pi - 2\alpha}{\pi}}{\pi r} = \frac{\pi - 2\alpha}{\pi},\]\\[-15mm]
	\begin{flushright}
	если $x \in [0,1]$
	\end{flushright}
	\end{enumerate}
\end{minipage}
	
	\[
	\cos \alpha = x \Rightarrow \alpha = \arccos x \Rightarrow F_L (x) = \frac{\pi - 2\arccos x}{\pi}, \, \text{если $x \in [0,1]$} \Rightarrow f_L(x)
	= \begin{cases}
	\frac{2}{\pi \sqrt{1-x^2}}, \text{ если } x \in [0,1] \\
	0, \text{ иначе}
	\end{cases}
	\]
	
	Мы рассмотрели только верхнюю часть Луны. Если мы будем рассматривать возможность копать туннель также в нижней части Луны, то в дроби $\P(L \le x) = \frac{\breve{IX}}{\breve{IF}}$ нам нужно будет удвоить и числитель, и знаменатель - отношение, которое мы ищем не изменится $\Rightarrow$ не изменится вероятность.
	
	Причем в данной задаче оказалось неважно, какой длины диаметр Луны.

\begin{enumerate}
	\item В случае объёмной Луны мы будем искать отношение площадей поверхностей.
	
	Илон Маск сидит в центре некоторой поляны Луны.
	Она образована точками, до которых туннель~$L$ не длиннее $x$.
	\[
	F_L (x) = \P(L \le x) = \frac{\text{площадь поляны}}{\text{площадь поверхности сферы}} = \frac{2\pi r x^2}{4\pi r^2} = \frac{x^2}{2r} = x^2, \, \text{если $x \in [0,1]$}
	\]
	
	Тогда $f_L(x)= \begin{cases}
	2x, \text{ если } x \in [0,1] \\
	0, \text{ иначе} \\
	\end{cases}$
\end{enumerate}
  \item xxx
  \item xxx
\end{enumerate}


\subsection[2018-2019]{\hyperref[sec:kr_01_ip_2018_2019]{2018-2019}}
\label{sec:sol_kr_01_ip_2018_2019}

\begin{enumerate}

\item Воспользуемся методом первого шага. Дерево игры выглядит так (в вершинах — длительность сответствующей подыгры, а на концах ветвей — доля разбавленных пинт среди тех, что Джо помнит):
\begin{center}
\begin{tikzpicture} \node {$a$} child {node {1} edge from parent node[left] {Р} node[right] {}}
child {node {$b$}
child {node {1/2} edge from parent node[left] {Р}}
child {node (x) {$c$}
child {node (y) {$d$}
child {node (s) {$e$}
child {node {2/3} edge from parent node[right] {Р}}
edge from parent node[left] {НР}}
child {node {2/3} edge from parent node[right] {Р}}
edge from parent node[right] {Р}}
edge from parent node[right] {НР}}
edge from parent node[right] {НР}};
\draw [->,black] (x) .. controls +(down:1cm) and +(left:1cm) .. node[below,sloped] {НР} (x);
\draw [->,black] (s) .. controls +(up:0.1cm) and +(left:2.5cm) .. node[below,sloped] {НР} (x);
\end{tikzpicture}
\end{center}

Если первая пинта разбавлена, то игра заканчивается (разбавлены 100\% рома).
Если первая пинта не разбавлена, то, если разбавлена вторая, игра заканчивается (50\%).
Если вторая не разбавлена, и третья тоже, то это равносильно тому, что не разбавлены только две (Джо не помнит больше трех).
Если вторая не разбавлена, а третья разбавлена, возможны три случая: если четвёртая разбавлена, то игра заканчивается;
если четвёртая не разбавлена, и пятая не разбавлена, то это эквивалентно тому, что не разбавлены только две;
если четвёртая не разбавлена, а пятая разбавлена, то игра заканчивается.
Из этих соображений получаем систему уравнений:
\[
\begin{cases}a=\frac{1}{8}+\frac{7}{8}(b+1)\\
b=\frac{1}{8}+\frac{7}{8}(c+1)\\
c=\frac{1}{8}(d+1)+\frac{7}{8}(c+1)\\
d=\frac{1}{8}+\frac{7}{8}(e+1)\\
e=\frac{1}{8}+\frac{7}{8}(c+3)
\end{cases}
\]

Отсюда $a=7514/192,b=1046/24,c=146/3,d=122/3,e=136/3$.

\item Пусть
\[
Z_i=\begin{cases}1, i \mbox{-я пинта разбавлена} \\
0, \mbox{иначе}
\end{cases}
\]

Пусть с вероятностью $x_n$ последовательность случайных величин $(Z_n)$ длиной
$n$ не содержит двух единиц подряд и оканчивается нулём, а с вероятностью $y_n$ —
не содержит двух единиц подряд и оканчивается единицей.

Пусть на последнем месте $(Z_n)$ стоит единица. Это может произойти с вероятностью
$1/2$. При этом перед последней единицей может стоять любая последовательность
длиной $n-1$, оканчивающаяся нулём. Значит, $y_n=0.5x_{n-1}$.

Пусть на последнем месте $(Z_n)$ стоит нуль. Это может произойти с вероятностью
$1/2$. При этом перед последним нулём может стоять любая последовательность длиной
$n-1$. Значит, $x_n=0.5(x_{n-1}+y_{n-1})$.

Таким образом, получена следующая разностная система:
\[
\begin{cases}x_n=0.5(x_{n-1}+y_{n-1})\\
y_n=0.5x_{n-1}
\end{cases}
\]

Более того, можно поставить задачу Коши. Так как $(Z_2)$ может равновероятно
иметь одну из 4 реализаций $(11, 10, 01, 00)$, из которых не содержат двух единиц
подряд и оканчиваются на 0 две, то $x_2=1/2$. Аналогично, $y_2=1/4$. Решив задачу
Коши, найдем формулы для $x_n$, $y_n$. Ответом будем число $x_{100}+y_{100}$.

\item
\begin{enumerate}
\item Оптимальная стратегия для Али-Бабы состоит в чередовании открытия двух
диагонально противоположных и двух соседних монет.

Изначально имеется только четыре варианта расположения монет:
\begin{center}
\begin{tikzpicture}[every node/.style={draw}]
\path[yshift=1.5cm,rectangle] (-6,0) node(a1){О} (-5,0) node(a2){Р} (-5,1) node(a3){О} (-6,1) node(a4){О};
\filldraw[fill=black] (a1) -- (a2) -- (a3) -- (a4) -- (a1);
\path[yshift=1.5cm,rectangle] (-4,0) node(a1){Р} (-3,0) node(a2){Р} (-3,1) node(a3){О} (-4,1) node(a4){О};
\filldraw[fill=black] (a1) -- (a2) -- (a3) -- (a4) -- (a1);
\path[yshift=1.5cm,rectangle] (-2,0) node(a1){Р} (-1,0) node(a2){О} (-1,1) node(a3){Р} (-2,1) node(a4){О};
\filldraw[fill=black] (a1) -- (a2) -- (a3) -- (a4) -- (a1);
\path[yshift=1.5cm,rectangle] (0,0) node(a1){Р} (1,0) node(a2){Р} (1,1) node(a3){Р} (0,1) node(a4){О};
\filldraw[fill=black] (a1) -- (a2) -- (a3) -- (a4) -- (a1);
\end{tikzpicture}
\end{center}

На первом ходе Али делает двух орлов на одной диагонали. Всего возможно восемь
случаев (по две диагонали в каждом из начальных вариантов). Из восьми равновозможных
случаев два приводят к успеху. При этом, если успех не достигнут, можно получить в
итоге только две комбинации орлов и решек (левую — в четырёх случаях, правую — в двух):
\begin{center}
\begin{tikzpicture}[every node/.style={draw}]

\path[yshift=1.5cm,rectangle] (-2,0) node(a1){О} (-1,0) node(a2){Р} (-1,1) node(a3){О} (-2,1) node(a4){О};
\filldraw[fill=black] (a1) -- (a2) -- (a3) -- (a4) -- (a1);
\path[yshift=1.5cm,rectangle] (0,0) node(a1){Р} (1,0) node(a2){О} (1,1) node(a3){Р} (0,1) node(a4){О};
\filldraw[fill=black] (a1) -- (a2) -- (a3) -- (a4) -- (a1);
\end{tikzpicture}
\end{center}

На втором ходе Али переворачивает две соседние монеты орлами вверх. При этом либо
наступает успех (8 случаев из 24), либо неуспех, приводящий к единтсвенно возможной
комбинации орлов и решек:
\begin{center}
\begin{tikzpicture}[every node/.style={draw}]
\path[yshift=1.5cm,rectangle] (-2,0) node(a1){О} (-1,0) node(a2){Р} (-1,1) node(a3){О} (-2,1) node(a4){О};
\filldraw[fill=black] (a1) -- (a2) -- (a3) -- (a4) -- (a1);
\end{tikzpicture}
\end{center}

На третьей попытке открывается или диагональ с орлами, или диагональ с орлом и решкой. В
о втором случае, очевидно, надо заменить решку на орла, и успех обеспечен. Но в первом
случае оптимально перевернуть орла и сделать 2 решки. Это даст неуспех, но Али точно
будет знать, что на четвёртой попытке монеты могут быть расположены только так:
\begin{center}
\begin{tikzpicture}[every node/.style={draw}]
\path[yshift=1.5cm,rectangle] (-2,0) node(a1){Р} (-1,0) node(a2){Р} (-1,1) node(a3){О} (-2,1) node(a4){О};
\filldraw[fill=black] (a1) -- (a2) -- (a3) -- (a4) -- (a1);
\end{tikzpicture}
\end{center}

На четвёртом ходе в любом из четырёх вариантов (Али не знает, на каком ребре решки)
нужно перевернуть обе открытые соседние монеты, тогда в двух случаях будет успех,
а в двух других -- неуспех, при котором монеты могут быть расположены только так:
\begin{center}
\begin{tikzpicture}[every node/.style={draw}]
\path[yshift=1.5cm,rectangle] (-2,0) node(a1){Р} (-1,0) node(a2){О} (-1,1) node(a3){Р} (-2,1) node(a4){О};
\filldraw[fill=black] (a1) -- (a2) -- (a3) -- (a4) -- (a1);
\end{tikzpicture}
\end{center}

Очевидно, что если на пятом ходе, открыв любую диагональ, Али перевернёт находящиеся
на ней монеты, то он гарантирует себе успех.

\item Как видим, в худшем случае потребовалось $5$ попыток.
\end{enumerate}
Источник: Кордемский, Математика изучает случайности.

\item Возьмём колоду, добавим в неё джокера. Разложим в открытую по окружности.
Джокер означает место разрыва окружности для её выкладывания в обычную колоду.
Замечаем, что джокер и четыре дамы разбивают окружность на пять случайных отрезко.
В силу симметрии ожидаемые длины этих отрезков равны и равны по $48/5$.
Значит первая дама попадается в среднем на $48/5 + 1$ месте.

\item Обозначим искомую вероятность быть в Неведении в момент $t$ значком $p_t$.
\[
p_{t+\Delta} = p_t (1-\lambda\Delta - o(\Delta))
\]

Отсюда получаем, что
\[
\frac{p_{t+\Delta} - p_t}{\Delta} = -\lambda p_t + \frac{o(\Delta)}{\Delta}
\]
Устремляем $\Delta$ к нулю и решаем получающееся дифференциальное уравнение
с начальным условием $p_0 = 1$, так как изначально Ученик находится в Неведении.

Итого:
\[
p_t = \exp(-\lambda t)
\]

\item
\[
\P(Y_4 \in [t; t+\Delta]) = C_{10}^1 \cdot \Delta \cdot C_{9}^3 t^3 (1-t)^6 + o(\Delta)
\]

Читаем вслух:
\begin{enumerate}
  \item Одна из десяти величин должна попасть в отрезок $[t; t + \Delta]$;
  \item Три из девяти оставшихся должны оказаться меньше $t$;
  \item Шесть из девяти оставших должны оказаться больше $t$;
\end{enumerate}
Вероятностью попадания двух и более величин в отрезок длины $\Delta$ пренебрегаем!
\end{enumerate}

\subsection[2017-2018]{\hyperref[sec:kr_01_ip_2017_2018]{2017-2018}}
\label{sec:sol_kr_01_ip_2017_2018}

\begin{enumerate}
\item Обозначим вероятность того, что сыр достанется Белому за $b$, если игра
начинается с его броска.

\begin{enumerate}
\item Получаем уравнение
\[
b = \frac{1}{12} + \frac{11}{12} \cdot \frac{11}{12} b
\]

Пояснение: Как Белый может победить в исходной игре? Либо сразу выкинуть 6 с вероятностью $1/12$.
Либо передать ход Серому ($11/12$), получить ход снова ($11/12$) и выиграть в продолжении игры.
Продолжение игры по сути совпадает с исходной игрой.

\item Игра продолжается до тех пор, пока кто-то не выкинет «6».
Для нахождения среднего количества бросков воспользуемся методом первого шага.

Обозначим среднее количество бросков нашей игры за $S$.
Когда Белый бросает кубик, с вероятностью $\frac{1}{12}$ игра закончится за один бросок,
а с вероятностью $\frac{11}{12}$ игра продолжится и ход перейдёт к Серому.
Но та игра, которая начнётся, когда бросать будет Серый, ничем не отличается от предыдущей,
поэтому среднее количество бросков в ней будет равно $S$.
Однако мы попадём в эту игру, «потратив» один бросок. Таким образом мы получаем:

\[
S = \frac{1}{12} \cdot 1 + \frac{11}{12}(S +1)
\]

Получается, что $S = 12$, значит игра длится в среднем 12 бросков.
\end{enumerate}

\item

\item Для того, чтобы выжить, мышам нужно ещё до начала игры договориться о стратегии,
которая позволит им с наибольшей вероятностью открыть нужные сундуки.
Если хотя бы две мыши выберут одинаковый сундук, то их в любом случае съедят.
Поэтому одной из оптимальных стратегий будет ещё до начала игры мышам договориться
и назвать левый сундук золотым, сундук посередине серебряным, а правый — платиновым.
Каждый мышонок должен открыть тот сундук, в честь которого назван необходимый ему металл.
Если внутри он обнаруживает свой металл, то он выбирает этот сундук,
если внутри находится не тот металл, мышонок открывает тот сундук,
на который указывает лежащий внутри предмет.

Например, первым заходит Микки Маус. Он открывает золотой (левый) ящик.
Если внутри лежит золото, то он выходит из комнаты. Если же внутри лежит, например, серебро,
то Микки Маус открывает сундук посередине.
Путём перебора можно посчитать, что в 4 случаях из 6 мыши смогут найти нужный металл,
поэтому вероятность выигрыша при данной стратегии равна $\frac{2}{3}$.

\item
\begin{enumerate}
  \item Обозначим буквой $X$ количество детей в случайной семье.
  Можно просуммировать ряд $\E(X) = 1\cdot 0.5 + 2\cdot 0.5^2 + 3\cdot 0.5^3 + \ldots$. 
  А можно воспользоваться методом первого шага и заметить, что либо первой рождается девочка 
  и мышиная семья больше не заводит мышат, либо рождается мальчик и семья ситуация превратилась в первоначальную
  плюс один ребёнок.

  \[
  \E(X) = 0.5 \cdot 1 + 0.5 \cdot (1 + \E(X))  
  \]
  Находим, $\E(X)=2$.
  \item В мышиной стране много семей. Занумеруем семьи. Обозначим буквой $X_i$ число детей в $i$-ой семье.
  Тогда доля $D$ — это суммарное количество мальчиков делить на суммарное количество детей:

  Получаем
  \[
  \frac{(X_1 - 1) + (X_2 - 2)+ \ldots + (X_n - 1)}{X_1 + X_2 + \ldots +X_n} = 1 - \frac{n}{\sum_{i=1}^n X_i}  
  \]

  Основной смысл математического ожидания $\E(X_i)$ — это среднее выборочное при большом количестве повторений опыта.

  Следовательно, $\frac{\sum_i X_i}{n} \to \E(X_i)=2$. И стало быть, при $n\to\infty$ получаем $D \to 1 - 1/2=1/2$.

  Интуитивный аргумент немного опасен, но всё же приведём его: вероятности рождения мальчиков и девочек равны,
  поэтому и доля будет половина. 
  Опасность аргумента состоит в том, что семьи могли бы в альтернативном условии рожать мышат по принципу: 
  пока доля мальчиков в стране не превысит 50\%. 

  \item Можно просто аккуратно суммировать ряд:
  \[
  \E(Y) = 0.5 \cdot \frac{0}{1} + 0.5^2 \cdot \frac{1}{2} + 0.5^3 \cdot \frac{2}{3} + \ldots
  \]
  
  Здесь мудрый студент может вспомнить ряд Тейлора для логарифма:
  \[
  \ln (1 + r) = r - r^2/2 + r^3/3 - r^4/4 +\ldots
  \]

  Стало быть,
  \[
  \ln (1 - 0.5) = -0.5 - 0.5^2/2 - 0.5^3/3 - 0.5^4/4  
  \]

  И исходное ожидание равно:

  \[
  \E(Y) = 1 + \ln 0.5 = 1 - \ln 2 \approx 0.31  
  \]

\item Неожиданно лёгкий вопрос, $\E(X - 1) = 2 -1=1$.
\end{enumerate}


\item Благосостояние кота Василия, положившего один гурд на вклад,
равно $m_t = 1\cdot e^{rt}$, где $r$ — процентная ставка, а $t$ — прошедшее время.

Откуда взялась экспонента?

Допустим время дискретно, а процентная ставка равна 5\% в год. 
Тогда сумма вклада каждый год домножается на $1.05$ и эволюционирует по формуле:

\[
m_t = 1 \cdot 1.05^t
\]

Но любое число можно представить в виде экспоненты, $1.05 = e^{\ln 1.05}$.

Поэтому
\[
m_t = e^{\ln 1.05 \cdot t}  = e^{rt}
\]


Момент закрытия вклада $T$ равномерно распределён на отрезке от 0 до $a$,
поэтому сумма, которую получит Василий, представима в виде $Z = e^{Y}$, где $Y \sim U[0; ra]$.
По условию, $a$ очень велико, поэтому $ra$ тоже очень велико.

Вероятность того, что первая цифра будет равна 1, равна вероятности того,
что доход Василия будет лежать в пределах от 1 до 2 гурдов, плюс вероятность того,
что он лежит в пределах от 10 до 20 гурдов и т.д.
Таким образом, можно представить эту вероятность, как:
\[
\P(N=1) = \P(e^Y \in [1;2) ) + \P(e^Y \in [10; 20) ) + \ldots
\]

Это выражение можно преобразовать таким образом:
\[
\P(N=1) = \P(Y \in [\ln 1; \ln2) ) + \P(Y \in [\ln 10; \ln 20) ) + \ldots
\]

Так как $Y$ — равномерно распределённая величина,
то $\P(Y \in [\ln 1; \ln2) ) = \frac{\ln 2 - \ln 1}{ra}$.
Для последующих слагаемых вероятность рассчитывается таким же образом.
Воспользовавшись свойством логарифма, можно заметить,
что $\frac{\ln 20 - \ln 10}{ra} = \frac{\ln 2}{ra}$.
Поэтому вероятность того, что на первом месте суммы вклада стоит единица,
равна $n\cdot \frac{\ln 2}{ra}$, где $n$ — количество слагаемых.
Путём аналогичных рассуждений получаем, что вероятность того,
что на первом месте стоит двойка, равна $n\cdot \frac{\ln 3- \ln 2}{ra}$.
Из-за того, что $a$ велико, можно считать, что число слагаемых одинаково.

На первом месте обязательно будет находиться какая-то цифра,
поэтому сумма вероятностей будет равна 1. Получаем:
\[
\dfrac{n}{ra}\left(\ln \frac{2}{1} + \ln \frac{3}{2} + \ldots + \ln \frac{10}{9}\right) = 1
\]

Таким образом $\frac{n}{ra} = \frac{1}{\ln 10}$.
Получается, что вероятность того, что на первом месте стоит единица, равна:
\[
\P (N=1) = \dfrac{\ln 2}{\ln 10}
\]

Закон распределения первой цифры выводится сложением соответствующих вероятностей.
\end{enumerate}


\subsection[2016-2017]{\hyperref[sec:kr_01_ip_2016_2017]{2016-2017}}
\label{sec:sol_kr_01_ip_2016_2017}

\begin{enumerate}
\item
\begin{enumerate}
\item Для удобства занумеруем макаронины и выделим у каждой левый и правый конец.
Взяли правый конец первой макаронины и подвязали случайной.
Взяли свободный конец только что подвязанной макаронины и подвязали случайно. И так далее.
\[
\frac{2n-2}{2n-1}\cdot \frac{2n-4}{2n-3}\cdot \ldots \cdot \frac{2}{3} \cdot 1
\]
\item Допустим, что при $n$ макаронинах в среднем образуется $e_n$ колец.
После первого соединения задача сводится к меньшему числу макаронин, важно только учесть,
образовалось ли кольцо при первом соединении:
\[
e_n = \frac{1}{2n-1}(e_{n-1}+1) + \frac{2n-2}{2n-1}e_{n-1} = e_{n-1} + \frac{1}{2n-1}
\]
\item Количество коротких колец можно разбить в сумму, $X=Z_1 + \ldots + Z_n$.
Вероятность завязывания конкретной макаронины в кольцо равна $1/(2n-1)$:
«левый конец» надо привазять именно к «правому». Значит, $\E(X)=n/(2n-1)$.
\end{enumerate}

\item
\begin{enumerate}
\item Рассмотрим обратную ситуацию: на планете есть точка, из которой связаться
хотя бы с одним пепелацем нельзя. Такое возможно, если все, кроме одного, сели
в одну полуокружность.
\[
\P(\text{есть точка без связи}) = n \cdot \left(\frac{1}{2}\right)^{n-1}
\Rightarrow \P(\text{из любой точки есть связь}) =
1 - n \cdot \left(\frac{1}{2}\right)^{n-1}
\]
\item Зафиксируем координату посадки первого пепелаца и возьмём её за точку отсчёта.
Изобразим на плоскости возможные значения центральных углов между первым пепелацем и
оставшимися и закрасим нужные участки. Получим $3/8$.
\item Зафиксируем координату посадки первого пепелаца. Обозначим центральный угол
между первым и вторым пепелацами $\alpha$. Функция плотности имеет вид:
$p(\alpha) = \frac{\sin(\alpha)}{2}$

Итог: $\int_0^{\pi/2} p(\alpha) \frac{\alpha + \pi}{2\pi} d \alpha + \int_{\pi/2}^{\pi}
p(\alpha) \frac{\pi - \alpha}{2\pi} d \alpha = \frac{\pi + 2}{4\pi}$
\end{enumerate}

\item Вспомним для начала, что площадь круга равна $\pi r^2$, а площадь сферы равна
$4\pi r^2$. Составим из маленьких треугольничков многогранник очень похожий на сферу
с единичным радиусом. Площадь этого многогранника будет примерно равна $4\pi$.
Проекция многогранника представляет собой примерно круг единичного радиуса.
Проекция имеет два слоя. С учётом обоих слоёв площадь проекции равна $2\pi$.
Значит отношение площади проекции к площади многогранника равно $1/2$.

От взаимного расположения треугольничков в пространстве ожидаемая площадь
проекции не зависит в силу аддитивности математического ожидания.

Ответ: 21 см$^2$.

\item
\begin{enumerate}
\item Мысленно отметим на окружности три точки: места ударов Брюса Ли и точку,
где схватился Чак Норрис. Можно считать, что эти три точки равномерно и независимо
распределены по окружности. Следовательно, среднее расстояние между соседними
точками равно $1/3$. Чак Норрис берёт два кусочка, слева и справа от своей точки.
Значит ему в среднем достаётся $2/3$ окружности.
\item Объявим точку, где схватился Чак Норрис нулём. Координаты двух ударов
изобразим на плоскости. Закрашиваем подходящий участок. Вероятность того, что
кусок Брюса Ли длиннее, равна $1/4$.
\end{enumerate}

\item  Рассмотрим совершенно конкурентный невольничий рынок начинающих певиц.
Певицы в хорошем настроении продаются по $V_1$, в депрессии — по $V_2$.
Получаем систему уравнений:
\[
\begin{cases}
  V_1 = 0.75 + (0.5 V_1 + 0.5 V_2) \\
  V_2 = \max_x \sqrt{x}V_1 + (1 - \sqrt{x})V_2 - x
\end{cases}
\]
Оптимизируем и получаем, $x^* = (V_1 - V_2)^2/4$. Из первого уравнения находим
$(V_1 - V_2)/2=0.75$.

\item Да. Например, такая. До общения с Джульеттой подкидывать монетку до выпадения
первого орла и запомнить число потребовавшихся подбрасываний. Пусть это будет число $X$.
Открыть равновероятно левую или правую руку Джульетты. Если открытое число больше $X$,
то сказать, что оно большее, иначе сказать, что меньшее.
\item Если $\gamma$ — вероятность самостоятельного познания Истины, а $\alpha$ —
передачи Истины отдельно в каждую из сторон, то
\[
p = \gamma + (1-\gamma) p \alpha.
\]

То есть $p=\gamma/(1-\alpha(1-\gamma))$.

Для решения второго пункта наложим на Абу Али Хусейн ибн Абдуллах ибн аль-Хасан
ибн Али ибн Сина обет молчания. Это не повлияет на вероятность постижения им Истины,
однако превратит задачу в две уже решённых :) Получаем
\[
q = \gamma + (1-\gamma)(2p\alpha - p^2\alpha^2)
\]
\end{enumerate}



\subsection[2015-2016]{\hyperref[sec:kr_01_ip_2015_2016]{2015-2016}}
\label{sec:sol_kr_01_ip_2015_2016}

\subsubsection*{Индивидуальный тур}

\begin{enumerate}
\item Сократ, эта — Η, η, дзета — Ζ, ζ, вега — нет, шо — ϸ, τ — тау, θ — тета, ξ — кси.
Греческая буква шо, ϸ, была введена Александром Македонским и ныне вышла из употребления.
По крайней мере, в греческом :) Заглавная примерно такая же, только её utf-код 03f7
не поддерживается шрифтом Linux Libertine.

\item Да. Cобытия независимы в совокупности, если для любого поднабора событий $A_1$,
\ldots, $A_k$ выполняется равенство $\P(A_1 \cap A_2 \cap \ldots \cap A_k) = \P(A_1)
\cdot \ldots \cdot \P(A_k)$. Нет.
\item $1/4$, $2/3$, $15$
\item $0$, $0.8^5\cdot 0.2$, $1-0.8^6$
\item $0.8^{10}$, $C_{10}^3 0.2^3 0.8^7$, $2$
\item $1/2$, $3/16$, $3/8$
\item
\begin{enumerate}
\item
\begin{flalign*}
F_X(x) &= \begin{cases}
0, \, x<0 \\
x^3, \, x \in [0;1] \\
1, \, x>1
\end{cases}&&
\end{flalign*}
\item $1/8$
\item $2^{-1/3}$
\item $56/63$
\item
\begin{flalign*}
F_Y(y) &= \begin{cases}
0, \, y<0 \\
1-1/y^3, \, y>0
\end{cases}&&
\end{flalign*}
\item
\begin{flalign*}
f_Y(y) &= \begin{cases}
0, \, y<0 \\
3y^{-4}, \, y>0
\end{cases}&&
\end{flalign*}

\end{enumerate}
\end{enumerate}

\subsubsection*{Командный тур}

\begin{enumerate}
\item Если отрублено 10 щупалец, значит либо был один удар породивший два новых
щупальца, либо было два удара, породивших по одному новому, а все остальные удары
не порождали новых щупалец.

Искомая вероятность равна: $8\cdot 0.5^9 \cdot 0.25^1 + C_8^2 0.5^8 0.25^2$.

Вероятность вечного боя равна нулю. Достаточно доказать, что с вероятностью один
за конечное время побеждается одноногий Кракен. А эта вероятность удовлетворяет
уравнению: $p=\frac{1}{4}p + \frac{1}{4}p^2 + \frac{1}{2} 1$. Единственный осмысленный
корень у этого уравнения — $1$.

Замечаем, что на победу над $k$-шупальцевым Кракеном уходим в $k$ раз больше ударов
в среднем чем на победу на $1$-щупальцевым. Отсюда:
\[
e_1=1 + 0.5\cdot 0 + 0.25\cdot e_1 + 0.25 \cdot 2e_1
\]
Решаем, получаем $e_1=4$ и $e_8=32$

\item Либо первая пинта разбавлена, либо первая неразбавлена, а вторая разбавлена,
то есть
\[
0.25 + 0.75\cdot 0.25 =0.4375
\]
Рисуем граф. %:

Составляем систему (индекс — количество выпитых неразбавленных пинт):
\[
\begin{cases}
e_0=\frac{1}{4} + \frac{3}{16}2 + \frac{9}{16}(2+e_2) \\
e_2=1+\frac{3}{4}e_2 + \frac{1}{4}e_0
\end{cases}
\]
Находим $e_0=64/7\approx 9$

\item Для $t>0$:
\[
\P(-\ln X \leq t)=\P(\ln X > -t)=\P(X > e^{-t})=1-e^{-t}
\]
Итого,
\[
F_{-\ln X}(t)=\begin{cases}
0, \, t < 0 \\
1-e^{-t}, \, t \geq 0
\end{cases}
\]
Из геометрических соображений легко найти $\P(XY < a)$ для $a\in (0;1)$:
\[
\P(XY < a)=a + \int_a^1 \frac{a}{x} \, dx=a-a\ln a
\]
Переходим ко второму пункту, для $t>0$:
\[
\P(-(\ln X + \ln Y) < t)=\P(XY > e^{-t})= 1-e^{-t} -t e^{-t}
\]
Итого:
\[
F_{-\ln X - \ln Y}(t)=\begin{cases}
0, \, t < 0 \\
1-e^{-t} - te^{-t}, \, t \geq 0
\end{cases}
\]
После дифференциирования получаем функцию плотности для $S=-\ln X - \ln Y$:
\[
f_S(s)=\begin{cases}
0, \, s < 0 \\
se^{-s}, \, s \geq 0
\end{cases}
\]
Приближаемся к финальной вероятности:
\[
\P(ZS > t)= \int_t^{\infty} \int_{t/s}^1  se^{-s} \, dz\, ds=
\int_t^{\infty} (s-t)\cdot e^{-s} \, ds= \ldots = e^{-t}
\]
Сравниваем результат с первым пунктом и приходим к выводу, что величина $(XY)^Z$
имеет равномерное распределение на $[0;1]$.

\item Если нанято $n$ пиратов, то вероятность, того, что в конкретный день все
работают равна $(364/365)^n$. Следовательно, ожидаемое количество праздничных дней
равно $365(1-(364/365)^n)$.

Решаем уравнение:
\[
1-(364/365)^n=100/365
\]
Получаем:
\[
n=\frac{\ln 265- \ln 365}{ \ln 364 - \ln 365}\approx 117
\]
Ожидаемое количество рабочих пирато-дней равно: $365n(364/365)^n$.

Получаем:
\[
n^*=1/(\ln 365 - \ln 364)\approx 364
\]

\item
\begin{enumerate}
\item $\P(R_{100})=1/100$ (максимум из 100 величин должен плюхнуться на сотое место),
$\P(B_{100})=1/3$ (максимум из трёх величин должен плюхнуться на второе место)
\item $\E(X)=1+\frac{1}{2} + \frac{1}{3}+\ldots + \frac{1}{100}\approx \ln 100 \approx 4.6$.
Т.к. $X=X_1+X_2+\ldots + X_{100}$ и $\E(X_i)=1/i$.
\item $\P(R_{99} | R_{100})=1/99$, $\P(R_{100}|B_{100})=3/101$

Для проверки: $\P(R_{99} \cap R_{100})=98!/100!$ ($100!$ — всего перестановок,
$98!$ — первые 98 можно переставлять свободно, а в конце должны идти второй
наибольшое и наибольшее). $\P(R_{100} \cap B_{100})=1/101$ (максимум из 101 числа
плюхнется на 100ое место).
\end{enumerate}

\item Если все пираты открывают первый и второй сундуки, то вероятность выигрыша
равна нулю.

Оптимальная стратегия (одна из). Три пирата заранее договариваются, о названиях сундуков.
Они называют эти три сундука (ещё до игры)  «рубиновым», «пиастровым» и «золотым».
Генри Рубинов должен начать с открытия рубинового сундука, Френсис Пиастров —
с пиастрового, Эдвард Золотов — с золотого. Далее каждый пират должен открыть тот
сундук, на который указывает предмет, лежащий в первом открытом им сундуке. Например,
если Генри Рубинов, открыв сначала рубиновый сундук обнаруживает там пиастры, он должен
открывать пиастровый сундук.

Вероятность победы при такой стратегии легко находится перебором 6 возможных вариантов
и равна\ldots Та-дам!!! $2/3$.
\end{enumerate}


\subsection[2014-2015]{\hyperref[sec:kr_01_ip_2014_2015]{2014-2015}}
\label{sec:sol_kr_01_ip_2014_2015}


\subsubsection*{Часть 1}

Не претендуя на единственность, решения претендуют на правильность!

\begin{enumerate}
\item
\begin{enumerate}
\item $\P(\text{все арбузы спелые}) = 0.9^2\cdot 0.7 = 0.567  $
\item $A$ = \{случайно выбранный арбуз — от тёти Маши\};
$B$ = \{случайно выбранный арбуз оказался спелым\}. Формула условной вероятности:
\[
\P(A|B) = \frac{\P(A \cap B)}{\P(B)} = \frac{2/3 \cdot 0.9}{2/3 \cdot 0.9 + 1/3 \cdot 0.7}
= \frac{18}{25}
\]
\item $A$ = \{второй и третий съеденные арбузы — от тёти Маши\};
$B$ = \{все три арбуза — спелые\}. Дает ли нам что-то о принадлежности арбузов
к тёте Маше или тёте Оле то, что все арбузы — спелые? События независимы!
\[
\P(A|B) = \P(A) = \frac{1}{3}
\]
\end{enumerate}

\item
\begin{enumerate}
\item $\E(X) = \sum \P(X_i) X_i = 1.9$

$\Var(X) = \E(X^2) - (\E(X))^2 = 0\cdot 0.1 + 1 \cdot 0.3 + 4 \cdot 0.2 +
9 \cdot 0.4 - 1.9^2 = 1.09$

\item Раз ребенок выбран, значит, в его семье дети есть! Всего детей $n\E(X) = 1.9n$.
Семей с одним ребенком — $0.3n$, значит, детей из семей с одним ребенком — $0.3n$.
Аналогично, детей из семей с двумя детьми — $0.4n$; детей из семей с тремя детьми — $1.2n$.

Теперь легко построить закон распределения случайной величины $Y$:

\begin{center}
\begin{tabular}{cccc}
\toprule
$y$ & $1$ & $2$ & $3$  \\
$\P(Y=y)$ & $3/19$ & $4/19$ & $12/19$  \\ \bottomrule
\end{tabular}
\end{center}

\[
\E(Y) = \frac{3}{19} + \frac{8}{19} + \frac{36}{19} = \frac{47}{19} > \E(X)
\]
\end{enumerate}

\item
Любителям (или нелюбителям) интегралов:
\begin{enumerate}
\item Да это же интеграл от функции плотности на всей числовой прямой! Ответ: единица!
\item \[
\E(X) = \int \limits_0^2 x f(x) dx = \int \limits_0^2 \frac{3}{8} x^3 dx =
\left. \frac{3}{32} x^4 \right|_0^2 = \frac{3}{2}
\]

\[
\E\left(X^2\right) = \int \limits_0^2 x^2 f(x) dx = \int \limits_0^2 \frac{3}{8} x^4 dx
= \left. \frac{3}{40} x^5 \right|_0^2 = \frac{12}{5}
\]

Формула дисперсии:
\[
\Var(X) = \E\left(X^2\right) - \left(\E(X) \right)^2 = \frac{12}{5} - \frac{9}{4} = \frac{3}{20}
\]

\item \[
\P(X>1.5) = \int \limits_{1.5}^2 f(x) dx = \int \limits_{1.5}^2 \frac{3}{8} x^2 dx =
\left.\frac{1}{8} x^3 \right|_{1.5}^2 = \frac{37}{64}
\]

Вычислим вероятность условия:

\[
\P(X>1) = \int \limits_1^2 f(x) dx = \int \limits_1^2 \frac{3}{8} x^2 dx
= \left. \frac{1}{8} x^3 \right|_1^2 = \frac{7}{8}
\]

\[
\P(X>1.5 | X>1) = \frac{\P(X>1.5)}{\P( X>1)} = \frac{37/64}{7/8} = \frac{37}{56}
\]

\item Должно выполниться следующее соотношение:
\[
\int \limits_{-\infty}^{+\infty} c x f(x) dx  = 1
\]

Применительно к нашей задаче:
\[
\frac{3c}{8} \int \limits_0^2 x^3 dx  = \left.\frac{3c}{32} x^4 \right|_0^2 = \frac{3c}{2} =
1 \Rightarrow c = \frac{2}{3}
\]
\end{enumerate}

\item
You have to learn the rules of the game. And then you have to play better than
anyone else. (А. Эйнштейн)

\begin{enumerate}
\item \[
\Var(Z) = \E\left(Z^2\right) - (\E(Z))^2 = 15 - 9 = 6
\]
\[
\Var(4 - 3Z) = 9\Var(Z) = 54
\]
\[
\E\left(5 + 3Z - Z^2\right) = 5 + 3\cdot (-3)  - 15 = -19
\]

\item
\[
\Var(X \pm Y) = \Var(X) + \Var(Y) \pm 2 \cdot \Cov(X, Y)
\]
Отсюда получаем:
\[
\Var(X + Y) - \Var(X - Y) = 4 \Cov(X, Y) \Rightarrow \Cov(X, Y) = 2.5
\]
\[
\Cov(6 - X, 3Y) = -3\cdot 2.5 = -7.5
\]

\item
\[
\Cov(X, Y) = 2.5 \ne 0
\]
Случайные величины действительно независимы.
\end{enumerate}

\item
В условии не сказано сколько ответов являются верными. Предположим, что правильный
ответ $0.25$. Но это невозможно, потому что вариантов ответа $0.25$ — два (1 и 4),
значит ответ $0.5$ тоже был бы правильный. Предположим, что правильный $0.5$.
Тогда $0.25$ тоже правильный — таких вариантов два из четырех, значит вероятность
попасть в $0.25$, выбрав ответ наугад, равна $0.5$. Ответ $0.6$, очевидно, неверен,
потому что вероятность попасть в него равна $0.25$. \\
\textbf{Правильный ответ:} $0$

\item
Удобно рассуждать следующим образом: предположим, что каждая опечатка наугад
(с равными вероятностями и независимо от других опечаток) выбирает, на какую
страницу ей попасть\footnote[1]{Ну очень самостоятельные!}.

\begin{enumerate}
\item Пусть $X$ — число опечаток на 13 странице.
\[
\P(X \geqslant 2) = 1 - \P(X=0) - \P(X=1)
\]
$\P(X=0) = \left( \frac{499}{500} \right)^{400}$ — каждая из 400 опечаток
не доложна попасть на 13 страницу.

$\P(X=1) = 400\cdot\frac{1}{500}\cdot\left( \frac{499}{500} \right)^{399}$ —
ровно одна опечатка (а есть 400 вариантов) должна попасть на 13 страницу,
а остальные — мимо. Соответственно:
\[
\P(X \geqslant 2) = 1 - \left( \frac{499}{500} \right)^{400} -
400\cdot\frac{1}{500}\cdot\left( \frac{499}{500} \right)^{399} \approx 0.1911357
\]
Это если считать в явном виде. А если пользоваться приближением Пуассона:
\[
p(k) = \P(X = k) = \frac{\lambda^k}{k!}e^{-\lambda}
\]
неплохо бы вспомнить, что параметр $\lambda$ это математическое ожидание $X$,
поэтому расчеты здесь пока оставим до лучших времен.

\item Пусть $X$ — число опечаток на 13 странице. Введем случайную величину
\[
X_i =
\begin{cases}
1 & \text{если } i\text{-ая опечатка попала на 13 страницу}\\
0 & \text{если нет}
\end{cases}
\]
Тогда $X = \sum\limits_{i=1}^{400}X_i$. Рассмотрим отдельно $X_i$:

\begin{center}
\begin{tabular}{@{}ccc@{}}
\toprule
$x$         & $1$             & $0$               \\
$\P(X=x)$ & $\frac{1}{500}$ & $\frac{499}{500}$ \\ \bottomrule
\end{tabular}
\end{center}

Так как $i$-ая опечатка наугад выбирает одну страницу из 500 и это должна быть именно 13.

Тогда:
\[
\E(X_i) = \frac{1}{500} = \E(X^2_i) \Rightarrow
\]
\[
\Var(X_i) = \E(X^2_i) - (\E(X_i))^2 = \frac{1}{500} - \left(\frac{1}{500}\right)^2 = \frac{499}{500^2}
\]
Значит
\[
\E(X) = \E\left(\sum\limits_{i=1}^{400}X_i\right) = \sum\limits_{i=1}^{400}\E(X_i)  = \frac{400}{500} = 0.8
\]

\[
\Var(X) = \Var\left(\sum\limits_{i=1}^{400}X_i\right) = \sum\limits_{i=1}^{400}\Var(X_i) = 400\cdot\frac{499}{500^2} = 0.8\cdot\frac{499}{500}
\]

Теперь мы знаем, что $\lambda = \E(X) = 0.8$ поэтому можем вернуться к пункту (а):
\[
\P(X \geqslant 2) = 1 - \P(X=0) - \P(X=1)  = 1 - \frac{0.8^0}{0!}e^{-0.8} - \frac{0.8^1}{1!}e^{-0.8} = 0.1912079
\] \vspace{-1.2cm}

\hspace{13cm} \fcolorbox{ForestGreen}{white}{So close!}

Осталось найти наиболее вероятное число опечаток на 13 странице:
\[
\P(X=k) = \frac{0.8^k}{k!}e^{-0.8} \rightarrow \max \limits_k
\]
Очевидно, что эта функция убывает по $k$, ведь с ростом $k$:\\
 $k!$ растет, а $0.8^k$ убывает. Значит наиболее вероятное число ошибок — $X = 0$


\item \href{https://en.wikipedia.org/wiki/Triskaidekaphobia}{Ох уж эти предрассудки!}
13-я страница точно такая же как и все остальные, ведь везде в решении можно просто
заменить номер 13 на любой другой и ничего не изменится.
\begin{center}
\includegraphics[width=6cm]{images/13}
\end{center}
\end{enumerate}

\item
Пусть $A = \{\text{«Лекция полезна»}\}$, $B = \{\text{«Лекция интересна»}\}$.
Заметим, что лекции вообще независимы друг от друга.

\begin{enumerate}
\item Пусть $X_A$ — число полезных лекций, прослушанных Васей,  $X_B$ —
число интересных лекций, прослушанных Васей. Введем случайную величину:
\[X_i =
\begin{cases}
1 & \text{если } i\text{-ая лекция была полезна}\\
0 & \text{если нет}
\end{cases}
\]

Тогда $X_A = \sum\limits_{i=1}^{30}X_i$. Рассмотрим отдельно $X_i$:

\begin{center}
\begin{tabular}{@{}ccc@{}}
\toprule
$x$         & $1$             & $0$               \\
$\P(X=x)$ & $0.9$ & $0.1$ \\ \bottomrule
\end{tabular}
\end{center}

Вероятность $0.9$ дана. Тогда:
\[
\E(X_i) = 0.9 = \E(X^2_i) \Rightarrow
\]
\[
\Var(X_i) = \E(X^2_i) - (\E(X_i))^2 = 0.9 - 0.9^2 = 0.09
\]

Значит,
\[
\E(X_A) = \E\left(\sum\limits_{i=1}^{30}X_i\right) = \sum\limits_{i=1}^{30}\E(X_i)  =
0.9\cdot30 = 27
\]
\[
\Var(X_A) = \Var\left(\sum\limits_{i=1}^{30}X_i\right) = \sum\limits_{i=1}^{30}\Var(X_i) =
0.09\cdot30 = 2.7
\]

Аналогично для числа интересных лекций можем получить:
\[
\E(X_B) = 0.7\cdot 30 = 21
\]
\[
\Var(X_B) = 0.21\cdot 30 = 6.3
\]


\item Так как интересность и полезность — независимые свойства лекций, то:

$\P(\bar{A} \cap \bar{B}) = \P(\bar{A})\cdot \P(\bar{B}) = 0.3\cdot0.1 = 0.03$, где $\bar{A}$ значит «не $A$». В свою очередь:\\
 $\P(A\cup B) = \P(A\cap\bar{B}) + \P(B\cap\bar{A}) + \P(A\cap B) = 1 - \P(\bar{A})\cdot \P(\bar{B}) = 0.97$ , где $(A\cup B)$ значит «$A$  или $B$». Аналогично, путем введения бинарной случайной величины можем получить:
 \[
 \E(X_{\bar{A} \cap \bar{B}}) = 0.03 \cdot  30 = 0.9
 \]
 \[
 \E(X_{A\cup B}) = 0.97\cdot30 = 29.1
\]
\end{enumerate}

\item
Будем пользоваться свойствами функций распределения и плотности. Для начала:
\[
\lim\limits_{x \rightarrow +\infty} F(x) = 1, \hspace{0.5cm} \lim\limits_{x \rightarrow -\infty} F(x) = 0,
\]
\[
\lim\limits_{x \rightarrow +\infty} \left(\frac{ae^x}{1+e^x}+b\right) = a+b := 1
\]
\[
\lim\limits_{x \rightarrow -\infty} \left(\frac{ae^x}{1+e^x}+b\right) = b :=0
\]
Откуда сразу получаем
\[
a =1, b = 0 \Rightarrow F(x) = \frac{e^x}{1+e^x}
\]
Для дальнейших развлечений нам понадобится функция плотности:
\[
f(x) = F'(x) = \frac{e^x}{(1+e^x)^2}
\]
Заметим, что она симметрична относительно нуля:
\[
f(-x) = \frac{\frac{1}{e^x}}{\left(1+\frac{1}{e^x}\right)^2} = \frac{e^x}{(1+e^x)^2} = f(x)
\]
Из того этого следует, что \textbf{математическое ожидание, а так же мода и медиана
равны нулю}. Более того, так как функция плотности симметрична относительно
\textbf{нулевого} математического ожидания, \textbf{центральный и начальный моменты
третьего порядка равны между собой и равны нулю.} Можно было выписать интегралы
для математического ожидания и третьего начального момента и сослаться на нечетность функции.

\begin{minipage}{0.6\textwidth}
\begin{center}
\includegraphics[scale=0.5]{auto_figures_tikz/2014_2015_fig_01_dlogis.pdf}
\end{center}
\end{minipage}
\end{enumerate}


\subsubsection*{Часть 2}

\begin{flushright}
 — Это невозможно! \\
— Нет. Это необходимо.\\
\textcopyright \hspace{0.1cm} Interstellar
\end{flushright}

\begin{enumerate}
\item
Алгоритм решения: рисуешь дерево $\rightarrow$ PROFIT

\begin{center}

 \begin{tikzpicture}[->,>=stealth',shorten >=1pt,auto,node distance=3cm,
  thick,main node/.style={circle,fill=blue!20,draw,font=\sffamily\Large\bfseries}]

  \node[main node] (1) {1};
  \node[main node] (2) [below of=1] {2};
  \node[main node] (3) [below of=2] {3};

  \path[every node/.style={font=\sffamily\small}]
    (1) edge [loop right] node {«не 6»} (1)
        edge [bend right] node[left] {«6»} (2)
    (2) edge [bend right] node[right] {«не 6»} (1)
        edge [above] node[left] {«6»} (3);
\end{tikzpicture}

\end{center}

Комментарии к построению дерева: состояние 1 — начальное, состояние 3 — конец игры,
когда выпало две «шестерки» подряд. Заметим, что выпадение любой «нешестерки» в
процессе игры приводит нас к состоянию, эквивалентному начальному.

Вероятность выпадения «шестерки» равна $1/6$, «нешестерки» — $5/6$.

Теперь мы готовы оседлать коня!

\begin{enumerate}
\item $\P(N = 1) = 0$ — невозможно за ход закончить игру.

$\P(N = 2) = \frac{1}{36}$

$\P(N = 3) = \frac{5}{6} \cdot \frac{1}{6} \cdot \frac{1}{6} = \frac{5}{216}$

\item А теперь будет видна вся сила рисования дерева:

Пусть $\E_1$ — число ходов, за которое мы ожидаем закончить игру, если игра начинается
в состоянии 1, $\E_2$ — число ходов, за которое мы ожидаем закончить игру, если игра
начинается в состоянии 2.

Получим два уравнения:

\[
\begin{cases} \E_2 = \frac{1}{6} \cdot 1 +  \frac{5}{6} (\E_1 + 1)   \\
\E_1 =  \frac{5}{6} (\E_1 + 1) + \frac{1}{6} ( \E_2 + 1) \end{cases}
\]

Решив эту систему, получим, что $\E_1 = 42$. А ведь это и есть $\E(N)$.

Аналогична логика для оставшихся математических ожиданий.

Найдем математическое ожидание суммы набранных очков. Ясно, что если выпадает «не 6»,
то мы ждем 3 очка. Тогда переопределив $\E_1$ и $\E_2$ следующим образом: пусть $\E_1$ —
число набранных очков, которое мы ожидаем получить за игру, если игра начинается
в состоянии 1, $\E_2$ — число набранных очков, которое мы ожидаем получить за игру,
если игра начинается в состоянии 2.

Новые два уравнения:

\[
\begin{cases} \E_2 = \frac{1}{6} \cdot 6 +  \frac{5}{6} (\E_1 + 3)   \\
\E_1 =  \frac{5}{6} (\E_1 + 3) + \frac{1}{6} ( \E_2 + 6) \end{cases}
\]

Решаем и получаем: $\E(S) = \E_1 = 147$

А можно было сделать еще круче! Выше показано, что  $\E(N) = 42$.
А сколько мы ждем очков за 1 ход? 3.5! Тогда $\E(S) = \E(N) \cdot 3.5 = 147$

Применяя схожую логику для $\E\left(N^2\right)$:

\[
\E(N^2) = \frac{5}{6} \cdot \E\left((N + 1)^2\right) + \frac{1}{6} \cdot \frac{5}{6}
\cdot \E\left((N + 2)^2\right) + \frac{1}{6} \cdot \frac{1}{6} \cdot 2^2
\]

Учитывая, что $\E(N) = 42$, получим: $\E(N^2) = 3414$.

\item Veni, vidi, vici
\begin{center}
  \begin{tabular}{@{}cc@{}}
  \toprule
  $x_n$       & $6$ \\
  $\P(X_n = x_n)$ & $1$ \\ \bottomrule
  \end{tabular}
\end{center}
\end{enumerate}

\item
\begin{enumerate}
\item $\P(V = 1) = 1/30$, так как именно этому равна вероятность того, что
Вовочка стоит ровно вторым в очереди;

$M = 1$ значит, что между Машенькой и Вовочкой ровно один человек в очереди.
Если Вовочка находится от 3 (включительно) до 28 позиции в очереди, то для
Машеньки есть две благоприятные позиции для события $M = 1$ (например, если
Вовочка стоит на 15 месте, то благоприятные позиции для Машеньки — стоять либо
13-ой, либо 17-ой). Если же Вовочка стоит на других позициях в очереди, то для
Машеньки существует ровно одна благоприятная позиция:

\[
\P(M = 1) = \frac{26}{30} \cdot \frac{2}{29} +  \frac{4}{30} \cdot \frac{1}{29} = \frac{56}{30\cdot 29} = \frac{28}{435}
\]

$M = V$ произойдет только, если Машенька стоит за Вовочкой. При этом для Машеньки
существует только одна благоприятная позиция и только в том случае, что Вовочка стоит
до 15 позиции (включительно):
\[
\P(M = V) = \frac{1}{2} \cdot \frac{1}{29} = \frac{1}{58}
\]

\item
\[
\E(V) = \frac{0 + 1 + \ldots + 29}{30} = \frac{30\cdot 14 + 15}{30} = 14.5
\]

Для $\E(M)$ можно решить в лоб, и получится красивая сумма, а можно вот так:

Сначала случайно кинем Вовочку и Машеньку на две из 30 позиций в очереди.
Образуется три отрезка: точки между Вовочкой и Машенькой и два крайних отрезка
(может быть, отрезок из 0 точек). Затем будем закидывать в очередь на оставшиеся
позиции случайно 28 оставшихся людей (назовем их «пропавшими»). Т.к. все броски
были случайны (или из соображений симметрии, как хотите), вероятность попасть в
отрезок между Машенькой и Вовочкой для «пропавшего» равна $1/3$, вне отрезка —
соответственно $2/3$, и независима от остальных бросков (!).

Введем случайную величину $X_i$ для $i$-го «пропавшего», которая равна $1$,
если он попал в отрезок между Машенькой и Вовочкой, $0$, если не попал:

\begin{center}
\begin{tabular}{ccc}
\toprule
$x_i$ & $1$ & $0$ \\
$\P(X_i = x_i)$ & $1/3$ & $2/3$ \\ \bottomrule
\end{tabular}
\end{center}

Легко считается: $\E(X_i) = 1/3$, $\E(X^2_i) = 1/3$, $\Var(X_i) = 1/3 - 1/9 = 2/9$.
Ясно, что $M = \sum_1^{28} X_i$. Тогда учитывая независимость $X_i$:

\[
\E(M) = \frac{28}{3}
\]
\[
\Var(M) = \frac{56}{9}
\]
\end{enumerate}

\item
Биномиальное распределение — \textit{À l’abordage!}.

Задача интерпретируется так: последний ход — это когда мы обратились к коробку,
в котором нет спичек (то есть к одному коробку нужно обратиться $n+1$ раз).

\begin{enumerate}
\item Пусть $\xi$ — это случайная величина, обозначающая число оставшихся спичек
в непустом коробке перед последним ходом.

Если $0<k \leqslant n$, будем считать успехом — попадание в коробок, к которому
мы на последнем ходу игры (пустому коробку) обратились. До этого момента из него
было вытащено $n$ спичек, а из другого $n-k$ спичек, то есть спички брались $2n - k$ раз.
Таким образом, перед последним ходом произошло $n$ успехов и $n-k$ неудач.
\[
\P(\xi = k) = C_{2n-k}^{n-k} = \left(\frac{1}{2}\right)^{n-k} \left(\frac{1}{2}\right)^{n} =
C_{2n-k}^{n-k} \left(\frac{1}{2}\right)^{2n-k}
\]
Теперь нужно учесть, что на последнем ходе был выбран именно пустой коробок.
Вероятность этого события — $1/2$, значит, искомая вероятность равна:
\[
\P(\text{в одном коробке осталось k спичек}) = C_{2n-k}^{n-k} \left(\frac{1}{2}\right)^{2n-k}
\cdot \frac{1}{2} = C_{2n-k}^{n-k} \left(\frac{1}{2}\right)^{2n-k+1}
\]
%Успехов — $n + 1$ (вытащено $n$ спичек, и на последнем ходу мы к нему обратились). По формуле Бернулли получаем следующее ($X$ — случайная величина, показывающая сколько спичек осталось в коробке, к которому мы не обратились на последнем ходу игры):
%\[\P(X = k) = C^{n+1}_{2n-k+1} \left(\frac{1}{2} \right)^{2n-k+1}\]

%Если $k = 0$, то мы вытащили все спички из обоих коробков к последнему ходу, и нам без разницы к какому коробку мы обратимся на последнем шагу, т.е.:
%\[\P(X = 0) =2 C^{n+1}_{2n+1} \left(\frac{1}{2} \right)^{2n+1}\]

\item Среднее спичек в другом коробке:

\[
\E(X) = \sum \limits_{k=1}^{n} k \cdot C^{n-k}_{2n-k} \left(\frac{1}{2} \right)^{2n-k+1}
\]

\end{enumerate}

\item
Для того чтобы количество упаковок, которые необходимо купить, равнялось 50,
нужно чтобы ни одну из наклеек Покупатель не встретил дважды, поэтому:
\[
\P(X=50) = 1\cdot\frac{49}{50}\cdot\frac{48}{50}\cdot\dots\cdot\frac{1}{50} =
\frac{49!}{50^{49}} \approx 3.4\cdot
10^{-21}
\] \vspace{-1cm}

\hspace{10.5cm}\fcolorbox{ForestGreen}{white}{Dum spiro, spero!}\footnote[2]{Надежда умирает последней!}

Теперь введём понятие «шаг». Переход на новый шаг происходит в тот момент,
когда покупатель получил наклейку, которой у него раньше не было. Начинаем с шага 0,
когда нет ни одной наклейки, и шагать будем до 49, потому что в момент перехода
на шаг 50 Покупатель получит последнюю необходимую наклейку и «прогулка» закончится.
Введём случайную величину $X_q$ равную количеству покупок в течение шага номер $q$.
Тогда $X = \sum \limits_{q=0}^{49}X_q$.  Найдём математическое ожидание $X_q$:
\[
\E(X_q) = \frac{n-q}{n}\cdot 1 + \frac{q}{n}\cdot\frac{n-q}{n}\cdot 2
+ \left(\frac{q}{n}\right)^2\cdot\frac{n-q}{n}\cdot 3 + \ldots
\]
здесь $\frac{n-q}{n}$ —  это вероятность найти наклейку, которой ещё нет,
а $\frac{q}{n}$, соответственно — вероятность повториться. Вопрос теперь в том,
как посчитать сумму:
\[
\E(X_q) = \frac{n-q}{n}\left( 1 + \frac{q}{n}\cdot 2 + \left(\frac{q}{n}\right)^2 \cdot 3
+ \ldots\right) = \frac{n-q}{n}\cdot\sum\limits_{k=0}^{\infty}\left(\frac{q}{n}\right)^k(k+1)
\]

Можем выписать в столбик несколько первых членов вышестоящей суммы:
\[
\begin{array}{l}
\hspace{0.3cm}1 \\
\vspace{0.2cm}
\left(\frac{q}{n}\right)^1  + \left(\frac{q}{n}\right)^1  \\
\vspace{0.2cm}
\left(\frac{q}{n}\right)^2 + \left(\frac{q}{n}\right)^2  + \left(\frac{q}{n}\right)^2 \\
\left(\frac{q}{n}\right)^3 + \left(\frac{q}{n}\right)^3 + \left(\frac{q}{n}\right)^3 + \left(\frac{q}{n}\right)^3 \\
\hspace{0.1cm}\cdots\cdots\cdots\cdots\cdots\cdots\cdots\cdots\cdots\cdots\cdots
\end{array}
\]
Достаточно! Можем скомпоновать всю сумму другим способом, а именно — по столбцам.
Заметим, что сумма элементов в каждом столбце это сумма бесконечно убывающей
геометрической прогрессии с одним и тем же знаменателем $\frac{q}{n}$ и различными
первыми членами. Соответственно:

\begin{align*}
\sum\limits_{k=0}^{\infty}\left(\frac{q}{n}\right)^k(k+1) &= \frac{1}{1-\frac{q}{n}}
+ \frac{\frac{q}{n}}{1-\frac{q}{n}} + \frac{\left(\frac{q}{n}\right)^2}{1-\frac{q}{n}}
+ \frac{\left(\frac{q}{n}\right)^3}{1-\frac{q}{n}} + \dots = \\
&= \frac{1}{1-\frac{q}{n}}\left( 1 + \frac{q}{n} + \left(\frac{q}{n}\right)^2
+ \left(\frac{q}{n}\right)^3 + \dots\right) = \frac{n}{n-q}\cdot\frac{n}{n-q}
= \left( \frac{n}{n-q}\right)^2
\end{align*}

Таким образом, получаем, что:
\[
\E(X_q) = \frac{n-q}{n}\cdot \left( \frac{n}{n-q}\right)^2 = \frac{n}{n-q}
\]
и это верно для любого q!

\begin{align*}
\E(X) &= \E\left(\sum \limits_{q=0}^{49}X_q\right) = \sum \limits_{q=0}^{49}\E(X_q) =
\frac{50}{50-0} + \frac{50}{50-1} + \dots + \frac{50}{50-49} \\
&= 50\left(\frac{1}{50} + \frac{1}{49} + \dots + 1\right)
\approx 50\int\limits_{1}^{50}\frac{1}{x}\mathbf{d}x = 50\ln(50) \approx 195.5
\end{align*}

А теперь ещё одно решение:


Величины $X_q$ независимы (но по разному распределены). Если долго пришлось ждать
$i$-го шага, это ничего не говорит о $j$-ом шаге. Величины $X_q$ имеют известный
закон распределения — это число опытов до первого успеха при заданной вероятности
успеха. Это геометрическое распределение, математическое ожидание которого равно
$\frac{1}{p}$, а дисперсия: $\frac{1-p}{p^2}$, где $p$ — \vspace{0.2cm} вероятность
успеха.

А те, кто забыл, могут \textbf{проще решить} методом первого шага:
если $X$ — число опытов до успеха при вероятности успеха $p$, то
\[
\E(X)=p\cdot 1 + (1-p)\cdot \E(X+1)
\]
Откуда $\E(X) = 1/p$ и дело в шляпе :)
Аналогично:
\[
\E(X^2)=p\cdot 1^2 + (1-p) \cdot \E((X+1)^2)
\]
и решая, находим $\E(X^2)$.

\item
\begin{enumerate}
\item Необходимое и достаточное условие — старушка не должна занять чужое место.
С вероятностью $1/n$ она угадает свое место, значит, для каждого входящего
его место будет свободно и он туда сядет.

\textbf{Ответ:} $1/n$
\item Будем искать вероятность того, что последний человек не сядет на свое место.

Пусть $A_i = \{\text{Старушка села на  место } i\text{-го} \}$,
$B_{(i,j)} = \{i \text{-ый  пассажир сел на место } j\text{-ого} \}$

\begin{align*}
P(n\text{-ый не сядет на свое место}) &= \P(A_n) + P(A_{n-1})P(B_{(n-1,n)}) + \\
&+ P(A_{n-2})(P(B_{(n-2,n)})
+ P(B_{(n-2,n-1)})P(B_{(n-1,n)}))) + \dots
\end{align*}

Можем заметить, что:
\begin{itemize}
\item $\P(A_i) = P(A_j) = \frac{1}{n}$ $\forall \hspace{0.1cm} i, j$
\item $\P(B_{(n-1,n)}) = \frac{1}{2}$, потому что $n-1$-ый выбирает из двух оставшихся мест
\item $\P(B_{(n-2,n)})
+ \P(B_{(n-2,n-1)}]P[B_{(n-1,n)})  = \frac{1}{3} + \frac{1}{3}\cdot \frac{1}{2} = \frac{1}{2}$
\item
\begin{align*}
\P(B_{(n-3,n)}) &+ \P(B_{(n-3,n-2)})(\P(B_{(n-2,n)})+\P(B_{(n-2,n-1)})\P(B_{(n-1,n)})) \\
&+ \P(B_{(n-3,n-1)})\P(B_{(n-1,n)}) = \frac{1}{4}+\frac{1}{4}\left(\frac{1}{3}+\frac{1}{3}\cdot\frac{1}{2}\right) +\frac{1}{4}\cdot \frac{1}{2} = \frac{1}{2}
\end{align*}
\item И так далее до того момента, пока старушка не сядет на место первого человека,
который заходит после нее, — всего $n-2$ вариантов.
\end{itemize}

Таким образом мы получаем сумму:
\[
\P[n\text{-ый не сядет на свое место}] = \frac{1}{n} + \frac{1}{n}\cdot \frac{1}{2}
+ \frac{1}{n}\cdot \frac{1}{2} + \dots = \frac{1}{n}  + \frac{1}{2n}(n-2) = \frac{1}{2}
\]
Значит,
\[
\P[n\text{-ый сядет на свое место}] = 1 - \frac{1}{2} = \frac{1}{2}
\]


А вот ещё один вариант решения:

\underline{Метод математической индукции:} допустим что это утверждение доказано
для одного, двух и так далее до $k$ человек. Рассмотрим $k+1$ человека. Когда
последний сядет на своё место? Если старушка сядет на своё место, а вероятность
этого равна $\frac{1}{k+1}$ или, с вероятностью $\frac{1}{2}$ (по индукции),
если старушка сядет на любое место кроме своего и последнего, то есть
$\frac{1}{2}\cdot\frac{k-1}{k+1}$. В этом случае тот\vspace{0.2cm} пассажир,
чьё место  она заняла, становится старушкой, и мы получаем задачу при меньшем $k$.
Складывая эти две дроби, получаем $\frac{1}{2} $.

Чтобы найти среднее число пассажиров, разобьем эту величину в сумму индикаторов:
$Y_1$ — сел ли первый на место, $\dots$, $Y_n$ — сел ли $n$-ый на место
(индикатор равен единице, если сел).

Стало быть $\E(Y)=\E(Y_1)+\E(Y_2)+\ldots+\E(Y_n)$. $\E(Y_n)=\frac{1}{2}$.

Почти аналогично можем рассуждать для предпоследнего:

База индукции: если пассажиров трое ($n=3$ включая старушку), то для предпоследнего
вероятность сесть на своё место равна $\frac{2}{3}$.

Шаг индукции: допустим что для $3, 4, \ldots n$ пассажиров эта вероятность равна
$\frac{2}{3}$.
Рассмотрим случай $(n+1)$-го пассажира.
Предпоследний сядет на своё место, если:

\renewcommand{\labelitemi}{\textbullet}

\begin{itemize}
\item старушка сядет на своё место или на место последнего $\frac{2}{n+1}$
\item в $\frac{2}{3}$ тех случаев, когда старушка сядет на место $2, 3, \ldots, (n-1)$,
то есть $\frac{2}{3}\cdot \frac{n-2}{n+1}$ складываем, получаем $\frac{2}{3}$.
То есть по индукции вероятность того, что предпоследний сядет на своё место равна
$\frac{2}{3}$.
\end{itemize}
И по аналогии можно увидеть, что вероятность того, что $k$-ый с конца пассажир
сядет на своё место равна $k/(k+1)$.

Если у нас $n$ пассажиров включая СС, то среднее количество севших на свои места
(раскладывая с конца) равно
\[
\frac{1}{2}+\frac{2}{3}+\frac{3}{4}+\dots+\frac{n-1}{n}+\frac{1}{n}
\]
\end{enumerate}
\end{enumerate}



\subsection[2013-2014]{\hyperref[sec:kr_01_ip_2013_2014]{2013-2014}}
\label{sec:sol_kr_01_ip_2013_2014}


\subsubsection*{Часть 1}

\begin{enumerate}
\item
\begin{enumerate}
\item Запишем все благоприятные исходы в таблицу:

\begin{center}
\begin{tabular}{@{}ll@{}}
\toprule
Исход & Вероятность             \\ \midrule
ООО   & $p^2 \cdot \frac{1}{2}$ \\
ООН   & $p^2 \cdot \frac{1}{2}$ \\
ОНО   & $ p(1-p)\frac{1}{2}$    \\
НОО   & $ (1-p)p\frac{1}{2}$    \\ \bottomrule
\end{tabular}
\end{center}

Нас устраивает любой из этих исходов, так что
\[
\P(\text{жюри одобрит конкурсанта}) = p^2 \cdot \frac{1}{2} \cdot 2 +
p(1-p)\frac{1}{2} \cdot 2 = p
\]
\item Исходя из результата предыдущего пункта, получаем, что конкурсанту безразлично.
\end{enumerate}

\item Введём обозначения:
\begin{itemize}
\item $\P(\text{В} | \text{A}^{c} \cap \text{М}^{c}) = p$ — Вася пришёл, а девушки — нет.
\item $\P(\text{В} | \text{A} \cap \text{М}) = 5p$ — пришли и Вася, и девушки.
\item $\P(\text{В} | \text{A}^{c} \cap \text{М}) = 3p$ — Вася пришёл, если пришла только Маша.
\item $\P(\text{В} | \text{A} \cap \text{М}^{c}) = 2p$ — Вася пришёл, если пришла только Алёна.
\item $\P(\text{М}) = 0.6$ — Маша пришла на лекцию.
\item $\P(\text{А}) = 0.3$ — Алёна пришла на лекцию.
\end{itemize}
\begin{enumerate}
\item По теореме умножения:
\[
\P(\text{А} | \text{В}) = \frac{\P(\text{А} \cap \text{В})}{\P(\text{В})}
\]
Выпишем числитель:
\begin{align*}
\P(\text{В} | \text{A}) \cdot \P(\text{А}) &= P(\text{В} | \text{A} \cap \text{М}) \cdot \P(\text{А}) \cdot \P(\text{М}) + \P(\text{В} | \text{A} \cap \text{М}^{c}) \cdot \P(\text{А}) = \cdot \P(\text{М}^{c}) \\
&= 5p \cdot  0.6 \cdot 0.3 \cdot 0.6 + 2p \cdot 0.36 \cdot 0.4 \cdot 0.3 = 1.14p
\end{align*}
И знаменатель:
\begin{align*}
\P(\text{В} | \text{A}^{c} \cap \text{М}^{c}) \cdot \P(\text{A}^{c} \cap \text{М}^{c}) &+ \P(\text{В} | \text{A} \cap \text{М}) \cdot \P(\text{A} \cap \text{М}) + \P(\text{В} | \text{A}^{c} \cap \text{М}) \cdot \P(\text{A}^{c} \cap \text{М})+ \\
&+  \P(\text{В} | \text{A} \cap \text{М}^{c}) \cdot \P(\text{A} \cap \text{М}^{c}) = p \cdot 0.4 \cdot 0.7 + 5p \cdot 0.6 \cdot 0.3 \\
&+ 3p \cdot 0.6 \cdot 0.7 + 2p \cdot 0.4 \cdot 0.3 = 2.68p
\end{align*}
Ответ:
\[
\P(\text{A} | \text{В} ) = \frac{\P(\text{A} \cap \text{В})}{\P(\text{В})} = \frac{1.14 p}{2.68p}  \approx 0.43
\]
\item Теперь необходимо найти
\[
\P(\text{М} | \text{В}) = \frac{\P(\text{М} \cap \text{В})}{\P(\text{В})}
\]
Знаменатель этой дроби посчитан в предыдущем пункте, посчитаем числитель:
\begin{align*}
\P(\text{М} \cap \text{В}) &= \P(\text{В} | \text{М}) \cdot \P(\text{М}) = P(\text{В} | \text{М} \cap \text{А}) \cdot \P(\text{А}) \cdot \P(\text{М}) \\
&+ \P(\text{В} | \text{A}^{c} \cap \text{М}) \cdot \P(\text{А}^{c})  \cdot \P(\text{М}) = 5p \cdot 0.6 \cdot 0.3 + 3p \cdot 0.6 \cdot 0.7 = 2.16p
\end{align*}
Ответ:
\[
\P(\text{М} | \text{В}) = \frac{\P(\text{М} \cap \text{В})}{\P(\text{В})} = \frac{2.16p}{2.68p} \approx 0.8
\]
Если Вася на лекции, вероятность застать на ней Машу выше.
\item $\P(\text{В}) = 0.5$, $ \P(\text{В}) = 2.68 p \Rightarrow p \approx 0.19$
\end{enumerate}
\item
\begin{enumerate}
\item Перед нами биномиальное распределение! Пусть $X$ — случайная величина, число туристов, которые не выехали за границу. Тогда:
\[
\P(X=5) = C_{100}^{5} \cdot 0.002^{5} \cdot 0.998^{95}
\]
\item
\begin{itemize}
\item $\E(X) = 2$
\item $\Var(X) = 0.2 \cdot 0.998$
\item Наиболее вероятное число невыехавших — 0.
\end{itemize}
\item Пусть случайная величина $S_i$ обозначает страховые выплаты, которые может
получить один турист. Она может принимать значение $0$, если турист выехал за границу
и не обратился за медицинской помощью, $ 2000$, когда он не выехал и $3000$,
когда турист выехал за границу и обратился за медицинской помощью. Тогда $S_i$
распределена следующим образом:
\begin{center}
\begin{tabular}{@{}lccc@{}}
\toprule
$s_i$       & $0$                & $2000$  & $3000$             \\ \midrule
$\P(S_i = s_i)$ & $0.998 \cdot 0.99$ & $0.002$ & $0.998 \cdot 0.01$ \\ \bottomrule
\end{tabular}
\end{center}
\begin{itemize}
\item $\E(S_i) = 2000 \cdot 0.002 + 3000 \cdot 0.998 \cdot 0.01 = 33.94 \Rightarrow \E(S) = 3394$
\item $\E(S_i^2) = 2000^2 \cdot 0.002 + 3000^2 \cdot 0.998 \cdot 0.01 = 97820 $
\item $\Var(S_i) = 97820 - 33.94^2 = 96668 \Rightarrow \Var(S) = 9666800$
\end{itemize}
\item
\end{enumerate}
\item
\begin{enumerate}
\item $\E(Y - 2X - 3) = \E(Y) - 2 \E(X) - 3 = 0$

$\Var(Y - 2X - 3) = \Var(Y) + 4\Var(X) - 2\Cov(Y, 2X) = 16$

$\Cov(X, Y) = \Corr(X,Y) \cdot \sqrt{\Var(X) \cdot \Var(Y)} = 6$
\item $\Corr(Y - 2X - 3, X) = \frac{\Cov(Y, X) - 2 \Var(X)}{\sqrt{\Var(Y - 2X - 3) \cdot \Var(X)}} = -1$, или проще: можно было заметить, что случайные величины линейно связаны.
\item Корреляция равна 1, значит, есть линейная взаимосвязь между переменными. Пусть $Y+ a X = b$, тогда $\Var(Y+ a X)=0$, $\E(Y) = -a + b =1 $. Решая уравнения, находим, что $a=-2/3, b=1/3$.
\end{enumerate}
\item
\begin{enumerate}
\item Частные распределения:

\begin{tabular}{@{}lccl@{}}
\toprule
$x$         & $-1$  & $0$   & $1$   \\ \midrule
$\P(X=x)$ & $0.3$ & $0.3$ & $0.4$ \\ \bottomrule
\end{tabular}
\hspace{1cm}
\begin{tabular}{@{}lcc@{}}
\toprule
$y$         & $-1$  & $1$   \\ \midrule
$\P(Y=y)$ & $0.5$ & $0.5$ \\ \bottomrule
\end{tabular}
\item
\begin{align*}
\Cov(X, Y) &= \E(XY) - \E(X)\E(Y) = (-1) \cdot (-1) \cdot 0.1 + (-1) \cdot 1 \cdot 0.2 + 1 \cdot (-1 ) \cdot 0.2 \\
 &+ 1 \cdot 1 \cdot 0.2 - ((-1) \cdot 0.3 + 1 \cdot 0.4) (-1\cdot 0.5 + 1 \cdot 0.5) = -0.1
\end{align*}
\item Да, так как $\Cov(X, Y) \neq 0$
\item Необходимо минимизировать дисперсию дохода:
\[\Var(\alpha X + (1- \alpha)Y) \to \min_{\alpha} \]
\begin{align*}
\Var(\alpha X + (1 - \alpha)Y)  &= \alpha^2 \Var(X) + (1 - \alpha)^2 \Var(Y) + 2 \alpha(1 - \alpha)\Cov(X, Y) \\
&= 0.69 \alpha^2  + (1 - \alpha)^2 -0.2 \alpha(1 - \alpha) \to \min_{\alpha}
\end{align*}
\[
\frac{\partial \Var(\alpha X + (1- \alpha)Y)}{\partial \alpha} = 2 \cdot 0.69 \alpha - 2(1-\alpha) -0.2 + 0.4 \alpha = 0 \Rightarrow
\alpha \approx 0.58
\]
\item Условное распределение:

\begin{center}
\begin{tabular}{@{}lclc@{}}
\toprule
$x$ & $-1$  & $0$   & $1$   \\ \midrule
$\P(X = x\mid Y=-1)$    & $0.2$ & $0.4$ & $0.4$ \\ \bottomrule
\end{tabular}
\end{center}

\item $\E(X \mid Y=-1) = -1 \cdot 0.2 + 1 \cdot 0.4 = 0.2$
\end{enumerate}
\end{enumerate}

% !TEX root = ../probability_hse_exams.tex
% \newpage
\thispagestyle{empty}
\section{Решения контрольной работы 2}


\subsection[2022-2023]{\hyperref[sec:sol_kr_02_2022_2023]{2022-2023}}
\label{sec:sol_kr_02_2022_2023}



\subsection[2021-2022]{\hyperref[sec:sol_kr_02_2021_2022]{2021-2022}}
\label{sec:sol_kr_02_2021_2022}

\begin{enumerate}
	\item $F_Z(t) = \frac{F(t-1) + F(t) + F(t+1)}{3}$, где $F(t)$ — функция распределения стандартной нормальное случайной величины.
	Аналогичное соотношение выполняется и для функции плотности:
	\[
	f_Z(t) = \frac{f(t-1) + f(t) + f(t+1)}{3}.
	\]
	Ожидание и дисперсия находятся без нахождения плотности или распределения:
	\[
	\E(Z) = \E(X) + \E(Y) = 0; \quad \Var(Z) = \Var(X) + \Var(Y) + 0 = 1 + 2/3.  
	\]
	\item 
	\item 
  \end{enumerate}
  

  \textbf{Случайная вариация задачи 1}:

  Складываемые величины симметрично распределены относительно $0.5$, поэтому сумма распределена симметрично относительно $1$. 
  Достаточно найти функцию плотности суммы на участке $z\in [0;1]$, а в силу симметрии получим $f_Z(1- t) = f_Z(1 + t)$ 
  или $f_Z(2-z) = f_Z(z)$.

  \begin{enumerate}
    \item $\E(X+Y) = 0.5 + 0.5 = 1$;
    \item $\Var(X+Y) = 1/20 + 1/20 = 0.1$;
    \item $f_{X+Y}(z) = \begin{cases}
	\frac{6}{5}z^5 - 6z^4 + 6z^3, \text{ если } z \in [0;1], \\
	\frac{6}{5}(2-z)^5 - 6(2-z)^4 + 6(2-z)^3, \text{ если } z \in [1;2], \\
	0, \text{ иначе.} \\
	\end{cases}$.
\end{enumerate}


\textbf{Вариация задачи 3 для ИП}:





\subsection[2020-2021]{\hyperref[sec:sol_kr_02_2020_2021]{2020-2021}}
\label{sec:sol_kr_02_2020_2021}

\begin{enumerate}
	\item 
	\begin{enumerate}
		\item 
		\[
			\plim_{n\to\infty} \frac{X_n}{n} = 0;
		\]
		\item 
		\[ 
			\plim_{n\to\infty} \frac{X_1 + X_2 + \ldots + X_{10}}{n} = 0;
		\]
		\item 
		\[
			 \plim_{n\to\infty} \left(\frac{X_1 + X_2 + \ldots + X_n}{n}\right)^2 = (\E(X_1))^2 = 1;
		\]
		
		\item 
		\[
			 \plim_{n\to\infty} \left(\frac{X_1^2 + X_2^2 + \ldots + X_{n}^2}{n} - \frac{X_1 + X_2 + \ldots + X_{n}}{n}\right) = \E(X_1^2) - \E(X_1) = 1/3.
		\]
	\end{enumerate}

\end{enumerate}


\subsection[2019-2020]{\hyperref[sec:kr_02_2019_2020]{2019-2020}}
\label{sec:sol_kr_02_2019_2020}

\begin{enumerate}
\item 

\begin{enumerate}
\item Поскольку рассматриваемые случайные величины являются равномерными, 
то их функции плотности имеют следующий вид:

\[
f_{\xi_{1}}(x)=f_{\xi_{2}}(x)=
\begin{cases}
\frac{1}{2}, \text{ если }x\in[0,2]\\
0, \text{ иначе}
\end{cases}.
\]

Поскольку данные случайные величины независимы, 
то их совместная функция плотности может быть рассчитана как произведение частных функций плотности. 
Полагая 
$\xi=
\begin{pmatrix}
\xi_{1} \\ 
\xi_{2}
\end{pmatrix}
$, 
имеем:

\[
f_{\xi}(x,y)=f_{\xi_{1}}(x)f_{\xi_{2}}(y)=
\begin{cases}\frac{1}{4}, \text{ если } x\in[0,2], y\in[0,2]\\
0\text{, иначе}
\end{cases}
\]

\item Плотность суммы независимых случайных величин можно найти при помощи формулы свертки:

\[
f_{\xi_{1}+\xi_{2}}(t)=\int_{-\infty}^{\infty}f_{\xi_{1}}(x)f_{\xi_{2}}(t-x)\, dx
\]


\[
f_{\xi_{1}+\xi_{2}}(t)=
\begin{cases}
\int_{0}^{t} \frac{1}{4} \, dx = \frac{t}{4}, \text{ если } t \in [0,2] \\ 
\int_{t-2}^{2} \frac{1}{4} \, dx = 1-\frac{t}{4}, \text{ если } t \in [2,4] \\
0, \text{иначе}
\end{cases}.
\]

\item Полученная функция плотности на интервале $t \in [0,4]$ зависит от значения параметра $t$, 
поэтому мы имеем дело не с равномерным распределением. 
Равномерное распределение неустойчиво относительно суммирования. 

\item Учитывая независимость рассматриваемых случайных величин, получаем:

\[
\E(\xi_{1}-\xi_{2}) = \E(\xi_{1})-\E(\xi_{2}) = \frac{2+0}{2} - \frac{2+0}{2} = 0
\]

\[
\Var(\xi_{1}-\xi_{2}) = \Var(\xi_{1}) + \Var(\xi_{2}) - 2\Cov(\xi_{1}, \xi_{2}) = 
\frac{(2-0)^2}{12} + \frac{(2-0)^2}{12} - 2 \cdot 0 = \frac{2}{3}
\]

Для нахождения математического ожидания $\E((\xi_1 - \xi_2)^{2019})$ достаточно заметить, 
что распределение разницы $(\xi_1 - \xi_2)$ симметрично вокруг нуля.  
Следовательно, любой нечётный начальный момент разницы будет обращаться в ноль. 
Можно и интеграл взять, если не удалось заметить симметрию. 

\item Используя свойства ковариации и дисперсии, получаем:

\[
	\Corr(\xi_{1}-\xi_{2},\xi_{1}) 
	= 
	\frac{\Var(\xi_{1}) - \Cov(\xi_{2},\xi_{1})}{\sqrt{\Var(\xi_{1}-\xi_{2})\Var(\xi_{1})}}
	=
	\frac{\Var(\xi_1)}{\sqrt{2\Var(\xi_1) \Var(\xi_1)}} 
	= 
	\frac{1}{\sqrt{2}}
\]

\item Ковариация линейная по каждому аргументу, поэтому:

\[
	\Cov(\xi_{1} - \xi_{2}, \xi_{1} + \xi_{2}) =
	\Var(\xi_{1}) - \Var(\xi_{2}) = 
	\frac{(2-0)^2}{12}-\frac{(2-0)^2}{12} = 0
\]

Данные случайные величины зависимы несмотря на то, что их ковариация равна нулю.
Например, знание того, что сумма $\xi_1 + \xi_2 \approx 4$ несёт в себе информацию,
что разность $\xi_1 - \xi_2 \approx 0$. 

\end{enumerate}



\item 
\begin{enumerate}

\item Введем обозначение $\xi=\begin{bmatrix}\xi_{1}\\ \xi_{2}\end{bmatrix}$. Обратим внимание, что при $x\geq y$ или $x,y\notin(0,1)$ функция плотности $\xi$ равняется нулю. Принимая во внимание это обстоятельство получаем, что:

\[f_{\xi_{2}}(y)=\int_{-\infty}^{\infty}f_{\xi}(x,y)dx=\begin{cases}\int_{0}^{y}3ydx=3y^2\text{, если }y\in[0,1]\\0\text{, в противном случае}\end{cases}\]

\item Ожидание равно:

\[
	\E(\xi_{2}^3)=\int_{0}^{1}y^3 \cdot 3y^2dy=\frac{1}{2}
\]

\item Находим интеграл

\[
	\int_{0}^{1}\int_{0}^{y}\left(\frac{x}{y}\cdot 3y\right)dxdy=\frac{1}{2}
\]

\item Условная функция плотности:

\[
	f_{\xi_{1}|\xi_{2}}(x|y)=\frac{f_{\xi}(x,y)}{f_{\xi_{2}}(y)}=
\begin{cases}
	\frac{3y}{3y^2}=\frac{1}{y}, \text{ если } 0 \leq x \leq y \leq 1 \\
	0, \text{ иначе}
\end{cases}
\]

\item Пользуясь полученным в предыдущем пункте результатом имеем:

\[
	\E(\xi_{1}|\xi_{2}=y)=\int_{0}^{y}x \cdot \frac{1}{y}dx=\frac{y}{2}
\]

\item По аналогии с предыдущим пунктом получаем:

\[
	\E(\xi_{1}|\xi_{2})=\int_{0}^{\xi_{2}}x*\frac{1}{\xi_{2}}dx=\frac{\xi_{2}}{2}
\]

Отсюда нетрудно найди функцию распределения условного математического ожидания:

\[
	F_{\E(\xi_{1}|\xi_{2})}(t)=P(E(\xi_{1}|\xi_{2})\leq t)=P\left(\frac{\xi_{2}}{2}\leq t\right)=P(\xi_{2}\leq 2t)=F_{\xi_{2}}(2t)
\]

Дифференцируя получаем функцию плотности:

\[
	f_{\E(\xi_{1}|\xi_{2})}(t)=\frac{dF_{\xi_{2}}(2t)}{dt}=2f_{\xi_{2}}(2t)=
\begin{cases}
2\cdot 3(2t)^2=24t^2, \text{ если } t \in (0,0.5) \\
0, \text{ иначе}
\end{cases}
\]

\item Ожидание от условного ожидания:

\[
	\E(\E(\xi_{1}|\xi_{2}))=\E(\xi_{1})=\int_{0}^{1}\int_{0}^{y} x \cdot 3y dxdy=\frac{3}{8}
\]

\item Функция плотности $\xi_{1}$ не будет совпадать с условной при некоторых значениях $\xi_{2}$, 
что доказывает отсутствие независимости:

\[
	f_{\xi_{1}}(0.5)=\int_{0.5}^{1}3y dy=\frac{3}{2}(1-0.5^2)=1.125\ne 
	2=f_{\xi_{1}|\xi_{2}=0.5}(0.5)
\]

\item Площадь под функцией плотности равна единице:

\[
	\int_{0}^{1}\int_{0}^{y}cx \cdot 3y=\frac{3}{8}c=1
\]

Отсюда следует, что $c=\frac{8}{3}$.

\end{enumerate}

\item 

\begin{enumerate}
\item Докажем, что данная последовательность сходится по вероятности к нулю:

\[
	\lim_{n\to\infty}\P(|\xi_{n}-0| > \epsilon) \leq \lim_{n\to\infty}\P(|\xi_{n}-0|>0)=
	\lim_{n\to\infty}\frac{1}{n} = 0
\]

\item Очевидно, что

\[
	\lim_{n\to\infty} \E(\xi_{n})=\lim_{n\to\infty}n^3 \cdot \frac{1}{n}+0\cdot (1-\frac{1}{n})=\lim_{n\to\infty} n^2=\infty
\]

\item Поскольку данная последовательность сходится к нулю по вероятности, то она сходится к нулю и по распределению.

\end{enumerate}

\item

\begin{enumerate}
\item Пусть случайная величина $X\sim \cBin(n=1000, p=0.01)$ отражает количество страховых случаев и имеет биномиальное распределение с параметрами $n=1000$ и $p=0.01$.

Оценим соответствующую вероятность при помощи неравенства Маркова:

\[
	\P(X\leq 20) = 1 - \P(X > 20) \geq 1 - \frac{\E(X)}{20} 
	= 1 - \frac{1000 \cdot 0.01}{20} = 0.5
\]

Теперь воспользуемся неравенством Чебышёва:

\[
	\P(X\leq 20) = \P(|X-10|\leq 10)
	= 1 - \P(|X - 10|\geq 10) \geq 1 - \frac{1000\cdot 0.01\cdot (1-0.01)}{10^2}
	= 0.901
\]

Пользуясь ЦПТ получаем, что:

\[
	X\dot\sim \cN(1000\cdot 0.01, 1000\cdot 0.01 \cdot (1-0.01)) \sim \cN(10, 9.9)
\]

В силу симметрии нормального распределения вокруг математического ожидания получаем, что:

\[
	\P(X\leq 20) \approx 0.999
\]

\item Найдем математическое ожидание и дисперсию выплат $Y$ в тысячах рублей:

\[
	\E(Y)=\E(25 X)=25\E(X)=25 \cdot 10=250
\]

\[
	\Var(Y)=\Var(25 X)=25^2 \Var(X) = 25^2\cdot 10 \cdot (1-0.01) = 6187.5
\]

Пользуясь ЦПТ получаем:

\[
	Y\dot\sim \cN(250, 6187.5)
\]

Необходимый резерв можно получить из решения неравенства:

\[
	\P(Y\leq S)=0.9
\]

Приступим к решению:

\[
	\P(Y\leq S)=\P\left(\frac{Y - 250}{\sqrt{6187.5}} \leq \frac{S-250}{\sqrt{6187.5}}\right)=0.9
\]

Находим соответствующую квантиль стандартного нормального распределения:

\[
	\frac{S-250}{\sqrt{6187.5}}\approx 1.28
\]

Решая, получаем, что $S \approx 350.8$.

\end{enumerate}

\end{enumerate}




\subsection[2018-2019]{\hyperref[sec:kr_02_2018_2019]{2018-2019}}
\label{sec:sol_kr_02_2018_2019}

\begin{enumerate}
\item
\begin{enumerate}
\item Таблица:
	
\begin{center}
\begin{tabular}{ccc}
	& $\xi = 1$ & $\xi = 0$  \\
	\hline
	$\eta = 1$ & 0 & 0.25 \\
	\hline
	$\eta = 0$ & 0.25 & 0.5 \\
	\hline
\end{tabular}
\end{center}
	
\item $\E(\xi) = 0.25 + 0 = 0.25$
	
$\E(\eta) = 0.25 + 0 = 0.25$
	
\item $\Cov(\xi, \, \eta) = \E(\xi\cdot \eta) - \E(\xi)\cdot \E(\eta) = 0 \cdot 1 + 0 \cdot 0,25 + 0 \cdot 0,2 + 0 \cdot 0,5 - 0,25 \cdot 0,25 = 0 - \frac{1}{16} = -\frac{1}{16}$
	
\item $\P\left(\xi = 1 | \eta = 0\right) = \frac{0,25}{0,75} = \frac{1}{3}$
	
$\P\left(\xi = 0 | \eta = 0\right) = \frac{0,5}{0,75} = \frac{2}{3}$
	
\item $\E\left(\xi | \eta = 0 \right) = \frac{1}{3} \cdot 1 + \frac{2}{3} \cdot 0 = \frac{1}{3}$
	
\item Таблица:
	
\begin{center}
\begin{tabular}{ccc}
	$u$ & 0 & $\frac{1}{3}$  \\
	\hline
	$\P(\E(\xi|\eta)=u)$ & 0.25 & 0.75 \\
	\hline
\end{tabular}
\end{center}
\end{enumerate}


\item
\begin{enumerate}
    \item Пусть $A$ — количество очков 1-го стрелка, $B$ — количество очков 2-го, тогда $\E(A) = 50 \cdot 0.5 = 25$, $\Var(A) = 50 \cdot 0.5 \cdot (1 - 0.5) = 12.5$.
        
    \[\P(A \ge 15) = \P\biggl(\frac{A - \E(A)}{\sqrt{\Var(A)}} \ge \frac{15 - \E(A)}{\sqrt{\Var(A)}}\biggl) 
    = \P\biggl(\frac{A - 25}{\sqrt{12.5}} \ge -2.83\biggl)\]
    
    Согласно ЦПТ, $\frac{A - 25}{\sqrt{12.5}} \sim \cN(0, 1)$, значит $\P\biggl(\frac{A - 25}{\sqrt{12.5}} \ge -2.83\biggl) = 0.9977$
    
    \item $\P(A > B) = \P(A - B > 0)$.~~Пусть $Z = A - B$.
    
    \[\E(Z) = \E(A) - \E(B) = 25 - 30 = -5\] 
    
    \[\Var(Z) = \Var(A) + \Var(B) - 2\Cov(A, B) = 12.5 + 12 = 24.5\] 
    
    \[\P(A - B > 0) = \P(Z > 0) = \P\biggl(\frac{Z + 5}{\sqrt{24.5}} > \frac{5}{\sqrt{24.5}}\biggl) = \P\biggl(\frac{Z + 5}{\sqrt{24.5}} > 1.01\biggl)\]
    
    Согласно ЦПТ, $\frac{Z + 5}{\sqrt{24.5}} \sim \cN(0, 1)$, значит $\P\Bigl(\frac{Z + 5}{\sqrt{24.5}} > 1.01\Bigl) = 0.1562$
\end{enumerate}

\item Известно:

$X$ — сколько рублей тратит Вася на обед

$\E(X) = 300$

$\sqrt{\Var(X)} = 30$


\begin{enumerate}
	\item Воспользуемся неравенством Маркова:
	
	\[
	\P(X \ge a) \le \frac{\E(X)}{a}
	\]
	
	\[
	\P(X \ge 1000) \le \frac{300}{1000} = 0.3
	\]
	
	\item Воспользуемся неравенством Чебышёва:
	
	\[
	\P(|X - \E(X)| \ge a) \le \frac{\Var(X)}{a^{2}}
	\]
	
	\[
	\P(|X - 300| \ge 100) \le \frac{900}{10000} = 0.09
	\]
\end{enumerate}

\item Пусть $T$ — это случайная величина равная $T_{1} + T_{2}$.	
\begin{enumerate}
	\item  
	Тогда по свойствам матожидания:
	\[
	\E(T) = \E(T_{1}) + \E(T_{2}) = \frac{0 + 20}{2} + \frac{10 + 20}{2} = 25
	\]
	Так как напомним, что  для $X\sim U(a, b)$ : $\E(X) = \frac{a + b}{2}$

	Тогда по свойствам дисперсии, учитывая факт независимости $T_{1}$ и $T_{2}$:
	\[
	\Var{(T)} = \Var{(T_{1} + T_{2})} = \Var{(T_{1})} + \Var{(T_{2})} = \frac{(20 - 0)^{2}}{12} + \frac{(20 - 10)^{2}}{12} = \frac{500}{12}
	\]
	\item Как это часто бывает удобно делать при работе с непрерывными случайными величинами, подойдем к вопросу подсчета вероятности с геометрической точки зрения. 
	В осях $T_{1}$ и $T_{2}$ изобразим множество всех возможных случаев и множество благоприятных случаев.
	Множество всех возможных исходов — это прямоугольник со сторонами длиной $10$ и $20$ минут. 
	Благоприятные исходы — это те, где в сумме на поездку потрачено меньше 15 минут, то есть $T_{1} + T_{2} < 15$. 
	На графике множество благоприятных исходов лежит внутри прямоугольного треугольника. 
	Находим отношение площадей
	\[
	\P[T_{1} + T_{2} < 15] = \frac{0.5 \cdot 5 \cdot 5}{10 \cdot 20} = 1/16 = 0.0625
	\] 
   \item Необходимо найти функцию плотности $T=T_1 + T_2$. Делаем это с помощью формулы свёртки:
	    
   \[
	   f_T(t)=\int_{-\infty}^{\infty}f_{T_{1}}(x)f_{T_{2}}(t-x)dx
	\]
   
   Нам достаточно брать соответствующий интеграл лишь по тем точкам, в которых произведение функций плотности под интегралом не обнуляется. Множество этих точек будет задавать для нас пределы интегрирования, а получить его можно из решения следующей системы уравнений для $x$:
   
   \[
	   \begin{cases}
				0 \leq x \leq 20 \\
				10 \leq t-x \leq 20 \\
		\end{cases}
\]
   
   Исходя из решения получаем, что:
   
\[
	\begin{cases}
		x \in [0, t-10] \text{, при } t \in [10, 20]\\ 
		x \in [t-20, t-10] \text{, при } t \in [20, 30]\\
	    x \in [t-20, 20] \text{, при } t \in [30, 40] \\
	\end{cases}
\]
   
   Таким образом искомая функция плотности принимает вид:
	\[
		f_T(t)=
			\begin{cases}
				\int_{0}^{t-10} \frac{1}{20} \frac{1}{10} \, dx = \frac{t-10}{200} \text{, при } t \in [10, 20] \\
				\int_{t-20}^{t-10} \frac{1}{20} \frac{1}{10} \, dx = \frac{1}{20} \text{, при } t \in [20, 30]\\ 
				\int_{t-20}^{20} \frac{1}{20} \frac{1}{10} \, dx = \frac{1}{5} - \frac{t}{200} \text{, при } t \in [20, 30]
			\end{cases}
	\]
	
	Функция плотности позволяет лишний раз проверить результат пункта б)
	
	\[
	    \P(T_{1}+T_{2} \le 15)= \int_{10}^{15} \frac{t-10}{200} \, dt = 0.0625
	\]
\end{enumerate}

\item 
\begin{enumerate}
	\item Пользуясь независимостью получаем:
	\[
	\E(\xi_{1}\xi_{2}\xi_{3})=\E(\xi_{1})\E(\xi_{2})\E(\xi_{3})=0 \cdot 0 \cdot 0=0
	\]
	\item По аналогии получаем:
	
	\[
	\E(\xi_{1}^2\xi_{2}^2\xi_{3}^2)=\E(\xi_{1}^2)\E(\xi_{2}^2)\E(\xi_{3}^2)=(\Var(\xi_{1})+\E(\xi_{1})^2)^3=(1+0^2)^3=1
	\]
	
	Отсюда следует, что:
	
	\[
	\Var(\xi_{1}\xi_{2}\xi_{3})=\E(\xi_{1}^2\xi_{2}^2\xi_{3}^2)-(\E(\xi_{1}\xi_{2}\xi_{3}))^2=1-0^2=1
	\]
	
	\item 
	\[
	\P(e^{\xi_1} < 1)=\P(\xi_{1}<\ln 1)=\P(\xi_1 < 0)=\frac{1}{2}
	\]
\end{enumerate}	


\item
\begin{enumerate}
	\item Ожидаемые доходности портфелей $A$ и $B$ — это $\E(R_A)$ и $\E(R_B)$. 
	Выпишем доходности портфелей, используя веса отдельных ценных бумаг, входящих в портфели:
	\[
		\E(R_A) = \E\Bigr(\frac{1}{2}R_1+\frac{1}{2}R_2\Bigl) = \frac{1}{2}\E(R_1) + \frac{1}{2}\E(R_2) = \frac{5}{2} + \frac{10}{2} = \frac{15}{2}
	\]
	\[
		\E(R_B) = \E\Bigr(\frac{1}{2}R_2+\frac{1}{2}R_3\Bigl) = \frac{1}{2}\E(R_2) + \frac{1}{2}\E(R_3) = \frac{10}{2} + \frac{15}{2} = \frac{25}{2}
	\]
	\item По условия, риски портфелей $A$ и $B$ определяются как стандартные отклонения $\sqrt{\Var(R_A)}$ и $\sqrt{\Var(R_B)}$. 
	Для портфеля $A$:
	\[
		\Var(R_A) = \Var{\Bigr(\frac{1}{2}R_1 + \frac{1}{2}R_2\Bigl)} = \frac{1}{4}\Var(R_1) + \frac{1}{4}\Var(R_2) + 2\Cov{\Bigr(\frac{1}{2}R_1, \frac{1}{2}R_2\Bigl)} 
	\]
	\[
		\sqrt{\Var(R_A)} =  \sqrt{\frac{50}{4} + \frac{100}{4} + \frac{2}{4}\Cov(R_1, R_2)} = \sqrt{\frac{190}{4}}
	\]
	И для $B$:
	\[
		\Var(R_B) = \Var{\Bigr(\frac{1}{2}R_2 + \frac{1}{2}R_3\Bigl)} = \frac{1}{4}\Var(R_2) + \frac{1}{4}\Var(R_3) + 2\Cov{\Bigr(\frac{1}{2}R_2, \frac{1}{2}R_3\Bigl)} 
	\]
	\[
		\sqrt{\Var(R_B)} =  \sqrt{\frac{100}{4} + \frac{150}{4} + \frac{2}{4}\Cov{(R_2, R_3)}} = \sqrt{\frac{230}{4}}
	\]
	\item Здесь всё как в жизни:
	портфель А имеет меньший риск и меньшую ожидаемую доходность, в то время как портфель B — большую ожидаемую доходность, а значит, больший риск.
	\item Посчитаем корреляцию:
	\[
		\Corr(R_A, R_B) = \frac{\Cov(R_A, R_B)}{\sqrt{\Var(R_A)}\cdot \sqrt{\Var(R_B)}} = 
		\frac{\Cov{\Bigr(\frac{1}{2}R_1 + \frac{1}{2}R_2,\frac{1}{2}R_2 + \frac{1}{2}R_3\Bigl)}}{\sqrt{\Var(R_A)}\cdot \sqrt{\Var(R_B)}} 
	\]
	\[
		\Cov{\Bigr(\frac{1}{2}R_1 + \frac{1}{2}R_2,\frac{1}{2}R_2 + \frac{1}{2}R_3\Bigl)} = 
		\Cov{\Bigr(\frac{1}{2}R_1,\frac{1}{2}R_2\Bigl)} + \Cov{\Bigr(\frac{1}{2}R_1, \frac{1}{2}R_3\Bigl)} + \Cov{\Bigr(\frac{1}{2}R_2,\frac{1}{2}R_2\Bigl)} + \Cov{\Bigr(\frac{1}{2}R_2, \frac{1}{2}R_3\Bigl)}
	\]
	Вспомним, что $\Cov(R_2,R_2) = \Var(R_2)$, и получим:
	\[
		\Cov{\Bigr(\frac{1}{2}R_1 + \frac{1}{2}R_2,\frac{1}{2}R_2 + \frac{1}{2}R_3\Bigl)} = \frac{20}{4} - \frac{10}{4} + \frac{100}{4} - \frac{10}{4} = 25
	\]
	Находим значение корреляции:
	\[
		\Corr(R_A, R_B) = \frac{25}{\sqrt{\frac{190}{4}\cdot \frac{230}{4}}} = \frac{100}{\sqrt{190} \cdot \sqrt{230}}
	\]
	\item {[непроверенное решение ассистента]} Для решения этого пункта достаточно найти хотя бы одно из решений следующей системы:
	\begin{equation*}
	\begin{cases}
	5w_{1} + 10w_{2} + 15w_{3} \ge \frac{15}{2}\\
	5w_{1} + 10w_{2} + 15w_{3} \ge \frac{25}{2}\\
	\sqrt{50w_{1}^{2} + 100w_{2}^{2} + 150w_{3}^{2} +2w_{1}w_{2}\cdot20 + 2w_{1}w_{3}\cdot(-10) + 2w_{2}w_{3}(-10)} \le \sqrt{\frac{230}{4}}\\
	\sqrt{50w_{1}^{2} + 100w_{2}^{2} + 150w_{3}^{2} +2w_{1}w_{2}\cdot20 + 2w_{1}w_{3}\cdot(-10) + 2w_{2}w_{3}(-10)} \le \sqrt{\frac{190}{4}}\\
	w_{1} + w_{2} + w_{3} = 1
	\end{cases}
	\end{equation*}
	Что эквивалентно:
	\begin{equation*}
	\begin{cases}
	5w_{1} + 10w_{2} + 15w_{3} \ge \frac{25}{2} (1)\\
	\sqrt{50w_{1}^{2} + 100w_{2}^{2} + 150w_{3}^{2} +2w_{1}w_{2}\cdot20 + 2w_{1}w_{3}\cdot(-10) + 2w_{2}w_{3}(-10)} \le \sqrt{\frac{190}{4}} (2)\\
	w_{1} + w_{2} + w_{3} = 1 (3)
	\end{cases}
	\end{equation*}
	(1) и (2) строчки вместе дают:
	\[w_{2} = 1.5 - 2w_{3}\]
	\[w_{1} = w_{3} - 0.5\]
	Осталось подставить во (2) и решить получившееся неравенство\ldots
	\[540w_{3}^{2} - 570w_{3} - 160 \ge 0\]
	Решая сначала квадратное уравнение и определяя знак интервалов, получаем:
	\[w_{3} \in \Bigr(-\infty, \frac{19}{36} - \frac{\sqrt{745}}{36}\Bigl]; \Bigl[\frac{19}{36} + \frac{\sqrt{745}}{36}, +\infty\Bigl)\]
	Это невозможно по экономическому смыслу, значит данный пункт не имеет решений.
\end{enumerate}

\item Можно использовать тождество Вальда, а можно решить и с помощью разложения в другую сумму. 
Если $X_i$ — время ответа $i$-го студента из 300, то оно может равняться нулю.
Заметьте, что $X_i \neq \tau_i$. А $X_i$ можно представить как произведение
$X_i = I_i R_i$, где $I_i$ равно нулю или единички, в зависимости от того, пришел ли студент на экзамен,
а $R_i$ имеет экспоненциальное распределение. 

\[
\E(T) = \E(X_1 + \ldots + X_{300}) = 300 \cdot (0.9\cdot 0 + 0.1 \cdot 0.5)
\]

\end{enumerate}

\subsection[2017-2018]{\hyperref[sec:kr_02_2017_2018]{2017-2018}}
\label{sec:sol_kr_02_2017_2018}

\begin{enumerate}

\item[7.]
\begin{enumerate}
\item Всем хватит места, если число явившихся на рейс пассажиров ($X$) не превысит $300$,
то есть нужно найти $\P(X \leq 300)$. Найдём матожидание и дисперсию
случайной величины $X$:
\begin{align*}
\E(X) &= np = 330 \cdot 0.9 = 297 \\
\Var(X) &= np(1-p) = 330 \cdot 0.9 \cdot 0.1 = 29.7
\end{align*}
Теперь посчитаем нужную вероятность:
\[
\P(X \leq 300) = \P \left(\frac{X - 297}{\sqrt{29.7}} \leq \frac{300 - 297}{\sqrt{29.7}} \right) = \P(\cN(0,1) \leq 0.55) \approx 0.709
\]
\item Вероятность переполнения не должна превышать $0.1$:
\begin{align*}
&\P(X > 300) < 0.1 \\
&\P\left(\frac{X - 0.9 \cdot n}{\sqrt{0.9 \cdot 0.1 \cdot n}} > \frac{300 - 0.9 \cdot n}{\sqrt{0.9 \cdot 0.1 \cdot n}} \right) < 0.1 \\
&\frac{300 - 0.9 \cdot n}{\sqrt{0.9 \cdot 0.1 \cdot n}}  > 1.28 \\
&300 - 0.9n > 1.28 \cdot 0.3 \sqrt{n} \\
&n < 325.6
\end{align*}
\end{enumerate}
\item[8.]
\begin{enumerate}
\item Выпишем случайную величину $X_i$ — цену акции после $i$-ого дня:
\[
X_i =
\begin{cases}
1.01, & p = 0.7 \\
0.99, & p = 0.2999 \\
0, & p = 0.0001
\end{cases}
\]
Нужно посчитать ожидание цены акциии после 20 дней:
\[
\E(X_1 \cdot \ldots \cdot X_{20}) \stackrel{\text{незав-ть}}{=} \E(X_1) \cdot \ldots \cdot \E(X_{20}) = 1.004^{20} \approx 1.083
\]
\item По ЗБЧ:
\[
\plim_{n\to\infty} \frac{1}{n} \sum_{i=1}^n X_i = \E(X_i) = 1.004
\]
\item Аналогично пункту (а):
\[
\E(X_1 \cdot \ldots \cdot X_{n}) = (\E(X_1))^n = 1.004^n
\]
И понятно, что $1.004^n \to_{n\to\infty} +\infty$.
\item
\begin{align*}
\P(\text{разорения}) &= 1 - \P(X_1 > 0, \ldots, X_n >0) = 1 - \prod_{i=1}^n \P(X_i > 0) \\
&= 1 - (1 - 0.0001)^n \to_{n\to\infty} 1
\end{align*}
\end{enumerate}
\end{enumerate}




\subsection[2016-2017]{\hyperref[sec:kr_02_2016_2017]{2016-2017}}
\label{sec:sol_kr_02_2016_2017}

\begin{enumerate}
\item \begin{enumerate}
\item $\E (2\xi - \eta +1) = 2 \E (\xi) - \E (\eta) + 1 = 2\cdot 1 - (-2) + 1 = 5 $

$\Cov (\xi, \eta) = \E (\xi \eta) - \E(\xi) \E(\eta) = -1 - \cdot 1 \cdot (-2) = 1$

$\Corr (\xi, \eta) = \frac{\Cov(\xi, \eta)}{\sqrt[]{\Var(X) \cdot \Var(Y)}} = \frac{1}{\sqrt{1 \cdot (8-(-2)^2)}} = \frac{1}{2}$

$\Var (2\xi - \eta + 1) = 4\Var(\xi) + \Var(\eta) - 2 \Cov (2\xi, \eta) = 4 \cdot 1 + 4 - 4 \cdot 1 = 4$

\item $\Cov(\xi + \eta, \xi + 1) = \Cov(\xi) + \Cov(\xi, 1) + \Cov(\eta, \xi) + \Cov(\eta, 1) = 1 +1 = 2$

$\Corr(\xi + \eta , \xi + 1) = \frac{\Cov(\xi + \eta , \xi + 1)}{\sqrt{\Var(\xi + \eta)\cdot \Var (\xi + 1)}} = \frac{2}{\sqrt{(1+4+2\cdot 1) \cdot 1}} = \frac{2}{\sqrt{7}}$

$\Corr(\xi + \eta - 24, 365 - \xi - \eta) = -1$

$\Cov(2016\cdot \xi, 2017) = 0$

\end{enumerate}
\item
\begin{enumerate}
\item Частные распределения:

\begin{center}
\begin{tabular}{cccc}
\toprule
$x$ & $-1$ & $0$ & $2$ \\
$\P(\xi = x)$ & $0.3$ & $0.4$ & $0.3$ \\ \bottomrule
\end{tabular}
\hspace{1cm}
\begin{tabular}{ccc}
\toprule
$y$ & $-1$ & $1$ \\
$\P(\eta = y)$ & $0.5$ & $0.5$ \\ \bottomrule
\end{tabular}
\end{center}

$\E(\xi) = -1 \cdot 0.3 + 0 \cdot 0.4 + 2 \cdot 0.3 = 0.3$

$\E (\xi^2) = (-1)^2 \cdot 0.3 + 0^2 \cdot 0.4 + 2^2 \cdot 0.3 = 1.5$

$\Var(\xi) = \E(\xi^2) - (\E(\xi))^2 = 1.5 - 0.3^2 = 1.41$

$\E(\eta) = -1 \cdot 0.5 + 1 \cdot 0.5 = 0$

$\E(\eta^2) = (-1)^2 \cdot 0.5 + 1^2 \cdot 0.5 = 1$

$\Var(\eta) = \E(\eta^2)-(\E(\eta))^2 = 1 - 0^2 = 1$

\item
\begin{center}
\begin{tabular}{cccccc}
\toprule
$xy$ & $-2$ & $-1$ & $0$ & $1$ & $2$ \\
$\P(\xi \cdot \eta = xy)$ & $0.2$ & $0.2$ & $0.4$ & $0.1$ & $0.1$ \\ \bottomrule
\end{tabular}
\end{center}

$\E(\xi\cdot\eta) = (-2) \cdot 0.2 + (-1) \cdot 0.2 + 0 \cdot 0.4 + 1\cdot 0.1 + 2 \cdot 0.1 = -0.3$

$\Cov(\xi, \eta) = \E(\xi\cdot\eta) - \E(\xi)\cdot\E(\eta) = -0.3 - 0.3 \cdot 0 = -0.3$
\item Пусть случайная величина $X$ принимает значения $a_1, \ldots, a_m$, случайная веилчина $Y$ принимает значения $b_1, \ldots, b_n$. Тогда случйаня величина $X$ и $Y$ называются независимыми, если $\forall i=1, \ldots, m \quad \forall j=1, \ldots, n: \P(X = a_i \cap Y = b_j) = P(X = a_i) \cdot P(Y = b_j)$
\item Заметим, что $\P (\xi = -1 \cap \eta=-1)=0.1$, $\P(\xi=-1)=0.3$ и $\P(\eta=-1)=0.5$.

Тогда поскольку $\P (\xi = -1 \cap \eta=-1) \neq \P(\xi=-1) \cdot \P(\eta=-1)$, случайные величины $\xi$ и $\eta$ не являются независимыми.
\item $\P (\xi = -1 \cap \eta=1) = \frac{\P (\xi = -1 \cap \eta=1)}{\P(\eta=1)} = \frac{0.2}{0.5} = \frac{2}{5}$

$\P (\xi = 0 \cap \eta=1) = \frac{\P (\xi = 0 \cap \eta=1)}{\P(\eta=1)} = \frac{0.2}{0.5} = \frac{2}{5}$

$\P (\xi = 2 \cap \eta=1) = \frac{\P (\xi = 2 \cap \eta=1)}{\P(\eta=1)} = \frac{0.1}{0.5} = \frac{1}{5}$

Следовательно, условное распределение случайной величины $\xi$ при условии $\{\eta=1\}$ может быть описано следующей таблицей:

\begin{center}
\begin{tabular}{cccc}
\toprule
$x$ & $-1$ & $0$ & $2$ \\
$\P(\xi = x)$ & $2/5$ & $2/5$ & $1/5$ \\ \bottomrule
\end{tabular}
\end{center}

\item $\E(\xi \mid \eta = 1) = -1 \cdot \frac{2}{5} + 0 \cdot \frac{2}{5} + 2 \cdot \frac{1}{5} = 0$
\item $\E(\pi) = \E(0.5 \xi + 0.5 \eta) = 0.5 \E(\xi) + 0.5 \E(\eta) = 0.15$

\begin{align*}
\Var(\pi) &= \Var(0.5 \xi + 0.5 \eta) = \Var(0.5 \xi) + \Var(0.5\eta) + 2 \Cov (0.5\xi, 0.5\eta) \\
&= 0.25\Var(\xi) + 0.25\Var(\eta) + 2 \cdot 0.5 \cdot 0.5 \Cov(\xi, \eta) \\
&= 0.25 \cdot 1.41 + 0.25 \cdot 1 + 2 \cdot 0.5 \cdot 0.5 \cdot (-0.3) = 0.4525
\end{align*}
\item
\begin{align*}
\Var(\pi(\alpha)) &= \Var(\alpha \xi + (1-\alpha)\eta) = \alpha^2\Var(\xi) + (1-\alpha)^2 \Var(\eta) \\
&+ 2\alpha(1-\alpha) \Cov(\xi, \eta) = 1.41 \cdot \alpha^2 + 1\cdot (1-\alpha)^2 + 2\alpha(1-\alpha) \cdot (-0.3) \\
&= 1.41 \cdot \alpha^2 + (1-\alpha)^2 - 0.6 \cdot (\alpha - \alpha^2) \to \min_\alpha \\
\frac{\partial}{\partial \alpha} \Var(\pi(\alpha)) &= 2 \cdot 1.41 \cdot \alpha -2(1-\alpha) -0.6\cdot(1-2\alpha) \\
&= 2.82 \cdot \alpha - 2 + 2\alpha - 0.6 + 1.2 \cdot \alpha = 6.02 \cdot \alpha - 2.6 = 0 \\
\alpha &= \frac{2.6}{6.02} = 0.4319
\end{align*}
\end{enumerate}
\item \begin{enumerate}
\item Для любой неотрицательной случайной величины $X$ и любого числа $\lambda > 0$ справедлива оценка: $\P(X>\lambda) \leq \frac{\E(X)}{\lambda}$

Пусть случайная величина $\xi_i$ означает число посетителей сайта за $i$-ый день. По условию, $\xi_i \sim Pois(\lambda=250)$. Известно, что если $\xi \sim Pois(\lambda)$, то $\E(\xi) = \Var(\xi) = \lambda$.

Имеем:
\[
\P(\xi_i >500) \leq \P(\xi_i \geq 500) \leq \frac{\E(\xi_i)}{500} = \frac{250}{500} = \frac{1}{2}
\]
\item Для любой случайной величины $X$ с конечным $\E(X)$ и любого положительного числа $\epsilon > 0$ имеет место неравенство: $\P(X-\E(X)\geq\epsilon)\leq\frac{\Var(X)}{\epsilon^2}$

Обозначим $\bar{\xi}_n := \frac{1}{n} \left(\xi_1 + \ldots + \xi_n\right)$ – среднее число посетителей сайта за $n$ дней. Тогда
\[
\E\left(\bar{\xi}_n\right) = \E\left(\frac{1}{n} \sum_{i=1}^{n} \xi_i\right) = \frac{1}{n} \sum_{i=1}^{n} \E(\xi_i) = \frac{1}{n} \cdot n \cdot \lambda = \lambda = 250
\]
\[
\Var\left(\bar{\xi}_n\right) = \Var\left(\frac{1}{n} \sum_{i=1}^{n} \xi_i\right) = \frac{1}{n^2} \sum_{i=1}^{n} \Var (\xi_i) = \frac{n \cdot \lambda}{n^2} = \frac{\lambda}{n} = \frac{250}{n}
\]
Оценим вероятность
\[
\P\left(\left\vert\bar{\xi}_n-250\right\vert > 10\right) \leq \frac{\Var\left(\bar{\xi}_n\right)}{100} = \frac{250}{100\cdot n}
\]
Следовательно, $1 - \frac{250}{100\cdot n} \leq \P\left(\left\vert\bar{\xi}_n-250\right\vert \leq 10\right)$.

Найдём наименьшее целое $n$, при котором левая часть неравенства ограничена снизу $0.99 \leq 1 - \frac{250}{100\cdot n}$.

Имеем:
\[
0.99 \leq 1 - \frac{250}{100\cdot n} \Leftrightarrow \frac{250}{100\cdot n} \leq 0.01 \Leftrightarrow n \geq \frac{250}{100 \cdot 0.01} \Leftrightarrow n  \geq 250
\]
Значит, $n=250$ – наименьшее число дней, при котором с вероятностью не менее $99\%$ среднее число посетителей будет отличаться от $250$ не более чем на $10$.
\item  Требуется найти наименьшее целое $n$, при котором $\P\left(\left\vert\bar{\xi}_n-250\right\vert \leq 10\right) = 0.99$

Имеем:
\begin{align*}
&\P(\vert\bar{\xi}_n - 250\vert \leq 10) = 0.99 \Leftrightarrow \P(-10\leq \bar{\xi}_n - 250 \leq 10) = 0.99 \\
&\P(-10n \leq S_n - 250n \leq 10n) = 0.99 \\
&\E(S_n) = \E(\xi_1 + \ldots + \xi_n) = \E(\xi_1) + \ldots + \E(\xi_n) = 250 \cdot n\\
&\Var(S_n) = \Var(\xi_1 + \ldots + \xi_n) = \Var(\xi_1) + \ldots + \Var(\xi_n) = 250 \cdot n \\
&\P\left(\frac{-10n}{\sqrt{250n}} \leq \frac{S_n - \E (S_n)}{\sqrt{\Var(S_n)}} \leq \frac{10n}{\sqrt{250n}}\right) = 0.99  \\
&2 \Phi \left(\frac{10n}{\sqrt{250n}}\right) -1 = 0.99 \\
&\Phi \left(\frac{10n}{\sqrt{250n}}\right) = \frac{1 + 0.99}{2}  \\
&\frac{10n}{\sqrt{250n}} = 2.58 \\
&\sqrt{n} = 2.58 \cdot \frac{\sqrt{250}}{10}  \\
&n = 16.641
\end{align*}
Следовательно, наименьшее целое $n$, есть $n=17$.
\item Пусть $X_1, X_2, \ldots, X_n, \ldots$ – последовательность независимых случайных величин с одинаковыми конечными математическими ожиданимяи и фиксированными конечными дисперсиями. Тогда $\frac{X_1 + \ldots + X_n}{n} \stackrel{\P}{\to} \E(X_i)$ при $n \to \infty$.

В нашем случае случаные величины $\xi_1^2, \xi_2^2, \ldots, \xi_n^2, \ldots$ – независимы,

$\E(\xi_1^2) = \ldots = \E(\xi_n^2) = \ldots < + \infty$ и $\Var(\xi_1^2) = \ldots = \Var(\xi_n^2) = \ldots < + \infty$ . Поэтому в соответствии с ЗБЧ имеем:
\[
\frac{\xi_1^2 +\ldots+ \xi_n^2}{n} \stackrel{\P}{\to} \E(\xi_i^2) = \Var(\xi_i) +\E(\xi_i)^2 = \lambda + \lambda^2 = \lambda(\lambda+1) = 250\cdot251 = 62750
\]
\end{enumerate}

\item \begin{enumerate}
\item Пусть $X_1, X_2, \ldots, X_n, \ldots$ – последовательность независимых,
одинаково распределённых случайных величин с $0<\Var(X_i)<\infty$, $i \in \mathbb{N}$.
Тогда для любого (борелевского) множества $B \subseteq R$ имеет место
\[
\lim_{n \to \infty} \P\left(\frac{S_n - \E(S_n)}{\sqrt{\Var(S_n)}} \in B\right) = \int_B \frac{1}{\sqrt{2\pi}}e^{-t^2/2} dt,
\]
где $S_n := X_1, \ldots, X_n$, $n \in \mathbb{N}$

\item Введём случайную величину
\[
X_i = \begin{cases}
1, & \text{если на i-ом шаге Винни-Пух пошёл направо} \\
-1, & \text{если пошёл налево}
\end{cases}
\quad i=1,\ldots, n;
\]
Тогда $S_n := X_1 +\ldots+X_n$ означает местоположение Винни-Пуха в $n$-ую минуту его блужданий по прямой.

$\E(X_i) = -1 \cdot 1/2 + 1 \cdot 1/2 = 0$,

$\E(X_i^2) = (-1)^2 \cdot 1/2 + (1)^2 \cdot 1/2 = 1$,

$\Var(X_i) = \E(X_i^2) - \E(X_i)^2 = 1$,

$\E(S_n) = \E(X_1 + \ldots + X_n ) = \E(X_1) + \ldots + \E(X_n) = 0$,

$\Var(S_n) = \Var(X_1 + \ldots + X_n ) = \Var(X_1) + \ldots + \Var(X_n) = n$

\begin{align*}
\P(S_n \in (-\infty, -5]) &= \P(S_n \leq -5) = \P\left( \frac{S_n-\E(S_n)}{\sqrt{\Var(S_n)}} \leq \frac{-5-0}{\sqrt{n}} \right)  \\
&\stackrel{n=60}{=}\P\left( \frac{S_n-\E(S_n)}{\sqrt{\Var(S_n)}} \leq -0.6454\right) \approx \int_{-\infty}^{-0.6454} \frac{1}{\sqrt{2\pi}} e^{-t^2/2} dt \\
&= \Phi(-0.6454) = 1-\Phi(0.6454) \approx0.2593
\end{align*}
\item Для любых $n \in \mathbb{N}$ и всех $x \in \mathbb{R}$ имеет место оценка:
\[
\bigl|F_{S_n^{*}}(x) - \Phi(x)\bigr| \leq 0.48 \cdot \frac{\E(|\xi_i - \E\xi_i|^3)}{\Var^{3/2}(\xi_i)\cdot\sqrt{n}} \text{,}
\]
где $\Phi(x) = \int_{-\infty}^{x}\frac{1}{\sqrt{2\pi}}e^{-\frac{t^2}{2}}\,dt$, \; $S_n^* = \frac{S_n - \E(S_n)}{\sqrt{\Var(S_n)}}$, \; $S_n = \xi_1 + \ldots + \xi_n$

В нашем случае:
\[
\P\left( \frac{S_{60} - \E(S_{60})}{\sqrt{\Var(S_{60})}} \leq -0.6454 \right) = \P(S^*_{60} \leq -0.6454) =
F_{S^*_{60}} (-0.6454)
\]
Согласно неравенству Берри-Эссеена, погрешность $\vert F_{S^*_{60}} (-0.6454) - \Phi(-0.6454) \vert$ оценивается сверху величиной
\[
0.48 \cdot \frac{\E(\vert X_i - \E(X_i) \vert^3 )}{\Var(X_i)^{3/2} \cdot \sqrt{n}} = 0.48 \cdot \frac{\E(\vert X_i \vert^3)}{1\cdot\sqrt{60}} = \frac{0.48}{\sqrt{60}} \approx0.062
\]
\end{enumerate}
\item \begin{enumerate}
\item Сначала найдём плотность распределения случайной величины $X$.
Пусть $x \leq 0 $, тогда $f_X (X) = \int_{-\infty}^{+\infty} f_{X, Y} (x, y) dy  = 0$.

Пусть $x >0 $, тогда
\begin{align*}
f_X (X) &= \int_{-\infty}^{+\infty} f_{X, Y} (x, y) dy = \int_{0}^{+\infty} 0.005 e^{-0.05x-0.1y} dy \\
&= 0.005e^{-0.05x} \int_{0}^{+\infty} e^{-0.1y} dy = 0.005e^{-0.05x} \cdot \left(-10e^{-0.1y} \right) \bigg\vert_{y=0}^{y=+\infty} = 0.05 e^{-0.05x}
\end{align*}
Таким образом, имеем:
\[
f_X (x) = \begin{cases}
0.05 e^{-0.05x} & \text{при } x>0 \\
0 & \text{при } x \leq 0
\end{cases}
\]
То есть $X \sim Exp(\lambda=0.05)$ – случайная величина $X$ имеет показательное
распределение с параметром $\lambda = 0.05$.

Теперь найдём плотность распределения случайной величины $Y$.

Пусть $y \leq 0 $, тогда $f_Y (y) = \int_{-\infty}^{+\infty} f_{X, Y} (x, y) dx  = 0$.

Пусть $y > 0 $, тогда
\begin{align*}
f_Y (y) &= \int_{-\infty}^{+\infty} f_{X, Y} (x, y) dx  = \int_{0}^{+\infty} 0.005 e^{-0.05x-0.1y} dx \\
&= 0.005e^{-0.1y} \int_{0}^{+\infty} e^{-0.05x} dx = 0.005e^{-0.1y} \cdot \left(-20e^{-0.05x} \right) \bigg\vert_{x=0}^{x=+\infty} = 0.1 e^{-0.1y}
\end{align*}
Таким образом, имеем:
\[
f_Y (y) = \begin{cases}
0.1 e^{-0.1y} & \text{при } y>0 \\
0 & \text{при } y \leq 0
\end{cases}
\]
То есть $Y \sim Exp(\lambda=0.1)$ – случайная величина $Y$ имеет показательное распределение с параметром $\lambda = 0.1$.
\item Поскольку для любых точек $x, y \in \mathbb{R}$ справедливо равенство $f_{X, Y} (x, y) = f_X (x) \cdot f_Y (y)$, случайные величины $X$ и $Y$ являются независимыми.
\item Найдём вероятность $\P(Y>5)$:
\[
\P(Y>5) = \int_{5}^{+\infty} f_Y (y) dy = \int_{5}^{+\infty}  0.1 e^{-0.1y} dy = 0.1 \cdot (-10 e^{-0.1x}) \bigg\vert_{y=5}^{y=+\infty} = e^{-0.5} \approx0.6065
\]
\item Требуется найти условную вероятность $\P(Y>8 \mid Y \geq 3)$. Для этого предварительно найдём вероятности $\P(Y>8)$ и $\P(y \geq 3)$:
\[
\P(Y>8) = \int_{8}^{+\infty} f_Y (y) dy  = \int_{8}^{+\infty}  0.1 e^{-0.1y} dy = 0.1 \cdot (-10 e^{-0.1x}) \bigg\vert_{y=8}^{y=+\infty} = e^{-0.8}
\]
\[
\P(Y \geq 3) =  \int_{3}^{+\infty} f_Y (y) dy   =  \int_{3}^{+\infty}  0.1 e^{-0.1y} dy = 0.1 \cdot (-10 e^{-0.1x}) \bigg\vert_{y=3}^{y=+\infty} = e^{-0.3}
\]
Теперь находим требуемую условную веростность:
\[
\P(Y>8 \mid Y \geq 3) = \frac{\P(Y > 8 \cap
Y \geq 3)}{\P(Y \geq 3)} = \frac{\P(Y>8)}{\P(Y \geq 3) } = \frac{e^{-0.8}}{e^{-0.3}} = e^{-0.5} \approx0.6065
\]
\item Сначала найдём условную плотность распределения случайной величины $X$ при условии $Y=y$:
\begin{align*}
f_{X \mid Y} (x \mid y) &=
\begin{cases}
\frac{f_{X \mid Y} (x, y)}{f_Y (y)} & \text{при } f_Y (y) \\
0 & \text{иначе}
\end{cases} \\
&=
\begin{cases}
\frac{0.005e^{-0.05x-0.1y}}{0.1e^{-0.1y}} & \text{при } x>0, \quad y>0 \\
0 & \text{иначе}
\end{cases} \\
&= \begin{cases}
0.05 e^{-0.05x} & \text{при } x>0, \quad y>0 \\
0 & \text{иначе}
\end{cases}\\
&=
\begin{cases}
f_X (x) & \text{при } y > 0 \\
0 & \text{при } y \leq 0
\end{cases}
\end{align*}
Теперь находим условное математическое ожидание
\[
\E(X \mid Y=5) = \int_{-\infty}^{+\infty} xf_{X\mid Y} (x \mid 5) dx =  \int_{-\infty}^{+\infty} xf_{X} (x) dx = \E(X) = \frac{1}{0.05} =20
\]
Здесь мы воспользовались известным фактом, что если $X\sim Exp(\lambda)$, то $\E(X) = \frac{1}{\lambda}$
\item Требуется найти вероятность $\P(X-Y > 2)$. Для этого введём множества

$B:=\{(x, y) \in \mathbb{R} : y < x-2 \}$ и $C := \{ (x,y) \in \mathbb{R} : y < x-2, x>0, y> 0  \}$.

Заметим, что искомая вероятность  $\P(X-Y > 2)$ может быть записана в виде
\[
\P(X-Y > 2) = \P((X, Y) \in B ) = \int \int_B f_{X, Y} (x, y) dx dy = \int \int_C f_{X, Y} (x, y) dx dy
\]
Стало быть, искомая вероятность
\begin{align*}
\P(X-Y > 2) &= \int \int_C f_{X, Y} (x, y) dx dy = \int_{2}^{+\infty} \left[ \int_{0}^{x-2} f_{X, Y} (x, y) dy \right] dx \\
&= \int_{2}^{+\infty} \left[ \int_{0}^{x-2} 0.005e^{-0.05x-0.1y}dy \right] dx \\
&= \int_{2}^{+\infty} \left[ 0.005e^{-0.05x} \cdot (-10e^{-0.1y}) \bigg\vert_{y=0}^{y=x-2} \right] dx  \\
&=  \int_{2}^{+\infty} \left[ 0.005e^{-0.05x} \cdot\left(1-e^{-0.1(x-2)}  \right) \right] dx  = \int_{2}^{+\infty} 0.005e^{-0.05x} dx \\
&- \int_{2}^{+\infty} 0.005e^{-0.05x-0.1x+0.2} dx
= 0.05 \cdot \left( -\frac{1}{0.05}e^{-0.05x}  \right) \bigg\vert_{x=2}^{x=+\infty} \\
&- e^{0.02} \cdot 0.05 \cdot \left( \frac{1}{0.15} e^{-0.15x} \right) \bigg\vert_{x=2}^{x=+\infty}
= e^{-0.1} -\frac{1}{3} e^{-0.1} = \frac{2}{3}e^{-0.1}  \approx 0.6032
\end{align*}
\end{enumerate}
\item Для решения задачи воспользуемся хорошо известными соотношениями:
\begin{align*}
&\int_{-\infty}^{+\infty} \frac{1}{\sqrt{2\pi\sigma^2}} e^{-\frac{(x-\mu)^2}{2\sigma^2}} dx = 1 \\
&\int_{-\infty}^{+\infty} x\frac{1}{\sqrt{2\pi\sigma^2}} e^{-\frac{(x-\mu)^2}{2\sigma^2}} dx = \mu \\
&\int_{-\infty}^{+\infty} x^2 \frac{1}{\sqrt{2\pi\sigma^2}} e^{-\frac{(x-\mu)^2}{2\sigma^2}} dx = \mu^2 + \sigma^2
\end{align*}
\begin{enumerate}
\item Указанная в задании функция $f_X$ является плотностью распределения, так как она удовлетворяет двум условиям:  $f_X$ является неотрицательной и интеграл от функции $f_X$ в пределах от  $-\infty$ до $+\infty$ равен единице:
\[
\int_{-\infty}^{+\infty} f_X (x) dx = \frac{1}{2} \cdot \int_{-\infty}^{+\infty}  \frac{1}{\sqrt{2\pi}} e^{-\frac{(x-1)^2}{2}} dx  + \frac{1}{2} \cdot \int_{-\infty}^{+\infty}  \frac{1}{\sqrt{2\pi}} e^{-\frac{(x+1)^2}{2}} dx = 1
\]
\item $\E(X) =\int_{-\infty}^{+\infty} xf_X (x) dx =  \frac{1}{2} \cdot \int_{-\infty}^{+\infty} x \frac{1}{\sqrt{2\pi}} e^{-\frac{(x-1)^2}{2}} dx  + \frac{1}{2} \cdot \int_{-\infty}^{+\infty}  x \frac{1}{\sqrt{2\pi}} e^{-\frac{(x+1)^2}{2}} dx = 1- 1=  0$
\item  \begin{align*}
\E(X^2) &= \int_{-\infty}^{+\infty} x^2 f_X (x) dx  =
\frac{1}{2} \cdot \int_{-\infty}^{+\infty} x^2 \frac{1}{\sqrt{2\pi}} e^{-\frac{(x-1)^2}{2}} dx  +
\frac{1}{2}  \cdot \int_{-\infty}^{+\infty} x^2 \frac{1}{\sqrt{2\pi}} e^{-\frac{(x+1)^2}{2}} dx \\
&= \frac{1}{2}(1^2 + 1^2 + (-1)^2 + 1^2) =  2
\end{align*}
\item $\Var(X) = \E(X^2) - (\E(X))^2 = 2 - 0^2 = 2$
\end{enumerate}
\end{enumerate}



\subsection[2015-2016]{\hyperref[sec:kr_02_2015_2016]{2015-2016}}
\label{sec:sol_kr_02_2015_2016}


\begin{enumerate}
\item
\begin{enumerate}
\item $ \E({\xi_1 \cdot \xi_2}) = \int_{0}^1 \int_0^1 xy f(x,y) \, dx \, dy = \int_0^1 \int_0^1 \frac{1}{2}\cdot x^2y + \frac{3}{2}\cdot xy^2 \, dx \, dy = \int_{0}^1 \frac{y}{6} + \frac{3y^2}{4} \,dy = \frac{1}{3}$
\item $f_{\xi_1 | \xi_2} (x | y) = \frac{f_{\xi_1, \xi_2}(x, y)}{f_{\xi_2}(y)} = \frac{0.5x + 1.5y}{0.25 + 1.5y}, \text{ при } y \in (0,1)$
\item
\begin{align*}
\E(\xi_1 | \xi_2 = y) &= \int_0^1 x f_{\xi_1 | \xi_2} (x | y) dx = \int_{0}^{1}  x \frac{0.5x + 1,5y}{0.25 + 1.5y} dx \\
&= \frac{1}{0.25 + 1,5y}  \left. \left( \dfrac{0.5x^3}{3} +  \dfrac{1.5yx^2}{2} \right) \right|_0^1  =  \frac{1/6 + 3/4y}{0.25 + 1.5y}
\end{align*}
\item
Для того, чтобы функция являлась совместной плотностью для пары случайных величин, должно выполнятся следующее:
\[
\int_{\Omega} kx f(x,y) \, dx \, dy = 1
\]
Вычислим, чему равняется левая часть:
\[
1 = \int_{\Omega} kx f(x,y) \, dx \, dy = \int_{0}^1 \int_{0}^1 kx \left(\frac{x + 3y}{2}\right) dx \, dy = \int_{0}^1 \frac{k}{6} + \frac{3ky}{4} \, dy = \frac{k}{6} + \frac{3k}{8} \Rightarrow
\]
\[
k = \frac{24}{13}
\]
\end{enumerate}
\item Заметим, что $\xi + \eta = 10$, тогда $\Cov(\xi, \eta) = \Cov(\xi, 10-\xi) = -\Var(\xi)$.

Представим случайную величину $\xi$ в виде суммы случайных величин $\xi = \xi_1 + \ldots + \xi_{10}$, где
\[
\xi_i = \begin{cases}
1, & \text{если у студента есть хотя бы один незачёт}, p=0.2 \\
0, & \text{иначе}, p=0.8
\end{cases} \quad i = 1, \ldots, 10
\]

Поскольку результаты каждого из стуентов независимы, $\Var(\xi) = 10\Var(\xi_1)$
\[
\Cov(\xi, \eta) = -10(1^2 \cdot 0.2 - \left(1\cdot 0.2)^2\right) = -1.6
\]

Так как случайные величины $\xi$ и $\eta$ связаны соотношением $\xi = 10 - \eta$, $\Corr(\xi, \eta)=-1$.

Подставив в $\Cov(\xi - \eta, \xi)$ выражение $\eta = 10 - \xi$, получим:
\[
\Cov(\xi - \eta, \xi) = 2 \Cov(\xi, \xi) = 2 \cdot 0.16 = 0.32
\]
Случайные величины $\xi - \eta$ и $\xi$ не являются независиыми.
\item Найдем ожидаемую доходность и риск портфеля $R = \alpha \xi + (1-\alpha) \eta$
для любого $\alpha$, тогда при $\alpha = 1$ получим результаты Пети,
при $\alpha = 0.5$ — результаты Васи.
\[
\E R = \alpha + (1 - \alpha) = 1 \: \, \forall \, \: \alpha \in [0,1]
\]

Находим дисперсию:
\[
\Var(R) = \alpha^2 \cdot 4 + (1-\alpha)^2 \cdot 9 - 6\alpha (1-\alpha) = 19\alpha^2 -24\alpha + 9 \to \min_{\alpha}
\]

Теперь, найдем оптимальное $\alpha$:
\[
\alpha = \frac{24}{38}
\]

Финальные цифры:
\[
\begin{cases}
\Var(R)^{P} = 4 \Rightarrow \sigma_{P} = 2 \\
\Var(R)^{V} = 1.75 \Rightarrow \sigma_{V} \approx 1.32 \\
\Var(R)^{M} = \frac{27}{19} \Rightarrow \sigma_{M} \approx 1.19 \\
\end{cases}
\]
\item
\begin{enumerate}
\item Пусть $S$ количество мальчиков, тогда используя \href{https://en.wikipedia.org/wiki/Markov%27s_inequality}{неравенство Маркова} получаем:
\[
\P(S \ge 750) \le \frac{\E(S)}{750} = \frac{2}{3}
\]
\item Пусть, теперь, $\bar{X}$ доля мальчиков, то есть, $\bar{X} = \sum_{i=1}^n X_i /n$, где
\[
X_i =
\begin{cases}
1, \text{ если }i\text{-ый ребёнок — мальчик }\\
0, \text{ иначе }
\end{cases}
\]
тогда используя \href{https://en.wikipedia.org/wiki/Markov%27s_inequality}{неравенство Чебышева} получаем:
\[
\P(|\bar{X} - 0.5| \ge 0.25) \le \frac{\Var(\bar{X})}{0.25^2} = \frac{1/4000}{0.25^2} = 0.004
\]
\item Вероятность из предыдущего пункта можно записать в таком виде:
\begin{align*}
\P\left(\left|\bar{X} - 0.5\right| \ge 0.25\right) &= \P\left(\bar{X} \ge 0.75\right) + \P\left(\bar{X} \le 0.25\right) = 2\P\left(\bar{X} \ge 0.75\right) = \\
&= 2\P(\cN(0;1)\geq 0.25\sqrt{4000})=2\P(\cN(0;1)\geq 15.8) = 1.3 \cdot 10^{-56} \approx 0
\end{align*}
\end{enumerate}
\item Пусть случайная величина $S$ —  это валютный курс через полгода. Заметим, что $S = 100 + \delta_1 + \ldots + \delta_{171}$.
Тогда по ЦПТ $S \sim \cN(142.75, 203.0625)$. Теперь можно искать нужную вероятность:
\[
\P(S > 250) = \P \left(\frac{S -  142.75}{\sqrt{203.0625}} > \frac{250-142.75}{\sqrt{203.0625}} \right) = \P(\cN(0, 1) > 7.6) \approx 0
\]
%\item $\lambda p$
\end{enumerate}



\subsection[2014-2015]{\hyperref[sec:kr_02_2014_2015]{2014-2015}}
\label{sec:sol_kr_02_2014_2015}


\begin{enumerate}
\item
\begin{enumerate}
\item
Так как $(X, Y)$ имеют совместное равномерное распределение,
$\P\left\{X+Y > 1/2 \right\}$ можно рассчитать как отношение
соответствующих площадей:

\begin{center}
\definecolor{zzttqq}{rgb}{0.6,0.2,0.}
\begin{tikzpicture}[line cap=round,line join=round,>=triangle 45,x=5.692307692307692cm,y=5.692307692307692cm]
\draw[->,color=black] (-0.1,0.) -- (1.2,0.);
\foreach \x in {,0.2,0.3,0.4,0.5,0.6,0.7,0.8,0.9,1.,1.1}
\draw[shift={(\x,0)},color=black] (0pt,2pt) -- (0pt,-2pt) node[below] {\footnotesize $\x$};
\draw[->,color=black] (0.,-0.1) -- (0.,1.2);
\foreach \y in {0.2,0.3,0.4,0.5,0.6,0.7,0.8,0.9,1.,1.1}
\draw[shift={(0,\y)},color=black] (2pt,0pt) -- (-2pt,0pt) node[left] {\footnotesize $\y$};
\draw[color=black] (0pt,-10pt) node[right] {\footnotesize $0$};
\clip(-0.1,-0.1) rectangle (1.2,1.2);
\fill[color=zzttqq,fill=zzttqq,fill opacity=0.1] (0.,0.5) -- (0.5,0.) -- (1.,0.) -- (0.,1.) -- cycle;
\draw (0.,1.)-- (1.,0.);
\draw [color=zzttqq] (0.,0.5)-- (0.5,0.);
\draw [color=zzttqq] (0.5,0.)-- (1.,0.);
\draw [color=zzttqq] (1.,0.)-- (0.,1.);
\draw [color=zzttqq] (0.,1.)-- (0.,0.5);
\draw (0.344196516991,0.436512272899) node[anchor=north west] {$S_0$};
\draw (0.129676687465,0.218325437741) node[anchor=north west] {$S_1$};
\draw (0.0233335241101,1.16441289103) node[anchor=north west] {$y$};
\draw (1.06109611823,0.08) node[anchor=north west] {$x$};
\end{tikzpicture}
\end{center}

Соответственно:
\[
\P\left(X+Y > \frac{1}{2}\right) = \frac{S_0}{S_0+S_1} = \frac{0.5-S_1}{0.5} = \frac{\frac{1}{2}-\frac{1}{8}}{\frac{1}{2}} = \frac{3}{4}
\]

\item
\[
f_Y(y) = \int _{0}^{1-y} f_{XY}(x,y) dx
\]

Поэтому, нам сначала нужно найти $f_{XY}(x,y)$, которая для равномерного
распределения должна быть константой. Это можно сделать из условия:
\begin{align*}
&\int_{0}^{1}\int _{0}^{1-x} f_{XY}(x, y) dxdy = \int_{0}^{1}\int _{0}^{1-x} C dxdy = 1 \Rightarrow \\
& \int_{0}^{1}\int _{0}^{1-x} C dxdy = \int_{0}^{1} C(1-x) dx = \left. \left( Cx-C\frac{x^2}{2}\right) \right| _{0}^{1} =
\frac{C}{2}  = 1
\end{align*}

Откуда имеем $f_{XY}(x, y) = C = 2$.
Теперь можем найти плотность распределения расходов Васи:
\[
f_Y(y) = \int _{0}^{1-y} 2 dx = 2(1-y)
\]

\item В данном случае площади немного другие, но смысл тот же:

\begin{center}
\definecolor{zzttqq}{rgb}{0.6,0.2,0.}
\begin{tikzpicture}[line cap=round,line join=round,>=triangle 45,x=5.692307692307692cm,y=5.692307692307692cm]
\draw[->,color=black] (-0.1,0.) -- (1.2,0.);
\foreach \x in {,0.2,0.3,0.4,0.5,0.6,0.7,0.8,0.9,1.,1.1}
\draw[shift={(\x,0)},color=black] (0pt,2pt) -- (0pt,-2pt) node[below] {\footnotesize $\x$};
\draw[->,color=black] (0.,-0.1) -- (0.,1.2);
\foreach \y in {,0.2,0.3,0.4,0.5,0.6,0.7,0.8,0.9,1.,1.1}
\draw[shift={(0,\y)},color=black] (2pt,0pt) -- (-2pt,0pt) node[left] {\footnotesize $\y$};
\draw[color=black] (0pt,-10pt) node[right] {\footnotesize $0$};
\clip(-0.1,-0.1) rectangle (1.2,1.2);
\fill[color=zzttqq,fill=zzttqq,fill opacity=0.1] (0.,0.5) -- (0.,1.) -- (0.333333333333,0.666666666667) -- (0.333333333333,0.5) -- cycle;
\draw (0.,1.)-- (1.,0.);
\draw[dashed] (1/3, 2/3) -- (1/3,0);
\draw (0.1004007648034,0.729040237615) node[anchor=north west] {$S_0$};
\draw (0.343364031073,0.5908861537646) node[anchor=north west] {$S_1$};
\draw (0.0233335241101,1.16441289103) node[anchor=north west] {$y$};
\draw (1.06109611823,0.08) node[anchor=north west] {$x$};
\draw (0.,0.5)-- (0.5,0.5);
\draw [color=zzttqq] (0.,0.5)-- (0.,1.);
\draw [color=zzttqq] (0.,1.)-- (0.333333333333,0.666666666667);
\draw [color=zzttqq] (0.333333333333,0.666666666667)-- (0.333333333333,0.5);
\draw [color=zzttqq] (0.333333333333,0.5)-- (0.,0.5);
\end{tikzpicture}
\end{center}


\[\P\left( \left. X<\frac{1}{3} \hspace{1mm} \right| Y>\frac{1}{2}\right) = \frac{S_0}{S_0+S_1} = \frac{\frac{1}{8} - S_1}{\frac{1}{8}} = \frac{\frac{1}{8}-\frac{1}{72}}{\frac{1}{8}} = \frac{8}{9}\]

\item При $Y = 1/2$, $X$ распределен равномерно от 0 до $1/2$, поэтому его плотность
равна
\[
f_X(x) = \frac{1}{\frac{1}{2} - 0} = 2
\]
Соответственно, условное математическое ожидание:
\[
\E \left( X \left| Y =\frac{1}{2}\right.\right) = \frac{1}{4}
\]

\item $\E(\E(X|Y)) = \E(X)$, а маргинальную функцию плотности для $X$ мы можем найти так же, как искали для $Y$, и получим $f_X(x) = 2(1-x)$. Отсюда:
\[
\E(X) = \int _{0}^{1} 2x(1-x)dx = \left .\left(x^2 - \frac{2}{3}x^3\right) \right|^{1}_{0} = \frac{1}{3}
\]

\item Если вспомнить формулу для корреляции:
\[
\rho_{XY} = \frac{\Cov(X, Y)}{\sigma_X\sigma_Y}  = \frac{\E(XY) - \E(X)\E(Y)}{\sigma_X\sigma_Y}
\]

то станет более-менее очевидно, что надо найти $\E(XY)$ и дисперсии $X$ и $Y$.

\begin{align*}
\E(XY) &= \int _{0}^{1} \int _{0}^{1-x} 2xy dxdy = \int _{0}^{1} 2xdx \int _{0}^{1-x}ydy =  \int _{0}^{1}x(x^2-2x+1)dx = \\
&= \left. \left(\frac{x^4}{4} - \frac{2}{3}x^3 + \frac{x^2}{2}\right) \right|^{1}_{0} = \frac{3}{4}-\frac{2}{3} = \frac{1}{12}
\end{align*}

Соответственно:

\[
\Cov(X, Y) = \frac{1}{12} - \frac{1}{3}\cdot \frac{1}{3} = -\frac{1}{36}
\]

Найдем теперь дисперсии $X$ и $Y$ (они будут одинаковыми, как и математические ождания, в силу симметрии):

\[
\E\left(X^2\right) = \int _{0}^{1} 2x^2(1-x)dx = \left. \left( \frac{2}{3}x^2 - \frac{x_4}{2} \right) \right|_{0}^{1} = \frac{1}{6}
\]

Поэтому:
\[
\Var(X) = \E\left(X^2\right) - (\E(X))^2 = \frac{1}{6} - \frac{1}{9} = \frac{1}{18}
\]

Теперь наконец-то можем найти корреляцию:
\[
\rho_{XY} = -\frac{\frac{1}{36}}{\sqrt{\frac{1}{18}}\sqrt{\frac{1}{18}}} = -\frac{1}{2}
\]
\end{enumerate}

\item

\begin{enumerate}

\item Закон больших чисел гласит, что $\bar{X} \rightarrow
\E(X)$ при $n\rightarrow \infty$. Проверим его выполнение в данном случае:
\[
\E(X_n) = \frac{1}{2n}(-\sqrt{n}) + \left(1-\frac{1}{n}\right)\cdot0 + \frac{1}{2n}\sqrt{n} = 0
\]

\[
\lim_{n\rightarrow\infty} \bar{X} = \lim_{n\rightarrow\infty} \frac{X_1 +\dots + X_n}{n} = 0
\]
так как числитель ограничен, а знаменатель бесконечно возрастает.
Видим, что ЗБЧ в данном случае, конечно, выполняется.

Как вариант, можно было сказать, что дисперсия ограничена, и из этого также следует выполнение ЗБЧ.
\item Неравенство Чебышева:
\[
\P(|X-\E(X)|\geqslant \varepsilon) \leqslant \frac{\Var(X)}{\varepsilon^2}
\]

Соответственно, искомую вероятность можем оценить следующим образом:
\[
\P\left(\left|\bar{X}\right| \leqslant 1\right) = 1 -\P\left(\left|\bar{X}\right| \geqslant 1\right) \Rightarrow \P\left(\left|\bar{X}\right| \leqslant 1\right) \geqslant 1 - \frac{\Var\left(\bar{X}\right)}{1}
\]
\[
\Var\left(\bar{X}\right) = \Var\left(\frac{\sum_{i=1}^{n} X_i}{n}\right) = \frac{1}{n^2}\sum _{i=1}^{n} \Var{X_i}
\]
В свою очередь:

\[
\E\left(X_i^2\right) = 2\cdot\frac{1}{2n}\cdot n + \left(1-\frac{1}{n}\right)\cdot0 = 1 \Rightarrow \Var(X_i) = 1 \Rightarrow \Var\left(\bar{X}\right) = \frac{1}{n}
\]

Поэтому:
\[
\P\left(\left|\bar{X}\right| \leqslant 1 \right) \geqslant 1 - \frac{1}{n}
\]

\item
\[
1 - \frac{1}{n} = 0.9  \Rightarrow n = 10
\]

\end{enumerate}

\item

Обозначим за $R$ — необходимое количество наличных денег в банке. 
Пусть $X$ — случайная величина, показывающее размер суммарной выплаты $60$ 
($n$ — достаточное большое для применения ЦПТ) клиентам. 
Выплаты отдельным клиентам независимы, 
поэтому \( \E X = 60 \cdot 5000 = 3\cdot 10^5 \); \( \Var X = 60 \cdot 2000^2 = 2.4 \cdot 10^8 \); 
\( \sigma_X = \sqrt{2.4} \cdot 10^4 \approx 1.55 \cdot 10^4\)

Теперь по ЦПТ:
\begin{align*}
&\P(R \geqslant X) = 0.95 \\
&\P \left(\frac{X - \E X}{\sigma_X} \leqslant \frac{R - \E X}{\sigma_X} \right) = 0.95 \\
&\P \left( Z \leqslant \frac{R - 3 \cdot 10^5}{1.55 \cdot 10^4} \right) = 0.95
\end{align*}
Слева функция распределения; подставляя 95-\% квантиль стандартного нормального распределения, получаем:
\[
\frac{R - 3 \cdot 10^5}{1.55 \cdot 10^4} = 1.64 \Rightarrow R = 325420
\]


\item

\begin{enumerate}
\item По предельной теореме Муавра-Лапласа:
\[ \frac{\hat{p} - p}{\sqrt{p(1-p)/n}} \sim \cN (0,1) \]
\[ \P \left( \frac{|\hat{p} - p|}{\sqrt{p(1-p)/n}} \leqslant \frac{0.1}{\sqrt{p(1-p)/n}} \right) \geqslant 0.99 \]
\[ \P \left( |Z| \leqslant \frac{0.1}{\sqrt{p(1-p)/n}} \right) \geqslant 0.99 \]
Из симметричности стандартного нормального распределения и зная его 99.5-\% квантиль, равный приблизительно 2.58, получаем:
\[ \frac{0.1}{\sqrt{p(1-p)/n}} \geqslant 2.58 \]
\[ \frac{\sqrt{n}}{\sqrt{p(1-p)}} \geqslant \frac{2.58}{0.1} \]
\[ \sqrt{n} \geqslant \frac{2.58}{0.1} \sqrt{p(1-p)} \]
\[ n \geqslant 665.64 \cdot p(1-p) \]

С помощью неравенства Чебышева:
\[ \P \left( |\hat{p} - p| \leqslant 0.1 \right) \geqslant 0.99 \]
\[ \P \left( |\hat{p} - p| \geqslant 0.1 \right) \leqslant 0.01 \]
Теперь просто смотрим на неравенство Чебышева и на строчку выше, на неравенство Чебышева и на строчку выше\ldots
\[ \frac{p(1-p)/n}{0.1^2} = 0.01\]
\[ n = 10^4 p(1-p) \]
Принимаются оба ответа!

\item По предельной теореме Муавра-Лапласа:
\[ \P \left( \frac{|\hat{p} - p|}{\sqrt{p(1-p)/n}} \leqslant \frac{\varepsilon}{\sqrt{p(1-p)/1000}} \right) \geqslant 0.99 \]
\[ \P \left( |Z| \leqslant \frac{\varepsilon}{\sqrt{p(1-p)/1000}} \right) \geqslant 0.99 \]
Аналогично пункту 1:
\[ \frac{\varepsilon}{\sqrt{p(1-p)/1000}} \geqslant 2.58 \]
\[ \varepsilon \geqslant 0.082 \sqrt{p(1-p)} \]

С помощью неравенства Чебышева:
\[ \P \left( |\hat{p} - p| \leqslant \varepsilon \right) \geqslant 0.99 \]
\[ \P \left( |\hat{p} - p| \geqslant \varepsilon \right) \leqslant 0.01 \]
Аналогично пункту 1:
\[ \frac{p(1-p)/1000}{\varepsilon^2} = 0.01\]
\[ \varepsilon^2 = \frac{p(1-p)}{10} \]
\[ \varepsilon = \sqrt{\frac{p(1-p)}{10}} \approx 0.316 \sqrt{p(1-p)} \]

\end{enumerate}

Нужно было показать, как мастерство владения теоремой Муавра-Лапласа, так и неравенством Чебышева.

\end{enumerate}




\subsection[2013-2014]{\hyperref[sec:kr_02_2013_2014]{2013-2014}}
\label{sec:sol_kr_02_2013_2014}


\begin{enumerate}
\item
\begin{enumerate}
\item
\begin{align*}
\P\left(Y<X^2\right) &= \int_0^1 \int_0^{x^2} (x+y) dy dx = \left. \int_0^1 \left(xy \frac{y^2}{2} \right) \right|_0^{x^2} dx =  \int_0^1 \left(x^3 + \frac{x^4}{2} \right) dx = \\
&= \left. \frac{x^4}{4} + \frac{x^5}{10} \right|_0^1 = 0.35
\end{align*}
\item $f_X (x) = \int_0^1 (x+y) dy = \left. xy + \frac{y^2}{2} \right|_0^1 = x + 0.5$

$\E(X) = \int_0^1 f_X (x) \cdot x dx = \int_0^1 (x^2 + 0.5x) dx = 7/12$
\item $f_{X|Y=0.2}(x) = \frac{f_{X, Y}(x, 0.2)}{f_Y (0.2)} = \frac{x+0.2}{0.7}$
\end{enumerate}
\item
\begin{enumerate}
\item Из условия находим, что $X \sim \cN (0, 9)$, тогда
\[
\P(X>1) = \P \left(\frac{X-0}{3} > \frac{1-0}{3} \right) = \P\left(\cN(0,1) > \frac{1}{3} \right) = 0.37
\]
\item Подготовимся: $\E(2X+Y) = 0$, $\Var(2X+Y) = 4\Var(X) + \Var(Y) + 4 \Cov(X, Y) = 36$
\[
\P(2X+Y > 3) = \P\left(\frac{2X+Y-0}{6} > \frac{3-0}{6} \right) = \P(\cN(0, 1) > 0.5) = 0.31
\]
\item Воспользуемся формулами для условного нормального распределения:
\[
	\E(Y|X=x) = \mu_{y} + \Corr(X, Y)\cdot \sigma_{y}\cdot \frac{x-\mu_{x}}{\sigma_{x}}
\]
\[
	\Var(Y|X=x) = \sigma^{2}_{y}\cdot (1 - \Corr^{2}(X, Y))
\]

Посчитаем\ldots

\[
	\E(Y|X=x) = 0 - \frac{1}{6}\cdot 2 \cdot \frac{1-0}{3} = -\frac{1}{9}
\]
\[
	\Var(Y|X=x) =4\cdot (1 - \frac{1}{36}) = \frac{35}{9} 
\]

А теперь стандартизируем:

\[
	\P(2X + Y >3|X=1) = 
	\P(Y>1|X=1) = \P(\frac{Y - (-1/9)}{\sqrt{35/9}}>\frac{1 - (-1/9)}{\sqrt{35/9}})
\]

\[
	\P(Y>1|X=1) = \P(\cN(0,1)>  \approx 0.56) = 1-\phi(0.56) = 1 - 0.56 = 0.44
\]
\item Заметим, что $\frac{X^2}{9}+\frac{Y^2}{4} \sim \chi^2_2$,
и тогда по таблице находим, что $\P\left(\frac{X^2}{9}+\frac{Y^2}{4} >12\right) = 0.0025$

\item В общем виде:

\[
	f(x,y) = \frac{1}{2\pi}\cdot\frac{1}{\sigma^{2}_{x}\cdot\sigma^{2}_{y}(1-\Corr^{2})}
	  \cdot 
	\exp^{-0.5(x-\mu_{x}, y - \mu_{y})\cdot C^{-1}\cdot (x-\mu_{x}, y - \mu_{y})^{T}}, 
\]
где $C$ — ковариационная матрица.

Подставляем и получаем:

\[
	f(x,y) = \frac{1}{2\pi}\cdot \frac{1}{\sqrt{35}}
	    \cdot 
	\exp^{-0.5\cdot (X,Y)\cdot C^{-1} \cdot (X,Y)^{T}}, 
\]
где 
\[
C^{-1} =
\begin{pmatrix}
  \frac{4}{35} & \frac{1}{35}\\
  \frac{1}{35} & \frac{9}{35}
\end{pmatrix}
\]

В итоге получаем:
\[
	f(x,y) = \frac{1}{2\pi}\cdot \frac{1}{\sqrt{35}}\cdot 
	\exp^{\frac{1}{70}\cdot(4x^{2}+2xy+9y^{2})}
\]
\end{enumerate}
\item
\begin{enumerate}
\item Заметим, что $\frac{X_1}{\sqrt{\frac{X_3^2+X_4^2+X_5^2}{3}}} \sim t_3$.
По таблице находим искомую вероятность: $0.15$.
\item
\item Заметим, что $\frac{X_1^2}{X_2^2+X_3^2} \sim F_{1, 2}$.
Нужное значение находим в таблице: $0.95$.
\end{enumerate}
\item
\begin{enumerate}
\item По неравенству Чебышёва:
$\P(|X_i-0.5|\geqslant 0.3) \leqslant \frac{1/12}{9/100} = \frac{25}{27}$
\item По неравенству Маркова: $\P(X_i \geqslant 0.8) \leqslant \frac{5}{8}$
\item $\E\left(\bar{X}\right) = \frac{1}{2}$,
$\Var\left(\bar{X}\right) = \frac{1}{36\cdot 12}$, $\bar{X}\sim\cN(\frac{1}{2}, \frac{1}{36\cdot12})$

$\P\left(\bar{X} > 0.8\right) = \P\left(\frac{\bar{X} - \frac{1}{2}}{\sqrt{ \frac{1}{36\cdot 12}}} \geq \frac{0.8-0.5}{ \frac{1}{36\cdot 12}} \right) = \P(\cN(0,1) \geq 6.235) \approx 0$
\item $ \P(|X_i-0.5|\geqslant 0.3) = 1 - \P(|X_i-0.5|\geqslant 0.3) = 1 - \P(-0.3 \leqslant X_i - 0.5 \leqslant 0.3) = 0.4$
\item Нужно воспользоваться неравенством Берри-Ессеена.
\[
\E\left(|X_1 - 0.5|^3\right) = \int_0^1 |x_1 -0.5|^3 \cdot 1 dx = 2 \int_{0.5}^1 (x_1 - 0.5)^3 dx = \frac{1}{2^5}
\]
\item $\P\left(\bar{X} - 0.5 > 0.3\right) = \frac{25}{27n} \to_{n \to \infty} 0$
\end{enumerate}
\item $\P\left(\hat{p} \leq 0.25\right) = \P\left(\frac{\hat{p} - 0.2}{\sqrt{\frac{0.2\cdot0.8}{n}}} \leq \frac{0.25}{\sqrt{\frac{0.2\cdot0.8}{n}}}\right) = 0.99$

По таблице: $\frac{0.25}{\sqrt{\frac{0.2\cdot0.8}{n}}} = 2.33 \Rightarrow n = 348$
\item
\begin{enumerate}
\item $3, 4, 5, 7, 8, 9$
\item $F(x) = \begin{cases}
0 &  x < 3 \\
1/6 & 3 < x \leq 4 \\
2/6 & 4 < x \leq 5 \\
3/6 & 5 < x \leq 7 \\
4/6 & 7 < x \leq 8 \\
5/6 & 8 < x \leq 9 \\
1 &  x > 9 \\
\end{cases}$
\item $\bar{X} = 6$, $\widehat{\Var}(X) = 28$
\end{enumerate}
\end{enumerate}




\subsection[2012-2013]{\hyperref[sec:kr_02_2012_2013]{2012-2013}}
\label{sec:sol_kr_02_2012_2013}

\begin{enumerate}
\item  $f(s,t)=f(s)\cdot f(t|s)=\frac{1}{3s}$ при $0\leq t\leq s\leq 3$.
Бонус тем, кто прочитал условие, $\P(S>T)=1$.
\[
\E\left(T^2\right) = \int_0^3\int_0^s \frac{t^2}{3s}\,dt\,ds = 1
\]
\item
\begin{enumerate}
\item $\P(X>0.5) = \P(Z > 0.13) \approx 0.45$, $\sigma_X \approx 0.38$
\item $\P(X+Y>0.5) =
\P(Z>-0.65)\approx 0.74$,
$\sigma_{X+Y} \approx 0.17$, $\E(X+Y) = 0.61$

\item $X = 0.25$ при нормировке даёт $\tilde{X} = -0.53$.
Получаем:
\begin{align*}
& \E\left(\tilde{Y} \mid \tilde{X} = -0.53\right) = 0.47 \\
& \Var\left(\tilde{Y} \mid \tilde{X} = -0.53\right) = 0.19
\end{align*}

Значит $\E\left(Y \mid \tilde{X} = -0.53\right) = 0.34$,
$\Var\left(Y \mid \tilde{X} = -0.53\right)=0.027$.

\item $\P\left(Y > 1/3 \mid \tilde{X} = -0.53\right) = \P(Z > -0.04) = 0.52$
\item Ноль
\end{enumerate}
\item $\E(\hat{p})=0.5$, $\Var(\hat{p})=0.25/n=1/16000$. По Чебышёву:
\[
\P(|\hat{p}-0.5|\leq 0.01)\geq 1-\frac{\Var(\hat{p})}{0.01^2}=\ldots=0.375
\]
Используя нормальную аппроксимацию:

\[
\P(|\hat{p}-0.5|\leq 0.01)=\P(|Z|\leq 1.26)\approx 0.79
\]
\item Обозначим $N$ — количество подключенных абонентов, тогда $N\sim Bin(n,0.3)$.
При больших $n$ биномиальное распределение можно заменить на нормальное,
$N\sim \cN(0.3n,0.21n)$.

\[
\P(120N>1\,080\,000)=\P(N>9000)=\P\left(Z>\frac{9000-0.3n}{\sqrt{0.21n}}\right)=0.99
\]

Из таблицы находим, что

\[
\frac{9000-0.3n}{\sqrt{0.21n}} = -2.33
\]

Решаем квадратное уравнение, находим корни, один — отрицательный, другой, $n\approx 30622$.
\item

Вариационный ряд: $3$, $5$, $6$, $7$, $8$, $9$. $\bar{X}\approx 6.3$,
$\frac{\sum (X_i-\bar{X})^2}{n-1}\approx 4.7$,
$\frac{\sum (X_i-\bar{X})^2}{n}\approx 3.9$


\begin{minipage}{0.6\textwidth}
\begin{center}
\includegraphics[scale=0.5]{auto_figures_tikz/2012_2013_fig_02_empirical_dist.pdf}
\end{center}
\end{minipage}
\end{enumerate}



\subsection[2011-2012]{\hyperref[sec:kr_02_2011_2012]{2011-2012}}
\label{sec:sol_kr_02_2011_2012}

\begin{enumerate}
\item
\begin{enumerate}
\item $\P(X + Y > 1) = 4/5$. Здесь нужно брать интеграл...
\item $\E(X) = 13/20 = 0.65$, $\E(XY) = 2/5 = 0.4$, $\Cov(X, Y) = -9/400 = -0.0225$
\item Нет, так как функция плотности не раскладывается в произведение $h(x) \cdot g(y)$.
\item Да, так как функция плотности симметрична по $x$ и $y$
\end{enumerate}
\item
\begin{enumerate}
\item Заметим, что величина $|X_i|$ распределена равномерно на $[0; b]$,
поэтому $\E(|X_i|) = b/2$ и $\Var(|X_i|) = b^2/12$. Значит $\E(\hat{b}) = cb$ и для
несмещённости $c = 1$.
\item Находим $MSE$ через $b$ и $c$:
\[
MSE=\Var(\hat{b})+(\E(\hat{b})-b)^2=2c^2\cdot \frac{b^2}{12}+(c-1)^2\cdot b^2=b^2\left(\frac{7}{6}c^2-2c+1\right)
\]
Отсюда $c=\frac{6}{7}$.
\end{enumerate}
\item
\begin{enumerate}
\item $\E(\hat{\mu}_1)=6\mu/6=\mu$, несмещённая
\item $\E(\hat{\mu}_2)=\alpha \mu+\beta \mu$ и $\Var(\hat{\mu}_2)=\alpha^2 \frac{\sigma^2}{3}+\beta^2 \frac{2\sigma^2}{3}$.
Для несмещённости необходимо условие $\alpha+\beta=1$. Для минимизации дисперсии
получаем уравнение
\[
\alpha-2(1-\alpha)=0
\]
Отсюда оценка имеет вид $\frac{2}{3}\bar{X}+\frac{1}{3}\bar{Y}$
\end{enumerate}
\item
\begin{enumerate}
\item $S=X_1+X_2+X_3$, слагаемых мало, использовать нормальное распределение
некорректно. Можно использовать неравенство Чебышева, $\E(S)=27$, $\Var(S)=27$, поэтому
\[
\P(S\in [20;34])=\P( |S-\E(X)| \leq 7) \geq 1-\frac{27}{7^2}=\frac{22}{49}
\]
\item Используем неравенство Маркова:
\[
\P(X_1 \geq 12)\leq \E(X_1)/12=9/12=0.75
\]
\item Если $S=X_1+\ldots+X_{50}$, то можно считать, что $S\sim \cN(450;450)$, поэтому
\[
\P(S \in [430;470])\approx \P( N(0;1) \in [-0.94;+0.94])\approx 0.6528
\]
\end{enumerate}
\item
\begin{enumerate}
\item Если $Y = X_1 + X_2$, то $\E(Y) = 3$ и $\Var(Y) = 1 + 9 - 2 = 8$, значит
$\P(Y > 1) = \P(\cN(0,1) > -2/\sqrt{8}) \approx \P(\cN(0,1) > -0.71) \approx 0.7602$
\item Находим $\Cov(X_1, Y) = 1 - 1 = 0$. Итого: вектор имеет совместное нормальное
распределение с
\[
(X_1,Y) \sim \cN\left(
\left(\begin{array}{l}
{1} \\
{3}
\end{array}\right);
\left(\begin{array}{cc}
{1} & {0} \\
{0} & {8}
\end{array}\right)
\right)
\]
\item Стандартизируем величины. То есть мы хотим представить их в виде:
\[
\begin{cases}
X_1 = 1 + aZ_1 + bZ_2 \\
X_2 = 2 + cZ_2
\end{cases}
\]
Единица и двойка — это математические ожидания $X_1$ и $X_2$. Мы хотим, чтобы
величины $Z_1$ и $Z_2$ были $\cN(0,1)$ и независимы.
Получаем систему:
\[
\begin{cases}
\Var(X_1) = 1 \\
\Var(X_2) = 9 \\
\Cov(X_1,X_2) = -1
\end{cases} \Leftrightarrow
\begin{cases}
a^2 + b^2 = 1 \\
c^2 = 9 \\
bc = -1
\end{cases}
\]
Одно из решений этой системы: $c = 3$, $b = -1/3$, $a = 2\sqrt{2}/3$

Используя это разложение получаем:
\begin{multline*}
\left( X_1 \mid X_2 = 2\right) \sim \left(1 + \frac{2\sqrt{2}}{3}Z_1-\frac{1}{3}Z_2 \mid 2 + 3Z_2 = 2\right)\sim \\
\sim\left(1 + \frac{2\sqrt{2}}{3}Z_1 - \frac{1}{3}Z_2\mid Z_2 = 0\right) \sim \left(1 + \frac{2\sqrt{2}}{3}Z_1\right)\sim \cN(1;8/9)
\end{multline*}

Еще возможные решения: выделить полный квадрат в совместной функции плотности, готовая формула, etc
\end{enumerate}
\item
\begin{enumerate}
\item $\Var(\hat{p}) = \frac{p(1-p)}{n}$. Максимально возможное значение $p(1-p)$
равно $1/4$, поэтому максимально возможное значение $\Var(\hat{p})=1/4n$.
\item У нас задано неравенство:
\[
\P(|\hat{p} - p| > 0.02) < 0.1
\]
Делим внутри вероятности на $\sqrt{\Var(\hat{p})}$:
\[
\P\left(|\cN(0;1)| > 0.02 \sqrt{4n} \right) < 0.1
\]
По таблицам получаем $0.02 \sqrt{4n} \approx 1.65$ и $n \approx 1691$

Если вместо ЦПТ использовать неравенство Чебышева, то можно получить менее точный
результат $n=6250$.
\end{enumerate}
\item
\begin{enumerate}
\item $\E(X_i) = (1 + 10) / 2 = 5.5$, $\E\left(X_1^2\right) = \frac{1}{10} \frac{10\cdot 11\cdot 21}{6}=77/2$,
$\Var(X_i) = 33/4 = \sigma^2$.
Можно найти $\Cov(X_1, X_2)$ по готовой формуле, но мы пойдем другим путем.
Заметим, что сумма номеров всех вариантов — это константа, поэтому
$\Cov(X_1, X_1 + \ldots + X_{40}) = 0$. Значит, $\Var(X_1) + 39\Cov(X_1, X_2) = 0$.
В итоге получаем $\Cov(X_1, X_2) = -\sigma^2 / 39$
\item $\E(\bar{X})=11/2$, $\Var(\bar{X})=4\frac{1}{52}$
\item Да, являются, так как и $X_1$ и $X_2$ — это номер случайно выбираемого варианта
\item Нет, если известно чему равно $X_1$, то шансы получить такой же $X_2$ падают
\end{enumerate}
\item
Если мы наняли $n$ работников, то ожидаемое количество рабочих человеко-дней равно:
\[
\E(X)=365\cdot n\cdot \left(\frac{364}{365}\right)^{n}
\]
Для удобства берем логарифм $\ln(\E(X))=c+\ln(n)+n\ln(364/365)$ и получаем условие
первого порядка $1/n+\ln(364/365)=0$. 
Пользуясь разложением в ряд Тейлора $\ln(1 + t) \approx t$ получаем: 
$1/n-1/365\approx 0$, $n\approx 365$.
\end{enumerate}



\subsection[2010-2011]{\hyperref[sec:kr_02_2010_2011]{2010-2011}}
\label{sec:sol_kr_02_2010_2011}


\begin{enumerate}
\item Перед нами функция плотности двумерного нормального распределения!

Поэтому: $\E(X) = 0$, $\Var(Y)=1$, $\Cov(X, Y) = \rho$
\item С помощью таблицы находим, что $\P(X > \sqrt{2}) = 1 - \P(X<1.14) \approx 0.13$

Заметим, что $X^2 + Y^2 \sim \chi^2_2$, и находим искомую вероятность по таблице: $\P(X^2 + Y^2 >4) \approx 0.87$

\item Вспомним важные формулы:
\[
\E(X | Y = y) = \E(X) + \Cov(X, Y) \cdot \Var^{-1}(Y) \cdot (y - \mu_y)
\]
\[
\Var(X | Y = y) = \Var(X) - \Cov(X, Y) \Var^{-1}(Y) \cdot \Cov(Y, X)
\]
Применив их, получим: $\E(X | Y = 0) = \frac{22}{9}$, $\Var(X | Y = 0) = \frac{32}{9}$. Тогда:
\[
\P(X > 0 | Y = 0) = \P\left(\cN(0,1) > \frac{0 - \frac{22}{9}}{\sqrt{\frac{32}{9}}}\right) \approx 0.9
\]
Далее, найдём дисперсию портфеля и минимизиреум её по $\alpha$:
\begin{multline*}
\Var(\alpha X + (1-\alpha) Y) = \alpha^2 \Var(X) + (1-\alpha)^2 \Var(Y) + 2 \alpha (1-\alpha) \Cov(X, Y) =\\
= 4 \alpha^2 + 9(1-\alpha)^2 -4\alpha(1-\alpha)  = 17 \alpha^2 - 22 \alpha + 5 \to \min_{\alpha}
\end{multline*}
\[
\alpha = \frac{11}{17}
\]
Нельзя, так как из $\Cov(X+Y, 7X - 2Y) = 0$ не следует независимость $X+Y$ и $7X - 2Y$.
\item Сначала подготовимся и найдём дисперсию случайной величины $X$:
\[
\E(X) = \int_1^{\infty} \frac{3}{x^4} \cdot x dx = \left. \frac{3x^{-2}}{-2} \right|_1^{\infty} = \frac{3}{2}
\]
\[
\E(X^2) = \int_1^{\infty} \frac{3}{x^4} \cdot x^2 dx = \left. \frac{3x^{-1}}{-1} \right|_1^{\infty} = 3
\]
\[
\Var(X) = 3 - \left(\frac{3}{2}\right)^2 = \frac{3}{4}
\]
Перепишем исходное неравенство в виде: $\P(|\bar{X} -\E(X)|<0.1) \geq 1 - 0.02$.
\[
\frac{\Var(\overline{X})}{0.1^2} \leq 0.02 \Rightarrow \frac{\Var(X)}{n} \leq 0.0002 \Rightarrow n \geq 3750
\]
\item Нужно найти $\P(\hat{p} > \frac{60}{90})$. Воспользуемся теоремой Муавра-Лапласа:
\[
\P\left(\hat{p} > \frac{60}{90}\right) = \P \left(\frac{\hat{p} - 0.8}{\frac{0.8 \cdot 0.2}{90}} > \frac{2/3 - 0.8}{\frac{0.8 \cdot 0.2}{90}} \right) = \P(\cN(0,1) > -3.16) \approx 1
\]
Найдём объём выборки:
\[
\P \left(\frac{\hat{p} - 0.8}{\frac{0.8 \cdot 0.2}{n}} > \frac{0.02}{\frac{0.8 \cdot 0.2}{n}} \right) = 0.95 \Rightarrow \frac{0.02}{\frac{0.8 \cdot 0.2}{n}} = 1.65 \Rightarrow n =33
\]
\item %$\E\left(\frac{X_1+X_2}{2}\right) = 40.5$

%$\Var\left(\frac{X_1+X_2}{2}\right) = 3.9$

%$\Cov(X_1, X_2) = -0.45$
\item Необоходимо решить следующую систему уравнений:
\[
\begin{cases}
D(\overline{X}_S) = \sum_{l=1}^L \left(\frac{w_l^2 \sigma_l^2}{n_l} \right) \to \min \\
n = n_1 + n_2 + n_3
\end{cases}
\]
Выпишем лагранжиан:
\[
L  = \sum_{l=1}^L \frac{w_l^2 \sigma_l^2}{n_l} - \lambda(n_1 + n_2 + n_3 -n)
\]
\[
\frac{\partial L}{\partial n_l} = 0 \Rightarrow n_l^o = \sqrt{\frac{w_l^2 \sigma_l^2}{-\lambda}} \Rightarrow \sum_{l=1}^L w_l \sigma_l = \sqrt{-\lambda} n \Rightarrow \sqrt{-\lambda} = \frac{\sum_{l=1}^L w_l \sigma_l }{n}
\]
Тогда находим объём выборки каждой группы по формуле: $n^o = \frac{w_l \sigma_l }{\sum_{k=1}^L w_k \sigma_k } n$
\begin{itemize}
\item $n^o_{\text{недорогие}} = 0.255n$
\item $n^o_{\text{средние}} = 0.532n$
\item $n^o_{\text{дорогие}} = 0.213n$
\end{itemize}
\item Выборочное среднее: $-0.2$; выборочная дисперсия: $70.98$;

вариационный ряд: $-5.6, -3.2, -0.8, 1.1, 2.9, 4.4$.
\item Для проверки свойства несмещённости найдём математические ожидания оценок:
\[
\E(T_1) = \E(2\overline{X}) = 2\E(X_1) = \theta
\]
\[
\E(T_2) = \E((n+1)X_{(1)}) = (n+1) \frac{\theta}{2}
\]
Несмещённой является только оценка $T_1$.

Для проверки оставшихся свойств посчитаем дисперсию оценок:
\[
\Var(T_1) = 4\Var(\overline{X}) = \frac{\theta^2}{3n} \to_{n \to \infty} 0
\]
\[
\Var(T_2) = (n+1)^2 \cdot \frac{\theta^2}{12}
\]
Оценка $T_1$ является более эффективной, и она состоятельна. $T_2$ не является состоятельной оценкой.
\item По формуле Байеса:
\[
\P(X=k | X+Y = k) = \frac{\P(X=k \cap X+Y = n)}{\P(X+Y=n)} = \frac{\P(X+Y = k | X=k)\P(X=k)}{\P(X+Y=n)}
\]
Находим числитель:
\[
\P(X+Y = n | X=k) \cdot \P(X=k) = \P(Y = n-k) \cdot \P(X=k) = p\cdot (1-p)^{n-k-1} \cdot p \cdot p^{k-1} = p^2 \cdot (1-p)^{n-2}
\]
И знаменатель:
\begin{align*}
\P(X+Y = n) &= \sum_{i=1}^n \P(X = i \cap Y = n-i) =  \sum_{i=1}^n p \cdot (1-p)^{i-1} \cdot p \cdot (1-p)^{n-i-1} \\
&= \sum_{i=1}^n p^2 (1-p)^n = np^2 (1-p)^n
\end{align*}
Итого:
\[
\P(X=k | X+Y = k) = \frac{ p^2 \cdot (1-p)^{n-2}}{np^2 (1-p)^n} = \frac{1}{n(1-p)^2}
\]
Второе выражение:
\[
\P(Y=k | X=Y) = \frac{\P(Y=k \cap X=Y)}{\P(X=Y)} = \P(Y=k) = p(1-p)^{k-1}
\]
\end{enumerate}



\subsection[2009-2010]{\hyperref[sec:kr_01_2009_2010]{2009-2010}}
\label{sec:sol_kr_01_2009_2010}

\begin{enumerate}
\item Из условия: $\Var(X) = 5^2 = 25$, $\Var(Y) = 4^2 = 16$, $\Var(X - Y) = 2^2 = 4$.
Есть такое тождество, $\Var(X - Y) = \Var(X) + \Var(Y) - 2\Cov(X,Y)$. Отсюда находим
$\Cov(X, Y) = 37/2$ и $\Corr(X, Y) = 37/40$.
\item По таблице: $\P(X<\sqrt{3}) = 0.9582$

Заметим, что $X^2 + Y^2 \sim \chi^2_2$. По таблице находим искомую вероятность:
$\P(X^2 + Y^2<6) = 0.95$
\item
\begin{enumerate}
\item $\P(|X - 16| > 4) \leq 0.75$
\item $\P(|X - 16| > 4) = 1 - \P(-4 < X-16 < 4) = 1 - \P(12 < X < 20) = \frac{1}{3}$
\item $\P(|X - 16| > 4) =  1 - \P\left(\frac{-4}{\sqrt{12}} < \frac{X-16}{\sqrt{12}}
< \frac{4}{\sqrt{12}}\right) = 2 \left(1 - \P\left(\cN(0, 1)<\frac{4}{\sqrt{12}}\right)\right) \approx 0.25$
\end{enumerate}
\item $\E(2X + Y) = 2$, $\Var(2X + Y) = 3$

$\P(2X+Y>1) = 1 - \P\left(\frac{2X+Y - 2}{\sqrt{3}} < \frac{1-2}{\sqrt{3}}\right) =
1 - \P\left(\cN(0,1) < \frac{-1}{\sqrt{3}}\right)= 0.72$

Замечаем, что $(X|Y)\sim \cN(-3 - 0.5Y; 3/4)$ и $(X|Y=2)\sim \cN(-4;3/4)$.
Получаем, что 
$\P(2X+Y>1 \mid Y=2) = \P(X>-0.5 \mid Y=2) = 
\P(\cN(0,1) > 7/\sqrt{3}) \approx 0.00$

\item Если $S$ — финальная стоимость акции, то $S = 1000 + X_1 + X_2 + \ldots + X_{100}$.
Тогда по ЦПТ $S\sim \cN(1000,100)$ и $\P(S > 1010) = \P(Z > 1)$.
\end{enumerate}



\subsection[2008-2009 Демо-версия]{\hyperref[sec:kr_02_2008_2009_demo]{2008-2009 Демо-версия}}
\label{sec:sol_kr_02_2008_2009_demo}

\begin{enumerate}
\item
$\P(Y>2X)=\int_0^{0.5} \int_{2x}^{1} (x+y) dydx = 5/24$

$f_X(x) = \int_0^1 (x+y) dy = x + \frac{1}{2} \Rightarrow \E(X) = \int_0^1 \left(x+\frac{1}{2}\right)x dx = \frac{7}{12}$
\item
\begin{enumerate}
\item $\E(X_1 + 3X_2) = -1$, $\Var(X_1 + 3X_2) = 207$

$\P(X_1 + 3X_2 > 20) = \P\left(\frac{X_1 + 3X_2 +1}{\sqrt{207}} > \frac{20+1}{\sqrt{207}} \right) = \P(\cN(0,1)>1.46) \approx 0.07$
\item $\E(X_1|X_2 = 0) = 2 - 4.5\cdot\frac{1}{25}(0+1) = 1.82$, $\Var(X_1 | X_2 = 0) = 9 - (-4.5) \cdot \frac{1}{25} \cdot(-4.5) = 8.19$

$X_1 | X_2 = 0 \sim \cN(1.82, 8.19)$
\end{enumerate}
\item
\begin{enumerate}
\item $\E(X)=1000$, $\Var(X) = 10^6$
\item Случайная величина $X$, сумма выплат по одному контракту, $X \sim \cExp(0.001)$.

$\P(1000X>110000) = \P(X>100) = \int_{110}^{+\infty} 0.001 \exp(-0.001x) dx \approx 0.9$
\end{enumerate}
\item $\P\left(\frac{(X-30)^{2}}{\Var(X)}<3\right) = \P(|X-30|< \sqrt{3\Var(X)}) \geq 1 - \frac{\Var(X)}{ \sqrt{3\Var(X)}} = \frac{2}{3}$

$\P(|X-30|< \sqrt{3\Var(X)}) = \P(X< \sqrt{3\cdot900}+30) = \int_{0}^{82}\frac{1}{30} \exp\left(-\frac{1}{30}x\right)dx\approx 0.94$
\item $a \approx 1.28$
\item[9-Б.]
Подразумевая под точками концы гирлянды, докажем следующее утверждение.

Бросим $2n \geq 4$ точек $X_1, X_2, \ldots, X_{2n}$ случайным образом на отрезок $[0;1]$. Пусть для $1 \leq i \leq n$ $J_i$ — это отрезок с концами $X_{2i-1}$ и $X_{2i}$.
Тогда вероятность того, что найдётся такой отрезок $J_i$, который пересекает все другие отрезки, равна $2/3$ и не зависит от $n$.

Доказательство. Бросим $2n+1$ точек на окружность, тогда $2n$ точек образуют пары, а оставшуюся обозначим $X$ и будем считать её и началом, и концом отрезка.
Каждому получившемуся отрезку присвоим уникальное имя.
Далее, будем двигаться от точки $X$ по часовой стрелке до тех пор, пока не встретим одно и то же имя дважды, например «а».
После этого пойдём в обратную сторону, и будем идти, пока не встретим какое-нибудь другое имя дважды, например, «б».
Теперь посмотрим на получившуюся последовательность между «б» и «а» на концах пути, читая её по часовой стрелке от «б» до «а».
Нас интересует взаимное расположение $X$, второй «а» и второй «б».
Зная, что «а» стоит после $X$, выпишем все возможные случаи, где может стоять «б»:
\begin{enumerate}
\item перед $X$
\item между $X$ и «а»
\item после «а»
\end{enumerate}
Покажем, что во втором и третьем случае отрезок «б» пересекает все остальные, а в первом такого отрезка вообще нет. Попутно заметим, что появление каждого и случаев равновероятно.

Действительно, если «б» стоит после $X$, и отрезок соответствующий этому имени, не пересекает какой-нибудь другой отрезок «в», то последовательность выглядела бы как «бвв$X$б» или «б$X$ввб», что противоречит описанному построению.
Если «б» стоит перед $X$ и отрезок «в» пересекает оба отрезка «а» и «б», то мы снова приходим в противоречие с построением.
В итоге, получаем, что искомая вероятность равна $2/3$.
\end{enumerate}


\subsection[2008-2009]{\hyperref[sec:kr_02_2008_2009]{2008-2009}}
\label{sec:sol_kr_02_2008_2009}

\begin{enumerate}
\item
\begin{enumerate}
\item $\int_{0}^{1}\int_{x}^{1}p(x,y)dydx = 5/12$

$\int_{0}^{1}\int_{0}^{1}y\cdot p(x,y)dydx = 13/24$
\item Нет, так как совместная функция плотности не разлагается в произведение
индивидуальных
\end{enumerate}
\item
\begin{enumerate}
\item $\P(|X_{1}+X_{2}+ \ldots +X_{7}|>14)\leq \frac{7}{14^2}=\frac{1}{28}$

$\P(X_{1}^{2} + \ldots + X_{7}^{2} > 14) = \P(X_{1}^{2} + \ldots + X_{7}^{2} - 7 > 7)
= \P(|X_{1}^{2} + \ldots + X_{7}^{2} - 7| > 7) \leq \frac{2 \cdot 7}{7^2} = \frac{2}{7}$
\item $\P(|X_{1} + \ldots + X_{7}| > 14) = \P(|\cN(0;1)| > 14/\sqrt{7}) = \P(|\cN(0;1)| > 5.29) \approx 0$

$\P(X_{1}^{2} + X_{2}^{2} + \ldots + X_{7}^{2} > 14) \approx 0.05$
\end{enumerate}
\item
\begin{enumerate}
\item $X_{1} + 2X_{2} \sim \cN(5;89)$, $\P(Z > 1.59) = 0.056$
\[
\Var(X_1 + 2X_2) = \Var(X_1) + 4\Var(X_2) + 4\Cov(X_1, X_2) = 89
\]
\item Нормальное, причем $\cN(1.4;8)$, корреляция равна $-1/3$
\end{enumerate}
\item $\beta = \frac{1}{3}a^{2}$

$\E(XY) = \frac{3}{4}a^{2}$

$\E\left(Y^{2}\right) = 3a^{2}$

$\hat{\beta}_{1} = \frac{4}{9}XY$

$\hat{\beta}_{2} = \frac{1}{9}Y^{2}$

Так как обе оценки несмещенные вместо сравнения дисперсий можно сравнить квадраты
ожиданий

$\frac{16}{81}\E\left(X^{2}Y^{2}\right)$ vs $\frac{1}{81}\E\left(Y^{4}\right)$

\ldots

$16 a^4$ vs $\frac{81}{5} a^{4}$

Дисперсия васиной оценки меньше.
\item Заметим, что Пуассоновская величина с положительной вероятностью принимает
значение ноль, значит бывает, что монстры дохнут от одного устрашающего взгляда Васи :)
\begin{enumerate}
\item Сумма трех независимых пуассоновских величин - пуассоновская с параметром: $3\lambda=6$.

$\P(X=6)=e^{-6}\frac{6^6}{6!}\approx 0.16$

Ответ с факториалам считается полным.
\item Сумма 80 величин имеет пуассоновское распределение, но при большом количестве
слагаемых пуассоновское мало отличается от нормального.

$\E(S)=160$, $\Var(S)=160$

$\P(S>200)=\P\left(\frac{S-160}{\sqrt{160}}>3.16\right)\approx 0$
\end{enumerate}
\item
\begin{enumerate}
\item $\lambda = 1/10$, $\P(X < 7) = 0.5$
\item $\P(\bar{X} > 0.55) = \P(\cN(0;1) > \frac{0.05\sqrt{1000}}{0.5}) = \P(\cN(0;1) > 3.16) \approx 0$
\end{enumerate}
\item $\Var(X) = 5 \cdot 0.1 \cdot 0.9 + 5 \cdot 0.5 \cdot 0.5 = 1.7$

$\Var(Y) = 3 \cdot 0.1 \cdot 0.9 + 7 \cdot 0.5 \cdot 0.5 = 2.02$

Пусть $Z$ — число правильных ответов на вопросы с 3-го по 7-ой (у Пети и у Васи)
\begin{align*}
\Cov(X,Y) &= \Cov(Z + (X - Z), Z + (Y - Z)) = \Var(Z) \\
&+ \Cov(X - Z, Z) + \Cov(Z, Y - Z) + \Cov(X - Z, Y - Z) = \Var(Z)
\end{align*}
$Y - Z$ — это сколько правильных ответов дал лично Вася и оно не зависит от числа
$Z$ правильных списанных ответов, значит, $\Cov(Y-Z,Z)=0$.

Аналогично все остальные ковариации равны нулю.

$\Var(Z)=3\cdot 0.1\cdot 0.9+2\cdot 0.5\cdot 0.5=0.77$
\item Любые совпадения с курсом экономической и социальный статистики случайны и
непреднамеренны.

Чтобы оценка среднего по всем трем стратам была несмещена, она должна строиться
по формуле:

$\bar{X}=w_{1}\bar{X}_{1}+w_{2}\bar{X}_{2}+w_{3}\bar{X}_{3}$
(здесь $\bar{X}_{i}$ — среднее арифметическое по $i$-ой страте)

Поэтому $\Var(\bar{X})$ (минимизируемая функция) равняется:

$\Var(\bar{X})=\sum \frac{w^{2}_{i}\sigma^{2}_{i}}{n_{i}}$

Принцип кота Матроскина\footnote{«Чтобы продать что-нибудь ненужное, нужно сначала
купить что-нибудь ненужное. А у нас денег нет!»} (aka бюджетное ограничение):
$4n_{1}+16n_{2}+25n_{3}=7000$

Решаем Лагранжем и получаем ответ: 35, 35, 252.

Некоторые маньяки наизусть знают:

$n_{i}=\frac{C}{\sum w_{i}\cdot \sigma_{i}\cdot\sqrt{c_{i}}}\frac{w_{i}\cdot \sigma_{i}}{\sqrt{c_{i}}}$
\item[9-А.] Замечание: неудачные переноски считаются, так как иначе решение
тривиально — пробовать нести по 1000 шашек.
\begin{enumerate}
\item  Так как $p$ небольшая будем считать, что $\ln(1-p)\approx -p$. Уже страшно, да?
\item Допустим, что $s(n)$ оптимальная стратегия, указывающая, сколько нужно брать
сейчас шашек, если осталось перенести $n$ шашек. Возможно, что $s$ зависит от $n$.
Обозначим $e(n)$ ожидаемое количество переносов при использовании оптимальной стратегии.
\item Начинаем:

$s(1)=1$, $e(1)=1/(1-p)$

$s(n)=\argmin_{a}(1/(1-p)^{a}+e(n-a))$, $e(n)=\min_{a}(1/(1-p)^{a}+e(n-a))$

Замечаем, что поначалу (где-то до $1/p$ шашек) все идет хорошо, а затем плохо...
\item Ищем упрощенное решение вида $s(n)=s$.

Ожидаемое число переносок равно $\frac{1000}{s}\frac{1}{(1-p)^{s}}$

Минимизируем по $s$. Получаем: $s=-1/\ln(1-p)\approx 1/p$.
\item Для тех кому интересно, точный график (10000 шашек, p=0.01):
\end{enumerate}

$[$Здесь оставлено место для картины Усама-Бен-Ладен, будь он не ладен, таскает шашки.$]$
%\begin{figure}[h]
%    \includegraphics{usama.eps}
%\end{figure}

ps. В оригинале мы сканировали ксерокопию учебника Микоша. Сканер был очень умный:
в него нужно положить стопку листов, а на выходе он выдавал готовый pdf файл.
Проблема была в том, что он иногда жевал бумагу. В этом случае он обрывал
сканирование и нужно было начинать все заново. Возник вопрос, какого размера должна
быть партия, чтобы минимизировать число подходов к ксероксу.

\item[9-Б.]
\begin{enumerate}
\item Если сейчас 0 долларов, то брать 1 доллар.

Назовем ситуацию, «шоколадной» если можно выиграть без риска. То есть если игр
осталось больше, чем недостающее количество денег.
\item Если игрок не в шоколаде, то оптимальным будет рисковать на первом ходе.

Почему? Получение одного доллара можно перенести на попозже.
\item В любой оптимальной стратегии достаточно одного успеха для выигрыша.

Почему? Допустим стратегии необходимо два успеха в двух рискованных играх. Заменим
их  на одну рискованную игру. Получим большую вероятность.

Оптимальная стратегия:

Если сейчас 0 долларов, то брать доллар.

Пусть $d$ — дефицит в долларах, а $k$ — число оставшихся попыток.

Если $d\le k$, то брать по доллару.

Если $d>k$, то с риском попробовать захапнуть $1+d-k$ долларов.
\end{enumerate}
\end{enumerate}



\subsection[2007-2008 Демо-версия]{\hyperref[sec:kr_01_2007_2008_demo]{2007-2008 Демо-версия}}
\label{sec:sol_kr_01_2007_2008_demo}


\begin{enumerate}
\item $c = 0.4$

$\P(Y>X) = \P(Y=1, X=-1) + \P(Y=2, X=-1) + \P(Y=2, X=1) = 0.7$

$\E(XY) = 0.1 -0.4 - 0.4 -0.1 + 0.1 +0.2 =-0.5$

$\E(X|Y>0) = -1\cdot\frac{0.6}{0.8} + 1\cdot \frac{0.2}{0.8} = -0.5$

Случайны величины $X$ и $Y$ не являются независимыми.
\item
\begin{enumerate}
\item Найдём распредление случайной величины $Z = X_1 + X_2$:
\[
\E(Z) = -1, \Var(Z) = \Var(X_1) + \Var(X_2) + 2 \Cov(X_1, X_2) = 37
\]
Получили, что $Z \sim \cN(-1, 37)$, тогда
\[
\P(Z>0) = \P\left(\frac{Z+1}{\sqrt{37}} > \frac{0+1}{\sqrt{37}}\right) = 0.4364
\]
\item $\E(X_1 | X_2 = -1) = -2 -4 \cdot \frac{1}{36} \cdot (-1 -1) = -\frac{16}{9}$

$\Var(X_1 | X_2 = -1) = 9 - (-4) \cdot \frac{1}{36} \cdot (-4) = \frac{77}{9}$

$X_1 | X_2 = -1 \sim \cN(-\frac{16}{9}, \frac{77}{9})$
\end{enumerate}
\item $c=6$

\[
	\P(3Y>X) = \int_{0}^{1} \int_{x/3}^{x} 6(x-y) dy dx = \int_0^1 \frac{4}{3} x^2 \, dx = 4/9
\]

$f_{X} (x) = \int_{0}^{x} 6(x-y) dy = 3x^2 \Rightarrow \E(X) = \int_{0}^{1} 3x^3 dx = 0.75$
\item Введём следующие случайные величины:

$
X = \begin{cases}
1 & \text{в субботу не будет дождя}, p=0.5 \\
0 & \text{иначе}, p=0.5
\end{cases}
$
\hspace{0.5cm}
$
Y = \begin{cases}
1 & \text{в воскресенье не будет дождя }, p=0.7 \\
0 & \text{иначе }, p=0.3
\end{cases}
$

Найдем их математические ожидания и дисперсии: $\E(X)=0.5$, $\Var(X)=0.25$,
$\E(Y)=0.3$, $\Var(Y)=0.21$.

В условии дана корееляция $X$ и $Y$, найдём ковариацию: $\Cov(X, Y) =
r \cdot 0.5 \sqrt{0.21}$.
По определению, $\Cov(X, Y) = \E(XY)-\E(X)\E(Y)$, откуда можно найти $\E(XY)$:
$\E(XY) = r \cdot 0.5 \sqrt{0.21} + 0.5 \cdot 0.7$.

Заметим, что $\E(XY)$ — это и есть искомая вероятность, потому что при подсчёте
совместного математического ожидания в~сумме будет только одно слагаемое, в~котором
$X = 1$ и $Y = 1$, остальные же будут равны нулю.
\item Пусть $X$ — случайная величина, обозначающая количество проданных книг.
Будем считать, что продажи каждой книги — независимые события.

$\E(50 + 5X) = 100$, $\Var(50 + 5X) = 25$
\item Пусть $X_i$ — случайная величина, обозначающая изменение цены акции очередной за день,
a $S$ —  финальную стоимость акции.
\begin{enumerate}
\item $\E(S) = \E(1000 + X_1 + \ldots + X_{60}) = 1000 + 60 (0.5 \cdot 3 - 0.5 \cdot 5) = 940$

$\Var(S) = \Var(1000 + X_1 + \ldots + X_{60}) = 60(0.5 \cdot 9 + 0.5 \cdot 25 - 1) = 960$
\item $\P(S > 900) = \P\left(\frac{S-940}{\sqrt{960}} > \frac{900-940}{\sqrt{960}} \right) =
\P(\cN(0,1) > -1.29) = \P(\cN(0, 1) < 1.29) \approx 0.9$
\end{enumerate}
\item $\P\left(|\hat{p} - 0.6| <0.01\right) = 0.99 \Rightarrow \P\left(\frac{|\hat{p}
- 0.6|}{\sqrt{\frac{0.6\cdot0.4}{n}}} < \frac{0.01}{\sqrt{\frac{0.6\cdot0.4}{n}}} \right) = 0.99 $

По таблице: $\frac{0.01}{\sqrt{\frac{0.6\cdot0.4}{n}}}  = 2.57$, следовательно, $n = 15851$.
\item
\begin{enumerate}
\item $\P(-2 < \cN(0,1) < 2) = 0.9544$, $1 - \frac{1}{4} < \P(-2\sigma<X-\mu<2\sigma) < 1$
\item $\P(8<X<12) = 0.2$, $1 - \frac{20^2}{12} < \P(-2 < X - \E(X) < 2)< 1$
\item $\P(-1 < X < 3) = \int_{-1}^{3} e^{-x} dx \stackrel{x>0}{=} 1 - e^{-3}$, $1- \frac{1}{4} < \P(-2<X-\E(X)<2) < 1$

\item[9-А.]

Если убийц нечетное число, то в живых останется только один убийца.
Если убийц чётное число, то в живых либо не останется никого, либо мирные граждане.

Следовательно, если кроме гостя в городе нечётное число убийц $u$, то шансов у гостя
никаких нет. Если гость мирный, то в живых останется в финале один из убийц,
если гость — убийца, то в финале убийц не останется.

Если кроме гостя в городе чётное число убийц $u$, и гость будет новым нечётным убийцей,
то в финале останется один убийца, и вероятность выжить для гостя равна $1/(u+1)$.

Рассмотрим случай чётного числа убийц и мирного гостя. Замечаем, что прочие мирные
лишь отдаляют по времени разборки и не влияют на вероятность выжить гостя.
Поэтому уберём остальных мирных жителей.

Чтобы гость выжил, все встречи должны быть между убийцами. Следовательно, вероятность
выжить равна:

\[
\frac{u}{u-1}\frac{u-1}{u-2} \cdot \frac{u-2}{u-3}\frac{u-3}{u-4}\cdot \ldots
\cdot \frac{2}{3}\frac{1}{2} = \frac{1}{u+1}
\]

От стратегии гостя ничего не зависит.
И вероятность выжить гостя либо 0, либо $1/(u+1)$ в зависимости от чётности числа убийц.

\item[9-Б.]
Подразумевая под точками концы гирлянды, докажем следующее утверждение.

Бросим $2n \geq 4$ точек $X_1, X_2, \ldots, X_{2n}$ случайным образом на отрезок
$[0;1]$. Пусть для $1 \leq i \leq n$ $J_i$ — это отрезок с концами $X_{2i-1}$ и $X_{2i}$.
Тогда вероятность того, что найдётся такой отрезок $J_i$, который пересекает все
другие отрезки, равна $2/3$ и не зависит от $n$.

Доказательство. Бросим $2n+1$ точек на окружность, тогда $2n$ точек образуют пары,
а оставшуюся обозначим $X$ и будем считать её и началом, и концом отрезка. Каждому
получившемуся отрезку присвоим уникальное имя. Далее, будем двигаться от точки $X$
по часовой стрелке до тех пор, пока не встретим одно и то же имя дважды, например «а».
После этого пойдём в обратную сторону, и будем идти, пока не встретим какое-нибудь
другое имя дважды, например, «б». Теперь посмотрим на получившуюся последовательность
между «б» и «а» на концах пути, читая её по часовой стрелке от «б» до «а». Нас
интересует взаимное расположение $X$, второй «а» и второй «б». Зная, что «а» стоит
после $X$, выпишем все возможные случаи, где может стоять «б»:
\begin{enumerate}
\item перед $X$
\item между $X$ и «а»
\item после «а»
\end{enumerate}
Покажем, что во втором и третьем случае отрезок «б» пересекает все остальные, а в
первом такого отрезка вообще нет. Попутно заметим, что появление каждого и случаев
равновероятно.

Действительно, если «б» стоит после $X$, и отрезок соответствующий этому имени,
не пересекает какой-нибудь другой отрезок «в», то последовательность выглядела бы
как «бвв$X$б» или «б$X$ввб», что противоречит описанному построению.
Если «б» стоит перед $X$ и отрезок «в» пересекает оба отрезка «а» и «б», то мы
снова приходим в противоречие с построением. В итоге, получаем, что искомая вероятность
равна $2/3$.
\end{enumerate}
\end{enumerate}



\subsection[2007-2008]{\hyperref[sec:kr_02_2007_2008]{2007-2008}}
\label{sec:sol_kr_02_2007_2008}

\begin{enumerate}
\item $c = 0.3$, далее $\P\left(Y>-X\right)=0.4$  и $\E\left(X\cdot Y\right)=0.1$

$\Corr(X,Y)=\frac{-0.02}{\sqrt{0.24\cdot 1.41}}$

$\E\left(Y|X>0\right)=0.25$
\item
\begin{enumerate}
\item $\E(S)=-1$, $\Var(S)=207$, $\P(Z>1.47)=1-0.9292=0.0708$
\item $p(x_{1}|0)\sim \exp\left(-\frac{1}{2}\left(\begin{array}{cc} {x_{1}-2} & {0+1} \end{array}\right) \left(\begin{array}{cc} {9} & {-4.5} \\ {-4.5} & {25}
\end{array}\right)^{-1}\left(\begin{array}{c} {x_{1}-2} \\ {0+1}
\end{array}\right)\right)$

$p(x_{1}|0)\sim \exp\left(-\frac{1}{2det}(25(x_{1}-2)^{2}+9(x_{1}-2)+9)\right)$

$p(x_{1}|0)\sim \exp\left(-\frac{1}{2\cdot 8.19}(x_{1}-1.82)^{2}\right)$

$\Var(X_{1}|X_{2}=0)=8.19$, $\E(X_{1}|X_{2}=0)=1.82$

Есть страшные люди, которые наизусть помнят, что:

$\Var(X_{1}|X_{2}=x_{2})=(1-\rho^{2})\sigma_{1}^{2}$

$\E(X_{1}|X_{2}=x_{2})=\mu_{1} + \rho\frac{\sigma_{1}}{\sigma_{2}}(x_{2}-\mu_{2})$
\end{enumerate}
\item $\P(Y>2X)=\int_{0}^{1}\int_{0}^{y/2}(x+y)dxdy=\frac{5}{24}$

$\E(X)=\int_{0}^{1}\int_{0}^{1}x(x+y)dxdy=\frac{7}{12}$

Зависимы
\item Рассмотрим $X=8-($Васин бал$)$ и $Y=($Петин бал$)-6$

$\Corr(X,Y)=-0.7$ (при линейном преобразовании может поменяться только знак корреляции)

$\Var(X)=\frac{1}{2}\left(1-\frac{1}{2}\right)$

$\Var(Y)=\frac{1}{3}\left(1-\frac{1}{3}\right)$

Интересующая нас величина - это $\P(X=1\cap Y=1)=\E(XY)=\Cov(X,Y)+\E(X)\E(Y)$

answer: $\frac{10-7\sqrt{2}}{60}\approx 0.001675$

key point: $\Cov=-\frac{7\sqrt{2}}{60}$
\item $\frac{37}{40}=0.925$
\item Частая ошибка в «а» — решение другой задачи, где проценты заменены на копейки.

Пусть $N$ — число подъемов акции.
\begin{enumerate}
\item
\begin{multline*}
\P(100\cdot 1.08^N\cdot 0.95^{64-N}>110)= \P(N \ln(1.08)+(64-N) \ln(0.95)> \ln(1.1))= \\
= \P\left(N>\frac{\ln(1.1)-64\ln(0.95)}{\ln(1.08)-\ln(0.95)}\right)
\end{multline*}
Заметим, что $N$ - биномиально распределена, примерно $N\left(64\cdot\frac{1}{2},64\cdot\frac{1}{4}\right)$

$Z=\frac{N-32}{4}$ - стандартная нормальная и $\P(Z>-1,42)=0.92$
\item $\E(N\ln(1.08)+(100-N)\ln(0.95))$

На этот раз $\E(N)=50$ и $\E(\ln(P_{100}))=1.28$
\end{enumerate}
\item $p_{break}=1-\exp(-5/7)=0.51=\int_{0}^{5}\frac{1}{7}e^{-\frac{t}{7}}dt$

$p=0.8\cdot 0.51\approx 0.4$

$\E(S)=1000p=400$, $\Var(S)=1000p(1-p)=240$

$\P(S>350)=\P(Z>-3.23)\approx 1$
\item
\begin{enumerate}
\item $\P(X^{2}>2.56\Var(X))=\P(|X-0|>1.6\sigma)\le
\frac{\Var{X}}{2.56\Var(X)}=\frac{100}{256}\approx 0.4$
\item $\P(X^{2}>2.56\Var(X))=\P(|Z|>1.6)=0.11$
\end{enumerate}
\item[9-А.] б) Искомая вероятность равна $\P(A)=f(k+1,n-k)/f(k,n-k)$, где

$f(a,b)=\int_{0}^{1}x^{a}(1-x)^{b}dx$

Проинтегрировав по частям, видим, что $f(a,b)=f(a+1,b-1)\frac{b}{a+1}$

Отсюда $f(a,b)=\frac{a!b!}{(a+b+1)!}$

Подставляем, и получаем: $\P(A)=\frac{k+1}{n+2}$

Если кто получит этот ответ более интуитивным образом — тому большой
дополнительный балл (!) — обращайтесь на \href{mailto:boris.demeshev@gmail.com}{boris.demeshev@gmail.com}

\item[9-Б.] Занумеруем детей в порядке появления на свет. Обозначим $M_{i}$ —
индикатор того, что $i$-ый ребенок — мальчик, и $F_{i}$ — индикатор того, что
$i$-ый ребенок — девочка. Конечно, $F_{i}+M_{i}=1$ и $F_{i}M_{i}=0$.
$M$, $F$ — общее число мальчиков и девочек соответственно.

Запасаемся простыми фактами:

$\E(F_{i})=\E(M_{i})=\E(F_{i}^{2})=\E(M_{i}^{2})=\frac{1}{2}$

$\E(F)=\E(M)=\frac{n}{2}$

$\Var(F_{i})=\Var(M_{i})=\frac{1}{4}$

$\Var(F)=\Var(M)=\frac{n}{4}$

$\E(F^{2})=\E(M^{2})=\Var(F)+\E(F)^{2}=\frac{n(n+1)}{4}$

$\E(FF_{i})=\frac{n+1}{4}$

Заметим, что $X_{i}=X_{i}+M_{i}F_{i}=M_{i}F$. Таким образом,

$X=MF=nF-F^{2}$

$Y_{i}=F-F_{i}-X_{i}$

$Y=(n-1)F-MF=(n-1)F-nF+F^{2}=F^{2}-F$

Далее берём математическое ожидание (легко) и дисперсию (громоздко):  $\E(X)=\E(Y)=\frac{n(n-1)}{4}$

\ldots (если кто решил до сих пор, то наверняка, он смог и дальше решить) \ldots
\end{enumerate}



\subsection[2006-2007]{\hyperref[sec:kr_02_2006_2007]{2006-2007}}
\label{sec:sol_kr_02_2006_2007}

\begin{enumerate}
\item $c=0.3$, $\P(Y>-X)=0.4$, $\E(XY^{2})=0.5$, $\E(Y|X>0)=\frac{0.1}{0.4}=0.25$
\item $\E(Y)=-1$, $\Var(Y)=207$, $\P(Y>20)=\P(Z>\frac{21}{\sqrt{207}})=\P(Z>1.46)=0.07$
\item $\P(Y>2X)=\int_{0}^{1}\int_{0}^{y/2}(x+y)dxdy=\frac{5}{24}$

$\E(X)=\int_{0}^{1}\int_{0}^{1}x(x+y)dxdy=\frac{7}{12}$
\item Используя метод множителей Лагранжа:

$L=\frac{(0.1\cdot 24)^{2}}{a}+\frac{(0.3\cdot 12)^{2}}{b}+\frac{(0.6\cdot 10)^{2}}{c}+\lambda(10-a-b-c)$

\ldots

$a=2$, $b=3$, $c=5$, можно было использовать готовую формулу
$n_{i}=\frac{w_{i}\sigma_{i}}{\sum w_{j}\sigma_{j}}$

$\Var(\bar X^{s})=14.4$
\item
\begin{enumerate}
\item $(2\theta-0.2)(1.2-3\theta)\rightarrow\max$

$\hat{\theta}=0.25$
\item $2.4-7\hat{\theta}=1$, $\hat{\theta}=0.2$
\end{enumerate}
\item $\P(\bar X>0.33)=\P\left(\frac{\bar{X}-0.3}{\sqrt{\frac{0.3\cdot
0.7}{200}}}>\frac{0.33-0.3}{\sqrt{\frac{0.3\cdot
0.7}{200}}}\right)=\P(Z>1.03)=0.15$
\item $\bar{X}=1$, $\hat{\sigma}^{2}=1$

$\P(\hat{\sigma}^{2}>3\sigma^{2})=\P\left(2\frac{\hat{\sigma}^{2}}{\sigma^{2}}>6\right)=\P(\chi_{2}^{2}>6)=0.05$
\item
\begin{enumerate}
\item $\P(X^{2}>4\Var(X))=\P(|X-0|>2\sigma)\le
\frac{\Var(X)}{4\Var(X)}=\frac{1}{4}$
\item $\P(X^{2}>4\Var(X))=\P(|Z|>2)=0.05$
\end{enumerate}
\item
\begin{enumerate}
\item $\bar X$
\item Да
\item $\Var(\bar X)=\frac{\theta^{2}}{n}$;
\item Да: несмещенность и предел дисперсии равный нулю
\end{enumerate}
\item
\begin{enumerate}
\item $\E(\bar X)=\frac{a}{2}$
$\theta=\frac{1}{\P(X_{i}<5)}=\frac{1}{5/a}=\frac{1}{5}a$

$\hat{\theta}=\frac{2}{5}\bar X$
\item $\Var(\hat{\theta}_{n})=(\frac{2}{5})^{2}\cdot\frac{a^{2}}{12n}$
\item $\lim \Var(\hat{\theta}_{n})=0$, оценка несмещённая,
следовательно, состоятельная.
\end{enumerate}
\item[11-А.] Обозначим $e_{n}$ — сколько дней осталось в~среднем ждать, если
уже набрано $n$ копеек.

Тогда:

$e_{100}=0$

$e_{99}=1$

$e_{98}=\frac{1}{100}e_{99}+\frac{99}{100}e_{100}+1=1+\frac{1}{100}$

$e_{97}=\frac{1}{100}e_{98}+\frac{1}{100}e_{99}+\frac{98}{100}e_{100}+1=(1+\frac{1}{100}))^{2}$

$e_{96}=\frac{1}{100}e_{97}+\frac{1}{100}e_{98}+\frac{1}{100}e_{99}+\frac{97}{100}e_{100}+1=(1+\frac{1}{100})^{3}$

\ldots

По~индукции легко доказать, что $e_{n}=(1+\frac{1}{100})^{99-n}$

Таким образом, $e_{0}=(1+\frac{1}{100})^{99}=2.718 \ldots$

\item[11-Б.]  $p_{0}=0.1$, $p_{1}=0.7\cdot 0.1$;

\[
p_{n}=\P(\text{в~первый день Петя познакомился с~одной девушкой})p_{n-1}+
\P(\text{в~первый день Петя познакомился с~двумя девушками})p_{n-2};
\]

Разностное уравнение: $p_{n}=0.7p_{n-1}+0.2p_{n-2}$
\end{enumerate}



\subsection[2005-2006]{\hyperref[sec:kr_02_2005_2006]{2005-2006}}
\label{sec:sol_kr_02_2005_2006}

\begin{enumerate}
\item $\P(X=0) = e^{-0.003} \approx 0.997$
\item $c=0.3$, $\P(Y > -X) = 0.4$, $\E(X\cdot Y^2) = 0.5$, $\E(Y|X>0) = 1/3$
\item $\P(X_1 +3X_2 > 20) = \P\left(\frac{X_1 +3X_2 +1}{\sqrt{207}}  > \frac{20+1}{\sqrt{207}}\right) = \P(\cN(0,1) > 1.46) = 0.0721$

$\E(X_1 +3X_2) = -1$, $\Var(X_1 +3X_2) = 9 + 9\cdot 25 +6 \cdot(-4.5) = 207$
\item $\P(Y > X) = \int_0^1 \int_x^1 (x+y) dy dx = 0.5$

$\E(X) = \int_0^1 (x+0.5)x dx = 7/12$

% f_{X| Y>X} = \int_x^1 (x+y) dy  = -1.5 x^2 + x + 0.5

% \E(X|Y>X) = \int_0^1 (-1.5 x^2 + x + 0.5)x dx  = 5/24
\item $\P(\hat{p} < 0.21) = \P\left(\frac{\hat{p} - 0.8}{\sqrt{\frac{0.8(1-0.8)}{200}}} < \frac{0.21 - 0.8}{\sqrt{\frac{0.8(1-0.8)}{200}}} \right) \approx 0$ % проверить!
\item
\item
\begin{enumerate}
\item $0.5118$
\item $7/30$
\item $-e^{-\frac{1}{20\cdot 23}} + e^{-\frac{1}{20\cdot 16}} \approx 0.13$
\end{enumerate}
\item Если $S$ — финальная стоимость акции, то $S=1000+X_1+X_2+\ldots+X_{100}$. 
Тогда по ЦПТ $S\sim \cN(1000,100)$ и $\P(S>1030)\approx 0.001$.
\item Узнаем нормальное распределение, $\E(X) = -1$, $\Var(X) = 1/4$. Константа, для полноты картины, равна $c=\sqrt{2}/\sqrt{\pi}$.
\item
\begin{multline*}
\P(X=k|X+Y=50) = \frac{\P(X+Y =50 | X=k)\P(X=k)}{\P(X+Y=50)} = \frac{\P(Y=50-k)\P(X=k)}{\P(X+Y=50)} = \\
\frac{\left(\frac{e^{-15} 15^{50-k}}{(50-k)!}\right) \left(\frac{e^{-5} 5^k}{k!}\right)}{\frac{e^{-(5+15)}(5+15)^{50}}{50!}} = C_{50}^k \frac{5^k \cdot 15^{50-k}}{(5+15)^{50}} = C_{50}^k \left(\frac{5}{5+15}\right)^k \left(\frac{15}{5+15}\right)^{50-k}
\end{multline*}
Получили биномиальное распределение с параметрами $p=1/4$, $n=50$.
\end{enumerate}



\subsection[2004-2005]{\hyperref[sec:kr_01_2004_2005]{2004-2005}}
\label{sec:sol_kr_01_2004_2005}


\begin{enumerate}
\item $\P\left(|X-\E(X)|>2\sqrt{\Var(X)}\right) = \P\left(\frac{|X-\E(X)|}{\sqrt{\Var(X)}} > 2\right) = 2\P(\cN(0,1) >2) \approx 0.05$
\item $\mu = -1$, $\sigma^2 = 9$
\item $C_{200}^4 \left(\frac{1}{130}\right)^4 \left(\frac{129}{130}\right)^{196}$
\item $C_{20}^5 0.52^{15}0.48^9$
\item $\P(|X| > 15) \leq \frac{16}{15^2}$
\item Подготовимся: $\E(2X-Y) = -5$, $\Var(2X-Y) = 16 + \Var(Y)$

$ \P(Z>0) = 0.9 \Rightarrow \P\left(\frac{2X-Y+5}{\sqrt{16+\Var(Y)}} > \frac{5}{\sqrt{16+\Var(Y)}}\right) \Rightarrow \frac{5}{\sqrt{16+\Var(Y)}} = 1.28  $

$ \Var(Y) = 0.74$
\item $\E(X) = \Var(X) = \lambda = 0.09$
\[
\P(| X - 0.09| > 0.18) = 1 - \P(-0.18 < X - 0.09 < 0.18)  \stackrel{X\geq0}{=} 1 - \P(X=0) = 1 - e^{-0.09}
\]
\item Пусть $X$ — случайная величина, число страховых случаев, $X \sim Bin(n=1000, p=0.05)$. $S$ — размер резерва.

Тогда условие можно записать в виде: $\P(1500X \leq S) = 0.95$
\[
\P\left(\frac{X - 50}{\sqrt{1000 \cdot 0.05 \cdot 0.95}} \leq \frac{\frac{S}{1500}-50}{\sqrt{1000 \cdot 0.05 \cdot 0.95}} \right) =0.95 \Rightarrow \frac{\frac{S}{1500}-50}{\sqrt{1000 \cdot 0.05 \cdot 0.95}} \approx 1.65 \Rightarrow S \approx 92058
\]
\item
\begin{enumerate}
\item Да
\item Нет
\item $1356$
\end{enumerate}
\item Пусть $X$ — случайная величина, число сочинённых песенок в день, когда Винни-Пуха кусает пчела, $X \sim Bin(n,p)$.

Из данной в условии выборки находим $\overline{X} = 19/18$, поскольку число наблюдений достаточно велико, $\E(X) = np = 36p = 19/18$, откуда получаем $p=19/(18\cdot36)$ и $\Var(X) = np(1-p) \approx 1$.

Нормальная аппроксимация: $X \sim \cN(19/18, 1)$.
\[
\P(X>1) = \P\left(X - \frac{19}{18} > -\frac{1}{18} \right) \approx 0.52
\]
\item
\begin{enumerate}
\item Заметим, что величину $X_t$ можно представить в виде:
\[
X_t = A_t \cdot X_{t-1} = A_t \cdot A_{t-1} \cdot X_{t-1} = \ldots = A_t \cdot A_{t-1} \cdot \ldots \cdot A_{2} \cdot X_1
\]
Тогда и предел тоже можно переписать:
\[
\lim_{n\to\infty}\frac{\ln X_n}{n} = \lim_{n\to\infty} \frac{\ln A_n + \ldots + \ln A_2 + \ln X_1}{n} \stackrel{X_1 = 1}{=} \lim_{n\to\infty} \frac{\ln A_n + \ldots + \ln A_2}{n} \stackrel{\text{ЗБЧ}}{=} \E(\ln A_1)
\]
Осталось найти математическое ожидание $\ln A_1$:
\[
\E(\ln A_1) = \int_0^{2a} \frac{1}{2a} \cdot \ln x dx = \ln(2a)-1
\]
\item Из неравенста $\ln(2a)-1>0$ получаем, что темп роста будет положительным при $a>e/2$.
\end{enumerate}
\end{enumerate}

% \thispagestyle{empty}
\section{Решения промежуточных экзаменов}


\subsection[2017-2018]{\hyperref[sec:midterm_exam_2017_2018]{2017-2018}}
\label{sec:sol_midterm_exam_2017_2018}

Здесь табличка с ответами

% !TEX root = ../probability_hse_exams.tex
\thispagestyle{empty}
\section{Решения контрольной номер 3}



\subsection[2019-2020]{\hyperref[sec:kr_03_2019_2020]{2019-2020}}
\label{sec:sol_kr_03_2019_2020}



\subsection[2018-2019]{\hyperref[sec:kr_03_2018_2019]{2018-2019}}
\label{sec:sol_kr_03_2018_2019}
\subsubsection*{Задачи}
\begin{enumerate}
	
	%Задача №1
  \item
  \item
  \item
  \item
  \item
	\item 
	Пусть $X_1, \ldots, X_5$ — цены кокошников. $\E (X_i) = 3500, \Var (X_i) = 500^2 = 250000, \ i = 1, \ldots, 5$.

 Василиса выбирает наугад и равновероятно 3 кокошника, обозначим их цены за $Y_1, Y_2, Y_3$.

Математическое ожидание вырученных Василисой денег равно  
\[ 
\E (Y_1 + Y_2 + Y_3) = 
\E (Y_1)  + \E (Y_2) + \E (Y_3) = 3 \cdot 3500 = 10500.
\] 

 Аналогично, дисперсия вырученных Василисой денег равна

\[ 
\Var (Y_1 + Y_2 + Y_3) = \Var (Y_1) + \Var (Y_2) + \Var (Y_3) + 2 \Cov (Y_1, Y_2) + 2 \Cov (Y_1, Y_3) + 2 \Cov (Y_2, Y_3)
\]

 Найдем ковариации соответствующих случайных величин, используя следующее свойство: 
 \[ \Cov (X_1, X_1 + X_2 + \ldots + X_5) = 0\]

  Это верно в силу того, что $X_1 + X_2 + \ldots + X_5 = const $

 А так как ковариации $X_1$ со всеми остальными $X_i$ будут одинаковы, то 
\[  \Cov (X_1, X_1 + X_2 + \ldots + X_5) = \Var (X_1) + 4 \Cov (X_1, X_i) = 0, \] 

 Отсюда получаем 
\[ \Cov (X_1, X_i) = - \frac{1}{4} \cdot \Var (X_1) \]
 
Получим: 
\[
 \Var (Y_1 + Y_2 + Y_3) = 3 \cdot \Var (Y_1) - 3 \cdot 2 \cdot \frac{1}{4} \cdot \Var (Y_1) 
  = \frac{3}{2}  \Var (Y_1) = \frac{3}{2} \cdot 250000 = 375000 
 \]
\item
В задаче один оцениваемый параметр $\theta$. Тогда для получения MM-оценки достаточно одного момента.

Выборочный первый начальный момент:
\[
\frac{\sum_{0}^{n}{x_{i}}}{n} = \bar X
\]

Теоретический первый начальный момент: 
\[
\E (X) = \int_{0}^{\theta} x \cdot \frac{2x}{\theta^{2}} dx = \left. \frac{4x}{\theta^{2}} \right|_{0}^{\theta} = \frac{4}{\theta}
\]
Приравняв выборочный и теоретический моменты, получаем $\hat \theta_{MM}=\frac{4}{\bar X}$
\item
\item

\item

\[
L(X, \theta)  = f(X_1, \theta) \cdot \ldots \cdot f(X_n, \theta) = \frac{1}{\theta^n}, \text{ если } X_i < \theta \text{ для любого } i 
\]

Значит $\hat{\theta}$ должна быть наименьшей при которой она ещё остаётся больше каждого $X_i$, $\hat{\theta} = \max\{X_1, \ldots, X_n\}$.

\end{enumerate}


\subsection[2017-2018]{\hyperref[sec:kr_03_2017_2018]{2017-2018}}
\label{sec:sol_kr_03_2017_2018}


\begin{enumerate}
\item[5.]
\begin{enumerate}
\item $L(X_1, \ldots, X_n, \mu) = \prod_{i=1}^n \frac{1}{\sqrt{2\pi}} e^{-\frac{1}{2}\sum_{i=1}^n (X_i - \mu)^2}$
\item $\hat\mu_{ML} = \bar X$
\item $\E(\hat\mu_{ML}) = \E(\bar X) = \mu \Rightarrow$ оценка несмещённая

$\plim \hat \mu_{ML} = \plim \bar X = \mu \Rightarrow$ оценка состоятельная
\item $I(\mu) = n$
\item $\Var(\theta) \geq \frac{1}{I(\theta)}$
\item $\Var(\hat \mu_{ML}) = \frac{1}{n}$, так как неравенство Рао-Крамера выполнено
как равенство, оценка является эффективной.
\item $\theta = \E\left(X^2\right) = \Var(X) + \mu^2 = 1 + \mu^2$.
Тогда в силу инвариантности оценок максимального правдоподобия: $\hat\theta_{ML} = 1 + \hat\mu^2$.
\item $\E(\hat \theta_{ML}) = 1 + \E(\hat \mu^2) = 1 + \E((\bar X)^2)$

Пользуясь соотношением $\E((\bar X)^2) = \Var(\bar X) + (\E(\bar X))^2$,
получим: $\E(\hat \theta_{ML}) = 1 + \frac{1}{n} + \mu^2$, то есть оценка смещена.

Однако, $\lim_{n \to \infty} \left(1 + \frac{1}{n} + \mu^2\right) = 1 + \mu^2$, значит,
оценка асимптотически несмещена.
\item $\hat \theta_{ML} \approx 1 + \mu^2 + 2\mu(\hat \mu - \mu)$

$\Var(\hat \theta_{ML}) \approx 4 \mu^2 \Var(\hat \mu) = \frac{4 \mu^2}{n}$
\item Так как $\hat \theta_{ML}$ асимптотически несмещена, то для проверки
состоятельности достаточно показать, что
$\Var(\hat \theta_{ML}) = \frac{4\mu^2}{n} \to_{n \to \infty} 0$.
\end{enumerate}
\item[6.]
\begin{enumerate}
\item $\E(X_1) = \int_{0}^{\theta} \frac{2}{\theta^2}(\theta - x)x dx = \frac{\theta}{3}$

$\frac{\hat \theta_{MM}}{3} = \bar X \Rightarrow \hat \theta_{MM} = 3 \bar X$
\item Оценка $\hat \theta$ состоятельна. если $\plim \hat \theta_n = \theta$.

$\plim \hat \theta_{MM} = \plim 3 \bar X = 3 \E(X_1) = \theta \Rightarrow$ оценка состоятельна.
\end{enumerate}
\item[7.]
\begin{enumerate}
\item $\E\left(\frac{X_1 + X_2 + X_3}{3} \right) = \frac{1}{3} \cdot 3 \E(X_1) = 132.5$

$\Var\left(\frac{X_1 + X_2 + X_3}{3} \right) = \frac{1}{9} \Var(X_1 + X_2 + X_3) =
\frac{1}{9} (\Var(X_1) + \Var(X_2) + \Var(X_3) + 2 \Cov(X_1, X_2) + 2\Cov(X_1, X_3) + 2\Cov(X_2, X_3)) =
\frac{1}{9}(3\Var(X_1) + 6\Cov(X_1,X_2))$

$\Var(X_1) = \E(X_1^2) - \E(X_1)^2 = \frac{1}{4} \cdot 30^2 + \frac{1}{4} \cdot 500^2 - 132.5^2 = 45168.75$

$\Cov(X_1, X_1 + \ldots + X_4 = \Var(X_1) + 3\Cov(X_1,X_2) = 0 \Rightarrow \Cov(X_1,X_2) = -\frac{45168.75}{3} = -15056.25$

$\Var\left(\frac{X_1 + X_2 + X_3}{3} \right) = 5018.75$

\item Вовочке удастся войти в метро, учитывая, что стоимость проезда по тройке составляет 35~рублей, только если одной из выбранных карт будет карта с 500~рублями на счету.

Всего Вовочка может выбрать три карты $C_4^3$ способами. Если одна из выбранных карт $-$ карта с~500~рублями, то выбрать оставшиеся две карты можно $C_3^2$ способами.

Тогда вероятность того, что одной из выбранных карт будет карта с 500 рублями равна $\displaystyle\frac{C_3^2}{C_4^3} = \frac{3}{4}$

\end{enumerate}
\item[8.] $\Delta_i = X_i - Y_i \sim \cN(\mu_x - \mu_y, \sigma^2)$

$\bar X = 297.5$, $\bar Y = 247.5$, $\bar \Delta = \bar X - \bar Y = 50$

$\hat \sigma^2 = \frac{1}{n-1} \sum_{i=1}^n (\Delta_i - \bar \Delta)^2 = 18266.(6)$.

Критическое значение — $t_{0.975, 3} = 3.182$ и доверительный интервал имеет вид:
\[
50 - 3.182 \sqrt{\frac{18266.(6)}{4}} < \mu_x - \mu_y < 50 + 3.182 \sqrt{\frac{18266.(6)}{4}}
\]
Так как $0$ входит в доверительный интервал, нельзя отвергнуть предположение о равенстве расхожов.
\item[9.]
\begin{enumerate}
\item $0.7 - 1.96 \sqrt{\frac{0.7 \cdot 0.3}{60}} < p < 0.7 + 1.96 \sqrt{\frac{0.7 \cdot 0.3}{60}} $
\item Да, так как $0.7667$ входит в доверительный интервал.
\item $\P(|p - \hat p| \leq 0.01) = 0.95$

$\P\left(\frac{|0.7 - p|}{\sqrt{\frac{0.7 \cdot 0.3}{n}}} < \frac{0.01}{\sqrt{\frac{0.7 \cdot 0.3}{n}}} \right) = 0.95$

$\frac{0.01}{\sqrt{\frac{0.7 \cdot 0.3}{n}}} = 1.96 \Rightarrow n \approx 8068$
\end{enumerate}
\end{enumerate}


\subsection[2016-2017]{\hyperref[sec:kr_03_2016_2017]{2016-2017}}
\label{sec:sol_kr_03_2016_2017}


\begin{enumerate}
\item
\begin{enumerate}
\item $-2, 1, 4, 7, 10$
\item $4$
\item $S^2 = \frac{1}{n} \sum_{i=1}^n (X_i - \bar{X})^2 = 18$
\item $\hat\sigma^2 = \frac{1}{n-1} \sum_{i=1}^n (X_i - \bar{X})^2 = 22.5$
\item $\frac{1}{n} \sum_{i=1}^n X_i^2 = 34$
\item $F_n(x) = \begin{cases}
0, & x < -2 \\
\frac{1}{5}, & -2 \leq x < 1 \\
\frac{2}{5}, & 1 \leq x < 4 \\
\frac{3}{5}, & 4 \leq x < 7 \\
\frac{4}{5}, & 7 \leq x < 10 \\
1, & x \geq 1
\end{cases}$
\end{enumerate}
\item $\E(X_1 + X_2) = 2 \cdot 11000 = 22000$

$\Var(X_1+X_2) = \Var(X_1) + \Var(X_2) + 2\Cov(X_1, X_2) = \Var(X_1) + \Var(X_2) - \frac{2\Var(X_1)}{N-1} = 2 \cdot 3000 - \frac{2\cdot3000}{3-1} = 3000$
\item
\begin{enumerate}
\item Необходимо решить следующую задачу:
\[
\begin{cases}
\frac{0.4^2 \cdot 10^2}{n_1} + \frac{0.5^2 \cdot 30^2}{n_2} + \frac{0.1^2 \cdot 60^2}{n_3} \to \min_{n_1, n_2, n_3} \\
150 n_1 + 300 n_2 + 600 n_3 \leq 30000
\end{cases}
\]
Выпишем функцию Лагранжа и найдём её частные производные по $n_1$, $n_2$ и $n_3$:
\begin{align*}
L(n_1, n_2, n_3, \lambda) &= \frac{0.4^2 \cdot 10^2}{n_1} + \frac{0.5^2 \cdot 30^2}{n_2} + \frac{0.1^2 \cdot 60^2}{n_3} + \lambda (150 n_1 + 300 n_2 + 600 n_3 - 30000) \\
\frac{\partial L}{\partial n_1} &= -\frac{0.4^2 \cdot 10^2}{n_1^2} + 150 \lambda \quad \Rightarrow \quad 150 \lambda = \frac{0.4^2 \cdot 10^2}{n_1^2} \\
\frac{\partial L}{\partial n_2} &= -\frac{0.5^2 \cdot 30^2}{n_2^2} + 300 \lambda \quad \Rightarrow \quad 150 \lambda = \frac{0.5^2 \cdot 30^2}{2n_2^2} \\
\frac{\partial L}{\partial n_2} &= -\frac{0.1^2 \cdot 60^2}{n_3^2} + 600 \lambda \quad \Rightarrow \quad 150 \lambda = \frac{0.1^2 \cdot 60^2}{4n_3^2}
\end{align*}
Выразим $n_2$ и $n_3$ через $n_1$:
\begin{align*}
\frac{0.4 \cdot 10}{n_1} = \frac{0.5 \cdot 30}{\sqrt{2}n_2} \Rightarrow n_2 = \frac{15n_1}{4\sqrt{2}} \\
\frac{0.4 \cdot 10}{n_1} = \frac{0.1 \cdot 60}{2n_3} \Rightarrow n_3 = \frac{6n_1}{8}
\end{align*}
Подставим вcё в бюджетное ограничение:
\[
150 n_1 + 300 \cdot \frac{15n_1}{4\sqrt{2}} + 600 \cdot \frac{6n_1}{8} = 30000
\]
Откуда получаем: $n_1 = 21.5 \approx 22$, $n_2 \approx 57$, $n_3 \approx 16$.
\item
$\Var(\bar{X}_S) = \sum_{l=1}^L \frac{w_l^2 \cdot \sigma_l^2}{n_l}
= \frac{0.4^2 \cdot 10^2}{22} + \frac{0.5^2 \cdot 30^2}{57} + \frac{0.1^2 \cdot 60^2}{16}
\approx 6.92$
\end{enumerate}
\item $\hat{p} = \frac{8000}{12300000} = \frac{2}{3075}$,
$\sqrt{\frac{\hat{p}(1-\hat{p})}{n}} \approx 7.27 \cdot 10^{-6}$, $z_{\frac{\alpha}{2}} = 1.96$

$\frac{2}{3075} - 1.96 \cdot 7.27 \cdot 10^{-6} < p < \frac{2}{3075} + 1.96 \cdot 7.27 \cdot 10^{-6}$

$0.00064 < p < 0.00066$

Поскольку $0$ не входит в доверительный интервал, утверждать, что доля статистически
не отличается от нуля нельзя.

\item
\begin{enumerate}
  \item
  \begin{enumerate}
    \item $\bar{Y} = 43$, $\hat{\sigma}_Y^2 = 32.5$, $t_{0.005, 4} = 4.6$

    $43 - 4.6 \cdot \sqrt{\frac{32.5}{5}} < \mu < 43 + 4.6 \cdot \sqrt{\frac{32.5}{5}}$

    $31.27 < \mu < 54.72$

    \item $\chi^2_{0.95, 4} = 9.49$, $\chi^2_{0.05, 4} = 0.71$

    $\frac{32.5 \cdot 4}{9.49} < \sigma^2 < \frac{32.5 \cdot 4}{0.71}$

    $13.7 < \sigma^2 < 183$
  \end{enumerate}
  \item
  \begin{enumerate}
  \item
  $X_1, \ldots, X_{n_X} \sim \cN(\mu_X, \sigma^2_X)$, $Y_1, \ldots, Y_{n_Y} \sim \cN(\mu_Y, \sigma^2_Y)$,
  $\sigma^2_X = \sigma^2_Y = \sigma^2_0$, выборки независимы

  \item $\bar{Y} - \bar{X} = 43 - 37 = 6$

  $\hat{\sigma}^2_0 = \frac{\sum_{i=1}^{n_X} (X_i - \bar X)^2 + \sum_{i=1}^{n_Y} (Y_i - \bar Y)^2}{n_X + n_Y - 2} = \frac{680+130}{5+5-2} = 101.25$

  $t_{0.95, 8} = 1.86$

  $6 - 1.86 \sqrt{101.25} \sqrt{\frac{1}{5} + \frac{1}{5}} < \mu_Y - \mu_X <  6 + 1.86 \sqrt{101.25} \sqrt{\frac{1}{5} + \frac{1}{5}} $

  $-5.83 < \mu_Y - \mu_X < 17.83$
  \item Да, так как ноль входит в доверительный интервал.
  \end{enumerate}
\end{enumerate}
\item
\begin{enumerate}
  \item Выборочный второй начальный момент: $\frac{1}{n} \sum_{i=1}^n X_i^2$.

  Теоретический второй начальный момент: $\E\left(X^2\right) = \Var(X) + (\E X)^2 = \theta$

  $\hat{\theta}_{MM} = \frac{1}{n} \sum_{i=1}^n X_i^2$
  \item $\E(\hat{\theta}_{MM}) = \frac{1}{n} \sum_{i=1}^n \E(X_i^2) = \theta$ —
  оценка несмещённая.
  \item $\Var(\hat{\theta}_{MM}) = \Var\left(\frac{1}{n} \sum_{i=1}^n X_i^2 \right) = \frac{1}{n^2} \sum_{i=1}^n \Var(X_i^2) = \frac{3\theta^2 - \theta^2}{n} \underset{n \to \infty}{\to} 0$
  — оценка состоятельная ($\E(X^4) = 3\theta^2$).
  \item
  \begin{align*}
    L(x,\theta) &= \prod_{i=1}^n \frac{1}{\sqrt{2\pi\theta}} \exp\left(-\frac{1}{2}\frac{x_i^2}{\theta} \right) = \frac{1}{(\sqrt{2\pi\theta})^n} \exp \left(-\frac{1}{2\theta} \sum_{i=1}^n x_i^2  \right) \\
    l(x, \theta) &= -\frac{n}{2}\ln(2\pi) - \frac{n}{2}\ln\theta -\frac{1}{2\theta} \sum_{i=1}^n x_i^2 \\
    \frac{\partial l}{\partial \theta} &= -\frac{n}{2\theta} + \frac{1}{2\theta^2} \sum_{i=1}^n x_i^2 \\
    \hat{\theta}_{ML} &= \frac{\sum_{i=1}^n x_i^2}{n}
  \end{align*}
  \item \begin{align*}
    \frac{\partial^2 l}{\partial \theta^2} &= \frac{n}{2\theta^2} - \frac{1}{\theta^3} \sum_{i=1}^n x_i^2 \\
    -\E\left( \frac{\partial^2 l}{\partial \theta^2} \right) &= -\frac{n}{2\theta^2} + \frac{1}{\theta^3} \cdot n \theta = \frac{n}{2\theta^2} \\
    I(\theta) &= \frac{n}{2\theta^2}
  \end{align*}
  \item $\Var(\hat{\theta}) \geq \frac{1}{I(\theta)}$
  \item $\Var(\hat{\theta}_{ML}) = \Var\left(\frac{\sum_{i=1}^n x_i^2}{n}\right) = \frac{1}{n^2}\cdot n \Var(X_1^2) = \frac{1}{n} (\E(X_1^4) - \E(X_1^2)^2) = \frac{2\theta^2}{n}$

  Так как $\Var(\hat{\theta}_{ML}) = \frac{1}{I(\theta)}$, $\hat{\theta}_{ML}$ — эффективная оценка.
\end{enumerate}
\item
\begin{enumerate}
\item Вспомним, что для распределения Пуассона $\E(X) = \Var(X) = \lambda$
\begin{align*}
  L(x, \lambda) &= \prod_{i=1}^n e^{-\lambda} \frac{\lambda^{x_i}}{x_i!} = e^{-n\lambda} \lambda^{\sum_{i=1}^n x_i} \prod_{i=1}^n \frac{1}{x_i!} \\
  l(x, \lambda) &= -n\lambda + \ln\lambda \sum_{i=1}^n x_i - \sum_{i=1}^n \ln x_i! \\
  \frac{\partial l}{\partial \lambda} &= -n + \frac{1}{\lambda} \sum_{i=1}^n x_i \\
  \hat{\lambda}_{ML} &= \bar{X}
\end{align*}
Значение по выборке: $\bar{X} = 14.5$
\item см. предыдущий пункт
\item $\hat \sigma^2 = \sqrt{\lambda_{ML}} = \sqrt{14.5}$
\item $\P(X=0) = \frac{\lambda^0 e^{-\lambda}}{0!} = e^{-\lambda} \Rightarrow \widehat{\P(X=0)} = e^{-\hat\lambda} = e^{-\bar{X}}$
\item $\left[14.5 - 1.96 \sqrt{\frac{14.5}{6}}; 14.5 + 1.96 \sqrt{\frac{14.5}{6}}\right]$, где $1.96$ — критическое значение $\cN(0;1)$. Конечно, этот результат верен только при больших $n$. Мы усиленно делаем вид, что $n=6$ велико. Полученный нами интервал может быть довольно далёк от 95\%-го.
\item В данном случае: $g(\hat{\lambda}) = e^{-\hat\lambda}$, $g'(\hat\lambda) = -e^{-\hat\lambda}$.
И доверительный интервал имеет вид:
\begin{align*}
  \left[e^{-\bar{X}} - 1.96 \sqrt{\frac{e^{-2\bar{X}}\bar{X}}{n}}; e^{-\bar{X}} + 1.96 \sqrt{\frac{e^{-2\bar{X}}\bar{X}}{n}} \right] \\
  \left[e^{-14.5} - 1.96 \sqrt{\frac{e^{-29}14.5}{6}}; e^{-14.5} + 1.96 \sqrt{\frac{e^{-29}14.5}{6}} \right]
\end{align*}
Снова отметим, что наш интервал может на самом деле быть далеко не 95\%-ым, так наше $n=6$ мало для серьёзного применения метода максимального правдоподобия.
\end{enumerate}
\end{enumerate}



\subsection[2015-2016]{\hyperref[sec:kr_03_2015_2016]{2015-2016}}
\label{sec:sol_kr_03_2015_2016}

\begin{enumerate}
\item Пусть случайная величина $S$ – это сумма поглощённых калорий

\begin{center}
\begin{tabular}{cccc}
\toprule
$s$ & $650$ & $800$ & $950$ \\
$\P(S = s)$ & $1/3$ & $1/3$ & $1/3$ \\ \bottomrule
\end{tabular}
\end{center}

Тогда
\begin{align*}
\E(S) &= \frac{1}{3}\cdot 650 +  \frac{1}{3}\cdot 800 +  \frac{1}{3}\cdot 950 = 800 \\
\Var(S) &= \frac{1}{3}(650-800)^2 + \frac{1}{3}(800-800)^2 + \frac{1}{3}(950-800)^2 = 15000
\end{align*}
\item Вариационный ряд: $4, 6, 11$; медиана: $6$; выборочное среднее: $7$;
несмещённая оценка дисперсии: $13$
\item Функция плотности двумерного нормального распределения имеет вид:
\begin{align*}
f(x,y) &= \frac{1}{2\pi}\cdot \frac{1}{\sigma_x \sigma_y \sqrt{1-\rho^2}} \\
&\cdot
\exp\left\{{-\frac{1}{2}\frac{1}{\sigma_x^2 \sigma_y^2\left(1-\rho^2\right)}\left[\sigma_y^2(x-\mu_x)^2-2\rho\sigma_x\sigma_y(x-\mu_x)(y-\mu_y)+\sigma_x^2(y-\mu_y)^2\right]}\right\}
\end{align*}

Откуда: $\mu_X=1$, $\mu_Y=0$, $\sigma_X = 1$, $\sigma_Y = 1$, $\rho = 0.2$

Обратите внимание, при скобке c $x$ коэффициент $\sigma^2_y$, там потом делится на их произведение!

\item
\begin{enumerate}
\item $X \sim \cN(178, 49)$
\begin{align*}
\P(X>185) &= 1  - \P(X<185) = 1- \P\left(\frac{X-178}{7} < \frac{185-178}{7}\right) \\
&= 1 - 0.8413 = 0.1587
\end{align*}
\item Нет, так как $\Cov(X, Y) = 5.6 \neq 0$
\item $Y \mid X \sim \cN\left(\mu_Y + \rho\sigma_Y\cdot\frac{X-\mu_X}{\sigma_X}; \sigma_Y^2\left(1-\rho^2\right)\right)$

$Y \mid X=185 \sim \cN(42.8;0.36)$

$\P(Y<42 \mid X=185) = \P\left(\frac{Y-42.8}{0.6} < \frac{42-42.8}{0.6}\mid X=185\right) \approx 0.09$
\end{enumerate}

\item
\begin{enumerate}
\item $\E(X) = \frac{0+2\theta}{2}\mid_{\hat{\theta}} = \bar{X}$, $\hat{\theta}_{MM} = \bar{X}$
\item $\forall \theta \in \Theta: \E\left(\hat{\theta}\right)=\theta \Rightarrow \hat{\theta}$ – несмещённая.

$\forall \theta \in \Theta, \forall \epsilon > 0 : \P\left(\vert \widehat{\theta}_n - \theta \vert > \epsilon\right) \to 0 \Rightarrow  \widehat{\theta}_n$ – состоятельная.

$\forall \theta \in \Theta: I_n^{-1} (\theta) = \Var\left(\hat{\theta}\right) \Rightarrow \hat{\theta} $ – эффективная.
\item $\E(\theta) = \E(\bar{X}) = \E(X_1) = \theta \Rightarrow \hat{\theta}$ – несмещённая оценка

$\Var\left(\hat{\theta_n}\right) = \Var\left(\bar{X}\right) = \frac{\Var(X_1)}{n} =
\frac{4\theta^2}{12\cdot n} \underset{n \to \infty}{\to} 0$; из условий
$\E\left(\widehat{\theta}_n\right) = \theta$ и $\Var\left(\widehat{\theta}_n\right)
\underset{n \to \infty}{\to} 0$ следует, что $\widehat{\theta}_n \stackrel{\P}{\to}
\theta$ при $n \to \infty$, т.е. $\widehat{\theta}_n$ является состоятельной.

\item
\begin{align*}
F_{X_{(n)}} &= \P(\max(X_1, \ldots, X_n) \leq x) = \P(X_1 \leq x) \cdot \ldots \cdot \P(X_n \leq x) = (\P(X_1 \leq x))^n \\
&= \begin{cases}
0 & \text{при } x<0 \\
\left(\frac{x}{2\theta}\right)^n & \text{при }  x \in [0, 2\theta] \\
1 & \text{при }  x > 2\theta
\end{cases}
\end{align*}

\[
f_{X_{(n)}} (x)  = \begin{cases}
0 & \text{при } x<0 \\
\frac{nx^{n-1}}{2^n \theta^n} & \text{при }  x \in [0, 2\theta] \\
0 & \text{при }  x > 2\theta
\end{cases}
\]

\begin{align*}
\E(X_{(n)}) &= \int_{-\infty}^{+\infty} x \cdot f_{X_{(n)}} (x) dx =  \int_{0}^{2\theta}	x \cdot \frac{nx^{n-1}}{2^n \theta^n} dx = \left. \frac{n}{2^n \theta^n} \cdot \frac{x^{n+1}}{n+1} \right|_{x=0}^{x=2\theta} \\
&= \frac{n}{2^n \theta^n}  \cdot \frac{2^{n+1}\cdot \theta^{n+1}}{n+1} = \frac{n2\theta}{n+1}
\end{align*}
Следовательно, $\E \left(\frac{n+1}{2n} \cdot X_{(n)}\right) = \theta$, а значит, $\tilde{\theta} = \frac{n+1}{2n} \cdot X_{(n)}$ – несмещённая оценка вида $c \cdot  X_{(n)}$
\item $\Var\left(\tilde{\theta}\right) = \frac{(n+1)^2}{4n^2} \Var(X_{(n)})$

\begin{align*}
\E\left(X_{(n)}^2\right) &= \int_{-\infty}^{+\infty} x^2 f_{X_{(n)}} (x) dx = \int_{0}^{2\theta} x^2 \frac{nx^{n-1}}{2^n \theta^n}  dx = \frac{n}{2^n \theta^n}  \int_{0}^{2\theta} x^{n+1} dx \\
&= \left. \frac{n}{2^n \theta^n} \cdot \frac{x^{n+2}}{n+2} \right|_{x=0}^{x=2\theta} = \frac{n}{2^n \theta^n} \cdot  \frac{2^{n+2}\cdot \theta^{n+2}}{n+2} = \frac{n\cdot4\cdot\theta^2}{n+2}
\end{align*}

\[
\Var(X_{(n)}) = \E\left(X_{(n)}^2\right)  - (\E(X_{(n)}))^2 = \frac{4n\theta^2}{n+2} - \frac{4 n^2 \cdot \theta^2}{(n+1)^2} = 4n\theta^2 \left(\frac{1}{n+2} - \frac{n}{(n+1)^2}\right)
\]

\[
\Var\left(\tilde{\theta}\right) = \frac{(n+1)^2}{4n^2} \Var(X_{(n)}) = \frac{(n+1)^2}{4n^2}  \cdot 4n\theta^2 \left(\frac{n^2+2n+1 - n^2-2n}{(n+2)(n+1)^2} \right) = \frac{\theta^2}{n(n+2)}
\]
Оценка $\tilde{\theta}_n$ является состоятельной, так как
$\E\left(\tilde{\theta}_n\right) = \theta$ и
$\Var\left(\tilde{\theta}_n\right) = \frac{\theta^2}{n(n+2)} \underset{n \to \infty}{\to} 0$
\item  Поскольку $\Var\left(\widehat{\theta}_n\right) = \frac{\theta^2}{3n}$,
$\Var\left(\tilde{\theta}_n\right) = \frac{\theta^2}{n(n+2)}$ при достаточно большом $n$
$\Var\left(\tilde{\theta}_n\right) < \Var\left(\widehat{\theta}_n\right)$.
Значит, при таких $n$ оценка $\tilde{\theta}_n$ будет более эффективной по сравнению
с оценкой $\widehat{\theta}_n$.
\end{enumerate}

\item
\begin{enumerate}
\item $X_i \sim Bin (n=10, p)$
\item $L(p)  = \prod_{i=1}^{n} C_{10}^{x_i} p^{x_i} (1-p)^{10-x_i}$
\item $\ln L(p) = \sum_{i=1}^{n} \ln C_{10}^{x_i} + \sum_{i=1}^n x_i \ln p + \sum_{i=1}^{n} (10-x_i)\ln (1-p) \to \max_p$

$\frac{\partial \ln L}{\partial p} = \frac{\sum_{i=1}^n x_i}{p} - \frac{\sum_{i=1}^{n} (10-x_i)}{1-p} \mid_{p=\hat{p}} = 0 \Rightarrow \hat{p} = \frac{\bar{X}}{n} = \frac{\sum_{i=1}^{n} x_i}{10n}$

$\frac{\partial^2 \ln L}{\partial p^2} = -\frac{\sum_{i=1}^n x_i}{p^2} -  \frac{\sum_{i=1}^{n} (10-x_i)}{(1-p)^2}$

\item $I(p) = -\E \left(\frac{\partial^2 \ln L}{\partial p^2}  \right) = \E \left(\frac{\sum_{i=1}^n x_i}{p^2} + \frac{\sum_{i=1}^{n} (10-x_i)}{(1-p)^2}\right) = \frac{10np}{p^2} + \frac{10n - 10np}{(1-p)^2} = \frac{10n}{p(1-p)}$

$i(p) = \frac{I(p)}{n} = \frac{10}{p(1-p)}$

\item $\Var(T) \geq \frac{1}{ni(T)}$

\item $\Var(\hat{p}_{ML}) = \Var\left(\frac{\sum_{i=1}^{n} x_i}{10n}\right) = \frac{1}{(10n)^2} n \Var(X_i) = \frac{1}{100n}10p(1-p) = \frac{p(1-p)}{10n}$

$\frac{p(1-p)}{10n} = \frac{1}{\frac{10n}{p(1-p)}} \Rightarrow$ да

\item $\E(X_i) = 10p \Rightarrow \widehat{\E(X_i)} = 10 \hat{p}_{ML} = \bar{X}$

$\Var(X_i) = 10p(1-p) \Rightarrow \widehat{\Var(X_i)} = \bar{X}\left(1-\frac{\bar{X}}{10}\right)$

\item $\hat{p} = \frac{3+4+0+2+6}{10\cdot 5} = 0.3$
\end{enumerate}

\item $L(x, \theta) = \prod_{i=1}^{n} (1 + \theta) x_i^\theta = (1+\theta)^n \prod_{i=1}^n x_i^\theta \to \max_\theta$

$\ln L (x, \theta) = n\ln (1+\theta) + \theta\sum_{i=1}^{n} \ln x_i \to \max_\theta$

$\frac{\partial \ln L}{\partial \theta} = \frac{n}{1+\theta} + \sum_{i=1}^{n} \ln x_i \mid_{\theta=\hat{\theta}} = 0 \Rightarrow \hat{\theta}_{ML} = -\frac{1}{\sum_{i=1}^{n} \ln x_i} -1$

\item $\bar{X} - 1.96 \frac{7}{\sqrt{20}} < \mu <\bar{X} + 1.96 \frac{7}{\sqrt{20}} $
\end{enumerate}



\subsection[2014-2015]{\hyperref[sec:kr_03_2014_2015]{2014-2015}}
\label{sec:sol_kr_03_2014_2015}

\begin{enumerate}
\item Пусть случайная величина $S$ – это сумма поглощённых калорий

\begin{center}
\begin{tabular}{cccc}
\toprule
$s$ & $650$ & $800$ & $950$ \\
$\P(S=s)$ & $1/3$ & $1/3$ & $1/3$ \\ \bottomrule
\end{tabular}
\end{center}

Тогда

\begin{align*}
\E(S) &= \frac{1}{3}\cdot 650 + \frac{1}{3}\cdot 800 +  \frac{1}{3}\cdot 950 = 800 \\
\Var(S) &= \frac{1}{3}(650-800)^2 + \frac{1}{3}(800-800)^2 + \frac{1}{3}(950-800)^2 = 15000
\end{align*}

\item Ответ — решение оптимизационной задачи:
\[
\begin{cases}
\Var(\bar{X}_S) = \frac{0.3^2\cdot 10^2}{n_1} + \frac{0.6^2 \cdot 30^2}{n_2} + \frac{0.1^2 \cdot 60^2}{n_3} \to min_{n_1, n_2, n_3} \\
150\cdot n_1 + 300 \cdot n_2 + 600 \cdot n_3 \leq 15000
\end{cases}
\]

\item
\begin{enumerate}
\item $\E(\hat \mu_1) = \E\left(\frac{X_1+X_2}{2}\right)  = \frac{1}{2}(\mu+\mu) = \mu \Rightarrow$  несмещённая

$\E(\hat \mu_2) = \E \left(\frac{X_1}{4} + \frac{X_2+\ldots+X_{n-1}}{2n-4} + \frac{X_n}{4}\right) = \frac{1}{4}\mu + \frac{n-2}{2n-4}\mu + \frac{1}{4}\mu = \mu \Rightarrow$ несмещённая

$\E(\bar{X}) = \E\left(\frac{X_1 + \ldots + X_n}{n}\right) = \mu \Rightarrow$ несмещённая

\item $\Var(\hat \mu_1) = \Var\left(\frac{X_1+X_2}{2}\right)  = \frac{1}{4}2\Var(X_1) = \frac{\sigma^2}{2}$

$\Var(\hat \mu_2) = \Var \left(\frac{X_1}{4} + \frac{X_2+\ldots+X_{n-1}}{2n-4} + \frac{X_n}{4}\right)  = \frac{\sigma^2}{16} + \frac{(n-2)\sigma^2}{(2n-4)^2} + \frac{\sigma^2}{16} = \sigma^2\left(\frac{1}{8} + \frac{1}{2(2n-4)} \right)$

$\Var(\hat \mu_3) = \Var\left(\frac{X_1 + \ldots + X_n}{n}\right)  = \frac{1}{n^2}n\sigma^2 = \frac{\sigma^2}{n}$
\end{enumerate}

\item
\begin{enumerate}
\item $X \sim \cN (1;1)$

$\P(X>1) = 0.5$, так как нормальное распределение симметрично относительно своего
математического ожидания.

\item $X \sim \cN (1;1)$, $2X \sim \cN(2; 4), Y \sim \cN(2, 4) \Rightarrow 2X+Y \sim \cN (4, 4)$

$\P(2X+Y > 2) = 1 - \P(2X+Y < 2) = 1 - \P\left(\frac{2X+Y - 4}{2} < \frac{2-4}{2}\right) \approx 1 - 0.16 = 0.84$

\item $Y \mid X \sim \cN\left(\mu_Y + \rho\sigma_Y\cdot\frac{X-\mu_X}{\sigma_X};
\sigma_Y^2\left(1 - \rho^2\right)\right)$, $(Y \mid X = 2) \sim \cN(1, 3)$

$\E(2X+Y \mid X=2) = 2\E (X\mid X=2) + \E(Y\mid X=2) = 4 + 1 = 5$
\end{enumerate}

\item
\begin{enumerate}
\item $X_1^2 + X_2^2 \sim \chi^2_2$, $\P(X_1^2 + X_2^2 > 6)  \approx 0.05$
\item $\P\left(\frac{X_1^2}{X_2^2+X_3^2} > 9.25\right) = \P\left(\frac{\frac{X_1^2}{1}}{\frac{X_2^2+X_3^2}{2}} > 18.5\right) \approx 0.05$, $\frac{\frac{X_1^2}{1}}{\frac{X_2^2+X_3^2}{2}} \sim F_{1, 2}$
\end{enumerate}
\item
\begin{enumerate}
\item \begin{align*}
F_{X_{(n)}} &= \P(\max(X_1, \ldots, X_n) \leq x) = \P(X_1 \leq x) \cdot \ldots \cdot \P(X_n \leq x) = (\P(X_1 \leq x))^n \\
&= \begin{cases}
0 & \text{при } x<0 \\
\left(\frac{x}{\theta}\right)^n & \text{при }  x \in [0, \theta] \\
1 & \text{при }  x > \theta
\end{cases}
\end{align*}

\[
f_{X_{(n)}} (x)  = \begin{cases}
0 & \text{при } x<0 \\
\frac{nx^{n-1}}{ \theta^n} & \text{при }  x \in [0, \theta] \\
0 & \text{при }  x > \theta
\end{cases}
\]

\begin{align*}
\E(X_{(n)}) &= \int_{-\infty}^{+\infty} x \cdot f_{X_{(n)}} (x) dx =  \int_{0}^{\theta}	x \cdot \frac{nx^{n-1}}{\theta^n} dx = \left. \frac{n}{\theta^n} \cdot \frac{x^{n+1}}{n+1} \right|_{x=0}^{x=\theta} \\
&= \frac{n}{\theta^n}  \cdot \frac{\cdot \theta^{n+1}}{n+1} = \frac{n\theta}{n+1}
\end{align*}
Следовательно, $\E \left(\frac{n+1}{n} \cdot X_{(n)}\right) = \theta$, а значит,
$\hat{\theta} = \frac{n+1}{n} \cdot X_{(n)}$ – несмещённая оценка вида $c \cdot  X_{(n)}$.

\item $\Var\left(\hat{\theta}\right) = \frac{(n+1)^2}{n^2} \Var(X_{(n)})$

\begin{align*}
\E\left(X_{(n)}^2\right) &= \int_{-\infty}^{+\infty} x^2 f_{X_{(n)}} (x) dx = \int_{0}^{\theta} x^2 \frac{nx^{n-1}}{\theta^n}  dx = \frac{n}{\theta^n}  \int_{0}^{\theta} x^{n+1} dx \\
&= \left. \frac{n}{\theta^n} \cdot \frac{x^{n+2}}{n+2} \right|_{x=0}^{x=\theta} = \frac{n}{\theta^n} \cdot  \frac{\theta^{n+2}}{n+2} = \frac{n\cdot\theta^2}{n+2}
\end{align*}

\[
\Var(X_{(n)}) = \E\left(X_{(n)}^2\right)  - (\E(X_{(n)}))^2 = \frac{n\theta^2}{n+2} - \frac{n^2 \cdot \theta^2}{(n+1)^2} = n\theta^2 \left(\frac{1}{n+2} - \frac{n}{(n+1)^2}\right)
\]

\[
\Var\left(\tilde{\theta}\right) = \frac{(n+1)^2}{n^2} \Var(X_{(n)}) = \frac{(n+1)^2}{n^2}  \cdot n\theta^2 \left(\frac{n^2+2n+1 - n^2-2n}{(n+2)(n+1)^2} \right) = \frac{\theta^2}{n(n+2)}
\]
Оценка $\hat{\theta}_n$ является состоятельной, так как $\E(\hat{\theta}_n) = \theta$ и $\Var(\hat{\theta}_n) = \frac{\theta^2}{n(n+2)} \underset{n \to \infty}{\to} 0$

\item $\E(X_1) = \left. \frac{\theta}{2} \right|_{\theta = \hat{\theta}_{MM}} = \bar{X} \Rightarrow \hat{\theta}_{MM} = 2\bar{X}$
\item $\Var\left(2\bar{X}\right) = \frac{4}{n^2}n\Var(X_1) = \frac{4\theta}{12n} = \frac{\theta}{3n} > \Var\left(\hat{\theta}_n\right)$
\end{enumerate}
\item
\begin{enumerate}
\item $L(x, p) = \prod_{i=1}^n \P(X_i = x_i) = p^{\sum_{i=1}^n(x_i-1)} (1-p)^n$

$\ln L (x, p) = \sum_{i=1}^n(x_i-1) \ln p  + n\ln(1-p)$

$\frac{\partial \ln L}{\partial p} = \frac{\sum_{i=1}^n(x_i-1)}{p} - \frac{n}{1-p} = 0 \Rightarrow \hat{p} = \frac{\sum_{i=1}^n x_i - n}{\sum_{i=1}^n x_i}$
\item $\E(X) = \frac{n}{p} \Rightarrow \hat{\E}(X) = \frac{n\sum_{i=1}^n x_i}{\sum_{i=1}^n x_i - n}$
\end{enumerate}
\end{enumerate}

% TODO: check typesetting below this line!!!!

\subsection[2011-2012]{\hyperref[sec:kr_03_2011_2012]{2011-2012}}
\label{sec:sol_kr_03_2011_2012}


\begin{enumerate}
\item
\begin{enumerate}
\item $L (x, \lambda) = \prod_{i=1}^{n}\lambda e^{-\lambda x_i} = \lambda^n e^{-\lambda \sum_{i=1}^{n}x_i}$

$\ln L (x, \lambda) = n\ln\lambda - \lambda\sum_{i=1}^n x_i \to \max_\lambda$

$\frac{\partial \ln L}{\partial \lambda} = \frac{n}{\lambda} - \sum_{i=1}^{n}x_i \mid_{\lambda = \hat{\lambda}} = 0 \Rightarrow \hat{\lambda}_{ML} = \frac{1}{\bar{X}}$

$\frac{\partial^2 \ln L}{\partial \lambda^2} = -\frac{n}{\lambda^2} \mid_{\lambda=\hat{\lambda}} < 0$
\item $\hat{a} = \bar{X}$
\item $\E(\hat{a}) = \E(\bar{X}) = \frac{1}{\lambda} \Rightarrow$  несмещённая
\item $\Var(\hat{a}) = \Var(\bar{X}) = \frac{1}{n}n\Var(X_i) = \frac{1}{n\lambda^2} \to_{n-\to\infty} 0 \Rightarrow$ состоятельная
\item
\item $\E(X) = \frac{1}{\lambda} \mid_{\lambda = \hat{\lambda}_{MM}} = \bar{X} \Rightarrow \lambda_{MM} = \frac{1}{\bar{X}}$
\end{enumerate}
\item
\begin{enumerate}
\item Истинное стандартное отклонение неизвестно, поэтому используем распределение Стьюдента:

$\bar{X} - t_{0.025, 99}\frac{\hat{\sigma}}{\sqrt{n}} < \mu < \bar{X} + t_{0.025, 99}\frac{\hat{\sigma}}{\sqrt{n}}$

$9.5 - 1.98 \frac{0.5}{\sqrt{100}} < \mu < 9.5 - 1.98 \frac{0.5}{\sqrt{100}} $

$9.4 < \mu < 9.6$
\item  Значение 10 не лежит в доверительном интервале, значит, гипотеза отвергается

$t_{obs} = \frac{9.5-10}{0.5/\sqrt{100}} = 10 \Rightarrow \text{p-value} \approx 0$
\end{enumerate}
\item
\begin{enumerate}
\item $\frac{\hat{\sigma}_\alpha^2}{\hat{\sigma}_\beta^2} \sim F_{n_{\alpha-1}, n_{\beta-1}}$

$F_{obs} = \frac{0.5}{0.6} \approx 0.83$, $\text{p-value}/ 2\approx 0.35 \Rightarrow$ на любом разумном уровне значимости оснований отвергать $H_0$ нет
\item $\hat{\sigma}_0^2 = \frac{\hat{\sigma}_\alpha^2(n_\alpha -1) + \hat{\sigma}_\beta^2(n_\beta-1)}{n_\alpha+n_\beta-2} = \frac{0.25\cdot19 + 0.36\cdot24}{20+25-2}=0.31$

$\frac{\bar{X} - \overline{Y}}{\hat{\sigma}_0 \sqrt{\frac{1}{n_\alpha}+\frac{1}{n_\beta}}} \sim t_{n_\alpha+n_\beta-2}$

$t_{obs} = \frac{9.5-9.8}{0.56\sqrt{\frac{1}{20}+ \frac{1}{25}}} = -1.79$, $\text{p-value}/2=0.04$

\end{enumerate}
\end{enumerate}


\subsection[2010-2011]{\hyperref[sec:kr_03_2010_2011]{2010-2011}}
\label{sec:sol_kr_03_2010_2011}


\begin{enumerate}
\item
\begin{enumerate}
\item Введём обозначения для событий:
\begin{itemize}
  \item $A$ — заболеть
  \item $\bar A$ — не заболеть
  \item $B$ — привиться
  \item $\bar B$ — не привиться
\end{itemize}

Тогда вероятность заболеть можно получить по формуле полной вероятности:
\[
\P(A) = \P(A | B) \cdot \P(B)  + \P(A | \bar B) \cdot \P(\bar B) = 0.15 \cdot 0.1 + 0.2 \cdot 0.9 = 0.195
\]
Значит, во время эпидемии заболевает $19.5\%$ людей.
\item Пусть всё население составляет $n$ человек.
Тогда доля заболевших — $0.195 n$.
Доля привитых заболевших — $0.015 n$.
Значит, среди заболевших
\[
\frac{0.015 n}{0.195 n} \cdot 100 \% \approx 7.7 \%
\]
привитых.
\end{enumerate}
\item
\begin{enumerate}
\item
\begin{align*}
\P(X > 4) &= \P\left(\frac{X - 3}{\sqrt{25}} > \frac{4 - 3}{\sqrt{25}} \right) = \P\left(\cN(0,1) > 0.2\right) \approx 0.42 \\
\P(4 < X \leq 5) &= \P \left(\frac{4 - 3}{\sqrt{25}} < \frac{X - 3}{\sqrt{25}} \leq \frac{5 - 3}{\sqrt{25}} \right)  \\
&= \P(0.2 < \cN(0,1) \leq 0.4) \approx 0.076
\end{align*}
\item Найдём математическое ожидание и дисперсию случайной величины $S = X - 2Y$:
\begin{align*}
\E(S) &= \E(X) - 2\E(Y) = 3 - 2 \cdot 1 = 1 \\
\Var(S) &= \Var(S)  + 4 \Var(Y) - 4 \Cov(X, Y) = 57
\end{align*}
Теперь можно посчитать вероятность:
\[
\P(X - 2Y < 4) = \P \left(\frac{S - 1}{\sqrt{57}} < \frac{4 - 1}{\sqrt{57}} \right) = \P(\cN(0, 1) < 0.4) \approx 0.66
\]
\item Замечаем, что дисперсия суммы равна сумме дисперсий. 
В силу совокупной нормальности величины $S$ и $Z$ независимы, а потому искомая условная вероятность равна безусловной,
$\P(X - 2Y < 4 | Z > 8 ) = \P(X - 2Y < 4) \approx 0.66$.
\end{enumerate}
\item
\begin{enumerate}
\item Сначала найдём среднее и несмещённую оценку дисперсии по выборке $Y$:
\[
\bar Y = \frac{480}{12} 40, \quad \hat{\sigma}^2_Y = \frac{358.3 \cdot 12}{11} \approx 391
\]
Доверительный интервал для математического ожидания при неизвестной дисперсии имеет вид:
\begin{align*}
\bar Y - t_{\frac{\alpha}{2}, n-1} \frac{\hat{\sigma}_Y}{\sqrt{n}} < \mu_Y < \bar Y + t_{\frac{\alpha}{2}, n-1} \frac{\hat{\sigma}_Y}{\sqrt{n}}
\end{align*}
Критическое значение $t_{0.95, 11} \approx 1.8$:
\[
40 - 1.8 \cdot \sqrt{\frac{391}{12}} < \mu_Y < 40 + 1.8 \cdot \sqrt{\frac{391}{12}}
\]
\item Найдём также среднее и несмещённую оценку дисперсии по выборке $X$:
\[
\bar X = \frac{540}{15} = 36, \quad \hat{\sigma}^2_X = \frac{410.264 \cdot 15}{14} \approx 440
\]
И посчитаем оценку общей дисперсии:
\[
\hat{\sigma}^2_0 = \frac{\sum_{i=1}^{n_X} \left(X_i - \bar X\right)^2 + \sum_{i=1}^{n_Y} \left(Y_i - \bar Y\right)^2}{n_X + n_Y - 2} =  \frac{410.264 \cdot 15 + 358.3 \cdot 12}{15 + 12 - 2} \approx 418
\]
Тогда, проверяя гипотезу $H_0: \mu_X = \mu_Y$ против альтернативной $H_a: \mu_X < \mu_Y$,
получим следующие наблюдаемое и критическое значения статистики:
\begin{align*}
t_{obs} &= \frac{\mu_X - \mu_Y}{\sqrt{\sigma^2_0 \left(\frac{1}{n_X} + \frac{1}{n_Y}\right)}} = \frac{36 - 40}{\sqrt{418 \cdot \left(\frac{1}{15} + \frac{1}{12}\right)}} \approx -0.51 \\
t_{crit, 0.05, 25} &\approx -1.7
\end{align*}
Так как $t_{obs} > t_{crit}$, нет оснований отвергать $H_0$.
\item Выпишем вариационный ряд по двум выборкам, выделяя наблюдения, относящиеся к
Юго-Западному округу, и присвоим им ранги:

\begin{center}
\begin{tabular}{@{}lcccccccccc@{}}
\toprule
Наблюдение & $8.4$           & $\textbf{14.4}$ & $15.2$          & $15.6$          & $\textbf{18.0}$ & $19.1$          & $21.2$          & $\textbf{22.0}$ & $\textbf{23.9}$ & $\textbf{26.6}$ \\
Ранг       & $1$             & $2$             & $3$             & $4$             & $5$             & $6$             & $7$             & $8$             & $9$             & $10$            \\ \midrule
Наблюдение & $28.2$          & $28.3$          & $\textbf{32.0}$ & $33.8$          & $34.5$          & $38.2$          & $\textbf{43.3}$ & $44.1$          & $45.0$          & $\textbf{46.7}$ \\
Ранг       & $11$            & $12$            & $13$            & $14$            & $15$            & $16$            & $17$            & $18$            & $19$            & $20$            \\ \midrule
Наблюдение & $\textbf{54.8}$ & $56.0$          & $\textbf{64.0}$ & $\textbf{65.1}$ & $68.2$          & $\textbf{69.2}$ & $84.2$ \\
Ранг       & $21$            & $22$            & $23$            & $24$            & $25$            & $26$            & $27$   \\ \bottomrule
\end{tabular}
\end{center}

Теперь посчитаем сумму рангов по выборке меньшего размера, то есть по Юго-Западному округу:
\[
T = \sum_{i=1}^{n_Y} R_i = 2 + 5 + 8 + 9 + 10 + 13 + 17 + 20 + 21 + 23 + 24 + 26 = 178
\]
Осталось найти значение наблюдаемой статистики и критическое значение:
\begin{align*}
\gamma_{obs} &= \frac{T - \frac{1}{2} n_X (n_X + n_Y + 1)}{\sqrt{\frac{1}{12} n_X n_Y (n_X + n_Y)} } = \frac{178 - \frac{1}{2} \cdot 12 (12 + 15 + 1)}{\sqrt{\frac{1}{12} \cdot 12 \cdot 15 \cdot(12 + 15)}} \approx 0.5 \\
\gamma_{crit} &= 1.96
\end{align*}
Так как $\gamma_{obs} < |\gamma_{crit}|$, нет оснований отвергать $H_0$.
\end{enumerate}
\item Будем проверять гипотезу о том, что вероятность падения маслом вверх равна $0.5$:
\[
\begin{cases}
H_0: p = 0.5 \\
H_a: p \neq 0.5
\end{cases}
\]
Найдём наблюдаемое и критическое значения статистики:
\begin{align*}
z_{obs} &= \frac{\hat p - p_0}{\sqrt{\frac{\hat p (1 - \hat p)}{n}}} = \frac{\frac{105}{200} - \frac{1}{2}}{\sqrt{\frac{\frac{105}{200} \cdot \frac{95}{200}}{200}}} \approx 0.71 \\
z_{crit} &= 1.96
\end{align*}
Так как $z_{obs} < |z_{crit}|$, нет оснований отвергать $H_0$.

\item
\begin{enumerate}
\item
\begin{align*}
\bar X &= \E(X_1) |_{\mu_1 = \hat{\mu}_1} = \hat{\mu}_1 \\
S^2 &= \Var(X_1) |_{\mu_1 = \hat{\mu}_1, \mu_2= \hat{\mu}_2} = \hat{\mu}_2 - \hat{\mu}_1^2 \Rightarrow \hat{\mu}_2 = \bar{X^2}
\end{align*}
\item Найдём максимум логарифмической функции правдоподобия по параметру $\theta$:
\begin{align*}
L &= \prod_{i=1}^n \frac{1}{\sqrt{2\pi}} \exp \left(-\frac{1}{2} (x_i - \theta)^2\right) \\
\ell &= \frac{n}{2} \ln 2\pi - \frac{1}{2} \sum_{i=1}^n (x_i - \theta)^2 \\
\frac{\partial \ell}{\partial \theta} &= \left. \sum_{i=1}^n (x_i - \theta) \right|_{\theta = \hat \theta} = 0 \\
\hat \theta &= \bar X
\end{align*}
Для проверки эффективности достаточно показать, что неравенство Рао-Крамера
для полученной оценки выполняется как равенство.

Найдём информацию Фишера:
\begin{align*}
\frac{\partial^2 \ell}{\partial \theta^2} &= -n \\
I(\theta) &= n
\end{align*}
и дисперсию оценки $\hat \theta$:
\[
\Var\left(\hat \theta \right) = \frac{1}{n}
\]
Подставив полученные значения в неравенство Рао-Крамера
\[
\Var\left(\hat \theta \right) \geq \frac{1}{I(\theta)},
\]
получим равенство, значит, найденная оценка является эффективной.
\end{enumerate}
\end{enumerate}

\subsection[2009-2010]{\hyperref[sec:kr_03_2009_2010]{2009-2010}}
\label{sec:sol_kr_03_2009_2010}


\begin{enumerate}
  \item 
  \item 
  \item 
  \item 
  \item 
  \item Пусть $S_i = X_i Y_i$. Замечаем, что $\E(S_i)=\E(X_i)\E(Y_i)=ab$, $\E(S_i)=\E(X_i^2)\E(Y_i^2)=(a^2+1)(b^2+1)$. Отсюда получаем систему
  \[
  \begin{cases}
  \hat a \hat b = \frac{\sum S_i}{n} \\
  (\hat a^2 + 1) (\hat b^2 + 1) = \frac{\sum S_i^2}{n} \\
  \end{cases}
  \]
  Для заданных чисел решением будет $\hat a = 2$ и $\hat b = 18$, или наоборот.
\end{enumerate}


\subsection[2008-2009]{\hyperref[sec:kr_03_2008_2009]{2008-2009}}
\label{sec:sol_kr_03_2008_2009}

\subsection[2007-2008. Демо]{\hyperref[sec:kr_03_2007_2008_demo]{2007-2008. Демо}}
\label{sec:sol_kr_03_2007_2008_demo}


Заметим, что $\hat{a}_{n}\geq a$.

$\P(|\hat{a}_{n}-a|>\varepsilon)=\P(\hat{a}_{n} a>\varepsilon)=
\P(\hat{a}_{n}>a+\varepsilon)=\P(\min\{X_{1},X_{2},\ldots,X_{n}\}>a+\varepsilon)= \\
=\P(X_{1}>a+\varepsilon \cap X_{2}> a+\varepsilon\cap \ldots)=
\P(X_{1}>a+\varepsilon)\cdot \P(X_{2}>a+\varepsilon)\cdot \ldots=
\left(\int_{a+\varepsilon}^{\infty}e^{a-t}dt\right)^{n}=\left(e^{-\varepsilon}\right)^{n}=e^{-n\varepsilon}$

$\lim_{n\to\infty} e^{-n\varepsilon} =0$

Оценка смещена при любых $n$, хотя смещение с ростом $n$ убывает

\subsection[2007-2008]{\hyperref[sec:kr_03_2007_2008]{2007-2008}}
\label{sec:sol_kr_03_2007_2008}

Да, Нет, Да, Нет, Нет, Да, Нет, Да, Да, Нет


\begin{enumerate}
\item
\begin{enumerate}
\item $[13.61;14.39]$
\item Отвергается ($t_{obs} = -2.12$, $t_{crit} = -1.29$)
\item $P_{value} \approx 0.017$
\end{enumerate}
\item Заменяем числа на цифры 0 и 1 (0 — меньше 19 цветков), (1 — больше).

$\hat{p}=\frac{8}{25}=0.32$

$H_{0}$: $p=0.5$

$H_{a}$: $p\neq 0.5$

$Z=\frac{0.32-0.5}{\sqrt{\frac{0.5\cdot 0.5}{25}}}=-1.8$

При уровне значимости 5\%, $Z_{critical}=1.96$. Значит, гипотеза $H_{0}$ не отвергается.
\item
\begin{enumerate}
\item $\hat{a}=\frac{5-\bar{X}}{10}$
\item Да, является
\end{enumerate}
\item $L=-\frac{n}{2}\ln(a)-\frac{na}{8}-\frac{\sum X_{i}^{2}}{8a}+c$

$L'=0$ равносильно $\hat{a}^{2}+4\hat{a}+4=4+\frac{\sum X_{i}^{2}}{n}$

$\hat{a}=-2+\sqrt{4+\frac{\sum X_{i}^{2}}{n}}$
\item $F_{29.39}=\frac{32}{20}=1.6$

$F_{crit}=1.74$

Гипотеза о том, что дисперсия одинакова не отвергается.
\item $\chi^{2}_{observed}=1.15$

$\chi^{2}_{2,5\%}=5.99$

Правдоподобно.
\item
\begin{enumerate}
\item $p=0.7\cdot 0.8+ 0.3\cdot 0.7=0.77$
\item $p=\frac{0.7\cdot0.8\cdot0.2}{0.7\cdot 0.8\cdot 0.2 + 0.7\cdot 0.2 \cdot 0.3}=\frac{8}{11}$
\end{enumerate}
\item
\begin{enumerate}
\item Заметим, что $\hat{a}_{n}\leq a$.
\begin{multline*}
\P(|\hat{a}_{n}-a|>\varepsilon)=\P(-(\hat{a}_{n}-a)>\varepsilon)=\P(\hat{a}_{n}<a-\varepsilon)=
\P(\max\{X_{1},X_{2}, \ldots, X_{n}\}<a-\varepsilon)= \\
=\P(X_{1}<a-\varepsilon \cap X_{2}< a-\varepsilon\cap \ldots)=
\P(X_{1}<a-\varepsilon)\cdot \P(X_{2}<a-\varepsilon)\cdot \ldots=(1-\varepsilon)^{n}
\end{multline*}
$\lim_{n\to\infty} (1-\varepsilon)^{n} =0$
\item Нет, не является ни при каких $n$, хотя смещение с ростом $n$ убывает.
\end{enumerate}
\item[9-А.] Да, \url{http://en.wikipedia.org/wiki/Simpson's_paradox}
\item[9-Б.]
\begin{enumerate}
\item Пусть истинные веса слитков равны $x$, $y$ и $z$.

Назовем оценку буквой $\hat{x}$, $\hat{x}=aX+bY+cZ$.

Несмещённость: $\E(\hat{x})=a\E(X)+b\E(Y)+c\E(Z)=ax+by+c(x+y)=x$

$a+c=1$, $b+c=0$

$\hat{x}=(1-c)X+(-c)Y+cZ$

Эффективность: $\Var(\hat{x})=((1-c)^{2}+c^{2}+c^{2})\cdot \sigma^{2}=(3c^{2}-2c+1)\sigma^{2}$

Чтобы минимизировать дисперсию, нужно выбрать $c=1/3$.

То есть $\hat{x}=\frac{2}{3}X-\frac{1}{3}Y+\frac{1}{3}Z$.
\item $\Var(\hat{x})=\frac{2}{3}\sigma^{2}$

$\Var\left(\frac{X_{1}+X_{2}}{2}\right)=\frac{1}{2}\sigma^{2}$

Усреднение двух взвешиваний первого слитка лучше.
\end{enumerate}
\end{enumerate}



\subsection[2006-2007]{\hyperref[sec:kr_03_2006_2007]{2006-2007}}
\label{sec:sol_kr_03_2006_2007}

Верный ответ на первый вопрос теста — нет! Некоррелированные одномерные нормальные распределения
могут быть не нормальными в совокупности.

\begin{enumerate}
\item
\begin{enumerate}
\item $\P(X=-1) = \frac{5}{12}\cdot\frac{4}{11} = \frac{5}{33}$

$\E(X) = -1 \cdot \frac{5}{33} + 0 \cdot 2 \cdot \frac{5}{12}\cdot\frac{7}{11} + 1 \cdot \frac{7}{12}\cdot\frac{6}{11} = \frac{1}{6}$

$\E(X^2) = (-1)^2 \cdot \frac{5}{33} + 1^2 \cdot \frac{7}{12}\cdot\frac{6}{11} = \frac{31}{66}$

$\Var(X) = \frac{175}{396}$
\item $F(x) = \begin{cases}
0, & x < -1 \\
\frac{5}{33}, & -1 \leq x < 0 \\
\frac{15}{22}, & 0 \leq x < 1 \\
1, & x \geq 1
\end{cases}$
\end{enumerate}
\item
\begin{enumerate}
\item $c=1$, $\P(0.5<X<2) = 0.75$, $0.5$
\item $\E(X) = \frac{2}{3}$, $\Var(X) = \frac{1}{18}$, $\Cov(X,-X) = - \frac{1}{18}$,
$\Corr(2X,3X) = 1$
\item $f(x) = \begin{cases}
0, & t < 0, t \geq 1 \\
2t, & 0 \leq t < 1
\end{cases}$
\end{enumerate}
\item
\begin{enumerate}
\item $\Corr(X,Y)=-\frac{1}{3}$
\item $\alpha=\frac{11}{17}$
\item Нет
\item Да
\end{enumerate}
\item
\begin{enumerate}
\item $\hat{p} = \frac{1}{20}$

$\left[\frac{1}{20} - 1.65 \cdot \sqrt{\frac{\frac{1}{20}\cdot\frac{19}{20}}{40}}; \frac{1}{20} + 1.65 \cdot \sqrt{\frac{\frac{1}{20}\cdot\frac{19}{20}}{40}}  \right]$
\item $\P(\vert \hat{p} - p \vert \leq 0.1) = 0.9 \Rightarrow \P\left(\frac{\vert \hat{p} - p \vert}{\sqrt{\frac{\hat{p}(1-\hat{p})}{n}}} \leq \frac{0.1}{\sqrt{\frac{\hat{p}(1-\hat{p})}{n}}} \right) = 0.9 \Rightarrow \frac{0.1}{\sqrt{\frac{\frac{1\cdot19}{20^2}}{n}}} = 1.65 \Rightarrow n \approx 13$
\end{enumerate}
\item
\begin{enumerate}
\item $\gamma_{obs} = \frac{\hat{\sigma}^2_Y}{\hat{\sigma}^2_X} \approx 1.32$, $\gamma_{crit, 0.95} \approx 1.64$,
оснований отвергать $H_0$ нет
\item $X_1, \ldots, X_n \sim \cN(\mu_X, \sigma^2_X)$, $Y_1, \ldots, Y_n \sim \cN(\mu_Y, \sigma^2_Y)$ — независимые выборки
\item $\left[17.51 - 1.66 \cdot 59.4 \sqrt{\frac{1}{40}+ \frac{1}{60}}; 17.51 + 1.66 \cdot 59.4 \sqrt{\frac{1}{40}+ \frac{1}{60}} \right]$

$[-2.61; 37.64]$

$\hat{\sigma}_0^2 = \frac{\hat{\sigma}^2_X(n_X-1) + \hat{\sigma}^2_Y(n_Y-1)}{n_X+n_Y-2} \approx 59.4$
\item Оснований считать новую методику более эффективной нет, так как $0$ входит в доверительный интервал.
\end{enumerate}
\item $\hat{p}_1 = \frac{57}{159}$, $\hat{p}_2 = \frac{48}{159}$, $\hat{p}_3 = \frac{54}{159}$

$Q_{obs} = \sum_{i=1}^n \frac{(n_i - n \cdot p_i)^2}{n \cdot p_i} = \frac{42}{53}$. $Q_{crit} = 3.84$.
Оснований отвергать нулевую гипотезу нет.
\item $\gamma_{obs} = \sum_{i=1}^s \sum_{j=1}^m \frac{\left(n_{ij} - \frac{n_{i\cdot}n_{\cdot j}}{n}\right)^2}{\frac{n_{i\cdot}n_{\cdot j}}{n}} \approx 1.15$,
$\gamma_{crit} = 3.84$, оснований отвергать $H_0$ нет.
\item
\begin{enumerate}
\item \begin{align*}
L(x_1, \ldots, x_n, \theta) &= \prod_{i=1}^n \frac{1}{\sqrt{2\pi\theta}}e^{-\frac{1}{2}\cdot\frac{x_i^2}{\theta}} = \frac{1}{(\sqrt{2\pi\theta})^n} e^{-\frac{1}{2\theta} \sum_{i=1}^n x_i^2} \\
l(x_1, \ldots, x_n, \theta) &= -\frac{n}{2} \ln (2\pi) - \frac{n}{2} \ln \theta -\frac{1}{2\theta} \sum_{i=1}^n x_i^2 \to \max_{\theta} \\
\frac{\partial l}{\partial \theta} &= - \frac{n}{2 \theta} + \frac{1}{2\theta^2} \sum_{i=1}^n x_i^2 \\
\hat{\theta}_{ML} &= \frac{\sum_{i=1}^n x_i^2}{n}
\end{align*}
\item $\E(\hat{\theta}_{ML}) = \E\left(\frac{\sum_{i=1}^n x_i^2}{n}\right) = \frac{1}{n}\cdot n \E(x_1^2) = \theta$,
оценка несмещённая.

$\Var(\hat{\theta}_{ML}) = \Var\left(\frac{\sum_{i=1}^n x_i^2}{n}\right) = \frac{1}{n^2}\cdot n \Var(x_1^2) = \frac{3\theta^2 - \theta^2}{n}\to_{n\to\infty} 0$,
оценка состоятельная.

$\frac{\partial^2 l}{\partial \theta^2} = \frac{n}{2 \theta^2} - \frac{1}{\theta^3}\sum_{i=1}^n x_i^2$

$-\E\left(\frac{\partial^2 l}{\partial \theta^2}\right) = -\frac{n}{2 \theta^2} + \frac{1}{\theta^3} \cdot n \theta = \frac{n}{2 \theta^2}$

$\Var(\hat{\theta}_{ML}) = \frac{2\theta^2}{n} = \frac{1}{\frac{n}{2 \theta^2}} = I(\theta)$,
оценка эффективная.
\end{enumerate}
\item
\begin{enumerate}
\item O1Р: выбрали $H_a$, но верна $H_0$, то есть $X \sim [0, 100]$, но при этом $X \geq c$.

О2Р: выбрали $H_0$, но верна $H_a$, то есть $X \sim [50, 150]$, но при этом $X > c$.
\item $\alpha = \begin{cases}
1, & c < 0 \\
\frac{100-c}{100}, & 0 \leq c \leq 100 \\
0, & c > 100
\end{cases}$

$\beta = \begin{cases}
0, & c < 50 \\
\frac{c-50}{100}, & 50 \leq c \leq 150 \\
1, & c > 150
\end{cases}$
\end{enumerate}
\item Считаем через сумму рангов, $T = 2 + 4 + 6 + 8 + 10 = 30$.

$\E(T) = \frac{1}{2}(n_x(n_x+n_y+1)) = 37.5$.

$\Var(T) = \frac{1}{12}(n_x n_y(n_x + n_y)) = \frac{75}{6}$

$\gamma_{obs} = \frac{30-37.5}{\sqrt{\frac{75}{6}}} \approx -2.12$,
$\gamma_{crit, 0.05} = -1.65$, основная гипотеза отвергается.

Можно было бы считать через $U$-статистику, у неё было бы другое ожидание, $n_x n_y/2$,
но после стандартизации снова вышло бы $-2.12$.

\end{enumerate}


\subsection[2005-2006]{\hyperref[sec:kr_03_2005_2006]{2005-2006}}
\label{sec:sol_kr_03_2005_2006}

Нет, Нет, Да, Да, Да, Нет, Нет, Нет, Да, Нет, Нет, Да, Нет, Да, Нет, Да



\begin{enumerate}

\item
\begin{enumerate}
\item
\item $\E(\hat{\theta})=1\cdot \P(X<3)+0\cdot \P(X \ge 3)=\theta$, да является
\item $\E\left(\left(\hat{\theta }-\theta \right)^{2} \right)=\E\left(\hat{\theta}^{2}-2\theta\hat{\theta}+\theta^{2}\right) \stackrel{\hat{\theta}^{2}=\hat{\theta}}{=} \theta-2\theta^{2}+\theta^{2}=\theta-\theta^{2}$
\end{enumerate}
\item
\begin{enumerate}
\item  $L=(k+1)^{n}(x_{1}\cdot x_{2} \cdot \ldots \cdot x_{n})^{k}$

$l=\ln{L}=n\ln(k+1)+k(\sum \ln{x_{i}})$

$\frac{\partial l}{\partial k}=\frac{n}{k+1}+\sum \ln{x_{i}}$

$\frac{n}{\hat{k}+1}+\sum \ln{x_{i}}=0$

$\hat{k}=-\left(1+\frac{n}{\sum \ln{x_{i}}} \right)$
\item  $\E(X_{i})=\int t\cdot p(t)dt=\int_{0}^{1} (k+1)t^{k+1}=\frac{k+1}{k+2}$

$\frac{\hat{k}+1}{\hat{k}+2}=\bar{X}$

$\hat{k}=\frac{2\bar{X}-1}{1-\bar{X}}$
\end{enumerate}
\item $C=\sum \frac{(X_{i,j}-n \hat{p}_{i,j})^{2}}{n\hat{p}_{i,j}}\sim \chi_{(r-1)(c-1)}^{2}$

$C\sim \chi_{1}^{2}$

$C=35$

Если $\alpha=0.1$, то $C_{crit}=2.706$.

Вывод: $H_{0}$ (гипотеза о независимости признаков) отвергается.
\item
\begin{enumerate}
\item Число наблюдений велико, используем нормальное распределение.

$\P\left(-1,65<\frac{\bar{X}-\bar{Y}-\triangle}{\sqrt{\frac{\hat{\sigma}_{x}^{2}}{40}+\frac{\hat{\sigma}_{y}^{2}}{50}}}<1,65\right)=0.9$

$\triangle \in 4 \pm 1.65\sqrt{\frac{49}{40}+\frac{64}{50}}$

$\triangle \in [1.4;6.6]$
\item Используем результат предыдущего пункта: $H_{0}$ отвергается, так как число 0 не входит в доверительный интервал.
\item $Z=2.505$ и $P_{value}=0.0114$
\end{enumerate}
\item
\begin{enumerate}
\item $\chi_{9}^{2}=\frac{9\hat{\sigma}^{2}}{\sigma^{2}} \in [4.17; 14.69]$

$\sigma^{2} \in [8822.3; 31080]$

$\sigma \in [93.9; 176.3] $
\item  $\P(|\hat{\sigma}^{2}-\sigma^{2}|<0.4\sigma^{2})=\P(0.6<\frac{\hat{\sigma}^{2}}{\sigma^{2}}<1.4)=\P(11.4<\chi_{19}^{2}<26.6)\approx 0.8$
\end{enumerate}
\item
\begin{enumerate}
\item $W_{1}=2+4+6+8=20$ или $W_{2}=1+3+5+7+9+10=35$

$U_{1}=10$ или $U_{2}=14$

$Z_{1}=-0.43=-Z_{2}$

Вывод: $H_{0}$ (гипотеза об отсутствии сдвига между законами распределения) не отвергается
\item Нет, так как наблюдения не являются парными.
\end{enumerate}
\item  $\P(-2.13<t_{4}<2.13)=0.9$

$\mu \in 1560 \pm 2.13\cdot \sqrt{\frac{670^{2}}{5}}$

$\mu \in [921.8;2198.2]$
\item
\begin{enumerate}
\item
\item  $\P(\text{1 type error})=\P(X>c|X\sim U[0;100])= \left\{
\begin{array}{ll}
  1, & c<0 \\
  1-\frac{c}{100}, & c \in [0;100] \\
  0, & c>100 \\
\end{array}
\right.$

$\P(\text{2 type error})=\P(X<c|X\sim U[50;150])= \left\{
\begin{array}{ll}
  0, & c<50 \\
  \frac{c-50}{100}, & c \in [50;150] \\
  1, & c>150 \\
\end{array}
\right.$

Построение оставлено читателю в качестве самостоятельного
упражнения :)
\end{enumerate}
\item $\P(\sqrt{X^{2}+Y^{2}}>2.45)=\P(X^{2}+Y^{2}>2.45^{2})=\P(\chi_{2}^{2}>6)=0.05$
\item
\begin{enumerate}
\item  $A$ = конспект забыт в 8-ой аудитории

$B$ = конспект был забыт в другом месте (не в аудиториях)

$C$ = конспект не был найден в первых 7-и

$\P(A|C)=\frac{\P(A)}{\P(C)}=\frac{0.3\cdot 0.1}{0.3\cdot 0.3+0.7}=\frac{3}{79}$
\item $\P(B|C)=\frac{\P(B)}{\P(C)}=\frac{0.7}{0.79}=\frac{70}{79}$
\end{enumerate}

\item[11-А.] О чем молчал учебник биологии 9 класса\ldots

Если:
\begin{enumerate}
  \item  ген имеет всего две аллели;
  \item  в популяции бесконечное число организмов;
  \item  одна аллель потомка выбирается наугад из аллелей матери, другая
— из аллелей отца;
\end{enumerate}
то распределение генотипов стабилизируется уже в первом поколении (!!!).

То есть $AA_{1}=AA_{2}=\ldots$ и $Aa_{1}=Aa_{2}=\ldots$.

Вероятность получить 'A' от родителя для рождающихся в поколении 1
равна: $p_{1}=0.3\cdot 1+0.6\cdot 0.5 + 0,1\cdot 0=0.6$.

В общем виде: $p_{1}=AA_{0}+0.5\cdot Aa_{0}$

$AA_{1}=p_{1}^{2}=0.36$, $Aa_{1}=2p_{1}(1-p_{1})=0.48$.

$p_{2}=AA_{1}+0.5\cdot Aa_{1}=p_{1}^{2}+p_{1}(1-p_{1})=p_{1}$
\item[11-Б.]
\begin{enumerate}
\item Безразлично.
Если я решил попробовать угадать $n$ букв, то выигрыш вырастает, а
вероятность падает в 2 раза по сравнению c попыткой угадать $(n-1)$-у букву.
\item В силу предыдущего пункта: $\E(X)=\frac{1}{2}\cdot 50=25$.
\end{enumerate}
\end{enumerate}

\thispagestyle{empty}
\section{Решения контрольной номер 3. ИП}




\subsection[2018-2019]{\hyperref[sec:kr_03_ip_2018_2019]{2018-2019}}
\label{sec:sol_kr_03_ip_2018_2019}

\begin{enumerate}
\item
\begin{enumerate}
\item[a)] Пары ортогональных величин: $(X_1,X_2);(X_2,X_3);(X_1,X_3);(S_2,X_3)$.
\item[б)] В качестве примера можно взять величину $X_2-X_1$, поскольку 
\begin{align*}
\Cov(X_2-X_1,S_3)&=\Cov(X_2-X_1,X_1+X_2+X_3)\\
&=\Var(X_1)-\Var(X_1)=0
\end{align*}
\item[в)] Как известно из линейной алгебры, если вектор $X^-_1$ – проекция $X_1$ на $S_3$, a $X^{\perp}_1$ – соответствующая ортогональная составляющая, то $X^{\perp}_1=X_1-X^-_1=X_1-\alpha S_3$, причем 
\[
\langle X^{\perp}_1,S_3\rangle=\Cov(X^{\perp}_1,S_3)=\Cov(X^{\perp}_1,X_1-\alpha S_3)=0,
\] 
откуда 
\[
\alpha=\frac{\Cov(X_1,S_3)}{\Var(S_3)}.
\]
Тогда искомая проекция равна 
\[
\frac{\Cov(X_1,S_3)}{S_3}\cdot S_3
\]
\item[г)] Искомая проекция равна ортогональной составляющей из предыдущего пункта:
\[
X^{\perp}_1=X_1-\frac{\Cov(X_1,S_3)}{S_3}\cdot S_3
\]
\item[д)] Пусть соответствующие проекции равны $X^-_1=\alpha S_3$, $X^-_2=\beta S_3$. Тогда 
\begin{align*}
\Cov(X^-_1,X^-_2)&=\Cov(X_1-\alpha S_3,X_2 - \beta S_3)\\
&=-\alpha\Var(X_1)-\beta\Var(X_2)+3\alpha\beta\Var(X_1)
\end{align*}
\begin{align*}
\Var(X^-_1)&=\Var(X_1-\alpha S_3)\\
&=\Var(X_1)+3\alpha^2\Var(X_1)-2\alpha\Var(X_1)
\end{align*}

Заметим, что дисперсии величин $X_1$ и $X_2$ равны, поэтому
\begin{align*}
\Var(X^-_2)&=\Var(X_2-\beta S_3)\\
&=\Var(X_1)+3\beta^2\Var(X_1)-2\beta\Var(X_1)
\end{align*}

Таким образом, искомая частная корреляция равна
\[
\frac{(3\alpha\beta-\alpha-\beta)\Var(X_1)}{\Var(X_1)\sqrt{(1-2\alpha+3\alpha^2)(1-2\beta+3\beta^2)}}
\]

Вспоминаем, что 
\[
\alpha=\frac{\Cov(X_1,S_3)}{\Var(S_3)}=\frac{\Var(X_1)}{3\Var(X_1)}=\frac{1}{3}=\beta,
\]

откуда искомая частная корреляция равна $-1/2$.
\end{enumerate}
\item 
\begin{enumerate}
\item[a)] Как видно по графику, $x-1\geq \ln x$.
\begin{minipage}{0.6\textwidth}
\begin{center}
\includegraphics[scale=0.5]{images/sol_kr_3_ip.png}
\end{center}
\end{minipage}
\item[б)] По свойству из предыдущего пункта, 
\begin{align*}
\frac{q(x)}{p(x)}-1&\geq\ln \left(\frac{q(x)}{p(x)}\right)\\
g(x)-p(x)&\geq\ p(x)\ln q(x)-p(x)\ln p(x)\\
-p(x)\ln q(x)&\geq -p(x)\ln p(x)+p(x)-q(x)
\end{align*}
Интегрируем:
\begin{align*}
-\int^{+\infty}_{-\infty}p(x)\ln q(x) \; dx \geq -\int^{+\infty}_{-\infty}p(x)\ln p(x) \; dx \\
&+\int^{+\infty}_{-\infty}p(x) \; dx - \int^{+\infty}_{-\infty}q(x) \; dx
\end{align*}
Поскольку $p(x)$, $q(x)$ – функции плотности, интегрирование каждой из них дает единицу, а значит, последние два слагаемых в сумме равны нулю:
\[
-\int^{+\infty}_{-\infty}p(x)\ln q(x) \; dx \geq -\int^{+\infty}_{-\infty}p(x)\ln p(x) \; dx
\]
\item[в)] Функция плотности для величины $X \sim \cN(\mu;\sigma^2)$ есть
\begin{align*}
q(x)&=\frac{1}{\sqrt{2\pi\sigma^2}}\exp{\left(\frac{-(x-\mu)^2}{2\sigma^2}\right)}\\
\ln q(x)&=-\frac{1}{2}\ln 2\pi-\ln \sigma-\frac{(x-\mu)^2}{2\sigma^2}\\
H(q)&=-\int^{+\infty}_{-\infty}q(x)\ln q(x) \; dx
\end{align*}
\begin{align*}
H(q)&=\int^{+\infty}_{-\infty}q(x)\left(\frac{1}{2}\ln 2\pi+\ln \sigma+\frac{(x-\mu)^2}{2\sigma^2}\right) \; dx\\
&=\frac{1}{2}\ln 2\pi + \ln \sigma+\int^{+\infty}_{-\infty}q(x)\frac{(x-\mu)^2}{2\sigma^2} \; dx\\
&=\frac{1}{2}\ln 2\pi + \ln \sigma+\int^{+\infty}_{-\infty}\frac{\left(q(x)x^2-2x\mu q(x)+\mu^2q(x)\right)}{2\sigma^2} \; dx\\
&=\frac{1}{2}\ln 2\pi + \ln \sigma+\frac{1}{2\sigma^2}\left(\E(X^2)-2\mu\E(X)+\mu^2\right)\\
&=\frac{1}{2}\ln 2\pi + \ln \sigma+\frac{1}{2\sigma^2}\left(\sigma^2+\mu^2-2\mu^2+\mu^2\right)\\
&=\frac{1}{2}\ln 2\pi + \ln \sigma+\frac{1}{2}
\end{align*}
\item[г)] Пусть $Y \sim \cN(\mu_N;\sigma^2_N)$, ее функция плотности равна $q(x)$. Тогда, аналогично пункту в), получаем:
\begin{align*}
CE_p(q)&=\frac{1}{2}\ln 2\pi + \ln \sigma_N\\
&+\int^{+\infty}_{-\infty}\frac{\left(p(x)x^2-2x\mu_N p(x)+\mu^2_Np(x)\right)}{2\sigma^2_N} \; dx\\
&=\frac{1}{2}\ln 2\pi + \ln \sigma_N+\frac{1}{2\sigma^2_N}\left(\sigma^2+\mu^2-2\mu\mu_N+\mu^2_N\right)
\end{align*}
\end{enumerate}
\item
\begin{enumerate}
\item[a)] 
\begin{align*}
L&=\lambda e^{-\lambda x_1}\lambda e^{-\lambda x_2}\\
\ln L&=2\ln \lambda-\lambda(x_1+x_2)\\
(\ln L)'_\lambda&=\frac{2}{\hat{\lambda}}-(x_1+x_2)=0\\
\hat{\lambda}&=\frac{2}{x_1+x_2}=\frac{1}{15}
\end{align*}
\item[б)] Пусть $X$ – время обслуживания одного клиента, тогда
\begin{align*}
\P(X>30)&=\int^{+\infty}_{30}\lambda e^{-\lambda x} \; dx \Rightarrow\\
L&=\lambda e^{-\lambda x_1}\lambda e^{-\lambda x_2}\prod^{10}_{i=3}\int^{+\infty}_{30}\lambda e^{-\lambda x_i} \; dx_i\\
\int^{+\infty}_{30}\lambda e^{-\lambda x} \; dx&=-e^{-\lambda x}|^{+\infty}_{30}=e^{-30\lambda}\\
L&=\lambda e^{-\lambda x_1}\lambda e^{-\lambda x_2}\prod^{10}_{i=3}e^{-30\lambda}\\
\ln L&=2\ln \lambda-\lambda(x_1+x_2)-240\lambda\\
(\ln L)'_\lambda&=\frac{2}{\hat{\lambda}}-(x_1+x_2)-240=0\\
\hat{\lambda}&=\frac{1}{135}
\end{align*}
\item[в)] 
\begin{align*}
\P(X<30)&=1-\int^{+\infty}_{30}\lambda e^{-\lambda x} \; dx=1-e^{-30\lambda} \Rightarrow\\
L&=\prod^{2}_{i=1}(1-e^{-30\lambda})\prod^{10}_{i=3}e^{-30\lambda}\\
\ln L&=2\ln (1-e^{-30\lambda})-240\lambda\\
(\ln L)'_\lambda&=\frac{2\cdot30e^{-30\hat{\lambda}}}{1-e^{-30\hat{\lambda}}}-240=0\\
\hat{\lambda}&\approx0,0074
\end{align*}
\item[г)]
\begin{align*}
L&=\lambda e^{-\lambda x_1}\lambda e^{-\lambda x_2}\prod^{10}_{i=3}e^{-20\lambda}\\
\ln L&=2\ln \lambda-\lambda(x_1+x_2)-160\lambda\\
\hat{\lambda}&=\frac{1}{95}
\end{align*}
\end{enumerate}
\item У всех одинаковые шансы стать Самым Главным, что можно доказать методом математической индукции. Пусть всего метеорологов двое. Тогда, если у них монеты упали одной стороной, они оба выбывают, если разными, то игра продолжается. Если метеорологов трое и у всех монеты выпали одной стороной, то все выходят из игры; если у одного из них выпала решка, а у остальных – орлы, то первый выбывает, получаем ситуацию для двух и т.д.
\item 
\begin{enumerate}
\item[a)] Обозначим количества орлов в соответствующие дни за $S_1$, $S_2$, $S_3$, а количества бросков – за $n_1$, $n_2$, $n_3$. По своим данным Анна может построить оценку $\hat{p}_a=(S_1+S_2)/(n_1+n_2)$, а Белла – $\hat{p}_b=(S_2+S_3)/(n_2+n_3)$. Наша цель – минимизировать по $\alpha$ дисперсию взвешенной оценки:
\[\Var(\hat{p})=\Var(\alpha\hat{p}_a+(1-\alpha)\hat{p}_b)\]
Или явно:
\begin{align*}
    p(1-p)\left(\frac{\alpha}{n_1+n_2}+\frac{(1-\alpha)^2}{n_2+n_3}+\frac{2\alpha(1-\alpha)n_2}{(n_1+n_2)(n_2+n_3)}\right)
\end{align*}
\[
\alpha^*=\frac{n_1}{n_1+n_3}
\]
Оценки: $\hat{p}_a=0.4$, $\hat{p}_b=0.5$, $\alpha=0.4$, $\hat{p}=0.46$.

\item[б)] Подставим найденную $\alpha^*$ в формулу для дисперсии и получим
\[
\frac{n_1n_2+n_2n_3+n_1n_3}{(n_1+n_2)(n_1+n_3)(n_2+n_3)}p(1-p)
\]
При наших данных $\hat{\Var}(\hat{p})\approx0.0458$.

\item[в)] Используя стандартную формулу, получим:
\[
[\hat{p}-1.96se(\hat{p});\hat{p}+1.96se(\hat{p})],
\]
где $\hat{p}=0.46$ и $se(\hat{p})\approx0.214$.
\end{enumerate}
\end{enumerate}

\subsection[2017-2018]{\hyperref[sec:kr_03_ip_2017_2018]{2017-2018}}
\label{sec:sol_kr_03_ip_2017_2018}



\begin{enumerate}
\item
\begin{enumerate}
	\item[а) - в)] См. картинку :)
\begin{figure}[h!]
\centering
\begin{tikzpicture}
\coordinate (e) at (1,0);
\coordinate (X) at (4,3);
\coordinate (hatX) at (4,0);
\coordinate (perpX) at (0,3);
\draw[->] (0,0) -- (e);
\node [below] at (e) {e};
\draw[->] (0,0) -- (X);
\node [right] at (X) {X};
\draw[->] (0,0) -- (hatX);
\node [below] at (hatX) {$\hat{X}$};
\draw[->] (0,0) -- (perpX);
\node [left] at (perpX) {$\hat{X}^{\perp}$};
\draw [dashed] (perpX) -- (X);
\draw [dashed] (hatX) -- (X);
\end{tikzpicture}
\end{figure}
\item[г)] $\hat{X}=e\cdot\bar{X}$

$\lVert \hat{X} \rVert=\sqrt{n}\cdot\bar{X}$

$\hat{X}^{\perp}=X-e\cdot\bar{X}=(X_1-\bar{X}, \dots ,X_n-\bar{X})$

$\lVert\hat{X}^{\perp} \rVert =\sqrt{\sum^n_{i=1}(X_i-\bar{X})^2}$

\item[д)] $ \lVert X \rVert^2=\lVert\hat{X}^{\perp}\rVert^2+\lVert\hat{X}\rVert^2$

$\sum^n_{i=1}X^2_i=\sum^n_{i=1}(X_i-\bar{X})^2+n\bar{X}^2$

\item[е)] t-статистика для построения доверительного интервала для $\mu$ имеет вид:

\begin{align*}
t &= \frac{\bar{X}-\mu}{\sqrt{\bar{\sigma}^2/n}} = \frac{\bar{X}-\mu}{\sqrt{\sum^n_{i=1}(X_i-\bar{X})^2/(n\cdot(n-1))}}\\
& =\sqrt{n-1}\cdot\frac{\sqrt{n}\cdot(\bar{X}-\mu)}{\sqrt{\sum^n_{i=1}(X_i-\bar{X})^2}}=\sqrt{n-1}\cdot\frac{\lVert \hat{X} \rVert-\sqrt{n}\cdot\mu}{\lVert\hat{X}^{\perp} \rVert}
\end{align*}

Заметим, что $\ctg \alpha$ есть отношение прилежащего катета к противолежащему, таким образом, нужный нам угол $\alpha$ образуется между векторами $X$ и $\hat{X}$. Зметим однако, что в нашем случае
\[
t=\sqrt{n-1}\cdot \ctg \alpha=\frac{\lVert\hat{X}\rVert}{\lVert\hat{X}^{\perp}\rVert},
\]
то есть наша статистика подойдёт только для проверки гипотезе о равенстве матожидания нулю.

Замечание. $t=\sqrt{n-1}\cdot \ctg \alpha$ будет t-статистикой только в том случае, если $X_i$ будут н.о.р.с.в. с нормальным распределением, о чём в условие сказано не было.
\end{enumerate}
\item Выпишем функцию правдоподобия для выборки из трёх видов, два из которых совпадают. Первый медведепришелец будет нового вида с вероятностью 1. Вероятность, что вид второго пойманного медведепришельца совпадёт с первмым, составляет $1/n$. После этого нужно поймать медведепришельца нового вида – это произойдёт с вероятностью $(n-1)/n$, и ещё одного нового вида – вероятность этого $(n-2)/n$. Поскольку медведепришелец, вид которого встречается дважды, мог встретить на любой из трёх позиций, функцию правдоподобия необходимо домножить на $C_3^1$. Таким образом, функция правдоподобия имеет вид:
\[
L(n) = C_3^1 \cdot 1 \cdot \frac{1}{n} \cdot \frac{n-1}{n} \cdot \frac{n-2}{n}, n \geq 3.
\]
Максимизируя её, внутри области определения получаем $\hat n = 5$.

Так как количество медведей велико и все они встречаются равновероятно, то $p_{1}=p_{2}=p_{3}=1/n$. Так же из выборки известно, что число видов космомедведей не меньше трёх. Потому $\hat{n} \ge 3$.

Найдите хитрую ошибку в предложенном решении:

$L(n)=\left(\frac{1}{n}\right)^{2} \cdot \frac{1}{n}\cdot\frac{1}{n}=n^{-4}$

$\frac{\partial L(n)}{\partial n} = -4\cdot n^{-5}=0$

Данное уравнение не имеет решений при конечных $n$, но заметим, что при всех $n \ge 3$ выполняется $\frac{\partial L(n)}{\partial n} = -4\cdot n^{-5} < 0$, таким образом максимальное значение находится в граничных точках.

$\lim\limits_{n\to\infty}\frac{1}{n^{-4}}=0 < \frac{1}{3^{-4}}$

Таким образом получаем, что $\hat{n}=3$.

\item \begin{enumerate}
\item $L(p_{1},p_{2})=p_{1}^{150}\cdot p_{2}^{100}\cdot(1-p_{1}-p_{2})^{50}$

$\ell(p_{1},p_{2}) = 150\ln p_{1} +100\ln p_{2}+50\ln (1-p_{1}-p_{2})$

$\begin{cases}
\frac{\partial \ell(p_{1},p_{2})}{\partial p_{1}}= \frac{150}{p_{1}} - \frac{50}{1-p_{1}-p_{2}}=0
\\
\frac{\partial \ell(p_{1},p_{2})}{\partial p_{2}}= \frac{100}{p_{2}} - \frac{50}{1-p_{1}-p_{2}}=0
\end{cases}$

Откуда получаем:

$\begin{cases}
\hat{p}_{1}=1/2
\\
\hat{p}_{2}=1/3
\end{cases}$
\item Найдём, какие значения должны стоять в теоретической ковариационной матрице.
Заметим, что случайная величина найти кальмаромедведя ($X$) или двурога ($Y$) есть бернулевская случайная величина с параметром $p_{1+2}=p_{1}+p_{2}$ и дисперсией $p_{1+2}\cdot(1-p_{1+2})$, но тогда:

$\Var(X + Y)=\Var(X)+\Var(Y)+2\cdot\Cov(X,Y)$

$\Cov(X,Y)=\frac{1}{2}\cdot(\Var(X+Y)-\Var(X)-\Var(Y))=\frac{1}{2}\cdot((p_{1}+p_{2})\cdot(1-p_{1}-p_{2})-p_{1}\cdot(1-p_{1})-p_{2}\cdot(1-p_{2}))=-p_{1}\cdot p_{2}$

Тогда подставляя в теоретическую ковариационную матрицу оценки параметров и домнажая всё на $1/300$, так как $\hat{p}_{1}$ и $\hat{p}_{2}$ являются средними, получим:

\[
\Var(\hat{p})=\frac{1}{n}\begin{pmatrix}
\hat{p}_{1}\cdot(1-\hat{p}_{1}) & -\hat{p}_{1}\cdot\hat{p}_{2}
\\
-\hat{p}_{1}\cdot\hat{p}_{2} & \hat{p}_{1}\cdot(1-\hat{p}_{1})
\end{pmatrix}=\frac{1}{300}\begin{pmatrix}
1/4 &-1/6
\\
-1/6 & 2/9
\end{pmatrix}
\]

\item Для начала, найдём теоретическую дисперсию $\Var(X-Y)$.

\[
\Var(X - Y)=\Var(X)+\Var(Y)-2\cdot\Cov(X,Y)=p_{1}\cdot(1-p_{1})+p_{2}\cdot(1-p_{2})+2\cdot p_{1}\cdot p_{2}
\]

Тогда подставляя оценки для $p_{1}$ и $p_{2}$ и учитывая, что это оценки среднего, получим оценку:

\[
\Var(\hat{p}_{1}-\hat{p}_{2})=1/300\cdot(1/4+2/9+2\cdot1/6)=29/(36\cdot300)
\]

\item Так как выборка достаточно велика, то статистика $\hat{p}_{1}-\hat{p}_{2}$,
являясь средним, будет иметь примерно нормальное распределение, и тогда:

$\frac{\hat{p}_{1}-\hat{p}_{2}-(p_{1}-p_{2})}{\sqrt{\Var(\hat{p}_{1}-\hat{p}_{2})}}\sim \cN(0,1)$

$\hat{p}_{1}-\hat{p}_{2}-z_{1-\frac{\alpha}{2}}\cdot\sqrt{\Var(\hat{p}_{1}-\hat{p}_{2})} \le p_{1}-p_{2} \le \hat{p}_{1}-\hat{p}_{2}-z_{\frac{\alpha}{2}}\cdot\sqrt{\Var(\hat{p}_{1}-\hat{p}_{2})}$
\end{enumerate}
\item \begin{enumerate}
\item $\hat{\alpha}=\bar{X}+\sqrt{\bar{X}+6}=10+\sqrt{10+6}=14$

\item Так как $\bar{X}$ сходится по распределению к нормальному распределению и $\hat{\alpha}=g(\bar{X})$, где $g(\bar{X})$ гладкая по $\bar{X}$ функция при $\bar{X}\ge0$, а также $\bar{X}$ сходится по вероятности к матожиданию, то можно абсолютно спокойно применить дельта-метод. Тогда:

\[
(\alpha-g(\bar{X}))\sim N(0;\sigma^{2}(g'(\E(X_{1})))^{2}/n)
\]

Но так как $\hat{\alpha}$ является состоятельной оценкой, то можно заменить $g'(\E(X_{1}))$ на $g'(\bar{X})$:
\[
g'(\bar{X})=1+\frac{1}{2\cdot\sqrt{\bar{X}+6}}=1+\frac{1}{2\cdot4}=\frac{9}{8}=1.125
\]
и тогда можно построить ассимтотический доверительный интервал:

\begin{align*}
%z_{2.5\%}\le&\frac{\hat{\alpha}-\alpha}{\sqrt{\sigma^{2}\cdot(g'(\bar{X}))^{2}/n}}\le z_{97.5\%} \\
\hat{\alpha}-z_{97.5\%}\cdot\sqrt{\sigma^{2}\cdot(g'(\bar{X}))^{2}/n}\le &\alpha \le
\hat{\alpha}-z_{2.5\%}\cdot\sqrt{\sigma^{2}\cdot(g'(\bar{X}))^{2}/n} \\
16-1.96\cdot2\cdot9/(8\cdot10)\le&\alpha\le 16+1.96\cdot2\cdot9/(8\cdot10) \\
13.559\le&\alpha\le 14.441
\end{align*}
\end{enumerate}
\item \begin{enumerate}
\item Так как не известно точно, кто сколько фотографий сделал, и так как метод оценки не указан,
то воспользуемся методом моментов для построения оценки.

\begin{align*}
N&=\E(\text{«фото Андрея»})+\E(\text{«фото Беллы»}) \\
130&=100\cdot 0.5+p\cdot100 \\
\hat{p}&=0.8
\end{align*}

Так как выборка достаточно велика, то $\frac{\hat{p}-p}{\sqrt{\hat{p}\cdot(1-\hat{p})/W}}\sim \cN(0,1)$

\begin{align*}
\hat{p}-z_{97.5\%}\sqrt{\hat{p}\cdot(1-\hat{p})/W} \le &p \le \hat{p}-z_{2.5\%}\sqrt{\hat{p}\cdot(1-\hat{p})/W} \\
0.8-1.96\cdot\sqrt{0.8\cdot0.2/100}\le &p\le0.8+1.96\cdot\sqrt{0.8\cdot0.2/100} \\
0.72\le &p\le 0.88
\end{align*}

\item Так как неизвестно, кто больше снимков сделал, то рассмотрим два случая: Андрей сделал 60 фото и Белла — 70 фото, Андрей сделал 70 фото и Белла — 60 фото. В каждом случае при помощи метода максимального правдоподобия оценим вероятность $p$, после чего сравним значения функции правдоподобия с оценёнными параметрами для каждого случая.
\begin{align*}
L(p)&=C^{60}_{100}\cdot 0.5^{60}\cdot 0.5^{40}\cdot C^{70}_{100}\cdot p^{70}\cdot(1-p)^{30} \\
\ell(p)&=const+70\ln p+30\ln(1-p) \\
\frac{\partial \ell (p)}{\partial p}&= \frac{70}{p}-\frac{30}{1-p}=0 \\
\hat{p}_1 &= 0.7
\end{align*}

Аналогично для второго случая получим оценку: $\hat{p}_2=0.6$.

Для простоты, будем сравнивать логарифмическии функции правдоподобия $\ell_1(p_1)$ и $\ell_2(p_2)$ и тогда получим:

\begin{align*}
\ell_1(p_1)&=const+70\ln 0.7+30\ln 0.3\approx const-70\cdot0.357-30\cdot 1.204=const-61.11 \\
\ell_2(p_2)&=const+60\ln 0.6+40\ln 0.4\approx const-60\cdot0.511-40\cdot0.916=const-67.3
\end{align*}

Так как $-67.3<-61.11$, то более вероятно, что $\hat{p}=0.7$

Тогда анологично предыдущему пункту получим доверительный интервал:

\begin{align*}
0.7-1.96\cdot\sqrt{0.7\cdot0.3/100}\le &p \le0.7+1.96\cdot\sqrt{0.7\cdot0.3/100} \\
0.61 \le &p \le 0.79
\end{align*}
\end{enumerate}
\end{enumerate}

\section{Решения контрольной номер 4}

\subsection[2017-2018]{\hyperref[sec:kr_04_2017_2018]{2017-2018}}
\label{sec:sol_kr_04_2017_2018}



\begin{enumerate}
\item Проверяем следующую гипотезу:
\[
\begin{cases}
H_0: \mu_{D} = 100 \\
H_a: \mu_{D} > 100
\end{cases}
\]
Считаем наблюдаемое значение статистики:
\[
t_{obs} = \frac{\bar X - \mu_{D}}{\frac{\sigma_D}{\sqrt{n_D}}} = \frac{136 - 100}{\frac{55}{\sqrt{40}}} \approx 4.14
\]
При верной $H_0$ $t$-статистика имеет распределение $t_{40 - 1}$, значит, $t_{crit} \approx 1.68$.
Поскольку $t_{crit} > t_{obs}$, основная гипотеза отвергается, $p-value \approx 0$.

\item Проверяем следующую гипотезу:
\[
\begin{cases}
H_0: \sigma^2_D = \sigma^2_T \\
H_a: \sigma^2_D \neq \sigma^2_T
\end{cases}
\]
Считаем наблюдаемое значение статистики:
\[
F_{obs} = \frac{\hat{\sigma}^2_D}{\hat{\sigma}^2_T} = \frac{55^2}{60^2} \approx 0.84
\]
При верной $H_0$ $F$-статистика имеет распределение $F_{40-1, 60-1}$.
Находим критические значения: $F_{left} \approx 0.6$, $F_{right} \approx 1.6$.
Поскольку $F_{left} < F_{obs} < F_{right}$, нет оснований отвергать $H_0$.
\item
\begin{enumerate}
Проверяем гипотезу
\[
\begin{cases}
H_0: \mu_{D} = \mu_{T} \\
H_a: \mu_{D} < \mu_{T}
\end{cases}
\]
\item Когда $n_D$, $n_T$ велики,
\[
\frac{\bar D - \bar T - (\mu_D - \mu_T)}{\sqrt{\frac{\sigma^2_D}{n_D} + \frac{\sigma^2_T}{n_T}}} \stackrel{H_0}{\sim} \cN(0, 1)
\]
Считаем наблюдаемое значение статистики:
\[
z_{obs} = \frac{136 - 139}{\sqrt{\frac{3025}{40} + \frac{3600}{60}}} \approx -0.25
\]
По таблице находим $z_{crit} = -1.28$.
Так как $z_{crit} < z_{obs}$, нет оснований отвергать $H_0$.
\item Когда считаем дисперсии одинаковыми, то:
\[
\hat{\sigma}^2_0 = \frac{\hat{\sigma}^2_D (n_D - 1) + \hat{\sigma}^2_T (n_T - 1)}{n_D + n_T - 2} = \frac{3025 \cdot 39 + 3600 \cdot 59}{30 + 60 - 2} \approx 3371
\]
и
\[
\frac{\bar D - \bar T - (\mu_D - \mu_T)}{\hat{\sigma}^2_0\sqrt{\frac{1}{n_D} + \frac{1}{n_T}}} \stackrel{H_0}{\sim} t_{n_D + n_T - 2}
\]
Считаем наблюдаемое значение статистики:
\[
t_{obs} = \frac{136 - 139}{\sqrt{3371}\sqrt{\frac{1}{40} + \frac{1}{60}}} \approx -0.25
\]
По таблице находим критическое значение: $t_{crit} \approx -1.29$.
Поскольку $t_{crit} < t_{obs}$, нет оснований отвергать $H_0$.
\end{enumerate}
\item
\begin{enumerate}
\item Сначала найдём оценку максимального правдоподобия параметра $\lambda$:
\begin{align*}
L &= \prod_{i=1}^n \lambda e^{-\lambda x_i} = \lambda^n e^{-\lambda \sum_{i=1}^n x_i} \\
\ell &= n \ln \lambda - \lambda \sum_{i=1}^n x_i \\
\frac{\partial \ell}{\partial \lambda} &= \left. \frac{n}{\lambda} \right|_{\lambda = \hat \lambda} = 0 \\
\frac{\partial^2 \ell}{\partial \lambda^2} &= -\frac{n}{\lambda^2} \\
\hat \lambda &= \frac{1}{0.52}
\end{align*}
Так как
\[
\frac{\hat \lambda - \lambda}{\sqrt{\frac{1}{I(\lambda)}}} \stackrel{as}{\sim} \cN(0,1),
\]
доверительный интервал имеет вид
\[
\frac{1}{0.52} - 1.96 \frac{1}{\frac{10}{0.52}} < \lambda < \frac{1}{0.52} + 1.96 \frac{1}{\frac{10}{0.52}}
\]
\item Найдём вероятность того, что наушник проработает без сбоев 45 минут:
\[
g(\lambda) = \P(X > 0.75) = 1 - F(0.75) = e^{-0.75\lambda}
\]
Тогда
\begin{align*}
g(\hat \lambda) &= e^{-0.75 / 0.52} \\
g'(\hat \lambda) &= -0.75 e^{-0.75 / 0.52}
\end{align*}
И доверительный интервал имеет вид:
\[
e^{-0.75 / 0.52} - 1.96 \cdot \frac{0.75 \cdot 0.52}{10} \cdot e^{-0.75 / 0.52} < g(\lambda) < e^{-0.75 / 0.52} + 1.96 \cdot \frac{0.75 \cdot 0.52}{10} \cdot e^{-0.75 / 0.52}
\]
\end{enumerate}
\item Выпишем функцию правдоподобия:
\begin{align*}
L &= p_1^{10} \cdot p_2^{10} \cdot p_3^{15} \cdot p_4^{15} \cdot p_5^{25} \cdot (1 - p_1 - p_2 - p_3 - p_4 - p_5)^{25} \\
\ell &= 10 \ln p_1 + 10 \ln p_2 + 15 \ln p_3 + 15 \ln p_4 + 25 \ln p_5 + 25 \ln (1 - p_1 - p_2 - p_3 - p_4 - p_5)
\end{align*}
Максимизируя логарифмическую функцию правдоподобия по всем параметрам,
получим следующие оценки для неограниченной модели:
\begin{align*}
& \hat p_1 = \hat p_2 = 0.1 \\
& \hat p_3 = \hat p_4 = 0.15 \\
& \hat p_5 = 0.25
\end{align*}
Подставив найденные значения в логарифмическую функцию правдоподобия, получим
\[
\ell_{UR} \approx -172
\]
В ограниченной модели $p_1 = \ldots = p_6 = 1/6$, и значение функции правдоподобия
будет
\[
\ell_R \approx -179
\]
Теперь можно посчитать наблюдаемое значение:
\[
LR = 2(\ell_{UR} - \ell_R) = 2(-172 - (-179)) = 14
\]
Критическое значение $\chi_{0.95, 5} = \approx 11 < 14$, значит, основная гипотеза
отвергается.
\end{enumerate}

% % !TEX root = ../probability_hse_exams.tex
\thispagestyle{empty}
\section{Решения контрольной номер 4. ИП}




\subsection[2018-2019]{\hyperref[sec:kr_04_ip_2018_2019]{2018-2019}}
\label{sec:sol_kr_04_ip_2018_2019}


\subsection[2017-2018]{\hyperref[sec:kr_04_ip_2017_2018]{2017-2018}}
\label{sec:sol_kr_04_ip_2017_2018}
% \subsection*{Решения финальных экзаменов}


\subsubsection*{\hyperref[sec:final_exam_2017_2018]{2017-2018}}
\label{sec:sol_final_exam_2017_2018}

Здесь табличка с ответами



\end{document}
