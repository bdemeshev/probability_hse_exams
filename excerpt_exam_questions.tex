\documentclass[12pt]{article}

\usepackage{etoolbox} % для условия if-else
\newtoggle{excerpt} % помечаем, что это отрывок, а далее в тексте может использовать
\toggletrue{excerpt}
% команду \iftoggle{excerpt}{yes}{no}

\usepackage{tikz} % картинки в tikz
\usepackage{microtype} % свешивание пунктуации

\usepackage{array} % для столбцов фиксированной ширины

\usepackage{indentfirst} % отступ в первом параграфе

\usepackage{sectsty} % для центрирования названий частей
\allsectionsfont{\centering}

\usepackage{amsmath} % куча стандартных математических плюшек

\usepackage{comment}
\usepackage{amsfonts}

\usepackage[top=2cm, left=1.2cm, right=1.2cm, bottom=2cm]{geometry} % размер текста на странице

\usepackage{lastpage} % чтобы узнать номер последней страницы

\usepackage{enumitem} % дополнительные плюшки для списков
%  например \begin{enumerate}[resume] позволяет продолжить нумерацию в новом списке
\usepackage{caption}


\usepackage{fancyhdr} % весёлые колонтитулы
\pagestyle{fancy}
\lhead{Теория вероятностей-ВШЭ}
\chead{}
\rhead{Вопросы к экзаменам по ТВ и МС}
\lfoot{}
\cfoot{}
\rfoot{\thepage/\pageref{LastPage}}
\renewcommand{\headrulewidth}{0.4pt}
\renewcommand{\footrulewidth}{0.4pt}



\usepackage{todonotes} % для вставки в документ заметок о том, что осталось сделать
% \todo{Здесь надо коэффициенты исправить}
% \missingfigure{Здесь будет Последний день Помпеи}
% \listoftodos --- печатает все поставленные \todo'шки


% более красивые таблицы
\usepackage{booktabs}
% заповеди из докупентации:
% 1. Не используйте вертикальные линни
% 2. Не используйте двойные линии
% 3. Единицы измерения - в шапку таблицы
% 4. Не сокращайте .1 вместо 0.1
% 5. Повторяющееся значение повторяйте, а не говорите "то же"



\usepackage{fontspec}
\usepackage{libertine}
\usepackage{polyglossia}

\setmainlanguage{russian}
\setotherlanguages{english}

% download "Linux Libertine" fonts:
% http://www.linuxlibertine.org/index.php?id=91&L=1
% \setmainfont{Linux Libertine O} % or Helvetica, Arial, Cambria
% why do we need \newfontfamily:
% http://tex.stackexchange.com/questions/91507/
% \newfontfamily{\cyrillicfonttt}{Linux Libertine O}

\AddEnumerateCounter{\asbuk}{\russian@alph}{щ} % для списков с русскими буквами
\setlist[enumerate, 2]{label=\asbuk*),ref=\asbuk*}

%% эконометрические сокращения
\DeclareMathOperator*{\argmin}{arg\,min}
\DeclareMathOperator*{\argmax}{arg\,max}
\DeclareMathOperator*{\amn}{arg\,min}
\DeclareMathOperator*{\amx}{arg\,max}
\DeclareMathOperator{\cov}{Cov}
\DeclareMathOperator{\Var}{Var}
\DeclareMathOperator{\Cov}{Cov}
\DeclareMathOperator{\Corr}{Corr}
\DeclareMathOperator{\pCorr}{pCorr}
\DeclareMathOperator{\E}{\mathbb{E}}
\let\P\relax
\DeclareMathOperator{\P}{\mathbb{P}}


\newcommand{\cN}{\mathrm{N}}
\newcommand{\cU}{\mathrm{U}}
\newcommand{\cBinom}{\mathrm{Binom}}
\newcommand{\cBin}{\cBinom}
\newcommand{\cExp}{\mathrm{Exp}}
\newcommand{\cPois}{\mathrm{Pois}}
\newcommand{\cBeta}{\mathrm{Beta}}
\newcommand{\cGamma}{\mathrm{Gamma}}

\newcommand \R{\mathbb{R}}
\newcommand \N{\mathbb{N}}
\newcommand \Z{\mathbb{Z}}


\begin{document}

\newpage
\thispagestyle{empty}
\section{Вопросы к экзаменам}

\subsection*{Промежуточный экзамен}

\begin{enumerate}
	\item Аксиоматика Колмогорова. Случайные величины. Функция распределения случайной величины и ее основные свойства. Функция плотности
	\item Виды сходимости последовательности случайных величин
	\item Основные дискретные распределения: биномиальное, Пуассона, гипергеометрическое, отрицательное биномиальное. Примеры непрерывных распределений (равномерное, экспоненциальное)
	\item Неравенство Маркова и неравенство Чебышёва. Закон больших чисел
	\item Понятие о случайном векторе. Совместное распределение нескольких случайных величин. Независимость случайных величин. Маргинальные распределения
	\item Центральная предельная теорема
	\item Условная вероятность. Формула полной вероятности. Формула Байеса
	\item Математическое ожидание и дисперсия случайной величины и их свойства. Распределение функции от случайной величины
	\item Случайные события и операции над ними. Вероятностное пространство. Вероятности и правила действий с ними. Классическое определение вероятности. Независимость событий (попарная и в совокупности). Схема испытаний Бернулли
	\item Математическое ожидание и ковариационная матрица случайного вектора. Коэффициент корреляции и его свойства
	\item Условное распределение и условное математическое ожидание
	\item Теорема Муавра – Лапласа
	\item Неравенство Маркова и неравенство Чебышёва. Закон больших чисел
\end{enumerate}

\subsection*{Финальный экзамен}
\begin{enumerate}

\item Многомерное нормальное распределение и его свойства.
\item Определение и свойства хи-квадрат распределения, распределения Стьюдента и Фишера. Их основные свойства. Работа с таблицами распределений.
\item Выборочное среднее, его математическое ожидание и дисперсия (с учетом поправки на конечный размер генеральной совокупности).
\item Выборочная дисперсия и ее математическое ожидание. Смещенная и несмещенная оценки для дисперсии по генеральной совокупности.
\item Стратифицированная случайная выборка. Выборочное среднее, его математическое ожидание. Дисперсия выборочного среднего при оптимальном и при пропорциональном размещении.
\item Статистические оценки. Свойства оценок; несмещенность, состоятельность, эффективность.
\item Методы получения оценок; метод моментов и метод максимального правдоподобия. Оценка параметров биномиального, нормального и равномерного распределений.
\item Информация Фишера. Неравенство Рао-Крамера-Фреше (без доказательства).
\item Доверительные интервалы. Доверительные интервалы для среднего при известной и неизвестной дисперсии. Доверительные интервалы для пропорции.
\item Доверительные интервалы для разности средних нормальных генеральных совокупностей.
\item Доверительный интервал для дисперсии нормальной генеральной совокупности.
\item Асимптотические доверительные интервалы параметров распределений, построенные с помощью оценок максимального правдоподобия.  Дельта-метод.
\item Проверка гипотез. Простые и сложные гипотезы. Критерий выбора между основной и альтернативной гипотезами. Уровень значимости. Мощность критерия. Ошибки первого и второго рода.
\item Проверка гипотез о конкретном значении для среднего, пропорции и дисперсии.
\item Проверка гипотез для разности двух средних и для разности двух пропорций. Проверка гипотез о равенстве двух дисперсий.
\item Лемма Неймана-Пирсона. Критерий отношения правдоподобия.
\item Критерии согласия. Статистика Колмогорова.
\item Критерий $\chi^2$. Проверка гипотез о соответствии наблюдений предполагаемому распределению вероятностей.
\item Критерий $\chi^2$. Проверка гипотезы о независимости признаков. Таблицы сопряженности признаков.
\item Непараметрические тесты. Критерий знаков. Ранговые критерии: Вилкоксона и Манна-Уитни.
\item Байесовский подход. Связь априорного и апостериорного распределения.
Отличия байесовского подхода к оцениванию параметров от метода максимального правдоподобия.
Байесовский доверительный интервал.
\item Байесовский подход. Алгоритм Гиббса. Алгоритм Метрополиса-Гастингса.
\end{enumerate}


\end{document}
