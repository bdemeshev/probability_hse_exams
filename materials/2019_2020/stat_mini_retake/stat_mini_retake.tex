\documentclass[12pt]{article}

\usepackage{tikz} % картинки в tikz
\usepackage{microtype} % свешивание пунктуации

\usepackage{array} % для столбцов фиксированной ширины

\usepackage{indentfirst} % отступ в первом параграфе

\usepackage{sectsty} % для центрирования названий частей
\allsectionsfont{\centering}

\usepackage{amsmath} % куча стандартных математических плюшек

\usepackage{comment}
\usepackage{amsfonts}

\usepackage{verbatim}

\usepackage{graphicx} % for png images

\usepackage[colorlinks=true, linkcolor=blue]{hyperref}

\usepackage[top=2cm, left=1cm, right=1cm, bottom=2cm]{geometry} % размер текста на странице

\usepackage{lastpage} % чтобы узнать номер последней страницы

\usepackage{enumitem} % дополнительные плюшки для списков
%  например \begin{enumerate}[resume] позволяет продолжить нумерацию в новом списке
\usepackage{caption}

\usepackage{hyperref} % гиперссылки

\usepackage{multicol} % текст в несколько столбцов


\usepackage{fancyhdr} % весёлые колонтитулы
\pagestyle{fancy}
\lhead{Теория вероятностей и математическая статистика-ВШЭ}
\chead{}
\rhead{Переписывание миниконтрольных}
\lfoot{2020-05-21}
\cfoot{}
\rfoot{}
\renewcommand{\headrulewidth}{0.4pt}
\renewcommand{\footrulewidth}{0.4pt}



\usepackage{todonotes} % для вставки в документ заметок о том, что осталось сделать
% \todo{Здесь надо коэффициенты исправить}
% \missingfigure{Здесь будет Последний день Помпеи}
% \listoftodos --- печатает все поставленные \todo'шки


% более красивые таблицы
\usepackage{booktabs}
% заповеди из докупентации:
% 1. Не используйте вертикальные линни
% 2. Не используйте двойные линии
% 3. Единицы измерения - в шапку таблицы
% 4. Не сокращайте .1 вместо 0.1
% 5. Повторяющееся значение повторяйте, а не говорите "то же"


\usepackage{fontspec}
\usepackage{polyglossia}

\setmainlanguage{russian}
\setotherlanguages{english}

\usepackage{xeCJK} % японские буквы
\setCJKmainfont{Noto Sans CJK JP} % и шрифт к ним :)

% download "Linux Libertine" fonts:
% http://www.linuxlibertine.org/index.php?id=91&L=1
\setmainfont{Linux Libertine O} % or Helvetica, Arial, Cambria
% why do we need \newfontfamily:
% http://tex.stackexchange.com/questions/91507/
\newfontfamily{\cyrillicfonttt}{Linux Libertine O}

\AddEnumerateCounter{\asbuk}{\russian@alph}{щ} % для списков с русскими буквами
\setlist[enumerate, 2]{label=\asbuk*),ref=\asbuk*}

%% эконометрические сокращения
\DeclareMathOperator{\Cov}{Cov}
\DeclareMathOperator{\Corr}{Corr}
\DeclareMathOperator{\Var}{Var}
\DeclareMathOperator{\E}{E}
\def \hb{\hat{\beta}}
\def \hs{\hat{\sigma}}
\def \htheta{\hat{\theta}}
\def \s{\sigma}
\def \hy{\hat{y}}
\def \hY{\hat{Y}}
\def \v1{\vec{1}}
\def \e{\varepsilon}
\def \he{\hat{\e}}
\def \z{z}
\def \hVar{\widehat{\Var}}
\def \hCorr{\widehat{\Corr}}
\def \hCov{\widehat{\Cov}}
\def \cN{\mathcal{N}}
\def \P{\mathbb{P}}
\newcommand \id {\mathrm{id}\_\mathrm{for}\_\mathrm{online}}

\begin{document}



\vspace{20mm}

\textbf{Буси-до, 武士道, кодекс чести самурая:}

\vspace{5mm}

\begin{enumerate}  
\item Подпишите работу сверху: \verb|id_for_online|, фамилию, имя, номер группы.
\item Для задач должно быть выписано решение, только ответ не засчитывается. 
В тесте проверяется только ответ. 
\item Для каждого пункта задания обведите полученный результат в торжественную рамочку.

\item Ссылка на гитхаб: \url{https://classroom.github.com/a/Nau7nIP5}.

\item Имя файла должно иметь вид \verb|retake_kr2_NNN.pdf| или \verb|retake_kr3_NNN.pdf| 
или \verb|retake_test_NNN.pdf| в зависимости от того, что переписывается. 
Вместо цифр \verb|NNN| следует написать \verb|id_for_online|.

\item При загрузке версий работы и на гитхаб, и в лмс, проверяется версия из лмс. 
\item Начало переписывания: 15:30. 

\item Пропустившим две контрольных предоставляется право переписать вторую из пропущенных, 
  за первую остаётся ноль. 
  Загрузите свою работу в лмс или на гитхаб в виде одного \verb|.pdf| файла.
  
  Дедлайн: 16:00 — без штрафа, 16:05 — со штрафом 30\%, 16:10 — со штрафом 60\%. 
\item Пропустившим три контрольных предоставляется право переписать вторую и третью, 
  за первую пропущенную остаётся ноль. 
  Загрузите свою работу в лмс или на гитхаб в виде двух \verb|.pdf| файлов.
  
  Дедлайн по первой: 16:00 — без штрафа, 16:05 — со штрафом 30\%, 16:10 — со штрафом 60\%.
  
  Дедлайн по второй: 16:30 — без штрафа, 16:35 — со штрафом 30\%, 16:40 — со штрафом 60\%.


\item Оперативные важные сообщения будут в телеграм-канале \verb|@room112|.
\end{enumerate}



\begin{center}
  \begin{minipage}{5cm}
    \includegraphics[height=30ex]{cheshire_cat.png}
  \end{minipage}
\end{center}
  
\newpage

\section*{Миниконтрольная №2}

Имеется случайная выборка $X_1$, $X_2$, \ldots, $X_n$ из распределения с функцией плотности

\[  
f(x) = \begin{cases}
\theta x^{\theta - 1}, \text{ при } x\in [0;1], \\
0, \text{ иначе.}
\end{cases}
\]

\begin{enumerate}
  \item Методом моментов, используя первый момент, найдите оценку параметра $\theta$.
  \item Методом максимального правдоподобия найдите
  \begin{enumerate}
    \item оценку параметра $\theta$;
    \item оценку $\E(X_i)$.
  \end{enumerate}
  \item Вычислите информацию Фишера о параметре $\theta$, содержащуюся во всей выборке.
  \item Вычислите асимптотическую дисперсию оценки максимального правдоподобия параметра. 
  \item Вычислите асимптотическую дисперсию оценки максимального правдоподобия.
  \item Найдите оценку асимптотической дисперсии оценки максимального правдоподобия $\E(X_i)$. 
\end{enumerate}


\section*{Миниконтрольная №3}

В этой миниконтрольной константа $k$ — это номер \verb|id_for_online|.

\begin{enumerate}
\item По случайной выборке $X_1$, $X_2$, \ldots, $X_n$ из равномерного распределения $U[0; \theta]$ построены две оценки

\[
T_1 = k_1 \cdot \bar X, \quad T_2 = k_2 \cdot \max\{X_1, X_2, \ldots, X_n \}.
\]

\begin{enumerate}
\item Найдите значения $k_1$ и $k_2$ при которых оценки являются несмещёнными.
\item Проверьте, будет ли несмещённая оценка $T_2$ состоятельной?
\item Какая из двух несмещённых оценок $T_1$ или $T_2$ является более эффективной?
\end{enumerate}
\item Дана случайная выборка $X_1$, $X_2$, \ldots, $X_n$ из дискретного распределения
\[
\P(X_i = m) = e^{-\lambda} \frac{\lambda^{m-k}}{(m-k)!}, \text{ где } m = k, k+1, k+2, \ldots
\]
с неизвестным параметром $\lambda$. Проверьте, будет ли оценка $\hat\lambda = \bar X$ эффективной?

\end{enumerate}

\newpage
\section*{Тест 1.  Выборочные характеристики}

Условие к вопросам 1-12. Дана реализация случайной выборки $x = (2, 0, -1, -1)$. 

\begin{enumerate}
\item Чему равно выборочное среднее?
\item Чему равна выборочная дисперсия (неисправленная выборочная дисперсия)?
\item Чему равна несмещенная оценка дисперсии (исправленная выборочная дисперсия)?
\item Чему равен выборочный второй начальный момент?
\item Чему равен выборочный третий центральный момент?
\item Чему равен первый член вариационного ряда?
\item Чему равен последний член вариационного ряда?
\item Чему равен теоретический второй начальный момент?
\item Что такое теоретический центральный момент второго порядка случайной величины $X$?
\item Что такое теоретический начальный момент второго порядка случайной величины $X$?
\item Чему равна выборочная функция распределения в точке $x = -0.7$?
\item Чему равна выборочная функция распределения в точке $x = 0$?
\item Дайте определение выборочной функции распределения в точке $x$.
\item Отметьте все правильные ответы. Какими свойствами обладает выборочная функция распределения?
\begin{enumerate}
\item выборочная функция распределения нестрого возрастает;
\item выборочная функция распределения строго возрастает;
\item выборочная функция распределения нестрого убывает;
\item выборочная функция распределения строго убывает;
\item при стремлении аргумента к плюс бесконечности выборочная функция распределения стремится к единице;
\item при стремлении аргумента к плюс бесконечности выборочная функция распределения стремится к плюс бесконечности;
\item при стремлении аргумента к плюс бесконечности выборочная функция распределения стремится к нулю;
\item при стремлении аргумента к минус бесконечности выборочная функция распределения стремится к минус бесконечности;
\item при стремлении аргумента к минус бесконечности выборочная функция распределения стремится к нулю;
\item при стремлении аргумента к минус бесконечности выборочная функция распределения стремится к единице;
\item в отличие от теоретической функции распределения выборочная функция распределения может принимать отрицательные значения.
\end{enumerate}

\end{enumerate}



\newpage
Таблица к вопросам 15-19: 
% Please add the following required packages to your document preamble:
% \usepackage{booktabs}
\begin{center}
\begin{tabular}{@{}llll@{}}
  \toprule
   & Страта 1 & Страта 2 & Страта 3 \\ \midrule
Вес   & $0.2$ & $0.3$ & $0.5$ \\ 
Стандартное отклонение  & $2$ & $1$ & $1$ \\ 
Стоимость наблюдения   & $4$ & $9$ & $1$ \\ 
Число наблюдений      & $n_1$ & $n_2$ & $n_3$ \\ \bottomrule
  \end{tabular}
\end{center}

\begin{enumerate}[resume]
\item Общее число наблюдений в выборке равно $n=n_1 + n_2 + n_3$, где $n_1$, $n_2$, $n_3$ — число наблюдений в 1-й, 2-й и 3-й стратах соответственно.
Какое количество наблюдений ($n_1$, $n_2$, $n_3$) должно быть в выборке из 1,2,3 страты при пропорциональном разбиении, если общее число наблюдений $n$ равно 10?

\item Общее число наблюдений в выборке равно $n=n_1 + n_2 + n_3$, где $n_1$, $n_2$, $n_3$ — число наблюдений в 1-й, 2-й и 3-й стратах соответственно.
Каким должно быть общее число наблюдений $n$ в выборке  при бюджетном ограничении 22 у.е. ($4 n_1 + 9 n_2 + n_3 = 22$ )?

\item Общее число наблюдений в выборке равно $n=n_1 + n_2 + n_3$, где $n_1$, $n_2$, $n_3$ — число наблюдений в 1-й, 2-й и 3-й стратах соответственно.
Вычислите дисперсию стратифицированного среднего при пропорциональном разбиении выборки из 10 наблюдений.

\item Общее число наблюдений в выборке равно $n=n_1 + n_2 + n_3$, где $n_1$, $n_2$, $n_3$ — число наблюдений в 1-й, 2-й и 3-й стратах соответственно.
Какое количество наблюдений ($n_1$, $n_2$, $n_3$) должно быть в выборке из 1,2,3 страты при оптимальном разбиении, если фиксировано общее число наблюдений $n$ равно 10?

\item Общее число наблюдений в выборке равно $n=n_1 + n_2 + n_3$, где $n_1$, $n_2$, $n_3$ — число наблюдений в 1-й, 2-й и 3-й стратах соответственно.
Найдите дисперсию стратифицированного среднего при оптимальном  разбиении выборки из 10 наблюдений.

\item В барабане три шара с написанными на них суммами выигрыша: 1, 2, 3. 
Ведущий раскручивает барабан и достаёт без возвращения два шара. 
Чему равна дисперсия среднего выигрыша по двум шарам выигрыша?
\end{enumerate}




\end{document}