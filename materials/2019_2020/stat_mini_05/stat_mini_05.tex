\documentclass[12pt]{article}

\usepackage{tikz} % картинки в tikz
\usepackage{microtype} % свешивание пунктуации

\usepackage{array} % для столбцов фиксированной ширины

\usepackage{indentfirst} % отступ в первом параграфе

\usepackage{sectsty} % для центрирования названий частей
\allsectionsfont{\centering}

\usepackage{amsmath} % куча стандартных математических плюшек

\usepackage{comment}
\usepackage{amsfonts}

\usepackage{verbatim}

\usepackage{graphicx} % for png images

\usepackage[colorlinks=true, linkcolor=blue]{hyperref}

\usepackage[top=2cm, left=1cm, right=1cm, bottom=2cm]{geometry} % размер текста на странице

\usepackage{lastpage} % чтобы узнать номер последней страницы

\usepackage{enumitem} % дополнительные плюшки для списков
%  например \begin{enumerate}[resume] позволяет продолжить нумерацию в новом списке
\usepackage{caption}

\usepackage{hyperref} % гиперссылки

\usepackage{multicol} % текст в несколько столбцов


\usepackage{fancyhdr} % весёлые колонтитулы
\pagestyle{fancy}
\lhead{Теория вероятностей и математическая статистика-ВШЭ}
\chead{}
\rhead{Миниконтрольная №5}
\lfoot{2020-05-28}
\cfoot{}
\rfoot{}
\renewcommand{\headrulewidth}{0.4pt}
\renewcommand{\footrulewidth}{0.4pt}



\usepackage{todonotes} % для вставки в документ заметок о том, что осталось сделать
% \todo{Здесь надо коэффициенты исправить}
% \missingfigure{Здесь будет Последний день Помпеи}
% \listoftodos --- печатает все поставленные \todo'шки


% более красивые таблицы
\usepackage{booktabs}
% заповеди из докупентации:
% 1. Не используйте вертикальные линни
% 2. Не используйте двойные линии
% 3. Единицы измерения - в шапку таблицы
% 4. Не сокращайте .1 вместо 0.1
% 5. Повторяющееся значение повторяйте, а не говорите "то же"


\usepackage{fontspec}
\usepackage{polyglossia}

\setmainlanguage{russian}
\setotherlanguages{english}

\usepackage{xeCJK}
\setCJKmainfont{Noto Sans CJK JP}

% download "Linux Libertine" fonts:
% http://www.linuxlibertine.org/index.php?id=91&L=1
\setmainfont{Linux Libertine O} % or Helvetica, Arial, Cambria
% why do we need \newfontfamily:
% http://tex.stackexchange.com/questions/91507/
\newfontfamily{\cyrillicfonttt}{Linux Libertine O}

\AddEnumerateCounter{\asbuk}{\russian@alph}{щ} % для списков с русскими буквами
\setlist[enumerate, 2]{label=\asbuk*),ref=\asbuk*}

%% эконометрические сокращения
\DeclareMathOperator{\Cov}{Cov}
\DeclareMathOperator{\Corr}{Corr}
\DeclareMathOperator{\Var}{Var}
\DeclareMathOperator{\E}{E}
\def \hb{\hat{\beta}}
\def \hs{\hat{\sigma}}
\def \htheta{\hat{\theta}}
\def \s{\sigma}
\def \hy{\hat{y}}
\def \hY{\hat{Y}}
\def \v1{\vec{1}}
\def \e{\varepsilon}
\def \he{\hat{\e}}
\def \z{z}
\def \hVar{\widehat{\Var}}
\def \hCorr{\widehat{\Corr}}
\def \hCov{\widehat{\Cov}}
\def \cN{\mathcal{N}}
\def \P{\mathbb{P}}
\newcommand \id {\mathrm{id}\_\mathrm{for}\_\mathrm{online}}

\begin{document}



\vspace{20mm}

\textbf{Буси-до, 武士道, кодекс чести самурая:}

\vspace{5mm}

\begin{enumerate}
\item Подпишите работу сверху: \verb|id_for_online|, фамилию, имя, номер группы.
\item Должно быть выписано решение задачи, только ответ не засчитывается.
\item Для каждого пункта задания обведите полученный результат в торжественную рамочку.
\item Загрузите свою работу в лмс или на гитхаб в виде одного \verb|.pdf| файла.

Ссылка на гитхаб: \url{https://classroom.github.com/a/FzuyWyxQ}.
\item Имя файла должно иметь вид \verb|kr5_NNN.pdf| где вместо цифр \verb|NNN| следует написать \verb|id_for_online|.
\item При загрузке версий работы и на гитхаб, и в лмс, проверяется версия из лмс. 
\item Начало контрольной: 13:00. 

Дедлайны: 13:40 — без штрафа, 13.45 — со штрафом 30\%, 13:50 — со штрафом 60\%.
\item Оперативные важные сообщения будут в телеграм-канале \verb|@room112|.
\end{enumerate}

\vspace{10mm}

\textbf{Bushido, 武士道, samurai code of honour:}

\vspace{5mm}

\begin{enumerate}
\item State your identity at the top of the sheet: \verb|id_for_online|, first name, last name, group number.
\item Full solutions are required, answer without explanations is not graded. 
\item You should draw a pretty box around every final numeric answer or formula.
\item Upload your work to the lms or gihtub as a unique \verb|.pdf| file.

Github link: \url{https://classroom.github.com/a/FzuyWyxQ}.
\item The filename should be of the form \verb|kr5_NNN.pdf| with your \verb|id_for_online| instead of \verb|NNN|.
\item If you submit your work on github and in lms, then the version in lms is checked.
\item Start time: 13:00. Deadlines: 13:40 — without penalty, 13:45 — 30\% penalty, 13:50 — 60\% penalty.
\item For prompt and important messages read the telegram channel \verb|@room112|.
\end{enumerate}


\begin{center}
  \begin{minipage}{5cm}
    \includegraphics[height=30ex]{cheshire_cat.png}
  \end{minipage}
\end{center}
  
\newpage

Номер выполняемого варианта $n$ определяется как $n=(\id \mod 6)+1$. 

\section*{Вариант 1}


Рассмотрим случайную выборку $X_1$, $X_2$, \ldots, $X_n$ из распределения Пуассона с параметром $\lambda$. 
Чеширский Кот хочет проверить гипотезу $H_0$: $\lambda = 3$ против альтернативной $H_1$: $\lambda = 4$ на уровне
значимости $0.05$. Помогите Чеширскому Коту разобраться с построением оптимального (равномерно наиболее мощного)
критерия с помощью леммы Неймана-Пирсона при различных $n$.

\begin{enumerate}
  \item {[5+4]} Для $n=100$ найдите оптимальный критерий и вероятность ошибки второго рода. Можно использовать нормальную аппроксимацию.
  \item {[6+4+1]} Для $n=1$ найдите оптимальный критерий и вероятность ошибки второго рода. Проверьте нулевую гипотезу для $x_1 = 5$. 
  \item {[бонусный пункт, 6+4]} Для $n=3$ найдите оптимальный критерий и вероятность ошибки второго рода.
\end{enumerate}

\vspace{15mm}


\textbf{Буси-до, 武士道, кодекс чести самурая:}

\vspace{5mm}

\begin{enumerate}
\item Подпишите работу сверху: \verb|id_for_online|, фамилию, имя, номер группы.
\item Должно быть выписано решение задачи, только ответ не засчитывается.
\item Для каждого пункта задания обведите полученный результат в торжественную рамочку.
\item Загрузите свою работу в лмс или на гитхаб в виде одного \verb|.pdf| файла.

Ссылка на гитхаб: \url{https://classroom.github.com/a/FzuyWyxQ}.
\item Имя файла должно иметь вид \verb|kr5_NNN.pdf| где вместо цифр \verb|NNN| следует написать \verb|id_for_online|.
\item При загрузке версий работы и на гитхаб, и в лмс, проверяется версия из лмс. 
\item Начало контрольной: 13:00. 

Дедлайны: 13:40 — без штрафа, 13.45 — со штрафом 30\%, 13:50 — со штрафом 60\%.
\item Оперативные важные сообщения будут в телеграм-канале \verb|@room112|.
\end{enumerate}


\newpage

Номер выполняемого варианта $n$ определяется как $n=(\id \mod 6)+1$. 

\section*{Вариант 2}


Рассмотрим случайную выборку $X_1$, $X_2$, \ldots, $X_n$ из распределения Пуассона с параметром $\lambda$. 
Чеширский Кот хочет проверить гипотезу $H_0$: $\lambda = 4$ против альтернативной $H_1$: $\lambda = 3$ на уровне
значимости $0.05$. Помогите Чеширскому Коту разобраться с построением оптимального (равномерно наиболее мощного)
критерия с помощью леммы Неймана-Пирсона при различных $n$.

\begin{enumerate}
  \item {[5+4]} Для $n=100$ найдите оптимальный критерий и вероятность ошибки второго рода. Можно использовать нормальную аппроксимацию.
  \item {[6+4+1]} Для $n=1$ найдите оптимальный критерий и вероятность ошибки второго рода. Проверьте нулевую гипотезу для $x_1 = 4$. 
  \item {[бонусный пункт, 6+4]} Для $n=3$ найдите оптимальный критерий и вероятность ошибки второго рода.
\end{enumerate}

\vspace{15mm}


\textbf{Буси-до, 武士道, кодекс чести самурая:}

\vspace{5mm}

\begin{enumerate}
\item Подпишите работу сверху: \verb|id_for_online|, фамилию, имя, номер группы.
\item Должно быть выписано решение задачи, только ответ не засчитывается.
\item Для каждого пункта задания обведите полученный результат в торжественную рамочку.
\item Загрузите свою работу в лмс или на гитхаб в виде одного \verb|.pdf| файла.

Ссылка на гитхаб: \url{https://classroom.github.com/a/FzuyWyxQ}.
\item Имя файла должно иметь вид \verb|kr5_NNN.pdf| где вместо цифр \verb|NNN| следует написать \verb|id_for_online|.
\item При загрузке версий работы и на гитхаб, и в лмс, проверяется версия из лмс. 
\item Начало контрольной: 13:00. 

Дедлайны: 13:40 — без штрафа, 13.45 — со штрафом 30\%, 13:50 — со штрафом 60\%.
\item Оперативные важные сообщения будут в телеграм-канале \verb|@room112|.
\end{enumerate}


\newpage

Номер выполняемого варианта $n$ определяется как $n=(\id \mod 6)+1$. 

\section*{Вариант 3}


Рассмотрим случайную выборку $X_1$, $X_2$, \ldots, $X_n$ из распределения Пуассона с параметром $\lambda$. 
Чеширский Кот хочет проверить гипотезу $H_0$: $\lambda = 5$ против альтернативной $H_1$: $\lambda = 4$ на уровне
значимости $0.05$. Помогите Чеширскому Коту разобраться с построением оптимального (равномерно наиболее мощного)
критерия с помощью леммы Неймана-Пирсона при различных $n$.

\begin{enumerate}
  \item {[5+4]} Для $n=100$ найдите оптимальный критерий и вероятность ошибки второго рода. Можно использовать нормальную аппроксимацию.
  \item {[6+4+1]} Для $n=1$ найдите оптимальный критерий и вероятность ошибки второго рода. Проверьте нулевую гипотезу для $x_1 = 3$. 
  \item {[бонусный пункт, 6+4]} Для $n=3$ найдите оптимальный критерий и вероятность ошибки второго рода.
\end{enumerate}

\vspace{15mm}


\textbf{Буси-до, 武士道, кодекс чести самурая:}

\vspace{5mm}

\begin{enumerate}
\item Подпишите работу сверху: \verb|id_for_online|, фамилию, имя, номер группы.
\item Должно быть выписано решение задачи, только ответ не засчитывается.
\item Для каждого пункта задания обведите полученный результат в торжественную рамочку.
\item Загрузите свою работу в лмс или на гитхаб в виде одного \verb|.pdf| файла.

Ссылка на гитхаб: \url{https://classroom.github.com/a/FzuyWyxQ}.
\item Имя файла должно иметь вид \verb|kr5_NNN.pdf| где вместо цифр \verb|NNN| следует написать \verb|id_for_online|.
\item При загрузке версий работы и на гитхаб, и в лмс, проверяется версия из лмс. 
\item Начало контрольной: 13:00. 

Дедлайны: 13:40 — без штрафа, 13.45 — со штрафом 30\%, 13:50 — со штрафом 60\%.
\item Оперативные важные сообщения будут в телеграм-канале \verb|@room112|.
\end{enumerate}


\newpage

Номер выполняемого варианта $n$ определяется как $n=(\id \mod 6)+1$. 

\section*{Вариант 4}


Рассмотрим случайную выборку $X_1$, $X_2$, \ldots, $X_n$ из распределения Пуассона с параметром $\lambda$. 
Чеширский Кот хочет проверить гипотезу $H_0$: $\lambda = 4$ против альтернативной $H_1$: $\lambda = 5$ на уровне
значимости $0.05$. Помогите Чеширскому Коту разобраться с построением оптимального (равномерно наиболее мощного)
критерия с помощью леммы Неймана-Пирсона при различных $n$.

\begin{enumerate}
  \item {[5+4]} Для $n=100$ найдите оптимальный критерий и вероятность ошибки второго рода. Можно использовать нормальную аппроксимацию.
  \item {[6+4+1]} Для $n=1$ найдите оптимальный критерий и вероятность ошибки второго рода. Проверьте нулевую гипотезу для $x_1 = 5$. 
  \item {[бонусный пункт, 6+4]} Для $n=3$ найдите оптимальный критерий и вероятность ошибки второго рода.
\end{enumerate}

\vspace{15mm}


\textbf{Буси-до, 武士道, кодекс чести самурая:}

\vspace{5mm}

\begin{enumerate}
\item Подпишите работу сверху: \verb|id_for_online|, фамилию, имя, номер группы.
\item Должно быть выписано решение задачи, только ответ не засчитывается.
\item Для каждого пункта задания обведите полученный результат в торжественную рамочку.
\item Загрузите свою работу в лмс или на гитхаб в виде одного \verb|.pdf| файла.

Ссылка на гитхаб: \url{https://classroom.github.com/a/FzuyWyxQ}.
\item Имя файла должно иметь вид \verb|kr5_NNN.pdf| где вместо цифр \verb|NNN| следует написать \verb|id_for_online|.
\item При загрузке версий работы и на гитхаб, и в лмс, проверяется версия из лмс. 
\item Начало контрольной: 13:00. 

Дедлайны: 13:40 — без штрафа, 13.45 — со штрафом 30\%, 13:50 — со штрафом 60\%.
\item Оперативные важные сообщения будут в телеграм-канале \verb|@room112|.
\end{enumerate}


\newpage

Номер выполняемого варианта $n$ определяется как $n=(\id \mod 6)+1$. 

\section*{Вариант 5}


Рассмотрим случайную выборку $X_1$, $X_2$, \ldots, $X_n$ из распределения Пуассона с параметром $\lambda$. 
Чеширский Кот хочет проверить гипотезу $H_0$: $\lambda = 3$ против альтернативной $H_1$: $\lambda = 5$ на уровне
значимости $0.05$. Помогите Чеширскому Коту разобраться с построением оптимального (равномерно наиболее мощного)
критерия с помощью леммы Неймана-Пирсона при различных $n$.

\begin{enumerate}
  \item {[5+4]} Для $n=100$ найдите оптимальный критерий и вероятность ошибки второго рода. Можно использовать нормальную аппроксимацию.
  \item {[6+4+1]} Для $n=1$ найдите оптимальный критерий и вероятность ошибки второго рода. Проверьте нулевую гипотезу для $x_1 = 4$. 
  \item {[бонусный пункт, 6+4]} Для $n=3$ найдите оптимальный критерий и вероятность ошибки второго рода.
\end{enumerate}

\vspace{15mm}


\textbf{Буси-до, 武士道, кодекс чести самурая:}

\vspace{5mm}

\begin{enumerate}
\item Подпишите работу сверху: \verb|id_for_online|, фамилию, имя, номер группы.
\item Должно быть выписано решение задачи, только ответ не засчитывается.
\item Для каждого пункта задания обведите полученный результат в торжественную рамочку.
\item Загрузите свою работу в лмс или на гитхаб в виде одного \verb|.pdf| файла.

Ссылка на гитхаб: \url{https://classroom.github.com/a/FzuyWyxQ}.
\item Имя файла должно иметь вид \verb|kr5_NNN.pdf| где вместо цифр \verb|NNN| следует написать \verb|id_for_online|.
\item При загрузке версий работы и на гитхаб, и в лмс, проверяется версия из лмс. 
\item Начало контрольной: 13:00. 

Дедлайны: 13:40 — без штрафа, 13.45 — со штрафом 30\%, 13:50 — со штрафом 60\%.
\item Оперативные важные сообщения будут в телеграм-канале \verb|@room112|.
\end{enumerate}


\newpage

Номер выполняемого варианта $n$ определяется как $n=(\id \mod 6)+1$. 

\section*{Вариант 6}


Рассмотрим случайную выборку $X_1$, $X_2$, \ldots, $X_n$ из распределения Пуассона с параметром $\lambda$. 
Чеширский Кот хочет проверить гипотезу $H_0$: $\lambda = 5$ против альтернативной $H_1$: $\lambda = 3$ на уровне
значимости $0.05$. Помогите Чеширскому Коту разобраться с построением оптимального (равномерно наиболее мощного)
критерия с помощью леммы Неймана-Пирсона при различных $n$.

\begin{enumerate}
  \item {[5+4]} Для $n=100$ найдите оптимальный критерий и вероятность ошибки второго рода. Можно использовать нормальную аппроксимацию.
  \item {[6+4+1]} Для $n=1$ найдите оптимальный критерий и вероятность ошибки второго рода. Проверьте нулевую гипотезу для $x_1 = 4$. 
  \item {[бонусный пункт, 6+4]} Для $n=3$ найдите оптимальный критерий и вероятность ошибки второго рода.
\end{enumerate}

\vspace{15mm}


\textbf{Буси-до, 武士道, кодекс чести самурая:}

\vspace{5mm}

\begin{enumerate}
\item Подпишите работу сверху: \verb|id_for_online|, фамилию, имя, номер группы.
\item Должно быть выписано решение задачи, только ответ не засчитывается.
\item Для каждого пункта задания обведите полученный результат в торжественную рамочку.
\item Загрузите свою работу в лмс или на гитхаб в виде одного \verb|.pdf| файла.

Ссылка на гитхаб: \url{https://classroom.github.com/a/FzuyWyxQ}.
\item Имя файла должно иметь вид \verb|kr5_NNN.pdf| где вместо цифр \verb|NNN| следует написать \verb|id_for_online|.
\item При загрузке версий работы и на гитхаб, и в лмс, проверяется версия из лмс. 
\item Начало контрольной: 13:00. 

Дедлайны: 13:40 — без штрафа, 13.45 — со штрафом 30\%, 13:50 — со штрафом 60\%.
\item Оперативные важные сообщения будут в телеграм-канале \verb|@room112|.
\end{enumerate}


\newpage
\lhead{Probability theory and statistics-HSE}
\rhead{Minitest №5}


Only non Russian speaking students are allowed to use English version of the test.
Preliminary approval by your class teacher is required. 

\section*{English version}


Consider a random sample $X_1$, $X_2$, \ldots, $X_n$ from Poisson distribution with rate $\lambda$. 
Cheshire Cat would like to test $H_0$: $\lambda = 3$ against alternative $H_1$: $\lambda = 4$ at
$0.05$ significance level. Please, assist Cheshire Cat in designing optimal statistical test
using Neyman-Pearson lemma for various $n$. 

\begin{enumerate}
  \item {[5+4]} For $n=100$ find the optimal criterion and the probability of second type error. You can use normal approximation.
  \item {[6+4+1]} For $n=1$ find the optimal criterion and the probability of second type error. Test the null hypothesis for $x_1 = 5$. 
  \item {[bonus point, 6+4]} For $n=3$ find the optimal criterion and the probability of second type error.
\end{enumerate}
 

\vspace{15mm}


\textbf{Bushido, 武士道, samurai code of honour:}

\vspace{5mm}

\begin{enumerate}
\item State your identity at the top of the sheet: \verb|id_for_online|, first name, last name, group number.
\item Full solutions are required, answer without explanations is not graded. 
\item You should draw a pretty box around every final numeric answer or formula.
\item Upload your work to the lms or gihtub as a unique \verb|.pdf| file.

Github link: \url{https://classroom.github.com/a/FzuyWyxQ}.
\item The filename should be of the form \verb|kr5_NNN.pdf| with your \verb|id_for_online| instead of \verb|NNN|.
\item If you submit your work on github and in lms, then the version in lms is checked.
\item Start time: 13:00. Deadlines: 13:40 — without penalty, 13:45 — 30\% penalty, 13:50 — 60\% penalty.
\item For prompt and important messages read the telegram channel \verb|@room112|.
\end{enumerate}





\end{document}