\documentclass[12pt]{article}

\usepackage{tikz} % картинки в tikz
\usepackage{microtype} % свешивание пунктуации

\usepackage{array} % для столбцов фиксированной ширины

\usepackage{indentfirst} % отступ в первом параграфе

\usepackage{sectsty} % для центрирования названий частей
\allsectionsfont{\centering}

\usepackage{amsmath} % куча стандартных математических плюшек

\usepackage{comment}
\usepackage{amsfonts}

\usepackage[top=2cm, left=1cm, right=1cm, bottom=2cm]{geometry} % размер текста на странице

\usepackage{lastpage} % чтобы узнать номер последней страницы

\usepackage{enumitem} % дополнительные плюшки для списков
%  например \begin{enumerate}[resume] позволяет продолжить нумерацию в новом списке
\usepackage{caption}

\usepackage{hyperref} % гиперссылки

\usepackage{multicol} % текст в несколько столбцов


\usepackage{fancyhdr} % весёлые колонтитулы
\pagestyle{fancy}
\lhead{Statistics, HSE}
\chead{2020-06-20}
\rhead{Final exam}
\lfoot{English edition}
\cfoot{}
\rfoot{May the Force be with You}
\renewcommand{\headrulewidth}{0.4pt}
\renewcommand{\footrulewidth}{0.4pt}



\usepackage{todonotes} % для вставки в документ заметок о том, что осталось сделать
% \todo{Здесь надо коэффициенты исправить}
% \missingfigure{Здесь будет Последний день Помпеи}
% \listoftodos --- печатает все поставленные \todo'шки


% более красивые таблицы
\usepackage{booktabs}
% заповеди из докупентации:
% 1. Не используйте вертикальные линни
% 2. Не используйте двойные линии
% 3. Единицы измерения - в шапку таблицы
% 4. Не сокращайте .1 вместо 0.1
% 5. Повторяющееся значение повторяйте, а не говорите "то же"


\usepackage{fontspec}
\usepackage{polyglossia}

\setmainlanguage{english}
\setotherlanguages{english}

% download "Linux Libertine" fonts:
% http://www.linuxlibertine.org/index.php?id=91&L=1
\setmainfont{Linux Libertine O} % or Helvetica, Arial, Cambria
% why do we need \newfontfamily:
% http://tex.stackexchange.com/questions/91507/
\newfontfamily{\cyrillicfonttt}{Linux Libertine O}

% \AddEnumerateCounter{\asbuk}{\russian@alph}{щ} % для списков с русскими буквами
% \setlist[enumerate, 2]{label=\asbuk*),ref=\asbuk*}

%% эконометрические сокращения
\DeclareMathOperator{\Cov}{Cov}
\DeclareMathOperator{\Corr}{Corr}
\DeclareMathOperator{\Var}{Var}
\DeclareMathOperator{\E}{E}
\newcommand \hb{\hat{\beta}}
\newcommand \hs{\hat{\sigma}}
\newcommand \htheta{\hat{\theta}}
\newcommand \s{\sigma}
\newcommand \hy{\hat{y}}
\newcommand \hY{\hat{Y}}
\newcommand \e{\varepsilon}
\newcommand \he{\hat{\e}}
\newcommand \z{z}
\newcommand \hVar{\widehat{\Var}}
\newcommand \hCorr{\widehat{\Corr}}
\newcommand \hCov{\widehat{\Cov}}
\newcommand \cN{\mathcal{N}}
\let\P\relax
\newcommand \P{\mathbb{P}}


\begin{document}


\fbox{
  \begin{minipage}{42em}
    Name, group:\vspace*{3ex}\par
    \noindent\dotfill\vspace{2mm}
  \end{minipage}
}


\begin{enumerate}

\item Consider the random sample $X_1$, $X_2$, \ldots, $X_n$ from the distribution with the density
\[
f(x) =  (a+1)x^a, \quad x\in [0;1].
\]

\begin{enumerate}
  \item Find the estimator of $a$ using maximum likelihood;
  \item Find the estimator of $a$ using method of moments.
\end{enumerate}

\item You found a strange coin. You toss it 200 times and it lands 120 times head up and 80 times tail up. 

Test the hypothesis that head and tail have the same probability using significance level $\alpha = 0.05$:
\begin{enumerate}
    \item using Pearson's test;
    \item using likelihood ratio.
\end{enumerate}
  

\item 
Maria measured her weight twice, before and after drinking 100g of smoothie. 
But the measures differ by 200g. The problem is with scales. Each measurement is the true weight plus 
random error with expected value zero. 

Provide an unbiased estimate of weighing error variance.  


\item You estimate the probability of some event using $1000$ observation. 
Find the maximal possible length of 95\% confidence interval for the unknown probability. 

\item You have two indepent observations $X_1$ and $X_2$. 
The null hypothesis states that the distribution is uniform on $[0;1]$. 
The alternative hypothesis states that both random variables have the density $f(x)=2x$ on $[0;1]$.

Using Neyman-Person lemma construct the most powerful test for the probability of first type error $\alpha = 0.05$.



\item You found a strange coin. You toss it $n$ times to test the hypothesis that head and tail have the same probability.
You use two statistics, $LR$ — likelihood ratio and $PT$ — Pearson's test. Under the null-hypothesis find 
probability limit of $LR/PT$ as $n\to \infty$.



\end{enumerate}

\end{document}
