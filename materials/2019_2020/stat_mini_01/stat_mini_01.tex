\documentclass[12pt]{article}

\usepackage{tikz} % картинки в tikz
\usepackage{microtype} % свешивание пунктуации

\usepackage{array} % для столбцов фиксированной ширины

\usepackage{indentfirst} % отступ в первом параграфе

\usepackage{sectsty} % для центрирования названий частей
\allsectionsfont{\centering}

\usepackage{amsmath} % куча стандартных математических плюшек

\usepackage{comment}
\usepackage{amsfonts}

\usepackage{verbatim}


\usepackage[colorlinks=true, linkcolor=blue]{hyperref}

\usepackage[top=2cm, left=1cm, right=1cm, bottom=2cm]{geometry} % размер текста на странице

\usepackage{lastpage} % чтобы узнать номер последней страницы

\usepackage{enumitem} % дополнительные плюшки для списков
%  например \begin{enumerate}[resume] позволяет продолжить нумерацию в новом списке
\usepackage{caption}

\usepackage{hyperref} % гиперссылки

\usepackage{multicol} % текст в несколько столбцов


\usepackage{fancyhdr} % весёлые колонтитулы
\pagestyle{fancy}
\lhead{Теория вероятностей и математическая статистика-ВШЭ}
\chead{}
\rhead{Миниконтрольная №1.}
\lfoot{2020-04-09}
\cfoot{}
\rfoot{}
\renewcommand{\headrulewidth}{0.4pt}
\renewcommand{\footrulewidth}{0.4pt}



\usepackage{todonotes} % для вставки в документ заметок о том, что осталось сделать
% \todo{Здесь надо коэффициенты исправить}
% \missingfigure{Здесь будет Последний день Помпеи}
% \listoftodos --- печатает все поставленные \todo'шки


% более красивые таблицы
\usepackage{booktabs}
% заповеди из докупентации:
% 1. Не используйте вертикальные линни
% 2. Не используйте двойные линии
% 3. Единицы измерения - в шапку таблицы
% 4. Не сокращайте .1 вместо 0.1
% 5. Повторяющееся значение повторяйте, а не говорите "то же"


\usepackage{fontspec}
\usepackage{polyglossia}

\setmainlanguage{russian}
\setotherlanguages{english}

% download "Linux Libertine" fonts:
% http://www.linuxlibertine.org/index.php?id=91&L=1
\setmainfont{Linux Libertine O} % or Helvetica, Arial, Cambria
% why do we need \newfontfamily:
% http://tex.stackexchange.com/questions/91507/
\newfontfamily{\cyrillicfonttt}{Linux Libertine O}

\AddEnumerateCounter{\asbuk}{\russian@alph}{щ} % для списков с русскими буквами
\setlist[enumerate, 2]{label=\asbuk*),ref=\asbuk*}

%% эконометрические сокращения
\DeclareMathOperator{\Cov}{Cov}
\DeclareMathOperator{\Corr}{Corr}
\DeclareMathOperator{\Var}{Var}
\DeclareMathOperator{\E}{E}
\def \hb{\hat{\beta}}
\def \hs{\hat{\sigma}}
\def \htheta{\hat{\theta}}
\def \s{\sigma}
\def \hy{\hat{y}}
\def \hY{\hat{Y}}
\def \v1{\vec{1}}
\def \e{\varepsilon}
\def \he{\hat{\e}}
\def \z{z}
\def \hVar{\widehat{\Var}}
\def \hCorr{\widehat{\Corr}}
\def \hCov{\widehat{\Cov}}
\def \cN{\mathcal{N}}
\def \P{\mathbb{P}}
\def \id {\mathrm{id}\_\mathrm{for}\_\mathrm{online}}

\begin{document}


\section*{Требования к оформлению:}

\begin{enumerate}
\item Подпишите работу сверху: \verb|id_for_online|, фамилию, имя, номер группы.
\item Должно быть выписано решение задачи, только ответ не засчитывается.
\item Для каждого пункта задания обведите полученный численный ответ или формулу в торжественную рамочку.
\end{enumerate}

\section*{Индивидуальные параметры:}

\begin{enumerate}
\item Исходя из своего индентификатора вычислите три константы: 
\begin{align*}
a = (\id \mod 2) + 1 \\
b = (\id \mod 3) + 1 \\
c = (\id \mod 4) + 1 \\
\end{align*}

Запись $\mod$ означает остаток от деления, например $16 \mod 3 = 1$. 

\item Номер вашего варианта равен константе $a$.
\end{enumerate}



\section*{Requirements:}

\begin{enumerate}
\item State your identity at the top of the sheet: \verb|id_for_online|, first name, last name, group number.
\item Full solutions are required, answer without explanations is not graded. 
\item You should draw a pretty box around every final numeric answer or formula.
\end{enumerate}

\section*{Individual parameters:}

\begin{enumerate}
\item Using your identification number calculate three constants:
\begin{align*}
  a = (\id \mod 2) + 1 \\
  b = (\id \mod 3) + 1 \\
  c = (\id \mod 4) + 1 \\
\end{align*}

Here $\mod$ denotes division remainder, i.e. $16 \mod 3 = 1$.

\item Your variant number is equal to the constant $a$.
\end{enumerate}


\newpage

\section*{Решите вариант 1, если $a=1$:}

Случайные величины $\xi$ и $\eta$ имеют совместное нормальное распределение:

\[
\begin{pmatrix}
  \xi \\
  \eta \\
\end{pmatrix}   \sim 
\cN \left(
\begin{pmatrix}
a \\
b \\
\end{pmatrix};
\begin{pmatrix}
  c & d \\
  d & 1 \\
\end{pmatrix}
\right).
\]

Корреляция между $\xi$ и $\eta$ равна $\Corr(\xi, \eta)= 0.5$.


\begin{enumerate}
\item Найдите вероятность $\P(\xi > 1 + a)$.
\item Выпишите функцию плотности $\xi$.
\item Найдите константу $d$. 
\item Вычислите $\P(2\xi -3\eta >10)$.
\item Выпишите функцию плотности случайной величины $2\xi -3\eta$.
\item Вычислите $\E(\xi |\eta = a)$.
\item Вычислите $\Var(\xi |\eta = a)$.
\item Вычислите $\P(\xi > 1+ a |\eta = a)$.
\end{enumerate}

\section*{Решите вариант 2, если $a=2$:}

Случайные величины $\xi$ и $\eta$ имеют совместное нормальное распределение:

\[
\begin{pmatrix}
  \xi \\
  \eta \\
\end{pmatrix}   \sim 
\cN \left(
\begin{pmatrix}
a \\
b \\
\end{pmatrix};
\begin{pmatrix}
  c & d \\
  d & 1 \\
\end{pmatrix}
\right).
\]

Корреляция между $\xi$ и $\eta$ равна $\Corr(\xi, \eta)= -0.5$.


\begin{enumerate}
\item Найдите вероятность $\P(\xi > b - 1)$.
\item Выпишите функцию плотности $\eta$.
\item Найдите константу $d$. 
\item Вычислите $\P(3\xi +2 \eta <10)$.
\item Выпишите функцию плотности случайной величины $3\xi +2\eta$.
\item Вычислите $\E(\eta |\xi = b)$.
\item Вычислите $\Var(\eta |\xi = b)$.
\item Вычислите $\P(\eta > b- 1 |\xi = b)$.
\end{enumerate}


\newpage

\section*{Solve variant 1, if $a=1$:}

Random variables $\xi$ and $\eta$ have joint normal distribution:

\[
\begin{pmatrix}
  \xi \\
  \eta \\
\end{pmatrix}   \sim 
\cN \left(
\begin{pmatrix}
a \\
b \\
\end{pmatrix};
\begin{pmatrix}
  c & d \\
  d & 1 \\
\end{pmatrix}
\right).
\]

Correlation between $\xi$ and $\eta$ is equal to $\Corr(\xi, \eta)= 0.5$.


\begin{enumerate}
\item Find the probability $\P(\xi > 1 + a)$.
\item Write down the density function of $\xi$.
\item Find the constant $d$. 
\item Calculate $\P(2\xi -3\eta >10)$.
\item Write down the density function of the random variable $2\xi -3\eta$.
\item Calculate $\E(\xi |\eta = a)$.
\item Calculate $\Var(\xi |\eta = a)$.
\item Calculate $\P(\xi > 1+ a |\eta = a)$.
\end{enumerate}

\section*{Solve variant 2, if $a=2$:}

Random variables $\xi$ and $\eta$ have joint normal distribution:

\[
\begin{pmatrix}
  \xi \\
  \eta \\
\end{pmatrix}   \sim 
\cN \left(
\begin{pmatrix}
a \\
b \\
\end{pmatrix};
\begin{pmatrix}
  c & d \\
  d & 1 \\
\end{pmatrix}
\right).
\]

Correlation between $\xi$ and $\eta$ is equal to $\Corr(\xi, \eta)= -0.5$.


\begin{enumerate}
\item Find the probability $\P(\xi > b - 1)$.
\item Write down the density function of $\eta$.
\item Find the constant $d$. 
\item Calculate $\P(3\xi +2 \eta <10)$.
\item Write down the density function of the random variable  $3\xi +2\eta$.
\item Calculate $\E(\eta |\xi = b)$.
\item Calculate $\Var(\eta |\xi = b)$.
\item Calculate $\P(\eta > b- 1 |\xi = b)$.
\end{enumerate}







\end{document}
