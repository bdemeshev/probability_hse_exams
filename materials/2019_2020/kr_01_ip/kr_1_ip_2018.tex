\documentclass[12pt]{article}

\usepackage{tikz} % картинки в tikz
\usepackage{microtype} % свешивание пунктуации

\usepackage{array} % для столбцов фиксированной ширины

\usepackage{indentfirst} % отступ в первом параграфе

\usepackage{sectsty} % для центрирования названий частей
\allsectionsfont{\centering}

\usepackage{amsmath} % куча стандартных математических плюшек

\usepackage{comment}
\usepackage{amsfonts}

\usepackage[top=2cm, left=1cm, right=1cm, bottom=2cm]{geometry} % размер текста на странице

\usepackage{lastpage} % чтобы узнать номер последней страницы

\usepackage{enumitem} % дополнительные плюшки для списков
%  например \begin{enumerate}[resume] позволяет продолжить нумерацию в новом списке
\usepackage{caption}

\usepackage{hyperref} % гиперссылки

\usepackage{multicol} % текст в несколько столбцов


\usepackage{fancyhdr} % весёлые колонтитулы
\pagestyle{fancy}
\lhead{Теория вероятностей-ВШЭ}
\chead{2019-10-23}
\rhead{Контрольная 1. Максимум :)}
\lfoot{}
\cfoot{}
\rfoot{}
\renewcommand{\headrulewidth}{0.4pt}
\renewcommand{\footrulewidth}{0.4pt}



\usepackage{todonotes} % для вставки в документ заметок о том, что осталось сделать
% \todo{Здесь надо коэффициенты исправить}
% \missingfigure{Здесь будет Последний день Помпеи}
% \listoftodos --- печатает все поставленные \todo'шки


% более красивые таблицы
\usepackage{booktabs}
% заповеди из докупентации:
% 1. Не используйте вертикальные линни
% 2. Не используйте двойные линии
% 3. Единицы измерения - в шапку таблицы
% 4. Не сокращайте .1 вместо 0.1
% 5. Повторяющееся значение повторяйте, а не говорите "то же"


\usepackage{fontspec}
\usepackage{polyglossia}

\setmainlanguage{russian}
\setotherlanguages{english}

% download "Linux Libertine" fonts:
% http://www.linuxlibertine.org/index.php?id=91&L=1
\setmainfont{Linux Libertine O} % or Helvetica, Arial, Cambria
% why do we need \newfontfamily:
% http://tex.stackexchange.com/questions/91507/
\newfontfamily{\cyrillicfonttt}{Linux Libertine O}

\AddEnumerateCounter{\asbuk}{\russian@alph}{щ} % для списков с русскими буквами
\setlist[enumerate, 2]{label=\asbuk*),ref=\asbuk*}

%% эконометрические сокращения
\DeclareMathOperator{\Cov}{Cov}
\DeclareMathOperator{\Corr}{Corr}
\DeclareMathOperator{\Var}{Var}
\DeclareMathOperator{\E}{E}
\def \hb{\hat{\beta}}
\def \hs{\hat{\sigma}}
\def \htheta{\hat{\theta}}
\def \s{\sigma}
\def \hy{\hat{y}}
\def \hY{\hat{Y}}
\def \v1{\vec{1}}
\def \e{\varepsilon}
\def \he{\hat{\e}}
\def \z{z}
\def \hVar{\widehat{\Var}}
\def \hCorr{\widehat{\Corr}}
\def \hCov{\widehat{\Cov}}
\def \cN{\mathcal{N}}
\def \P{\mathbb{P}}


\begin{document}


\fbox{
  \begin{minipage}{42em}
    Имя, фамилия и номер группы:\vspace*{3ex}\par
    \noindent\dotfill\vspace{2mm}
  \end{minipage}
}


\begin{enumerate}
\item Илон Маск подбрасывает правильную монетку до тех пор, пока не выпадет последовательность
РОР. В процессе игры за каждого выброшенного «орла» Илон получает 100 тысяч долларов от компании Tesla.

\begin{enumerate}
  \item Сколько в среднем денег зарабатывает Илон за одну игру?
  \item Как изменится ответ, если за каждого «орла», идущего подряд, выигрыш растёт на 100 тысяч,
  а промелькнувшая «решка» сбрасывает выигрыш за «орлов» до 100 тысяч?

Пояснение: например, за ООРООО Илон получит $100 + 200 + 100 + 200 + 300$ тысяч долларов.
\end{enumerate}


\item Попугай Кеша подбрасывает кубик до первой шестёрки. 

\begin{enumerate}
  \item Сколько в среднем подбрасываний ему потребуется?
  \item Сколько в среднем подбрасываний ему потребовалось, если известно, что
  пятерок в процессе игры выпало ровно две?
  \item Сколько в среднем подбрасываний ему потребовалось, если известно, что
  ни разу не выпадала нечётная грань?
\end{enumerate}

\item Илон Маск и Грета Тунберг оказались на Луне. 
Судьба забросила их в случайные независимые точки, равномерно распределённые по поверхности Луны. 

Илон Маск тут же бросился рыть совершенно прямой туннель к
Грете. Диаметр Луны будем считать равным единице.

Найдите функцию плотности длины туннеля $L$ для случаев:

\begin{enumerate}
  \item Луна — плоская;
  \item Луна — объёмная.
\end{enumerate}


\item У Джеки Чана в коробке лежат бумажки с числами от 1 до 100. 
Джеки Чан случайно извлекает 30 этих бумажек, без возвращения бумажек в коробку.
Затем он считает среднее арифметическое извлечённых чисел, и получает случайную величину $Y$.

Найдите $\E(Y)$, $\Var(Y)$.

\item Преподаватель Николай Арефьеф любит короткие решения нетривиальных
задач. 
Однажды он пообещал студентам, что включит в контрольную
работу задачу на новую тему с вероятностью $1/3$.

У Николая есть только правильная монетка.

\begin{enumerate}
\item Какой эксперимент с монеткой позволяет выполнить данное обещание?
\item Какова минимальная ожидаемая продолжительность подобного эксперимента?
\end{enumerate}

Перед контрольной Николай подбросил монетку два раза и на основании этих
бросков принял решение о том, включить ли задачу в контрольную.

\begin{enumerate}[resume]
  \item Попала ли задача в контрольную работу?
\end{enumerate}


\end{enumerate}

\end{document}
