\documentclass[12pt]{article}

\usepackage{tikz} % картинки в tikz
\usepackage{microtype} % свешивание пунктуации

\usepackage{array} % для столбцов фиксированной ширины

\usepackage{indentfirst} % отступ в первом параграфе

\usepackage{sectsty} % для центрирования названий частей
\allsectionsfont{\centering}

\usepackage{amsmath} % куча стандартных математических плюшек

\usepackage{comment}
\usepackage{amsfonts}

\usepackage{verbatim}

\usepackage{graphicx} % for png images

\usepackage[colorlinks=true, linkcolor=blue]{hyperref}

\usepackage[top=2cm, left=1cm, right=1cm, bottom=2cm]{geometry} % размер текста на странице

\usepackage{lastpage} % чтобы узнать номер последней страницы

\usepackage{enumitem} % дополнительные плюшки для списков
%  например \begin{enumerate}[resume] позволяет продолжить нумерацию в новом списке
\usepackage{caption}

\usepackage{hyperref} % гиперссылки

\usepackage{multicol} % текст в несколько столбцов


\usepackage{fancyhdr} % весёлые колонтитулы
\pagestyle{fancy}
\lhead{Теория вероятностей и математическая статистика-ВШЭ}
\chead{}
\rhead{Миниконтрольная №2}
\lfoot{2020-04-16}
\cfoot{}
\rfoot{}
\renewcommand{\headrulewidth}{0.4pt}
\renewcommand{\footrulewidth}{0.4pt}



\usepackage{todonotes} % для вставки в документ заметок о том, что осталось сделать
% \todo{Здесь надо коэффициенты исправить}
% \missingfigure{Здесь будет Последний день Помпеи}
% \listoftodos --- печатает все поставленные \todo'шки


% более красивые таблицы
\usepackage{booktabs}
% заповеди из докупентации:
% 1. Не используйте вертикальные линни
% 2. Не используйте двойные линии
% 3. Единицы измерения - в шапку таблицы
% 4. Не сокращайте .1 вместо 0.1
% 5. Повторяющееся значение повторяйте, а не говорите "то же"


\usepackage{fontspec}
\usepackage{polyglossia}

\setmainlanguage{russian}
\setotherlanguages{english}

% download "Linux Libertine" fonts:
% http://www.linuxlibertine.org/index.php?id=91&L=1
\setmainfont{Linux Libertine O} % or Helvetica, Arial, Cambria
% why do we need \newfontfamily:
% http://tex.stackexchange.com/questions/91507/
\newfontfamily{\cyrillicfonttt}{Linux Libertine O}

\AddEnumerateCounter{\asbuk}{\russian@alph}{щ} % для списков с русскими буквами
\setlist[enumerate, 2]{label=\asbuk*),ref=\asbuk*}

%% эконометрические сокращения
\DeclareMathOperator{\Cov}{Cov}
\DeclareMathOperator{\Corr}{Corr}
\DeclareMathOperator{\Var}{Var}
\DeclareMathOperator{\E}{E}
\def \hb{\hat{\beta}}
\def \hs{\hat{\sigma}}
\def \htheta{\hat{\theta}}
\def \s{\sigma}
\def \hy{\hat{y}}
\def \hY{\hat{Y}}
\def \v1{\vec{1}}
\def \e{\varepsilon}
\def \he{\hat{\e}}
\def \z{z}
\def \hVar{\widehat{\Var}}
\def \hCorr{\widehat{\Corr}}
\def \hCov{\widehat{\Cov}}
\def \cN{\mathcal{N}}
\def \P{\mathbb{P}}
\def \id {\mathrm{id}\_\mathrm{for}\_\mathrm{online}}

\begin{document}


\section*{Требования к оформлению:}

\begin{enumerate}
\item Подпишите работу сверху: \verb|id_for_online|, фамилию, имя, номер группы.
\item Должно быть выписано решение задачи, только ответ не засчитывается.
\item Для каждого пункта задания обведите полученный численный ответ или формулу в торжественную рамочку.
\end{enumerate}

\section*{Requirements:}

\begin{enumerate}
\item State your identity at the top of the sheet: \verb|id_for_online|, first name, last name, group number.
\item Full solutions are required, answer without explanations is not graded. 
\item You should draw a pretty box around every final numeric answer or formula.
\end{enumerate}



\newpage

Каждый студент в качестве значения $k$ выбирает свой \verb|id_for_online|.


\section*{Задача:}

Имеется случайная выборка $X_1$, \ldots, $X_n$ из распределения с функцией плотности
\[
f(x) = \begin{cases}
  \lambda e^{-\lambda(x-k)}, \text{ если } x > k; \\
  0, \text{ иначе.}  
\end{cases}
\]
  
\begin{enumerate}
\item Методом моментов, используя первый начальный момент, найдите оценку параметра $\lambda$.
\item Методом максимального правдоподобия найдите:
\begin{enumerate}
\item оценку параметра $\lambda$;
\item оценку вероятности $\P(X_1 > k + 1)$.
\end{enumerate}
\item Вычислите информацию Фишера о параметре $\lambda$, содержащуюся во всей выборке.
\item Вычислите асимптотическую дисперсию оценки максимального правдоподобия $\hat\lambda$.
\item Вычислите асимптотическую дисперсию оценки максимального правдоподобия $\hat\P(X_1 > k+1)$.
\item Найдите оценку максимального правдоподобия асимптотической дисперсии оценки максимального правдоподобия  
$\hat\P(X_1 > k+1)$.
\end{enumerate}
\begin{center}
\begin{minipage}{3cm}
  \includegraphics[height=20ex]{cheshire_cat.png}
\end{minipage}
\end{center}
  
\newpage
You should use your \verb|id_for_online| as the value of constant $k$.

\section*{Problem:}

We have a random sample $X_1$, \ldots, $X_n$ from a distribution with the density function
\[
f(x) = \begin{cases}
  \lambda e^{-\lambda(x-k)}, \text{ if } x > k; \\
  0, \text{ otherwise.}  
\end{cases}
\]
  
\begin{enumerate}
\item Estimate the parameter $\lambda$ using method of moments with first moment.
\item Using maximum likelihood estimate:
\begin{enumerate}
\item the parameter $\lambda$;
\item the probability $\P(X_1 > k + 1)$.
\end{enumerate}
\item Finde the theoretical Fisher information on $\lambda$ in the whole sample.
\item Find the asymptotic variance of maximum likelihood estimator $\hat\lambda$.
\item Find the asymptotic variance of maximum likelihood estimator $\hat\P(X_1 > k+1)$.
\item Find the maximum likelihood estimator for asymptotic variance of maximum likelihood estimator   
$\hat\P(X_1 > k+1)$.
\end{enumerate}
\begin{center}
\begin{minipage}{3cm}
  \includegraphics[height=20ex]{cheshire_cat.png}
\end{minipage}
\end{center}



\end{document}
