\documentclass[12pt]{article}

\usepackage{tikz} % картинки в tikz
\usepackage{microtype} % свешивание пунктуации

\usepackage{array} % для столбцов фиксированной ширины

\usepackage{indentfirst} % отступ в первом параграфе

\usepackage{sectsty} % для центрирования названий частей
\allsectionsfont{\centering}

\usepackage{amsmath} % куча стандартных математических плюшек

\usepackage{comment}
\usepackage{amsfonts}

\usepackage{verbatim}

\usepackage{graphicx} % for png images

\usepackage[colorlinks=true, linkcolor=blue]{hyperref}

\usepackage[top=2cm, left=1cm, right=1cm, bottom=2cm]{geometry} % размер текста на странице

\usepackage{lastpage} % чтобы узнать номер последней страницы

\usepackage{enumitem} % дополнительные плюшки для списков
%  например \begin{enumerate}[resume] позволяет продолжить нумерацию в новом списке
\usepackage{caption}

\usepackage{hyperref} % гиперссылки

\usepackage{multicol} % текст в несколько столбцов


\usepackage{fancyhdr} % весёлые колонтитулы
\pagestyle{fancy}
\lhead{Теория вероятностей и математическая статистика-ВШЭ}
\chead{}
\rhead{Миниконтрольная №3}
\lfoot{2020-04-23}
\cfoot{}
\rfoot{}
\renewcommand{\headrulewidth}{0.4pt}
\renewcommand{\footrulewidth}{0.4pt}



\usepackage{todonotes} % для вставки в документ заметок о том, что осталось сделать
% \todo{Здесь надо коэффициенты исправить}
% \missingfigure{Здесь будет Последний день Помпеи}
% \listoftodos --- печатает все поставленные \todo'шки


% более красивые таблицы
\usepackage{booktabs}
% заповеди из докупентации:
% 1. Не используйте вертикальные линни
% 2. Не используйте двойные линии
% 3. Единицы измерения - в шапку таблицы
% 4. Не сокращайте .1 вместо 0.1
% 5. Повторяющееся значение повторяйте, а не говорите "то же"


\usepackage{fontspec}
\usepackage{polyglossia}

\setmainlanguage{russian}
\setotherlanguages{english}

% download "Linux Libertine" fonts:
% http://www.linuxlibertine.org/index.php?id=91&L=1
\setmainfont{Linux Libertine O} % or Helvetica, Arial, Cambria
% why do we need \newfontfamily:
% http://tex.stackexchange.com/questions/91507/
\newfontfamily{\cyrillicfonttt}{Linux Libertine O}

\AddEnumerateCounter{\asbuk}{\russian@alph}{щ} % для списков с русскими буквами
\setlist[enumerate, 2]{label=\asbuk*),ref=\asbuk*}

%% эконометрические сокращения
\DeclareMathOperator{\Cov}{Cov}
\DeclareMathOperator{\Corr}{Corr}
\DeclareMathOperator{\Var}{Var}
\DeclareMathOperator{\E}{E}
\def \hb{\hat{\beta}}
\def \hs{\hat{\sigma}}
\def \htheta{\hat{\theta}}
\def \s{\sigma}
\def \hy{\hat{y}}
\def \hY{\hat{Y}}
\def \v1{\vec{1}}
\def \e{\varepsilon}
\def \he{\hat{\e}}
\def \z{z}
\def \hVar{\widehat{\Var}}
\def \hCorr{\widehat{\Corr}}
\def \hCov{\widehat{\Cov}}
\def \cN{\mathcal{N}}
\def \P{\mathbb{P}}
\newcommand \id {\mathrm{id}\_\mathrm{for}\_\mathrm{online}}

\begin{document}


\section*{Требования к оформлению:}

\begin{enumerate}
\item Подпишите работу сверху: \verb|id_for_online|, фамилию, имя, номер группы.
\item Должно быть выписано решение задачи, только ответ не засчитывается.
\item Для каждого пункта задания обведите полученный численный ответ или формулу в торжественную рамочку.
\end{enumerate}

\section*{Requirements:}

\begin{enumerate}
\item State your identity at the top of the sheet: \verb|id_for_online|, first name, last name, group number.
\item Full solutions are required, answer without explanations is not graded. 
\item You should draw a pretty box around every final numeric answer or formula.
\end{enumerate}


\begin{center}
  \begin{minipage}{5cm}
    \includegraphics[height=30ex]{cheshire_cat.png}
  \end{minipage}
\end{center}
  


\newpage

Номер выполняемого варианта $n$ определяется как $n=(\id \mod 3)+1$. 

Во всех задачах $k=\id$.

\section*{Вариант 1}
\begin{enumerate}

\item  Пусть $X_1$, $X_2$, $X_3$, \ldots, $X_n$ — случайная выборка из дискретного распределения, 
заданного с помощью таблицы

\begin{tabular}{@{}llll@{}}
  \toprule
   $x$           & $-k$ & $0$ & $k$ \\ 
   $\P(X_i = x)$ & $0.5-\theta$ & $0.5$ & $\theta$ \\
  \bottomrule
\end{tabular}
			
Рассмотрите оценку $\hat \theta = \bar X$.
\begin{enumerate}
\item Найдите $\E(\hat \theta)$. 
Является ли оценка $\hat \theta$ несмещенной оценкой неизвестного параметра $\theta$?
\item Подберите константы $a$ и $b$ так, 
чтобы оценка $\tilde \theta = a + b\bar X$ оказалась несмещенной оценкой неизвестного параметра $\theta$.
\end{enumerate} 


\item Пусть $X_1$, $X_2$, $X_3$, \ldots, $X_n$ — 
случайная выборка из распределения с плотностью распределения
\[
f(x, \theta) = \begin{cases}
  6x(\theta - x)/\theta^3, \text{ при } x\in [0;\theta]; \\
  0, \text{ иначе}
\end{cases},
\]

где $\theta$ — неизвестный параметр распределения. 
Является ли оценка $\hat\theta_n = \frac{2n+1}{n}\bar X_n$ состоятельной оценкой неизвестного параметра $\theta$?

\item Пусть $X_1$, $X_2$, $X_3$, \ldots, $X_n$ — случайная выборка из распределения с плотностью
\[
  f(x, \theta) = \begin{cases}
    \frac{1}{\theta}\exp\left(-\frac{x-k}{\theta}\right), \text{ при } x\geq k; \\
    0, \text{ иначе}
  \end{cases},    
\]
где $\theta$ — неизвестный параметр. Проверьте, будет ли оценка $\hat\theta = \bar X - k$ эффективной?

\item  Время от начала контрольной работы до её загрузки в систему lms  подчинено закону с функцией плотности
\[
  f(x, \theta) = \begin{cases}
    \frac{1}{\theta}\exp\left(-\frac{x-40}{\theta}\right), \text{ при } x\geq 40; \\
    0, \text{ иначе}
  \end{cases},    
\]
По случайной выборке $X_1$, $X_2$, $X_3$, \ldots, $X_n$ оценивают параметр $\theta$. 

\begin{enumerate}
\item Найдите такие константы $a$ и $b$, чтобы оценка  $\hat \theta_1 = a + b\min\{X_1, \ldots, X_n\}$ была несмещённой.
\item Какая из двух несмещённых оценок $\hat \theta_1$ или $\hat \theta_2 = \bar X - 40$ является более эффективной? Обоснуйте ответ.
\end{enumerate}

\end{enumerate}

\newpage
Номер выполняемого варианта $n$ определяется как $n=(\id \mod 3)+1$. 

Во всех задачах $k=\id$.


\section*{Вариант 2}
\begin{enumerate}

\item Пусть $X_1$, $X_2$, $X_3$, \ldots, $X_n$ — 
случайная выборка из распределения с плотностью распределения
\[
f(x, \theta) = \begin{cases}
  6x(\theta - x)/\theta^3, \text{ при } x\in [0;\theta]; \\
  0, \text{ иначе}
\end{cases},
\]

где $\theta$ — неизвестный параметр распределения. 

\begin{enumerate}
\item Является ли оценка $\hat\theta=\bar X$ несмещенной оценкой неизвестного параметра $\theta$?
\item Подберите константу $c$ так, 
чтобы оценка $\tilde \theta = c\bar X$ оказалась несмещенной оценкой неизвестного параметра $\theta$. 
\end{enumerate}


\item Пусть $X_1$, $X_2$, $X_3$, \ldots, $X_n$ — случайная выборка из распределения с плотностью
\[
  f(x, \theta) = \begin{cases}
    \frac{1}{\theta}\exp\left(-\frac{x-k}{\theta}\right), \text{ при } x\geq k; \\
    0, \text{ иначе}
  \end{cases},    
\]
где $\theta$ — неизвестный параметр.

Является ли оценка $\hat\theta_n = \frac{X_1 + X_2 + \ldots + X_n}{n+1} - k$ состоятельной оценкой неизвестного параметра $\theta$?



\item  Пусть $X_1$, $X_2$, $X_3$, \ldots, $X_n$ — 
случайная выборка из распределения Бернулли с неизвестным параметром $p\in (0;1)$. 
Проверьте, будет ли эффективной оценка $\hat p = \bar X$.




\item  Время от начала контрольной работы до её загрузки в систему lms  подчинено закону с функцией плотности
\[
  f(x, \theta) = \begin{cases}
    \frac{1}{\theta}\exp\left(-\frac{x-30}{\theta}\right), \text{ при } x\geq 30; \\
    0, \text{ иначе}
  \end{cases},    
\]
По случайной выборке $X_1$, $X_2$, $X_3$, \ldots, $X_n$ оценивают параметр $\theta$. 

\begin{enumerate}
\item Найдите такие константы $a$ и $b$, чтобы оценка  $\hat \theta_1 = a + b\min\{X_1, \ldots, X_n\}$ была несмещённой.
\item Какая из двух несмещённых оценок $\hat \theta_1$ или $\hat \theta_2 = \bar X - 30$ является более эффективной? Обоснуйте ответ.
\end{enumerate}

\end{enumerate}

\newpage
Номер выполняемого варианта $n$ определяется как $n=(\id \mod 3)+1$. 

Во всех задачах $k=\id$.


\section*{Вариант 3}
\begin{enumerate}

\item  Пусть $X_1$, $X_2$, $X_3$, \ldots, $X_n$ — случайная выборка из дискретного распределения, 
заданного с помощью таблицы

\begin{tabular}{@{}llll@{}}
  \toprule
   $x$           & $-k$ & $0$ & $k$ \\ 
   $\P(X_i = x)$ & $0.5-\theta$ & $0.5$ & $\theta$ \\
  \bottomrule
\end{tabular}
			
Рассмотрите оценку $\hat \theta = \bar X$.
\begin{enumerate}
\item Найдите $\E(\hat \theta)$. 
Является ли оценка $\hat \theta$ несмещенной оценкой неизвестного параметра $\theta$?
\item Подберите константы $a$ и $b$ так, 
чтобы оценка $\tilde \theta = a + b\bar X$ оказалась несмещенной оценкой неизвестного параметра $\theta$.
\end{enumerate} 



\item Пусть $X_1$, $X_2$, $X_3$, \ldots, $X_n$ — случайная выборка из распределения с плотностью
\[
  f(x, \theta) = \begin{cases}
    \frac{1}{\theta}\exp\left(-\frac{x-k}{\theta}\right), \text{ при } x\geq k; \\
    0, \text{ иначе}
  \end{cases},    
\]
где $\theta$ — неизвестный параметр.

Является ли оценка $\hat\theta_n = \frac{X_1 + X_2 + \ldots + X_n}{n+1} - k$ состоятельной оценкой неизвестного параметра $\theta$?


\item Пусть $X_1$, $X_2$, $X_3$, \ldots, $X_n$ 
— случайная выборка из распределения Пуассона
\[
\P(X_i = m) = \exp(-\lambda)\frac{\lambda^m}{m!}, \text{ если } m \in \{0, 1, 2, \ldots\},
\]
где $\lambda$ — неизвестный параметр. 
Проверьте, будет ли эффективной оценка $\hat\theta = \bar X$?


\item  Время от начала контрольной работы до её загрузки в систему lms  подчинено закону с функцией плотности
\[
  f(x, \theta) = \begin{cases}
    \frac{1}{\theta}\exp\left(-\frac{x-20}{\theta}\right), \text{ при } x\geq 20; \\
    0, \text{ иначе}
  \end{cases},    
\]
По случайной выборке $X_1$, $X_2$, $X_3$, \ldots, $X_n$ оценивают параметр $\theta$. 

\begin{enumerate}
\item Найдите такие константы $a$ и $b$, чтобы оценка  $\hat \theta_1 = a + b\min\{X_1, \ldots, X_n\}$ была несмещённой.
\item Какая из двух несмещённых оценок $\hat \theta_1$ или $\hat \theta_2 = \bar X - 20$ является более эффективной? Обоснуйте ответ.
\end{enumerate}

\end{enumerate}




\newpage

The variant $n$ is defined as $n=(\id \mod 3)+1$. 

In all problems $k=\id$.

\section*{Variant 1}
\begin{enumerate}

\item  Let $X_1$, $X_2$, $X_3$, \ldots, $X_n$ — be a random sample from 
discrete distribution given by the table

\begin{tabular}{@{}llll@{}}
  \toprule
   $x$           & $-k$ & $0$ & $k$ \\ 
   $\P(X_i = x)$ & $0.5-\theta$ & $0.5$ & $\theta$ \\
  \bottomrule
\end{tabular}
			
Consider the estimator $\hat \theta = \bar X$.
\begin{enumerate}
  \item Find $\E(\hat \theta)$. 
  Check whether the estimator $\hat \theta$ is unbiased for the parameter $\theta$?
  \item Find $a$ and $b$ such, 
  that $\tilde \theta = a + b\bar X$ is unbiased for the parameter $\theta$.
  \end{enumerate} 
  

\item Let $X_1$, $X_2$, $X_3$, \ldots, $X_n$ 
 be a random sample from a distribution with density
\[
f(x, \theta) = \begin{cases}
  6x(\theta - x)/\theta^3, \text{ if } x\in [0;\theta]; \\
  0, \text{ otherwise}
\end{cases},
\]
where $\theta$  is the unknown paramater. 
Is the estimator $\hat\theta_n = \frac{2n+1}{n}\bar X_n$ consistent for the parameter $\theta$?

\item Let $X_1$, $X_2$, $X_3$, \ldots, $X_n$  be a random sample from a distribution with density
\[
  f(x, \theta) = \begin{cases}
    \frac{1}{\theta}\exp\left(-\frac{x-k}{\theta}\right), \text{ if } x\geq k; \\
    0, \text{ otherwise}
  \end{cases},    
\]
where $\theta$  is the unknown paramater. 
Check whether the estimator $\hat\theta = \bar X - k$ is efficient?

\item  The time between the beginning of the test and the upload to the lms system
has the density
\[
  f(x, \theta) = \begin{cases}
    \frac{1}{\theta}\exp\left(-\frac{x-40}{\theta}\right), \text{ if } x\geq 40; \\
    0, \text{ otherwise}
  \end{cases},    
\]
Given the random sample $X_1$, $X_2$, $X_3$, \ldots, $X_n$ we estimate $\theta$. 

\begin{enumerate}
\item Find the constants  $a$ and $b$ such that the estimator  $\hat \theta_1 = a + b\min\{X_1, \ldots, X_n\}$ is unbiased.
\item Which estimator $\hat \theta_1$ or $\hat \theta_2 = \bar X - 40$ is more efficient? Justify.
\end{enumerate}

\end{enumerate}

\newpage

The variant $n$ is defined as $n=(\id \mod 3)+1$. 

In all problems $k=\id$.

\section*{Variant 2}
\begin{enumerate}

\item Let $X_1$, $X_2$, $X_3$, \ldots, $X_n$ — 
be a random sample from distribution with density
\[
f(x, \theta) = \begin{cases}
  6x(\theta - x)/\theta^3, \text{ if } x\in [0;\theta]; \\
  0, \text{ otherwise}
\end{cases},
\]

where $\theta$ — is the unknown parameter. 

\begin{enumerate}
\item Check whether $\hat\theta=\bar X$ is unbiased for the parameter $\theta$?
\item Find the constant $c$ such, 
that the estimator $\tilde \theta = c\bar X$ is unbiased for the parameter $\theta$. 
\end{enumerate}


\item Let $X_1$, $X_2$, $X_3$, \ldots, $X_n$ be a random sample from distribution with density

\[
  f(x, \theta) = \begin{cases}
    \frac{1}{\theta}\exp\left(-\frac{x-k}{\theta}\right), \text{ if } x\geq k; \\
    0, \text{ otherwise}
  \end{cases},    
\]
where $\theta$ is the unknown parameter.

Check whether the estimator $\hat\theta_n = \frac{X_1 + X_2 + \ldots + X_n}{n+1} - k$ is consistent for $\theta$?



\item  Let $X_1$, $X_2$, $X_3$, \ldots, $X_n$ — 
be a random sample from Bernoulli distribution with unknown parameter $p\in (0;1)$. 
Check whether the estimator $\hat p = \bar X$ is efficient.




\item  The time between the beginning of the test and the upload to the lms system
has the density
\[
  f(x, \theta) = \begin{cases}
    \frac{1}{\theta}\exp\left(-\frac{x-30}{\theta}\right), \text{ if } x\geq 30; \\
    0, \text{ otherwise}
  \end{cases},    
\]
Given the random sample $X_1$, $X_2$, $X_3$, \ldots, $X_n$ we estimate $\theta$. 

\begin{enumerate}
\item Find the constants  $a$ and $b$ such that the estimator  $\hat \theta_1 = a + b\min\{X_1, \ldots, X_n\}$ is unbiased.
\item Which estimator $\hat \theta_1$ or $\hat \theta_2 = \bar X - 30$ is more efficient? Justify.
\end{enumerate}

\end{enumerate}

\newpage

The variant $n$ is defined as $n=(\id \mod 3)+1$. 

In all problems $k=\id$.

\section*{Variant 3}
\begin{enumerate}

\item  Let $X_1$, $X_2$, $X_3$, \ldots, $X_n$  be a random sample from 
discrete distribution given by the table

\begin{tabular}{@{}llll@{}}
  \toprule
   $x$           & $-k$ & $0$ & $k$ \\ 
   $\P(X_i = x)$ & $0.5-\theta$ & $0.5$ & $\theta$ \\
  \bottomrule
\end{tabular}
			
Consider the estimator $\hat \theta = \bar X$.
\begin{enumerate}
\item Find $\E(\hat \theta)$. 
Check whether the estimator $\hat \theta$ is unbiased for the parameter $\theta$?
\item Find $a$ and $b$ such, 
that $\tilde \theta = a + b\bar X$ is unbiased for the parameter $\theta$.
\end{enumerate} 



\item Пусть $X_1$, $X_2$, $X_3$, \ldots, $X_n$ случайная выборка из распределения с плотностью
\[
  f(x, \theta) = \begin{cases}
    \frac{1}{\theta}\exp\left(-\frac{x-k}{\theta}\right), \text{ при } x\geq k; \\
    0, \text{ иначе}
  \end{cases},    
\]
where $\theta$  is the unknown parameter. 

Is the estimator $\hat\theta_n = \frac{X_1 + X_2 + \ldots + X_n}{n+1} - k$ consistent?


\item Let $X_1$, $X_2$, $X_3$, \ldots, $X_n$ 
be a random sample from Poisson distribution
\[
\P(X_i = m) = \exp(-\lambda)\frac{\lambda^m}{m!}, \text{ если } m \in \{0, 1, 2, \ldots\},
\]
where $\lambda$  is the unknown parameter. 
Check whether the estimator $\hat\theta = \bar X$ is efficient?


\item  The time between the beginning of the test and the upload to the lms system
has the density
\[
  f(x, \theta) = \begin{cases}
    \frac{1}{\theta}\exp\left(-\frac{x-20}{\theta}\right), \text{ if } x\geq 20; \\
    0, \text{ otherwise}
  \end{cases},    
\]
Given the random sample $X_1$, $X_2$, $X_3$, \ldots, $X_n$ we estimate $\theta$. 

\begin{enumerate}
\item Find the constants  $a$ and $b$ such that the estimator  $\hat \theta_1 = a + b\min\{X_1, \ldots, X_n\}$ is unbiased.
\item Which estimator $\hat \theta_1$ or $\hat \theta_2 = \bar X - 20$ is more efficient? Justify.
\end{enumerate}

\end{enumerate}






\end{document}
