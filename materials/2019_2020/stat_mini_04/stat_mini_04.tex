\documentclass[12pt]{article}

\usepackage{tikz} % картинки в tikz
\usepackage{microtype} % свешивание пунктуации

\usepackage{array} % для столбцов фиксированной ширины

\usepackage{indentfirst} % отступ в первом параграфе

\usepackage{sectsty} % для центрирования названий частей
\allsectionsfont{\centering}

\usepackage{amsmath} % куча стандартных математических плюшек

\usepackage{comment}
\usepackage{amsfonts}

\usepackage{verbatim}

\usepackage{graphicx} % for png images

\usepackage[colorlinks=true, linkcolor=blue]{hyperref}

\usepackage[top=2cm, left=1cm, right=1cm, bottom=2cm]{geometry} % размер текста на странице

\usepackage{lastpage} % чтобы узнать номер последней страницы

\usepackage{enumitem} % дополнительные плюшки для списков
%  например \begin{enumerate}[resume] позволяет продолжить нумерацию в новом списке
\usepackage{caption}

\usepackage{hyperref} % гиперссылки

\usepackage{multicol} % текст в несколько столбцов


\usepackage{fancyhdr} % весёлые колонтитулы
\pagestyle{fancy}
\lhead{Теория вероятностей и математическая статистика-ВШЭ}
\chead{}
\rhead{Миниконтрольная №4}
\lfoot{2020-05-14}
\cfoot{}
\rfoot{}
\renewcommand{\headrulewidth}{0.4pt}
\renewcommand{\footrulewidth}{0.4pt}



\usepackage{todonotes} % для вставки в документ заметок о том, что осталось сделать
% \todo{Здесь надо коэффициенты исправить}
% \missingfigure{Здесь будет Последний день Помпеи}
% \listoftodos --- печатает все поставленные \todo'шки


% более красивые таблицы
\usepackage{booktabs}
% заповеди из докупентации:
% 1. Не используйте вертикальные линни
% 2. Не используйте двойные линии
% 3. Единицы измерения - в шапку таблицы
% 4. Не сокращайте .1 вместо 0.1
% 5. Повторяющееся значение повторяйте, а не говорите "то же"


\usepackage{fontspec}
\usepackage{polyglossia}

\setmainlanguage{russian}
\setotherlanguages{english}

\usepackage{xeCJK}
\setCJKmainfont{Noto Sans CJK JP}

% download "Linux Libertine" fonts:
% http://www.linuxlibertine.org/index.php?id=91&L=1
\setmainfont{Linux Libertine O} % or Helvetica, Arial, Cambria
% why do we need \newfontfamily:
% http://tex.stackexchange.com/questions/91507/
\newfontfamily{\cyrillicfonttt}{Linux Libertine O}

\AddEnumerateCounter{\asbuk}{\russian@alph}{щ} % для списков с русскими буквами
\setlist[enumerate, 2]{label=\asbuk*),ref=\asbuk*}

%% эконометрические сокращения
\DeclareMathOperator{\Cov}{Cov}
\DeclareMathOperator{\Corr}{Corr}
\DeclareMathOperator{\Var}{Var}
\DeclareMathOperator{\E}{E}
\def \hb{\hat{\beta}}
\def \hs{\hat{\sigma}}
\def \htheta{\hat{\theta}}
\def \s{\sigma}
\def \hy{\hat{y}}
\def \hY{\hat{Y}}
\def \v1{\vec{1}}
\def \e{\varepsilon}
\def \he{\hat{\e}}
\def \z{z}
\def \hVar{\widehat{\Var}}
\def \hCorr{\widehat{\Corr}}
\def \hCov{\widehat{\Cov}}
\def \cN{\mathcal{N}}
\def \P{\mathbb{P}}
\newcommand \id {\mathrm{id}\_\mathrm{for}\_\mathrm{online}}

\begin{document}



\vspace{20mm}

\textbf{Буси-до, 武士道, кодекс чести самурая:}

\vspace{5mm}

\begin{enumerate}
\item Подпишите работу сверху: \verb|id_for_online|, фамилию, имя, номер группы.
\item Должно быть выписано решение задачи, только ответ не засчитывается.
\item Для каждого пункта задания обведите полученный результат в торжественную рамочку.
\item Загрузите свою работу в лмс или на гитхаб в виде одного \verb|.pdf| файла.

Ссылка на гитхаб: \url{https://classroom.github.com/a/tciY6H9J}.
\item Имя файла должно иметь вид \verb|kr4_NNN.pdf| где вместо цифр \verb|NNN| следует написать \verb|id_for_online|.
\item При загрузке версий работы и на гитхаб, и в лмс, проверяется версия из лмс. 
\item Начало контрольной: 13:00. 

Дедлайны: 13:40 — без штрафа, 13.45 — со штрафом 30\%, 13:50 — со штрафом 60\%.
\item Оперативные важные сообщения будут в телеграм-канале \verb|@room112|.
\end{enumerate}

\vspace{20mm}

\textbf{Bushido, 武士道, samurai code of honour:}

\vspace{5mm}

\begin{enumerate}
\item State your identity at the top of the sheet: \verb|id_for_online|, first name, last name, group number.
\item Full solutions are required, answer without explanations is not graded. 
\item You should draw a pretty box around every final numeric answer or formula.
\item Upload your work to the lms or gihtub as a unique \verb|.pdf| file.

Github link: \url{https://classroom.github.com/a/tciY6H9J}.
\item The filename should be of the form \verb|kr4_NNN.pdf| with your \verb|id_for_online| instead of \verb|NNN|.
\item If you submit your work on github and in lms, then the version in lms is checked.
\item Start time: 13:00. Deadlines: 13:40 — without penalty, 13:45 — 30\% penalty, 13:50 — 60\% penalty.
\item For prompt and important messages read the telegram channel \verb|@room112|.
\end{enumerate}





\begin{center}
  \begin{minipage}{5cm}
    \includegraphics[height=30ex]{cheshire_cat.png}
  \end{minipage}
\end{center}
  
\newpage

Номер выполняемого варианта $n$ определяется как $n=(\id \mod 7)+1$. 


\section*{Вариант 1}
\begin{enumerate}

  \item {[5 баллов]} Монета подбрасывается 100 раз. По выборке из распределения Бернулли
  постройте 90\% доверительный интервал для вероятности выпадения орла в общем виде и найдите его
  реализацию, если выпал 61 орел.
  
  \item Рассмотрим выборку $X_1$, $X_2$, \ldots, $X_n$ из нормального распределения $\cN(1, \theta)$, 
  где $\theta$ — неизвестная дисперсия:
  
  \[
  \sum_{i=1}^{100} X_i = 150, \quad \sum_{i=1}^{100} X_i^2 = 600, \quad \sum_{i=1}^{100} (X_i - 1)^2 = 400.  
  \]
  
  С помощью метода максимального правдоподобия 
  постройте асимптотические доверительные интервалы с уровнем доверия 95\% в общем виде и найдите их реализации для  
  
  \begin{enumerate}
    \item параметра $\theta$ {[7 баллов]};
    \item стандартного отклонения $X_1$ {[8 баллов]};
    \item вероятности $\P(X_1 < 3)$ {[бонусный пункт в 10 баллов]}.
  \end{enumerate}
  
  
  
\end{enumerate}


\textbf{Буси-до, 武士道, кодекс чести самурая:}

\vspace{5mm}

\begin{enumerate}
\item Подпишите работу сверху: \verb|id_for_online|, фамилию, имя, номер группы.
\item Должно быть выписано решение задачи, только ответ не засчитывается.
\item Для каждого пункта задания обведите полученный результат в торжественную рамочку.
\item Загрузите свою работу в лмс или на гитхаб в виде одного \verb|.pdf| файла.

Ссылка на гитхаб: \url{https://classroom.github.com/a/tciY6H9J}.
\item Имя файла должно иметь вид \verb|kr4_NNN.pdf| где вместо цифр \verb|NNN| следует написать \verb|id_for_online|.
\item При загрузке версий работы и на гитхаб, и в лмс, проверяется версия из лмс. 
\item Начало контрольной: 13:00. 

Дедлайны: 13:40 — без штрафа, 13.45 — со штрафом 30\%, 13:50 — со штрафом 60\%.
\item Оперативные важные сообщения будут в телеграм-канале \verb|@room112|.
\end{enumerate}


\newpage
Номер выполняемого варианта $n$ определяется как $n=(\id \mod 7)+1$. 

\section*{Вариант 2}


\begin{enumerate}

\item {[5 баллов]} Монета подбрасывается 196 раз. По выборке из распределения Бернулли
постройте 95\% доверительный интервал для вероятности выпадения орла в общем виде и найдите его
реализацию, если выпало 54 орла.

\item Рассмотрим выборку $X_1$, $X_2$, \ldots, $X_n$ из нормального распределения $\cN(1, \theta^2)$, 
где $\theta^2$ — неизвестная дисперсия:

\[
\sum_{i=1}^{100} X_i = 150, \quad \sum_{i=1}^{100} X_i^2 = 600, \quad \sum_{i=1}^{100} (X_i - 1)^2 = 400.  
\]

С помощью метода максимального правдоподобия 
постройте асимптотические доверительные интервалы с уровнем доверия 90\% в общем виде и найдите их реализации для  

\begin{enumerate}
\item параметра $\theta$ {[7 баллов]}; \textit{Обратите внимание, именно $\theta$!}
\item дисперсии $X_1$ {[8 баллов]};
\item вероятности $\P(X_1 < 2)$ {[бонусный пункт в 10 баллов]}.
\end{enumerate}

\end{enumerate}

\textbf{Буси-до, 武士道, кодекс чести самурая:}

\vspace{5mm}

\begin{enumerate}
\item Подпишите работу сверху: \verb|id_for_online|, фамилию, имя, номер группы.
\item Должно быть выписано решение задачи, только ответ не засчитывается.
\item Для каждого пункта задания обведите полученный результат в торжественную рамочку.
\item Загрузите свою работу в лмс или на гитхаб в виде одного \verb|.pdf| файла.

Ссылка на гитхаб: \url{https://classroom.github.com/a/tciY6H9J}.
\item Имя файла должно иметь вид \verb|kr4_NNN.pdf| где вместо цифр \verb|NNN| следует написать \verb|id_for_online|.
\item При загрузке версий работы и на гитхаб, и в лмс, проверяется версия из лмс. 
\item Начало контрольной: 13:00. 

Дедлайны: 13:40 — без штрафа, 13.45 — со штрафом 30\%, 13:50 — со штрафом 60\%.
\item Оперативные важные сообщения будут в телеграм-канале \verb|@room112|.
\end{enumerate}



\newpage
Номер выполняемого варианта $n$ определяется как $n=(\id \mod 7)+1$. 

\section*{Вариант 3}


\begin{enumerate}

\item {[5 баллов]} Монета подбрасывается 169 раз. По выборке из распределения Бернулли
постройте 99\% доверительный интервал для вероятности выпадения орла в общем виде и найдите его
реализацию, если выпало 58 орлов.

\item Рассмотрим выборку $X_1$, $X_2$, \ldots, $X_n$ из нормального распределения $\cN(2, \theta)$, 
где $\theta$ — неизвестная дисперсия:

\[
\sum_{i=1}^{100} X_i = 125, 
\quad \sum_{i=1}^{100} X_i^2 = 200, 
\quad \sum_{i=1}^{100} (X_i - 2)^2 = 100.  
\]

С помощью метода максимального правдоподобия 
постройте асимптотические доверительные интервалы с уровнем доверия 95\% в общем виде и найдите их реализации для  

\begin{enumerate}
  \item параметра $\theta$ {[7 баллов]};
  \item стандартного отклонения $X_1$ {[8 баллов]};
  \item вероятности $\P(X_1 < 3)$ {[бонусный пункт в 10 баллов]}.
\end{enumerate}

\end{enumerate}

\textbf{Буси-до, 武士道, кодекс чести самурая:}

\vspace{5mm}

\begin{enumerate}
\item Подпишите работу сверху: \verb|id_for_online|, фамилию, имя, номер группы.
\item Должно быть выписано решение задачи, только ответ не засчитывается.
\item Для каждого пункта задания обведите полученный результат в торжественную рамочку.
\item Загрузите свою работу в лмс или на гитхаб в виде одного \verb|.pdf| файла.

Ссылка на гитхаб: \url{https://classroom.github.com/a/tciY6H9J}.
\item Имя файла должно иметь вид \verb|kr4_NNN.pdf| где вместо цифр \verb|NNN| следует написать \verb|id_for_online|.
\item При загрузке версий работы и на гитхаб, и в лмс, проверяется версия из лмс. 
\item Начало контрольной: 13:00. 

Дедлайны: 13:40 — без штрафа, 13.45 — со штрафом 30\%, 13:50 — со штрафом 60\%.
\item Оперативные важные сообщения будут в телеграм-канале \verb|@room112|.
\end{enumerate}


\newpage
Номер выполняемого варианта $n$ определяется как $n=(\id \mod 7)+1$. 

\section*{Вариант 4}


\begin{enumerate}

  \item {[5 баллов]} Монета подбрасывается 144 раза. По выборке из распределения Бернулли
  постройте 95\% доверительный интервал для вероятности выпадения орла в общем виде и найдите его
  реализацию, если выпало 45 орлов.
  
  \item Рассмотрим выборку $X_1$, $X_2$, \ldots, $X_n$ из нормального распределения $\cN(2, \theta^2)$, 
  где $\theta^2$ — неизвестная дисперсия:
  
  \[
  \sum_{i=1}^{100} X_i = 125, \quad \sum_{i=1}^{100} X_i^2 = 200, 
  \quad \sum_{i=1}^{100} (X_i - 2)^2 = 100.  
  \]
  
  С помощью метода максимального правдоподобия 
  постройте асимптотические доверительные интервалы с уровнем доверия 90\% в общем виде и найдите их реализации для  
  
  \begin{enumerate}
    \item параметра $\theta$ {[7 баллов]}; \textit{Обратите внимание, именно $\theta$!}
    \item дисперсии $X_1$ {[8 баллов]};
    \item вероятности $\P(X_1 < 1)$ {[бонусный пункт в 10 баллов]}.
  \end{enumerate}
  
\end{enumerate}
    
\textbf{Буси-до, 武士道, кодекс чести самурая:}

\vspace{5mm}

\begin{enumerate}
\item Подпишите работу сверху: \verb|id_for_online|, фамилию, имя, номер группы.
\item Должно быть выписано решение задачи, только ответ не засчитывается.
\item Для каждого пункта задания обведите полученный результат в торжественную рамочку.
\item Загрузите свою работу в лмс или на гитхаб в виде одного \verb|.pdf| файла.

Ссылка на гитхаб: \url{https://classroom.github.com/a/tciY6H9J}.
\item Имя файла должно иметь вид \verb|kr4_NNN.pdf| где вместо цифр \verb|NNN| следует написать \verb|id_for_online|.
\item При загрузке версий работы и на гитхаб, и в лмс, проверяется версия из лмс. 
\item Начало контрольной: 13:00. 

Дедлайны: 13:40 — без штрафа, 13.45 — со штрафом 30\%, 13:50 — со штрафом 60\%.
\item Оперативные важные сообщения будут в телеграм-канале \verb|@room112|.
\end{enumerate}




\newpage

Номер выполняемого варианта $n$ определяется как $n=(\id \mod 7)+1$. 


\section*{Вариант 5}
\begin{enumerate}

  \item {[5 баллов]} Монета подбрасывается 100 раз. По выборке из распределения Бернулли
  постройте 90\% доверительный интервал для вероятности выпадения орла в общем виде и найдите его
  реализацию, если выпал 61 орел.
  
  \item Рассмотрим выборку $X_1$, $X_2$, \ldots, $X_n$ из нормального распределения $\cN(1, \theta)$, 
  где $\theta$ — неизвестная дисперсия:
  
  \[
  \sum_{i=1}^{100} X_i = 150, \quad \sum_{i=1}^{100} X_i^2 = 600, \quad \sum_{i=1}^{100} (X_i - 1)^2 = 400.  
  \]
  
  С помощью метода максимального правдоподобия 
  постройте асимптотические доверительные интервалы с уровнем доверия 95\% в общем виде и найдите их реализации для  
  
  \begin{enumerate}
    \item параметра $\theta$ {[7 баллов]};
    \item стандартного отклонения $X_1$ {[8 баллов]};
    \item вероятности $\P(X_1 > 3)$ {[бонусный пункт в 10 баллов]}.
  \end{enumerate}
  
  
  
\end{enumerate}



\textbf{Буси-до, 武士道, кодекс чести самурая:}

\vspace{5mm}

\begin{enumerate}
\item Подпишите работу сверху: \verb|id_for_online|, фамилию, имя, номер группы.
\item Должно быть выписано решение задачи, только ответ не засчитывается.
\item Для каждого пункта задания обведите полученный результат в торжественную рамочку.
\item Загрузите свою работу в лмс или на гитхаб в виде одного \verb|.pdf| файла.

Ссылка на гитхаб: \url{https://classroom.github.com/a/tciY6H9J}.
\item Имя файла должно иметь вид \verb|kr4_NNN.pdf| где вместо цифр \verb|NNN| следует написать \verb|id_for_online|.
\item При загрузке версий работы и на гитхаб, и в лмс, проверяется версия из лмс. 
\item Начало контрольной: 13:00. 

Дедлайны: 13:40 — без штрафа, 13.45 — со штрафом 30\%, 13:50 — со штрафом 60\%.
\item Оперативные важные сообщения будут в телеграм-канале \verb|@room112|.
\end{enumerate}


\newpage
Номер выполняемого варианта $n$ определяется как $n=(\id \mod 7)+1$. 

\section*{Вариант 6}


\begin{enumerate}

\item {[5 баллов]} Монета подбрасывается 196 раз. По выборке из распределения Бернулли
постройте 95\% доверительный интервал для вероятности выпадения орла в общем виде и найдите его
реализацию, если выпало 54 орла.

\item Рассмотрим выборку $X_1$, $X_2$, \ldots, $X_n$ из нормального распределения $\cN(1, \theta^2)$, 
где $\theta^2$ — неизвестная дисперсия:

\[
\sum_{i=1}^{100} X_i = 150, \quad \sum_{i=1}^{100} X_i^2 = 600, \quad \sum_{i=1}^{100} (X_i - 1)^2 = 400.  
\]

С помощью метода максимального правдоподобия 
постройте асимптотические доверительные интервалы с уровнем доверия 90\% в общем виде и найдите их реализации для  

\begin{enumerate}
\item параметра $\theta$ {[7 баллов]}; \textit{Обратите внимание, именно $\theta$!}
\item дисперсии $X_1$ {[8 баллов]};
\item вероятности $\P(X_1 > 2)$ {[бонусный пункт в 10 баллов]}.
\end{enumerate}

\end{enumerate}


\textbf{Буси-до, 武士道, кодекс чести самурая:}

\vspace{5mm}

\begin{enumerate}
\item Подпишите работу сверху: \verb|id_for_online|, фамилию, имя, номер группы.
\item Должно быть выписано решение задачи, только ответ не засчитывается.
\item Для каждого пункта задания обведите полученный результат в торжественную рамочку.
\item Загрузите свою работу в лмс или на гитхаб в виде одного \verb|.pdf| файла.

Ссылка на гитхаб: \url{https://classroom.github.com/a/tciY6H9J}.
\item Имя файла должно иметь вид \verb|kr4_NNN.pdf| где вместо цифр \verb|NNN| следует написать \verb|id_for_online|.
\item При загрузке версий работы и на гитхаб, и в лмс, проверяется версия из лмс. 
\item Начало контрольной: 13:00. 

Дедлайны: 13:40 — без штрафа, 13.45 — со штрафом 30\%, 13:50 — со штрафом 60\%.
\item Оперативные важные сообщения будут в телеграм-канале \verb|@room112|.
\end{enumerate}


\newpage
Номер выполняемого варианта $n$ определяется как $n=(\id \mod 7)+1$. 

\section*{Вариант 7}


\begin{enumerate}

\item {[5 баллов]} Монета подбрасывается 169 раз. По выборке из распределения Бернулли
постройте 99\% доверительный интервал для вероятности выпадения орла в общем виде и найдите его
реализацию, если выпало 58 орлов.

\item Рассмотрим выборку $X_1$, $X_2$, \ldots, $X_n$ из нормального распределения $\cN(2, \theta)$, 
где $\theta$ — неизвестная дисперсия:

\[
\sum_{i=1}^{100} X_i = 125, 
\quad \sum_{i=1}^{100} X_i^2 = 200, 
\quad \sum_{i=1}^{100} (X_i - 2)^2 = 100.  
\]

С помощью метода максимального правдоподобия 
постройте асимптотические доверительные интервалы с уровнем доверия 95\% в общем виде и найдите их реализации для  

\begin{enumerate}
  \item параметра $\theta$ {[7 баллов]};
  \item стандартного отклонения $X_1$ {[8 баллов]};
  \item вероятности $\P(X_1 > 3)$ {[бонусный пункт в 10 баллов]}.
\end{enumerate}

\end{enumerate}


\textbf{Буси-до, 武士道, кодекс чести самурая:}

\vspace{5mm}

\begin{enumerate}
\item Подпишите работу сверху: \verb|id_for_online|, фамилию, имя, номер группы.
\item Должно быть выписано решение задачи, только ответ не засчитывается.
\item Для каждого пункта задания обведите полученный результат в торжественную рамочку.
\item Загрузите свою работу в лмс или на гитхаб в виде одного \verb|.pdf| файла.

Ссылка на гитхаб: \url{https://classroom.github.com/a/tciY6H9J}.
\item Имя файла должно иметь вид \verb|kr4_NNN.pdf| где вместо цифр \verb|NNN| следует написать \verb|id_for_online|.
\item При загрузке версий работы и на гитхаб, и в лмс, проверяется версия из лмс. 
\item Начало контрольной: 13:00. 

Дедлайны: 13:40 — без штрафа, 13.45 — со штрафом 30\%, 13:50 — со штрафом 60\%.
\item Оперативные важные сообщения будут в телеграм-канале \verb|@room112|.
\end{enumerate}



\newpage

The variant $n$ is defined as $n=(\id \mod 7)+1$. 


\section*{Variant 1}
\begin{enumerate}

\item {[5 points]} A coin was tossed 100 times and landed tails up in 61 tosses.
Provide the general formula and particular values
of the 90\% confidence interval for the probability of landing tails up.

\item Consider a random sample $X_1$, $X_2$, \ldots, $X_n$ from normal distribution $\cN(1, \theta)$, 
with unknown variance $\theta$:

\[
\sum_{i=1}^{100} X_i = 150, \quad \sum_{i=1}^{100} X_i^2 = 600, \quad \sum_{i=1}^{100} (X_i - 1)^2 = 400.  
\]

Using maximum likelihood 
provide general formula and particular values
of the 95\% asymptotic confidence interval for:

\begin{enumerate}
  \item parameter $\theta$ {[7 points]};
  \item standard deviation $X_1$ {[8 points]};
  \item probability $\P(X_1 < 1)$ {[bonus question for 10 points]}.
\end{enumerate}

\end{enumerate}


\textbf{Bushido, 武士道, samurai code of honour:}

\vspace{5mm}

\begin{enumerate}
\item State your identity at the top of the sheet: \verb|id_for_online|, first name, last name, group number.
\item Full solutions are required, answer without explanations is not graded. 
\item You should draw a pretty box around every final numeric answer or formula.
\item Upload your work to the lms or gihtub as a unique \verb|.pdf| file.

Github link: \url{https://classroom.github.com/a/tciY6H9J}.
\item The filename should be of the form \verb|kr4_NNN.pdf| with your \verb|id_for_online| instead of \verb|NNN|.
\item If you submit your work on github and in lms, then the version in lms is checked.
\item Start time: 13:00. Deadlines: 13:40 — without penalty, 13:45 — 30\% penalty, 13:50 — 60\% penalty.
\item For prompt and important messages read the telegram channel \verb|@room112|.
\end{enumerate}


\newpage

The variant $n$ is defined as $n=(\id \mod 7)+1$. 


\section*{Variant 2}
\begin{enumerate}

\item {[5 points]} A coin was tossed 196 times and landed tails up in 54 tosses.
Provide the general formula and particular values
of the 95\% confidence interval for the probability of landing tails up.

\item Consider a random sample $X_1$, $X_2$, \ldots, $X_n$ from normal distribution $\cN(1, \theta^2)$, 
with unknown variance $\theta^2$:

\[
\sum_{i=1}^{100} X_i = 150, \quad \sum_{i=1}^{100} X_i^2 = 600, \quad \sum_{i=1}^{100} (X_i - 1)^2 = 400.  
\]

Using maximum likelihood 
provide general formula and particular values
of the 95\% asymptotic confidence interval for:

\begin{enumerate}
  \item parameter $\theta$ {[7 points]}; \textit{Be careful, exactly for $\theta$!}
  \item variance $X_1$ {[8 points]};
  \item probability $\P(X_1 < 2)$ {[bonus question for 10 points]}.
\end{enumerate}

\end{enumerate}

\textbf{Bushido, 武士道, samurai code of honour:}

\vspace{5mm}

\begin{enumerate}
\item State your identity at the top of the sheet: \verb|id_for_online|, first name, last name, group number.
\item Full solutions are required, answer without explanations is not graded. 
\item You should draw a pretty box around every final numeric answer or formula.
\item Upload your work to the lms or gihtub as a unique \verb|.pdf| file.

Github link: \url{https://classroom.github.com/a/tciY6H9J}.
\item The filename should be of the form \verb|kr4_NNN.pdf| with your \verb|id_for_online| instead of \verb|NNN|.
\item If you submit your work on github and in lms, then the version in lms is checked.
\item Start time: 13:00. Deadlines: 13:40 — without penalty, 13:45 — 30\% penalty, 13:50 — 60\% penalty.
\item For prompt and important messages read the telegram channel \verb|@room112|.
\end{enumerate}


\newpage

The variant $n$ is defined as $n=(\id \mod 7)+1$. 


\section*{Variant 3}
\begin{enumerate}

\item {[5 points]} A coin was tossed 169 times and landed tails up in 58 tosses.
Provide the general formula and particular values
of the 99\% confidence interval for the probability of landing tails up.

\item Consider a random sample $X_1$, $X_2$, \ldots, $X_n$ from normal distribution $\cN(2, \theta)$, 
with unknown variance $\theta$:

\[
\sum_{i=1}^{100} X_i = 125, \quad \sum_{i=1}^{100} X_i^2 = 200, \quad \sum_{i=1}^{100} (X_i - 2)^2 = 100.  
\]

Using maximum likelihood 
provide general formula and particular values
of the 95\% asymptotic confidence interval for:

\begin{enumerate}
  \item parameter $\theta$ {[7 points]};
  \item standard deviation $X_1$ {[8 points]};
  \item probability $\P(X_1 < 3)$ {[bonus question for 10 points]}.
\end{enumerate}

\end{enumerate}


\textbf{Bushido, 武士道, samurai code of honour:}

\vspace{5mm}

\begin{enumerate}
\item State your identity at the top of the sheet: \verb|id_for_online|, first name, last name, group number.
\item Full solutions are required, answer without explanations is not graded. 
\item You should draw a pretty box around every final numeric answer or formula.
\item Upload your work to the lms or gihtub as a unique \verb|.pdf| file.

Github link: \url{https://classroom.github.com/a/tciY6H9J}.
\item The filename should be of the form \verb|kr4_NNN.pdf| with your \verb|id_for_online| instead of \verb|NNN|.
\item If you submit your work on github and in lms, then the version in lms is checked.
\item Start time: 13:00. Deadlines: 13:40 — without penalty, 13:45 — 30\% penalty, 13:50 — 60\% penalty.
\item For prompt and important messages read the telegram channel \verb|@room112|.
\end{enumerate}


\newpage

The variant $n$ is defined as $n=(\id \mod 7)+1$. 


\section*{Variant 4}
\begin{enumerate}

\item {[5 points]} A coin was tossed 144 times and landed tails up in 45 tosses.
Provide the general formula and particular values
of the 95\% confidence interval for the probability of landing tails up.

\item Consider a random sample $X_1$, $X_2$, \ldots, $X_n$ from normal distribution $\cN(2, \theta^2)$, 
with unknown variance $\theta^2$:

\[
\sum_{i=1}^{100} X_i = 125, \quad \sum_{i=1}^{100} X_i^2 = 200, \quad \sum_{i=1}^{100} (X_i - 2)^2 = 100.  
\]

Using maximum likelihood 
provide general formula and particular values
of the 90\% asymptotic confidence interval for:

\begin{enumerate}
  \item parameter $\theta$ {[7 points]}; \textit{Be careful, exactly for $\theta$!}
  \item variance $X_1$ {[8 points]};
  \item probability $\P(X_1 < 4)$ {[bonus question for 10 points]}.
\end{enumerate}

\end{enumerate}

\textbf{Bushido, 武士道, samurai code of honour:}

\vspace{5mm}

\begin{enumerate}
\item State your identity at the top of the sheet: \verb|id_for_online|, first name, last name, group number.
\item Full solutions are required, answer without explanations is not graded. 
\item You should draw a pretty box around every final numeric answer or formula.
\item Upload your work to the lms or gihtub as a unique \verb|.pdf| file.

Github link: \url{https://classroom.github.com/a/tciY6H9J}.
\item The filename should be of the form \verb|kr4_NNN.pdf| with your \verb|id_for_online| instead of \verb|NNN|.
\item If you submit your work on github and in lms, then the version in lms is checked.
\item Start time: 13:00. Deadlines: 13:40 — without penalty, 13:45 — 30\% penalty, 13:50 — 60\% penalty.
\item For prompt and important messages read the telegram channel \verb|@room112|.
\end{enumerate}


\newpage

The variant $n$ is defined as $n=(\id \mod 7)+1$. 


\section*{Variant 5}
\begin{enumerate}

\item {[5 points]} A coin was tossed 100 times and landed tails up in 61 tosses.
Provide the general formula and particular values
of the 90\% confidence interval for the probability of landing tails up.

\item Consider a random sample $X_1$, $X_2$, \ldots, $X_n$ from normal distribution $\cN(1, \theta)$, 
with unknown variance $\theta$:

\[
\sum_{i=1}^{100} X_i = 150, \quad \sum_{i=1}^{100} X_i^2 = 600, \quad \sum_{i=1}^{100} (X_i - 1)^2 = 400.  
\]

Using maximum likelihood 
provide general formula and particular values
of the 95\% asymptotic confidence interval for:

\begin{enumerate}
  \item parameter $\theta$ {[7 points]};
  \item standard deviation $X_1$ {[8 points]};
  \item probability $\P(X_1 > 1)$ {[bonus question for 10 points]}.
\end{enumerate}

\end{enumerate}

\textbf{Bushido, 武士道, samurai code of honour:}

\vspace{5mm}

\begin{enumerate}
\item State your identity at the top of the sheet: \verb|id_for_online|, first name, last name, group number.
\item Full solutions are required, answer without explanations is not graded. 
\item You should draw a pretty box around every final numeric answer or formula.
\item Upload your work to the lms or gihtub as a unique \verb|.pdf| file.

Github link: \url{https://classroom.github.com/a/tciY6H9J}.
\item The filename should be of the form \verb|kr4_NNN.pdf| with your \verb|id_for_online| instead of \verb|NNN|.
\item If you submit your work on github and in lms, then the version in lms is checked.
\item Start time: 13:00. Deadlines: 13:40 — without penalty, 13:45 — 30\% penalty, 13:50 — 60\% penalty.
\item For prompt and important messages read the telegram channel \verb|@room112|.
\end{enumerate}


\newpage

The variant $n$ is defined as $n=(\id \mod 7)+1$. 


\section*{Variant 6}
\begin{enumerate}

\item {[5 points]} A coin was tossed 196 times and landed tails up in 54 tosses.
Provide the general formula and particular values
of the 95\% confidence interval for the probability of landing tails up.

\item Consider a random sample $X_1$, $X_2$, \ldots, $X_n$ from normal distribution $\cN(1, \theta^2)$, 
with unknown variance $\theta^2$:

\[
\sum_{i=1}^{100} X_i = 150, \quad \sum_{i=1}^{100} X_i^2 = 600, \quad \sum_{i=1}^{100} (X_i - 1)^2 = 400.  
\]

Using maximum likelihood 
provide general formula and particular values
of the 95\% asymptotic confidence interval for:

\begin{enumerate}
  \item parameter $\theta$ {[7 points]}; \textit{Be careful, exactly for $\theta$!}
  \item variance $X_1$ {[8 points]};
  \item probability $\P(X_1 > 2)$ {[bonus question for 10 points]}.
\end{enumerate}

\end{enumerate}

\textbf{Bushido, 武士道, samurai code of honour:}

\vspace{5mm}

\begin{enumerate}
\item State your identity at the top of the sheet: \verb|id_for_online|, first name, last name, group number.
\item Full solutions are required, answer without explanations is not graded. 
\item You should draw a pretty box around every final numeric answer or formula.
\item Upload your work to the lms or gihtub as a unique \verb|.pdf| file.

Github link: \url{https://classroom.github.com/a/tciY6H9J}.
\item The filename should be of the form \verb|kr4_NNN.pdf| with your \verb|id_for_online| instead of \verb|NNN|.
\item If you submit your work on github and in lms, then the version in lms is checked.
\item Start time: 13:00. Deadlines: 13:40 — without penalty, 13:45 — 30\% penalty, 13:50 — 60\% penalty.
\item For prompt and important messages read the telegram channel \verb|@room112|.
\end{enumerate}


\newpage

The variant $n$ is defined as $n=(\id \mod 7)+1$. 


\section*{Variant 7}
\begin{enumerate}

\item {[5 points]} A coin was tossed 169 times and landed tails up in 58 tosses.
Provide the general formula and particular values
of the 99\% confidence interval for the probability of landing tails up.

\item Consider a random sample $X_1$, $X_2$, \ldots, $X_n$ from normal distribution $\cN(2, \theta)$, 
with unknown variance $\theta$:

\[
\sum_{i=1}^{100} X_i = 125, \quad \sum_{i=1}^{100} X_i^2 = 200, \quad \sum_{i=1}^{100} (X_i - 2)^2 = 100.  
\]

Using maximum likelihood 
provide general formula and particular values
of the 95\% asymptotic confidence interval for:

\begin{enumerate}
  \item parameter $\theta$ {[7 points]};
  \item standard deviation $X_1$ {[8 points]};
  \item probability $\P(X_1 > 3)$ {[bonus question for 10 points]}.
\end{enumerate}

\end{enumerate}

\textbf{Bushido, 武士道, samurai code of honour:}

\vspace{5mm}

\begin{enumerate}
\item State your identity at the top of the sheet: \verb|id_for_online|, first name, last name, group number.
\item Full solutions are required, answer without explanations is not graded. 
\item You should draw a pretty box around every final numeric answer or formula.
\item Upload your work to the lms or gihtub as a unique \verb|.pdf| file.

Github link: \url{https://classroom.github.com/a/tciY6H9J}.
\item The filename should be of the form \verb|kr4_NNN.pdf| with your \verb|id_for_online| instead of \verb|NNN|.
\item If you submit your work on github and in lms, then the version in lms is checked.
\item Start time: 13:00. Deadlines: 13:40 — without penalty, 13:45 — 30\% penalty, 13:50 — 60\% penalty.
\item For prompt and important messages read the telegram channel \verb|@room112|.
\end{enumerate}





\end{document}