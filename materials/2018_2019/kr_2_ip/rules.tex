\documentclass[a4paper,12pt]{article}

\usepackage{cancel}
%%% Работа с русским языком
\usepackage{cmap}					% поиск в PDF
\usepackage{mathtext} 				% русские буквы в формулах
\usepackage[T2A]{fontenc}			% кодировка
\usepackage[utf8]{inputenc}			% кодировка исходного текста
\usepackage[english,russian]{babel}	% локализация и переносы
\usepackage{booktabs}

%%% Дополнительная работа с математикой
\usepackage{amsfonts,amssymb,amsthm,mathtools} % AMS
\usepackage{amsmath}
\usepackage{icomma} % "Умная" запятая: $0,2$ --- число, $0, 2$ --- перечисление

%% Номера формул
%\mathtoolsset{showonlyrefs=true} % Показывать номера только у тех формул, на которые есть \eqref{} в тексте.

%% Шрифты
\usepackage{euscript}	 % Шрифт Евклид
\usepackage{mathrsfs} % Красивый матшрифт

\usepackage{ulem}
%% Свои команды
\DeclareMathOperator{\sgn}{\mathop{sgn}}

%% Перенос знаков в формулах (по Львовскому)
\newcommand*{\hm}[1]{#1\nobreak\discretionary{}
{\hbox{$\mathsurround=0pt #1$}}{}}

%%% Работа с картинками
\usepackage{graphicx}  % Для вставки рисунков
\graphicspath{{images/}{images2/}}  % папки с картинками
\setlength\fboxsep{3pt} % Отступ рамки \fbox{} от рисунка
\setlength\fboxrule{1pt} % Толщина линий рамки \fbox{}
\usepackage{wrapfig} % Обтекание рисунков и таблиц текстом

%%% Работа с таблицами
\usepackage{array,tabularx,tabulary,booktabs} % Дополнительная работа с таблицами
\usepackage{longtable}  % Длинные таблицы
\usepackage{multirow} % Слияние строк в таблице
\usepackage{upgreek}
\usepackage{enumerate}
\usepackage{ dsfont }

%%% Заголовок
\author{Равноапостольные адепты вероятностей}
\title{Захват королевства Пуассона}
\date{\today}

\begin{document} % конец преамбулы, начало документа

\section*{Правила сражений}

Итак, наточите мечи и запаситесь маной. Мы отправляемся в удивительный тур по полному опасностей королевству! Не бойтесь трудностей и не хватайтесь за головы, ведь у каждого отряда будет свой проводник из числа смелых ассистентов, живущих на земле Пуассона.

Командный тур проходит в несколько этапов. Первый этап будет проходить в окрестностях замка Пуассона. Каждой команде будет выдана карта. Подобраться к окрестностям замка можно по трём тропинкам. На каждой вас ждёт несколько локаций с увлекательными задачками. Начинать можно с любой тропинки. Чем ближе локация к замку, тем больше ценятся решённые задачи. Задача каждой команды: пробиться к замку и набрать максимальное количество монет за решение задач. 

 На каждой локации, вне зависимости от уровня команде выдаётся листок с пятью задачами: двумя средней сложности и тремя небольшой сложности. Для прохода на следующую локацию необходимо решить либо одну среднюю задачу, либо три лёгких.  Темы задач варьируются от локации к локации.

По мере приближения к замку задачи возрастают по цене, но не по сложности. Финальные локации на обеих картах: замок Пуассона и сокровищница Пуассона. Оставь незнание, всяк в него входящий! При достижении последней локации команде выдаётся листок с тремя сложными задачами.  У сложных задач принимаются частичные решения, и, соответственно, награда за них дробится. 

%Ниже можно увидеть таблицу с ценностью одной задачи на каждом этапе, выраженную в золотых монетах.

%\begin{center}
%	\hspace*{-2cm}
%	\begin{tabular}{ccccc}
%		\toprule
%		$ \text{Тип задачи/уровень} $ & Граница & Владения короля &Окрестности замка& Замок Пуассона \\ \midrule
%		Лёгкая задача & $1$ & $ 2 $  & $ 2 $&\\
%		Средняя задача & 3 & $ 5 $  & $ 6 $&\\
%		Сложная задача &  &  & &10\\
%		Бонус за весь листок & 2 &  3&4 &5\\ \bottomrule
%	\end{tabular}
%\end{center}
%
Если команда считает, что неспособна пройти на следующую локацию, но имеет достаточно монет, то может подкупить злобных, но жадных экспоненциальных гоблинов, которые проведут её на следующий уровень. Цена: 20 монет.


Второй этап проходит аналогично первому, но наши отважные путешественники находятся уже внутри замка Пуассона. Карта аналогична предыдущей, однако изменены локации и наполняющие их задачи. Новая цель: отыскать сокровищницу и снять магические вероятностные печати с её входа.  

В замке Пуассона гостит сумасшедший король соседних земель. Он любит проводить дуэли. Если он обратит на отряд взор, то команде придётся сражаться. Ниже официальный свод правил дуэли, принятый в королевстве Пуассона: 
Один участник команды проходит за дуэльный стол, где он должен один на один с соперником в течение 5 минут решить задачу. Выигрывает тот, кто первым решает правильно задачу. В случае неправильного ответа обоими участниками, магистр дуэлей выбирает победителя по более точному решению.


Король считает своим долгом, вовлечь все отряды в дуэли. Однако по истечению некоторого времени, магический куб вероятностей будет указывать ему на случайный отряд.





\end{document}