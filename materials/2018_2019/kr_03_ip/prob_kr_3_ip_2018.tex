\documentclass[12pt]{article}

\usepackage{tikz} % картинки в tikz
\usepackage{microtype} % свешивание пунктуации

\usepackage{array} % для столбцов фиксированной ширины

\usepackage{indentfirst} % отступ в первом параграфе

\usepackage{sectsty} % для центрирования названий частей
\allsectionsfont{\centering}

\usepackage{amsmath} % куча стандартных математических плюшек

\usepackage{comment}
\usepackage{amsfonts}

\usepackage[top=2cm, left=1cm, right=1cm, bottom=2cm]{geometry} % размер текста на странице

\usepackage{lastpage} % чтобы узнать номер последней страницы

\usepackage{enumitem} % дополнительные плюшки для списков
%  например \begin{enumerate}[resume] позволяет продолжить нумерацию в новом списке
\usepackage{caption}

\usepackage{hyperref} % гиперссылки

\usepackage{multicol} % текст в несколько столбцов


\usepackage{fancyhdr} % весёлые колонтитулы
\pagestyle{fancy}
\lhead{Теория вероятностей-ВШЭ}
\chead{2019-03-23}
\rhead{Контрольная 3. Максимум :)}
\lfoot{}
\cfoot{}
\rfoot{}
\renewcommand{\headrulewidth}{0.4pt}
\renewcommand{\footrulewidth}{0.4pt}



\usepackage{todonotes} % для вставки в документ заметок о том, что осталось сделать
% \todo{Здесь надо коэффициенты исправить}
% \missingfigure{Здесь будет Последний день Помпеи}
% \listoftodos --- печатает все поставленные \todo'шки


% более красивые таблицы
\usepackage{booktabs}
% заповеди из докупентации:
% 1. Не используйте вертикальные линни
% 2. Не используйте двойные линии
% 3. Единицы измерения - в шапку таблицы
% 4. Не сокращайте .1 вместо 0.1
% 5. Повторяющееся значение повторяйте, а не говорите "то же"


\usepackage{fontspec}
\usepackage{polyglossia}

\setmainlanguage{russian}
\setotherlanguages{english}

% download "Linux Libertine" fonts:
% http://www.linuxlibertine.org/index.php?id=91&L=1
\setmainfont{Linux Libertine O} % or Helvetica, Arial, Cambria
% why do we need \newfontfamily:
% http://tex.stackexchange.com/questions/91507/
\newfontfamily{\cyrillicfonttt}{Linux Libertine O}

\AddEnumerateCounter{\asbuk}{\russian@alph}{щ} % для списков с русскими буквами
\setlist[enumerate, 2]{label=\asbuk*),ref=\asbuk*}

%% эконометрические сокращения
\DeclareMathOperator{\Cov}{Cov}
\DeclareMathOperator{\Corr}{Corr}
\DeclareMathOperator{\Var}{Var}
\DeclareMathOperator{\pCorr}{pCorr}
\DeclareMathOperator{\E}{E}
\def \hb{\hat{\beta}}
\def \hs{\hat{\sigma}}
\def \htheta{\hat{\theta}}
\def \s{\sigma}
\def \hy{\hat{y}}
\def \hY{\hat{Y}}
\def \v1{\vec{1}}
\def \e{\varepsilon}
\def \he{\hat{\e}}
\def \z{z}
\def \hVar{\widehat{\Var}}
\def \hCorr{\widehat{\Corr}}
\def \hCov{\widehat{\Cov}}
\def \cN{\mathcal{N}}
\def \P{\mathbb{P}}


\begin{document}


\fbox{
  \begin{minipage}{42em}
    Имя, фамилия и номер группы:\vspace*{3ex}\par
    \noindent\dotfill\vspace{2mm}
  \end{minipage}
}


\begin{enumerate}
\item Маша подбрасывает правильную монетку три раза. Величина $X_i$ равна единице,
если в $i$-м броске выпал орёл,и нулю иначе. Определим также суммы $S_2 = X_1 + X_2$
и $S_3 = X_1 + X_2 + X_3$.

Рассмотрим геометрию, порождаемую скалярным произведением $\langle L, R\rangle = \Cov(L, R)$.



\begin{enumerate}
  \item Какие величины из набора $X_1$, $X_2$, $X_3$, $S_2$, $S_3$ ортогональны?
  \item Приведите пример любой непостоянной случайной величины, лежащей в $S_3^{\perp}$, ортогональном дополнении к $S_3$.
  \item Выразите через $X_1$ и $S_3$ проекцию $X_1$ на $S_3$.
  \item Выразите через $X_1$ и $S_3$ проекцию $X_1$ на $S_3^{\perp}$.
\end{enumerate}

Определение :) Частной корреляцией между величинами $L$ и $R$ при фиксированной величине $M$,
$\pCorr(L, R; M)$, называется корреляция между проекциями $L$ и $R$ на подпространство $M^{\perp}$.

\begin{enumerate}[resume]
  \item Найдите частную корреляцию между $X_1$ и $X_2$ при фиксированной $S$.
\end{enumerate}

\item В данном упражнении храбрый Винни-Пух докажет, что нормальное распределение
обладает максимальной энтропией среди всех распределений с заданным ожиданием и дисперсией.

\begin{enumerate}
  \item Помогите медведю с опилками в голове определить, что больше, $\ln t$ или $t-1$?
  \item У Винни-Пуха есть две функции плотности, $q(x)$ и $p(x)$.
  Подставив в найденное неравенство вместо $t$ отношение плотностей докажите, что
\[
 CE_p(q) = - \int_{-\infty}^{\infty} p(x) \ln q(x) \; dx \geq  - \int_{-\infty}^{\infty} p(x) \ln p(x) \; dx = H(p)
\]
\item Помогите Винни-Пуху вспомнить формулу плотности $q(x)$ для нормального распределения.
И найдите энтропию данного распределения, $H(q)$.

\item Для произвольной случайной величины с ожиданием $\mu$, дисперсией $\sigma^2$ и плотностью $p(x)$, и для нормальной плотности $q(x)$ найдите кросс-энтропию $CE_p(q)$ и завершите доказательство.

\end{enumerate}

\newpage

\item
В банке 10 независимых клиентских «окошек». В момент открытия в банк вошло 10
человек. Других клиентов банке не было. Предположим, что время обслуживания одного клиента
распределено экспоненциально с параметром $\lambda$.

Оцените параметр $\lambda$ методом максимального правдоподобия в каждой из ситуаций:
\begin{enumerate}
\item Менеджер записал время обслуживания первого клиента в каждом окошке. Первое окошко
  обслужило своего первого клиента за 10 минут, второе, своего первого, — за 20 минут; оставшуюся
  часть записей менеджер благополучно затерял.
\item Менеджер наблюдал за окошками в течение получаса и записывал время обслуживания первого
  клиента. Первое окошко обслужило своего первого клиента за 10 минут, второе, своего первого,
  — за 20 минут; остальные окошки еще обслуживали своих первых клиентов в тот момент, когда
  менеджер удалился.
\item Менеджер наблюдал за окошками в течение получаса.
 За эти полчаса два окошка успели обслужить своих первых клиентов. Остальные окошки
  ещё обслуживали своих первых клиентов в тот момент, когда менеджер удалился.
\item Менеджер наблюдал за окошками и решил записать время обслуживания первых двух клиентов.
  Первое окошко обслужило своего первого клиента за 10 минут, другое, своего первого, — за 20 минут.
  Сразу после того, как был обслужен второй клиент менеджер прекратил наблюдение.
\end{enumerate}

\item В день метеоролога, 23 марта, 23 метеоролога собрались сыграть в странную игру :)
У каждого из них есть монетка.
Монетка первого метеоролога выпадает орлом с вероятностью $1/2$, второго — $2/3$,
третьего — $3/4$, и так далее.

Метеорологи садятся за круглым столом в случайном порядке и одновременно подкидывают монетки.
Затем каждый смотрит на результаты подбрасываний двух своих соседей.
Если результаты бросков соседей совпадают между собой, то метеоролог покидает игру.
Оставшиеся в игре метеорологи повторяют подбрасывание монеток до тех пор,
пока в игре не останется один метеоролог или вообще никого.

Если в финале остался один метеоролог, то он объявляется Самым Главным Метеорологом Года.

У кого больше шансов стать Самым Главным?

\item Аня и Белла нашли неправильную монетку, что падает на орла с вероятностью $p$.
В первый день они подкинули её 100 раз, во второй — 200 раз, в третий — 400 раз.

Аня запомнила суммарное количество орлов за первые два дня — 120 орлов. А Белла —
суммарное количество за второй и третий день — 300 орлов.
\begin{enumerate}
  \item Найдите оценку $p$ с наименьшей дисперсией.
  \item Оцените дисперсию полученной оценки.
  \item Постройте 95\%-й доверительный интервал для $p$.
\end{enumerate}

\end{enumerate}

\end{document}
