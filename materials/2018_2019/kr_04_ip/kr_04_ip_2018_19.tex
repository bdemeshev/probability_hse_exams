\documentclass[12pt]{article}

\usepackage{tikz} % картинки в tikz
\usepackage{microtype} % свешивание пунктуации

\usepackage{array} % для столбцов фиксированной ширины

\usepackage{indentfirst} % отступ в первом параграфе

\usepackage{sectsty} % для центрирования названий частей
\allsectionsfont{\centering}

\usepackage{amsmath} % куча стандартных математических плюшек

\usepackage{comment}
\usepackage{amsfonts}

\usepackage[top=2cm, left=1cm, right=1cm, bottom=2cm]{geometry} % размер текста на странице

\usepackage{lastpage} % чтобы узнать номер последней страницы

\usepackage{enumitem} % дополнительные плюшки для списков
%  например \begin{enumerate}[resume] позволяет продолжить нумерацию в новом списке
\usepackage{caption}

\usepackage{hyperref} % гиперссылки

\usepackage{multicol} % текст в несколько столбцов


\usepackage{fancyhdr} % весёлые колонтитулы
\pagestyle{fancy}
\lhead{Теория вероятностей-ВШЭ}
\chead{2019-06-01}
\rhead{Контрольная 4. Бери от жизни всё :)}
\lfoot{}
\cfoot{}
\rfoot{}
\renewcommand{\headrulewidth}{0.4pt}
\renewcommand{\footrulewidth}{0.4pt}



\usepackage{todonotes} % для вставки в документ заметок о том, что осталось сделать
% \todo{Здесь надо коэффициенты исправить}
% \missingfigure{Здесь будет Последний день Помпеи}
% \listoftodos --- печатает все поставленные \todo'шки


% более красивые таблицы
\usepackage{booktabs}
% заповеди из докупентации:
% 1. Не используйте вертикальные линни
% 2. Не используйте двойные линии
% 3. Единицы измерения - в шапку таблицы
% 4. Не сокращайте .1 вместо 0.1
% 5. Повторяющееся значение повторяйте, а не говорите "то же"


\usepackage{fontspec}
\usepackage{polyglossia}

\setmainlanguage{russian}
\setotherlanguages{english}

% download "Linux Libertine" fonts:
% http://www.linuxlibertine.org/index.php?id=91&L=1
\setmainfont{Linux Libertine O} % or Helvetica, Arial, Cambria
% why do we need \newfontfamily:
% http://tex.stackexchange.com/questions/91507/
\newfontfamily{\cyrillicfonttt}{Linux Libertine O}

\AddEnumerateCounter{\asbuk}{\russian@alph}{щ} % для списков с русскими буквами
\setlist[enumerate, 2]{label=\asbuk*),ref=\asbuk*}

%% эконометрические сокращения
\DeclareMathOperator{\Cov}{Cov}
\DeclareMathOperator{\Corr}{Corr}
\DeclareMathOperator{\Var}{Var}
\DeclareMathOperator{\E}{E}
\DeclareMathOperator{\plim}{plim}
\def \hb{\hat{\beta}}
\def \hs{\hat{\sigma}}
\def \htheta{\hat{\theta}}
\def \s{\sigma}
\def \hy{\hat{y}}
\def \hY{\hat{Y}}
\def \v1{\vec{1}}
\def \e{\varepsilon}
\def \he{\hat{\e}}
\def \z{z}
\def \hVar{\widehat{\Var}}
\def \hCorr{\widehat{\Corr}}
\def \hCov{\widehat{\Cov}}
\def \cN{\mathcal{N}}
\def \P{\mathbb{P}}


\begin{document}


\fbox{
  \begin{minipage}{42em}
    Имя, фамилия и номер группы:\vspace*{3ex}\par
    \noindent\dotfill\vspace{2mm}
  \end{minipage}
}

Ровно 189 лет назад, 1 июня 1830 британский учёный Джон Росс открыл северный магнитный полюс :)


\begin{enumerate}
\item Пусть $y$ — стандартный нормальный $n$-мерный вектор. 
Он случайный, просто Джону Россу лень писать заглавные буквы для векторов :) 
Вектор $a$ — неслучайный, но тоже гордый $n$-мерный.

Пусть $H$ — матрица, проецирующая любой вектор на $(n-1)$-мерное подпространство $a^{\perp}$, 
являющееся ортогональным дополнением к вектору $a$. 
То есть, для любого вектора $v$ вектор $Hv$ перпендикулярен вектору $a$.

\begin{enumerate}
    \item Найдите матрицу $H$, если $n=3$ и $a=(1,2,2)'$.
    \item Найдите матрицу $H$ для произвольного $n$ и $a$;
    \item Найдите $\E(y)$ и $\Var(y)$;
    \item Найдите $\E(Hy)$ и $\Var(Hy)$;
    \item Укажите закон распределения $y'Hy$, где $y'$ — это транспонированный вектор $y$.
\end{enumerate}

\item Рассмотрим формулу, здорово упрощающую подсчёт критерия Пирсона:
\[
 \sum_{j=1}^m \frac{(X_j - np_j)^2}{np_j} + n = \sum_{j=1}^m \frac{X_j^2}{np_j}
\]

\begin{enumerate}
    \item Докажите формулу.
    \item Нарисуйте картинку к этой формуле. На картинке подпишите прямой угол, катеты и гипотенузу.
\end{enumerate}


\item На Земле короля Уильяма Джон Росс нашёл странную монетку. 
Он подбрасывает её $n$ раз и обнаруживает, что она выпадает на орла, решку и ребро. 
Джон Росс проверяет гипотезу $H_0$ о том, что все три вероятности равны.

Пусть $y = (y_1, y_2, y_3)'$ — количество выпадений орла, решки и ребра. Рассмотрим так же вектор
$z = (z_1, z_2, z_3)'$, такой, что $z_i = (y_i - \E(y_i)) / \sqrt{\E(y_i)}$. 
Джон Росс сознательно перепутал ожидание и дисперсию в классической формуле!

Предположим, что гипотеза $H_0$ верна.
\begin{enumerate}
    \item Укажите закон распределения каждой величины $y_i$;
    \item Найдите вектор $\E(y)$ и матрицу $\Var(y)$;
    \item Найдите вектор $\E(z)$ и матрицу $\Var(z)$;
    \item Докажите, что матрица $H=\Var(z)$ является проектором на ортогональное дополнение к некоторому вектору $a$. 
  Явно выпишите вектор $a$.
  \item Объясните, почему критерий Пирсона имеет хи-квадрат распределение с нужным числом степеней свободы.
  % \item Обобщите решение на случай прозвольных вероятностей или произвольного количества граней у монетки.
\end{enumerate}

\newpage

\item На Земле короля Уильяма Джон Росс нашёл странную монетку. 
Он подбрасывает её $n$ раз и обнаруживает, что она выпадает на орла, решку и ребро. 
Джон Росс проверяет гипотезу о том, что все три вероятности равны с помощью двух статистики: 
$LR$, отношения правдоподобия, и $CP$, критерия Пирсона. 

\begin{enumerate}
\item Найдите $\plim_{n\to\infty} \frac{LR}{CP}$;
\item Обобщите решение на случай произвольного количества равновероятных граней у монетки.    
% \item Обобщите решение на случай произвольного количества граней и произвольных вероятностей в гипотезе $H_0$.
\end{enumerate}

\item Идея доказательства состоятельности ML оценки :)

Пусть наблюдения $y_1$, \ldots, $y_n$ независимы и одинаково распределены с функцией плотности, зависящей от параметра $a$.
Истинное значение параметра обозначим буквой $a_0$. Оценку максимального правдоподобия обозначим $\hat a$.

Рассмотрим отмасштабированную логарифмическую функцию правдоподобия $\ell_n(a)=\ell(a) / n$, и
ожидаемую логарифмическую функцию правдоподобия\footnote{Внимание:
ожидание считается с помощью истинного $a_0$ от функции, в которую входит константа $a$.},
$\tilde \ell(a)=\E(\ell(a))$.
\begin{enumerate}
\item Что больше, $\ln x$ или $x-1$? Докажите!
\item В какой точке находится максимум функции $\ell_n(a)$?
\item В какой точке находится максимум функции $\tilde \ell(a)$?

Подсказка: рассмотрите выражение $\tilde \ell(a) - \tilde \ell(a_0)$ и примените доказанное неравество :)
\item К чему сходится $\ell_n(a)$ по вероятности?
%\item К чему сходится $\ell^{\prime\prime}_n(a)$?

\end{enumerate}


\end{enumerate}

\end{document}
