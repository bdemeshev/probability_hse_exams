\documentclass[12pt]{article}

\usepackage{tikz} % картинки в tikz
\usepackage{microtype} % свешивание пунктуации

\usepackage{array} % для столбцов фиксированной ширины

\usepackage{indentfirst} % отступ в первом параграфе

\usepackage{sectsty} % для центрирования названий частей
\allsectionsfont{\centering}

\usepackage{amsmath} % куча стандартных математических плюшек

\usepackage{comment}
\usepackage{amsfonts}

\usepackage[top=2cm, left=1.2cm, right=1.2cm, bottom=2cm]{geometry} % размер текста на странице

\usepackage{lastpage} % чтобы узнать номер последней страницы

\usepackage{enumitem} % дополнительные плюшки для списков
%  например \begin{enumerate}[resume] позволяет продолжить нумерацию в новом списке
\usepackage{caption}

\usepackage{longtable}
\usepackage{multicol}
\usepackage{multirow}


\usepackage{fancyhdr} % весёлые колонтитулы
\pagestyle{fancy}
\lhead{Теория вероятностей}
\chead{}
\rhead{Минимум к контрольной \textnumero 4 по ТВ и МС}
\lfoot{}
\cfoot{}
\rfoot{\thepage/\pageref{LastPage}}
\renewcommand{\headrulewidth}{0.4pt}
\renewcommand{\footrulewidth}{0.4pt}



\usepackage{todonotes} % для вставки в документ заметок о том, что осталось сделать
% \todo{Здесь надо коэффициенты исправить}
% \missingfigure{Здесь будет Последний день Помпеи}
% \listoftodos --- печатает все поставленные \todo'шки


% более красивые таблицы
\usepackage{booktabs}
% заповеди из документации:
% 1. Не используйте вертикальные линии
% 2. Не используйте двойные линии
% 3. Единицы измерения - в шапку таблицы
% 4. Не сокращайте .1 вместо 0.1
% 5. Повторяющееся значение повторяйте, а не говорите "то же"



\usepackage{fontspec}
\usepackage{polyglossia}

\setmainlanguage{russian}
\setotherlanguages{english}

% download "Linux Libertine" fonts:
% http://www.linuxlibertine.org/index.php?id=91&L=1
\setmainfont{Linux Libertine O} % or Helvetica, Arial, Cambria
% why do we need \newfontfamily:
% http://tex.stackexchange.com/questions/91507/
\newfontfamily{\cyrillicfonttt}{Linux Libertine O}

\AddEnumerateCounter{\asbuk}{\russian@alph}{щ} % для списков с русскими буквами
\setlist[enumerate, 2]{label=\asbuk*),ref=\asbuk*}

%% эконометрические сокращения
\DeclareMathOperator{\Cov}{Cov}
\DeclareMathOperator{\Corr}{Corr}
\DeclareMathOperator{\Var}{Var}
\DeclareMathOperator{\E}{E}
\def \hb{\hat{\beta}}
\def \hs{\hat{\sigma}}
\def \htheta{\hat{\theta}}
\def \s{\sigma}
\def \hy{\hat{y}}
\def \hY{\hat{Y}}
\def \v1{\vec{1}}
\def \e{\varepsilon}
\def \he{\hat{\e}}
\def \z{z}
\def \hVar{\widehat{\Var}}
\def \hCorr{\widehat{\Corr}}
\def \hCov{\widehat{\Cov}}
\def \cN{\mathcal{N}}
\def \P{\mathbb{P}}


\begin{document}

\section{Теоретический минимум}


\begin{enumerate}
  \item Дайте определение ошибки первого и второго рода, критической области.
  \item Укажите формулу доверительного интервала с уровнем доверия $(1-\alpha)$ для вероятности успеха, построенного по случайной выборке большого размера из распределения Бернулли $Bin(1, p)$.
\end{enumerate}

Для следующего блока вопросов предполагается, что величины $X_1$, $X_2$, \ldots, $X_n$ независимы и нормальны $\cN(\mu;\sigma^2)$.
Укажите формулу для статистики:

\begin{enumerate}[resume]
  \item Статистика, проверяющая гипотезу о математическом ожидании при известной дисперсии $\sigma^2$,
    и её распределение при справедливости основной гипотезы  $H_0$: $\mu = \mu_0$.
  \item Статистика, проверяющая гипотезу о математическом ожидании при неизвестной дисперсии $\sigma^2$,
    и её распределение при справедливости основной гипотезы  $H_0$: $\mu = \mu_0$.
\end{enumerate}


Для следующего блока вопросов предполагается, что есть две независимые случайные выборки:
выборка $X_1$, $X_2$, \ldots{ }размера $n_x$ из нормального распределения $\cN(\mu_x;\sigma^2_x)$
и выборка $Y_1$, $Y_2$, \ldots{ }размера $n_y$ из нормального распределения $\cN(\mu_y;\sigma^2_y)$.

Укажите формулу для статистики или границ доверительного интервала:

\begin{enumerate}[resume]
  \item Доверительный интервал для разницы математических ожиданий, когда дисперсии известны;
  \item Доверительный интервал для разницы математических ожиданий, когда дисперсии не известны, но равны;
  \item Статистика, проверяющая гипотезу о разнице математических ожиданий при известных дисперсиях,
    и её распределение при справедливости основной гипотезы $H_0$: $\mu_x - \mu_y = \Delta_0$;
  \item Статистика, проверяющая гипотезу о разнице математических ожиданий при неизвестных, но равных дисперсиях,
    и её распределение при справедливости основной гипотезы $H_0$: $\mu_x - \mu_y = \Delta_0$;
  \item Статистика, проверяющая гипотезу о равенстве дисперсий,
    и её распределение при справедливости основной гипотезы $H_0$: $\sigma^2_x = \sigma^2_y$.
\end{enumerate}


\section{Задачный минимум}



\begin{enumerate}

\item Пусть $X = (X_{1},..., X_{n})$  — случайная выборка из нормального
распределения с параметрами $\mu$ и ${\sigma}^2 = 4$. Используя реализацию случайной выборки, $x_{1} = -1.11$, $x_{2} = -6.10$, $x_{3} =  2.42$, постройте 90\%-ый доверительный интервал для неизвестного параметра $\mu$.

\item Пусть $X = (X_{1},..., X_{n})$ — случайная выборка из нормального распределения с неизвестными параметрами $\mu$ и ${\sigma}^2$. Используя реализацию случайной выборки, $x_{1} = -1.11$, $x_{2} = -6.10$, $x_{3} = 2.42$, постройте 90\%-ый доверительный интервал для неизвестного параметра $\mu$.

\item Пусть $X = (X_{1},..., X_{n})$ — случайная выборка из нормального распределения с неизвестными параметрами $\mu$ и ${\sigma}^2$. Используя реализацию случайной выборки, $x_{1} = 1.07$, $x_{2} = 3.66$, $x_{3} = -4.51$, постройте 80\%-ый доверительный интервал для неизвестного параметра ${\sigma}^2$.

\item Пусть $X = (X_{1},..., X_{n})$ и $Y = (Y_{1},..., Y_{m})$ — независимые случайные выборки из нормального распределения с параметрами $(\mu_{X},{\sigma^2_{X}})$ и $(\mu_{Y},{\sigma^2_{Y}})$ соответственно, причем $\sigma^2_{X} = 2$ и $\sigma^2_{Y} = 1$. Используя реализации случайных выборок\newline
\hspace*{3cm}$x_{1} = -1.11, x_{2} = -6.10, x_{3} = 2.42, y_{1} = -2.29, y_{2} = -2.91$,\newline постройте 95\%-ый доверительный интервал для разности математических ожиданий $\mu_{X} - \mu_{Y}$.

\item Пусть $X = (X_{1},..., X_{n})$ и $Y = (Y_{1},..., Y_{m})$— независимые случайные выборки из нормального распределения с параметрами $(\mu_{X},{\sigma^2_{X}})$ и $(\mu_{Y},{\sigma^2_{Y}})$ соответственно. Известно, что $\sigma^2_{X} = \sigma^2_{Y}$. Используя реализации случайных выборок\newline
\hspace*{3cm}$x_{1} = 1.53, x_{2} = 2.83, x_{3} = -1.25, y_{1} = -0.8, y_{2} = 0.06$,\newline постройте 95\%-ый доверительный интервал для разности математических ожиданий $\mu_{X} - \mu_{Y}$.

\item Пусть $X = (X_{1},..., X_{n})$ — случайная выборка из распределения Бернулли с параметром $p$. Используя реализацию случайной выборки $x = (x_{1},..., x_{n})$, в которой 55 нулей и 45 единиц, постройте приближенный 95\%-ый доверительный интервал для неизвестного параметра $p$.

\item Пусть $X = (X_{1},..., X_{n})$ и $Y = (Y_{1},..., Y_{m})$ — независимые случайные
выборки из распределения Бернулли с параметрами $p_{X} \in (0;1)$ и $p_{Y} \in (0;1)$ соответственно. Известно, что $n = 100$, $\bar{x}_{n} = 0.6$, $m = 200$, $\bar{y}_{m} = 0.4$. Постройте приближенный 95\%-ый доверительный интервал для отношения разности вероятностей успеха $p_{X} - p_{Y}$.

\item Дядя Вова (Владимир Николаевич) и Скрипач (Гедеван) зарабатывают на Плюке чатлы, чтобы купить гравицапу. Число заработанных за $i$-ый день чатлов имеет распределение Пуассона с неизвестным параметром $\lambda$. Заработки в различные дни независимы. За прошедшие 100 дней они заработали 250 чатлов.

С помощью метода максимального правдоподобия постройте приближенный 95\%-ый доверительный интервал для неизвестного параметра $\lambda$.

\item Пусть $X = (X_{1},..., X_{n})$ — случайная выборка из показательного (экспоненциального) распределения с плотностью распределения
\[
f(x,\lambda)=
\begin{cases}
\lambda e^{-\lambda x}\text{ при } x\geq 0 \\
0 \text{ при } x < 0 \\
\end{cases}
\]
где $\lambda > 0$ — неизвестный параметр распределения. Известно, что $n = 100$ и $\bar{x}_n = 0.52$.

С помощью метода максимального правдоподобия постройте приближенный 95\%-ый доверительный интервал для параметра $\lambda$.

% \item Пусть $X = (X_{1},..., X_{n})$  — случайная выборка из равномерного распределения на отрезке $[0; \theta]$, где  $\theta > 0$ — неизвестный параметр распределения. Известно, что $n = 100$ и $\bar{x}_n = 0.57$.

% С помощью метода максимального правдоподобия постройте приближенный 95\%-ый доверительный интервал для параметра $\theta$.
% тут брутальный дельта-метод :) это не минимум :)


\item Пусть $X = (X_{1},..., X_{n})$ — случайная выборка из нормального распределения с неизвестным математическим ожиданием $\mu$ и известной дисперсией $\sigma^2 = 4$. Объем выборки $n = 16$. Для тестирования основной гипотезы $H_{0}:\mu = 0$ против
альтернативной гипотезы $H_{1}:\mu = 2$ вы используете критерий: если $\bar{X} \leq 1$, то вы не отвергаете гипотезу $H_{0}$ , в противном случае вы отвергаете гипотезу $H_{0}$ в пользу гипотезы $H_{1}$. Найдите

\begin{enumerate}
   \item  вероятность ошибки 1-го рода;
   \item вероятность ошибки 2-го рода;
   \item мощность критерия;
\end{enumerate}


\item На основе случайной выборки, содержащей одно наблюдение $X_{1}$, тестируется гипотеза $H_{0} : X_{1} \sim U[-0.7;0.3]$ против альтернативной гипотезы $H_{1}: X_{1} \sim U[-0.3;0.7]$. Рассматривается критерий вида: если $X_{1} > c$ , то гипотеза $H_{0}$ отвергается в пользу гипотезы $H_{1}$. Выберите константу $c$ так, чтобы уровень значимости этого критерия составлял 0.1.

\item Пусть $X = (X_{1},..., X_{n})$ — случайная выборка из нормального распределения с параметрами $\mu$ и $\sigma^2 = 4$. Уровень значимости  $\alpha = 0.1$. Используя
реализацию случайной выборки $x_{1} = -1.11, x_{2} = -6.10, x_{3} = 2.42$, проверьте следующую гипотезу:$\begin{cases}
H_{0}:\mu = 0, \\
H_{1}:\mu > 0 \\
\end{cases}$

\item Пусть $X = (X_{1},..., X_{n})$ — случайная выборка из нормального распределения с параметрами $\mu$ и $\sigma^2$. Уровень значимости  $\alpha = 0.1$. Используя
реализацию случайной выборки $x_{1} = -1.11, x_{2} = -6.10, x_{3} = 2.42$, проверьте следующую гипотезу:$\begin{cases}
H_{0}:\mu = 0, \\
H_{1}:\mu > 0 \\
\end{cases}$

\item Пусть $X = (X_{1},..., X_{n})$ и $Y = (Y_{1},..., Y_{m})$ — независимые случайные
выборки из нормального распределения с параметрами $(\mu_{X},\sigma^2_{X})$ и $(\mu_{Y},\sigma^2_{Y})$ соответственно, причем  $\sigma^2_{X} = 2$ и $\sigma^2_{Y} = 1$. Уровень значимости $\alpha = 0.05$. Используя реализации случайных выборок\newline
\hspace*{3cm}$x_{1} = -1.11, x_{2} = -6.10, x_{3} = 2.42, y_{1} = -2.29, y_{2} = -2.91$,

проверьте следующую гипотезу:$\begin{cases}
H_{0}:\mu_{X} = \mu_{Y}, \\
H_{1}:\mu_{X} < \mu_{Y} \\
\end{cases}$

\item Пусть $X = (X_{1},..., X_{n})$ и $Y = (Y_{1},..., Y_{m})$ — независимые случайные
выборки из нормального распределения с параметрами $(\mu_{X},\sigma^2_{X})$ и $(\mu_{Y},\sigma^2_{Y})$ соответственно. Известно, что $\sigma^2_{X} = \sigma^2_{Y}$. Уровень значимости $\alpha = 0.05$. Используя реализации случайных выборок\newline
\hspace*{3cm}$x_{1} = 1.53, x_{2} = 2.83, x_{3} = -1.25, y_{1} = -0.8, y_{2} = 0.06$,

проверьте следующую гипотезу:$\begin{cases}
H_{0}:\mu_{X} = \mu_{Y}, \\
H_{1}:\mu_{X} < \mu_{Y} \\
\end{cases}$


\item Пусть $X = (X_{1},..., X_{n})$ и $Y = (Y_{1},..., Y_{m})$ — независимые случайные
выборки из нормального распределения с параметрами $(\mu_{X},\sigma^2_{X})$ и $(\mu_{Y},\sigma^2_{Y})$ соответственно. Уровень значимости $\alpha = 0.05$. Используя реализации случайных выборок\newline
\hspace*{3cm}$x_{1} = -1.11, x_{2} = -6.10, x_{3} = 2.42, y_{1} = -2.29, y_{2} = -2.91$,

проверьте следующую гипотезу:$\begin{cases}
H_{0}:\sigma^2_{X} = \sigma^2_{Y}, \\
H_{1}:\sigma^2_{X} > \sigma^2_{Y} \\
\end{cases}$

\item Пусть  $X = (X_{1},..., X_{n})$ — случайная выборка из распределения Бернулли с
неизвестным параметром $p \in (0;1)$.Имеется следующая информация о реализации
случайной выборки, содержащей $n = 100$ наблюдений: $\sum_{i=0}^{n} x_{i} = 60$. На уровне значимости $\alpha = 0.05$ требуется протестировать следующую гипотезу:$\begin{cases}
H_{0}:p = 0.5, \\
H_{1}:p > 0.5 \\
\end{cases}$

\item Пусть $X = (X_{1},..., X_{n})$ и $Y = (Y_{1},..., Y_{m})$-две независимые случайные выборки из распределения Бернулли с неизвестными параметрами $p_{X} \in (0; 1)$ и $p_{Y} \in (0; 1)$. Имеется следующая информация о реализациях этих случайных выборок: $n = 100$, $\sum_{i=1}^{n} x_{i} = 60$, $m = 150$,$\sum_{j=1}^{m} y_{j} = 50$. На уровне значимости $\alpha = 0.05$ требуется протестировать следующую гипотезу:
$\begin{cases}
H_{0}:p_{X} = p_{Y}, \\
H_{1}:p_{X} \neq p_{Y} \\
\end{cases}$

\item Вася Сидоров утверждает, что ходит в кино в два раза чаще, чем в спортзал, а в спортзал в два раза чаще, чем в театр. За последние полгода он 10 раз был в театре, 17 раз – в спортзале и 39 раз в кино. На уровне значимости 5\% проверьте утверждение Васи.

\item Вася очень любит тестировать статистические гипотезы. В этот раз Вася собирается проверить утверждение о том, что его друг Пётр звонит Васе исключительно в то время, когда Вася ест. Для этого Вася трудился целый год и провел серию из 365 испытаний. Результаты приведены в таблице ниже.
\begin{center}\begin{tabular}{r|rr}
\toprule
   & Пётр звонит   & Пётр не звонит  \\ \midrule
Вася ест           & $200$ & $40$ \\
 Вася не ест       & $25$ & $100$  \\ \bottomrule
\end{tabular}\end{center}
На уровне значимости 5\% протестируйте гипотезу о том, что Пётр звонит Васе
независимо от момента приема пищи Васей.

\item Пусть $X = (X_{1},..., X_{n})$ — случайная выборка из нормального распределения
с математическим ожиданием $\mu \in \mathbb{R}$ и дисперсией $v > 0$, где $\mu$ и $v$ — неизвестные параметры. Известно, что выборка состоит из $n = 100$ наблюдений, $\sum_{i=1}^{n} x_{i} = 30$, $\sum_{i=1}^{n} x^2_{i} = 146$. При помощи теста отношения правдоподобия протестируйте гипотезу $H_{0}:v = 1$ на уровне значимости 5\%.

\end{enumerate}

\section{Ответы}


\begin{enumerate}
\item $\left[-1.6 - 1.65 \cdot \frac{2}{\sqrt{3}}; -1.6 + 1.65 \cdot \frac{2}{\sqrt{3}} \right]$
\item $\left[-1.6 - 2.92 \cdot \sqrt{\frac{18.33}{3}}; -1.6 + 2.92 \cdot \sqrt{\frac{18.33}{3}} \right]$
\item $\left[\frac{17.43 \cdot 2}{4.61}; \frac{17.43 \cdot 2}{0.21} \right]$
\item $\left[-1.6 - (-2.6) - 1.96 \cdot \sqrt{\frac{2}{3} + \frac{1}{2}}; -1.6 - (-2.6) + 1.96 \cdot \sqrt{\frac{2}{3} + \frac{1}{2}} \right]$
\item $\left[1.04 - (-0.37) - 3.18 \cdot \sqrt{3.02} \sqrt{\frac{1}{3} + \frac{1}{2}}; 1.04 - (-0.37) + 3.18 \cdot \sqrt{3.02} \sqrt{\frac{1}{3} + \frac{1}{2}} \right]$
\item $\left[0.45 - 1.96 \cdot \sqrt{\frac{0.45 \cdot 0.55}{100}}; 0.45 + 1.96 \cdot \sqrt{\frac{0.45 \cdot 0.55}{100}} \right]$
\item $\left[0.6 - 0.4 - 1.96  \cdot \sqrt{\frac{0.6\cdot0.4}{100} + \frac{0.4 \cdot 0.6}{200}}; 0.6 - 0.4 + 1.96 \cdot \sqrt{\frac{0.6\cdot0.4}{100} + \frac{0.4 \cdot 0.6}{200}} \right]$
\item $\left[2.5 - 1.96 \cdot \sqrt{\frac{1}{40}}; 2.5 + 1.96 \cdot \sqrt{\frac{1}{40}} \right]$
\item $\left[\frac{1}{0.52} - 1.96 \cdot \sqrt{\frac{1}{100 \cdot 0.52^2}}; \frac{1}{0.52} + 1.96 \cdot \sqrt{\frac{1}{100 \cdot 0.52^2}} \right]$
\item
\begin{enumerate}
\item $\approx 0.02$
\item $\approx 0.02$
\item $\approx 0.98$
\end{enumerate}
\item $0.2$
\item $z_{obs} \approx -1.39 $, $z_{crit} = 1.28$, нет оснований отвергать $H_0$.
\item $t_{obs} \approx -0.65$, $t_{crit} = 1.89$, нет оснований отвергать $H_0$.
\item $z_{obs} \approx 0.93$, $z_{crit} = -1.65$, нет оснований отвергать $H_0$.
\item $t_{obs} \approx 0.89$, $t_{crit} = -2.35$, нет оснований отвергать $H_0$.
\item $F_{obs} \approx 95.37$, $F_{crit} = 199.5$, нет оснований отвергать $H_0$.
\item $z_{obs} \approx 2.04$, $z_{crit} = 1.65$, основная гипотеза отвергается.
\item $z_{obs} \approx 4.16$, $z_{crit} = 1.96$, основная гипотеза отвергается.
\item $\gamma_{obs} \approx 0.26$, $\gamma_{crit} = 5.99$, нет оснований отвергать $H_0$.
\item $\gamma_{obs} \approx 139.4$, $\gamma_{crit} = 3.84$, основная гипотеза отвергается.
\item $LR_{obs} \approx 5.5$, $LR_{crit} = 3.84$, основная гипотеза отвергается.
\end{enumerate}
\end{document}
