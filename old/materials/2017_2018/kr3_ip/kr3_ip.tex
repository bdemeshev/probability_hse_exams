\documentclass[11pt]{article} % размер шрифта

\usepackage{tikz} % картинки в tikz
\usepackage{microtype} % свешивание пунктуации

\usepackage{array} % для столбцов фиксированной ширины
\usepackage{verbatim} % для вставки комментариев

\usepackage{indentfirst} % отступ в первом параграфе

\usepackage{sectsty} % для центрирования названий частей
\allsectionsfont{\centering} % приказываем центрировать все sections

\usepackage{amsmath, amsfonts} % куча стандартных математических плюшек

\usepackage[top=1.5cm, left=1.5cm, right=1.5cm, bottom=1.5cm]{geometry} % размер текста на странице

\usepackage{lastpage} % чтобы узнать номер последней страницы

\usepackage{enumitem} % дополнительные плюшки для списков
%  например \begin{enumerate}[resume] позволяет продолжить нумерацию в новом списке
\usepackage{caption} % подписи к картинкам без плавающего окружения figure


\usepackage{fancyhdr} % весёлые колонтитулы
\pagestyle{fancy}
\lhead{Теория вероятностей}
\chead{}
\rhead{2018-03-24, Комоедица — праздник пробуждения медведя}
\lfoot{}
\cfoot{}
\rfoot{\thepage/\pageref{LastPage}}
\renewcommand{\headrulewidth}{0.4pt}
\renewcommand{\footrulewidth}{0.4pt}



\usepackage{todonotes} % для вставки в документ заметок о том, что осталось сделать
% \todo{Здесь надо коэффициенты исправить}
% \missingfigure{Здесь будет картина Последний день Помпеи}
% команда \listoftodos — печатает все поставленные \todo'шки

\usepackage{booktabs} % красивые таблицы
% заповеди из документации:
% 1. Не используйте вертикальные линии
% 2. Не используйте двойные линии
% 3. Единицы измерения помещайте в шапку таблицы
% 4. Не сокращайте .1 вместо 0.1
% 5. Повторяющееся значение повторяйте, а не говорите "то же"

\usepackage{fontspec} % поддержка разных шрифтов
\usepackage{polyglossia} % поддержка разных языков

\setmainlanguage{russian}
\setotherlanguages{english}

\setmainfont{Linux Libertine O} % выбираем шрифт
% можно также попробовать Helvetica, Arial, Cambria и т.Д.

% чтобы использовать шрифт Linux Libertine на личном компе,
% его надо предварительно скачать по ссылке
% http://www.linuxlibertine.org/index.php?id=91&L=1

\newfontfamily{\cyrillicfonttt}{Linux Libertine O}
% пояснение зачем нужно шаманство с \newfontfamily
% http://tex.stackexchange.com/questions/91507/

\AddEnumerateCounter{\asbuk}{\russian@alph}{щ} % для списков с русскими буквами
\setlist[enumerate, 2]{label=\asbuk*),ref=\asbuk*} % списки уровня 2 будут буквами а) б) ...

%% эконометрические и вероятностные сокращения
\DeclareMathOperator{\Cov}{Cov}
\DeclareMathOperator{\Corr}{Corr}
\DeclareMathOperator{\Var}{Var}
\DeclareMathOperator{\E}{E}
\def \hb{\hat{\beta}}
\def \hs{\hat{\sigma}}
\def \htheta{\hat{\theta}}
\def \s{\sigma}
\def \hy{\hat{y}}
\def \hY{\hat{Y}}
\def \v1{\vec{1}}
\def \e{\varepsilon}
\def \he{\hat{\e}}
\def \z{z}
\def \hVar{\widehat{\Var}}
\def \hCorr{\widehat{\Corr}}
\def \hCov{\widehat{\Cov}}
\def \cN{\mathcal{N}}

\let\P\relax
\DeclareMathOperator{\P}{\mathbb{P}}

\DeclareMathOperator{\Lin}{\mathrm{Lin}}
\DeclareMathOperator{\Linp}{\Lin^{\perp}}

 
\begin{document}

\begin{enumerate}
  \item Медведь Михайло-Потапыч уснул в берлоге и ему снится сон про $n$-мерное пространство.
    Особенно ярко ему снится вектор $X=(X_1, X_2, \ldots, X_n)$ и вектор $e=(1, 1, 1, \ldots, 1)$.
    \begin{enumerate}
      \item Изобразите векторы $X$ и $e$ в $n$-мерном пространстве;
      \item Изобразите проекцию $X$ на $\Lin \{e\}$, обозначим её $\hat X$;
      \item Изобразите проекцию $X$ на $\Linp \{e\}$, обозначим её $\hat X^{\perp}$;
      \item Выпишите явно вектора $\hat X$ и $\hat X^{\perp}$, и найдите их длины;
      \item Сформулируйте теорему Пифагора для нарисованного прямоугольного треугольника;
      \item Изобразите на рисунке такой угол $\alpha$, что обычная $t$-статистика, используемая при построении доверительного интервала для $\mu$, имела бы вид $t = \sqrt{n-1} \cdot \ctg \alpha$.
    \end{enumerate}

  \item Исследователь Михаил предполагает, что все виды медведепришельцев встречаются равновероятно. 
    Отправившись на охоту в район Малой Медведицы Михаил поймал двух лиловых кальмаромедведей, 
    одного двурога медведеспинного и одного медведезавра ящероголового. 
    
      Помогите Михаилу оценитель общее количество видов медведепришельцев с помощью метода максимального правдоподобия.


    \item Помотавшись по просторам Вселенной Михаил изменил своё мнение. 
      Никто кроме кальмаромедведей, двурогов и медведезавров не попадается, однако попадаются они явно с разной вероятностью. 
      Из 300 отловленных пришельцев оказалось 150 кальмаромедведей, 100 двурогов и 50 медведезавров. 
      Михаил считает, что медведепришельцы встречаются независимо, $p_1$ — вероятность встретить кальмаромедведя, $p_2$ — двурога.

      \begin{enumerate}
	\item Оцените вектор $p = (p_1, p_2)$ методом максимального правдоподобия;
	\item Оцените ковариационную матрицу $\Var(\hat p)$;
	\item Оцените дисперсию $\Var(\hat p_1 - \hat p_2)$;
	\item Постройте доверительный интервал для разницы долей $p_1 - p_2$.
      \end{enumerate}

  \item Винни-Пух лично измерил количество мёда (в кг) на 100 деревьях и обнаружил, что $\bar X = 10$ и $\hat\sigma^2 = 4$. 
    По мнению Кролика, состоятельная оценка для параметра $\alpha$ правильности мёда имеет вид $\hat \alpha = \bar X + \sqrt{\bar X + 6}$. 

    \begin{enumerate}
      \item «Халява, сэр!» Найдите точечную оценку параметра $\alpha$;
      \item Найдите 95\%-ый доверительный интервал для $\alpha$, симметричный относительно $\hat\alpha$.
    \end{enumerate}

  \item Фотографы Андрей и Белла независимо друг от друга пытаются фотографировать кадьяков. 
    Андрею удаётся сфотографировать одного кадьяка в неделю с вероятностью $0.5$, а Белле — с вероятностью $p$, 
    независимо друг от друга и от прошлого.
    За 100 недель они вместе сфотографировали 130 кадьяков.

    \begin{enumerate}
      \item Оцените $p$ и постройте 95\%-ый доверительный интервал для $p$;
      \item Оцените $p$ и постройте 95\%-ый доверительный интервал для $p$, если дополнительно известно, что один фотограф опередил другого на 10 фото.
    \end{enumerate}

\end{enumerate}
 

\textbf{Просто красивая задачка}. Эту задачу \textbf{не нужно} решать на кр :) 

Медведю Мишутке никак не удаётся заснуть в берлоге, и потому он подбрасывает правильную монетку $n$ раз. 
Обозначим вероятность того, что ни разу не идёт двух решек подряд буквой $q_n$. 
   

        \begin{enumerate}[label=\asbuk*)]
	  \item Найдите $2^8q_8$ и \textbf{назовите} это число;
          \item Найдите $\lim 2q_{n+1}/q_n$ и \textbf{назовите} это число.
	\end{enumerate}





\end{document}
