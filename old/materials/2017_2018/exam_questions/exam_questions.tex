\documentclass[11pt]{article} % размер шрифта

\usepackage{tikz} % картинки в tikz
\usepackage{microtype} % свешивание пунктуации

\usepackage{array} % для столбцов фиксированной ширины
\usepackage{verbatim} % для вставки комментариев

\usepackage{indentfirst} % отступ в первом параграфе

\usepackage{sectsty} % для центрирования названий частей
\allsectionsfont{\centering} % приказываем центрировать все sections

\usepackage{amsmath, amsfonts} % куча стандартных математических плюшек

\usepackage[top=1.5cm, left=1.5cm, right=1.5cm, bottom=1.5cm]{geometry} % размер текста на странице

\usepackage{lastpage} % чтобы узнать номер последней страницы

\usepackage{enumitem} % дополнительные плюшки для списков
%  например \begin{enumerate}[resume] позволяет продолжить нумерацию в новом списке
\usepackage{caption} % подписи к картинкам без плавающего окружения figure


\usepackage{fancyhdr} % весёлые колонтитулы
\pagestyle{fancy}
\lhead{Теория вероятностей и математическая статистика}
\chead{}
\rhead{2017-2018, вопросы к экзамену}
\lfoot{}
\cfoot{}
\rfoot{\thepage/\pageref{LastPage}}
\renewcommand{\headrulewidth}{0.4pt}
\renewcommand{\footrulewidth}{0.4pt}



\usepackage{todonotes} % для вставки в документ заметок о том, что осталось сделать
% \todo{Здесь надо коэффициенты исправить}
% \missingfigure{Здесь будет картина Последний день Помпеи}
% команда \listoftodos — печатает все поставленные \todo'шки

\usepackage{booktabs} % красивые таблицы
% заповеди из документации:
% 1. Не используйте вертикальные линии
% 2. Не используйте двойные линии
% 3. Единицы измерения помещайте в шапку таблицы
% 4. Не сокращайте .1 вместо 0.1
% 5. Повторяющееся значение повторяйте, а не говорите "то же"

\usepackage{fontspec} % поддержка разных шрифтов
\usepackage{polyglossia} % поддержка разных языков

\setmainlanguage{russian}
\setotherlanguages{english}

\setmainfont{Linux Libertine O} % выбираем шрифт
% можно также попробовать Helvetica, Arial, Cambria и т.Д.

% чтобы использовать шрифт Linux Libertine на личном компе,
% его надо предварительно скачать по ссылке
% http://www.linuxlibertine.org/index.php?id=91&L=1

\newfontfamily{\cyrillicfonttt}{Linux Libertine O}
% пояснение зачем нужно шаманство с \newfontfamily
% http://tex.stackexchange.com/questions/91507/

\AddEnumerateCounter{\asbuk}{\russian@alph}{щ} % для списков с русскими буквами
\setlist[enumerate, 2]{label=\asbuk*),ref=\asbuk*} % списки уровня 2 будут буквами а) б) ...

%% эконометрические и вероятностные сокращения
\DeclareMathOperator{\Cov}{Cov}
\DeclareMathOperator{\Corr}{Corr}
\DeclareMathOperator{\Var}{Var}
\DeclareMathOperator{\E}{E}
\def \hb{\hat{\beta}}
\def \hs{\hat{\sigma}}
\def \htheta{\hat{\theta}}
\def \s{\sigma}
\def \hy{\hat{y}}
\def \hY{\hat{Y}}
\def \v1{\vec{1}}
\def \e{\varepsilon}
\def \he{\hat{\e}}
\def \z{z}
\def \hVar{\widehat{\Var}}
\def \hCorr{\widehat{\Corr}}
\def \hCov{\widehat{\Cov}}
\def \cN{\mathcal{N}}

\let\P\relax
\DeclareMathOperator{\P}{\mathbb{P}}

\DeclareMathOperator{\Lin}{\mathrm{Lin}}
\DeclareMathOperator{\Linp}{\Lin^{\perp}}


\begin{document}

\begin{enumerate}

\item Многомерное нормальное распределение и его свойства.
\item Определение и свойства хи-квадрат распределения, распределения Стьюдента и Фишера. Их основные свойства. Работа с таблицами распределений.
\item Выборочное среднее, его математическое ожидание и дисперсия (с учетом поправки на конечный размер генеральной совокупности).
\item Выборочная дисперсия и ее математическое ожидание. Смещенная и несмещенная оценки для дисперсии по генеральной совокупности.
\item Стратифицированная случайная выборка. Выборочное среднее, его математическое ожидание. Дисперсия выборочного среднего при оптимальном и при пропорциональном размещении.
\item Статистические оценки. Свойства оценок; несмещенность, состоятельность, эффективность.
\item Методы получения оценок; метод моментов и метод максимального правдоподобия. Оценка параметров биномиального, нормального и равномерного распределений.
\item Информация Фишера. Неравенство Рао-Крамера-Фреше (без доказательства).
\item Доверительные интервалы. Доверительные интервалы для среднего при известной и неизвестной дисперсии. Доверительные интервалы для пропорции.
\item Доверительные интервалы для разности средних нормальных генеральных совокупностей.
\item Доверительный интервал для дисперсии нормальной генеральной совокупности.
\item Асимптотические доверительные интервалы параметров распределений, построенные с помощью оценок максимального правдоподобия.  Дельта-метод.
\item Проверка гипотез. Простые и сложные гипотезы. Критерий выбора между основной и альтернативной гипотезами. Уровень значимости. Мощность критерия. Ошибки первого и второго рода.
\item Проверка гипотез о конкретном значении для среднего, пропорции и дисперсии.
\item Проверка гипотез для разности двух средних и для разности двух пропорций. Проверка гипотез о равенстве двух дисперсий.
\item Лемма Неймана-Пирсона. Критерий отношения правдоподобия.
\item Критерии согласия. Статистика Колмогорова.
\item Критерий $\chi^2$. Проверка гипотез о соответствии наблюдений предполагаемому распределению вероятностей.
\item Критерий $\chi^2$. Проверка гипотезы о независимости признаков. Таблицы сопряженности признаков.
\item Непараметрические тесты. Критерий знаков. Ранговые критерии: Вилкоксона и Манна-Уитни.
\item Байесовский подход. Связь априорного и апостериорного распределения.
Отличия байесовского подхода к оцениванию параметров от метода максимального правдоподобия.
Байесовский доверительный интервал.
\item Байесовский подход. Алгоритм Гиббса. Алгоритм Метрополиса-Гастингса.
\end{enumerate}





\end{document}
