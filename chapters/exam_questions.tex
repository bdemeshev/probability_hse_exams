\newpage
\thispagestyle{empty}
\section{Вопросы к экзаменам}

\subsection*{Промежуточный экзамен}

\begin{enumerate}
	\item Аксиоматика Колмогорова. Случайные величины. Функция распределения случайной величины и ее основные свойства. Функция плотности
	\item Виды сходимости последовательности случайных величин
	\item Основные дискретные распределения: биномиальное, Пуассона, гипергеометрическое, отрицательное биномиальное. Примеры непрерывных распределений (равномерное, экспоненциальное)
	\item Неравенство Маркова и неравенство Чебышёва. Закон больших чисел
	\item Понятие о случайном векторе. Совместное распределение нескольких случайных величин. Независимость случайных величин. Маргинальные распределения
	\item Центральная предельная теорема
	\item Условная вероятность. Формула полной вероятности. Формула Байеса
	\item Математическое ожидание и дисперсия случайной величины и их свойства. Распределение функции от случайной величины
	\item Случайные события и операции над ними. Вероятностное пространство. Вероятности и правила действий с ними. Классическое определение вероятности. Независимость событий (попарная и в совокупности). Схема испытаний Бернулли
	\item Математическое ожидание и ковариационная матрица случайного вектора. Коэффициент корреляции и его свойства
	\item Условное распределение и условное математическое ожидание
	\item Теорема Муавра – Лапласа
	\item Неравенство Маркова и неравенство Чебышёва. Закон больших чисел
\end{enumerate}

\subsection*{Финальный экзамен}
\begin{enumerate}

\item Многомерное нормальное распределение и его свойства.
\item Определение и свойства хи-квадрат распределения, распределения Стьюдента и Фишера. Их основные свойства. Работа с таблицами распределений.
\item Выборочное среднее, его математическое ожидание и дисперсия (с учетом поправки на конечный размер генеральной совокупности).
\item Выборочная дисперсия и ее математическое ожидание. Смещенная и несмещенная оценки для дисперсии по генеральной совокупности.
\item Стратифицированная случайная выборка. Выборочное среднее, его математическое ожидание. Дисперсия выборочного среднего при оптимальном и при пропорциональном размещении.
\item Статистические оценки. Свойства оценок; несмещенность, состоятельность, эффективность.
\item Методы получения оценок; метод моментов и метод максимального правдоподобия. Оценка параметров биномиального, нормального и равномерного распределений.
\item Информация Фишера. Неравенство Рао-Крамера-Фреше (без доказательства).
\item Доверительные интервалы. Доверительные интервалы для среднего при известной и неизвестной дисперсии. Доверительные интервалы для пропорции.
\item Доверительные интервалы для разности средних нормальных генеральных совокупностей.
\item Доверительный интервал для дисперсии нормальной генеральной совокупности.
\item Асимптотические доверительные интервалы параметров распределений, построенные с помощью оценок максимального правдоподобия.  Дельта-метод.
\item Проверка гипотез. Простые и сложные гипотезы. Критерий выбора между основной и альтернативной гипотезами. Уровень значимости. Мощность критерия. Ошибки первого и второго рода.
\item Проверка гипотез о конкретном значении для среднего, пропорции и дисперсии.
\item Проверка гипотез для разности двух средних и для разности двух пропорций. Проверка гипотез о равенстве двух дисперсий.
\item Лемма Неймана-Пирсона. Критерий отношения правдоподобия.
\item Критерии согласия. Статистика Колмогорова.
\item Критерий $\chi^2$. Проверка гипотез о соответствии наблюдений предполагаемому распределению вероятностей.
\item Критерий $\chi^2$. Проверка гипотезы о независимости признаков. Таблицы сопряженности признаков.
\item Непараметрические тесты. Критерий знаков. Ранговые критерии: Вилкоксона и Манна-Уитни.
\item Байесовский подход. Связь априорного и апостериорного распределения.
Отличия байесовского подхода к оцениванию параметров от метода максимального правдоподобия.
Байесовский доверительный интервал.
\item Байесовский подход. Алгоритм Гиббса. Алгоритм Метрополиса-Гастингса.
\end{enumerate}
