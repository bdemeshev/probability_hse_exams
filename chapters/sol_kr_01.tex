\thispagestyle{empty}
\section{Решения контрольной номер 1}

\subsection[2017-2018]{\hyperref[sec:kr_01_2017_2018]{2017-2018}}
\label{sec:sol_kr_01_2017_2018}

\begin{enumerate}
\item
\begin{enumerate}
\item События называются независимыми, если  $ \P(A \cap B) = \P(A) \cdot \P(B)$
\item Запасёмся всеми нужными вероятностями:

$\P(A) = \frac{1}{2}$

$\P(B) = \frac{1}{3}$

$\P(C) = \frac{1}{2}$

$\P(A \cap C) = \frac{1}{3} $ — выпадет чётое число больше трёх

$\P(A \cap B)  = \frac{1}{6}$ — выпадет чётное число, кратное трём

$\P(A \cap C) = \frac{1}{6}$ — выпадет число, большее трёх и кратное трём

Теперь можно проверять независимость:

$\P(A \cap C) \neq \P(A) \cdot \P(C) \Rightarrow$  не являются независимыми

$ \P(A \cap B) = \P(A) \cdot \P(B) \Rightarrow$ являются независимыми

$ \P(B \cap C) = \P(B) \cdot \P(C) \Rightarrow$ являются независимыми

\end{enumerate}
\item
\begin{enumerate}
\item Количество возможных вариантов ТМ: $ C_{10}^2 $,  количество возможных
вариантов ЗМ: $ C_{24}^2 $. Количество их возможных сочетаний: $ C_{10}^2 \cdot C_{24}^2$,
где $ C_n^k = \frac{n!}{k!(n-k)!}$.
\item По классическому определению вероятностей, предполагая исходы равновероятными,
искомая вероятность равна $\frac{C_{16}^2}{C_{24}^2}$.
\item По тому же принципу:
\[
\frac{C_k^2}{C_{10}^2} = \frac{1}{15} \Rightarrow \frac{\frac{k!}{2!(k-2)!}}{\frac{10!}{2! \cdot 8!}} = \frac{1}{15} \Rightarrow \frac{(k-1)k}{2}\frac{ 2}{9 \cdot 10} = \frac{1}{15}
\]
Получаем квадратное уравнение вида $ k^2 - k - 6 = 0 $ с корнями $-2$ и $3$.
Так как $k$ не может быть отрицательным, ответ $3$.
\end{enumerate}
\item
\begin{enumerate}
\item Если эксперт отдаёт предпочтение Fit, то это можно интерпретировать как
«успех» в схеме Бернулли. Так как $\xi$ - количество успехов,
$ k \in [0;4]$, $p = \frac{1}{3} $, то
\[
\P(\xi = k) = C_n^k(p)^k(1-p)^{n-k}
\]

Большинство означает, что либо три, либо четыре эксперта выбрали Fit.
\[
\P(\xi = 3) = C_4^3\left(\frac{1}{3}\right)^3 \left(\frac{2}{3}\right)^{1} = \frac{8}{81}
\]
\[
\P(\xi = 4) = C_4^4\left(\frac{1}{3}\right)^4 \left(\frac{2}{3}\right)^{0} = \frac{1}{81}
\]
\[
\P( \xi > 2) =  \frac{9}{81}
\]
\item Аналогично:

\[ \P(\xi = 0) = C_4^0\left(\frac{1}{3}\right)^0 \left(\frac{2}{3}\right)^{4} = \frac{16}{81}\]

\[ \P(\xi = 1) = C_4^1\left(\frac{1}{3}\right)^1 \left(\frac{2}{3}\right)^{3} = \frac{32}{81}\]

\[ \P(\xi = 2) = C_4^2\left(\frac{1}{3}\right)^2 \left(\frac{2}{3}\right)^{2} = \frac{24}{81}\]

\begin{figure}[h!]
    \noindent\centering{
    \includegraphics[width=80mm]{images/kr1_2017_3.png}
    }
    \caption{Функция распределения}
    \label{cdf_kr2017}
\end{figure}

\item Все вероятности посчитаны, видим, что наибольшая достигается при $\xi=1$.
\item $\E(X) = np = \frac{4}{3} $, $ \Var(X) = npq = \frac{8}{9}$
\end{enumerate}
\item
\begin{enumerate}
\item Так как указано, что цена сметаны распределена равномерно на отерзке
$[250, 1000]$, максимальное значение цены — $1000$, это и есть необходимая сумма.
\item Вспомним, что функция распределения $F(x) = \P(X \leq x)$, нужно найти
такой $x$, что $ \P(X \leq x)=0.9$:
\[
0.9 = 1 - \exp({-x^{2}}) \Rightarrow \exp(-x^{2}) = 0.1 \Rightarrow -x^2 = \ln(0.1)  \Rightarrow x=  \sqrt{-\ln(0.1)}
\]
\item Взяв производную от функции распределения списка без сметаны, получим функцию
плотности:
\[
f_X(x) =
\begin{cases}
2x\exp(-x^2) & x \ge 0 \\
0 & \text{иначе}
\end{cases}
\]
Найдём математическое ожидание:
\[
\int_{0}^{+\infty}2x^2\exp({-x^2}) dx = -x \exp({-x^2})\big|_0^{+\infty} + \int_{0}^{+\infty}\exp({-x^2}) dx = \frac{\sqrt{\pi}}{2}
\]
\item Математическое ожидание суммы случайных величин равно сумме математических
ожиданий случайных влечин, если они существуют. Математическое ожидание от цены
сметаны равно: $ \frac{1000 + 250}{2} = 625$.
Математическое ожидание списка без сметаны было найдено в предыдущем пункте, его
осталось перевести в рубли. Получаем ответ: $ 625 + \frac{\sqrt{\pi}}{2} \cdot 1000 $.
\item Так как обе величины имеют абсолютно непрерывные распределения, вероятность
попасть в конкретную точку равна нулю.
\end{enumerate}
\item
\begin{enumerate}
\item $\P(\text{детектор показал ложь и подозреваемый лжёт}) = 0.9 \cdot 0.1 + 0.1 \cdot 0.95 = 0.185$
\item $\P(\text{невиновен}|\text{детектор показал ложь}) = \frac{0.9\cdot0.1}{0.185} = \frac{90}{185}$
\item $\P(\text{эксперт точно выявит преступника}) = (0.9)^9 \cdot 0.95$
\item $\P(\text{эксперт ошибочно выявит преступника}) = 9 \cdot 0.1 \cdot 0.9^8\cdot 0.05$
\end{enumerate}
\end{enumerate}


\subsection[2016-2017]{\hyperref[sec:kr_01_2016_2017]{2016-2017}}
\label{sec:sol_kr_01_2016_2017}

\begin{enumerate}
\item
\begin{enumerate}
\item Возможны четыре равновероятные ситуации:
\[
\P(\text{ММ}) = \P(\text{МД}) = \P(\text{ДМ}) = \P(\text{ДД}) = 1/4
\]

Посчитаем условную вероятность:
\[
\P(B \mid A) = \frac{\P(B \cap A)}{\P(A)} = \frac{\P(\text{МД, ДМ})}{\P(\text{ДМ, МД, ДД})} = \frac{2/4}{3/4} = \frac{2}{3}
\]

\item События $A$ и $B$ называются независимыми, если $\P(A \cap B) = \P(A) \cdot \P(B)$

В нашем случае: $\P(A \cap B) = \P (\text{МД, ДМ}) = 2/4$,
$\P(A) \cdot \P (B) = 3/4 \cdot 3/4$.

Следовательно, $\P(A \cap B) \neq \P(A) \cdot \P (B)$,
значит, события $A$ и $B$ не являются независимыми.
\end{enumerate}

\item Пусть событие $A_i$ означает, что $i$-ый узел системы дал сбой,
а событие $B_N$, что вся система дала сбой.

В условии сказано, что $\P(A_i) = 10^{-6}$,
а найти нужно такое максимальное $N \in \mathbb{N}$, при котором

\[
\P(B_N) \leq \frac{1}{10^2}
\]

\begin{align*}
\P(B_N) &= \P\left(\cup_{i=1}^n A_i\right) = 1 - \P (\left(\cup_{i=1}^n A_i\right)^c) \\
&\stackrel{\text{ф-ла де Моргана}}{=} 1 - \P \left(\cup_{i=1}^N A_i^c\right) \stackrel{A_1, \ldots, A_N \text{– независ.}}{=} 1 - \P(A_1^c) \cdot \ldots \cdot \P(A_N^c) \\
&= 1 - \left(1-\frac{1}{10^6}\right)^N
\end{align*}
Чтобы найти такое максимальное $N \in \mathbb{N}$, надо решить следующее неравенство
\begin{align*}
& 1 - \left(1-10^{-6}\right)^N \leq 10^{-2} \\
& 1 - 10^{-2} \leq \left(1-10^{-6}\right)^N \\
& \ln\left(1 - 10^{-2}\right) \leq N \ln \left(1 - 10^{-6}\right) \\
& N \leq \frac{\ln\left(1 - 10^{-2}\right)}{ \ln \left(1 - 10^{-6}\right)} \approx 10050.33
\end{align*}
Значит, максимальное $N$ равно $10050$.

\item Введём обозначения для событий.
Пусть $A$ означает, что человек имеет заболевание лёгких,
а $B$, что человек работал в шахте.

В условии сказано, что $\P(B \mid A) = 0.22$, $\P(B \mid A^c) = 0.14$, $\P(A) = 0.04$.
\begin{enumerate}
\item Нужно найти
\[
\P(A \mid B) = \frac{\P(A\cap B)}{\P (B)} = \frac{\P(B|A)\P(A)}{\P(B)}
\]
Для этого с помощью формулы полной вероятности посчитаем
\[
\P (B) = \P (B \mid A) \P(A) + \P (B \mid A^c) \P (A^c) = 0.22 \cdot 0.04 + 0.14 \cdot 0.96 = 0.1432
\]
Осталось подставить значения:
\[
\P(A \mid B) = \frac{0.22 \cdot 0.04}{0.1432} \approx 0.0615
\]

\item Все необходимые значения для второго пункта у нас есть,
осталось применить формулу условной вероятности:
\begin{align*}
\P  (A \mid B^c) &=  \frac{\P(A\cap B^c)}{\P (B^c)} =  \frac{\P (B^c \cap A)}{\P(A)} \cdot \frac{\P(A)}{\P (B^c)} = \P (B^c \mid A) \cdot \frac{\P(A)}{\P (B^c)} \\
&= (1-\P (B \mid A)) \cdot \frac{\P(A)}{1-\P (B)} = (1-0.22) \cdot \frac{0.04}{1-0.1432} \approx 0.0364
\end{align*}
\end{enumerate}
\item Введём индикатор события «Петя дал верный ответ на $i$-ый вопрос»:
\[
X_i =
\begin{cases}
1, & \text{если на } i \text{-ый вопрос теста Петя дал верный ответ} \\
0, & \text{иначе}
\end{cases}
\]

Заметим, что $X_i \sim Be\left(p = 1/5 \right)$, $X_1, \ldots, X_{17}$ – независимы,
$X = X_1 + \ldots + X_{17}$ – общее число верных ответов,
$X \sim Bin\left(n=17, p=1/5\right)$.

\begin{enumerate}
\item Наибольшее вероятное число правильных ответов $m_0$ может быть нвйдено по формуле:
\begin{enumerate}
\item[1)] если число $(n\cdot p - q)$ – не целое, где $q:=1-p$, то
\[
m_0 = [np-q] +1,
\]
\item[2)] если число  $(n\cdot p - q)$ – целое, то наиболее вероятных значений $m_0$ два:
\[
m_0' = np-q \text{ и } m_0'' = np-q+1
\]
\end{enumerate}
Итак, поскольку $np-q = 17\cdot\frac{1}{5} - \frac{4}{5} = 2.6$ – не целое, наиболее вероятное число верных ответов $m_0$ может быть найдено по формуле из пункта (1):
\[
m_0 = [np-q] +1 = [2.6] + 1 = 3
\]
\item \[\E(X) = np = 17 \cdot \frac{1}{5}=3.4\]

\[\Var(X) = npq = 17 \cdot \frac{1}{5} \cdot \frac{4}{5} = 2.72\]

\item
\begin{align*}
\P (\text{у Пети «отлично»}) &= \P (X\geq 15) = \P (X = 15) + \P (X= 16) + \P (X = 17) \\
& = C^{15}_{17} \cdot \left(\frac{1}{5}\right)^{15} \cdot \left(\frac{4}{5}\right)^2 + C^{16}_{17} \cdot \left(\frac{1}{5}\right)^{16} \cdot \left(\frac{4}{5}\right)^1 + C^{17}_{17} \cdot \left(\frac{1}{5}\right)^{17} \cdot \left(\frac{4}{5}\right)^0 \\
&= 136 \cdot \frac{16}{5^{17}} + 17 \cdot \frac{4}{5^{17}} + \frac{1}{5^{17}} \approx 2.94 \cdot 10^{-9}
\end{align*}
\item Рассмотрим первый вопрос теста. Петя может выбрать первый ответ с вероятностью $1/5$, и Вася
может выбрать первый ответ с вероятностью $1/5$. Тогда они оба выберут одинаковый ответ с вероятностью $1/25$.
Вариантов ответа в каждом вопросе $5$, значит, вероятность совпадения ответа в одном вопросе равна $1/5$.
Всего вопросов 17, тогда получаем
\[
\P(\text{все ответы Пети и Васи совпадают}) = \left(\frac{1}{5}\right)^{17}
\]

\end{enumerate}
\item Введём случайную велчину $\eta$, которая означает число потенциальных покупателей, с которыми контактировал продавец оборудования. По условию задачи, $\eta$ имеет таблицу распеределения:
\begin{center}
\begin{tabular}{ccc}
\toprule
$\eta$ & $ 1 $ & $2$ \\
$\P_{\eta}$ & $1/3$ & $2/3$ \\ \bottomrule
\end{tabular}
\end{center}
Случайная величина $\xi$ может принимать значения $0, 50000$ и $100000$
\begin{enumerate}

\item Найдём $\P (\xi = 0 )$. По формуле полной вероятности, имеем:
\begin{align*}
\P (\xi = 0) &= \P (\xi = 0 \mid \eta = 1 ) \cdot \P ( \eta = 1 ) + \P (\xi = 0 \mid \eta = 2 )  \cdot \P ( \eta = 2 ) \\
&= 0.9 \cdot \frac{1}{3} + 0.9\cdot0.9 \cdot \frac{2}{3} = 0.84
\end{align*}

\item Найдём $\P (\xi = 50000 )$ и $\P (\xi = 100000 )$ :
\begin{align*}
\P (\xi = 50000 ) &= \P (\xi = 50000 \mid \eta = 1 ) \cdot \P ( \eta = 1  ) +  \P (\xi = 50000 \mid \eta = 2 ) \cdot  \P ( \eta = 2 ) \\
&= 0.1 \cdot \frac{1}{3} + 2 \cdot 0.1 \cdot 0.9 \cdot \frac{2}{3} = 0.15(3)
\end{align*}
\begin{align*}
\P (\xi = 100000 ) &=  \P (\xi = 100000 \mid \eta = 1 ) \cdot \P ( \eta = 1 ) +  \P (\xi = 100000 \mid \eta = 2 ) \cdot  \P ( \eta = 2  ) \\
&= 0 \cdot \frac{1}{3} + 0.1\cdot 0.1  \cdot \frac{2}{3} = 0.00(6)
\end{align*}
Таблица распределения случайной величина $\xi$ имеет вид:

\begin{center}
\begin{tabular}{cccc}
\toprule
$\xi$ & $ 0 $ & $5000$ & $100000$ \\
$\P_{\xi}$ & $0.84$ & $0.15(3)$ & $0.00(6)$ \\ \bottomrule
\end{tabular}
\end{center}

Тогда функция распределения случайной величины $\xi$ имеет вид:
\[
F_{\xi} (X) =
\begin{cases}
0 & \text{при } x<0 \\
0.84 & \text{при } 0 \leq x < 50000 \\
0.84 + 0.15(3) & \text{при } 50000
\leq x < 100000 \\
1 & \text{при } x > 100000
\end{cases}
\]
Опр.: $F_{\xi} = \P (\xi \leq x ), x \in \mathbb{R}$
\item \[
\E (X) = 0 \cdot 0.84 + 50000 \cdot 0.15(3) + 100000 \cdot 0.00(6) = 8333.(3)
\]
\begin{align*}
\Var(X) &= (0 - 8333.(3))^2 \cdot 0.84 + (50000-8333.(3))^2 \cdot 0.15(3) \\
&+ (100000 - 8333.(3))^2 \cdot 0.00(6) = 380555555.(5)
\end{align*}
\end{enumerate}
\item
\begin{enumerate}
\item $ f_{\xi} (x)=
\begin{cases}
\frac{1}{b} & \text{при } x \in [0, b] \\
0 & \text{при } x \notin [0, b]
\end{cases}
$
\item  Известно, что если $\xi \sim U[a, b]$, то $\E (\xi) = \frac{a+b}{2}$. Стало быть, из уравнения $\E (\xi) = 1$ получаем $\frac{b}{2} = 1$,  то есть $b=2$.
\item Известно, что если $\xi \sim U[a, b]$, то $\Var (\xi) = \frac{(b-a)^2}{12}$. Значит, $\Var (\xi) = \frac{2^2}{12} = \frac{1}{3}$
\item Воспользуемся формулой $\P (\xi \in B ) = \int_B f_{\xi} (x) dx$. Имеем:
\[
\P (\xi > 1 ) = \P (\xi \in (1, + \infty) ) = \int_{1}^{+ \infty} f_{\xi} (x) dx = \int_{1}^{2} \frac{1}{2} dx = \frac{1}{2}
\]
\item Требуется найти такое минимальное число $q_{0.25}$, что $\int_{-\infty}^{q_{0.25}} f_{\xi} (x) dx = 0.25$. Итак:
\[
\int_{-\infty}^{q_{0.25}} f_{\xi} (x) dx = 0.25 \Leftrightarrow \int_{-\infty}^{q_{0.25}} \frac{1}{2} dx = 0.25 \Leftrightarrow \frac{1/2}{q_{0.25}} = 0.25 \Leftrightarrow
\]
\[
q_{0.25} = 2 \cdot 0.25 = 0.5
\]
\item
\begin{align*}
\E [ (\xi - \E(\xi))^{2017} ] &= \int_{-\infty}^{+\infty} (x- \E(\xi) )^{2017} \cdot f_{\xi} (x) dx = \int_{-\infty}^{+\infty} (x-1)^{2017} f_{\xi} (x) dx \\
&= \int_{0}^{2} (x-1)^{2017} \cdot \frac{1}{2} dx = \frac{(x-1)^{2018}}{2018} \cdot \frac{1}{2} \bigg\rvert_{x=0}^{x=2} =0
\end{align*}
\item $F_{\xi} (x) =
\begin{cases}
0 & \text{при } x < 0 \\
\frac{x}{2} & \text{при } 0 \leq x \leq 2 \\
1 & \text{при } x > 2
\end{cases}
$
\item Согласно условиям задачи, время до прихода 1-го поезда есть $\xi$; время до прихода 2-го поезда равно $\xi + b$; время до прихода 3-го (заветного) поезда есть $\xi + 2b$. Таким образом, Марья Ивановна в среднем ожидает «своего» поезда $\E (\xi + 2b) = 1 + 2b = 1 + 2 \cdot 2 = 5 $ минут. При этом $\Var (\xi + 2b) = \Var (\xi) = 1/3$
\item[к)] Пусть $\tau$ – наименьший номер поезда без «подозрительных лиц». По условию задачи, таблица распределения случайной величины $\tau$ имеет вид:

\begin{center}
\begin{tabular}{cccccc}
\toprule
$\tau$ & $ 1 $ & $2$ & $3$ & $4$ & \ldots \\
$\P_{\tau}$ & $1/4$ & $3/4\cdot1/4$ & $(3/4)^2 \cdot 1/4$ & $(3/4)^3 \cdot 1/4$ & \ldots\\ \bottomrule
\end{tabular}
\end{center}

То есть случайная величина $\tau$ имеет геометрическое распределение с параметром $p=1/4$ $(\tau \sim G(p=1/4))$.

Несложно сообразить, что время ожидания Глафирой Петровной «своего» поезда составляет: $\eta := \xi + b(\tau- 1)$. Стало быть, $\E (\eta) = \E (\xi) + b \cdot (\E(\tau)-1)  = 1 + 2 \cdot (4-1) = 7$ минут.

Здесь мы воспользовались тем фактом, что если $\eta \sim G(p)$, то $\E (\eta) = 1/p$
\item[и)] Найдём теперь вероятность $\P (\eta \geq 5 )$. Для нахождения искомой вероятности воспользуемся формулой полной вероятности:
\[
	\P (\eta \geq 5 ) = \P(\eta \geq 5, \tau < 3) +\P(\eta \geq 5, \tau = 3)+\P(\eta \geq 5, \tau > 3)
\]

Если Глафира уехала на первом или втором поезде,
то ждать больше 5 минут она не могла, то есть $\P(\eta \geq 5, \tau <3)=0$.

Если Глафира уехала на третьем поезде, то чтобы ждать больше пяти минут,
ей нужно ждать первый поезд больше минуты,
то есть $\P(\eta \geq 5, \tau = 3)=0.5 \P(\tau = 3)$.

Если Глафира уехала на четвертом поезде или позже, то она точно ждала больше 5 минут,
$\P(\eta \geq 5, \tau >3)=\P(\tau>3)$.

\[
\P(\eta \geq 5) = 0.5\P(\tau = 3) + \P(\tau > 3) = 0.5 \cdot (3/4)^2 \cdot (1/4) + (3/4)^3 = 63 / 128
\]

\end{enumerate}
\item Пусть $\xi$ — случайная величина, обозначающая число остановок лифта. Предствим её в виде суммы $\xi = \xi_2 + \ldots + \xi_{10}$, где $\xi_i$ — индикатор
того, что лифт остановился на $i$-ом этаже, то есть
\[
\xi_i = \begin{cases}
1 & \text{если лифт остановился} \\
0 & \text{иначе}
\end{cases}
\quad \forall i = 2, \ldots, 10
\]
Найдём соответсвующие вероятности:
\[
\P(\xi_i = 0) = \left(\frac{8}{9}\right)^9
\]
\[
\P(\xi_i = 1) = 1 - \P(\xi = 0) = 1 - \left(\frac{8}{9}\right)^9
\]
Тогда $\E(\xi_i) = \P(\xi_i = 0) \cdot 0 + \P(\xi_i = 1) \cdot 1 = 1 - \left(\frac{8}{9}\right)^9$, и в итоге получаем:
\[
\E(\xi) = 9 \cdot \E(\xi_i) = 9 \cdot \left(1 - \left(\frac{8}{9}\right)^9\right)
\]
\end{enumerate}

\subsection[2015-2016]{\hyperref[sec:kr_01_2015_2016]{2015-2016}}
\label{sec:sol_kr_01_2015_2016}

\begin{enumerate}
\item
\begin{enumerate}
\item[$\alpha$)] Найдём вероятности каждого события:
$\P(A) = 1/2$, $\P(B) = 1/2$, $\P(C) = 1/2$.

Проверим попарную независимость:
\begin{itemize}
\item $\P(A \cap B) = 1/4$, $\P(A) \cdot \P(B) = 1/2 \cdot 1/2 = 1/4$
\item $\P(A \cap C) = 1/4$, $\P(A) \cdot \P(C) = 1/2 \cdot 1/2 = 1/4$
\item $\P(B \cap C) = 1/4$, $\P(B) \cdot \P(C) = 1/2 \cdot 1/2 = 1/4$
\end{itemize}
Значит, события попарно независимы.
\item[$\beta$)] События $A_1, A_2, A_3$ называются независимыми в совокупности,
если $\P(A_1 \cap A_2 \cap A_3) = \P(A_1) \cdot \P(A_2) \cdot \P(A_3)$.

В нашем случае: $\P(A \cap B \cap C) = 0$, $ \P(A) \cdot \P(B) \cdot \P(C) = (1/2)^3$,
следовательно, события не являются независимыми в совокупности.
\end{enumerate}

\item
\begin{enumerate}
\item[$\alpha$)] Воспользуемся формулой полной вероятности:
\begin{align*}
\P(\text{выпала «6»}) &= \P(\text{выпала «6»} \mid \text{взят белый кубик}) \cdot \P(\text{взят белый кубик}) \\
&+ \P(\text{выпала «6»} \mid \text{взят красный кубик}) \cdot \P(\text{взят красный кубик}) \\
&= \frac{1}{6} \cdot \frac{1}{2} + \frac{1}{3} \cdot \frac{1}{2} = \frac{1}{4}
\end{align*}
\item[$\beta$)] Воспользуемся формулой условной вероятности и результатом предыдущего пункта:
\begin{align*}
\P(\text{взят красный кубик} \mid \text{выпала «6»}) &= \frac{\P(\text{взят красный кубик} \cap \text{выпала «6»})}{\P(\text{выпала «6»})}  \\
&= \frac{\frac{1}{2}\cdot \frac{1}{3}}{\frac{1}{4}} = \frac{2}{3}
\end{align*}
\end{enumerate}

\item
\begin{enumerate}
\item[$\alpha$)] Совместное распределение имеет вид:
\begin{center}
\begin{tabular}{@{}lllllll@{}}
\toprule
$\eta$ $\backslash$ $\xi$ & $1$                            & $2$                            & $3$                            & $4$                            & $5$                            & $6$                            \\ \midrule
$1$           & $\frac{2}{15}\cdot\frac{1}{6}$ & $\frac{2}{15}\cdot\frac{1}{6}\mbox{*}$  & $\frac{2}{15}\cdot\frac{1}{6}\mbox{*}$   & $\frac{2}{15}\cdot\frac{1}{6} \mbox{*}$   & $\frac{2}{15}\cdot\frac{1}{6} \mbox{*}$   & $\frac{1}{3}\cdot\frac{1}{6} \mbox{*}$   \\
$2$           & $\frac{2}{15}\cdot\frac{1}{6}$ & $\frac{2}{15}\cdot\frac{1}{6}$ & $\frac{2}{15}\cdot\frac{1}{6}\mbox{*}$   & $\frac{2}{15}\cdot\frac{1}{6}\mbox{*}$   & $\frac{2}{15}\cdot\frac{1}{6}\mbox{*}$   & $\frac{1}{3}\cdot\frac{1}{6} \mbox{*}$   \\
$3$           & $\frac{2}{15}\cdot\frac{1}{6}$ & $\frac{2}{15}\cdot\frac{1}{6}$ & $\frac{2}{15}\cdot\frac{1}{6}$ & $\frac{2}{15}\cdot\frac{1}{6} \mbox{*}$   & $\frac{2}{15}\cdot\frac{1}{6} \mbox{*}$   & $\frac{1}{3}\cdot\frac{1}{6} \mbox{*}$   \\
$4$           & $\frac{2}{15}\cdot\frac{1}{6}$ & $\frac{2}{15}\cdot\frac{1}{6}$ & $\frac{2}{15}\cdot\frac{1}{6}$ & $\frac{2}{15}\cdot\frac{1}{6}$ & $\frac{2}{15}\cdot\frac{1}{6} \mbox{*}$ & $\frac{1}{3}\cdot\frac{1}{6} \mbox{*}$   \\
$5$           & $\frac{2}{15}\cdot\frac{1}{6}$ & $\frac{2}{15}\cdot\frac{1}{6}$ & $\frac{2}{15}\cdot\frac{1}{6}$ & $\frac{2}{15}\cdot\frac{1}{6}$ & $\frac{2}{15}\cdot\frac{1}{6}$ & $\frac{1}{3}\cdot\frac{1}{6} \mbox{*}$   \\
$6$           & $\frac{2}{15}\cdot\frac{1}{6}$ & $\frac{2}{15}\cdot\frac{1}{6}$ & $\frac{2}{15}\cdot\frac{1}{6}$ & $\frac{2}{15}\cdot\frac{1}{6}$ & $\frac{2}{15}\cdot\frac{1}{6}$ & $\frac{1}{3}\cdot\frac{1}{6}$ \\ \bottomrule
\end{tabular}
\end{center}
\item[$\beta$)] $\P(\text{выиграет белый кубик}) = (6 + 5 + 4 + 3 + 2) \cdot \frac{2}{15}\cdot\frac{1}{6} + 1 \cdot \frac{1}{3}\cdot\frac{1}{6} = \frac{1}{2}$.

Значит, Пете безразлично, какой кубик брать.
\item[$\gamma)$] $F_{\zeta}(x) = \P(\zeta \leq x)$

Выпишем таблицу распределения случайной величины $\zeta$:

\begin{center}
\begin{tabular}{@{}lcccccc@{}}
\toprule
$\zeta$     & $1$                              & $2$                                      & $3$                                      & $4$                                      & $5$                                      & $6$                                                                              \\ \midrule
$\P(\cdot)$ & $\frac{2}{15} \cdot \frac{1}{6}$ & $\frac{2}{15} \cdot \frac{1}{6} \cdot 3$ & $\frac{2}{15} \cdot \frac{1}{6} \cdot 5$ & $\frac{2}{15} \cdot \frac{1}{6} \cdot 7$ & $\frac{2}{15} \cdot \frac{1}{6} \cdot 9$ & $\frac{1}{3} \cdot \frac{1}{6} \cdot 6 + \frac{2}{15} \cdot \frac{1}{6} \cdot 5$ \\ \bottomrule
\end{tabular}
\end{center}

Тогда функция распределения имеет вид:
\[
F_{\zeta}(x) =
\begin{cases}
0 & x \leq 1 \\
\frac{1}{45} & 1 < x \leq 2 \\
\frac{4}{45} & 2 < x \leq 3 \\
\frac{9}{45} & 3 < x \leq 4 \\
\frac{16}{45} & 4 < x \leq 5 \\
\frac{25}{45} & 5 < x \leq 6 \\
1 & x > 6
\end{cases}
\]
\item[$\delta$)] $\E(\zeta) = \frac{2}{15} \cdot \frac{1}{6} \cdot 1 + \frac{2}{15} \cdot \frac{1}{6} \cdot 3 \cdot 2 + \frac{2}{15} \cdot \frac{1}{6} \cdot 5 \cdot 3 + \frac{2}{15} \cdot \frac{1}{6} \cdot 7 \cdot 4 + \frac{2}{15} \cdot \frac{1}{6} \cdot 9 \cdot 5 + \frac{1}{3} \cdot \frac{1}{6} \cdot 6 + \frac{2}{15} \cdot \frac{1}{6} \cdot 6 = \frac{43}{9} \approx 4.8 $
\end{enumerate}
\item Пусть $x$ — вероятность того, что мужчина честно любит петь в душе.

Распишем по формуле полной вероятности вероятность получить ответ «да»:
\begin{align*}
P(\text{ответ «Да»}) &= 1 \cdot \P(\text{выпала «6»}) + x \cdot(\P(\text{выпала «2»}) + \P(\text{выпала «3»}) \\
&+ \P(\text{выпала «4»}) + \P(\text{выпала «5»})) = 1 \cdot \frac{1}{6} + x \cdot \frac{4}{6} \Rightarrow x = \frac{3}{4}
\end{align*}
Тогда истинный процент «певцов» составляет $75 \%$

\item Предположим, что ваше имя — Студент (7 букв), а фамилия — Идеальный (9 букв).
\begin{enumerate}
\item[$\alpha$)] $\P(\text{напишет фаимлию правильно}) = (0.9)^9$
\item[$\beta$)] $\P(\text{ровно 2 ошибки в имени}) = C_{7}^2 \cdot 0.1^2 \cdot 0.9^5$
\item[$\gamma$)] Наиболее вероятное число ошибок — 1
\item[$\delta$)] $\P(\text{допустит хотя бы одну ошибку}) = 1 - \P(\text{не допустит ни одной ошибки}) = 1 - (0.9)^{16}$
\end{enumerate}

\item
\begin{enumerate}
\item[$\alpha$)] Из условия $\int_{0}^{1} (cy^2 + y) dy = 1$ получаем, что $c=3/2$.
\item[$\beta$)]
$F_{Y} (y) =
\begin{cases}
1 & y > 1 \\
\frac{y^3 + y^2}{2} & 0 < y \leq 1 \\
0 & y < 0
\end{cases} $
\item[$\gamma$)] $\P(Y < 0.5) = \int_{0}^{0.5} \left(\frac{3}{2} y^2 + y   \right) dy = \frac{3}{16}$
\item[$\delta$)] $F_{Y} (y) = 0.5 \Rightarrow y \approx 0.75 $
\item[$\epsilon$)] $\P(Y > 0.5 \mid Y \geq 0.25) = \frac{\P(Y > 0.5)}{\P(Y \geq 0.25)} = \frac{1 - \frac{3}{16}}{\int_{0.25}^{1} \left(\frac{3}{2} y^2 + y   \right) dy} = \frac{104}{123}$
\end{enumerate}

\item
\begin{enumerate}
\item[$\alpha$)] $\P(\text{кисточка окажется на слоне}) = \frac{1}{1.5} = \frac{2}{3}$
\item[$\beta$)] $f_{\xi, \eta}(x, y) = \frac{1}{1.5}$
\item[$\gamma$)] $f_{\xi} (x) = \int_{0}^{1} \frac{1}{1.5} dy = 1.5$

$f_{\eta}(y) = \int_{0}^{1.5} \frac{1}{1.5} dx = 1$
\item[$\delta$)] Да, поскольку $ f_{\xi} (x) \cdot f_{\eta}(y) = f_{\xi, \eta}(x, y)$
\item[$\epsilon$)] $f_{\xi+\eta} (t) = \int_{-\infty}^{+\infty} f_{\xi}(u) f_{\eta}(t-u) du $
\end{enumerate}
\end{enumerate}



\subsection[2014-2015]{\hyperref[sec:kr_01_2014_2015]{2014-2015}}
\label{sec:sol_kr_01_2014_2015}



\begin{enumerate}
\item Внимательно читайте примечание! Всего 6 возможных ситуаций, только 1 — благоприятная.
Требуемая вероятность равна $1/6$.

\item
Два события $A$ и $B$ независимы, если: $\P(AB) = \P(A) \P(B)$.

Проверим, независимы ли события $A = \{ \xi < 1/2 \} $ и  $B = \{ \eta < 1/2 \} $:

$\P(AB)$ ищется как отношение площади квадрата с вершинами в $(0,\,0)$, $(0,\,1/2)$,
$(1/2,\,1/2)$, $(1/2,\,0)$ к площади данного треугольника, то есть:
\[
\P(AB) = \frac{(1/2)^2}{1/2}= \frac{1}{2}
\]

$\P(A)$ ищется как отношение площади трапеции с вершинами в $(0,\,0)$, $(0,\,1)$,
$(1/2,\,1/2)$, $(1/2,\,0)$ к площади данного треугольника, то есть:
\[
\P(A) = \frac{(1/2)\cdot (3/2) \cdot (1/2)}{1/2}= \frac{3}{4}
\]

$\P(B)$ ищется как отношение площади трапеции с вершинами в $(0,\,0)$, $(1,\,0)$,
$(1/2,\,1/2)$, $(0,\,1/2)$ к площади данного треугольника, то есть:
\[
\P(B) = \frac{(1/2)\cdot (3/2) \cdot (1/2)}{1/2}= \frac{3}{4}
\]

\[
\P(A)\cdot \P(B) = \frac{3}{4} \cdot  \frac{3}{4} = \frac{9}{16} \ne \frac{1}{2} = \P(AB)
\]

Получается, события $A$ и $B$ зависимы.

\item
Пусть событие $A$ = \{Цель была поражена первым самолетом\},
событие $B$ = \{Цель была поражена только одним самолетом\}.
Тогда событие $AB$ = \{Первый самолет поразил цель, второй и третий — промахнулись\}.
По формуле условной вероятности:

\[\P(A|B) = \frac{\P(AB)}{\P(B)} = \frac{0.6 \cdot 0.6 \cdot 0.7}{0.6\cdot 0.6 \cdot 0.7 + 0.4 \cdot 0.4 \cdot 0.7 + 0.4 \cdot 0.6 \cdot 0.3} = \frac{0.252}{0.436} \approx 0.578\]

\item
Удобно рассуждать следующим образом: предположим, что каждая опечатка наугад
(с равными вероятностями и независимо от других опечаток) выбирает, на какую
страницу ей попасть.

\begin{enumerate}
\item Пусть $X$ — число опечаток на 13 странице. \[\P(X \geqslant 2) = 1 - \P(X=0) - \P(X=1) \]
$\P(X=0) = \left( \frac{499}{500} \right)^{400}$ — каждая из 400 опечаток не должна попасть на 13 страницу.\\
$\P(X=1) = 400\cdot\frac{1}{500}\cdot\left( \frac{499}{500} \right)^{399}$ — ровно одна опечатка (а есть 400 вариантов) должна попасть на 13 страницу, а остальные — мимо. Соответственно:
\[
\P(X \geqslant 2) = 1 - \left( \frac{499}{500} \right)^{400} - 400\cdot\frac{1}{500}\cdot\left( \frac{499}{500} \right)^{399} \approx 0.19
\]
Это если считать в явном виде. А если пользоваться приближением Пуассона:
\[
p(k) = \P(X = k) = \frac{\lambda^k}{k!}e^{-\lambda}
\]
неплохо бы вспомнить, что параметр $\lambda$ это математическое ожидание $X$, поэтому расчеты здесь пока оставим до лучших времен.

\item Пусть $X$ — число опечаток на 13 странице. Введем случайную величину
\[X_i =
\begin{cases}
1, & \text{если } i\text{-ая опечатка попала на 13 страницу}\\
0, & \text{если нет}
\end{cases}
\]
Тогда $X = \sum\limits_{i=1}^{400}X_i$. Рассмотрим отдельно $X_i$:

\begin{center}
\begin{tabular}{@{}ccc@{}}
\toprule
$x$         & $1$             & $0$               \\ \midrule
$\P(X=x)$ & $\frac{1}{500}$ & $\frac{499}{500}$ \\ \bottomrule
\end{tabular}
\end{center}

Так как $i$-ая опечатка наугад выбирает одну страницу из 500 и это должна быть именно 13.

Тогда:
\begin{align*}
\E(X_i) &= \frac{1}{500} = \E(X^2_i)  \\
\Var(X_i) &= \E(X^2_i) - (\E(X_i))^2 = \frac{1}{500} - \left(\frac{1}{500}\right)^2 = \frac{499}{500^2}
\end{align*}
Значит
\begin{align*}
\E(X) &= \E\left(\sum\limits_{i=1}^{400}X_i\right) = \sum\limits_{i=1}^{400}\E(X_i)  = \frac{400}{500} = 0.8 \\
\Var(X) &= \Var\left(\sum\limits_{i=1}^{400}X_i\right) = \sum\limits_{i=1}^{400}\Var(X_i) = 400\cdot\frac{499}{500^2} = 0.8\cdot\frac{499}{500}
\end{align*}

Теперь мы знаем, что $\lambda = \E(X) = 0.8$ поэтому можем вернуться к пункту (а):
\[
\P(X \geqslant 2) = 1 - \P(X=0) - \P(X=1)  = 1 - \frac{0.8^0}{0!}e^{-0.8} - \frac{0.8^1}{1!}e^{-0.8} \approx 0.19
\]

Осталось найти наиболее вероятное число опечаток на 13 странице:
\[
\P(X=k) = \frac{0.8^k}{k!}e^{-0.8} \rightarrow \max \limits_k
\]
Очевидно, что эта функция убывает по $k$, ведь с ростом $k$:\\
 $k!$ растет, а $0.8^k$ убывает. Значит наиболее вероятное число ошибок — $X = 0$

\item \href{https://en.wikipedia.org/wiki/Triskaidekaphobia}{Ох уж эти предрассудки!}
13-я страница точно такая же как и все остальные, ведь везде в решении можно просто заменить номер 13 на любой другой и ничего не изменится.

\end{enumerate}

\item
Пусть событие $A$ означает, что медицинский тест показал наличие заболевания.
Событие $B$ — заболевание на самом деле есть.

Перепишем условие задачи:

Чувствительность теста $=\P(A |B)$

Специфичность теста $=\P(A^c | B^c)$

Прогностическая сила теста $=\P(B | A)$

$\P(B) = 0.01 \Rightarrow \P(B^c) = 0.99 $

По условию, чувствительность теста равна $0.9$, тогда из формулы условной вероятности:
\[
\P(A | B) = \frac{\P(A \cap B)}{\P(B)} \Rightarrow
\P(A \cap B) = 0.9 \cdot 0.01 = 0.009
\]

При этом очевидно, что:
\[
\P(B) = \P(A \cap B) +  \P(A^c \cap B) \Rightarrow
\P(A^c \cap B) = 0.01 - 0.009 = 0.001
\]

По условию специфичность теста равна 0.95, тогда из формулы условной вероятности:
\[
\P(A^c | B^c) = \frac{\P(A^c \cap B^c)}{\P(B^c)} \Rightarrow
\P(A^c \cap B^c) =0.95 \cdot 0.99 = 0.9405
\]

При этом очевидно, что:
\[
\P(B^c) = \P(A \cap B^c) + \P(A^c \cap B^c) \Rightarrow
\P(A \cap B^c) = 0.99 - 0.9405 = 0.0495
\]

Теперь мы готовы отвечать на заданные вопросы:

\begin{enumerate}
\item
\[
\P(A) = \P(A \cap B^c) + \P(A \cap B) = 0.009+0.0495 = 0.0585
\]

\item Прогностическая сила теста:

\[
\P(B | A) = \frac{\P(A \cap B)}{\P(A) } = \frac{0.009}{0.0585} \approx 0.154
\]

Для того, чтобы повысить прогностическую силу теста, необходимо понизить
$\P(A \cap B^c) $, а для этого необходимо повысить специфичность теста.
\end{enumerate}

\item
\begin{enumerate}
\item
Должно выполняться условие нормировки:

\begin{align*}
& \int \limits_{-a}^0 1.5(x+a)^2 dx + \int \limits_0^a 1.5(x- a)^2  dx = 1   \\
& \left. 0.5(x+a)^3 \right|_{-a}^0 + \left. 0.5(x- a)^3 \right|_0^a  = 1  \\
& 0.5a^3 + 0.5a^3 = 1 \Rightarrow a = 1
\end{align*}

Теперь легко понять, как выглядит функция распределения (смотри определение функции распределения):

\[
F(x) = \begin{cases}
0, & x < 1 \\
0.5 (x+1)^3, & -1 \leqslant x <0 \\
1 + 0.5 (x-1)^3, & 0 \leqslant x < 1 \\
1, & x \geqslant 1
\end{cases}
\]

И с её помощью всё посчитать:
\begin{align*}
& P\left(X \in \left[\frac{1}{2}, 2 \right]  \right) = F(2) - F\left(\frac{1}{2} \right) =
1 - 1 +0.5^4 = 0.5^4
\end{align*}
\begin{align*}
\E(X) &= \int \limits_{-1}^0 x \cdot 1.5 (x + 1)^2 dx +  \int \limits_0^1 x \cdot 1.5 (x - 1)^2 dx \\
& = 1.5 \int \limits_{-1}^0\left( x^3 + 2x^2 + x\right) dx + 1.5 \int \limits_0^1\left( x^3 -2x^2 + x\right) dx \\
& =  \frac{3}{8} x^4 |_{-1}^0 + x^3 |_{-1}^0 + \frac{3}{4} x^2|_{-1}^0+    \frac{3}{8} x^4 |_0^1   - x^3 |_0^1 + \frac{3}{4} x^2|_0^1  = - \frac{3}{8}  + 1- \frac{3}{4} + \frac{3}{8} - 1 +\frac{3}{4} = 0
\end{align*}
А можно было заметить, что функция плотности — четная функция, поэтому сразу $\E(X) = 0$

Вычислим $\E\left(X^2\right)$:

\begin{align*}
\E(X^2) &= \int \limits_{-1}^0 x^2 \cdot 1.5 (x + 1)^2 dx +  \int \limits_0^1 x^2 \cdot 1.5 (x - 1)^2 dx \\
&= 1.5 \int \limits_{-1}^0\left( x^4 + 2x^3 + x^2\right) dx + 1.5 \int \limits_0^1\left( x^4 -2x^3 + x^2\right) dx \\
& =  \frac{3}{10} x^5 |_{-1}^0 + \frac{3}{4} x^4|_{-1}^0 + \frac{1}{2} x^3 |_{-1}^0 +  \frac{3}{10} x^5 |_0^1 - \frac{3}{4} x^4|_0^1  + \frac{1}{2} x^3 |_0^1 =  \frac{1}{10} \\
\end{align*}
Тогда:
\begin{align*}
&\Var(X) = \E(X^2) - (\E(X))^2 = 0.1
\end{align*}

\item Верим, что график $F(x)$, выписанной выше, вы построить можете :)
\end{enumerate}
\item
Пусть $A = \{\text{«Лекция полезна»}\}$, $B = \{\text{«Лекция интересна»}\}$. Заметим, что лекции вообще независимы друг от друга.

\begin{enumerate}
\item Пусть $X_A$ — число полезных лекций, прослушанных Васей,  $X_B$ — число интересных лекций, прослушанных Васей. Введем случайную величину:
\[X_i =
\begin{cases}
1 & \text{если } i\text{-ая лекция была полезна}\\
0 & \text{если нет}
\end{cases}
\]

Тогда $X_A = \sum\limits_{i=1}^{30}X_i$. Рассмотрим отдельно $X_i$:

\begin{center}
\begin{tabular}{@{}ccc@{}}
\toprule
$x$         & $1$             & $0$               \\
$\P(X=x)$ & $0.9$ & $0.1$ \\ \bottomrule
\end{tabular}
\end{center}

Вероятность $0.9$ дана. Тогда:
\begin{align*}
\E(X_i) &= 0.9 = \E(X^2_i) \Rightarrow \\
\Var(X_i) &= \E(X^2_i) - (\E(X_i))^2 = 0.9 - 0.9^2 = 0.09
\end{align*}

Значит
\begin{align*}
\E(X_A) &= \E\left(\sum\limits_{i=1}^{30}X_i\right) = \sum\limits_{i=1}^{30}\E(X_i)  = 0.9\cdot30 = 27 \\
\Var(X_A) &= \Var\left(\sum\limits_{i=1}^{30}X_i\right) = \sum\limits_{i=1}^{30}\Var(X_i) = 0.09\cdot30 = 2.7
\end{align*}

Аналогично для числа интересных лекций можем получить:
\begin{align*}
\E(X_B) &= 0.7\cdot 30 = 21 \\
\Var(X_A) &= 0.21\cdot 30 = 6.3
\end{align*}


\item Так как интересность и полезность — независимые свойства лекций, то:
\[
\P(A^c \cap B^c) = \P(A^c)\cdot \P(B^c) = 0.3\cdot0.1 = 0.03,
\]
где $A^c$ значит «не $A$».
В свою очередь:
\[
\P(A\cup B) = \P(A\cap B^c) + \P(B\cap A^c) + \P(A\cap B) = 1 - \P(A^c)\cdot \P(B^c) = 0.97,
\]
где $(A\cup B)$ значит «$A$ или $B$», а $(A\cap B)$ — «$A$ и $B$».
Аналогично, путем введения бинарной случайной величины можем получить:
\begin{align*}
& \E(X_{A^c \cap B^c}) = 0.03 \cdot  30 = 0.9 \\
& \E(X_{A\cup B}) = 0.97\cdot30 = 29.1
\end{align*}
\end{enumerate}

\item
Дано: $\E(X) = 1$, $\E(Y) = 2$, $\E(X^2) = 5$, $\E(Y^2) = 8$, $\E(XY) = -1$.

Будем использовать только свойства математического ожидания, ковариации и дисперсии, и ничего больше. Ни-че-го.

\begin{itemize}
\item  $\E(2X + Y - 4) = 2\E(X) + \E(Y) + \E(-4) = 2 + 2 - 4 = 0 $
\item $\Var(X) = \E(X^2) - (\E(X))^2 = 5 - 1 = 4 $
\item $\Var(Y) = \E(Y^2) - (\E(Y))^2 = 8 - 4 = 4 $
\item $\Cov(X, Y) = \E(XY) - \E(X)\E(Y) = -1 - 2 = -3$
\item $\Corr(X, Y) = \frac{\Cov(X, Y)}{\sqrt{\Var(X)}\sqrt{\Var(Y)}} = -\frac{3}{2\cdot 2} = -0.75$
\item $\Var(X-Y-1) = \Var(X) + \Var(Y) - 2\Cov(X, Y) = 4+4 -2(-3) = 14$
\item $\Var(X+Y+1) = \Var(X) + \Var(Y) + 2\Cov(X, Y) = 4+4+2(-3) =2 $
\item \begin{align*}
\Cov(X-Y-1, X+Y+1)&=\E((X-Y)(X+Y))-\E(X-Y)\E(X+Y) \\
&= \E\left(X^2-Y^2\right) - (\E(X)-\E(Y))(\E(X) + \E(Y)) \\
&= \E\left(X^2\right) - \E\left(Y^2\right) - \left((\E(X))^2 -(\E(Y))^2\right)\\
&= \Var(X)-\Var(Y) = 0
\end{align*}
 \item $\Cov(X-Y-1, X+Y+1)=0 \Rightarrow \Corr(X-Y-1, X+Y+1) = 0 $
\end{itemize}

\item Найдём частные распределения $Y$ и $Y^2$:
\begin{center}
\begin{tabular}{cccc}
\toprule
 & $X=1$ & $X=2$ & $\sum$ \\ \midrule
$Y=-1$ & $0.1$ & $0.2$ & $0.3$ \\
$Y=0$ & $0.2$ & $0.3$ & $0.5$ \\
$Y=1$ & $0$ & $0.2$ & $0.2$ \\
$\sum$ & $0.3$ & $0.7$ & \\ \bottomrule
\end{tabular}
\end{center}

\begin{center}
\begin{tabular}{@{}cccc@{}}
\toprule
$y$         & $-1$             & $0$      & $1$         \\
$\P(Y=y)$ & $0.3$ & $0.5$  & $0.2$\\ \bottomrule
\end{tabular}
\end{center}

Так как $Y^2$ может принимать только значения 0 или 1:

\begin{center}
\begin{tabular}{@{}ccc@{}}
\toprule
$y^2$         & $0$             & $1$               \\
$\P(Y^2 = y^2)$ & $0.5$ & $0.5$ \\ \bottomrule
\end{tabular}
\end{center}
А ковариация:
\begin{align*}
\Cov(X, Y) &= \E(XY) - \E(X)\E(Y) =
((-1)\cdot 1\cdot0.1 + (-1)\cdot2 \cdot 0.2 + 1\cdot2\cdot 0.2) \\
&- (0.3\cdot1 + 0.7 \cdot 2)\cdot(0.3\cdot(-1) + 0.1\cdot 0.2) = 0.07
\end{align*}

Так как $\Cov(X, Y) \ne 0$ — величины зависимы

\item Бонусная задача

Предположим, что правильный ответ 0.25. Но это невозможно, потому что вариантов ответа 0.25 — два (1 и 4), значит ответ 0.5 тоже был бы правильный. Предположим, что правильный 0.5. Тогда 0.25 тоже правильный — таких вариантов два из четырех, значит вероятность попасть в 0.25, выбрав ответ наугад, равна 0.5. Ответ 0.6, очевидно, неверен, потому что вероятность попасть в него равна 0.25. \\
\textbf{Правильный ответ:} 0
\end{enumerate}



\subsection[2013-2014]{\hyperref[sec:kr_01_2013_2014]{2013-2014}}
\label{sec:sol_kr_01_2013_2014}

\begin{enumerate}
\item Введём обозначения:
\begin{itemize}
\item $\P(\text{В} | \text{A}^{c} \cap \text{М}^{c}) = 0.18$ — Вася пришёл, а девушки — нет
\item $\P(\text{В} | \text{A} \cap \text{М}) = 0.9$ — пришли и Вася, и девушки
\item $\P(\text{В} | \text{A}^{c} \cap \text{М}) = 0.54$ — Вася пришёл, если пришла только Маша
\item $\P(\text{В} | \text{A} \cap \text{М}^{c}) = 0.36$ — Вася пришёл, если пришла только Алёна
\item $\P(\text{М}) = 0.4$ — Маша пришла на лекцию
\item $\P(\text{А}) = 0.6$ — Алёна пришла на лекцию
\end{itemize}
\begin{enumerate}
\item Используя формулы Байеса и полной вероятности, получим:
\[
\P(\text{A} | \text{В} ) = \frac{\P(\text{A} \cap \text{В})}{\P(\text{В})}
\]
В числителе:
\begin{align*}
\P(\text{В} | \text{A}) \cdot \P(\text{А}) &= P(\text{В} | \text{A} \cap \text{М}) \cdot \P(\text{А}) \cdot \P(\text{М}) + \P(\text{В} | \text{A} \cap \text{М}^{c}) \cdot \P(\text{А}) = \cdot \P(\text{М}^{c}) \\
&= 0.9 \cdot 0.4 \cdot 0.6 + 0.36 \cdot 0.6 \cdot 0.6 = 0.3456
\end{align*}
А в знаменателе:
\begin{align*}
\P(\text{В}) &=
\P(\text{В} | \text{A}^{c} \cap \text{М}^{c}) \cdot \P(\text{A}^{c} \cap \text{М}^{c})+\P(\text{В} | \text{A} \cap \text{М}) \cdot \P(\text{A} \cap \text{М}) + \P(\text{В} | \text{A}^{c} \cap \text{М}) \cdot \P(\text{A}^{c} \cap \text{М}) \\
&+  \P(\text{В} | \text{A} \cap \text{М}^{c}) \cdot \P(\text{A} \cap \text{М}^{c}) \\
&= 0.18 \cdot 0.6 \cdot 0.4 + 0.9 \cdot 0.4 \cdot 0.6 + 0.54 \cdot 0.4 \cdot 0.4 + 0.36 \cdot 0.6 \cdot 0.6 = 0.4752
\end{align*}
Ответ:
\[
\P(\text{A} | \text{В} ) = \frac{\P(\text{A} \cap \text{В})}{\P(\text{В})} = \frac{0.3456}{0.4752}  =0.(72)
\]

\item Необходимо найти
\[
\P(\text{М} | \text{В}) = \frac{\P(\text{М} \cap \text{В})}{\P(\text{В})}
\]
Знаменатель этой дроби посчитан в предыдущем пункте, посчитаем числитель:
\begin{align*}
\P(\text{М} \cap \text{В}) &= \P(\text{В} | \text{М}) \cdot \P(\text{М}) \\
&= P(\text{В} | \text{М} \cap \text{А}) \cdot \P(\text{А}) \cdot \P(\text{М}) + \P(\text{В} | \text{A}^{c} \cap \text{М}) \cdot \P(\text{А}^{c})  \cdot \P(\text{М}) \\
&= 0.9 \cdot 0.4 \cdot 0.6 + 0.54 \cdot 0.4 \cdot 0.4 = 0.3024
\end{align*}
Ответ:
\[
\P(\text{М} | \text{В}) = \frac{\P(\text{М} \cap \text{В})}{\P(\text{В})} = \frac{0.3024}{0.4752} = 0.(63)
\]
Если Вася на лекции, вероятность застать на ней Алёну выше.
\end{enumerate}


\item $\P(X = 5) = C_{100}^5 0.002^5 0.998^{95}$,

$\E(X) = 0.2$,

$\Var(X) = 0.2\cdot 0.998$,

наиболее вероятно событие $X = 0$.
\item $c = 1/2$,
$\P(X \in [\ln 0.5,\ln 4]) = 5/8$,
$\E(X) = 0$,
$\Var(X)=2$,
$\E(X^{2k+1})=0$,
$\E(X^{2k})=(2k)!$
\item
\begin{enumerate}
\item $\E(Y - 2X - 3) = \E(Y) - 2 \E(X) - 3 = 0$

$\Var(Y - 2X - 3) = \Var(Y) + 4\Var(X) - 2\Cov(Y, 2X) = 16$

$\Cov(X, Y) = \Corr(X,Y) \cdot \sqrt{\Var(X) \cdot \Var(Y)} = 6$
\item $\Corr(Y - 2X - 3, X) = \frac{\Cov(Y, X) - 2 \Var(X)}{\sqrt{\Var(Y - 2X - 3) \cdot \Var(X)}} = -1$.
\item Корреляция равна $-1$, значит, есть линейная взаимосвязь между переменными.
Пусть $Y+ a X = b$, тогда $\Var(Y+ a X)=0$, $\E(Y) = -a + b =1 $.
Решая уравнения, находим, что $a=-2/3, b=1/3$.
\end{enumerate}

\item \begin{enumerate}
\item Таблицы распределения имеют вид:
\begin{center}
\begin{tabular}{@{}cccc@{}}
\toprule
$x$         & $-1$  & $0$   & $1$   \\ \midrule
$\P(X=x)$ & $0.3$ & $0.3$ & $0.4$ \\ \bottomrule
\end{tabular}
\hspace{1cm}
\begin{tabular}{@{}ccc@{}}
\toprule
$y$         & $-1$  & $1$   \\ \midrule
$\P(Y=y)$ & $0.5$ & $0.5$ \\ \bottomrule
\end{tabular}
\end{center}

\item
\begin{multline*}
\Cov(X, Y) = \E(XY) - \E(X) \E(Y)  = (-1)\cdot (-1) \cdot 0.1 + (-1) \cdot 0 \cdot 0.2 + \\
+ (-1) \cdot 1 \cdot 0.2 + 1 \cdot (-1) \cdot 0.2 + 1 \cdot 0 \cdot 0.1 + 1 \cdot 1 \cdot 0.1 -
0.1 \cdot 0 = -0.1
\end{multline*}
\item Да, поскольку если случайные величины независимы, то их ковариция равна нулю.
\item Условное распределение:
\begin{center}
\begin{tabular}{@{}cccc@{}}
\toprule
$X|Y=-1$    & $-1$  & $0$   & $1$   \\ \midrule
$\P(\cdot)$ & $0.2$ & $0.4$ & $0.4$ \\ \bottomrule
\end{tabular}
\end{center}
\item $\E(X | Y = -1) = -1 \cdot 0.2 + 0 \cdot 0.4 + 1 \cdot 0.4 = 0.2$
\end{enumerate}
\end{enumerate}




\subsection[2012-2013]{\hyperref[sec:kr_01_2012_2013]{2012-2013}}
\label{sec:sol_kr_01_2012_2013}

\begin{enumerate}
\item
\begin{enumerate}
\item $\P(A)=0.8\cdot 0.3+0.7\cdot 0.2=0.38$
\item $\P(B)=0.9$
\item $\P(C|A)=\frac{0.3\cdot 0.8}{0.38}=0.632$
\item $\P(C|D)=\frac{0.3\cdot (0.9\cdot 0.8+0.1\cdot 0.2)}{0.9\cdot 0.38+0.1\cdot (1-0.38)}=0.55$
\end{enumerate}
\item Это была задачка-неберучка!
\item
\begin{enumerate}
\item $1$
\item $\E(X)=45/28\approx 1.61$, $\E(X^2)=93/35\approx 2.66$, $\Var(X)=291/3920\approx 0.07$
\item $37/56\approx 0.66$
\item $F(x)=\begin{cases} 0,\, x<1 \\
\frac{x^3-1}{7},\, x\in [1;2] \\
1,\, x>1 \end{cases}$
\end{enumerate}
\item
\begin{enumerate}
\item $a=0.1$
\item $\P(X>-1)=0.7$, $\P(X>Y)=0.1$
\item $\E(X)=-0.2$, $\E(X^2)=2$
\item $\Corr(X,Y)=0.117$
\end{enumerate}
\item
\begin{enumerate}
\item Правильные: $\E(X)=10$, $\Var(X)=9$, неправильные: $\E(Y)=9$, $\Var(Y)=0.9$
\item Наиболее вероятное число укусов равно математическому ожиданию
\item Лучше идти к неправильным пчёлам, так как $\P(X\leq 2)<\P(Y\leq 2)$.
\end{enumerate}
\end{enumerate}
