\subsection*{Решения контрольной номер 1}

\subsubsection*{\hyperref[sec:kr_01_2017_2018]{2017-2018}}
\label{sec:sol_kr_01_2017_2018}

\begin{enumerate}
\item
\begin{enumerate}
\item События называются независимыми, если  $ \P(A \cap B) = \P(A) \cdot \P(B)$
\item Запасёмся всеми нужными вероятностями:

$\P(A) = \frac{1}{2}$

$\P(B) = \frac{1}{3}$

$\P(C) = \frac{1}{2}$

$\P(A \cap C) = \frac{1}{3} $ — выпадет чётое число больше трёх

$\P(A \cap B)  = \frac{1}{6}$ — выпадет чётное число, кратное трём

$\P(A \cap C) = \frac{1}{6}$ — выпадет число, большее трёх и кратное трём

Теперь можно проверять независимость:

$\P(A \cap C) \neq \P(A) \cdot \P(C) \Rightarrow$  не являются независимыми

$ \P(A \cap B) = \P(A) \cdot \P(B) \Rightarrow$ являются независимыми

$ \P(B \cap C) = \P(B) \cdot \P(C) \Rightarrow$ являются независимыми

\end{enumerate}
\item
\begin{enumerate}
\item Количество возможных вариантов ТМ: $ C_{10}^2 $,  количество возможных вариантов ЗМ: $ C_{24}^2 $. Количество их возможных сочетаний: $ C_{10}^2 \cdot C_{24}^2$ , где $ C_n^k = \frac{n!}{k!(n-k)!}$.
\item По классическому определению вероятностей, предполагая исходы равновероятными, искомая вероятность равна $ \frac{C_{16}^2}{C_{24}^2} $
\item По тому же принципу:
\[
\frac{C_k^2}{C_{10}^2} = \frac{1}{15} \Rightarrow \frac{\frac{k!}{2!(k-2)!}}{\frac{10!}{2! \cdot 8!}} = \frac{1}{15} \Rightarrow \frac{(k-1)k}{2}\frac{ 2}{9 \cdot 10} = \frac{1}{15}
\]
Получаем квадратное уравнение вида $ k^2 - k - 6 = 0 $ с корнями $-2$ и $3$. Так как $k$ не может быть отрицательным, ответ $3$.
\end{enumerate}
\item
\begin{enumerate}
\item Если эксперт отдаёт предпочтение Fit, то это можно интерпретировать как «успех» в схеме Бернулли. Так как $\xi$ - количество успехов, $ k \in [0;4]$, $p = \frac{1}{3} $, то
\[
\P(\xi = k) = C_n^k(p)^k(1-p)^{n-k}
\]

Большинство означает, что либо три, либо четыре эксперта выбрали Fit.
\[
\P(\xi = 3) = C_4^3\left(\frac{1}{3}\right)^3 \left(\frac{2}{3}\right)^{1} = \frac{8}{81}
\]
\[
\P(\xi = 4) = C_4^4\left(\frac{1}{3}\right)^4 \left(\frac{2}{3}\right)^{0} = \frac{1}{81}
\]
\[
\P( \xi > 2) =  \frac{9}{81}
\]
\item Аналогично:

\[ \P(\xi = 0) = C_4^0\left(\frac{1}{3}\right)^0 \left(\frac{2}{3}\right)^{4} = \frac{16}{81}\]

\[ \P(\xi = 1) = C_4^1\left(\frac{1}{3}\right)^1 \left(\frac{2}{3}\right)^{3} = \frac{32}{81}\]

\[ \P(\xi = 2) = C_4^2\left(\frac{1}{3}\right)^2 \left(\frac{2}{3}\right)^{2} = \frac{24}{81}\]

\begin{figure}[h!]
    \noindent\centering{
    \includegraphics[width=80mm]{images/kr1_2017_3.png}
    }
    \caption{Функция распределения}
    \label{cdf_kr2017}
\end{figure}

\item Все вероятности посчитаны, видим, что наибольшая достигается при $\xi=1$.
\item $\E(X) = np = \frac{4}{3} $, $ \Var(X) = npq = \frac{8}{9}$
\end{enumerate}
\item
\begin{enumerate}
\item Так как указано, что цена сметаны распределена равномерно на отерзке $[250, 1000]$, максимальное значение цены — $1000$, это и есть необходимая сумма.
\item Вспомним, что функция распределения $F(x) = \P(X \leq x)$, нужно найти такой $x$, что $ \P(X \leq x)=0.9$:
\[
0.9 = 1 - \exp({-x^{2}}) \Rightarrow \exp(-x^{2}) = 0.1 \Rightarrow -x^2 = \ln(0.1)  \Rightarrow x=  \sqrt{-\ln(0.1)}
\]
\item Взяв производную от функции распределения списка без сметаны, получим функцию плотности:
\[
f_X(x) =
\begin{cases}
2x\exp(-x^2) & x \ge 0 \\
0 & \text{иначе}
\end{cases}
\]
Найдём математическое ожидание:
\[
\int_{0}^{+\infty}2x^2\exp({-x^2}) dx = -x \exp({-x^2})\big|_0^{+\infty} + \int_{0}^{+\infty}\exp({-x^2}) dx = \frac{\sqrt{\pi}}{2}
\]
\item Математическое ожидание суммы случайных величин равно сумме математических ожиданий случайных влечин, если они существуют. Математическое ожидание от цены сметаны равно: $ \frac{1000 + 250}{2} = 625 $
Математическое ожидание списка без сметаны было найдено в предыдущем пункте, его осталось перевести в рубли. Получаем ответ: $ 625 + \frac{\sqrt{\pi}}{2} \cdot 1000 $.
\item Так как обе величины имеют абсолютно непрерывные распределения, вероятность попасть в конкретную точку равна нулю.
\end{enumerate}
\item
\begin{enumerate}
\item $\P(\text{детектор показал ложь и подозреваемый лжёт}) = 0.9 \cdot 0.1 + 0.1 \cdot 0.95 = 0.185$
\item $\P(\text{невиновен}|\text{детектор показал ложь}) = \frac{0.9\cdot0.1}{0.185} = \frac{90}{185}$
\item $\P(\text{эксперт точно выявит преступника}) = (0.9)^9 \cdot 0.95$
\item $\P(\text{эксперт ошибочно выявит преступника}) = 9 \cdot 0.1 \cdot 0.9^8\cdot 0.05$
\end{enumerate}
\end{enumerate}


\subsubsection*{\hyperref[sec:kr_01_2016_2017]{2016-2017}}
\label{sec:sol_kr_01_2016_2017}

\begin{enumerate}
\item
\begin{enumerate}
\item Возможны четыре равновероятные ситуации:
\[
\P(\text{ММ}) = \P(\text{МД}) = \P(\text{ДМ}) = \P(\text{ДД}) = 1/4
\]

Посчитаем условную вероятность:
\[
\P(B \mid A) = \frac{\P(B \cap A)}{\P(A)} = \frac{\P(\text{МД, ДМ})}{\P(\text{ДМ, МД, ДД})} = \frac{2/4}{3/4} = \frac{2}{3}
\]

\item События $A$ и $B$ называются независимыми, если $\P(A \cap B) = \P(A) \cdot \P(B)$

В нашем случае: $\P(A \cap B) = \P (\text{МД, ДМ}) = 2/4$,
$\P(A) \cdot \P (B) = 3/4 \cdot 3/4$.

Следовательно, $\P(A \cap B) \neq \P(A) \cdot \P (B)$,
значит, события $A$ и $B$ не являются независимыми.
\end{enumerate}

\item Пусть событие $A_i$ означает, что $i$-ый узел системы дал сбой,
а событие $B_N$, что вся система дала сбой.

В условии сказано, что $\P(A_i) = 10^{-6}$,
а найти нужно такое максимальное $N \in \mathbb{N}$, при котором

\[
\P(B_N) \leq \frac{1}{10^2}
\]

\begin{align*}
\P(B_N) &= \P\left(\cup_{i=1}^n A_i\right) = 1 - \P (\left(\cup_{i=1}^n A_i\right)^c) \\
&\stackrel{\text{ф-ла де Моргана}}{=} 1 - \P \left(\cup_{i=1}^N A_i^c\right) \stackrel{A_1, \ldots, A_N \text{– независ.}}{=} 1 - \P(A_1^c) \cdot \ldots \cdot \P(A_N^c) \\
&= 1 - \left(1-\frac{1}{10^6}\right)^N
\end{align*}
Чтобы найти такое максимальное $N \in \mathbb{N}$, надо решить следующее неравенство
\begin{align*}
& 1 - \left(1-10^{-6}\right)^N \leq 10^{-2} \\
& 1 - 10^{-2} \leq \left(1-10^{-6}\right)^N \\
& \ln\left(1 - 10^{-2}\right) \leq N \ln \left(1 - 10^{-6}\right) \\
& N \leq \frac{\ln\left(1 - 10^{-2}\right)}{ \ln \left(1 - 10^{-6}\right)} \approx 10050.33
\end{align*}
Значит, максимальное $N$ равно $10050$.

\item Введём обозначения для событий.
Пусть $A$ означает, что человек имеет заболевание лёгких,
а $B$, что человек работал в шахте.

В условии сказано, что $\P(B \mid A) = 0.22$, $\P(B \mid A^c) = 0.14$, $\P(A) = 0.04$.
\begin{enumerate}
\item Нужно найти
\[
\P(A \mid B) = \frac{\P(A\cap B)}{\P (B)} = \frac{\P(B|A)\P(A)}{\P(B)}
\]
Для этого с помощью формулы полной вероятности посчитаем
\[
\P (B) = \P (B \mid A) \P(A) + \P (B \mid A^c) \P (A^c) = 0.22 \cdot 0.04 + 0.14 \cdot 0.96 = 0.1432
\]
Осталось подставить значения:
\[
\P(A \mid B) = \frac{0.22 \cdot 0.04}{0.1432} \approx 0.0615
\]

\item Все необходимые значения для второго пункта у нас есть,
осталось применить формулу условной вероятности:
\begin{multline*}
\P  (A \mid B^c) =  \frac{\P(A\cap B^c)}{\P (B^c)} =  \frac{\P (B^c \cap A)}{\P(A)} \cdot \frac{\P(A)}{\P (B^c)} = \P (B^c \mid A) \cdot \frac{\P(A)}{\P (B^c)} = \\
= (1-\P (B \mid A)) \cdot \frac{\P(A)}{1-\P (B)} = (1-0.22) \cdot \frac{0.04}{1-0.1432} \approx 0.0364
\end{multline*}
\end{enumerate}
\item Введём индикатор события «Петя дал верный ответ на $i$-ый вопрос»:
\[
X_i =
\begin{cases}
1, & \text{если на } i \text{-ый вопрос теста Петя дал верный ответ} \\
0, & \text{иначе}
\end{cases}
\]

Заметим, что $X_i \sim Be\left(p = 1/5 \right)$, $X_1, \ldots, X_{17}$ – независимы,
$X = X_1 + \ldots + X_{17}$ – общее число верных ответов,
$X \sim Bin\left(n=17, p=1/5\right)$.

\begin{enumerate}
\item Наибольшее вероятное число правильных ответов $m_0$ может быть нвйдено по формуле:
\begin{enumerate}
\item[1)] если число $(n\cdot p - q)$ – не целое, где $q:=1-p$, то
\[
m_0 = [np-q] +1,
\]
\item[2)] если число  $(n\cdot p - q)$ – целое, то наиболее вероятных значений $m_0$ два:
\[
m_0' = np-q \text{ и } m_0'' = np-q+1
\]
\end{enumerate}
Итак, поскольку $np-q = 17\cdot\frac{1}{5} - \frac{4}{5} = 2.6$ – не целое, наиболее вероятное число верных ответов $m_0$ может быть найдено по формуле из пункта (1):
\[
m_0 = [np-q] +1 = [2.6] + 1 = 3
\]
\item \[\E(X) = np = 17 \cdot \frac{1}{5}=3.4\]

\[\Var(X) = npq = 17 \cdot \frac{1}{5} \cdot \frac{4}{5} = 2.72\]

\item
\begin{multline*}
\P (\text{Петя получит «отлично»}) = \P (X\geq 15) = \P (X = 15) + \P (X= 16) + \\
+ \P (X = 17) = C^{15}_{17} \cdot \left(\frac{1}{5}\right)^{15} \cdot \left(\frac{4}{5}\right)^2 + C^{16}_{17} \cdot \left(\frac{1}{5}\right)^{16} \cdot \left(\frac{4}{5}\right)^1 + C^{17}_{17} \cdot \left(\frac{1}{5}\right)^{17} \cdot \left(\frac{4}{5}\right)^0 = \\
= 136 \cdot \frac{16}{5^{17}} + 17 \cdot \frac{4}{5^{17}} + \frac{1}{5^{17}} \approx 2.94 \cdot 10^{-9}
\end{multline*}
\item Рассмотрим первый вопрос теста. Петя может выбрать первый ответ с вероятностью $1/5$, и Вася
может выбрать первый ответ с вероятностью $1/5$. Тогда они оба выберут одинаковый ответ с вероятностью $1/25$.
Вариантов ответа в каждом вопросе $5$, значит, вероятность совпадения ответа в одном вопросе равна $1/5$.
Всего вопросов 17, тогда получаем
\[
\P(\text{все ответы Пети и Васи совпадают}) = \left(\frac{1}{5}\right)^{17}
\]

\end{enumerate}
\item Введём случайную велчину $\eta$, которая означает число потенциальных покупателей, с которыми контактировал продавец оборудования. По условию задачи, $\eta$ имеет таблицу распеределения:
\begin{center}
\begin{tabular}{ccc}
\toprule
$\eta$ & $ 1 $ & $2$ \\ \midrule
$\P_{\eta}$ & $1/3$ & $2/3$ \\ \bottomrule
\end{tabular}
\end{center}
Случайная величина $\xi$ может принимать значения $0, 50000$ и $100000$
\begin{enumerate}

\item Найдём $\P (\xi = 0 )$. По формуле полной вероятности, имеем:
\begin{multline*}
\P (\xi = 0) = \P (\xi = 0 \mid \eta = 1 ) \cdot \P ( \eta = 1 ) + \P (\xi = 0 \mid \eta = 2 )  \cdot \P ( \eta = 2 )  = \\
= 0.9 \cdot \frac{1}{3} + 0.9\cdot0.9 \cdot \frac{2}{3} = 0.84
\end{multline*}

\item Найдём $\P (\xi = 50000 )$ и $\P (\xi = 100000 )$ :
\begin{multline*}
\P (\xi = 50000 ) =  \P (\xi = 50000 \mid \eta = 1 ) \cdot \P ( \eta = 1  ) +  \P (\xi = 50000 \mid \eta = 2 ) \cdot  \P ( \eta = 2 )  = \\
= 0.1 \cdot \frac{1}{3} + 2 \cdot 0.1 \cdot 0.9 \cdot \frac{2}{3} = 0.15(3)
\end{multline*}
\begin{multline*}
\P (\xi = 100000 ) =  \P (\xi = 100000 \mid \eta = 1 ) \cdot \P ( \eta = 1 ) +  \P (\xi = 100000 \mid \eta = 2 ) \cdot  \P ( \eta = 2  )  =  \\
= 0 \cdot \frac{1}{3} + 0.1\cdot 0.1  \cdot \frac{2}{3} = 0.00(6)
\end{multline*}
Таблица распределения случайной величина $\xi$ имеет вид:

\begin{center}
\begin{tabular}{cccc}
\toprule
$\xi$ & $ 0 $ & $5000$ & $100000$ \\ \midrule
$\P_{\xi}$ & $0.84$ & $0.15(3)$ & $0.00(6)$ \\ \bottomrule
\end{tabular}
\end{center}

Тогда функция распределения случайной величины $\xi$ имеет вид:
\[
F_{\xi} (X) =
\begin{cases}
0 & \text{при } x<0 \\
0.84 & \text{при } 0 \leq x < 50000 \\
0.84 + 0.15(3) & \text{при } 50000
\leq x < 100000 \\
1 & \text{при } x > 100000
\end{cases}
\]
Опр.: $F_{\xi} = \P (\xi \leq x ), x \in \mathbb{R}$
\item \[\E (X) = 0 \cdot 0.84 + 50000 \cdot 0.15(3) + 100000 \cdot 0.00(6) = 8333.(3)	\]
\begin{multline*}
\Var(X) = (0 - 8333.(3))^2 \cdot 0.84 + (50000-8333.(3))^2 \cdot 0.15(3) + \\
+ (100000 - 8333.(3))^2 \cdot 0.00(6) = 380555555.(5)
\end{multline*}
\end{enumerate}
\item
\begin{enumerate}
\item $ f_{\xi} (x)=
\begin{cases}
\frac{1}{b} & \text{при } x \in [0, b] \\
0 & \text{при } x \notin [0, b]
\end{cases}
$
\item  Известно, что если $\xi \sim U[a, b]$, то $\E (\xi) = \frac{a+b}{2}$. Стало быть, из уравнения $\E (\xi) = 1$ получаем $\frac{b}{2} = 1$,  то есть $b=2$.
\item Известно, что если $\xi \sim U[a, b]$, то $\Var (\xi) = \frac{(b-a)^2}{12}$. Значит, $\Var (\xi) = \frac{2^2}{12} = \frac{1}{3}$
\item Воспользуемся формулой $\P (\xi \in B ) = \int_B f_{\xi} (x) dx$. Имеем:
\[
\P (\xi > 1 ) = \P (\xi \in (1, + \infty) ) = \int_{1}^{+ \infty} f_{\xi} (x) dx = \int_{1}^{2} \frac{1}{2} dx = \frac{1}{2}
\]
\item Требуется найти такое минимальное число $q_{0.25}$, что $\int_{-\infty}^{q_{0.25}} f_{\xi} (x) dx = 0.25$. Итак:
\[
\int_{-\infty}^{q_{0.25}} f_{\xi} (x) dx = 0.25 \Leftrightarrow \int_{-\infty}^{q_{0.25}} \frac{1}{2} dx = 0.25 \Leftrightarrow \frac{1/2}{q_{0.25}} = 0.25 \Leftrightarrow
\]
\[
q_{0.25} = 2 \cdot 0.25 = 0.5
\]
\item
\begin{multline*}
\E [ (\xi - \E(\xi))^{2017} ] = \int_{-\infty}^{+\infty} (x- \E(\xi) )^{2017} \cdot f_{\xi} (x) dx = \int_{-\infty}^{+\infty} (x-1)^{2017} f_{\xi} (x) dx = \\
= \int_{0}^{2} (x-1)^{2017} \cdot \frac{1}{2} dx = \frac{(x-1)^{2018}}{2018} \cdot \frac{1}{2} \bigg\rvert_{x=0}^{x=2} =0
\end{multline*}
\item $F_{\xi} (x) =
\begin{cases}
0 & \text{при } x < 0 \\
\frac{x}{2} & \text{при } 0 \leq x \leq 2 \\
1 & \text{при } x > 2
\end{cases}
$
\item Согласно условиям задачи, время до прихода 1-го поезда есть $\xi$; время до прихода 2-го поезда равно $\xi + b$; время до прихода 3-го (заветного) поезда есть $\xi + 2b$. Таким образом, Марья Ивановна в среднем ожидает «своего» поезда $\E (\xi + 2b) = 1 + 2b = 1 + 2 \cdot 2 = 5 $ минут. При этом $\Var (\xi + 2b) = \Var (\xi) = 1/3$
\item[к)] Пусть $\tau$ – наименьший номер поезда без «подозрительных лиц». По условию задачи, таблица распределения случайной величины $\tau$ имеет вид:

\begin{center}
\begin{tabular}{cccccc}
\toprule
$\tau$ & $ 1 $ & $2$ & $3$ & $4$ & \ldots \\ \midrule
$\P_{\tau}$ & $1/4$ & $3/4\cdot1/4$ & $(3/4)^2 \cdot 1/4$ & $(3/4)^3 \cdot 1/4$ & \ldots\\ \bottomrule
\end{tabular}
\end{center}

То есть случайная величина $\tau$ имеет геометрическое распределение с параметром $p=1/4$ $(\tau \sim G(p=1/4))$.

Несложно сообразить, что время ожидания Глафирой Петровной «своего» поезда составляет: $\eta := \xi + b(\tau- 1)$. Стало быть, $\E (\eta) = \E (\xi) + b \cdot (\E(\tau)-1)  = 1 + 2 \cdot (4-1) = 7$ минут.

Здесь мы воспользовались тем фактом, что если $\eta \sim G(p)$, то $\E (\eta) = 1/p$
\item[и)] Найдём теперь вероятность $\P (\eta \geq 5 )$. Для нахождения искомой вероятности воспользуемся формулой полной вероятности:
\[
	\P (\eta \geq 5 ) = \P(\eta \geq 5, \tau < 3) +\P(\eta \geq 5, \tau = 3)+\P(\eta \geq 5, \tau > 3)
\]

Если Глафира уехала на первом или втором поезде,
то ждать больше 5 минут она не могла, то есть $\P(\eta \geq 5, \tau <3)=0$.

Если Глафира уехала на третьем поезде, то чтобы ждать больше пяти минут,
ей нужно ждать первый поезд больше минуты,
то есть $\P(\eta \geq 5, \tau = 3)=0.5 \P(\tau = 3)$.

Если Глафира уехала на четвертом поезде или позже, то она точно ждала больше 5 минут,
$\P(\eta \geq 5, \tau >3)=\P(\tau>3)$.

\[
\P(\eta \geq 5) = 0.5\P(\tau = 3) + \P(\tau > 3) = 0.5 \cdot (3/4)^2 \cdot (1/4) + (3/4)^3 = 63 / 128
\]

\end{enumerate}
\item Пусть $\xi$ — случайная величина, обозначающая число остановок лифта. Предствим её в виде суммы $\xi = \xi_2 + \ldots + \xi_{10}$, где $\xi_i$ — индикатор
того, что лифт остановился на $i$-ом этаже, то есть
\[
\xi_i = \begin{cases}
1 & \text{если лифт остановился} \\
0 & \text{иначе}
\end{cases}
\quad \forall i = 2, \ldots, 10
\]
Найдём соответсвующие вероятности:
\[
\P(\xi_i = 0) = \left(\frac{8}{9}\right)^9
\]
\[
\P(\xi_i = 1) = 1 - \P(\xi = 0) = 1 - \left(\frac{8}{9}\right)^9
\]
Тогда $\E(\xi_i) = \P(\xi_i = 0) \cdot 0 + \P(\xi_i = 1) \cdot 1 = 1 - \left(\frac{8}{9}\right)^9$, и в итоге получаем:
\[
\E(\xi) = 9 \cdot \E(\xi_i) = 9 \cdot \left(1 - \left(\frac{8}{9}\right)^9\right)
\]
\end{enumerate}
