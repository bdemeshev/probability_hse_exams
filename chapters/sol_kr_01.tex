\subsection*{Решения контрольной номер 1}

\subsubsection*{\hyperref[sec:kr_01_2017_2018]{2017-2018}}
\label{sec:sol_kr_01_2017_2018}

\begin{enumerate}
\item
\begin{enumerate}
\item События называются независимыми, если  $ \P(A \cap B) = \P(A) \cdot \P(B)$
\item Запасёмся всеми нужными вероятностями:

$\P(A) = \frac{1}{2}$

$\P(B) = \frac{1}{3}$

$\P(C) = \frac{1}{2}$

$\P(A \cap C) = \frac{1}{3} $ — выпадет чётое число больше трёх

$\P(A \cap B)  = \frac{1}{6}$ — выпадет чётное число, кратное трём

$\P(A \cap C) = \frac{1}{6}$ — выпадет число, большее трёх и кратное трём

Теперь можно проверять независимость:

$\P(A \cap C) \neq \P(A) \cdot \P(C) \Rightarrow$  не являются независимыми

$ \P(A \cap B) = \P(A) \cdot \P(B) \Rightarrow$ являются независимыми

$ \P(B \cap C) = \P(B) \cdot \P(C) \Rightarrow$ являются независимыми

\end{enumerate}
\item
\begin{enumerate}
\item Количество возможных вариантов ТМ: $ C_{10}^2 $,  количество возможных вариантов ЗМ: $ C_{24}^2 $. Количество их возможных сочетаний: $ C_{10}^2 \cdot C_{24}^2$ , где $ C_n^k = \frac{n!}{k!(n-k)!}$.
\item По классическому определению вероятностей, предполагая исходы равновероятными, искомая вероятность равна $ \frac{C_{16}^2}{C_{24}^2} $
\item По тому же принципу:
\[
\frac{C_k^2}{C_{10}^2} = \frac{1}{15} \Rightarrow \frac{\frac{k!}{2!(k-2)!}}{\frac{10!}{2! \cdot 8!}} = \frac{1}{15} \Rightarrow \frac{(k-1)k}{2}\frac{ 2}{9 \cdot 10} = \frac{1}{15}
\]
Получаем квадратное уравнение вида $ k^2 - k - 6 = 0 $ с корнями $-2$ и $3$. Так как $k$ не может быть отрицательным, ответ $3$.
\end{enumerate}
\item
\begin{enumerate}
\item Если эксперт отдаёт предпочтение Fit, то это можно интерпретировать как «успех» в схеме Бернулли. Так как $\xi$ - количество успехов, $ k \in [0;4]$, $p = \frac{1}{3} $, то
\[
\P(\xi = k) = C_n^k(p)^k(1-p)^{n-k}
\]

Большинство означает, что либо три, либо четыре эксперта выбрали Fit.
\[
\P(\xi = 3) = C_4^3\left(\frac{1}{3}\right)^3 \left(\frac{2}{3}\right)^{1} = \frac{8}{81}
\]
\[
\P(\xi = 4) = C_4^4\left(\frac{1}{3}\right)^4 \left(\frac{2}{3}\right)^{0} = \frac{1}{81}
\]
\[
\P( \xi > 2) =  \frac{9}{81}
\]
\item Аналогично:

\[ \P(\xi = 0) = C_4^0\left(\frac{1}{3}\right)^0 \left(\frac{2}{3}\right)^{4} = \frac{16}{81}\]

\[ \P(\xi = 1) = C_4^1\left(\frac{1}{3}\right)^1 \left(\frac{2}{3}\right)^{3} = \frac{32}{81}\]

\[ \P(\xi = 2) = C_4^2\left(\frac{1}{3}\right)^2 \left(\frac{2}{3}\right)^{2} = \frac{24}{81}\]

\begin{figure}[h!]
    \noindent\centering{
    \includegraphics[width=80mm]{images/kr1_2017_3.png}
    }
    \caption{Функция распределения}
    \label{cdf_kr2017}
\end{figure}

\item Все вероятности посчитаны, видим, что наибольшая достигается при $\xi=1$.
\item $\E(X) = np = \frac{4}{3} $, $ \Var(X) = npq = \frac{8}{9}$
\end{enumerate}
\item
\begin{enumerate}
\item Так как указано, что цена сметаны распределена равномерно на отерзке $[250, 1000]$, максимальное значение цены — $1000$, это и есть необходимая сумма.
\item Вспомним, что функция распределения $F(x) = \P(X \leq x)$, нужно найти такой $x$, что $ \P(X \leq x)=0.9$:
\[
0.9 = 1 - \exp({-x^{2}}) \Rightarrow \exp(-x^{2}) = 0.1 \Rightarrow -x^2 = \ln(0.1)  \Rightarrow x=  \sqrt{-\ln(0.1)}
\]
\item Взяв производную от функции распределения списка без сметаны, получим функцию плотности:
\[
f_X(x) =
\begin{cases}
2x\exp(-x^2) & x \ge 0 \\
0 & \text{иначе}
\end{cases}
\]
Найдём математическое ожидание:
\[
\int_{0}^{+\infty}2x^2\exp({-x^2}) dx = -x \exp({-x^2})\big|_0^{+\infty} + \int_{0}^{+\infty}\exp({-x^2}) dx = \frac{\sqrt{\pi}}{2}
\]
\item Математическое ожидание суммы случайных величин равно сумме математических ожиданий случайных влечин, если они существуют. Математическое ожидание от цены сметаны равно: $ \frac{1000 + 250}{2} = 625 $
Математическое ожидание списка без сметаны было найдено в предыдущем пункте, его осталось перевести в рубли. Получаем ответ: $ 625 + \frac{\sqrt{\pi}}{2} \cdot 1000 $.
\item Так как обе величины имеют абсолютно непрерывные распределения, вероятность попасть в конкретную точку равна нулю.
\end{enumerate}
\item
\begin{enumerate}
\item $\P(\text{детектор показал ложь и подозреваемый лжёт}) = 0.9 \cdot 0.1 + 0.1 \cdot 0.95 = 0.185$
\item $\P(\text{невиновен}|\text{детектор показал ложь}) = \frac{0.9\cdot0.1}{0.185} = \frac{90}{185}$
\item $\P(\text{эксперт точно выявит преступника}) = (0.9)^9 \cdot 0.95$
\item $\P(\text{эксперт ошибочно выявит преступника}) = 9 \cdot 0.1 \cdot 0.9^8\cdot 0.05$
\end{enumerate}

\end{enumerate}
