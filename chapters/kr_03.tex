% !TEX root = ../probability_hse_exams.tex
\newpage
\thispagestyle{empty}
\section{Контрольная работа 3}

\subsection[2022-2023]{\hyperref[sec:sol_kr_03_2022_2023]{2022-2023}}
\label{sec:kr_03_2022_2023}

Основную часть писали 70 минут, очно, 2023-03-20. Минимум писали на семинарах. 

\begin{enumerate}
  \item Вася тратит на обед время $Х$, которое хорошо описывается равномерным распределением на отрезке $[ \theta, 2 \theta ]$. 

  \begin{enumerate}
  \setlength\itemsep{0.01cm}
  \item (3) Методом моментов, используя первый момент, найдите оценку $\theta$.
  \item (3) Проверьте, будет ли эта оценка несмещённой?
  \item (4) Проверьте, будет ли эта оценка состоятельной?
  \item (10) Маша утверждает, что оценка $\tilde{\theta}_2 = \min\{X_1, \ldots, X_n\}$ эффективнее, чем оценка метода моментов, найденная с помощью первого момента. 
  Проверьте утверждение Маши для оценок, построенных по выборке из двух наблюдений.
  \end{enumerate}
  

  \item Срок службы холодильника описывается показательным распределением c плотностью 
  \[
  f(x,\theta) =
  \begin{cases}
  \frac{1}{\theta}  \exp(-\frac{x}{\theta}) & \text{ при } x \geq 0, \\
  0 & \text{ при } x < 0,
  \end{cases}
  \] 
  По выборке из $n$ наблюдений найдите
  \begin{enumerate}
  \item(7) оценку максимального правдоподобия  $\hat{\theta}$ для $\theta$,
  \item (3)  дисперсию оценки  $\hat{\theta}$ ,
  \item (3)  информацию Фишера о параметре, содержащуюся в выборке,
  \item (2) оценку асимптотической дисперсии оценки  $\hat{\theta}$,
  \item (5)  оценку максимального правдоподобия для вероятности того, что холодильник проработает более 10 лет,
  \item (5)  асимптотическую дисперсию оценки из пункта (д),
  \item (2)  предел по вероятности для последовательности оценок из пункта (д) при $n \to \infty$.
  \end{enumerate}
  
  \item Евгений опросил жителей Москвы и Санкт-Петербурга, 
  какой из трёх видов отдыха они предпочитают: прогулки, чтенье, сон глубокий. 
  Из 300 опрошенных москвичей прогулки предпочитают 100 человек, чтенье — 50 и сон глубокий — 150 человек. Из 200 опрошенных санктпетербуржцев прогулки предпочитают 80 человек, 
  чтение — 70 и сон глубокий — 50 человек.
  
  \begin{enumerate}
  \item (5) Постройте 95\% асимптотический доверительный интервал для доли москвичей, предпочитающих прогулки.
  \item (8) Постройте 95\% асимптотический доверительный интервал для разницы доли москвичей и доли петербуржцев, предпочитающих прогулки.
  \item (7) Сколько человек нужно было опросить Евгению, чтобы с вероятностью не менее 0.9 быть уверенным, что доли москвичей в выборке отличаются от их истинных значений не более, чем на 0.01?
  \end{enumerate}
  
\end{enumerate}


\subsection[2021-2022]{\hyperref[sec:sol_kr_03_2021_2022]{2021-2022}}
\label{sec:kr_03_2021_2022}

Минимум и основную часть писали очно, за ислючением одной онлайн группы.
Минимум не был блокирующим, но проверялся жестко: одна смысловая ошибка — ноль за задачу.
Минимум писали 25 минут, основную часть — 80 минут. 
В минимуме было 5 задач по 10 баллов, при этом можно было набрать не более 40 баллов, 
одна лишняя задача давала возможность ошибиться без штрафа.

\subsubsection*{Минимум}


\begin{enumerate}
  
\item Дайте определение эффективности оценки $\hat \theta$ среди множества оценок $\hat \Theta$.
  
  \item Величины $X_1, X_2, \dots, X_n$ независимы и нормально распределены, $\cN(\mu, \sigma^2)$. 
  Укажите закон распределения следующих величин:
  \[
  S_a = \bar X, \quad S_b = \frac{\bar X - \mu}{\sigma / \sqrt{n}}, \quad S_c = \frac{\bar X - \mu}{\hat \sigma / \sqrt{n}}, \quad S_d = \frac{\hat \sigma (n-1)}{\sigma}.  
  \]
  
  Если величина не имеет именного распределения, то укажите преобразование с помощью которого можно свести величину к именному распределению. 
  
  \item Случайная выборка  $X_1, X_2, X_3$ взята из распределения Бернулли с неизвестным параметром $p \in (0, 1)$. 

  \begin{enumerate}
    \item   Какие из следующих оценок являются несмещенными?   
    \[
      \hat p_a = \frac{X_1 + X_3}{2}, \quad \hat p_b = \frac{1}{4}X_1 + \frac{1}{2}X_2 + \frac{1}{2}X_3, \quad \hat p_c =  \frac{1}{3}X_1 + \frac{1}{3}X_2 + \frac{1}{3}X_3.
    \]
  \item Среди несмещенных оценок найдите наиболее эффективную оценку.
  \end{enumerate}
  
  \item Величины $X_1, \dots, X_n$ независимы и одинаково распределены с плотностью
  
  \[
  f(x, \theta) = 
   \begin{cases}
     \frac{1}{\theta}e^{-\frac{1}{\theta}} \text{ при }x \geq 0,\\
     0, \text{ иначе}
   \end{cases},
  \]
  где $\theta > 0$ — неизвестный параметр. 
  
  Является ли оценка $\hat \theta_n = \frac{X_1 + \dots + X_n}{n+1}$ состоятельной?
  
  \item Вес взрослого бобра в килограммах, случайная величина $X$, и длина его хвоста в сантиметрах, случайная величина $Y$, 
  являются двумерным нормальным случайным вектором $Z = (X, Y)$ 
  с математическим ожиданием $\E(Z) = (30, 28)$ и ковариационной матрицей
  \[
  \Var(Z) = \begin{pmatrix}
    16 & 8\\
    8 & 9
  \end{pmatrix}.
  \]
  
  Найдите условную вероятность того, что бобр имеет вес более 35 килограммов, если длина его хвоста равна 32 сантиметрам.
  
  \end{enumerate}
  



\subsubsection*{Основная часть}


\begin{enumerate}
  
  \item \textbf{(5 баллов)} Величины $X_1, \dots, X_n$ — случайная выборка из распределения с плотностью
  \begin{equation*}
  f(x, \theta) = 
   \begin{cases}
     \frac{3x^2}{\theta^3} \text{ при } x \in [0,\theta],\\
     0, \text{ иначе}
   \end{cases},
  \end{equation*}
  где $\theta > 0$. 
  
  Используя центральный момент второго порядка, найдите оценку параметра $\theta$ методом моментов.
  
  \item \textbf{(10 баллов)} Случайные величины $X, Y, Z$ независимы и нормально распределены, $\cN(0, 4)$.
  Рассмотрим две новые величины,
  \[
  \gamma_1 = X^2 + Y^2 + Z^2, \quad \gamma_2 = \frac{X}{\sqrt{Y^2 + Z^2}}.
  \]
  
  \begin{enumerate}
    \item Найдите $\E(\gamma_1)$ и $\E(\gamma_2)$.
    \item Найдите $\Var(\gamma_1)$ и $\Var(\gamma_2)$.
    \item Найдите $\P(\gamma_1 > 2.34)$ и $\P(\gamma_2 < 0.75)$.
  \end{enumerate}
  
  \item \textbf{(5 баллов)} Случайный опрос показал, что 87 студентов из 100 с симпатией относятся к Джокеру. Постройте доверительный интервал для истинной доли симпатизирующих Джокеру. Можно ли утверждать, что 90\% студентов симпатизируют Джокеру? 
  
  \item \textbf{(5 баллов)} С помощью ЦПТ и теоремы Слуцкого обоснуйте выбор распределения для построения доверительного интервала в предыдущем пункте.
  
  \item \textbf{(5 баллов)} Студенты Петя, Вася и Маша пообедали один раз в «Груше» и один раз в столовой. 
  В таблице приведены стоимости обедов. 
  Разница расходов на обед хорошо описывается нормальным распределением, одинаковым для всех студентов. 
  Постройте доверительный интервал для разницы математических ожиданий стоимостей обеда в «Груше» и столовой.

  \begin{tabular}{@{}lccc@{}}
     & Петя & Вася & Маша \\
    \toprule
     «Груша»  & 380 & 600 & 250 \\ 
     Столовая & 350 & 450 & 350 \\
    \bottomrule
  \end{tabular}
  
  \item \textbf{(30 баллов)} Величины $X_1, \dots, X_n$ — случайная выборка из нормального распределения $\cN(0, \theta)$.
  
  \begin{enumerate}
  
      \item \textbf{(7 баллов)} Методом максимального правдоподобия оцените параметр $\theta$.
      
      \item \textbf{(8 баллов)} Вычислите математическое ожидание и дисперсию найденной оценки.
      
      \item \textbf{(2 балла)} Проверьте несмещенность оценки.
      
      \item \textbf{(1 балл)} Проверьте, будет ли оценка асимптотически несмещена.
      
      \item \textbf{(2 балла)} Проверьте состоятельность оценки.
      
      \item \textbf{(2 балла)} Вычислите информацию Фишера, содержащуюся в выборке.
      
      \item \textbf{(2 балла)} Проверьте, является ли найденная оценка эффективной в классе оценок с таким же математическим ожиданием. 
      
      \item \textbf{(2 балла)} Найдите оценку максимального правдоподобия для стандартного отклонения $X_1$.
      
      \item \textbf{(4 балла)} Найдите асимптотическую дисперсию оценки из предыдущего пункта.
      
  \end{enumerate}
  
  
  \end{enumerate}
  



\subsection[2020-2021]{\hyperref[sec:sol_kr_03_2020_2021]{2020-2021}}
\label{sec:kr_03_2020_2021}


Основную часть писали 80 минут очно. Минимум проходил в виде онлайн теста за несколько дней до основной части. 
Минимум состоял из 12 вопросов (тестовых и с числовым ответом) и на него было выделено 30 минут.


\begin{enumerate}
\item Среди курьеров проводится соревнование по скоростной доставке заказов. 
Каждый участник имеет $20$ попыток, 
лучшая из которых (минимальное время) засчитывается в качестве итогового результата. 
Время, за которое случайно выбранный участник (в пределах одной попытки) доставляет заказ, 
хорошо описывается экспоненциальным распределением с параметром $\lambda$ и не зависит ни от результатов предыдущих попыток, 
ни от результатов других участников. 
Имеется выборка $X_{1},\dots, X_{500}$  из итоговых (лучших) результатов $500$ участников.

\begin{enumerate}
\item Найдите функцию плотности итогового (лучшего) результата для одного участника.
\item При помощи метода моментов, используя второй центральный момент, по выборке $X$  оцените параметр $\lambda$.
\item При помощи метода максимального правдоподобия по выборке $X$  оцените параметр $\lambda$.
\item Найдите оценку асимптотической дисперсии параметра $\lambda$. Проверьте, является ли ММП оценка параметра $\lambda$ эффективной.
\item Оцените вероятность того, что худший результат среди первых 50 участников соревнований окажется меньше 1.
\item Оцените асимптотическую дисперсию найденной в предыдущем пункте оценки.
\end{enumerate}

\item Галактический путешественник бороздит просторы космоса в поисках инопланетян. 
Каждая посещаемая им планета с вероятностью $p=0.1$ оказывается непригодной для жизни. 
В противном случае на ней могу обитать инопланетяне, число которых описывается распределением Пуассона с параметром $\lambda$. 
У исследователя имеется выборка из числа инопланетян, повстречавшихся на посещенных планетах (в том числе непригодных для жизни).

\begin{enumerate}
\item В качестве оценки параметра $\lambda$ исследователь использует статистику $\hat{\lambda} = \bar{X}$. 
Проверьте, будет ли данная оценка несмещенной и состоятельной.

\item Найдите оценку параметра $\lambda$ при помощи метода моментов, используя первый начальный момент.

\item Пусть $\lambda=0.5$. Найдите примерно минимальный объем выборки, 
при котором вероятность того, 
что найденная в предыдущем пункте оценка отклонится от истинного значения параметра менее, чем на 1, превысит 0.5.
\end{enumerate}

\item Прибыли малых, средних и крупных фирм хорошо описываются нормальным распределением. 
Дисперсии прибыли составляют 1, 25 и 100 для малых, средних и крупных фирм соответственно. 
Доля малых фирм составляет 0.4, средних — 0.5, крупных — 0.1. С использованием пропорционального размещения была сформирована выборка, 
содержащая информацию о прибыли 500 фирм.
\begin{enumerate}
\item Рассчитайте дисперсию стратифицированного среднего.
\item Вычислите вероятность того, что стратифицированное среднее окажется меньше истинного значения на единицу или больше.
\item Найдите разбиение выборки, максимизирующее вероятность, рассчитанную в предыдущем пункте (снимается допущение о пропорциональном разбиении, 
теперь нужно самостоятельно предложить разбиение, исходя из указанного критерия).
\end{enumerate}


\item Используя реализацию $x=(0, 1, 10)$ постройте 90\%-й доверительный интервал для параметра $\theta$ в двух случаях:

\begin{enumerate}

\item $(X_{1}-\theta) ^ 2\sim \chi^2(1)$, а величины $X_1, X_2, X_3$ — случайная выборка.

\item $\frac{1}{\theta}\sum\limits_{i=1}^{3}\frac{X_{i}^3}{3}\sim U\left(0, 3\right)$ (равномерное распределение), 
а величины $X_1, X_2, X_3$ могут быть зависимы. 
\end{enumerate}

\item Доходности акций независимы и хорошо описываются нормальным распределением. Доходность первой акции имеет дисперсию $25$, второй — $100$. 

\begin{enumerate}

\item Используя реализации выборок доходности первой $x=(0,1,2)$ и второй $y=(0,1)$ акций, 
постройте 95\%-й доверительный интервал математического ожидания разности дневной доходности двух акций.

\item Повторите предыдущий пункт предполагая, что дисперсии доходностей равны и неизвестны.

\item Рассмотрим еще одну акцию с дисперсией доходности равной $64$. 
Постройте 95\%-й доверительный интервал для математического ожидания \textbf{суммарной} дневной доходности трех акций, учитывая, 
что реализации выборок имеют вид $x=(0, 1, 2)$, $y=(0, 1)$, $z = (1, 2)$ для первой, второй и третьей акций соответственно.

\end{enumerate}
\end{enumerate}
  






\subsection[2019-2020]{\hyperref[sec:sol_kr_03_2019_2020]{2019-2020}}
\label{sec:kr_03_2019_2020}

Весной 2020 года контрольная 3 была разрезана на несколько онлайн миниконтрольных из-за дистанционного режима обучения.
На каждую миниконтрольную отводилось примерно 40 минут времени. 

\subsubsection*{Миникр 1}

Случайные величины $\xi$ и $\eta$ имеют совместное нормальное распределение:

\[
\begin{pmatrix}
  \xi \\
  \eta \\
\end{pmatrix}   \sim 
\cN \left(
\begin{pmatrix}
1 \\
1 \\
\end{pmatrix};
\begin{pmatrix}
  1 & d \\
  d & 1 \\
\end{pmatrix}
\right).
\]

Корреляция между $\xi$ и $\eta$ равна $\Corr(\xi, \eta)= 0.5$.


\begin{enumerate}
\item Найдите вероятность $\P(\xi > 2)$.
\item Выпишите функцию плотности $\xi$.
\item Найдите константу $d$. 
\item Вычислите $\P(2\xi -3\eta >10)$.
\item Выпишите функцию плотности случайной величины $2\xi -3\eta$.
\item Вычислите $\E(\xi |\eta = 1)$.
\item Вычислите $\Var(\xi |\eta = 1)$.
\item Вычислите $\P(\xi > 2 |\eta = 1)$.
\end{enumerate}


\subsubsection*{Миникр 2}

Имеется случайная выборка $X_1$, \ldots, $X_n$ из распределения с функцией плотности
\[
f(x) = \begin{cases}
  \lambda e^{-\lambda(x-100)}, \text{ если } x > 100; \\
  0, \text{ иначе.}  
\end{cases}
\]
  
\begin{enumerate}
\item Методом моментов, используя первый начальный момент, найдите оценку параметра $\lambda$.
\item Методом максимального правдоподобия найдите:
\begin{enumerate}
\item оценку параметра $\lambda$;
\item оценку вероятности $\P(X_1 > 101)$.
\end{enumerate}
\item Вычислите информацию Фишера о параметре $\lambda$, содержащуюся во всей выборке.
\item Вычислите асимптотическую дисперсию оценки максимального правдоподобия $\hat\lambda$.
\item Вычислите асимптотическую дисперсию оценки максимального правдоподобия $\hat\P(X_1 > 101)$.
\item Найдите оценку максимального правдоподобия асимптотической дисперсии оценки максимального правдоподобия  
$\hat\P(X_1 > 101)$.
\end{enumerate}
\begin{center}
\begin{minipage}{3cm}
  \includegraphics[height=20ex]{images/cheshire_cat.png}
\end{minipage}
\end{center}



\subsubsection*{Миникр 3}

\begin{enumerate}

  \item  Пусть $X_1$, $X_2$, $X_3$, \ldots, $X_n$ — случайная выборка из дискретного распределения, 
  заданного с помощью таблицы
  
  \begin{tabular}{@{}llll@{}}
    \toprule
     $x$           & $-100$ & $0$ & $100$ \\ 
     $\P(X_i = x)$ & $0.5-\theta$ & $0.5$ & $\theta$ \\
    \bottomrule
  \end{tabular}
        
  Рассмотрите оценку $\hat \theta = \bar X$.
  \begin{enumerate}
  \item Найдите $\E(\hat \theta)$. 
  Является ли оценка $\hat \theta$ несмещенной оценкой неизвестного параметра $\theta$?
  \item Подберите константы $a$ и $b$ так, 
  чтобы оценка $\tilde \theta = a + b\bar X$ оказалась несмещенной оценкой неизвестного параметра $\theta$.
  \end{enumerate} 
  
  
  \item Пусть $X_1$, $X_2$, $X_3$, \ldots, $X_n$ — 
  случайная выборка из распределения с плотностью распределения
  \[
  f(x, \theta) = \begin{cases}
    6x(\theta - x)/\theta^3, \text{ при } x\in [0;\theta]; \\
    0, \text{ иначе}
  \end{cases},
  \]
  
  где $\theta$ — неизвестный параметр распределения. 
  Является ли оценка $\hat\theta_n = \frac{2n+1}{n}\bar X_n$ состоятельной оценкой неизвестного параметра $\theta$?
  
  \item Пусть $X_1$, $X_2$, $X_3$, \ldots, $X_n$ — случайная выборка из распределения с плотностью
  \[
    f(x, \theta) = \begin{cases}
      \frac{1}{\theta}\exp\left(-\frac{x-100}{\theta}\right), \text{ при } x\geq 100; \\
      0, \text{ иначе}
    \end{cases},    
  \]
  где $\theta$ — неизвестный параметр. Проверьте, будет ли оценка $\hat\theta = \bar X - 100$ эффективной?
  
  \item  Время от начала контрольной работы до её загрузки в систему lms  подчинено закону с функцией плотности
  \[
    f(x, \theta) = \begin{cases}
      \frac{1}{\theta}\exp\left(-\frac{x-40}{\theta}\right), \text{ при } x\geq 40; \\
      0, \text{ иначе}
    \end{cases},    
  \]
  По случайной выборке $X_1$, $X_2$, $X_3$, \ldots, $X_n$ оценивают параметр $\theta$. 
  
  \begin{enumerate}
  \item Найдите такие константы $a$ и $b$, чтобы оценка  $\hat \theta_1 = a + b\min\{X_1, \ldots, X_n\}$ была несмещённой.
  \item Какая из двух несмещённых оценок $\hat \theta_1$ или $\hat \theta_2 = \bar X - 40$ является более эффективной? Обоснуйте ответ.
  \end{enumerate}
  
  \end{enumerate}
  
  



\subsection[2018-2019]{\hyperref[sec:sol_kr_03_2018_2019]{2018-2019}}
\label{sec:kr_03_2018_2019}

Примечание: минимум писали 30 минут без чит-листа, потом перерыв 10 минут,
потом задачи писали час сорок с чит-листом.

\subsubsection*{Минимум}

\begin{enumerate}
\item У случайно выбираемого взрослого мужчины рост в сантиметрах, $X$, и вес
в килограммах, $Y$,
являются нормальным случайным вектором $Z=(X,Y)$ с математическим ожиданием $E(Z)=(180,90)$ и ковариационной матрицей:

\[
\Var(Z)=\begin{bmatrix}100 & 35\\35 & 25\end{bmatrix}
\]

Рассмотрим величину $U=X-Y$. Считается, что человек страдает избыточным весом, если $U<80$.
\begin{enumerate}
\item Укажите распределение случайной величины $U$. Выпишите её плотность распределения.
\item Найдите вероятность того, что случайно выбранный мужчина страдает избыточным весом.
\item Найдите условную вероятность того, что случайно выбранный мужчина страдает избыточным весом, при условии, что его рост равен 185 см.
\end{enumerate}
\item Стоимость выборочного исследования генеральной совокупности, состоящей из двух страт, определяется по формуле $TC=c_{1}n_{1}+c_{2}n_{2}$, где $c_{i}$ — цена одного наблюдения в $i$-ой страте, а $n_{i}$ — число наблюдений, которое приходится на $i$-ую страту. Найдите $n_{1}$ и $n_{2}$, при которых дисперсия стратифицированного среднего достигает наименьшего значения, если бюджет исследования 10000 и имеется следующая информация:

			\begin{tabular}{c|c|c|c}
			Страта & 1 & 2 \\
			\hline
			Среднее значение & 10 & 20  \\ \hline
			Стандартная ошибка & 20 & 10  \\ \hline
			Вес & 10\% & 90\% \\
			\hline
			Цена наблюдения & 1 & 4 \\
			\end{tabular}

\item Для независимых нормальных $\cN(\mu,\sigma^2)$ случайных величин $(X_{1}, \ldots, X_{n})$ укажите формулу доверительного интервала с уровнем доверия $(1-\alpha)$ для неизвестного математического ожидания $\mu$ при \textbf{известной} дисперсии $\sigma^2$.

\item Дайте определение распределения Стьюдента с помощью нормальных
случайных величин. Укажите диапазон возможных значений. Нарисуйте функцию плотности распределения Стьюдента при разных степенях свободы на фоне нормальной стандартной функции плотности.

Дополнительная задача для пропустивших мини-контрольную на лекции по уважительной причине

\item Для реализации случайной выборки $x=(2,2,-1,2,1)$:
\begin{enumerate}
\item Найдите вариационный ряд;
\item Найдите выборочный второй начальный момент;
\item Постройте график выборочной функции распределения.
\end{enumerate}

\end{enumerate}


\subsubsection*{Задачи}



\begin{enumerate}[resume]

%Задача №1
\item Василиса Премудрая в рамках борьбы с гендерным дресс-кодом в Тридевятом Царстве устроила распродажу всех своих пяти кокошников. 
Ожидаемая цена случайно выбираемого кокошника составляет $3500$ у.е., а стандартное отклонение – $500$ у.е.. За неделю было продано три кокошника.

Найдите математическое ожидание и дисперсию вырученных Василисой денег, если она продаёт кокошники по себестоимости и вероятность покупки любого из них одна и та же. \textbf{(5 баллов)}

%Задача №2
\item К Весеннему слёту по обмену премудростями
независимо готовятся $n$ Василис Премудрых.
Время подготовки каждой Василисы в часах, $X_{i}$,
имеет функцию плотности:


\[
f(x;\theta)=\begin{cases}
  \frac{2x}{\theta^2}\text{, при }x\in[0;\theta]\\
  0\text{, при }x\notin[0;\theta]
\end{cases},
\]

где $\theta>0$ — неизвестный параметр.
Найдите оценку $\theta$ методом моментов. \textbf{(5 баллов)}

%Задача №3
\item Величины $X_{1}, \ldots, X_{n}$ — случайная выборка из распределения с плотностью
\[
f(x;\theta)=\begin{cases}
  \frac{2x}{\theta^2}\text{, при }x\in[0;\theta]\\
  0\text{, при }x\notin[0;\theta]
\end{cases},
\]
где $\theta>0$ — неизвестный параметр. Для параметра $\theta$ предлагаются две оценки: $\hat{\theta}_{n}=\frac{3}{2}\bar X_n$ и $T_n=\max(X_{1}, \ldots, X_{n})$:

\begin{enumerate}
\item Является ли оценка $\hat{\theta}_{n}$ несмещенной оценкой неизвестного параметра $\theta$? \textbf{(3 балла)}
\item Найдите $\Var(\hat{\theta}_{n})$ \textbf{(3 балла)}.
\item Проверьте состоятельность оценки $\hat{\theta}_{n}$.
\textbf{(4 балла)}
\item Проверьте несмещенность оценки $T_{n}$ и вычислите величину смещения \textbf{(7 баллов)}.
\item Какая из оценок $\hat{\theta}_{n}$ или $T_{n}$ является более эффективной согласно критерию MSE? \textbf{(8 баллов)}
\end{enumerate}

%Задача №4
\item Величины $X_{1}, \ldots, X_{n}$ — случайная выборка из нормального распределения с нулевым математическим ожиданием и дисперсией $\theta$.

\begin{enumerate}
\item С помощью метода максимального правдоподобия найдите оценку $\hat{\theta}_{n}$ параметра $\theta$ \textbf{(6 баллов)}.
\item Проверьте несмещенность найденной оценки \textbf{(3 балла)}.
\item Вычислите информацию Фишера о параметре $\theta$, заключенную в выборке \textbf{(2 балла)}.
\item Проверьте, является ли найденная оценка эффективной \textbf{(4 балла)}.

\textbf{Подсказка}: четвёртый момент стандартной нормальной случайной величины равен 3.
\end{enumerate}

\item Пусть $X_{1}, \ldots, X_{n}$ — случайная выборка из равномерного распределения с плотностью


\[
f(x;\theta)=\begin{cases}\frac{1}{\theta}\text{, если }x\in[0,\theta]\\ 0\text{, если }x\notin[0,\theta]\end{cases},
\]

где $\theta>0$. Постройте оценку параметра $\theta$ методом максимального правдоподобия \textbf{(10 баллов)}.

\end{enumerate}



\subsection[2017-2018]{\hyperref[sec:sol_kr_03_2017_2018]{2017-2018}}
\label{sec:kr_03_2017_2018}

\subsubsection*{Минимум}

\begin{enumerate}
\item Дайте определение выборочной функции распределения.
\item Предположим, что величины $X_1$, $X_2$, \ldots, $X_n$ независимы и нормальны
$\cN(\mu;\sigma^2)$. Укажите закон распределения выборочного среднего, величины
$\frac{\bar X - \mu}{\sigma/\sqrt{n}}$, величины $\frac{\bar X - \mu}{\hat\sigma/\sqrt{n}}$,
величины $\frac{\hat\sigma^2(n-1)}{\sigma^2}$.
\item Рост в сантиметрах, случайная величина $X$, и вес в килограммах, случайная
величина $Y$, взрослого мужчины является нормальным случайным вектором $Z = (X, Y)$
с математическим ожиданием $\E(Z) = (175, 75)$ и ковариационной матрицей

\[
\Var(Z) =
\begin{pmatrix}
49 & 28 \\
28 & 36
\end{pmatrix}
\]

\begin{enumerate}
\item Найдите средний вес мужчины при условии, что его рост составляет $172$ см.
\item Выпишите условную плотность распределения веса мужчины при условии, что его
рост составляет $172$ см.
\item Найдите условную вероятность того, что человек будет иметь вес, больший $92$ кг,
при условии, что его рост составляет $172$ см.
\end{enumerate}

\item Стоимость выборочного исследования генеральной совокупности, состоящей из трёх
страт, определяется по формуле $TC = c_1n_1 + c_2n_2 + c_3n_3$, где $c_i$ — цена
одного наблюдения в $i$-ой страте, a $n_i$ — число наблюдений, которые приходятся
на $i$-ую страту. Найдите $n_1$, $n_2$ и $n_3$, при которых дисперсия стратифицированного
среднего достигает наименьшего значения, если бюджет исследования 8000 и имеется
следующая информация:

\begin{center}
\begin{tabular}{cccc}
\toprule
Страта & $1$ & $2$ & $3$  \\
\midrule
Среднее значение & $30$ & $40$ & $50$ \\
Стандартная ошибка  & $5$ & $10$ & $20$ \\
Вес & $25\%$ & $25\%$ & $50\%$ \\
Цена наблюдения & $2$ & $5$ & $8$ \\
\bottomrule
\end{tabular}
\end{center}
\end{enumerate}


\subsubsection*{Задачи}

\begin{enumerate}[resume]
	%Задача 1
\item Пусть $X_{1}, \ldots, X_{n}$ — выборка из нормального распределения $\cN(\mu,1)$.
\begin{enumerate}
\item Выпишите функцию правдоподобия;
\item Методом максимального правдоподобия найдите оценку $\hat{\mu}$ математического
ожидания $\mu$;
\item Проверьте состоятельность и несмещённость оценки $\hat{\mu}$;
\item Вычислите информацию Фишера о параметре $\mu$, содержащуюся во всей выборке;
\item Для произвольной несмещённой оценки $\mu$ выпишите неравенство Рао-Крамера-Фреше;
\item Проверьте свойство эффективности оценки $\hat{\mu}$;
\item Найдите оценку максимального правдоподобия $\hat{\theta}$ для второго начального
момента;
\item Проверьте свойства несмещенности и асимптотической несмещенности оценки $\hat{\theta}$;
\item С помощью дельта-метода вычислите, примерно, дисперсию оценки $\hat{\theta}$;
\item Проверьте состоятельность оценки $\hat{\theta}$.
\end{enumerate}

 %Задача 2
 \item Пусть $X_{1}$, \ldots, $X_{n}$ выборка из распределения с функцией плотности:

\[
f(x)=\begin{cases}
\frac{2}{\theta^2}(\theta-x),&\text{при }x\in[0,\theta]\\
0,&\text{при }x\notin[0,\theta]
\end{cases}
\]

\begin{enumerate}
\item Методом моментов найдите оценку параметра $\theta$;
\item Приведите определение состоятельности оценки и проверьте, будет ли найденная
оценка состоятельной.
\end{enumerate}

%Задача 3
\item В прихожей лежат четыре карты «тройка». На двух из них нет денег, на двух
других 30 и 500 рублей. Вовочка не помнит, на какой из карт есть деньги, поэтому
берёт три карточки.

\begin{enumerate}
\item Найдите математическое ожидание и дисперсию средней по выбранным карточкам
суммы денег;
\item Определите, какова вероятность того, что Вовочке удастся войти в метро,
если стоимость проезда по тройке составляет 35 рублей.
\end{enumerate}

%Задача 4
\item По выборочному опросу студенческих семейных пар о расходах на ланч были
получены следующие результаты:

\begin{center}
\begin{tabular}{ccccc}
\toprule
Номер семьи & 1 & 2 & 3 & 4\\
Расходы мужа & 450 & 370 & 170 & 200\\
Расходы жены & 210 & 350 & 250 & 180\\
\bottomrule
\end{tabular}
\end{center}

Считая, что разница в расходах мужа и жены хорошо описываются нормальным распределением,
постройте 95\%-ый доверительный интервал для разницы математических ожиданий расходов
супругов. Есть ли основания утверждать, что расходы одинаковы?

	%Задание №5
\item Наблюдатель Алексей Недопускальный решил проверить честность выборов.
Ему удалось подглядеть, как проголосовали 60 избирателей. Из них 42 выбрали
действующего президента.

\begin{enumerate}
\item Постройте 95\%-ый доверительный интервал для истинной доли избирателей,
проголосовавших «за» действующего президента.
\item По результатам ЦентрИзберКома «за» действующего президента проголосовало
76.67\% населения. Согласуются ли эти данные с данными Алексея?
\item Сколько бюллетеней нужно подглядеть Алексею, чтобы с вероятностью
$0.95$ отклонение от выборочной доли проголосовавших «за» действующего
президента от истинной не превышало $0.01$?
\end{enumerate}
\end{enumerate}



\newpage
\subsection[2016-2017]{\hyperref[sec:sol_kr_03_2016_2017]{2016-2017}}
\label{sec:kr_03_2016_2017}

\begin{enumerate}

\item Дана реализация случайной выборки: $1$, $10$, $7$, $4$, $-2$. Выпишите
определения и найдите значения следующих характеристик:
\begin{enumerate}
  \item вариационного ряда,
  \item выборочного среднего,
  \item выборочной дисперсии,
  \item несмещенной оценки дисперсии,
  \item выборочного второго начального момента.
  \item Постройте выборочную функцию распределения.
\end{enumerate}


\item
Мама дяди Фёдора каждое лето приезжает в Простоквашино с тремя вечерними платьями.
Если выбирать одно платье из трёх случайно, то ожидание стоимости равно 11 тысяч рублей,
а дисперсия стоимости равна 3 тысячи квадратных рублей.
Рачительный кот Матроскин случайным образом выбирает одно из платьев и продаёт его как ненужное. Вычислите математическое ожидание и дисперсию стоимости двух оставшихся платьев.

\item
Ресторанный критик ходит по трём типам ресторанов (дешевых, бюджетных и дорогих)
города N для того, чтобы оценить среднюю стоимость бизнес-ланча. В городе 40\%
дешевых ресторанов, 50\% — бюджетных и 10\% — дорогих. Стандартное отклонение
цены бизнес-ланча составляет 10, 30 и 60 рублей соответственно. В ресторане
критик заказывает только кофе. Стоимость кофе в дешевых/бюджетных/дорогих ресторанах
составляет 150, 300 и 600 рублей соответственно, а бюджет исследования — 30\,000 рублей.
\begin{enumerate}
  \item Какое количество ресторанов каждого типа нужно посетить критику, чтобы
	как можно точнее оценить среднюю стоимость бизнес-ланча при заданном бюджетном
	ограничении (округлите полученные значения до ближайших целых)?
  \item Вычислите дисперсию соответствующего стратифицированного среднего.
\end{enumerate}

\item
В «акции протеста против коррупции» в Москве 26.03.2017 по данным МВД приняло
участие 8\,000 человек. Считая, что население Москвы составляет 12\,300\,000 человек,
постройте 95\% доверительный интервал для истинной доли желающих участвовать в
подобных акциях жителей России. Можно ли утверждать, что эта доля статистически
не отличается от нуля?

\item
Для некоторой отрасли проведено исследование об оплате труда мужчин и женщин. Их зарплаты (тыс. руб. в месяц) приведены ниже:
\begin{center}
\begin{tabular}{cccccc}
  \toprule
  \text{мужчины}         &$50$    &$40$    &$45$   &$45$   &$35$   \\
  \text{женщины}         &$60$    &$30$    &$30$   &$35$   &$30$   \\ \bottomrule
\end{tabular}
\end{center}

\begin{enumerate}
  \item Считая, что распределение заработных плат мужчин хорошо описывается
	нормальным распределением, постройте
  \begin{enumerate}
    \item 99\%-ый доверительный интервал для математического ожидания заработной
		платы мужчин,
    \item 90\%-ый доверительный интервал для стандартного отклонения заработной
		платы мужчин.
  \end{enumerate}
  \item
  \begin{enumerate}
    \item Сформулируйте предпосылки, необходимые для построения доверительно интервала
		для разности математических ожиданий заработных плат мужчин и женщин.
    \item Считая предпосылки выполненными, постройте 90\%-ый доверительный интервал
		для разности математических ожиданий заработных плат мужчин и женщин.
    \item Можно ли считать зарплаты мужчин и женщин одинаковыми?
  \end{enumerate}
\end{enumerate}

\item
Пусть $X = (X_1, \, \ldots, \, X_n)$ — случайная выборка из нормального распределения
с нулевым математическим ожиданием и дисперсией $\theta$.
\begin{enumerate}
  \item Используя второй начальный момент, найдите оценку параметра $\theta$
	методом моментов.
  \item Сформулируйте определение несмещённости оценки и проверьте выполнение
	данного свойства для оценки, найденной в пункте а).
  \item Сформулируйте определение состоятельности оценки и проверьте выполнение
	данного свойства для оценки, найденной в пункте а).
  \item Найдите оценку параметра $\theta$ методом максимального правдоподобия.
  \item Вычислите информацию Фишера о параметре $\theta$, заключенную в $n$
	наблюдениях случайной выборки.
  \item Сформулируйте неравенство Рао-Крамера-Фреше.
  \item Сформулируйте определение эффективности оценки и проверьте выполнение
	данного свойства для оценки, найденной в пункте г).
\end{enumerate}

\item
Аэрофлот утверждает, что 10\% пассажиров, купивших билет, не являются на рейс.
В случайной выборке из шести рейсов аэробуса А320, имеющего 180 посадочных мест,
число не явившихся оказалось: $5$, $10$, $25$, $0$, $17$, $30$. Пусть число пассажиров
$X$, не явившихся на рейс, хорошо описывается распределением Пуассона $\P(X = k)
= \tfrac{\lambda^{k}}{k!}e^{-\lambda}$, $k \in \{0,\, 1,\, 2,\, \ldots\}$. При
помощи метода максимального правдоподобия найдите:
\begin{enumerate}
  \item оценку $\E(X)$ и её числовое значение по выборке,
  \item оценку дисперсии $X$ и её числовое значение по выборке,
  \item оценку стандартного отклонения $X$ и её числовое значение по выборке,
  \item оценку вероятности того, что на рейс явятся все пассажиры, а также найдите
	её числовое значение по выборке.
  \item Используя асимптотические свойства оценок максимального правдоподобия,
	постройте 95\% доверительный интервал для $\E(X)$.
  \item С помощью дельта-метода найдите 95\% доверительный интервал для вероятности
	полной загруженности самолёта.
\end{enumerate}
\end{enumerate}


\newpage
\subsection[2015-2016]{\hyperref[sec:sol_kr_03_2015_2016]{2015-2016}}
\label{sec:kr_03_2015_2016}

\epigraph{Ищите и обрящете, толцыте и отверзется вам}{Лука 11:9}

\begin{enumerate}
\item В студенческом буфете осталось только три булочки одинаковой привлекательности
и цены, но разной калорийности: 250, 400 и 550 ккал. Голодные Маша и Саша, не глядя
на калорийность, покупают по булочке. Найдите математическое ожидание и дисперсию
суммы поглощенных студентами калорий.
\item Дана реализация случайной выборки  независимых одинаково распределенных
случайных величин: 11, 4, 6.
\begin{enumerate}
  \item Выпишите вариационный ряд;
  \item Постройте выборочную функцию распределения;
  \item Найдите выборочную медиану распределения;
  \item Вычислите выборочное среднее и несмещенную оценку дисперсии.
\end{enumerate}

\item Найдите математическое ожидание, дисперсию и коэффициент корреляции случайных
величин $X$ и $Y$, совместное распределение которых имеет функцию плотности
\[
f(x, y) = \frac{5}{4\pi \sqrt{6}} \exp\left(
-\frac{25}{48}\left( (x-1)^2 -0.4(x-1)y + y^2 \right)
\right)
\]

\item Рост и размер обуви $(X, Y)$ взрослого мужчины хорошо описывается двумерным
нормальным распределением с математическим ожиданием $(178, 42)$ и ковариационной
матрицей
\[
C = \begin{pmatrix}
49 & 5.6 \\
5.6 & 1 \\
\end{pmatrix}
\]
\begin{enumerate}
  \item Какой процент мужчин обладает ростом выше 185 см?
  \item Являются ли рост и размер обуви случайно выбранного мужчины независимыми?
	Обоснуйте ответ.
  \item Среди мужчин с ростом 185 см, каков процент тех, кто имеет размер обуви,
	меньший сорок второго  $\P(Y < 42 \mid X=185)$?
\end{enumerate}


\item Дана случайная выборка $X_1$, \ldots, $X_n$ из равномерного распределения
$U[0; 2\theta]$.
\begin{enumerate}
  \item С помощью первого момента найдите оценку параметра  $\theta$ методом моментов;
  \item Сформулируйте определения несмещенности, состоятельности и эффективности оценок;
  \item Проверьте, будет ли найденная в пункте (а) оценка несмещенной и состоятельной.
  \item С помощью статистики $X_{(n)}= \max\{ X_1,\ldots, X_n \}$ постройте несмещенную
	оценку параметра $\theta$  вида $cX_{(n)}$. Укажите значение $c$.
  \item Проверьте, будет ли данная оценка состоятельной;
  \item Какая из двух оценок является более эффективной? Обоснуйте ответ.
\end{enumerate}

\item Вовочка хочет проверить утверждение организаторов юбилейной лотереи
«Метро-80 лет в ритме столицы», что почти треть всех билетов выигрышные.
Для этого он попросил $n$ своих друзей купить по 10 лотерейных билетов.
Пусть  $X_i$ — число выигрышных билетов друга $i$ и $p$ — вероятность выигрыша
одного билета.
\begin{enumerate}
  \item  Какое распределение имеет величина $X_i$?
  \item Запишите функцию правдоподобия $L(p)$  для выборки $X_1$, \ldots, $X_n$;
  \item Методом максимального правдоподобия найдите оценку $p$;
  \item Найдите информацию Фишера для одного наблюдения $i(p)$;
  \item Для произвольной несмещенной оценки $T(X_1, \ldots, X_n)$ запишите
	неравенство Рао-Крамера-Фреше;
  \item Будет оценка $\hat p_{ML}$ эффективной?
  \item Найдите оценку максимального правдоподобия математического ожидания и
	дисперсии выигранных произвольным другом билетов;
  \item Дана реализация случайной выборки 5 Вовочкиных друзей. Число выигрышных
	билетов  оказалось равно (3, 4, 0, 2, 6). Найдите значение точечной оценки
	вероятности выигрыша $p$. Как Вы думаете, похоже ли утверждение организаторов на правду?
\end{enumerate}

\item  Дана выборка $X_1$, $X_2$, \ldots, $X_n$ независимых одинаково распределенных
величин из распределения с функцией плотности
\[
f(x)=\begin{cases}
(1+\theta)x^\theta, \text{ если } 0<y<1, \theta+1>0 \\
0, \text{ иначе}
\end{cases}.
\]

Методом максимального правдоподобия найдите оценку параметра $\theta$.

\item Пробег (в 1000 км) автомобиля «Лада Калина» до капитального ремонта двигателя
является нормальной случайной величиной с неизвестным математическим ожиданием $\mu$
и известной дисперсией 49. По выборке из 20 автомобилей найдите значение доверительного
интервала для математического ожидания пробега с уровнем доверия $0.95$.
\end{enumerate}


\newpage
\subsection[2014-2015]{\hyperref[sec:sol_kr_03_2014_2015]{2014-2015}}
\label{sec:kr_03_2014_2015}

\begin{enumerate}

\item В студенческом буфете осталось только три булочки одинаковой привлекательности
и цены, но разной калорийности: 250, 400 и 550 ккал. Голодные Маша и Саша, не глядя
на калорийность, покупают по булочке. Найдите математическое ожидание и дисперсию
суммы поглощённых студентами калорий.

\item Ресторанный критик ходит по трём типам ресторанов (дешёвых, бюджетных и дорогих)
города N для того, чтобы оценить среднюю стоимость бизнес-ланча. В городе N 30\%
дешёвых ресторанов, 60\% бюджетных  и 10\% дорогих. Стандартное отклонение цены
бизнес-ланча составляет 10, 30 и 60 рублей соответственно. В ресторане критик
заказывает только кофе.  Стоимость кофе в дешёвых/бюджетных/дорогих ресторанах
составляет 150, 300 и 600 рублей соответственно, а бюджет  исследования — 15 000
рублей. Какое количество ресторанов каждого типа нужно посетить критику, чтобы
как можно точнее оценить среднюю стоимость бизнес-ланча при заданном бюджетном
ограничении (округлите полученные значения до ближайших целых)? Вычислите дисперсию
соответствующего стратифицированного среднего.

\item Дана случайная выборка $X_1$, \ldots, $X_n$  из некоторого распределения
с математическим ожиданием $\mu$ и дисперсией $\sigma^2$. Даны три оценки $\mu$:
 \[
\hat\mu_1 = (X_1 + X_2)/2, \quad \hat\mu_2 = X_1/4 + (X_2 + \ldots + X_{n-1})/(2n-4)
+ X_n/4, \quad \hat\mu_3 = \bar X
 \]
\begin{enumerate}
\item Какая из оценок является несмещённой?
\item Какая из оценок является более эффективной, чем остальные?
\end{enumerate}

\item Случайный вектор $(X, Y)^T$ имеет двумерное нормальное распределение
с математическим ожиданием  $(1, 2)^T$ и ковариационной матрицей
\[
C=\begin{pmatrix}
1 & -1 \\
-1 & 4
\end{pmatrix}
\]
\begin{enumerate}
\item $\P(X>1)$
\item $\P(2X+Y>2)$
\item $\E(2X+Y|X=2)$, $\Var(2X+Y|X=2)$, $\P(2X+Y>2|X=2)$
\item Сравните вероятности двух предыдущих пунктов, объясните, почему они отличаются.
Являются ли компоненты случайного вектора независимыми?
\end{enumerate}


\item Величины $X_1$, $X_2$ и $X_3$  независимы и стандартно нормально распределены.
Вычислите:
\begin{enumerate}
\item $\P(X_1^2 + X_2^2 > 6)$
\item $\P(X_1^2 / (X_2^2 + X_3^2) > 9.25 )$
\end{enumerate}

\item Дана случайная выборка $X_1$, \ldots, $X_n$ из равномерного распределения
$U[0, \theta]$.
\begin{enumerate}
\item С помощью статистики $X_{(n)}=\max\{X_1, \ldots, X_n \}$ постройте несмещённую
оценку параметра $\theta$  вида $cX_{(n)}$ (укажите значение $c$).
\item Будет ли данная оценка состоятельной?
\item Найдите оценку параметра $\theta$ методом моментов.
\item Какая из двух оценок является более эффективной?
\end{enumerate}

\item Каждый из $n$ биатлонистов одинакового уровня подготовки стреляет по мишеням
до первого промаха.  Пусть $X_i$ — число выстрелов $i$-го биатлониста,
$\P(X_i = x_i)=p^{x_i-1}(1-p)$, где $p$ — вероятность попадания при одном выстреле.
\begin{enumerate}
\item Методом максимального правдоподобия найдите оценку $p$.
\item Методом максимального правдоподобия найдите оценку математического ожидания
числа выстрелов.
\item Сформулируйте определения несмещенности, состоятельности и эффективности
оценок, и проверьте выполнение данных свойств для найденной в предыдущем пункте
оценки математического ожидания.
\end{enumerate}
\end{enumerate}


\newpage
\subsection[2013-2014]{\hyperref[sec:sol_kr_03_2013_2014]{2013-2014}}
\label{sec:kr_03_2013_2014}


Вычислите константы $B_1=\{\text{Цифра, соответствующая первой букве}$
Вашей\\ фамилии$\}$ и $B_2=\{\text{Цифра, соответствующая первой букве}$
 Вашего имени$\}$.\\
Уровень значимости для всех проверяемых гипотез $0.0\alpha$, уровень доверия
для всех доверительных интервалов $(1-0.0\alpha)$, где  $\alpha = 1+
\{\text{остаток от деления } B_1 \text{ на }  5\}$.\\

\begin{center}
\begin{tabular}{|c|c|c|c|c|c|c|c|c|c|c|c|c|c|}
\hline  А & Б & В & Г & Д & Е & Ж & З & И & К & Л & М & Н & О \\
\hline 1 & 2 & 3 & 4 & 5 & 6 & 7 & 8 & 9 & 10 & 11 & 12 & 13 & 14 \\
\hline  П & Р & С & Т & У & Ф & Х & Ц & Ч & Ш & Щ & Э & Ю & Я \\
\hline 15& 16  &  17 &  18&  19&  20&  21& 22 & 23 &  24& 25 & 26  &  27 & 28 \\
\hline
\end{tabular}
\end{center}

\begin{enumerate}
\item Вес упаковки с лекарством является нормальной случайной величиной с
неизвестными математическим ожиданием  $\mu$ и дисперсией $\sigma^2$. Контрольное
взвешивание $(10+B_1)$ упаковок показало, что выборочное среднее  $\bar{X} =
(50+B_2)$, а  несмещенная оценка дисперсии равна $B_1\cdot B_2$. Постройте
доверительные интервалы для математического ожидания и дисперсии веса упаковки
(для дисперсии односторонний с нижней границей).

\item Экзамен принимают два преподавателя, случайным образом выбирая студентов.
По выборкам из 85 и 100 наблюдений, выборочные доли не сдавших экзамен студентов
составили соответственно $\frac{1}{B_1+1}$ и $\frac{1}{B_2+1}$ . Можно ли утверждать,
что преподаватели предъявляют к студентам одинаковый уровень требований? Вычислите
минимальный уровень значимости, при котором основная гипотеза (уровень требований одинаков)
отвергается (p-value).

\item Даны независимые выборки доходов выпускников двух ведущих экономических вузов
A и B, по $(10+B_1)$ и $(10+B_2)$ выпускников соответственно: $\bar{X}_A=45$,
$\hat{\sigma}_A=5$, $\bar{X}_B=50$, $\hat{\sigma}_B=6$.
Предполагая, что распределение доходов подчиняется нормальному закону, проверьте
гипотезу об отсутствии преимуществ выпускников вуза B.

\item 	По выборке независимых одинаково распределенных случайных величин
$X_1,\dots,X_n$ с функцией плотности $f(x)=\frac{1}{\theta} x^{-1+\frac{1}{\theta}}$,
$x\in(0, 1)$, найдите оценки максимального правдоподобия параметра $\theta$.
Сформулируйте определения свойств несмещенности, состоятельности и эффективности
и проверьте, выполняются ли эти свойства для найденной оценки.
\end{enumerate}
\underline{Примечание.} В помощь несчастным, забывшим формулу интегрирования по
частям и таблицу неопределенных интегралов, или просто ленивым студентам:
\[
\int_0^1 t^\alpha \ln(t) dt = -\frac{1}{(\alpha+1)^2}
\]

% TODO: check typsetting below this line

\subsection[2011-2012]{\hyperref[sec:sol_kr_03_2011_2012]{2011-2012}}
\label{sec:kr_03_2011_2012}


Условия: 80 минут, без официальной шпаргалки.

\begin{enumerate}
\item Наблюдения $X_1$, $X_2$, \ldots, $X_n$ независимы и одинаково распределены с функцией плотности $f(x)=\lambda \exp(-\lambda x)$ при $x\geq 0$.
\begin{enumerate}
\item Методом максимального правдоподобия найдите оценку параметра  $\lambda$.
\item Найдите оценку максимального правдоподобия $\hat{a}$ для параметра $a=1/\lambda$.
\item Сформулируйте определение несмещенности оценки и проверьте выполнение данного свойства для оценки $\hat{a}$.
\item Сформулируйте определение состоятельности оценки и проверьте выполнение данного свойства для оценки $\hat{a}$.
\item Сформулируйте определение эффективности  оценки и проверьте выполнение данного свойства для оценки $\hat{a}$.
\item Оцените параметр $\lambda$ методом моментов.
\end{enumerate}
Подсказка: $\E(X_i^2)=2/\lambda^2$

\item В ходе анкетирования 100 сотрудников банка «Альфа» ответили на вопрос о том, сколько времени они проводят на работе ежедневно. Среднее выборочное оказалось равно $9.5$ часам при выборочном стандартном отклонении $0.5$ часа.
\begin{enumerate}
\item Постройте 95\% доверительный интервал для математического ожидания времени проводимого сотрудниками на работе.
\item Проверьте гипотезу о том, что в среднем люди проводят на работе 10 часов, против альтернативной гипотезы о том, что в среднем люди проводят на работе меньше 10 часов, укажите точное Р-значение.
\item Сформулируйте предпосылки, которые были использованы для проведения теста.
\end{enumerate}

\item В ходе анкетирования 20 сотрудников банка «Альфа» ответили на вопрос о том, сколько времени они проводят на работе ежедневно. Среднее выборочное оказалось равно 9.5 часам при стандартном отклонении 0.5 часа. Аналогичные показатели для 25 сотрудников банка «Бета» составили 9.8 и 0.6 часа соответственно.
\begin{enumerate}
\item Проверьте гипотезу о равенстве дисперсий времени, проводимого на работе, сотрудниками банков «Альфа» и «Бета». Укажите необходимые предпосылки относительно распределения наблюдаемых значений.
\item Проверьте гипотезу о том, что сотрудники банка «Альфа» проводят на работе столько же времени, что и сотрудники банка «Бета». Укажите необходимые предпосылки относительно распределения наблюдаемых значений.
\end{enumerate}
\end{enumerate}


\subsection[2010-2011]{\hyperref[sec:sol_kr_03_2010_2011]{2010-2011}}
\label{sec:kr_03_2010_2011}

Решение задач с обозначением «\verb|MIN|» необходимо и достаточно для получения удовлетворительной оценки за данную контрольную работу.\par\smallskip

\begin{enumerate}
\item Во время эпидемии гриппа среди привитых людей заболевают в среднем 15\,\%, среди непривитых — 20\,\%. Ежегодно прививаются 10\,\% всего населения (прививка действует один год).
\begin{enumerate*}
\item \verb|MIN| Какой процент населения заболевает во время эпидемии гриппа?
\item Каков процент привитых среди заболевших людей?
\end{enumerate*}

\item Известно, что случайная величина $X\sim\cN(3,25)$.
\begin{enumerate*}
\item \verb|MIN| Найдите вероятности $\P(X>4)$ и $\P(4<X\leqslant 5)$.
\item Дополнительно известно, что случайная величина $Y$ имеет распределение $\cN(1,16)$, 
что $X$ и $Y$ имеют совместное нормальное распределение и что $\Corr(X,Y)=0.4$. Найдите $\P(X-2Y<4)$.
\item Случайная величина $Z\sim \cN(6,49)$ обладает тем свойством, 
что $\Var\left(X-2Y+\frac{1}{\sqrt{7}}Z\right)=64$ и вектор $(X, Y, Z)$ имеет совместное нормальное распределение. 
Найдите условную вероятность $\P(X-2Y<4 | Z>8)$.
\end{enumerate*}

\item Опрос домохозяйств, проживающих в Южном (выборка $X$) и Юго-Западном (выборка $Y$)
административных округах города Москвы, выявил следующие результаты:

\begin{tabular}{@{}lccccccccccccccc@{}}
\toprule
$X$  & $8.4$ & $15.6$ & $21.2$ & $15.2$ & $38.2$ & $28.3$ & $19.1$ & $44.1$ & $68.2$ & $56.0$ & $34.5$ & $33.8$ & $84.2$ & $45.0$ & $28.2$  \\
$Y$ & $54.8$ & $26.6$ & $14.4$ & $22.0$ & $23.9$ & $43.3$ & $65.1$ & $18.0$ & $69.2$ & $32.0$ & $46.7$ & $64.0$  \\ \bottomrule
\end{tabular}

Вычислены следующие суммы: $\sum\limits_i X_i=540$, $\sum\limits_i Y_i=480$, $\sum\limits_i \frac{X_i^2}{15}=1\,706.264$, $\sum\limits_i \frac{Y_i^2}{12}=1\,958.3$, $\sum\limits_i \frac{(X_i-36)^2}{15}\hm=410.264$, $\sum\limits_i \frac{(Y_i-40)^2}{12}=358.3$, $\sum\limits_i \frac{(X_i-40)^2}{15}=426.264$, $\sum\limits_i \frac{(Y_i-36)^2}{12}=374.3$.
\begin{enumerate*}
\item \verb|MIN| Постройте 90\,\% доверительный интервал для математического ожидания дохода в Юго-Западном АО.
\item На 5\,\% уровне значимости проверьте гипотезу о том, что средний доход в Юго-Западном АО не превышает среднего дохода в Южном АО, предполагая, что распределения доходов нормальны.
\item Проверьте гипотезу о равенстве распределений доходов в двух округах, используя статистику Вилкоксона-Манна-Уитни, на 5\,\% уровне значимости. Разрешается использование нормальной аппроксимации.
\end{enumerate*}

\item Вася решил проверить известное утверждение о том, что бутерброд падает маслом вниз. Для этого он провёл серию из 200 испытаний. Ниже приведена таблица с результатами:

\begin{center}
  \begin{tabular}{@{}ccc@{}}
  \toprule
  Бутерброд        & Маслом вверх & Маслом вниз \\ \midrule
  Число наблюдений & $105$        & $95$        \\ \bottomrule
  \end{tabular}
\end{center}\par\smallskip
\verb|MIN| Можно ли утверждать, что бутерброд падает маслом вниз так же часто, как и маслом вверх? (Уровень значимости 0.05.)
\par\medskip
\item
\begin{enumerate*}
\item \verb|MIN| По случайной выборке $X_1, \ldots, X_n$ из нормального распределения
$\cN(\mu_1, \mu_2-\mu_1^2)$ методом моментов оценить параметры $\mu_1$, $\mu_2$.
Дайте определения несмещённости и состоятельности и проверьте выполнение этих свойств
для оценки $\mu_1$.
\item По случайной выборке $X_1, \ldots, X_n$ из нормального распределения
$\cN(\theta, 1)$ методом максимального правдоподобия оцените параметр $\theta$.
Будет ли найденная оценка эффективной?
Ответ обоснуйте.
\end{enumerate*}
\end{enumerate}




\subsection[2009-2010]{\hyperref[sec:sol_kr_03_2009_2010]{2009-2010}}
\label{sec:kr_03_2009_2010}


\begin{enumerate}
\item Имеются наблюдения $-1.5$, $2.6$, $1.2$, $-2.1$, $0.1$, $0.9$. Найдите выборочное среднее, выборочную дисперсию. Постройте эмпирическую функцию распределения.
\item Известно, что в урне всего $n_{t}$ шаров. Часть этих шаров — белые. Количество белых шаров, $n_{w}$, неизвестно. Мы извлекаем из урны $n$ шаров без возвращения. Количество белых шаров в выборке, $X$, — это случайная величина и $\nu=X/n$. Найдите $\E(\nu)$, $\\Var(\nu)$. Будет ли $\nu$ состоятельной оценкой неизвестной доли $p=n_{w}/n_{t}$ белых шаров в выборке? Будет ли оценка $\nu$ несмещенной? Дайте определение несмещенной оценки.
\item Стоимость выборочного исследования генеральной совокупности, состоящей из трёх страт определяется по формуле $TC=150n_1+40n_2+15n_3$, 
где $n_i$ — количество наблюдений в выборке, относящихся к $i$-ой страте. 
Стоимость исследования фиксирована и равна $60000$. 
Цель исследования — получить несмещенную оценку среднего по генеральной совокупности с наименьшей дисперсией. 
Сколько наблюдений нужно взять из каждой страты, если:

\begin{tabular}{@{}lrrr@{}}
\toprule
Страта             & 1      & 2      & 3      \\ \midrule
Стандартная ошибка & $50$   & $20$   & $10$   \\
Вес                & $10\%$ & $30\%$ & $60\%$ \\
Цена наблюдения    & $150$  & $40$   & $15$   \\ \bottomrule
\end{tabular}

\item По выборке $X_1$, $X_2$, \ldots, $X_n$ найдите методом моментов оценку параметра $\theta$ равномерного распределения $\cU[0;\theta]$. Является ли она несмещенной? Является ли она состоятельной? Какая оценка эффективнее, оценка метода моментов или оценка $T=\frac{n+1}{n}\max\{X_1,\ldots,X_n\}$?
\item Неправильная монетка подбрасывается $n$ раз. Количество выпавших орлов — случайная величина $X$.  Найдите оценку вероятности выпадения орла. Проверьте несмещенность, состоятельность и эффективность этой оценки.
\item «Насяльника» отправил Равшана и Джамшуда измерить ширину и длину земельного участка. Равшан и Джамшуд для надежности измеряют длину и ширину 100 раз. Равшан меряет длину, результат каждого измерения — случайная величина $X_i=a+e_i$, где $a$ — истинная длина участка, а $e_i\sim \cN(0,1)$ — ошибка измерения. Джамшуд меряет ширину, результат каждого измерения — случайная величина $Y_i=b+u_i$, где $b$ — истинная ширина участка, а $u_i\sim \cN(0,1)$ — ошибка измерения. Все ошибки независимы. Думая, что «насяльника» хочет измерить площадь участка, Равшан и Джамшуд каждый раз сообщают «насяльнику» только величину $S_i = X_iY_i$.

Помогите «насяльнику» оценить параметры $a$ и $b$ по отдельности методом моментов. По выборке оказалось, что $\sum s_i=3600$ сотен метров, $\sum s_i^2 =162500$ квадратных сотен метров.
\end{enumerate}







\subsection[2008-2009]{\hyperref[sec:sol_kr_03_2008_2009]{2008-2009}}
\label{sec:kr_03_2008_2009}

\subsubsection*{Часть I}

\begin{enumerate}
\item Если $X\sim \cN(0;1)$, то $X^{2}\sim \chi^{2}_{1}$. Верно. Нет.
\item Если $X\sim t_{n}$ и $Y\sim t_{m}$, то $\frac{X/n}{Y/m}\sim F_{n,m}$. Верно. Нет.
\item Если основная гипотеза отвергается  при 1\% уровне значимости, то она будет отвергаться и при 5\% уровне значимости. Верно. Нет.
\item Неравенство Рао-Крамера справедливо только для оценок максимального правдоподобия. Верно. Нет.
\item Оценки метода максимального правдоподобия всегда несмещенные. Верно. Нет.
\item Ошибка второго рода происходит при отвержении основной гипотезы, когда она верна. Верно. Нет.
\item Из несмещенности оценки следует её состоятельность. Верно. Нет.
\item Длина доверительного интервала увеличивается при увеличении уровня доверия (доверительной вероятности) Верно. Нет.
\item Выборочное среднее независимых одинаково распределенных случайных величин с конечной дисперсией имеет асимптотически нормальное распределение. Верно. Нет.
\item Теорема Муавра-Лапласа  является частным случаем ЦПТ. Верно. Нет.
\item Оценка, получаемая за эту контрольную, является несмещенной. Верно. Нет.
\end{enumerate}

Любой ответ на 11 считается правильным.

\subsubsection*{Часть II-A.}

Стоимость задач 10 баллов. Теория вероятностей.

Нужно решить любые \textbf{\underbar{3 (три)}} задачи из части II-A.

\begin{enumerate}
% числа выверены
\item При контроле правдивости показаний подозреваемого на «детекторе лжи» вероятность признать ложью ответ, не соответствующий действительности, равна 0.99, вероятность ошибочно признать ложью правдивый ответ равна 0.01. Известно, что ответы, не соответствующие действительности, составляют 1\% всех ответов подозреваемого.
Какова вероятность того, что ответ, признанный ложью, и в самом деле не соответствует действительности?
\item Предположим, что вероятность того, что среднегодовой доход наугад выбранного жителя некоторого города не превосходит уровень $t$, равна $\P(I\le t)=a+be^{-t/300}$ при $t\ge 0$.
\begin{enumerate}
\item Найдите числа $a$ и $b$.
\item Найдите математическое ожидание, моду и медиану дохода жителей города. Какую из данных характеристик следует использовать для рапорта о высоком уровне жизни?
\end{enumerate}

\item Доходности акций двух компаний являются случайными величинами $X$ и $Y$ с одинаковым математическим ожиданием и ковариационной матрицей $\left( \begin{array}{cc}
   4 & -2  \\
   -2 & 9  \\
\end{array}\right)$
\begin{enumerate}
\item Найдите $\Corr(X,Y)$.
\item В какой пропорции нужно приобрести акции этих двух компаний, чтобы дисперсия доходности получившегося портфеля была наименьшей?

Подсказка: Если $R$ — доходность портфеля, то $R=\alpha X+(1-\alpha)Y$
\item Можно ли утверждать, что величины $X+Y$ и $7X-2Y$ независимы?
\end{enumerate}

\item Волшебный Сундук

Если присесть на Волшебный Сундук, то сумма денег, лежащих в нем, увеличится в два раза. Изначально в Сундуке был один рубль. Предположим, что «посадки» на Сундук — Пуассоновский процесс с интенсивностью $\lambda$. Каково ожидаемое количество денег в Сундуке к моменту времени $t$?

\item На окружности единичной длины случайным образом равномерно и независимо друг от друга выбирают две дуги: длины 0.3 и длины 0.4.
\begin{enumerate}
\item Найдите функцию распределения длины пересечения этих отрезков.
\item Найдите среднюю длину пересечения.
\end{enumerate}
\end{enumerate}

\subsubsection*{Часть II-B.}

Стоимость задач 10 баллов. Построение и свойства оценок.

Нужно решить любые \textbf{\underbar{2 (две)}} задачи из части II-B.

\begin{enumerate}
\item[6.] Асимметричная монета подбрасывается $n$ раз. При этом $X$ раз выпал «орел».
\begin{enumerate}
\item Методом максимального правдоподобия найдите оценку вероятности «орла».
\item Проверьте является ли полученная оценка состоятельной, несмещенной и эффективной.
\item Считая, что $n$ велико, укажите, в каких пределах с вероятностью 0,95 должно находиться значение оценки, если монета симметрична.
\end{enumerate}

\item[7.] Вася попадает по мишени с вероятностью $p$ при каждом выстреле независимо от других. Он стрелял до 3-х промахов (не обязательно подряд). При этом у него получилось $X$ попаданий.
\begin{enumerate}
\item Постройте оценку $p$ с помощью метода максимального правдоподобия.
\item Является ли полученная оценка несмещенной?
\end{enumerate}

\item[8.] Известно, что величины $X_{1}$, \ldots, $X_{n}$ независимы и равномерны на $[0;b]$. Пусть $Y$ — это минимум этих $n$ величин. Вася знает $n$ и $Y$.
\begin{enumerate}
\item Найдите оценку $b$ методом моментов.
\item Является ли полученная оценка несмещенной?
\end{enumerate}
\end{enumerate}

\subsubsection*{Часть II-C.}

Стоимость задач 10 баллов. Проверка гипотез и доверительные интервалы.

Нужно решить любые \textbf{\underbar{3 (три)}} задачи из части II-C.

\begin{enumerate}
\item[9.] Вес выпускаемого заводом кирпича распределен по нормальному закону. По выборке из 16 штук средний вес кирпича составил 2.9 кг, выборочное стандартное отклонение 0.3. Постройте 80\% доверительные интервалы для истинного значения веса кирпича и стандартного отклонения.

Примечание: можно строить односторонний интервал для стандартного отклонения, если таблицы не хватает, чтобы построить двусторонний.

\item[10.] В городе N за год родились 520 мальчиков и 500 девочек.
\begin{enumerate}
\item Проверьте гипотезу о том, что мальчики и девочки рождаются одинаково чаще против гипотезы о том, что мальчиков рождается больше, чем девочек.
\item Вычислите Р-значение (минимальный уровень значимости, при котором основная гипотеза отвергается).
\item Каким должен быть размер выборки, чтобы с вероятностью 0.95 можно было утверждать, что выборочная доля отличается от теоретической не более, чем на 0.02?
\end{enumerate}

\item[11.] Проверьте гипотезу о независимости пола респондента и предпочитаемого им сока.

\begin{tabular}{@{}cccc@{}}
\toprule
  & Апельсиновый & Томатный & Вишнёвый \\ \midrule
Мужчины & $70$         & $40$     & $25$     \\
Женщины & $75$         & $60$     & $35$     \\ \bottomrule
\end{tabular}

\item[12.] Даны независимые выборки доходов выпускников двух ведущих экономических вузов A и B, по 50 выпускников каждого вуза: $\bar{X}_{A}=650$, $\bar{X}_{B}=690$, $\hat{\sigma}_{A}=50$, $\hat{\sigma}_{B}=70$.

Предполагая, что распределение доходов подчиняется нормальному закону, проверьте гипотезу об отсутствии преимуществ выпускников вуза B на уровне значимости 0.05.

\item[13.] Величины $X_{1}$, $X_{2}$, \ldots, $X_{100}$ независимы и распределены $\cN(10,16)$. Вася знает дисперсию, но не знает среднего. Поэтому он строит 60\% доверительный интервал для истинного среднего значения.
\begin{enumerate}
\item Найдите вероятность того, что доверительный интервал накрывает настоящее среднее.
\item Найдите вероятность того, что доверительный интервал накрывает число 9.
\end{enumerate}
\end{enumerate}

\subsubsection*{Часть III.}

Стоимость задачи 20 баллов.

Нужно решить любую \textbf{\underbar{1 (одну)}} задачу из части III.

\begin{enumerate}
\item[14-A.] Набранную книгу независимо друг от друга вычитывают два корректора. Первый корректор обнаружил $m_{1}$ опечаток, второй заметил $m_{2}$ опечаток. При этом $m$ опечаток оказались обнаруженными и первым, и вторым корректорами.
\begin{enumerate}
\item Постройте любым методом состоятельную оценку для общего числа опечаток, замеченных и незамеченных.
\item Является ли построенная оценка несмещенной?
\end{enumerate}

\item[14-B.] Вася хочет купить чудо-швабру! Магазинов, где продается чудо-швабра, бесконечно много. Любое посещение магазина связано с издержками равными $c>0$. Цена чудо-швабры в каждом магазине имеет равномерное распределение на отрезке $[0;M]$. Цены в магазинах не меняются, то есть при желании Вася может вернуться в уже посещенный им магазин для совершения покупки.
\begin{enumerate}
\item Как выглядит оптимальная стратегия Васи? Вася нейтрален к риску.
\item Каковы ожидаемые Васины затраты при использовании оптимальной стратегии?
\item Сколько магазинов в среднем будет посещено?
\end{enumerate}
\emph{Подсказка}: Думайте!
\end{enumerate}


\subsection[2007-2008. Демо]{\hyperref[sec:sol_kr_03_2007_2008_demo]{2007-2008. Демо}}
\label{sec:kr_03_2007_2008_demo}

\subsubsection*{Часть II.}

Стоимость задач 10 баллов.

\begin{enumerate}
\item Вася и Петя метают дротики по мишени. Каждый из них сделал
по 100 попыток. Вася оказался метче Пети в 59 попытках.
\begin{enumerate}
\item На уровне
значимости 5\% проверьте гипотезу о том, что меткость Васи и Пети
одинаковая, против альтернативной гипотезы о том, что Вася метче
Пети.
\item Чему равно точное $P$-значение при проверке гипотезы в п. «а»?
\end{enumerate}

% числа выверены
\item Из 10 опрошенных студентов часть предпочитала готовиться по
синему учебнику, а часть - по зеленому. В таблице представлены их
итоговые баллы.

\begin{tabular}{@{}lcccccc@{}}
\toprule
Синий   & 76 & 45 & 57 & 65 &    &    \\
Зелёный & 49 & 59 & 66 & 81 & 38 & 88 \\ \bottomrule
\end{tabular}


С помощью теста Манна-Уитни (Mann-Whitney) проверьте гипотезу о
том, что выбор учебника не меняет закона распределения оценки.

% числа выверены
\item Имеется случайная выборка $X_{1}$, $X_{2}$, \ldots, $X_{n}$, где все $X_{i}$ имеют распределение, задаваемое табличкой:

\begin{tabular}{@{}lccc@{}}
\toprule
$x$         & $1$ & $2$   & $5$     \\ \midrule
$\P(X=x)$ & $a$ & $0.1$ & $0.9-1$ \\ \bottomrule
\end{tabular}
\begin{enumerate}
\item Постройте оценку неизвестного $a$ методом моментов.
\item Является ли построенная оценка состоятельной?
\end{enumerate}

% числа выверены
\item Имеется случайная выборка $X_{1}$, $X_{2}$, \ldots, $X_{n}$, где все $X_{i}$ имеют $\cN(27,a)$ распределение.
Найдите оценку неизвестного $a$ методом максимального правдоподобия.

Напоминалка: не забудьте проверить условия второго порядка

% числа выверены
\item На курсе два потока, на первом потоке учатся 40 человек, на втором
потоке 50 человек. Средний балл за контрольную на первом потоке
равен 78 при (выборочном) стандартном отклонении в 7 баллов. На
втором потоке средний балл равен 74 при (выборочном) стандартном
отклонении в 8 баллов.
\begin{enumerate}
\item Постройте 90\% доверительный интервал для разницы баллов между
двумя потоками.
\item На 10\%-ом уровне значимости проверьте гипотезу о том, что
результаты контрольной между потоками не отличаются.
\end{enumerate}


% числа выверены
\item Проверьте независимость пола респондента и предпочитаемого
им сока:

\begin{tabular}{@{}cccc@{}}
\toprule
  & Апельсиновый & Томатный & Вишнёвый \\ \midrule
М & $69$         & $40$     & $23$     \\
Ж & $74$         & $62$     & $34$     \\ \bottomrule
\end{tabular}

% числа выверены
\item На Древе познания Добра и Зла растет 6 плодов познания Добра и 5 плодов познания Зла. Адам и Ева съели по 2 плода. Какова вероятность того, что Ева познала Зло, если Адам познал Добро?

 % числа выверены
\item Пусть $X_{i}$ — независимы и имеют функцию плотности $p(t)=e^{a-t}$ при $t>a$, где $a$ - неизвестный параметр. В качестве оценки неизвестного $a$ используется $\hat{a}_{n}=\min\{X_{1},X_{2}, \ldots, X_{n}\}$.
\begin{enumerate}
\item Является ли предлагаемая оценка состоятельной?
\item Является ли предлагаемая оценка несмещенной?
\end{enumerate}


\end{enumerate}

\subsubsection*{Часть III.}

Стоимость задачи 20 баллов.

Требуется решить \textbf{\underbar{одну}} из двух 9-х задач по
выбору!

\begin{enumerate}
\item[9-A.] Имеются две монетки. Одна правильная, другая — выпадает орлом с
вероятностью $0.45$. Одну из них, неизвестно какую, подкинули $n$
раз и сообщили Вам, сколько раз выпал орел. Ваша задача проверить
гипотезу $H_{0}$: «подбрасывалась правильная монетка» против
$H_{a}$:
«подбрасывалась неправильная монетка».

Каким должно быть наименьшее $n$ и критерий выбора гипотезы, чтобы
вероятность ошибок первого рода не превышала 10\%, а вероятность
ошибки второго рода не превышала 15\%?

\item[9-B.] Пусть величины $X_{i}$ — независимы и равномерны $\cU[-b;b]$. Имеется выборка из 2-х наблюдений. Вася строит оценку для $b$ по формуле $\hat{b}=c\cdot (|X_{1}|+|X_{2}|)$.
\begin{enumerate}
\item При каком $c$ оценка будет несмещенной?
\item При каком $c$ оценка будет минимизировать средне-квадратичную ошибку, $MSE=\E((\hat{b}-b)^{2})$?
\end{enumerate}
\end{enumerate}



\subsection[2007-2008]{\hyperref[sec:sol_kr_03_2007_2008]{2007-2008}}
\label{sec:kr_03_2007_2008}



\subsubsection*{Часть I.}

Какие утверждения являются истинными?

\begin{enumerate}
\item Мощность теста можно рассчитать заранее, до проведения теста.
\item Точное $P$-значение можно рассчитать заранее, до проведения теста.
\item Если гипотеза отвергает при 5\%-ом уровне значимости, то
она обязательно будет отвергаться и при 10\%-ом уровне значимости.
\item Мощность больше у того теста, у которого вероятность ошибки
1-го рода меньше.
\item Функция плотности $F$-распределения $p(t)$ не определена при $t<0$.
\item При большом $k$ случайную величину, имеющую $\chi_{k}^{2}$ распределение, можно считать нормально распределенной.
\item Оценки метода моментов всегда несмещенные.
\item Оценки метода максимального правдоподобия асимптотически несмещенные.
\item Непараметрические тесты можно использовать, даже если закон распределения выборки неизвестен.
\item Неравенство Крамера-Рао применимо только к оценкам метода максимального правдоподобия.
\end{enumerate}

Тест не является блокирующим.

Обозначения: $\E(X)$ — математическое ожидание, $\Var(X)$ — дисперсия.


\subsubsection*{Часть II.}

Стоимость задач 10 баллов.

\begin{enumerate}
\item Школьник Вася аккуратно замерял время, которое ему требовалось, чтобы добраться от школы до дома. По результатам 90 наблюдений, среднее выборочное оказалось равным 14 мин, а несмещенная оценка дисперсии — 5 мин$^{2}$.
\begin{enumerate}
\item Постройте 90\% доверительный интервал для среднего времени на дорогу.
\item На уровне значимости 10\% проверьте гипотезу о том, что среднее время равно 14,5 мин, против альтернативной гипотезы о меньшем времени.
\item Чему равно точное $P$-значение при проверке гипотезы в пункте «б»?
\end{enumerate}

% числа выверены
\item Садовник осматривал розовые кусты и записывал число цветков. Всего в саду растет 25 розовых кустов. Предположим, что количество цветков на разных кустах независимы и одинаково распределены.
Вот заметки садовника:

12, 17, 21, 14, 15; 21, 16, 24, 11, 14; 22, 17, 21, 14, 15; 12, 26, 14, 21, 14; 11, 31, 18, 13, 18.

Проверьте гипотезу о том, что медиана количества цветков равна 19.

% числа выверены
\item Имеется случайная выборка $X_{1}$, $X_{2}$, \ldots, $X_{n}$, где все $X_{i}$ имеют распределение, задаваемое табличкой:

\begin{tabular}{@{}cccc@{}}
\toprule
$x$         & $1$ & $2$  & $5$    \\ \midrule
$\P(X=x)$ & $a$ & $2a$ & $1-3a$ \\ \bottomrule
\end{tabular}
\begin{enumerate}
\item Постройте оценку неизвестного $a$ методом моментов.
\item Является ли построенная оценка несмещенной?
\end{enumerate}

% числа выверены
\item Имеется случайная выборка $X_{1}$, $X_{2}$, \ldots, $X_{n}$, где все $X_{i}$ имеют $\cN(a,4a)$ распределение.
Найдите оценку неизвестного $a$ методом максимального правдоподобия.

Напоминалка: не забудьте проверить условия второго порядка.

 % числа выверены
\item Допустим, что логарифм дохода семьи имеет нормальное распределение. В городе А была проведена случайная выборка 40 семей, показавшая выборочную дисперсию 20 (тыс.р.)$^{2}$. В городе Б по 30 семьям выборочная дисперсия оказалась равной 32 (тыс.р.)$^{2}$.
На уровне значимости 5\% проверьте гипотезу о том, что дисперсия (логарифма дохода) одинакова, против альтернативной гипотезы о том, что город А более однородный.

 % числа выверены
\item Учебная часть утверждает, что все три факультатива («Вязание крючком для экономистов», «Экономика вышивания крестиком» и «Статистические методы в макраме») одинаково популярны. В этом году на эти факультативы соответственно записалось 35, 31 и 40 человек. Правдоподобно ли заявление учебной части?

% может изменить одно из 0.7 на 0.6?
% числа выверены
\item Снайпер попадает в «яблочко» с вероятностью 0.8, если в предыдущий раз он попал в «яблочко» и с вероятностью 0.7, если в предыдущий раз он не попал в «яблочко» или если это был первый выстрел. Снайпер стрелял по мишени 3 раза.
\begin{enumerate}
\item Какова вероятность попадания в «яблочко» при втором выстреле?
\item Какова вероятность попадания в «яблочко» при втором выстреле, если при первом снайпер попал, а при третьем — промазал?
\end{enumerate}

% числа выверены
\item Пусть $X_{i}$ — независимы и распределены равномерно на $[a-1;a]$, где $a$ — неизвестный параметр. В качестве оценки неизвестного $a$ используется $\hat{a}_{n}=\max\{X_{1},X_{2},...,X_{n}\}$.
\begin{enumerate}
\item Является ли предлагаемая оценка состоятельной?
\item Является ли предлагаемая оценка несмещенной?
\end{enumerate}
\end{enumerate}

\subsubsection*{Часть III.}

Стоимость задачи 20 баллов.

Требуется решить \textbf{\underbar{одну}} из двух 9-х задач по
выбору!

\begin{enumerate}
\item[9-A.] Два лекарства испытывали на мужчинах и женщинах. Каждый
человек принимал только одно лекарство. Общий процент людей,
почувствовавших улучшение, больше среди принимавших лекарство А.
Процент мужчин, почувствовавших улучшение, больше среди принимавших лекарство В. Процент женщин, почувствовавших улучшение, больше среди принимавших лекарство В. Возможно ли это?

\item[9-B.] Есть два золотых слитка, разных по весу. Сначала взвесили первый слиток и получили результат $X$. Затем взвесили второй слиток и получили результат $Y$. Затем взвесили оба слитка и получили результат $Z$. Допустим, что ошибка каждого взвешивания — это случайная величина с нулевым средним и дисперсией $\sigma^{2}$.
\begin{enumerate}
\item Придумайте наилучшую оценку веса первого слитка.
\item Сравните придуманную Вами оценку с оценкой, получаемой путем усреднения двух взвешиваний первого слитка.
\end{enumerate}
\end{enumerate}


\subsection[2006-2007]{\hyperref[sec:sol_kr_03_2006_2007]{2006-2007}}
\label{sec:kr_03_2006_2007}

Нужные и ненужные формулы:


$T$ — сумма чего-то там. \\
Если $H_{0}$ верна, то $\E(T)=\frac{n}{2}$ и $\Var(T)=\frac{n}{4}$ \\ \\
$T$ — сумма каких-то рангов. \\
Если $H_{0}$ верна, то $\E(T)=\frac{n(n+1)}{4}$ и
$\Var(T)=\frac{n(n+1)(2n+1)}{24}$. \\ \\
$T$ — сумма каких-то рангов. \\
Если $H_{0}$ верна, то $\E(T)=\frac{n_{1}(n_{1}+n_{2}+1)}{2}$,
$\Var(T)=\frac{n_{1}n_{2}(n_{1}+n_{2}+1)}{12}$. \\ \\
$\cos[2](x)+\sin[2](x)=1$ \\

\textbf{УДАЧИ!}

\subsubsection*{Часть I.}

Обведите нужный ответ

\begin{enumerate}
\item Если $X\sim \cN(0;12)$, $Y\sim \cN(12,24)$, $\Corr(X,Y)=0$, то
$X+Y\sim \cN(12,36)$.
Да. Нет.

\item Если закон распределения $X$ задан табличкой

\begin{tabular}{@{}ccc@{}}
\toprule
$x$      & $0$   & $1$   \\ \midrule
$\P(X=x)$ & $0.5$ & $0.5$ \\ \bottomrule
\end{tabular}, то $X$ — нормально распределена. Да. Нет.

\item Непараметрические тесты неприменимы, если выборка имеет
$\chi^{2}$ распределение. Да. Нет.
\item P-значение показывает вероятность отвергнуть нулевую
гипотезу, когда она верна. Да. Нет.
\item Если $t$-статистика равна нулю, то P-значение также равно
нулю. Да. Нет.
\item Если гипотеза отвергает при 5\%-ом уровне значимости, то
она будет отвергаться и при 1\%-ом уровне значимости. Да. Нет.
\item При прочих равных 90\% доверительный интервал шире 95\%-го. Да. Нет.
\item Значение функции плотности может превышать единицу. Да. Нет.
\item Для любой случайной величины  $\E(X^{2} )\ge
(\E(X))^{2}$. Да. Нет.
\item Если $\Corr(X,Y)>0$, то $\E(X)\E(Y)<\E(XY)$. Да. Нет.
\item На экзаменационной работе не шутят! Нет, шутят.
\end{enumerate}

Ответ «да» означает истинное утверждение, ответ «нет» — ложное.

Тест не является блокирующим.


\subsubsection*{Часть II.}

Стоимость задач 10 баллов.

\begin{enumerate}
 % числа выверены
\item Из урны с 5 белыми и 7 черными шарами случайным образом вынимается
2 шара. Случайная величина $X$ принимает значение (-1), если оба
шара — белые; 0, если шары разного цвета и 1, если оба шара
черные.
\begin{enumerate}
\item Найдите $\P(X=-1)$, $\E(X)$, $\Var(X)$.
\item Постройте функцию распределения величины $X$.
\end{enumerate}

 % числа выверены
\item Случайная величина $X$ имеет функцию распределения
\[
F_{X}(t)=
\begin{cases}
  0, & t<0 \\
  ct^{2}, & 0\le t <1 \\
  1, & 1\le t \\
\end{cases}.
\]
\begin{enumerate}
\item Найдите $c$, $\P(0.5<X<2)$, 25\%-ый квантиль.
\item Найдите $\E(X)$, $\Var(X)$, $\Cov(X,-X)$, $\Corr(2X,3X)$.
\item Выпишите функцию плотности величины $X$.
\end{enumerate}

% числа выверены
\item Доходности акций двух компаний являются случайными величинами $X$
и $Y$ с одинаковым математическим ожиданием и ковариационной
матрицей  $\left(%
\begin{array}{cc}
  4 & -2 \\
  -2 & 9 \\
\end{array}%
\right).$
\begin{enumerate}
\item Найдите $\Corr(X,Y)$.
\item В какой пропорции нужно приобрести акции этих двух
компаний, чтобы дисперсия доходности получившегося портфеля была наименьшей?
\item Можно ли утверждать, что величины $X+Y$ и $7X-2Y$ независимы?
\item Изменится ли ответ на пункт «в», если дополнительно
известно, что величины $X$ и $Y$ в совокупности нормально распределены?
\end{enumerate}
Подсказка: Если $R$ — доходность портфеля, то $R=\alpha
X+(1-\alpha)Y$

% числа выверены
\item Проверка 40 случайно выбранных лекций показала, что студент
Халявин присутствовал только на двух из них.
\begin{enumerate}
\item Найдите 90\%-ый доверительный интервал для вероятности
увидеть Халявина на лекции.
\item Укажите минимальный размер выборки, необходимый для того,
чтобы с вероятностью 0.9 выборочная доля посещаемых Халявиным.
лекций отличалась от соответствующей вероятности не более, чем на 0.1.
\item Какие предпосылки и теоремы использовались при ответах на предыдущие пункты?
\end{enumerate}

% числа выверены
\item Изучается эффективность нового метода обучения. У группы из 40
студентов, обучавшихся по новой методике, средний бал на экзамене
составил 322.12, а выборочное стандартное отклонение 54.53.
Аналогичные показатели для независимой выборки из 60 студентов
того же курса, обучавшихся по старой методике,
приняли значения 304.61 и 62.61 соответственно.
\begin{enumerate}
\item Проверьте гипотезу о равенстве дисперсий оценок в двух
группах.
\item Какие предпосылки использовались при ответе на «а»?
\item Постройте 90\% доверительный интервал для разницы
математических ожиданий оценок в двух группах.
\item Можно ли считать новую методику более эффективной?
\end{enumerate}

% числа выверены
\item В парке отдыха за час 57 человек посетило аттракцион «Чертово
колесо», 48 — «Призрачные гонки» и 54 — «Американские горки».
Можно ли на 5\% уровне значимости считать, что посетители
одинаково любят эти три аттракциона?

% числа выверены
\item Можно ли по имеющейся таблице утверждать о независимости пола и
доминирующей руки на 5\% уровне значимости?

\begin{tabular}{@{}ccc@{}}
\toprule
Пол / Рука & Правша & Левша \\ \midrule
Мужчины    & $16$     & $76$    \\
Женщина    & $25$     & $81$ \\ \bottomrule
\end{tabular}

% числа выверены
\item Пусть $X_{i}$ нормально распределены, независимы, $\E(X_{i})=0$,
$\Var(X_{i})=\theta$.
\begin{enumerate}
\item Постройте оценку $\hat{\theta}$ методом максимального
правдоподобия.
\item Проверьте свойства несмещенности, состоятельности,
эффективности у построенной оценки.
\item Какая оценка более предпочтительна: построенная или
привычная
$\hat{\sigma}^{2}=\frac{\sum(X_{i}-\bar{X})^{2}}{n-1}$?
\end{enumerate}

% числа выверены
\item Имеются две конкурирующие гипотезы:
\begin{enumerate}
\item[$H_0$:] Случайная величина $X$ распределена равномерно на (0,100);
\item[$H_a$:] Случайная величина $X$ распределена равномерно на (50,150).
\end{enumerate}
Исследователь выбрал следующий критерий: если $X<c$, принимать гипотезу $H_0$, иначе  $H_a$.
\begin{enumerate}
\item Дайте определение ошибок первого и второго рода.
\item Постройте графики зависимостей ошибок первого и второго рода от $c$.
\end{enumerate}

%числа выверены
\item Вася измерил длину 10 пойманных им рыб. Часть рыб была поймана на
левом берегу реки, а часть — на правом. Бывалые рыбаки говорят,
что на правом берегу реки рыба крупнее.

\begin{tabular}{@{}lccccc@{}}
\toprule
Левый берег  & $25$ & $45$ & $37$ & $47$ & $51$ \\
Правый берег & $49$ & $28$ & $39$ & $46$ & $57$ \\ \bottomrule
\end{tabular}
\begin{enumerate}
\item С помощью теста Манна-Уитни (Mann-Whitney) проверьте
гипотезу о том, что выбор берега реки не влияет на среднюю длину
рыбы против
альтернативной гипотезы, что на правом берегу рыба длиннее.

\emph{Разрешается использование нормальной аппроксимации}
\item{} $[$Не оценивался$]$ Возможно ли в этой задаче использовать
(Wilcoxon Signed Rank Test)?
\end{enumerate}
\end{enumerate}

\subsubsection*{Часть III.}

Стоимость задачи 20 баллов.

Требуется решить \textbf{\underbar{одну}} из двух 11-х задач по
выбору!

\begin{enumerate}
\item[11-А.] Имеются две монетки. Одна правильная, другая — выпадает орлом с
вероятностью $0.45$. Одну из них, неизвестно какую, подкинули $n$
раз и сообщили Вам, сколько раз выпал орел. Ваша задача проверить
гипотезу $H_{0}$: «подбрасывалась правильная монетка» против
$H_{a}$:
«подбрасывалась неправильная монетка».

Каким должно быть наименьшее $n$ и критерий выбора гипотезы, чтобы
вероятность ошибок первого рода не превышала 10\%, а вероятность
ошибки второго рода не превышала 15\%?

\item[11-Б.] Время горения лампочки – экспоненциальная случайная величина с
математическим ожиданием равным $\theta $. Вася включил
одновременно 20 лампочек. Величина  $Y$ обозначает время самого
первого перегорания.
\begin{enumerate}
\item Найдите $\E(Y)$.
\item Постройте с помощью  $Y$ несмещённую оценку для  $\theta$.
\item Сравните по эффективности оценку построенную в пункте
«б» и обычное выборочное среднее.
\end{enumerate}
\end{enumerate}


\subsection[2005-2006]{\hyperref[sec:sol_kr_03_2005_2006]{2005-2006}}
\label{sec:kr_03_2005_2006}



Просто из сил выбьешься, пока вдруг как-то само не уладится;
что-то надо подчеркнуть, что-то — выбросить, не договорить, а
где-то — ошибиться, без ошибки такая пакость, что глядеть тошно. \\
В.А. Серов \\

\subsubsection*{Часть I.}

Проверьте, являются ли утверждения истинными:

\begin{enumerate}
\item Если $X\sim \chi_{n}^{2}$ и $Y\sim \chi_{n+1}^{2}$, $X$ и $Y$ -
независимы, то  $X$ не превосходит $Y$.
\item В тесте Манна-Уитни предполагается нормальность хотя бы одной
из сравниваемых выборок.
\item График функции плотности случайной величины, имеющей
$t$-распределение симметричен относительно 0.
\item Мощность больше у того теста, у которого вероятность ошибки
2-го рода меньше.
\item Если $X\sim t_{n}$, то $X^{2}\sim F_{1,n}$.
\item При прочих равных 90\% доверительный интервал шире 95\%-го.
\item Несмещенная выборочная оценка дисперсии не превосходит квадрата
выборочного среднего.
\item Если гипотеза отвергает при 5\%-ом уровне значимости, то она
будет отвергаться и при 1\%-ом уровне значимости.
\item У t-распределения более толстые «хвосты», чем у стандартного
нормального.
\item P-значение показывает вероятность отвергнуть нулевую гипотезу,
когда она верна.
\item Если t-статистика равна нулю, то P-значение также равно нулю.
\item Если $X\sim \cN(0;1)$, то $X^{2}\sim \chi_{1}^{2}$.
\item Пусть $X_{i}$ — длина $i$-го удава в сантиметрах, а $Y_{i}$ —
в дециметрах. Выборочный коэффициент корреляции между этими
наборами данных равен $\frac{1}{10}$.
\item Математическое ожидание выборочного среднего не зависит от
объема выборки, если $X_{i}$ одинаково распределены.
\item Зная закон распределения $X$ и закон распределения $Y$
можно восстановить совместный закон распределения пары $(X,Y)$.
\item Если ты отвечаешь на вопросы этого теста наугад, то число
правильных ответов — случайная величина, имеющая биномиальное
распределение с дисперсией $4$.
\end{enumerate}



\subsubsection*{Часть II.}

Стоимость задач 10 баллов.

\begin{enumerate}
\item Пусть случайная величина  $X$  распределена
равномерно на отрезке $\left[0;a\right]$, где  $a>3$ .
Исследователь хочет оценить параметр  $\theta =\P\left(X<3\right)$. Рассмотрим следующую оценку

$\hat{\theta
}=\left\{\begin{array}{l} {1,\; X<3} \\ {0,\; X\ge 3}
\end{array}\right. $
\begin{enumerate}
\item Объясните, что означают термины «несмещенность»,
«состоятельность», «эффективность».
\item Верно ли, что оценка $\hat{\theta}$ является несмещенной?
\item Найдите $\E\left(\left(\hat{\theta }-\theta \right)^{2} \right)$.
\end{enumerate}

\item Величины $X_{1}$, $X_{2}$, \ldots, $X_{n}$ независимы и их функции
плотности имеет вид:
\[
f(x)=
\begin{cases}
    (k+1)x^{k}, & x \in [0;1]; \\
    0, & x \notin [0;1].
\end{cases}
\]

\begin{enumerate}
\item Оцените $k$ методом максимального правдоподобия.
\item Оцените $k$ методом моментов.
\end{enumerate}

\item У 200 человек записали цвет глаз и волос. На уровне значимости
10\% проверьте гипотезу о независимости этих признаков.

\begin{center}
\begin{tabular}{@{}cccc@{}}
\toprule
Цвет глаз / волос & Светлые & Тёмные & Итого \\ \midrule
Зелёные           & 49      & 25     & 74    \\
Другие            & 30      & 96     & 126   \\ \midrule
Итого             & 79      & 121    & 200   \\ \bottomrule
\end{tabular}
\end{center}

\item На курсе два потока, на первом потоке учатся 40 человек, на втором
потоке 50 человек. Средний балл за контрольную на первом потоке
равен 78 при (выборочном) стандартном отклонении в 7 баллов. На
втором потоке средний балл равен 74 при (выборочном) стандартном
отклонении в 8 баллов.
\begin{enumerate}
\item Постройте 90\% доверительный интервал для разницы баллов между двумя потоками.
\item На 10\%-ом уровне значимости проверьте гипотезу о том, что результаты контрольной между потоками не отличаются.
\item Рассчитайте точное P-значение (P-value) теста в пункте «б».
\end{enumerate}

\item Предположим, что время жизни лампочки распределено нормально. По
10 лампочкам оценка стандартного отклонения времени жизни
оказалась равной 120 часам.
\begin{enumerate}
\item Найдите 80\%-ый двусторонний
доверительный интервал для истинного стандартного отклонения.
\item Допустим, что выборку увеличат до 20 лампочек. Какова
вероятность того, что выборочная оценка дисперсии будет отличаться
от истинной дисперсии меньше, чем на 40\%?
\end{enumerate}

\item Из 10 опрошенных студентов часть предпочитала готовиться по синему
учебнику, а часть — по зеленому. В таблице представлены их
итоговые баллы.

\begin{tabular}{@{}lcccccc@{}}
\toprule
Синий   & 76 & 45 & 57 & 65 &    &    \\
Зелёный & 49 & 59 & 66 & 81 & 38 & 88 \\ \bottomrule
\end{tabular}
\begin{enumerate}
\item С помощью теста Манна-Уитни (Mann-Whitney) проверьте
гипотезу о том, что выбор учебника не меняет закона распределения оценки.

\emph{Разрешается использование нормальной аппроксимации}

\item Возможно ли в этой задаче использовать Wilcoxon Signed Rank Test?
\end{enumerate}

\item Вася очень любит играть в преферанс. Предположим, что Васин
выигрыш распределен нормально. За последние 5 партий средний
выигрыш составил 1560 рублей, при оценке стандартного отклонения
равной 670 рублям. Постройте 90\%-ый доверительный интервал для
математического ожидания Васиного выигрыша.

\item Имеется две конкурирующие гипотезы:
\begin{enumerate}
\item[$H_{0}$:] Величина $X$ распределена равномерно на отрезке $[0;100]$;
\item[$H_{a}$:] Величина $X$ распределена равномерно на отрезке $[50;150]$.
\end{enumerate}
Исследователь выбрал такой критерей: если $X<c$, то использовать $H_{0}$, иначе использовать $H_{a}$.
\begin{enumerate}
\item Что такое «ошибка первого
рода», «ошибка второго рода»,
«мощность теста»?
\item Постройте графики зависимостей ошибок первого и второго рода от $c$.
\end{enumerate}

\item На плоскости выбирается точка со случайными координатами. Абсцисса
и ордината независимы и распределены $\cN(0;1)$. Какова вероятность
того, что расстояние от точки до начала координат будет больше
2.45?

\item С вероятностью 0.3 Вася оставил конспект в одной из 10 посещенных
им сегодня аудиторий. Вася осмотрел 7 из 10 аудиторий и конспекта в них не нашел.
\begin{enumerate}
\item Какова вероятность того, что конспект будет найден в
следующей осматриваемой им аудитории?
\item Какова (условная) вероятность того, что конспект оставлен
где-то в другом месте?
\end{enumerate}
\end{enumerate}

\subsubsection*{Часть III.}

Стоимость задачи 20 баллов.

Требуется решить {\bf \underbar{одну}} из двух 11-х задач по
выбору!

\begin{enumerate}
\item[11-А.] [Hardy-Weinberg theorem]

У диплоидных организмов наследственные характеристики определяются
парой генов. Вспомним знакомые нам с 9-го класса горошины чешского
монаха Менделя. Ген, определяющий форму горошины, имеет две
аллели:  'А' (гладкая) и 'а' (морщинистая). 'А' доминирует 'а'. В
популяции бесконечное количество организмов. Родители каждого
потомка определяются случайным образом, согласно имеющемуся
распределению генотипов. Одна аллель потомка выбирается наугад из
аллелей матери, другая - из аллелей отца. Начальное распределение
генотипов имеет вид: 'АА' - 30\%, 'Аа' - 60\%, 'аа' - 10\%.
\begin{enumerate}
\item Каким будет распределение генотипов в $n$-ом поколении?
\item Заметив закономерность, сформулируйте и докажите теорему
Харди-Вайнберга для произвольного начального распределения
генотипов.
\end{enumerate}

\item[11-Б.] В киосках продается «открытка-подарок». На открытке есть
прямоугольник размером 2 на 7. В каждом столбце в случайном
порядке находятся очередная буква слова «подарок» и звёздочка.
Например, вот так:

\begin{tabular}{|c|c|c|c|c|c|c|}
  \hline
  П & * & * & А & * & О & К \\
  \hline
  * & О & Д & * & Р & * & * \\
  \hline
\end{tabular}

Прямоугольник закрыт защитным слоем, и покупатель не видит, где
буква, а где звёздочка. Следует стереть защитный слой в одном
квадратике в каждом столбце. Можно попытаться угадать любое число
букв. Если открыто $n$ букв слова «подарок» и не открыто ни одной
звёздочки, то открытку можно обменять на $50\cdot 2^{n-1}$ рублей.
Если открыта хотя бы одна звёздочка, то открытка
остается просто открыткой.
\begin{enumerate}
\item Какой стратегии следует придерживаться покупателю, чтобы
максимизировать ожидаемый выигрыш?
\item Чему равен максимальный ожидаемый выигрыш?
\end{enumerate}
\emph{Подсказка}: Думайте!
\end{enumerate}
