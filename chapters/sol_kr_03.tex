% !TEX root = ../probability_hse_exams.tex
\thispagestyle{empty}
\section{Решения контрольной номер 3}



\subsection[2019-2020]{\hyperref[sec:kr_03_2019_2020]{2019-2020}}
\label{sec:sol_kr_03_2019_2020}



\subsection[2018-2019]{\hyperref[sec:kr_03_2018_2019]{2018-2019}}
\label{sec:sol_kr_03_2018_2019}
\subsubsection*{Задачи}
\begin{enumerate}
	
	%Задача №1
  \item
  \item
  \item
  \item
  \item
	\item 
	Пусть $X_1, \ldots, X_5$ — цены кокошников. $\E (X_i) = 3500, \Var (X_i) = 500^2 = 250000, \ i = 1, \ldots, 5$.

 Василиса выбирает наугад и равновероятно 3 кокошника, обозначим их цены за $Y_1, Y_2, Y_3$.

Математическое ожидание вырученных Василисой денег равно  
\[ 
\E (Y_1 + Y_2 + Y_3) = 
\E (Y_1)  + \E (Y_2) + \E (Y_3) = 3 \cdot 3500 = 10500.
\] 

 Аналогично, дисперсия вырученных Василисой денег равна

\[ 
\Var (Y_1 + Y_2 + Y_3) = \Var (Y_1) + \Var (Y_2) + \Var (Y_3) + 2 \Cov (Y_1, Y_2) + 2 \Cov (Y_1, Y_3) + 2 \Cov (Y_2, Y_3)
\]

 Найдем ковариации соответствующих случайных величин, используя следующее свойство: 
 \[ \Cov (X_1, X_1 + X_2 + \ldots + X_5) = 0\]

  Это верно в силу того, что $X_1 + X_2 + \ldots + X_5 = const $

 А так как ковариации $X_1$ со всеми остальными $X_i$ будут одинаковы, то 
\[  \Cov (X_1, X_1 + X_2 + \ldots + X_5) = \Var (X_1) + 4 \Cov (X_1, X_i) = 0, \] 

 Отсюда получаем 
\[ \Cov (X_1, X_i) = - \frac{1}{4} \cdot \Var (X_1) \]
 
Получим: 
\[
 \Var (Y_1 + Y_2 + Y_3) = 3 \cdot \Var (Y_1) - 3 \cdot 2 \cdot \frac{1}{4} \cdot \Var (Y_1) 
  = \frac{3}{2}  \Var (Y_1) = \frac{3}{2} \cdot 250000 = 375000 
 \]
\item
В задаче один оцениваемый параметр $\theta$. Тогда для получения MM-оценки достаточно одного момента.

Выборочный первый начальный момент:
\[
\frac{\sum_{0}^{n}{x_{i}}}{n} = \bar X
\]

Теоретический первый начальный момент: 
\[
\E (X) = \int_{0}^{\theta} x \cdot \frac{2x}{\theta^{2}} dx = \left. \frac{4x}{\theta^{2}} \right|_{0}^{\theta} = \frac{4}{\theta}
\]
Приравняв выборочный и теоретический моменты, получаем $\hat \theta_{MM}=\frac{4}{\bar X}$
\item
\item

\item

\[
L(X, \theta)  = f(X_1, \theta) \cdot \ldots \cdot f(X_n, \theta) = \frac{1}{\theta^n}, \text{ если } X_i < \theta \text{ для любого } i 
\]

Значит $\hat{\theta}$ должна быть наименьшей при которой она ещё остаётся больше каждого $X_i$, $\hat{\theta} = \max\{X_1, \ldots, X_n\}$.

\end{enumerate}


\subsection[2017-2018]{\hyperref[sec:kr_03_2017_2018]{2017-2018}}
\label{sec:sol_kr_03_2017_2018}


\begin{enumerate}
\item[5.]
\begin{enumerate}
\item $L(X_1, \ldots, X_n, \mu) = \prod_{i=1}^n \frac{1}{\sqrt{2\pi}} e^{-\frac{1}{2}\sum_{i=1}^n (X_i - \mu)^2}$
\item $\hat\mu_{ML} = \bar X$
\item $\E(\hat\mu_{ML}) = \E(\bar X) = \mu \Rightarrow$ оценка несмещённая

$\plim \hat \mu_{ML} = \plim \bar X = \mu \Rightarrow$ оценка состоятельная
\item $I(\mu) = n$
\item $\Var(\theta) \geq \frac{1}{I(\theta)}$
\item $\Var(\hat \mu_{ML}) = \frac{1}{n}$, так как неравенство Рао-Крамера выполнено
как равенство, оценка является эффективной.
\item $\theta = \E\left(X^2\right) = \Var(X) + \mu^2 = 1 + \mu^2$.
Тогда в силу инвариантности оценок максимального правдоподобия: $\hat\theta_{ML} = 1 + \hat\mu^2$.
\item $\E(\hat \theta_{ML}) = 1 + \E(\hat \mu^2) = 1 + \E((\bar X)^2)$

Пользуясь соотношением $\E((\bar X)^2) = \Var(\bar X) + (\E(\bar X))^2$,
получим: $\E(\hat \theta_{ML}) = 1 + \frac{1}{n} + \mu^2$, то есть оценка смещена.

Однако, $\lim_{n \to \infty} \left(1 + \frac{1}{n} + \mu^2\right) = 1 + \mu^2$, значит,
оценка асимптотически несмещена.
\item $\hat \theta_{ML} \approx 1 + \mu^2 + 2\mu(\hat \mu - \mu)$

$\Var(\hat \theta_{ML}) \approx 4 \mu^2 \Var(\hat \mu) = \frac{4 \mu^2}{n}$
\item Так как $\hat \theta_{ML}$ асимптотически несмещена, то для проверки
состоятельности достаточно показать, что
$\Var(\hat \theta_{ML}) = \frac{4\mu^2}{n} \to_{n \to \infty} 0$.
\end{enumerate}
\item[6.]
\begin{enumerate}
\item $\E(X_1) = \int_{0}^{\theta} \frac{2}{\theta^2}(\theta - x)x dx = \frac{\theta}{3}$

$\frac{\hat \theta_{MM}}{3} = \bar X \Rightarrow \hat \theta_{MM} = 3 \bar X$
\item Оценка $\hat \theta$ состоятельна. если $\plim \hat \theta_n = \theta$.

$\plim \hat \theta_{MM} = \plim 3 \bar X = 3 \E(X_1) = \theta \Rightarrow$ оценка состоятельна.
\end{enumerate}
\item[7.]
\begin{enumerate}
\item $\E\left(\frac{X_1 + X_2 + X_3}{3} \right) = \frac{1}{3} \cdot 3 \E(X_1) = 132.5$

$\Var\left(\frac{X_1 + X_2 + X_3}{3} \right) = \frac{1}{9} \Var(X_1 + X_2 + X_3) =
\frac{1}{9} (\Var(X_1) + \Var(X_2) + \Var(X_3) + 2 \Cov(X_1, X_2) + 2\Cov(X_1, X_3) + 2\Cov(X_2, X_3)) =
\frac{1}{9}(3\Var(X_1) + 6\Cov(X_1,X_2))$

$\Var(X_1) = \E(X_1^2) - \E(X_1)^2 = \frac{1}{4} \cdot 30^2 + \frac{1}{4} \cdot 500^2 - 132.5^2 = 45168.75$

$\Cov(X_1, X_1 + \ldots + X_4 = \Var(X_1) + 3\Cov(X_1,X_2) = 0 \Rightarrow \Cov(X_1,X_2) = -\frac{45168.75}{3} = -15056.25$

$\Var\left(\frac{X_1 + X_2 + X_3}{3} \right) = 5018.75$

\item Вовочке удастся войти в метро, учитывая, что стоимость проезда по тройке составляет 35~рублей, только если одной из выбранных карт будет карта с 500~рублями на счету.

Всего Вовочка может выбрать три карты $C_4^3$ способами. Если одна из выбранных карт $-$ карта с~500~рублями, то выбрать оставшиеся две карты можно $C_3^2$ способами.

Тогда вероятность того, что одной из выбранных карт будет карта с 500 рублями равна $\displaystyle\frac{C_3^2}{C_4^3} = \frac{3}{4}$

\end{enumerate}
\item[8.] $\Delta_i = X_i - Y_i \sim \cN(\mu_x - \mu_y, \sigma^2)$

$\bar X = 297.5$, $\bar Y = 247.5$, $\bar \Delta = \bar X - \bar Y = 50$

$\hat \sigma^2 = \frac{1}{n-1} \sum_{i=1}^n (\Delta_i - \bar \Delta)^2 = 18266.(6)$.

Критическое значение — $t_{0.975, 3} = 3.182$ и доверительный интервал имеет вид:
\[
50 - 3.182 \sqrt{\frac{18266.(6)}{4}} < \mu_x - \mu_y < 50 + 3.182 \sqrt{\frac{18266.(6)}{4}}
\]
Так как $0$ входит в доверительный интервал, нельзя отвергнуть предположение о равенстве расхожов.
\item[9.]
\begin{enumerate}
\item $0.7 - 1.96 \sqrt{\frac{0.7 \cdot 0.3}{60}} < p < 0.7 + 1.96 \sqrt{\frac{0.7 \cdot 0.3}{60}} $
\item Да, так как $0.7667$ входит в доверительный интервал.
\item $\P(|p - \hat p| \leq 0.01) = 0.95$

$\P\left(\frac{|0.7 - p|}{\sqrt{\frac{0.7 \cdot 0.3}{n}}} < \frac{0.01}{\sqrt{\frac{0.7 \cdot 0.3}{n}}} \right) = 0.95$

$\frac{0.01}{\sqrt{\frac{0.7 \cdot 0.3}{n}}} = 1.96 \Rightarrow n \approx 8068$
\end{enumerate}
\end{enumerate}


\subsection[2016-2017]{\hyperref[sec:kr_03_2016_2017]{2016-2017}}
\label{sec:sol_kr_03_2016_2017}


\begin{enumerate}
\item
\begin{enumerate}
\item $-2, 1, 4, 7, 10$
\item $4$
\item $S^2 = \frac{1}{n} \sum_{i=1}^n (X_i - \bar{X})^2 = 18$
\item $\hat\sigma^2 = \frac{1}{n-1} \sum_{i=1}^n (X_i - \bar{X})^2 = 22.5$
\item $\frac{1}{n} \sum_{i=1}^n X_i^2 = 34$
\item $F_n(x) = \begin{cases}
0, & x < -2 \\
\frac{1}{5}, & -2 \leq x < 1 \\
\frac{2}{5}, & 1 \leq x < 4 \\
\frac{3}{5}, & 4 \leq x < 7 \\
\frac{4}{5}, & 7 \leq x < 10 \\
1, & x \geq 1
\end{cases}$
\end{enumerate}
\item $\E(X_1 + X_2) = 2 \cdot 11000 = 22000$

$\Var(X_1+X_2) = \Var(X_1) + \Var(X_2) + 2\Cov(X_1, X_2) = \Var(X_1) + \Var(X_2) - \frac{2\Var(X_1)}{N-1} = 2 \cdot 3000 - \frac{2\cdot3000}{3-1} = 3000$
\item
\begin{enumerate}
\item Необходимо решить следующую задачу:
\[
\begin{cases}
\frac{0.4^2 \cdot 10^2}{n_1} + \frac{0.5^2 \cdot 30^2}{n_2} + \frac{0.1^2 \cdot 60^2}{n_3} \to \min_{n_1, n_2, n_3} \\
150 n_1 + 300 n_2 + 600 n_3 \leq 30000
\end{cases}
\]
Выпишем функцию Лагранжа и найдём её частные производные по $n_1$, $n_2$ и $n_3$:
\begin{align*}
L(n_1, n_2, n_3, \lambda) &= \frac{0.4^2 \cdot 10^2}{n_1} + \frac{0.5^2 \cdot 30^2}{n_2} + \frac{0.1^2 \cdot 60^2}{n_3} + \lambda (150 n_1 + 300 n_2 + 600 n_3 - 30000) \\
\frac{\partial L}{\partial n_1} &= -\frac{0.4^2 \cdot 10^2}{n_1^2} + 150 \lambda \quad \Rightarrow \quad 150 \lambda = \frac{0.4^2 \cdot 10^2}{n_1^2} \\
\frac{\partial L}{\partial n_2} &= -\frac{0.5^2 \cdot 30^2}{n_2^2} + 300 \lambda \quad \Rightarrow \quad 150 \lambda = \frac{0.5^2 \cdot 30^2}{2n_2^2} \\
\frac{\partial L}{\partial n_2} &= -\frac{0.1^2 \cdot 60^2}{n_3^2} + 600 \lambda \quad \Rightarrow \quad 150 \lambda = \frac{0.1^2 \cdot 60^2}{4n_3^2}
\end{align*}
Выразим $n_2$ и $n_3$ через $n_1$:
\begin{align*}
\frac{0.4 \cdot 10}{n_1} = \frac{0.5 \cdot 30}{\sqrt{2}n_2} \Rightarrow n_2 = \frac{15n_1}{4\sqrt{2}} \\
\frac{0.4 \cdot 10}{n_1} = \frac{0.1 \cdot 60}{2n_3} \Rightarrow n_3 = \frac{6n_1}{8}
\end{align*}
Подставим вcё в бюджетное ограничение:
\[
150 n_1 + 300 \cdot \frac{15n_1}{4\sqrt{2}} + 600 \cdot \frac{6n_1}{8} = 30000
\]
Откуда получаем: $n_1 = 21.5 \approx 22$, $n_2 \approx 57$, $n_3 \approx 16$.
\item
$\Var(\bar{X}_S) = \sum_{l=1}^L \frac{w_l^2 \cdot \sigma_l^2}{n_l}
= \frac{0.4^2 \cdot 10^2}{22} + \frac{0.5^2 \cdot 30^2}{57} + \frac{0.1^2 \cdot 60^2}{16}
\approx 6.92$
\end{enumerate}
\item $\hat{p} = \frac{8000}{12300000} = \frac{2}{3075}$,
$\sqrt{\frac{\hat{p}(1-\hat{p})}{n}} \approx 7.27 \cdot 10^{-6}$, $z_{\frac{\alpha}{2}} = 1.96$

$\frac{2}{3075} - 1.96 \cdot 7.27 \cdot 10^{-6} < p < \frac{2}{3075} + 1.96 \cdot 7.27 \cdot 10^{-6}$

$0.00064 < p < 0.00066$

Поскольку $0$ не входит в доверительный интервал, утверждать, что доля статистически
не отличается от нуля нельзя.

\item
\begin{enumerate}
  \item
  \begin{enumerate}
    \item $\bar{Y} = 43$, $\hat{\sigma}_Y^2 = 32.5$, $t_{0.005, 4} = 4.6$

    $43 - 4.6 \cdot \sqrt{\frac{32.5}{5}} < \mu < 43 + 4.6 \cdot \sqrt{\frac{32.5}{5}}$

    $31.27 < \mu < 54.72$

    \item $\chi^2_{0.95, 4} = 9.49$, $\chi^2_{0.05, 4} = 0.71$

    $\frac{32.5 \cdot 4}{9.49} < \sigma^2 < \frac{32.5 \cdot 4}{0.71}$

    $13.7 < \sigma^2 < 183$
  \end{enumerate}
  \item
  \begin{enumerate}
  \item
  $X_1, \ldots, X_{n_X} \sim \cN(\mu_X, \sigma^2_X)$, $Y_1, \ldots, Y_{n_Y} \sim \cN(\mu_Y, \sigma^2_Y)$,
  $\sigma^2_X = \sigma^2_Y = \sigma^2_0$, выборки независимы

  \item $\bar{Y} - \bar{X} = 43 - 37 = 6$

  $\hat{\sigma}^2_0 = \frac{\sum_{i=1}^{n_X} (X_i - \bar X)^2 + \sum_{i=1}^{n_Y} (Y_i - \bar Y)^2}{n_X + n_Y - 2} = \frac{680+130}{5+5-2} = 101.25$

  $t_{0.95, 8} = 1.86$

  $6 - 1.86 \sqrt{101.25} \sqrt{\frac{1}{5} + \frac{1}{5}} < \mu_Y - \mu_X <  6 + 1.86 \sqrt{101.25} \sqrt{\frac{1}{5} + \frac{1}{5}} $

  $-5.83 < \mu_Y - \mu_X < 17.83$
  \item Да, так как ноль входит в доверительный интервал.
  \end{enumerate}
\end{enumerate}
\item
\begin{enumerate}
  \item Выборочный второй начальный момент: $\frac{1}{n} \sum_{i=1}^n X_i^2$.

  Теоретический второй начальный момент: $\E\left(X^2\right) = \Var(X) + (\E X)^2 = \theta$

  $\hat{\theta}_{MM} = \frac{1}{n} \sum_{i=1}^n X_i^2$
  \item $\E(\hat{\theta}_{MM}) = \frac{1}{n} \sum_{i=1}^n \E(X_i^2) = \theta$ —
  оценка несмещённая.
  \item $\Var(\hat{\theta}_{MM}) = \Var\left(\frac{1}{n} \sum_{i=1}^n X_i^2 \right) = \frac{1}{n^2} \sum_{i=1}^n \Var(X_i^2) = \frac{3\theta^2 - \theta^2}{n} \underset{n \to \infty}{\to} 0$
  — оценка состоятельная ($\E(X^4) = 3\theta^2$).
  \item
  \begin{align*}
    L(x,\theta) &= \prod_{i=1}^n \frac{1}{\sqrt{2\pi\theta}} \exp\left(-\frac{1}{2}\frac{x_i^2}{\theta} \right) = \frac{1}{(\sqrt{2\pi\theta})^n} \exp \left(-\frac{1}{2\theta} \sum_{i=1}^n x_i^2  \right) \\
    l(x, \theta) &= -\frac{n}{2}\ln(2\pi) - \frac{n}{2}\ln\theta -\frac{1}{2\theta} \sum_{i=1}^n x_i^2 \\
    \frac{\partial l}{\partial \theta} &= -\frac{n}{2\theta} + \frac{1}{2\theta^2} \sum_{i=1}^n x_i^2 \\
    \hat{\theta}_{ML} &= \frac{\sum_{i=1}^n x_i^2}{n}
  \end{align*}
  \item \begin{align*}
    \frac{\partial^2 l}{\partial \theta^2} &= \frac{n}{2\theta^2} - \frac{1}{\theta^3} \sum_{i=1}^n x_i^2 \\
    -\E\left( \frac{\partial^2 l}{\partial \theta^2} \right) &= -\frac{n}{2\theta^2} + \frac{1}{\theta^3} \cdot n \theta = \frac{n}{2\theta^2} \\
    I(\theta) &= \frac{n}{2\theta^2}
  \end{align*}
  \item $\Var(\hat{\theta}) \geq \frac{1}{I(\theta)}$
  \item $\Var(\hat{\theta}_{ML}) = \Var\left(\frac{\sum_{i=1}^n x_i^2}{n}\right) = \frac{1}{n^2}\cdot n \Var(X_1^2) = \frac{1}{n} (\E(X_1^4) - \E(X_1^2)^2) = \frac{2\theta^2}{n}$

  Так как $\Var(\hat{\theta}_{ML}) = \frac{1}{I(\theta)}$, $\hat{\theta}_{ML}$ — эффективная оценка.
\end{enumerate}
\item
\begin{enumerate}
\item Вспомним, что для распределения Пуассона $\E(X) = \Var(X) = \lambda$
\begin{align*}
  L(x, \lambda) &= \prod_{i=1}^n e^{-\lambda} \frac{\lambda^{x_i}}{x_i!} = e^{-n\lambda} \lambda^{\sum_{i=1}^n x_i} \prod_{i=1}^n \frac{1}{x_i!} \\
  l(x, \lambda) &= -n\lambda + \ln\lambda \sum_{i=1}^n x_i - \sum_{i=1}^n \ln x_i! \\
  \frac{\partial l}{\partial \lambda} &= -n + \frac{1}{\lambda} \sum_{i=1}^n x_i \\
  \hat{\lambda}_{ML} &= \bar{X}
\end{align*}
Значение по выборке: $\bar{X} = 14.5$
\item см. предыдущий пункт
\item $\hat \sigma^2 = \sqrt{\lambda_{ML}} = \sqrt{14.5}$
\item $\P(X=0) = \frac{\lambda^0 e^{-\lambda}}{0!} = e^{-\lambda} \Rightarrow \widehat{\P(X=0)} = e^{-\hat\lambda} = e^{-\bar{X}}$
\item $\left[14.5 - 1.96 \sqrt{\frac{14.5}{6}}; 14.5 + 1.96 \sqrt{\frac{14.5}{6}}\right]$, где $1.96$ — критическое значение $\cN(0;1)$. Конечно, этот результат верен только при больших $n$. Мы усиленно делаем вид, что $n=6$ велико. Полученный нами интервал может быть довольно далёк от 95\%-го.
\item В данном случае: $g(\hat{\lambda}) = e^{-\hat\lambda}$, $g'(\hat\lambda) = -e^{-\hat\lambda}$.
И доверительный интервал имеет вид:
\begin{align*}
  \left[e^{-\bar{X}} - 1.96 \sqrt{\frac{e^{-2\bar{X}}\bar{X}}{n}}; e^{-\bar{X}} + 1.96 \sqrt{\frac{e^{-2\bar{X}}\bar{X}}{n}} \right] \\
  \left[e^{-14.5} - 1.96 \sqrt{\frac{e^{-29}14.5}{6}}; e^{-14.5} + 1.96 \sqrt{\frac{e^{-29}14.5}{6}} \right]
\end{align*}
Снова отметим, что наш интервал может на самом деле быть далеко не 95\%-ым, так наше $n=6$ мало для серьёзного применения метода максимального правдоподобия.
\end{enumerate}
\end{enumerate}



\subsection[2015-2016]{\hyperref[sec:kr_03_2015_2016]{2015-2016}}
\label{sec:sol_kr_03_2015_2016}

\begin{enumerate}
\item Пусть случайная величина $S$ – это сумма поглощённых калорий

\begin{center}
\begin{tabular}{cccc}
\toprule
$s$ & $650$ & $800$ & $950$ \\
$\P(S = s)$ & $1/3$ & $1/3$ & $1/3$ \\ \bottomrule
\end{tabular}
\end{center}

Тогда
\begin{align*}
\E(S) &= \frac{1}{3}\cdot 650 +  \frac{1}{3}\cdot 800 +  \frac{1}{3}\cdot 950 = 800 \\
\Var(S) &= \frac{1}{3}(650-800)^2 + \frac{1}{3}(800-800)^2 + \frac{1}{3}(950-800)^2 = 15000
\end{align*}
\item Вариационный ряд: $4, 6, 11$; медиана: $6$; выборочное среднее: $7$;
несмещённая оценка дисперсии: $13$
\item Функция плотности двумерного нормального распределения имеет вид:
\begin{align*}
f(x,y) &= \frac{1}{2\pi}\cdot \frac{1}{\sigma_x \sigma_y \sqrt{1-\rho^2}} \\
&\cdot
\exp\left\{{-\frac{1}{2}\frac{1}{\sigma_x^2 \sigma_y^2\left(1-\rho^2\right)}\left[\sigma_y^2(x-\mu_x)^2-2\rho\sigma_x\sigma_y(x-\mu_x)(y-\mu_y)+\sigma_x^2(y-\mu_y)^2\right]}\right\}
\end{align*}

Откуда: $\mu_X=1$, $\mu_Y=0$, $\sigma_X = 1$, $\sigma_Y = 1$, $\rho = 0.2$

Обратите внимание, при скобке c $x$ коэффициент $\sigma^2_y$, там потом делится на их произведение!

\item
\begin{enumerate}
\item $X \sim \cN(178, 49)$
\begin{align*}
\P(X>185) &= 1  - \P(X<185) = 1- \P\left(\frac{X-178}{7} < \frac{185-178}{7}\right) \\
&= 1 - 0.8413 = 0.1587
\end{align*}
\item Нет, так как $\Cov(X, Y) = 5.6 \neq 0$
\item $Y \mid X \sim \cN\left(\mu_Y + \rho\sigma_Y\cdot\frac{X-\mu_X}{\sigma_X}; \sigma_Y^2\left(1-\rho^2\right)\right)$

$Y \mid X=185 \sim \cN(42.8;0.36)$

$\P(Y<42 \mid X=185) = \P\left(\frac{Y-42.8}{0.6} < \frac{42-42.8}{0.6}\mid X=185\right) \approx 0.09$
\end{enumerate}

\item
\begin{enumerate}
\item $\E(X) = \frac{0+2\theta}{2}\mid_{\hat{\theta}} = \bar{X}$, $\hat{\theta}_{MM} = \bar{X}$
\item $\forall \theta \in \Theta: \E\left(\hat{\theta}\right)=\theta \Rightarrow \hat{\theta}$ – несмещённая.

$\forall \theta \in \Theta, \forall \epsilon > 0 : \P\left(\vert \widehat{\theta}_n - \theta \vert > \epsilon\right) \to 0 \Rightarrow  \widehat{\theta}_n$ – состоятельная.

$\forall \theta \in \Theta: I_n^{-1} (\theta) = \Var\left(\hat{\theta}\right) \Rightarrow \hat{\theta} $ – эффективная.
\item $\E(\theta) = \E(\bar{X}) = \E(X_1) = \theta \Rightarrow \hat{\theta}$ – несмещённая оценка

$\Var\left(\hat{\theta_n}\right) = \Var\left(\bar{X}\right) = \frac{\Var(X_1)}{n} =
\frac{4\theta^2}{12\cdot n} \underset{n \to \infty}{\to} 0$; из условий
$\E\left(\widehat{\theta}_n\right) = \theta$ и $\Var\left(\widehat{\theta}_n\right)
\underset{n \to \infty}{\to} 0$ следует, что $\widehat{\theta}_n \stackrel{\P}{\to}
\theta$ при $n \to \infty$, т.е. $\widehat{\theta}_n$ является состоятельной.

\item
\begin{align*}
F_{X_{(n)}} &= \P(\max(X_1, \ldots, X_n) \leq x) = \P(X_1 \leq x) \cdot \ldots \cdot \P(X_n \leq x) = (\P(X_1 \leq x))^n \\
&= \begin{cases}
0 & \text{при } x<0 \\
\left(\frac{x}{2\theta}\right)^n & \text{при }  x \in [0, 2\theta] \\
1 & \text{при }  x > 2\theta
\end{cases}
\end{align*}

\[
f_{X_{(n)}} (x)  = \begin{cases}
0 & \text{при } x<0 \\
\frac{nx^{n-1}}{2^n \theta^n} & \text{при }  x \in [0, 2\theta] \\
0 & \text{при }  x > 2\theta
\end{cases}
\]

\begin{align*}
\E(X_{(n)}) &= \int_{-\infty}^{+\infty} x \cdot f_{X_{(n)}} (x) dx =  \int_{0}^{2\theta}	x \cdot \frac{nx^{n-1}}{2^n \theta^n} dx = \left. \frac{n}{2^n \theta^n} \cdot \frac{x^{n+1}}{n+1} \right|_{x=0}^{x=2\theta} \\
&= \frac{n}{2^n \theta^n}  \cdot \frac{2^{n+1}\cdot \theta^{n+1}}{n+1} = \frac{n2\theta}{n+1}
\end{align*}
Следовательно, $\E \left(\frac{n+1}{2n} \cdot X_{(n)}\right) = \theta$, а значит, $\tilde{\theta} = \frac{n+1}{2n} \cdot X_{(n)}$ – несмещённая оценка вида $c \cdot  X_{(n)}$
\item $\Var\left(\tilde{\theta}\right) = \frac{(n+1)^2}{4n^2} \Var(X_{(n)})$

\begin{align*}
\E\left(X_{(n)}^2\right) &= \int_{-\infty}^{+\infty} x^2 f_{X_{(n)}} (x) dx = \int_{0}^{2\theta} x^2 \frac{nx^{n-1}}{2^n \theta^n}  dx = \frac{n}{2^n \theta^n}  \int_{0}^{2\theta} x^{n+1} dx \\
&= \left. \frac{n}{2^n \theta^n} \cdot \frac{x^{n+2}}{n+2} \right|_{x=0}^{x=2\theta} = \frac{n}{2^n \theta^n} \cdot  \frac{2^{n+2}\cdot \theta^{n+2}}{n+2} = \frac{n\cdot4\cdot\theta^2}{n+2}
\end{align*}

\[
\Var(X_{(n)}) = \E\left(X_{(n)}^2\right)  - (\E(X_{(n)}))^2 = \frac{4n\theta^2}{n+2} - \frac{4 n^2 \cdot \theta^2}{(n+1)^2} = 4n\theta^2 \left(\frac{1}{n+2} - \frac{n}{(n+1)^2}\right)
\]

\[
\Var\left(\tilde{\theta}\right) = \frac{(n+1)^2}{4n^2} \Var(X_{(n)}) = \frac{(n+1)^2}{4n^2}  \cdot 4n\theta^2 \left(\frac{n^2+2n+1 - n^2-2n}{(n+2)(n+1)^2} \right) = \frac{\theta^2}{n(n+2)}
\]
Оценка $\tilde{\theta}_n$ является состоятельной, так как
$\E\left(\tilde{\theta}_n\right) = \theta$ и
$\Var\left(\tilde{\theta}_n\right) = \frac{\theta^2}{n(n+2)} \underset{n \to \infty}{\to} 0$
\item  Поскольку $\Var\left(\widehat{\theta}_n\right) = \frac{\theta^2}{3n}$,
$\Var\left(\tilde{\theta}_n\right) = \frac{\theta^2}{n(n+2)}$ при достаточно большом $n$
$\Var\left(\tilde{\theta}_n\right) < \Var\left(\widehat{\theta}_n\right)$.
Значит, при таких $n$ оценка $\tilde{\theta}_n$ будет более эффективной по сравнению
с оценкой $\widehat{\theta}_n$.
\end{enumerate}

\item
\begin{enumerate}
\item $X_i \sim Bin (n=10, p)$
\item $L(p)  = \prod_{i=1}^{n} C_{10}^{x_i} p^{x_i} (1-p)^{10-x_i}$
\item $\ln L(p) = \sum_{i=1}^{n} \ln C_{10}^{x_i} + \sum_{i=1}^n x_i \ln p + \sum_{i=1}^{n} (10-x_i)\ln (1-p) \to \max_p$

$\frac{\partial \ln L}{\partial p} = \frac{\sum_{i=1}^n x_i}{p} - \frac{\sum_{i=1}^{n} (10-x_i)}{1-p} \mid_{p=\hat{p}} = 0 \Rightarrow \hat{p} = \frac{\bar{X}}{n} = \frac{\sum_{i=1}^{n} x_i}{10n}$

$\frac{\partial^2 \ln L}{\partial p^2} = -\frac{\sum_{i=1}^n x_i}{p^2} -  \frac{\sum_{i=1}^{n} (10-x_i)}{(1-p)^2}$

\item $I(p) = -\E \left(\frac{\partial^2 \ln L}{\partial p^2}  \right) = \E \left(\frac{\sum_{i=1}^n x_i}{p^2} + \frac{\sum_{i=1}^{n} (10-x_i)}{(1-p)^2}\right) = \frac{10np}{p^2} + \frac{10n - 10np}{(1-p)^2} = \frac{10n}{p(1-p)}$

$i(p) = \frac{I(p)}{n} = \frac{10}{p(1-p)}$

\item $\Var(T) \geq \frac{1}{ni(T)}$

\item $\Var(\hat{p}_{ML}) = \Var\left(\frac{\sum_{i=1}^{n} x_i}{10n}\right) = \frac{1}{(10n)^2} n \Var(X_i) = \frac{1}{100n}10p(1-p) = \frac{p(1-p)}{10n}$

$\frac{p(1-p)}{10n} = \frac{1}{\frac{10n}{p(1-p)}} \Rightarrow$ да

\item $\E(X_i) = 10p \Rightarrow \widehat{\E(X_i)} = 10 \hat{p}_{ML} = \bar{X}$

$\Var(X_i) = 10p(1-p) \Rightarrow \widehat{\Var(X_i)} = \bar{X}\left(1-\frac{\bar{X}}{10}\right)$

\item $\hat{p} = \frac{3+4+0+2+6}{10\cdot 5} = 0.3$
\end{enumerate}

\item $L(x, \theta) = \prod_{i=1}^{n} (1 + \theta) x_i^\theta = (1+\theta)^n \prod_{i=1}^n x_i^\theta \to \max_\theta$

$\ln L (x, \theta) = n\ln (1+\theta) + \theta\sum_{i=1}^{n} \ln x_i \to \max_\theta$

$\frac{\partial \ln L}{\partial \theta} = \frac{n}{1+\theta} + \sum_{i=1}^{n} \ln x_i \mid_{\theta=\hat{\theta}} = 0 \Rightarrow \hat{\theta}_{ML} = -\frac{1}{\sum_{i=1}^{n} \ln x_i} -1$

\item $\bar{X} - 1.96 \frac{7}{\sqrt{20}} < \mu <\bar{X} + 1.96 \frac{7}{\sqrt{20}} $
\end{enumerate}



\subsection[2014-2015]{\hyperref[sec:kr_03_2014_2015]{2014-2015}}
\label{sec:sol_kr_03_2014_2015}

\begin{enumerate}
\item Пусть случайная величина $S$ – это сумма поглощённых калорий

\begin{center}
\begin{tabular}{cccc}
\toprule
$s$ & $650$ & $800$ & $950$ \\
$\P(S=s)$ & $1/3$ & $1/3$ & $1/3$ \\ \bottomrule
\end{tabular}
\end{center}

Тогда

\begin{align*}
\E(S) &= \frac{1}{3}\cdot 650 + \frac{1}{3}\cdot 800 +  \frac{1}{3}\cdot 950 = 800 \\
\Var(S) &= \frac{1}{3}(650-800)^2 + \frac{1}{3}(800-800)^2 + \frac{1}{3}(950-800)^2 = 15000
\end{align*}

\item Ответ — решение оптимизационной задачи:
\[
\begin{cases}
\Var(\bar{X}_S) = \frac{0.3^2\cdot 10^2}{n_1} + \frac{0.6^2 \cdot 30^2}{n_2} + \frac{0.1^2 \cdot 60^2}{n_3} \to min_{n_1, n_2, n_3} \\
150\cdot n_1 + 300 \cdot n_2 + 600 \cdot n_3 \leq 15000
\end{cases}
\]

\item
\begin{enumerate}
\item $\E(\hat \mu_1) = \E\left(\frac{X_1+X_2}{2}\right)  = \frac{1}{2}(\mu+\mu) = \mu \Rightarrow$  несмещённая

$\E(\hat \mu_2) = \E \left(\frac{X_1}{4} + \frac{X_2+\ldots+X_{n-1}}{2n-4} + \frac{X_n}{4}\right) = \frac{1}{4}\mu + \frac{n-2}{2n-4}\mu + \frac{1}{4}\mu = \mu \Rightarrow$ несмещённая

$\E(\bar{X}) = \E\left(\frac{X_1 + \ldots + X_n}{n}\right) = \mu \Rightarrow$ несмещённая

\item $\Var(\hat \mu_1) = \Var\left(\frac{X_1+X_2}{2}\right)  = \frac{1}{4}2\Var(X_1) = \frac{\sigma^2}{2}$

$\Var(\hat \mu_2) = \Var \left(\frac{X_1}{4} + \frac{X_2+\ldots+X_{n-1}}{2n-4} + \frac{X_n}{4}\right)  = \frac{\sigma^2}{16} + \frac{(n-2)\sigma^2}{(2n-4)^2} + \frac{\sigma^2}{16} = \sigma^2\left(\frac{1}{8} + \frac{1}{2(2n-4)} \right)$

$\Var(\hat \mu_3) = \Var\left(\frac{X_1 + \ldots + X_n}{n}\right)  = \frac{1}{n^2}n\sigma^2 = \frac{\sigma^2}{n}$
\end{enumerate}

\item
\begin{enumerate}
\item $X \sim \cN (1;1)$

$\P(X>1) = 0.5$, так как нормальное распределение симметрично относительно своего
математического ожидания.

\item $X \sim \cN (1;1)$, $2X \sim \cN(2; 4), Y \sim \cN(2, 4) \Rightarrow 2X+Y \sim \cN (4, 4)$

$\P(2X+Y > 2) = 1 - \P(2X+Y < 2) = 1 - \P\left(\frac{2X+Y - 4}{2} < \frac{2-4}{2}\right) \approx 1 - 0.16 = 0.84$

\item $Y \mid X \sim \cN\left(\mu_Y + \rho\sigma_Y\cdot\frac{X-\mu_X}{\sigma_X};
\sigma_Y^2\left(1 - \rho^2\right)\right)$, $(Y \mid X = 2) \sim \cN(1, 3)$

$\E(2X+Y \mid X=2) = 2\E (X\mid X=2) + \E(Y\mid X=2) = 4 + 1 = 5$
\end{enumerate}

\item
\begin{enumerate}
\item $X_1^2 + X_2^2 \sim \chi^2_2$, $\P(X_1^2 + X_2^2 > 6)  \approx 0.05$
\item $\P\left(\frac{X_1^2}{X_2^2+X_3^2} > 9.25\right) = \P\left(\frac{\frac{X_1^2}{1}}{\frac{X_2^2+X_3^2}{2}} > 18.5\right) \approx 0.05$, $\frac{\frac{X_1^2}{1}}{\frac{X_2^2+X_3^2}{2}} \sim F_{1, 2}$
\end{enumerate}
\item
\begin{enumerate}
\item \begin{align*}
F_{X_{(n)}} &= \P(\max(X_1, \ldots, X_n) \leq x) = \P(X_1 \leq x) \cdot \ldots \cdot \P(X_n \leq x) = (\P(X_1 \leq x))^n \\
&= \begin{cases}
0 & \text{при } x<0 \\
\left(\frac{x}{\theta}\right)^n & \text{при }  x \in [0, \theta] \\
1 & \text{при }  x > \theta
\end{cases}
\end{align*}

\[
f_{X_{(n)}} (x)  = \begin{cases}
0 & \text{при } x<0 \\
\frac{nx^{n-1}}{ \theta^n} & \text{при }  x \in [0, \theta] \\
0 & \text{при }  x > \theta
\end{cases}
\]

\begin{align*}
\E(X_{(n)}) &= \int_{-\infty}^{+\infty} x \cdot f_{X_{(n)}} (x) dx =  \int_{0}^{\theta}	x \cdot \frac{nx^{n-1}}{\theta^n} dx = \left. \frac{n}{\theta^n} \cdot \frac{x^{n+1}}{n+1} \right|_{x=0}^{x=\theta} \\
&= \frac{n}{\theta^n}  \cdot \frac{\cdot \theta^{n+1}}{n+1} = \frac{n\theta}{n+1}
\end{align*}
Следовательно, $\E \left(\frac{n+1}{n} \cdot X_{(n)}\right) = \theta$, а значит,
$\hat{\theta} = \frac{n+1}{n} \cdot X_{(n)}$ – несмещённая оценка вида $c \cdot  X_{(n)}$.

\item $\Var\left(\hat{\theta}\right) = \frac{(n+1)^2}{n^2} \Var(X_{(n)})$

\begin{align*}
\E\left(X_{(n)}^2\right) &= \int_{-\infty}^{+\infty} x^2 f_{X_{(n)}} (x) dx = \int_{0}^{\theta} x^2 \frac{nx^{n-1}}{\theta^n}  dx = \frac{n}{\theta^n}  \int_{0}^{\theta} x^{n+1} dx \\
&= \left. \frac{n}{\theta^n} \cdot \frac{x^{n+2}}{n+2} \right|_{x=0}^{x=\theta} = \frac{n}{\theta^n} \cdot  \frac{\theta^{n+2}}{n+2} = \frac{n\cdot\theta^2}{n+2}
\end{align*}

\[
\Var(X_{(n)}) = \E\left(X_{(n)}^2\right)  - (\E(X_{(n)}))^2 = \frac{n\theta^2}{n+2} - \frac{n^2 \cdot \theta^2}{(n+1)^2} = n\theta^2 \left(\frac{1}{n+2} - \frac{n}{(n+1)^2}\right)
\]

\[
\Var\left(\tilde{\theta}\right) = \frac{(n+1)^2}{n^2} \Var(X_{(n)}) = \frac{(n+1)^2}{n^2}  \cdot n\theta^2 \left(\frac{n^2+2n+1 - n^2-2n}{(n+2)(n+1)^2} \right) = \frac{\theta^2}{n(n+2)}
\]
Оценка $\hat{\theta}_n$ является состоятельной, так как $\E(\hat{\theta}_n) = \theta$ и $\Var(\hat{\theta}_n) = \frac{\theta^2}{n(n+2)} \underset{n \to \infty}{\to} 0$

\item $\E(X_1) = \left. \frac{\theta}{2} \right|_{\theta = \hat{\theta}_{MM}} = \bar{X} \Rightarrow \hat{\theta}_{MM} = 2\bar{X}$
\item $\Var\left(2\bar{X}\right) = \frac{4}{n^2}n\Var(X_1) = \frac{4\theta}{12n} = \frac{\theta}{3n} > \Var\left(\hat{\theta}_n\right)$
\end{enumerate}
\item
\begin{enumerate}
\item $L(x, p) = \prod_{i=1}^n \P(X_i = x_i) = p^{\sum_{i=1}^n(x_i-1)} (1-p)^n$

$\ln L (x, p) = \sum_{i=1}^n(x_i-1) \ln p  + n\ln(1-p)$

$\frac{\partial \ln L}{\partial p} = \frac{\sum_{i=1}^n(x_i-1)}{p} - \frac{n}{1-p} = 0 \Rightarrow \hat{p} = \frac{\sum_{i=1}^n x_i - n}{\sum_{i=1}^n x_i}$
\item $\E(X) = \frac{n}{p} \Rightarrow \hat{\E}(X) = \frac{n\sum_{i=1}^n x_i}{\sum_{i=1}^n x_i - n}$
\end{enumerate}
\end{enumerate}

% TODO: check typesetting below this line!!!!

\subsection[2011-2012]{\hyperref[sec:kr_03_2011_2012]{2011-2012}}
\label{sec:sol_kr_03_2011_2012}


\begin{enumerate}
\item
\begin{enumerate}
\item $L (x, \lambda) = \prod_{i=1}^{n}\lambda e^{-\lambda x_i} = \lambda^n e^{-\lambda \sum_{i=1}^{n}x_i}$

$\ln L (x, \lambda) = n\ln\lambda - \lambda\sum_{i=1}^n x_i \to \max_\lambda$

$\frac{\partial \ln L}{\partial \lambda} = \frac{n}{\lambda} - \sum_{i=1}^{n}x_i \mid_{\lambda = \hat{\lambda}} = 0 \Rightarrow \hat{\lambda}_{ML} = \frac{1}{\bar{X}}$

$\frac{\partial^2 \ln L}{\partial \lambda^2} = -\frac{n}{\lambda^2} \mid_{\lambda=\hat{\lambda}} < 0$
\item $\hat{a} = \bar{X}$
\item $\E(\hat{a}) = \E(\bar{X}) = \frac{1}{\lambda} \Rightarrow$  несмещённая
\item $\Var(\hat{a}) = \Var(\bar{X}) = \frac{1}{n}n\Var(X_i) = \frac{1}{n\lambda^2} \to_{n-\to\infty} 0 \Rightarrow$ состоятельная
\item
\item $\E(X) = \frac{1}{\lambda} \mid_{\lambda = \hat{\lambda}_{MM}} = \bar{X} \Rightarrow \lambda_{MM} = \frac{1}{\bar{X}}$
\end{enumerate}
\item
\begin{enumerate}
\item Истинное стандартное отклонение неизвестно, поэтому используем распределение Стьюдента:

$\bar{X} - t_{0.025, 99}\frac{\hat{\sigma}}{\sqrt{n}} < \mu < \bar{X} + t_{0.025, 99}\frac{\hat{\sigma}}{\sqrt{n}}$

$9.5 - 1.98 \frac{0.5}{\sqrt{100}} < \mu < 9.5 - 1.98 \frac{0.5}{\sqrt{100}} $

$9.4 < \mu < 9.6$
\item  Значение 10 не лежит в доверительном интервале, значит, гипотеза отвергается

$t_{obs} = \frac{9.5-10}{0.5/\sqrt{100}} = 10 \Rightarrow \text{p-value} \approx 0$
\end{enumerate}
\item
\begin{enumerate}
\item $\frac{\hat{\sigma}_\alpha^2}{\hat{\sigma}_\beta^2} \sim F_{n_{\alpha-1}, n_{\beta-1}}$

$F_{obs} = \frac{0.5}{0.6} \approx 0.83$, $\text{p-value}/ 2\approx 0.35 \Rightarrow$ на любом разумном уровне значимости оснований отвергать $H_0$ нет
\item $\hat{\sigma}_0^2 = \frac{\hat{\sigma}_\alpha^2(n_\alpha -1) + \hat{\sigma}_\beta^2(n_\beta-1)}{n_\alpha+n_\beta-2} = \frac{0.25\cdot19 + 0.36\cdot24}{20+25-2}=0.31$

$\frac{\bar{X} - \overline{Y}}{\hat{\sigma}_0 \sqrt{\frac{1}{n_\alpha}+\frac{1}{n_\beta}}} \sim t_{n_\alpha+n_\beta-2}$

$t_{obs} = \frac{9.5-9.8}{0.56\sqrt{\frac{1}{20}+ \frac{1}{25}}} = -1.79$, $\text{p-value}/2=0.04$

\end{enumerate}
\end{enumerate}


\subsection[2010-2011]{\hyperref[sec:kr_03_2010_2011]{2010-2011}}
\label{sec:sol_kr_03_2010_2011}


\begin{enumerate}
\item
\begin{enumerate}
\item Введём обозначения для событий:
\begin{itemize}
  \item $A$ — заболеть
  \item $\bar A$ — не заболеть
  \item $B$ — привиться
  \item $\bar B$ — не привиться
\end{itemize}

Тогда вероятность заболеть можно получить по формуле полной вероятности:
\[
\P(A) = \P(A | B) \cdot \P(B)  + \P(A | \bar B) \cdot \P(\bar B) = 0.15 \cdot 0.1 + 0.2 \cdot 0.9 = 0.195
\]
Значит, во время эпидемии заболевает $19.5\%$ людей.
\item Пусть всё население составляет $n$ человек.
Тогда доля заболевших — $0.195 n$.
Доля привитых заболевших — $0.015 n$.
Значит, среди заболевших
\[
\frac{0.015 n}{0.195 n} \cdot 100 \% \approx 7.7 \%
\]
привитых.
\end{enumerate}
\item
\begin{enumerate}
\item
\begin{align*}
\P(X > 4) &= \P\left(\frac{X - 3}{\sqrt{25}} > \frac{4 - 3}{\sqrt{25}} \right) = \P\left(\cN(0,1) > 0.2\right) \approx 0.42 \\
\P(4 < X \leq 5) &= \P \left(\frac{4 - 3}{\sqrt{25}} < \frac{X - 3}{\sqrt{25}} \leq \frac{5 - 3}{\sqrt{25}} \right)  \\
&= \P(0.2 < \cN(0,1) \leq 0.4) \approx 0.076
\end{align*}
\item Найдём математическое ожидание и дисперсию случайной величины $S = X - 2Y$:
\begin{align*}
\E(S) &= \E(X) - 2\E(Y) = 3 - 2 \cdot 1 = 1 \\
\Var(S) &= \Var(S)  + 4 \Var(Y) - 4 \Cov(X, Y) = 57
\end{align*}
Теперь можно посчитать вероятность:
\[
\P(X - 2Y < 4) = \P \left(\frac{S - 1}{\sqrt{57}} < \frac{4 - 1}{\sqrt{57}} \right) = \P(\cN(0, 1) < 0.4) \approx 0.66
\]
\item Замечаем, что дисперсия суммы равна сумме дисперсий. 
В силу совокупной нормальности величины $S$ и $Z$ независимы, а потому искомая условная вероятность равна безусловной,
$\P(X - 2Y < 4 | Z > 8 ) = \P(X - 2Y < 4) \approx 0.66$.
\end{enumerate}
\item
\begin{enumerate}
\item Сначала найдём среднее и несмещённую оценку дисперсии по выборке $Y$:
\[
\bar Y = \frac{480}{12} 40, \quad \hat{\sigma}^2_Y = \frac{358.3 \cdot 12}{11} \approx 391
\]
Доверительный интервал для математического ожидания при неизвестной дисперсии имеет вид:
\begin{align*}
\bar Y - t_{\frac{\alpha}{2}, n-1} \frac{\hat{\sigma}_Y}{\sqrt{n}} < \mu_Y < \bar Y + t_{\frac{\alpha}{2}, n-1} \frac{\hat{\sigma}_Y}{\sqrt{n}}
\end{align*}
Критическое значение $t_{0.95, 11} \approx 1.8$:
\[
40 - 1.8 \cdot \sqrt{\frac{391}{12}} < \mu_Y < 40 + 1.8 \cdot \sqrt{\frac{391}{12}}
\]
\item Найдём также среднее и несмещённую оценку дисперсии по выборке $X$:
\[
\bar X = \frac{540}{15} = 36, \quad \hat{\sigma}^2_X = \frac{410.264 \cdot 15}{14} \approx 440
\]
И посчитаем оценку общей дисперсии:
\[
\hat{\sigma}^2_0 = \frac{\sum_{i=1}^{n_X} \left(X_i - \bar X\right)^2 + \sum_{i=1}^{n_Y} \left(Y_i - \bar Y\right)^2}{n_X + n_Y - 2} =  \frac{410.264 \cdot 15 + 358.3 \cdot 12}{15 + 12 - 2} \approx 418
\]
Тогда, проверяя гипотезу $H_0: \mu_X = \mu_Y$ против альтернативной $H_a: \mu_X < \mu_Y$,
получим следующие наблюдаемое и критическое значения статистики:
\begin{align*}
t_{obs} &= \frac{\mu_X - \mu_Y}{\sqrt{\sigma^2_0 \left(\frac{1}{n_X} + \frac{1}{n_Y}\right)}} = \frac{36 - 40}{\sqrt{418 \cdot \left(\frac{1}{15} + \frac{1}{12}\right)}} \approx -0.51 \\
t_{crit, 0.05, 25} &\approx -1.7
\end{align*}
Так как $t_{obs} > t_{crit}$, нет оснований отвергать $H_0$.
\item Выпишем вариационный ряд по двум выборкам, выделяя наблюдения, относящиеся к
Юго-Западному округу, и присвоим им ранги:

\begin{center}
\begin{tabular}{@{}lcccccccccc@{}}
\toprule
Наблюдение & $8.4$           & $\textbf{14.4}$ & $15.2$          & $15.6$          & $\textbf{18.0}$ & $19.1$          & $21.2$          & $\textbf{22.0}$ & $\textbf{23.9}$ & $\textbf{26.6}$ \\
Ранг       & $1$             & $2$             & $3$             & $4$             & $5$             & $6$             & $7$             & $8$             & $9$             & $10$            \\ \midrule
Наблюдение & $28.2$          & $28.3$          & $\textbf{32.0}$ & $33.8$          & $34.5$          & $38.2$          & $\textbf{43.3}$ & $44.1$          & $45.0$          & $\textbf{46.7}$ \\
Ранг       & $11$            & $12$            & $13$            & $14$            & $15$            & $16$            & $17$            & $18$            & $19$            & $20$            \\ \midrule
Наблюдение & $\textbf{54.8}$ & $56.0$          & $\textbf{64.0}$ & $\textbf{65.1}$ & $68.2$          & $\textbf{69.2}$ & $84.2$ \\
Ранг       & $21$            & $22$            & $23$            & $24$            & $25$            & $26$            & $27$   \\ \bottomrule
\end{tabular}
\end{center}

Теперь посчитаем сумму рангов по выборке меньшего размера, то есть по Юго-Западному округу:
\[
T = \sum_{i=1}^{n_Y} R_i = 2 + 5 + 8 + 9 + 10 + 13 + 17 + 20 + 21 + 23 + 24 + 26 = 178
\]
Осталось найти значение наблюдаемой статистики и критическое значение:
\begin{align*}
\gamma_{obs} &= \frac{T - \frac{1}{2} n_X (n_X + n_Y + 1)}{\sqrt{\frac{1}{12} n_X n_Y (n_X + n_Y)} } = \frac{178 - \frac{1}{2} \cdot 12 (12 + 15 + 1)}{\sqrt{\frac{1}{12} \cdot 12 \cdot 15 \cdot(12 + 15)}} \approx 0.5 \\
\gamma_{crit} &= 1.96
\end{align*}
Так как $\gamma_{obs} < |\gamma_{crit}|$, нет оснований отвергать $H_0$.
\end{enumerate}
\item Будем проверять гипотезу о том, что вероятность падения маслом вверх равна $0.5$:
\[
\begin{cases}
H_0: p = 0.5 \\
H_a: p \neq 0.5
\end{cases}
\]
Найдём наблюдаемое и критическое значения статистики:
\begin{align*}
z_{obs} &= \frac{\hat p - p_0}{\sqrt{\frac{\hat p (1 - \hat p)}{n}}} = \frac{\frac{105}{200} - \frac{1}{2}}{\sqrt{\frac{\frac{105}{200} \cdot \frac{95}{200}}{200}}} \approx 0.71 \\
z_{crit} &= 1.96
\end{align*}
Так как $z_{obs} < |z_{crit}|$, нет оснований отвергать $H_0$.

\item
\begin{enumerate}
\item
\begin{align*}
\bar X &= \E(X_1) |_{\mu_1 = \hat{\mu}_1} = \hat{\mu}_1 \\
S^2 &= \Var(X_1) |_{\mu_1 = \hat{\mu}_1, \mu_2= \hat{\mu}_2} = \hat{\mu}_2 - \hat{\mu}_1^2 \Rightarrow \hat{\mu}_2 = \bar{X^2}
\end{align*}
\item Найдём максимум логарифмической функции правдоподобия по параметру $\theta$:
\begin{align*}
L &= \prod_{i=1}^n \frac{1}{\sqrt{2\pi}} \exp \left(-\frac{1}{2} (x_i - \theta)^2\right) \\
\ell &= \frac{n}{2} \ln 2\pi - \frac{1}{2} \sum_{i=1}^n (x_i - \theta)^2 \\
\frac{\partial \ell}{\partial \theta} &= \left. \sum_{i=1}^n (x_i - \theta) \right|_{\theta = \hat \theta} = 0 \\
\hat \theta &= \bar X
\end{align*}
Для проверки эффективности достаточно показать, что неравенство Рао-Крамера
для полученной оценки выполняется как равенство.

Найдём информацию Фишера:
\begin{align*}
\frac{\partial^2 \ell}{\partial \theta^2} &= -n \\
I(\theta) &= n
\end{align*}
и дисперсию оценки $\hat \theta$:
\[
\Var\left(\hat \theta \right) = \frac{1}{n}
\]
Подставив полученные значения в неравенство Рао-Крамера
\[
\Var\left(\hat \theta \right) \geq \frac{1}{I(\theta)},
\]
получим равенство, значит, найденная оценка является эффективной.
\end{enumerate}
\end{enumerate}

\subsection[2009-2010]{\hyperref[sec:kr_03_2009_2010]{2009-2010}}
\label{sec:sol_kr_03_2009_2010}


\begin{enumerate}
  \item 
  \item 
  \item 
  \item 
  \item 
  \item Пусть $S_i = X_i Y_i$. Замечаем, что $\E(S_i)=\E(X_i)\E(Y_i)=ab$, $\E(S_i)=\E(X_i^2)\E(Y_i^2)=(a^2+1)(b^2+1)$. Отсюда получаем систему
  \[
  \begin{cases}
  \hat a \hat b = \frac{\sum S_i}{n} \\
  (\hat a^2 + 1) (\hat b^2 + 1) = \frac{\sum S_i^2}{n} \\
  \end{cases}
  \]
  Для заданных чисел решением будет $\hat a = 2$ и $\hat b = 18$, или наоборот.
\end{enumerate}


\subsection[2008-2009]{\hyperref[sec:kr_03_2008_2009]{2008-2009}}
\label{sec:sol_kr_03_2008_2009}

\subsection[2007-2008. Демо]{\hyperref[sec:kr_03_2007_2008_demo]{2007-2008. Демо}}
\label{sec:sol_kr_03_2007_2008_demo}


Заметим, что $\hat{a}_{n}\geq a$.

$\P(|\hat{a}_{n}-a|>\varepsilon)=\P(\hat{a}_{n} a>\varepsilon)=
\P(\hat{a}_{n}>a+\varepsilon)=\P(\min\{X_{1},X_{2},\ldots,X_{n}\}>a+\varepsilon)= \\
=\P(X_{1}>a+\varepsilon \cap X_{2}> a+\varepsilon\cap \ldots)=
\P(X_{1}>a+\varepsilon)\cdot \P(X_{2}>a+\varepsilon)\cdot \ldots=
\left(\int_{a+\varepsilon}^{\infty}e^{a-t}dt\right)^{n}=\left(e^{-\varepsilon}\right)^{n}=e^{-n\varepsilon}$

$\lim_{n\to\infty} e^{-n\varepsilon} =0$

Оценка смещена при любых $n$, хотя смещение с ростом $n$ убывает

\subsection[2007-2008]{\hyperref[sec:kr_03_2007_2008]{2007-2008}}
\label{sec:sol_kr_03_2007_2008}

Да, Нет, Да, Нет, Нет, Да, Нет, Да, Да, Нет


\begin{enumerate}
\item
\begin{enumerate}
\item $[13.61;14.39]$
\item Отвергается ($t_{obs} = -2.12$, $t_{crit} = -1.29$)
\item $P_{value} \approx 0.017$
\end{enumerate}
\item Заменяем числа на цифры 0 и 1 (0 — меньше 19 цветков), (1 — больше).

$\hat{p}=\frac{8}{25}=0.32$

$H_{0}$: $p=0.5$

$H_{a}$: $p\neq 0.5$

$Z=\frac{0.32-0.5}{\sqrt{\frac{0.5\cdot 0.5}{25}}}=-1.8$

При уровне значимости 5\%, $Z_{critical}=1.96$. Значит, гипотеза $H_{0}$ не отвергается.
\item
\begin{enumerate}
\item $\hat{a}=\frac{5-\bar{X}}{10}$
\item Да, является
\end{enumerate}
\item $L=-\frac{n}{2}\ln(a)-\frac{na}{8}-\frac{\sum X_{i}^{2}}{8a}+c$

$L'=0$ равносильно $\hat{a}^{2}+4\hat{a}+4=4+\frac{\sum X_{i}^{2}}{n}$

$\hat{a}=-2+\sqrt{4+\frac{\sum X_{i}^{2}}{n}}$
\item $F_{29.39}=\frac{32}{20}=1.6$

$F_{crit}=1.74$

Гипотеза о том, что дисперсия одинакова не отвергается.
\item $\chi^{2}_{observed}=1.15$

$\chi^{2}_{2,5\%}=5.99$

Правдоподобно.
\item
\begin{enumerate}
\item $p=0.7\cdot 0.8+ 0.3\cdot 0.7=0.77$
\item $p=\frac{0.7\cdot0.8\cdot0.2}{0.7\cdot 0.8\cdot 0.2 + 0.7\cdot 0.2 \cdot 0.3}=\frac{8}{11}$
\end{enumerate}
\item
\begin{enumerate}
\item Заметим, что $\hat{a}_{n}\leq a$.
\begin{multline*}
\P(|\hat{a}_{n}-a|>\varepsilon)=\P(-(\hat{a}_{n}-a)>\varepsilon)=\P(\hat{a}_{n}<a-\varepsilon)=
\P(\max\{X_{1},X_{2}, \ldots, X_{n}\}<a-\varepsilon)= \\
=\P(X_{1}<a-\varepsilon \cap X_{2}< a-\varepsilon\cap \ldots)=
\P(X_{1}<a-\varepsilon)\cdot \P(X_{2}<a-\varepsilon)\cdot \ldots=(1-\varepsilon)^{n}
\end{multline*}
$\lim_{n\to\infty} (1-\varepsilon)^{n} =0$
\item Нет, не является ни при каких $n$, хотя смещение с ростом $n$ убывает.
\end{enumerate}
\item[9-А.] Да, \url{http://en.wikipedia.org/wiki/Simpson's_paradox}
\item[9-Б.]
\begin{enumerate}
\item Пусть истинные веса слитков равны $x$, $y$ и $z$.

Назовем оценку буквой $\hat{x}$, $\hat{x}=aX+bY+cZ$.

Несмещённость: $\E(\hat{x})=a\E(X)+b\E(Y)+c\E(Z)=ax+by+c(x+y)=x$

$a+c=1$, $b+c=0$

$\hat{x}=(1-c)X+(-c)Y+cZ$

Эффективность: $\Var(\hat{x})=((1-c)^{2}+c^{2}+c^{2})\cdot \sigma^{2}=(3c^{2}-2c+1)\sigma^{2}$

Чтобы минимизировать дисперсию, нужно выбрать $c=1/3$.

То есть $\hat{x}=\frac{2}{3}X-\frac{1}{3}Y+\frac{1}{3}Z$.
\item $\Var(\hat{x})=\frac{2}{3}\sigma^{2}$

$\Var\left(\frac{X_{1}+X_{2}}{2}\right)=\frac{1}{2}\sigma^{2}$

Усреднение двух взвешиваний первого слитка лучше.
\end{enumerate}
\end{enumerate}



\subsection[2006-2007]{\hyperref[sec:kr_03_2006_2007]{2006-2007}}
\label{sec:sol_kr_03_2006_2007}

Верный ответ на первый вопрос теста — нет! Некоррелированные одномерные нормальные распределения
могут быть не нормальными в совокупности.

\begin{enumerate}
\item
\begin{enumerate}
\item $\P(X=-1) = \frac{5}{12}\cdot\frac{4}{11} = \frac{5}{33}$

$\E(X) = -1 \cdot \frac{5}{33} + 0 \cdot 2 \cdot \frac{5}{12}\cdot\frac{7}{11} + 1 \cdot \frac{7}{12}\cdot\frac{6}{11} = \frac{1}{6}$

$\E(X^2) = (-1)^2 \cdot \frac{5}{33} + 1^2 \cdot \frac{7}{12}\cdot\frac{6}{11} = \frac{31}{66}$

$\Var(X) = \frac{175}{396}$
\item $F(x) = \begin{cases}
0, & x < -1 \\
\frac{5}{33}, & -1 \leq x < 0 \\
\frac{15}{22}, & 0 \leq x < 1 \\
1, & x \geq 1
\end{cases}$
\end{enumerate}
\item
\begin{enumerate}
\item $c=1$, $\P(0.5<X<2) = 0.75$, $0.5$
\item $\E(X) = \frac{2}{3}$, $\Var(X) = \frac{1}{18}$, $\Cov(X,-X) = - \frac{1}{18}$,
$\Corr(2X,3X) = 1$
\item $f(x) = \begin{cases}
0, & t < 0, t \geq 1 \\
2t, & 0 \leq t < 1
\end{cases}$
\end{enumerate}
\item
\begin{enumerate}
\item $\Corr(X,Y)=-\frac{1}{3}$
\item $\alpha=\frac{11}{17}$
\item Нет
\item Да
\end{enumerate}
\item
\begin{enumerate}
\item $\hat{p} = \frac{1}{20}$

$\left[\frac{1}{20} - 1.65 \cdot \sqrt{\frac{\frac{1}{20}\cdot\frac{19}{20}}{40}}; \frac{1}{20} + 1.65 \cdot \sqrt{\frac{\frac{1}{20}\cdot\frac{19}{20}}{40}}  \right]$
\item $\P(\vert \hat{p} - p \vert \leq 0.1) = 0.9 \Rightarrow \P\left(\frac{\vert \hat{p} - p \vert}{\sqrt{\frac{\hat{p}(1-\hat{p})}{n}}} \leq \frac{0.1}{\sqrt{\frac{\hat{p}(1-\hat{p})}{n}}} \right) = 0.9 \Rightarrow \frac{0.1}{\sqrt{\frac{\frac{1\cdot19}{20^2}}{n}}} = 1.65 \Rightarrow n \approx 13$
\end{enumerate}
\item
\begin{enumerate}
\item $\gamma_{obs} = \frac{\hat{\sigma}^2_Y}{\hat{\sigma}^2_X} \approx 1.32$, $\gamma_{crit, 0.95} \approx 1.64$,
оснований отвергать $H_0$ нет
\item $X_1, \ldots, X_n \sim \cN(\mu_X, \sigma^2_X)$, $Y_1, \ldots, Y_n \sim \cN(\mu_Y, \sigma^2_Y)$ — независимые выборки
\item $\left[17.51 - 1.66 \cdot 59.4 \sqrt{\frac{1}{40}+ \frac{1}{60}}; 17.51 + 1.66 \cdot 59.4 \sqrt{\frac{1}{40}+ \frac{1}{60}} \right]$

$[-2.61; 37.64]$

$\hat{\sigma}_0^2 = \frac{\hat{\sigma}^2_X(n_X-1) + \hat{\sigma}^2_Y(n_Y-1)}{n_X+n_Y-2} \approx 59.4$
\item Оснований считать новую методику более эффективной нет, так как $0$ входит в доверительный интервал.
\end{enumerate}
\item $\hat{p}_1 = \frac{57}{159}$, $\hat{p}_2 = \frac{48}{159}$, $\hat{p}_3 = \frac{54}{159}$

$Q_{obs} = \sum_{i=1}^n \frac{(n_i - n \cdot p_i)^2}{n \cdot p_i} = \frac{42}{53}$. $Q_{crit} = 3.84$.
Оснований отвергать нулевую гипотезу нет.
\item $\gamma_{obs} = \sum_{i=1}^s \sum_{j=1}^m \frac{\left(n_{ij} - \frac{n_{i\cdot}n_{\cdot j}}{n}\right)^2}{\frac{n_{i\cdot}n_{\cdot j}}{n}} \approx 1.15$,
$\gamma_{crit} = 3.84$, оснований отвергать $H_0$ нет.
\item
\begin{enumerate}
\item \begin{align*}
L(x_1, \ldots, x_n, \theta) &= \prod_{i=1}^n \frac{1}{\sqrt{2\pi\theta}}e^{-\frac{1}{2}\cdot\frac{x_i^2}{\theta}} = \frac{1}{(\sqrt{2\pi\theta})^n} e^{-\frac{1}{2\theta} \sum_{i=1}^n x_i^2} \\
l(x_1, \ldots, x_n, \theta) &= -\frac{n}{2} \ln (2\pi) - \frac{n}{2} \ln \theta -\frac{1}{2\theta} \sum_{i=1}^n x_i^2 \to \max_{\theta} \\
\frac{\partial l}{\partial \theta} &= - \frac{n}{2 \theta} + \frac{1}{2\theta^2} \sum_{i=1}^n x_i^2 \\
\hat{\theta}_{ML} &= \frac{\sum_{i=1}^n x_i^2}{n}
\end{align*}
\item $\E(\hat{\theta}_{ML}) = \E\left(\frac{\sum_{i=1}^n x_i^2}{n}\right) = \frac{1}{n}\cdot n \E(x_1^2) = \theta$,
оценка несмещённая.

$\Var(\hat{\theta}_{ML}) = \Var\left(\frac{\sum_{i=1}^n x_i^2}{n}\right) = \frac{1}{n^2}\cdot n \Var(x_1^2) = \frac{3\theta^2 - \theta^2}{n}\to_{n\to\infty} 0$,
оценка состоятельная.

$\frac{\partial^2 l}{\partial \theta^2} = \frac{n}{2 \theta^2} - \frac{1}{\theta^3}\sum_{i=1}^n x_i^2$

$-\E\left(\frac{\partial^2 l}{\partial \theta^2}\right) = -\frac{n}{2 \theta^2} + \frac{1}{\theta^3} \cdot n \theta = \frac{n}{2 \theta^2}$

$\Var(\hat{\theta}_{ML}) = \frac{2\theta^2}{n} = \frac{1}{\frac{n}{2 \theta^2}} = I(\theta)$,
оценка эффективная.
\end{enumerate}
\item
\begin{enumerate}
\item O1Р: выбрали $H_a$, но верна $H_0$, то есть $X \sim [0, 100]$, но при этом $X \geq c$.

О2Р: выбрали $H_0$, но верна $H_a$, то есть $X \sim [50, 150]$, но при этом $X > c$.
\item $\alpha = \begin{cases}
1, & c < 0 \\
\frac{100-c}{100}, & 0 \leq c \leq 100 \\
0, & c > 100
\end{cases}$

$\beta = \begin{cases}
0, & c < 50 \\
\frac{c-50}{100}, & 50 \leq c \leq 150 \\
1, & c > 150
\end{cases}$
\end{enumerate}
\item Считаем через сумму рангов, $T = 2 + 4 + 6 + 8 + 10 = 30$.

$\E(T) = \frac{1}{2}(n_x(n_x+n_y+1)) = 37.5$.

$\Var(T) = \frac{1}{12}(n_x n_y(n_x + n_y)) = \frac{75}{6}$

$\gamma_{obs} = \frac{30-37.5}{\sqrt{\frac{75}{6}}} \approx -2.12$,
$\gamma_{crit, 0.05} = -1.65$, основная гипотеза отвергается.

Можно было бы считать через $U$-статистику, у неё было бы другое ожидание, $n_x n_y/2$,
но после стандартизации снова вышло бы $-2.12$.

\end{enumerate}


\subsection[2005-2006]{\hyperref[sec:kr_03_2005_2006]{2005-2006}}
\label{sec:sol_kr_03_2005_2006}

Нет, Нет, Да, Да, Да, Нет, Нет, Нет, Да, Нет, Нет, Да, Нет, Да, Нет, Да



\begin{enumerate}

\item
\begin{enumerate}
\item
\item $\E(\hat{\theta})=1\cdot \P(X<3)+0\cdot \P(X \ge 3)=\theta$, да является
\item $\E\left(\left(\hat{\theta }-\theta \right)^{2} \right)=\E\left(\hat{\theta}^{2}-2\theta\hat{\theta}+\theta^{2}\right) \stackrel{\hat{\theta}^{2}=\hat{\theta}}{=} \theta-2\theta^{2}+\theta^{2}=\theta-\theta^{2}$
\end{enumerate}
\item
\begin{enumerate}
\item  $L=(k+1)^{n}(x_{1}\cdot x_{2} \cdot \ldots \cdot x_{n})^{k}$

$l=\ln{L}=n\ln(k+1)+k(\sum \ln{x_{i}})$

$\frac{\partial l}{\partial k}=\frac{n}{k+1}+\sum \ln{x_{i}}$

$\frac{n}{\hat{k}+1}+\sum \ln{x_{i}}=0$

$\hat{k}=-\left(1+\frac{n}{\sum \ln{x_{i}}} \right)$
\item  $\E(X_{i})=\int t\cdot p(t)dt=\int_{0}^{1} (k+1)t^{k+1}=\frac{k+1}{k+2}$

$\frac{\hat{k}+1}{\hat{k}+2}=\bar{X}$

$\hat{k}=\frac{2\bar{X}-1}{1-\bar{X}}$
\end{enumerate}
\item $C=\sum \frac{(X_{i,j}-n \hat{p}_{i,j})^{2}}{n\hat{p}_{i,j}}\sim \chi_{(r-1)(c-1)}^{2}$

$C\sim \chi_{1}^{2}$

$C=35$

Если $\alpha=0.1$, то $C_{crit}=2.706$.

Вывод: $H_{0}$ (гипотеза о независимости признаков) отвергается.
\item
\begin{enumerate}
\item Число наблюдений велико, используем нормальное распределение.

$\P\left(-1,65<\frac{\bar{X}-\bar{Y}-\triangle}{\sqrt{\frac{\hat{\sigma}_{x}^{2}}{40}+\frac{\hat{\sigma}_{y}^{2}}{50}}}<1,65\right)=0.9$

$\triangle \in 4 \pm 1.65\sqrt{\frac{49}{40}+\frac{64}{50}}$

$\triangle \in [1.4;6.6]$
\item Используем результат предыдущего пункта: $H_{0}$ отвергается, так как число 0 не входит в доверительный интервал.
\item $Z=2.505$ и $P_{value}=0.0114$
\end{enumerate}
\item
\begin{enumerate}
\item $\chi_{9}^{2}=\frac{9\hat{\sigma}^{2}}{\sigma^{2}} \in [4.17; 14.69]$

$\sigma^{2} \in [8822.3; 31080]$

$\sigma \in [93.9; 176.3] $
\item  $\P(|\hat{\sigma}^{2}-\sigma^{2}|<0.4\sigma^{2})=\P(0.6<\frac{\hat{\sigma}^{2}}{\sigma^{2}}<1.4)=\P(11.4<\chi_{19}^{2}<26.6)\approx 0.8$
\end{enumerate}
\item
\begin{enumerate}
\item $W_{1}=2+4+6+8=20$ или $W_{2}=1+3+5+7+9+10=35$

$U_{1}=10$ или $U_{2}=14$

$Z_{1}=-0.43=-Z_{2}$

Вывод: $H_{0}$ (гипотеза об отсутствии сдвига между законами распределения) не отвергается
\item Нет, так как наблюдения не являются парными.
\end{enumerate}
\item  $\P(-2.13<t_{4}<2.13)=0.9$

$\mu \in 1560 \pm 2.13\cdot \sqrt{\frac{670^{2}}{5}}$

$\mu \in [921.8;2198.2]$
\item
\begin{enumerate}
\item
\item  $\P(\text{1 type error})=\P(X>c|X\sim U[0;100])= \left\{
\begin{array}{ll}
  1, & c<0 \\
  1-\frac{c}{100}, & c \in [0;100] \\
  0, & c>100 \\
\end{array}
\right.$

$\P(\text{2 type error})=\P(X<c|X\sim U[50;150])= \left\{
\begin{array}{ll}
  0, & c<50 \\
  \frac{c-50}{100}, & c \in [50;150] \\
  1, & c>150 \\
\end{array}
\right.$

Построение оставлено читателю в качестве самостоятельного
упражнения :)
\end{enumerate}
\item $\P(\sqrt{X^{2}+Y^{2}}>2.45)=\P(X^{2}+Y^{2}>2.45^{2})=\P(\chi_{2}^{2}>6)=0.05$
\item
\begin{enumerate}
\item  $A$ = конспект забыт в 8-ой аудитории

$B$ = конспект был забыт в другом месте (не в аудиториях)

$C$ = конспект не был найден в первых 7-и

$\P(A|C)=\frac{\P(A)}{\P(C)}=\frac{0.3\cdot 0.1}{0.3\cdot 0.3+0.7}=\frac{3}{79}$
\item $\P(B|C)=\frac{\P(B)}{\P(C)}=\frac{0.7}{0.79}=\frac{70}{79}$
\end{enumerate}

\item[11-А.] О чем молчал учебник биологии 9 класса\ldots

Если:
\begin{enumerate}
  \item  ген имеет всего две аллели;
  \item  в популяции бесконечное число организмов;
  \item  одна аллель потомка выбирается наугад из аллелей матери, другая
— из аллелей отца;
\end{enumerate}
то распределение генотипов стабилизируется уже в первом поколении (!!!).

То есть $AA_{1}=AA_{2}=\ldots$ и $Aa_{1}=Aa_{2}=\ldots$.

Вероятность получить 'A' от родителя для рождающихся в поколении 1
равна: $p_{1}=0.3\cdot 1+0.6\cdot 0.5 + 0,1\cdot 0=0.6$.

В общем виде: $p_{1}=AA_{0}+0.5\cdot Aa_{0}$

$AA_{1}=p_{1}^{2}=0.36$, $Aa_{1}=2p_{1}(1-p_{1})=0.48$.

$p_{2}=AA_{1}+0.5\cdot Aa_{1}=p_{1}^{2}+p_{1}(1-p_{1})=p_{1}$
\item[11-Б.]
\begin{enumerate}
\item Безразлично.
Если я решил попробовать угадать $n$ букв, то выигрыш вырастает, а
вероятность падает в 2 раза по сравнению c попыткой угадать $(n-1)$-у букву.
\item В силу предыдущего пункта: $\E(X)=\frac{1}{2}\cdot 50=25$.
\end{enumerate}
\end{enumerate}
