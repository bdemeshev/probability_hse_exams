% !TEX root = ../probability_hse_exams.tex
\newpage
\thispagestyle{empty}
\section{Контрольная работа 2. ИП}

\subsection[2023-2024]{\hyperref[sec:sol_kr_02_ip_2023_2024]{2023-2024}}
\label{sec:kr_02_ip_2023_2024}

\begin{enumerate}
	\item  (6) Величины $(X_n)_{n=1}^{\infty}$ независимы и одинаково распределены с функцией плотности
	\[
	f(x)=\begin{cases}
	\frac{2}{9} x,  \text { если }  x \in[0 ; 3], \\
	0,  \text { иначе. }
	\end{cases}
	\]
	
	 \begin{enumerate}
	\item   (3) Найдите $\plim_{n \to \infty} \frac{X_1^2+\ldots+X_n^2}{n}$.
	\item   (3) Найдите $\plim_{n \to \infty} \frac{4}{3} \frac{\sum_{i=1}^n X_i^2}{\sum_{i=1}^n X_i}$.
	\end{enumerate}
	
	\item  (12) Величины $X$ и $Y$ независимы. 
	Cлучайная величина $Y$ равномерно распределена на отрезке $[0, 1]$.
	Случайная величина $X$ имеет плотность распределения
	\[
	f_X(x)=\begin{cases}
	2 x,  \text { если }  x \in[0 ; 1], \\
	0,  \text { иначе. }  
	\end{cases}
	\]
	
	 \begin{enumerate}
	\item  (7) Найдите функцию плотности суммы $f_{X+Y}(z)$.
	\item (2) Найдите $\E(X+Y)$.
	\item  (3) Найдите интерквартильный размах величины  $X$.
	\end{enumerate}
	
	
	\item (15) Сибирский крокодил Утундрий решил заняться риск-менеджментом. 
	 В этот раз он изучает теорию эффективных портфелей ценных бумаг. 
	 Величины $R_1$ и $R_2$ — это доходности двух рисковых ценных бумаг с 
	 $\E(R_1) = \E(R_2) = 0.15$, $\Var(R_1) = 0.49$, $\Var(R_2)=1$ и $\Cov(R_1, R_2) = -0.35$.
	
	Крокодил Утундрий знает, что: 
	\begin{itemize}
	\item \textit{Доходность} портфеля вычисляется по формуле $R=w_1 R_1+w_2 R_2$, где $w_1 \geq 0$ и $w_2 \geq 0$ — это доли первой и второй ценный бумаг в портфеле, соответственно, и $w_1 + w_2 = 1$.
	\item \textit{Ожидаемая доходность} портфеля равна $\E(R)$.
	\item \textit{Квадратичный риск} портфеля определяется как $\Var(R)$.
	\end{itemize}

	Утундрий составил два портфеля $A$ и $B$. 
	Доли ценных бумаг, с которыми ценные бумаги входят в портфель $A$, заданы вектором $w^{A}=(0.7, 0.3)$, 
	а в портфель $B$ — вектором $w^{B} = (1, 0)$. 
	
	Помогите Утундрию выполнить следующие задания:
	 \begin{enumerate}
	\item (1) Найдите ожидаемые доходности портфелей $A$ и $B$.
	\item (4) Какой из портфелей $A$ или $B$ предпочтительнее, если Утундрий является рискофобом, то есть не приемлет риск?
	\item (3) Найдите ковариацию доходностей портфелей $A$ и $B$.
	\item (7) Составьте портфель $C$, который имеет наименьший квадратичной риск среди всех портфелей с ожидаемой доходностью $0.15$.
	\end{enumerate}
	
	
	\item (14) Величины $X$ и $Y$, описывающие расходы семейной пары, 
	равномерно распределены в треугольнике с вершинами $(0,0)$, $(0,1)$ и $(1,0)$.
	 \begin{enumerate}
	\item (2) Найдите $\P(X^2 + Y^2 > 0.25)$.
	\item (5) Найдите частные фукнции плотности и математические ожидания величин $X$ и $Y$.
	\item (2) Проверьте независимость величин $X$ и $Y$.
	\item (5) Вычислите ковариацию величин $X$ и $Y$.
	\end{enumerate}
	
	\item (8) Рассмотрим матрицу
	 \[
	A = \begin{pmatrix}
	 4 & b \\
	 a & 9
	\end{pmatrix}.
	\]
	 \begin{enumerate}
	\item (3) Каким условиям должны удовлетворять константы $a$ и $b$, 
	чтобы матрица $A$ была ковариационной матрицей некоторого случайного вектора?
	\item  (5) Известно, что $A$ — ковариационная матрица вектора $(X, Y)$ и $ab =36$. 
	Каким соотношением связаны компоненты случайного вектора?
	\end{enumerate}
	
	\item (5) Юрий Долгорукий проводит уличный опрос и спрашивает москвичей: «Знаете ли Вы год основания Москвы?» 
	 С помощью неравенства Чебышёва определите, сколько нужно опросить человек, 
	 чтобы с вероятностью не менее 0.9 доля знающих ответ среди опрошенных отличалась бы от истинной вероятности не более, чем на 0.05?
	
	\item Бонусная задача  (4). 
	
	Второй начальный момент неотрицательной случайной величины $\xi$ равен 3. 
	 
	 Оцените сверху вероятность $\P(\xi\geq 3)$.
	
	
	\item Бонусная задача (6). 
	Величина $\xi$ имеет плотность распределения
	\[
	f(x)=\frac{\exp(-x)}{(1+\exp(-x))^2}.
	\]
	
	Для величины $\xi$ вычислите математическое ожидание, медиану, моду и начальный момент порядка 2023.
	
\end{enumerate}


\subsection[2022-2023]{\hyperref[sec:sol_kr_02_ip_2022_2023]{2022-2023}}
\label{sec:kr_02_ip_2022_2023}

Контрольная 2 состояла из двух частей, минимум и максимум.
Максимум писали очно, 80 минут с чит-листом А4. 

\begin{enumerate}
	\item Машенькина оценка за контрольную $M$ распределена равномерно на отрезке $[0,1]$. Вовочка списывает у Маши, но с ошибками, поэтому его оценка $V$ за контрольную распределена равномерно на отрезке $[0, M]$.
	Найдите
		\begin{enumerate}
	\item (3) совместную плотность оценок $(M,V)$; 
	\item (2) вероятность того, что Вовочка получит оценку больше 0.5;
	\item (3) вероятность того, что Вовочка получит оценку больше 0.5, если известно, что Машина оценка больше 0.5;
	\item (3) безусловную плотность распределения оценки Вовочки $f_V (v)$;
	\item (2) условную плотность распределения оценки Вовочки $f_{V|M} (v|m)$;
	\item (3) условное математическое ожидание $\E(V|M)$;
	\item (2) математическое ожидание условного математического ожидания $\E(\E(V|M))$;
	\item (7) ковариационную матрицу вектора $\left(\begin{array}{c}
		V \\
		M
	\end{array}\right)$.
\end{enumerate}

\item Дана матрица 
\[\left(\begin{array}{cc}
	4 & -1 \\
	-1 & 4
\end{array}\right)
\]
	\begin{enumerate}
		\item (5) Поясните, может ли эта матрица быть ковариационной для некоторого случайного вектора.
		\item (5) Предложите такое линейное преобразование $A$, чтобы вектор $\eta=A\xi$ имел 
		некоррелированные компоненты с единичной дисперсией.
	\end{enumerate}
	
    \item Количество дневных сообщений, получаемых Сереной ван дер Вудсен в Телеграм, 
    — пуассоновская случайная величина $\xi$ с математическим ожиданием 100, а сообщений, 
    получаемых ею же в Инсте (запрещенная на территории РФ организация Мета), 
    — независимая пуассоновская случайна величина $\eta$ с математическим ожиданием 10. 
    
    Найдите 
	\begin{enumerate}
	\item (8) условное распределение числа сообщений в Инсте (запрещенная на территории РФ организация Мета), если известно, что всего Серена получит 90 сообщений за день;
	\item (2) условное математическое ожидание числа сообщений в Инсте (запрещенная на территории РФ организация Мета), если известно, что всего Серена получит 90 сообщений за день;
	\item (5) условную корреляцию $\xi$ и $\eta$, если известно что всего Серена получит 90 сообщений за день.
	\end{enumerate}
	
	
	\item Случайная величина $\xi$ имеет плотность распределения 
\[
    f_{\xi}(x)=\frac{1}{2} \frac{1}{\sqrt{2 \pi}} e^{-\frac{(x-1)^2}{2}}+\frac{1}{2} \frac{1}{\sqrt{2 \pi}} e^{-\frac{(x+1)^2}{2}}
\]
	\begin{enumerate}
		\item (3) Найдите  $\E(\xi)$.
		\item (3) Найдите $\Var(\xi)$.
		\item (2) Найдите  начальный момент 2023-го порядка $\E(\xi^{2023})$.
		\item (7) Покажите, что случайную величину $\xi$ можно представить в виде суммы независимых случайных величин: дискретной и непрерывной, и найдите распределения этих величин.
	\end{enumerate}
	Подсказка: проще начать с последнего пункта.

\end{enumerate}
