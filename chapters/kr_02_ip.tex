% !TEX root = ../probability_hse_exams.tex
\newpage
\thispagestyle{empty}
\section{Контрольная работа 2. ИП}


\subsection[2022-2023]{\hyperref[sec:sol_kr_02_ip_2022_2023]{2022-2023}}
\label{sec:kr_02_ip_2022_2023}

Контрольная 2 состояла из двух частей, минимум и максимум.
Максимум писали очно, 80 минут с чит-листом А4. 

\begin{enumerate}
	\item Машенькина оценка за контрольную $M$ распределена равномерно на отрезке $[0,1]$. Вовочка списывает у Маши, но с ошибками, поэтому его оценка $V$ за контрольную распределена равномерно на отрезке $[0, M]$.
	Найдите
		\begin{enumerate}
	\item (3) совместную плотность оценок $(M,V)$; 
	\item (2) вероятность того, что Вовочка получит оценку больше 0.5;
	\item (3) вероятность того, что Вовочка получит оценку больше 0.5, если известно, что Машина оценка больше 0.5;
	\item (3) безусловную плотность распределения оценки Вовочки $f_V (v)$;
	\item (2) условную плотность распределения оценки Вовочки $f_{V|M} (v|m)$;
	\item (3) условное математическое ожидание $\E(V|M)$;
	\item (2) математическое ожидание условного математического ожидания $\E(\E(V|M))$;
	\item (7) ковариационную матрицу вектора $\left(\begin{array}{c}
		V \\
		M
	\end{array}\right)$.
\end{enumerate}

\item Дана матрица 
\[\left(\begin{array}{cc}
	4 & -1 \\
	-1 & 4
\end{array}\right)
\]
	\begin{enumerate}
		\item (5) Поясните, может ли эта матрица быть ковариационной для некоторого случайного вектора.
		\item (5) Предложите такое линейное преобразование $A$, чтобы вектор $\eta=A\xi$ имел 
		некоррелированные компоненты с единичной дисперсией.
	\end{enumerate}
	
    \item Количество дневных сообщений, получаемых Сереной ван дер Вудсен в Телеграм, 
    — пуассоновская случайная величина $\xi$ с математическим ожиданием 100, а сообщений, 
    получаемых ею же в Инсте (запрещенная на территории РФ организация Мета), 
    — независимая пуассоновская случайна величина $\eta$ с математическим ожиданием 10. 
    
    Найдите 
	\begin{enumerate}
	\item (8) условное распределение числа сообщений в Инсте (запрещенная на территории РФ организация Мета), если известно, что всего Серена получит 90 сообщений за день;
	\item (2) условное математическое ожидание числа сообщений в Инсте (запрещенная на территории РФ организация Мета), если известно, что всего Серена получит 90 сообщений за день;
	\item (5) условную корреляцию $\xi$ и $\eta$, если известно что всего Серена получит 90 сообщений за день.
	\end{enumerate}
	
	
	\item Случайная величина $\xi$ имеет плотность распределения 
\[
    f_{\xi}(x)=\frac{1}{2} \frac{1}{\sqrt{2 \pi}} e^{-\frac{(x-1)^2}{2}}+\frac{1}{2} \frac{1}{\sqrt{2 \pi}} e^{-\frac{(x+1)^2}{2}}
\]
	\begin{enumerate}
		\item (3) Найдите  $\E(\xi)$.
		\item (3) Найдите $\Var(\xi)$.
		\item (2) Найдите  начальный момент 2023-го порядка $\E(\xi^{2023})$.
		\item (7) Покажите, что случайную величину $\xi$ можно представить в виде суммы независимых случайных величин: дискретной и непрерывной, и найдите распределения этих величин.
	\end{enumerate}
	Подсказка: проще начать с последнего пункта.

\end{enumerate}
