\newpage
\thispagestyle{empty}
\section{Контрольная работа 4}



\subsection[2017-2018]{\hyperref[sec:sol_kr_04_2017_2018]{2017-2018}}
\label{sec:kr_04_2017_2018}



\subsubsection*{Минимум}

\begin{enumerate}


	\item Пусть $X=(X_{1}, \ldots,X_{n})$ — случайная выборка из нормального распределения с неизвестным математическим ожиданием $\mu$ и неизвестной дисперсией $\sigma^2=9$. Объем выборки $n=20$. Для тестирования основной гипотезы $H_{0}:\mu=0$ против альтернативной гипотезы $H_{1}:\mu=5$ вы используете критерий: если $\overline{X}\leq2$, то вы не отвергаете гипотезу $H_{0}$, в противном случае вы отвергаете гипотезу $H_{0}$ в пользу гипотезы $H_{1}$. Найдите
	\begin{enumerate}
	\item Вероятность ошибки 1-го рода
	\item Вероятносто ошибки 2-го рода
	\end{enumerate}


		\item Пусть $X=(X_{1}, \ldots,X_{n})$ — случайная выборка из нормального распределения с неизвестными параметрами $\mu$ и $\sigma^2$. Используя реализацию случайной выборки: $x_{1}=2$, $x_{3}=8$, $x_{3}=5$, постройте 95\%-ый доверительный интервал для неизвестного параметра $\sigma^2$.


	\item Вася очень любит тестировать статистические гипотезы. В этот раз Вася собирается проверить утверждение о том, что его друг Пётр звонит Васе исключительно в то время, когда Вася ест. Для этого Вася трудится целый год и проводит серию из 365 испытаний. Результаты приведены в табилце ниже.

	\begin{center}
		\begin{tabular}{c|cc}
			\toprule
			& Пётр не звонит & Пётр звонит\\
			\midrule
			Вася ест & $100$ & $50$\\
			Вася не ест  & $125$ & $90$\\
			\bottomrule
		\end{tabular}
	\end{center}

	На уровне значимости 10\% протестируйте гипотезу о том, что Пётр звонит Васе независимо от момента приема пищи.


\item Вася Сидоров утверждает, что ходит в кино в четыре раза чаще, чем в спортзал, а в спортзал в четыре раза чаще, чем в театр. За последние полгода он 105 раз был в театре, 63 раза - в спортазе и 42 раза в кино. На уровне значимости 10\% проверьте утверждение Васи.

\end{enumerate}


	\begin{center}
	Квантили $\chi^2$ распределения c 1, 2 и 3 степенями свободы\\
		\begin{tabular}{c|cccccc}
			\toprule
			& 0.025 & 0.05 & 0.1 & 0.9 & 0.95 & 0.975\\
			\midrule
							1 & 0.001 & 0.004 & 0.016 & 2.706 & 3.841 & 5.024 \\
							2 & 0.051 & 0.103 & 0.211 & 4.605 & 5.991 & 7.378 \\
							3 & 0.216 & 0.352 & 0.584 & 6.251 & 7.815 & 9.348 \\
			\bottomrule
		\end{tabular}
	\end{center}

\newpage

\subsubsection*{Задачи}

При решении задач пять–семь используйте данные обследования Росстата за первый квартал 2018 года:

	\begin{center}
		\begin{tabular}{c|ccc}
			\toprule
			& Число наблюдений & Среднее (тыс. руб.) & Выборочное отклонение (тыс. руб.) \\
			\midrule
							Врачи & 40 & 136 & 55  \\
							Преподаватели & 60 & 139 & 60 \\
			\bottomrule
		\end{tabular}
	\end{center}

Распределение заработной платы работников любой отрасли хорошо описывается нормальным законом.

\begin{enumerate}[resume]

	%Задача 1
	\item На уровне значимости 5\% проверьте гипотезу о том, что средняя зарплата врача составляет 100 т.р., против альтернативы, что она больше 100 т.р.  Вычислите минимальный уровень значимости, при котором основная гипотеза отвергается (Р–значение).

	%Задача 2
	\item На уровне значимости 10\% проверьте гипотезу о том, что разброс в зарплатах врачей и преподавателей одинаков, против двухсторонней альтернативы.

	%Задача 3
	\item На уровне значимости 10\% проверьте гипотезу о том, что средняя зарплата врачей и преподавателей совпадают, против альтернативы, что у преподавателей зарплата выше:

	\begin{enumerate}
	\item Считая объемы выборок достаточно большими
			\item Считая дисперсии одинаковыми
	 \end{enumerate}

			%Задача 4
			\item Время в часах безотказной работы  микронаушника, величина $X$, подчиняется экспоненциальному (показательному) закону распределения с неизвестным параметром $\lambda$:
\[
f(x;\lambda)
			=\begin{cases}
			\lambda e^{-\lambda x},\text{ при } x\geq0\\
			0, \text{ при }  x<0\\
			\end{cases}
\]
По выборке из 100 независимых наблюдений $\bar x=0.52$. С помощью асимптотических свойств оценок максимального правдоподобия постройте приближенный 95\%-ый доверительный интервал:

\begin{enumerate}
	\item Для параметра $\lambda$
			\item Для вероятности того, что наушник проработает без сбоев весь тест — 45 минут
	 \end{enumerate}

	 %Задача 5
	 \item Приглашенный на Петербургский международный экономический форум Германом Грефом индийский мистик Садхгуру подарил Грефу древнюю шестигранную кость для принятия решений в сложных макроэкономических ситуациях. Служба безопасности Сбербанка провела серию из 100 испытаний и составила таблицу:

			 \begin{center}
		\begin{tabular}{c|cccccc}
			\toprule
			Грань & 1 & 2 & 3 & 4 & 5 & 6\\
			\midrule
							Число выпадений & 10 & 10 & 15 & 15 & 25 & 25 \\
			\bottomrule
		\end{tabular}
	\end{center}

С помощью теста отношения правдоподобия на уровне значимости 5\% проверьте гипотезу о том, что все грани равновероятны.

\[
\ln(1/6)=-1.79, \; \ln(0.15)=-1.90, \; \ln(0.25)=-1.39, \; \ln(0.1)=-2.30
\]

\end{enumerate}
