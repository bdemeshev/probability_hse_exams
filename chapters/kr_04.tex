\newpage
\thispagestyle{empty}
\section{Контрольная работа 4}



\subsection[2017-2018]{\hyperref[sec:sol_kr_04_2017_2018]{2017-2018}}
\label{sec:kr_04_2017_2018}


\subsubsection*{Минимум}

\begin{enumerate}


	\item Пусть $X=(X_{1}, \ldots,X_{n})$ — случайная выборка из нормального распределения с неизвестным математическим ожиданием $\mu$ и неизвестной дисперсией $\sigma^2=9$. Объем выборки $n=20$. Для тестирования основной гипотезы $H_{0}:\mu=0$ против альтернативной гипотезы $H_{1}:\mu=5$ вы используете критерий: если $\overline{X}\leq2$, то вы не отвергаете гипотезу $H_{0}$, в противном случае вы отвергаете гипотезу $H_{0}$ в пользу гипотезы $H_{1}$. Найдите
	\begin{enumerate}
	\item Вероятность ошибки 1-го рода
	\item Вероятносто ошибки 2-го рода
	\end{enumerate}


		\item Пусть $X=(X_{1}, \ldots,X_{n})$ — случайная выборка из нормального распределения с неизвестными параметрами $\mu$ и $\sigma^2$. Используя реализацию случайной выборки: $x_{1}=2$, $x_{3}=8$, $x_{3}=5$, постройте 95\%-ый доверительный интервал для неизвестного параметра $\sigma^2$.


	\item Вася очень любит тестировать статистические гипотезы. В этот раз Вася собирается проверить утверждение о том, что его друг Пётр звонит Васе исключительно в то время, когда Вася ест. Для этого Вася трудится целый год и проводит серию из 365 испытаний. Результаты приведены в табилце ниже.

	\begin{center}
		\begin{tabular}{c|cc}
			\toprule
			& Пётр не звонит & Пётр звонит\\
			\midrule
			Вася ест & $100$ & $50$\\
			Вася не ест  & $125$ & $90$\\
			\bottomrule
		\end{tabular}
	\end{center}

	На уровне значимости 10\% протестируйте гипотезу о том, что Пётр звонит Васе независимо от момента приема пищи.


\item Вася Сидоров утверждает, что ходит в кино в четыре раза чаще, чем в спортзал, а в спортзал в четыре раза чаще, чем в театр.
За последние полгода он 105 раз был в театре, 63 раза — в спортазе и 42 раза в кино.
На уровне значимости 10\% проверьте утверждение Васи.

\end{enumerate}


	\begin{center}
	Квантили $\chi^2$ распределения c 1, 2 и 3 степенями свободы\\
		\begin{tabular}{c|cccccc}
			\toprule
			& 0.025 & 0.05 & 0.1 & 0.9 & 0.95 & 0.975\\
			\midrule
							1 & 0.001 & 0.004 & 0.016 & 2.706 & 3.841 & 5.024 \\
							2 & 0.051 & 0.103 & 0.211 & 4.605 & 5.991 & 7.378 \\
							3 & 0.216 & 0.352 & 0.584 & 6.251 & 7.815 & 9.348 \\
			\bottomrule
		\end{tabular}
	\end{center}

\newpage

\subsubsection*{Задачи}

При решении задач пять–семь используйте данные обследования Росстата за первый квартал 2018 года:

	\begin{center}
		\begin{tabular}{c|ccc}
			\toprule
			& Число наблюдений & Среднее (тыс. руб.) & Выборочное отклонение (тыс. руб.) \\
			\midrule
							Врачи & 40 & 136 & 55  \\
							Преподаватели & 60 & 139 & 60 \\
			\bottomrule
		\end{tabular}
	\end{center}

Распределение заработной платы работников любой отрасли хорошо описывается нормальным законом.

\begin{enumerate}[resume]

	%Задача 1
	\item На уровне значимости 5\% проверьте гипотезу о том, что средняя зарплата врача составляет 100 т.р., против альтернативы, что она больше 100 т.р.  Вычислите минимальный уровень значимости, при котором основная гипотеза отвергается (Р–значение).

	%Задача 2
	\item На уровне значимости 10\% проверьте гипотезу о том, что разброс в зарплатах врачей и преподавателей одинаков, против двухсторонней альтернативы.

	%Задача 3
	\item На уровне значимости 10\% проверьте гипотезу о том, что средняя зарплата врачей и преподавателей совпадают, против альтернативы, что у преподавателей зарплата выше:

	\begin{enumerate}
	\item Считая объемы выборок достаточно большими
			\item Считая дисперсии одинаковыми
	 \end{enumerate}

			%Задача 4
			\item Время в часах безотказной работы  микронаушника, величина $X$, подчиняется экспоненциальному (показательному) закону распределения с неизвестным параметром $\lambda$:
\[
f(x;\lambda)
			=\begin{cases}
			\lambda e^{-\lambda x},\text{ при } x\geq0\\
			0, \text{ при }  x<0\\
			\end{cases}
\]
По выборке из 100 независимых наблюдений $\bar x=0.52$. С помощью асимптотических свойств оценок максимального правдоподобия постройте приближенный 95\%-ый доверительный интервал:

\begin{enumerate}
	\item Для параметра $\lambda$
			\item Для вероятности того, что наушник проработает без сбоев весь тест — 45 минут
	 \end{enumerate}

	 %Задача 5
	 \item Приглашенный на Петербургский международный экономический форум Германом Грефом индийский мистик Садхгуру подарил Грефу древнюю шестигранную кость для принятия решений в сложных макроэкономических ситуациях. Служба безопасности Сбербанка провела серию из 100 испытаний и составила таблицу:

			 \begin{center}
		\begin{tabular}{c|cccccc}
			\toprule
			Грань & 1 & 2 & 3 & 4 & 5 & 6\\
			\midrule
							Число выпадений & 10 & 10 & 15 & 15 & 25 & 25 \\
			\bottomrule
		\end{tabular}
	\end{center}

С помощью теста отношения правдоподобия на уровне значимости 5\% проверьте гипотезу о том, что все грани равновероятны.

\[
\ln(1/6)=-1.79, \; \ln(0.15)=-1.90, \; \ln(0.25)=-1.39, \; \ln(0.1)=-2.30
\]

\end{enumerate}


\newpage
\subsection[2016-2017]{\hyperref[sec:sol_kr_04_2016_2017]{2016-2017}}
\label{sec:kr_04_2016_2017}


\subsubsection*{I. Теоретический минимум}


В пунктах 1, 3, 11 и 12 предполагается, что $X = (X_1, \, \ldots, \, X_n)$ и
$Y = (Y_1, \, \ldots, \, Y_m)$ — две независимые случайные выборки из нормальных
распределений $N(\mu_X, \sigma_X^2)$ и $N(\mu_Y, \sigma_Y^2)$ соответственно.

\begin{enumerate}
  \item Приведите формулу статистики, при помощи которой можно проверить гипотезу
	$H_0 \colon \sigma_X^2 = \sigma_Y^2$. Укажите распределение этой статистики при
	верной гипотезе $H_0$.
  \item Приведите формулу информации Фишера о параметре $\theta$, содержащейся в
	одном наблюдении случайной выборки.
  \item Приведите формулу статистики, при помощи которой можно проверить гипотезу
	$H_0 \colon \mu_X - \mu_Y = \Delta_0$ при условии, что дисперсии $\sigma_X^2$
	и $\sigma_Y^2$ неизвестны, но равны между собой. Укажите распределение этой
	статистики при верной гипотезе $H_0$.
  \item Дайте определение критической области.
  \item Приведите формулу плотности нормального распределения $\cN(\mu, \sigma^2)$.
  \item Приведите формулы границ доверительного интервала с уровнем доверия
	$(1 - \alpha)$, $\alpha \in (0;\,1)$, для вероятности появления успеха в
	случайной выборке $X = (X_1, \, \ldots, \, X_n)$ из распределения Бернулли с
	параметром $p \in (0;\,1)$.
  \item Дайте определение несмещенной оценки $\hat{\theta}$ для неизвестного
	параметра $\theta \in \Theta$.
  \item Дайте определение эффективной оценки $\hat{\theta}$ для неизвестного
	параметра $\theta \in \Theta$.
  \item Приведите формулу выборочной дисперсии.
  \item Приведите формулу выборочной функции распределения.
  \item Приведите формулы границ доверительного интервала с уровнем доверия
	$(1 - \alpha)$, $\alpha \in (0;\,1)$, для $\mu_X$ при условии, что дисперсия
	$\sigma_X^2$ известна.
  \item Укажите распределение статистики $\frac{\overline{X} - \mu_X}{\sigma / \sqrt{n}}$.
\end{enumerate}


\subsubsection*{II. Задачи}

\begin{enumerate}
\item В ходе анкетирования ста сотрудников банка «Альфа» были получены ответы на
вопрос о том, сколько времени они проводят на работе ежедневно. Среднее выборочное
оказалось равным $9.5$ часам, а выборочное стандартное отклонение $0.5$ часа.
\begin{enumerate}
  \item На уровне значимости 5\,\% проверьте гипотезу о том, что сотрудники банка
	«Альфа» в среднем проводят на работе $10$ часов, против альтернативной гипотезы о
	том, что сотрудники банка «Альфа» в среднем проводят на работе менее $10$ часов.
  \item Найдите точное $P$-значение для наблюдаемой статистики из пункта (a).
  \item Сформулируйте предпосылки, которые были использованы вами для выполнения
	пункта (a).
  \item На уровне значимости 5\,\% проверьте гипотезу о $H_0 \colon \sigma^2 = 0.3$.
\end{enumerate}


\item Проверка сорока случайно выбранных лекций показала, что студент Халявин
присутствовал только на 16 из них. На уровне значимости 5\,\% проверьте гипотезу
о том, что Халявин посещает в среднем половину лекций.

\item В ходе анкетирования двадцати сотрудников банка «Альфа» были получены ответы
на вопрос о том, сколько времени они проводят на работе ежедневно. Среднее выборочное
оказалось равным $9.5$ часам, а выборочное стандартное отклонение $0.5$ часа.
Аналогичные показатели для 25 сотрудников банка «Бета» составили $9.8$ и $0.6$
часа соответственно.
\begin{enumerate}
  \item На уровне значимости 5\,\% проверьте гипотезу о равенстве математических
	ожиданий времени, проводимого на работе сотрудниками банков «Альфа» и «Бета».
  \item Сформулируйте предпосылки, которые были использованы вами для выполнения
	пункта (a).
  \item На уровне значимости 5\,\% проверьте гипотезу о равенстве дисперсий времени,
	проводимого на работе сотрудниками банков «Альфа» и «Бета».
\end{enumerate}

\item Вася решил проверить известное утверждение о том, что бутерброд падает
маслом вниз. Для этого он провел серию из 200 испытаний. Ниже приведена таблица
с результатами:
\begin{center}
\begin{tabular}{ccc}
  \toprule
  \text{Бутерброд}                &\text{Маслом вниз}    &\text{Маслом вверх}       \\ \midrule
  \text{Число наблюдений}         &$105$    &$95$       \\ \bottomrule
\end{tabular}
\end{center}
Можно ли утверждать, что бутерброд падает маслом вниз так же часто, как и маслом
вверх? При ответе на вопрос используйте уровень значимости 5\,\%.

\item Пусть $X = (X_1, \, \ldots, \, X_{100})$ — случайная выборка из нормального
распределения с математическим ожиданием $\mu$ и дисперсией $\nu$. Оба параметра
$\mu$ и $\nu$ неизвестны. Используя следующие данные $\sum_{i=1}^{100}x_i = 30$,
$\sum_{i=1}^{100}x_i^2 = 146$ и $\sum_{i=1}^{100}x_i^3 = 122$ с помощью теста
отношения правдоподобия проверьте гипотезу $H_0 \colon \nu = 1$ на уровне значимости 5\,\%.
\end{enumerate}


\newpage
\subsection[2015-2016]{\hyperref[sec:sol_kr_04_2015_2016]{2015-2016}}
\label{sec:kr_04_2015_2016}

\begin{enumerate}
\item	Сформулируйте определения несмещённости, состоятельности и эффективности оценок.
\item На курсе учится 250 человек. Предположим, что число студентов, не явившихся
на экзамен, хорошо описывается законом Пуассона.
\begin{enumerate}
\item	Методом максимального правдоподобия найдите оценку параметра распределения Пуассона.
\item	Проверьте выполнение свойств несмещенности, эффективности и состоятельности
для данной оценки.
\item	Найдите оценку максимального правдоподобия для вероятности стопроцентной
явки студентов на экзамен.
\item	Используя дельта-метод, постройте для этой вероятности асимптотический
доверительный интервал.
\end{enumerate}

\item	Фармацевтическая компания выпустила новое лекарство от бессонницы, утверждая,
что оно помогает 80\% людей, страдающих бессонницей. Чтобы проверить утверждение
компании, случайным образом выбираются 20 человек, страдающих бессонницей. Обозначим
за $Y$ количество человек из выборки, которым лекарство помогло. Основная гипотеза,
$H_0$: $p=0.8$, альтернативная гипотеза $H_a$: $p=0.6$. Критическая область: $\{Y<12\}$.
\begin{enumerate}
\item	В терминах этой задачи сформулируйте, что является ошибкой первого рода.
Найдите уровень значимости, соответствующий заданной критической области.
\item	В терминах этой задачи сформулируйте, что является ошибкой второго рода.
Найдите вероятность ошибки второго рода.
\item	Найдите такое значение $c$, что вероятность ошибки первого рода $\alpha
\approx 0.1$ при критической области вида $\{Y<c\}$. Найдите соответствующее
значение вероятности ошибки второго рода.
\item	Каким должен быть размер выборки, чтобы выборочная доля страдающих бессонницей
отличалась от истинной вероятности не более, чем на $0.01$ с вероятностью не менее,
чем $0.95$?
\end{enumerate}

\item	Вася Сидоров утверждает, что ходит в кино в два раза чаще, чем на лекции
по статистике, на лекции по статистике в два раза чаще, чем в спортзал. За
последние полгода он 10 раз был в спортзале, 1 раз — на лекциях по статистике и
39 раз в кино.

При помощи критерия хи-квадрат Пирсона на уровне значимости $0.05$ проверьте,
правдоподобно ли Васино утверждение.

\item У Евдокла есть случайная выборка из экспоненциального распределения с
неизвестным параметром $\lambda$ в 50 наблюдений, $X_1$, $X_2$, \ldots, $X_{50}$.
Оказалось, что $\bar X = 1.1$. Евдокл хочет проверить гипотезу о равенстве
$\lambda = 1$ против альтернативной гипотезы о неравенстве $\lambda \neq 1$ на
уровне значимости $0.1$.

Помогите Евдоклу и проверьте гипотезу с помощью критерия отношения правдоподобия.

Пачка логарифмов: $\ln 50 \approx 3.9$, $\ln 55 \approx 4.0$, $\ln 11 \approx 2.4$,
$\ln 60 \approx 4.1$, $\ln 12 \approx 2.5$

\item Американский демографический журнал опубликовал исследование, в котором
утверждается, что посетители крупных торговых центров за одно посещение тратят
в выходные дни больше, чем в будние. Наибольшие расходы приходятся на воскресенье
в период с 4 до 6 часов вечера. Для двух независимых выборок посетителей средние
расходы и выборочные стандартные отклонения расходов составили
\begin{center}
\begin{tabular}{lrr}
\toprule
 & Выходные & Рабочие дни \\
\midrule
Число наблюдений & 21 & 19 \\
Средние расходы (\$) & 78 & 67 \\
Выборочное стандартное отклонение (\$) & 22 & 20 \\
\bottomrule
\end{tabular}
\end{center}

\begin{enumerate}
\item Проверьте гипотезу о равенстве дисперсий расходов
\item Предполагая, что дисперсии расходов одинаковы, проверьте гипотезу об отсутствии
разницы в расходах в выходные и будние дни.
\item Сформулируйте все необходимые для проверки гипотез предыдущих пунктов предпосылки.
\end{enumerate}

\item Винни Пух знает, что пчёлы и мёд бывают правильные и неправильные.
По результатам 100~попыток добыть мёд Винни Пух составил таблицу сопряженности признаков.

\begin{center}
\begin{tabular}{lrr}
\toprule
 & Мёд правильный & Мёд неправильный \\
\midrule
Пчёлы правильные & 12	& 36 \\
Пчёлы неправильные & 32	& 20 \\
\bottomrule
\end{tabular}
\end{center}

На уровне значимости $0.05$ проверьте гипотезу о независимости характеристик пчёл и мёда.

\begin{figure}[b]
\centering
\includegraphics[width=9cm]{images/winnie_kr_4}
\end{figure}
\end{enumerate}



\newpage
\subsection[2014-2015]{\hyperref[sec:sol_kr_04_2014_2015]{2014-2015}}
\label{sec:kr_04_2014_2015}

\begin{enumerate}

\item[1.] \textbf{Задача для первого потока.}

Проверка 40 случайно выбранных лекций показала, что студент Халявин присутствовал
только на 16 из них.
\begin{enumerate}
\item Найдите 95\% доверительный интервал для вероятности увидеть Халявина на лекции.
\item На уровне значимости 5\% проверьте гипотезу о том, что Халявин посещает
в среднем половину лекций.
\item Вычислите минимальный уровень значимости, при котором основная гипотеза
отвергается (P-значение).
\end{enumerate}

\item[1.] \textbf{Задача для второго потока.}

Вес упаковки с лекарством является нормальной случайной величиной.
Взвешивание 20~упаковок показало, что выборочное среднее равно 51 г., а
несмещенная оценка дисперсии равна~4.
\begin{enumerate}
\item На уровне значимости 10\% проверьте гипотезу, что в среднем вес упаковки
составляет~55 г.
\item Контрольное взвешивание 30 упаковок такого же лекарства другого производителя
показало, что несмещенная оценка дисперсии веса равна 6. На уровне значимости 10\%
проверьте гипотезу о равенстве дисперсий веса упаковки двух производителей.
\end{enumerate}

\item[2.] \textbf{Задача для первого потока.}

В ходе анкетирования 15 сотрудников банка «Альфа» ответили на вопрос о том,
сколько времени они проводят на работе ежедневно. Среднее выборочное оказалось
равно $9.5$ часам при выборочном стандартном отклонении $0.5$ часа. Аналогичные
показатели для 12 сотрудников банка «Бета» составили $9.8$ и $0.6$ часа соответственно.

Считая распределение времени нормальным, на уровне значимости 5\% проверьте
гипотезу о том, что сотрудники банка «Альфа» в среднем проводят на работе столько
же времени, сколько и сотрудники банка «Бета».

\item[2.] \textbf{Задача для второго потока.}

Экзамен принимают два преподавателя, случайным образом выбирая студентов.
По выборке из 85 и 100 наблюдений, выборочные доли не сдавших экзамен студентов составили
соответственно $0.2$ и $0.17$.
\begin{enumerate}
\item Можно ли при уровне значимости в 1\% утверждать, что преподаватели предъявляют
к студентам одинаковый уровень требований?
\item Вычислите минимальный уровень значимости, при котором основная гипотеза
отвергается (P-значение).
\end{enumerate}

\item[3.] Методом максимального правдоподобия найдите оценку параметра $\theta$
для выборки $X_1$, \ldots, $X_n$ из распределения с функцией плотности
\[
f(x)=\begin{cases}
\frac{1}{\theta^2}xe^{-\frac{x}{\theta}}, \; x>0 \\
0, \; x\leq 0
\end{cases}
\]

\item[4.]
Пусть $X_1$, \ldots, $X_{100}$ — случайная выборка из нормального распределения
с математическим ожиданием $\mu$ и дисперсией $\nu$, где $\mu$ и $\nu$ — неизвестные
параметры. По 100 наблюдениям $\sum x_i=30$, $\sum x_i^2=146$, $\sum x_i^3=122$.

При помощи теста отношения правдоподобия протестируйте гипотезу $H_0: \nu=1$
на уровне значимости 5\%.

\item[5.] \textbf{Исследовательская задача.}

Пусть $X_1$, \ldots, $X_{n}$ — случайная выборка из нормального распределения
с математическим ожиданием $\mu$ и дисперсией $\nu$, где $\mu$ и $\nu$ — неизвестные
параметры. Рассмотрим три классических теста, отношения правдоподобия, $LR$,
множителей Лагранжа, $LM$ и Вальда, $W$, для тестирования гипотезы $H_0: \; \mu=0$.

\begin{enumerate}
\item Сравните  статистики $LR$, $LM$ и $W$ между собой. Какая — наибольшая,
какая — наименьшая?
\item Изменится ли упорядоченность статистик, если проверять гипотезу $H_0: \; \mu=\mu_0$?
\end{enumerate}

Подсказка: $\frac{x}{1+x} \leq \ln(1+x) \leq x\, \; \text{ при } x>-1$

\item[6.] \textbf{Исследовательская задача.}

Величины $X_1$, \ldots, $X_n$ независимы и одинаково распределены с функцией плотности
\[
f(x)=\begin{cases}
a^2xe^{-ax}, \; x>0 \\
0, \; x\leq 0
\end{cases}
\]

По выборке из 100 наблюдений оказалось, что $\sum x_i =300$, $\sum x_i^2=1000$,
$\sum x_i^3=3700$.

\begin{enumerate}
\item Найдите оценку неизвестного параметра $a$ методом моментов
\item Используя дельта-метод или иначе оцените дисперсию полученной оценки $a$
\item Постройте 95\%-ый доверительный интервал используя оценку метода моментов
\end{enumerate}
\end{enumerate}
