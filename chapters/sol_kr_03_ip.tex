\section{Решения контрольной номер 3. ИП}

\subsection[2017-2018]{\hyperref[sec:kr_03_ip_2017_2018]{2017-2018}}
\label{sec:sol_kr_03_ip_2017_2018}



\begin{enumerate}
\item
\begin{enumerate}
	\item[а) - в)] См. картинку :)
\begin{figure}[h!]
\centering
\begin{tikzpicture}
\coordinate (e) at (1,0);
\coordinate (X) at (4,3);
\coordinate (hatX) at (4,0);
\coordinate (perpX) at (0,3);
\draw[->] (0,0) -- (e);
\node [below] at (e) {e};
\draw[->] (0,0) -- (X);
\node [right] at (X) {X};
\draw[->] (0,0) -- (hatX);
\node [below] at (hatX) {$\hat{X}$};
\draw[->] (0,0) -- (perpX);
\node [left] at (perpX) {$\hat{X}^{\perp}$};
\draw [dashed] (perpX) -- (X);
\draw [dashed] (hatX) -- (X);
\end{tikzpicture}
\end{figure}
\item[г)] $\hat{X}=e\cdot\bar{X}$

$\lVert \hat{X} \rVert=\sqrt{n}\cdot\bar{X}$

$\hat{X}^{\perp}=X-e\cdot\bar{X}=(X_1-\bar{X}, \dots ,X_n-\bar{X})$

$\lVert\hat{X}^{\perp} \rVert =\sqrt{\sum^n_{i=1}(X_i-\bar{X})^2}$

\item[д)] $ \lVert X \rVert^2=\lVert\hat{X}^{\perp}\rVert^2+\lVert\hat{X}\rVert^2$

$\sum^n_{i=1}X^2_i=\sum^n_{i=1}(X_i-\bar{X})^2+n\bar{X}^2$

\item[е)] t-статистика для построения доверительного интервала для $\mu$ имеет вид:

\begin{align*}
t &= \frac{\bar{X}-\mu}{\sqrt{\bar{\sigma}^2/n}} = \frac{\bar{X}-\mu}{\sqrt{\sum^n_{i=1}(X_i-\bar{X})^2/(n\cdot(n-1))}}\\
& =\sqrt{n-1}\cdot\frac{\sqrt{n}\cdot(\bar{X}-\mu)}{\sqrt{\sum^n_{i=1}(X_i-\bar{X})^2}}=\sqrt{n-1}\cdot\frac{\lVert \hat{X} \rVert-\sqrt{n}\cdot\mu}{\lVert\hat{X}^{\perp} \rVert}
\end{align*}

Заметим, что $\ctg \alpha$ есть отношение прилежащего катета к противолежащему, таким образом, нужный нам угол $\alpha$ образуется между векторами $X$ и $\hat{X}$. Зметим однако, что в нашем случае
\[
t=\sqrt{n-1}\cdot \ctg \alpha=\frac{\lVert\hat{X}\rVert}{\lVert\hat{X}^{\perp}\rVert},
\]
то есть наша статистика подойдёт только для проверки гипотезе о равенстве матожидания нулю.

Замечание. $t=\sqrt{n-1}\cdot \ctg \alpha$ будет t-статистикой только в том случае, если $X_i$ будут н.о.р.с.в. с нормальным распределением, о чём в условие сказано не было.
\end{enumerate}
\item Выпишем функцию правдоподобия для выборки из трёх видов, два из которых совпадают. Первый медведепришелец будет нового вида с вероятностью 1. Вероятность, что вид второго пойманного медведепришельца совпадёт с первмым, составляет $1/n$. После этого нужно поймать медведепришельца нового вида – это произойдёт с вероятностью $(n-1)/n$, и ещё одного нового вида – вероятность этого $(n-2)/n$. Поскольку медведепришелец, вид которого встречается дважды, мог встретить на любой из трёх позиций, функцию правдоподобия необходимо домножить на $C_3^1$. Таким образом, функция правдоподобия имеет вид:
\[
L(n) = C_3^1 \cdot 1 \cdot \frac{1}{n} \cdot \frac{n-1}{n} \cdot \frac{n-2}{n}, n \geq 3.
\]
Максимизируя её, внутри области определения получаем $\hat n = 5$.

Так как количество медведей велико и все они встречаются равновероятно, то $p_{1}=p_{2}=p_{3}=1/n$. Так же из выборки известно, что число видов космомедведей не меньше трёх. Потому $\hat{n} \ge 3$.

Найдите хитрую ошибку в предложенном решении:

$L(n)=\left(\frac{1}{n}\right)^{2} \cdot \frac{1}{n}\cdot\frac{1}{n}=n^{-4}$

$\frac{\partial L(n)}{\partial n} = -4\cdot n^{-5}=0$

Данное уравнение не имеет решений при конечных $n$, но заметим, что при всех $n \ge 3$ выполняется $\frac{\partial L(n)}{\partial n} = -4\cdot n^{-5} < 0$, таким образом максимальное значение находится в граничных точках.

$\lim\limits_{n\to\infty}\frac{1}{n^{-4}}=0 < \frac{1}{3^{-4}}$

Таким образом получаем, что $\hat{n}=3$.

\item \begin{enumerate}
\item $L(p_{1},p_{2})=p_{1}^{150}\cdot p_{2}^{100}\cdot(1-p_{1}-p_{2})^{50}$

$\ell(p_{1},p_{2}) = 150\ln p_{1} +100\ln p_{2}+50\ln (1-p_{1}-p_{2})$

$\begin{cases}
\frac{\partial \ell(p_{1},p_{2})}{\partial p_{1}}= \frac{150}{p_{1}} - \frac{50}{1-p_{1}-p_{2}}=0
\\
\frac{\partial \ell(p_{1},p_{2})}{\partial p_{2}}= \frac{100}{p_{2}} - \frac{50}{1-p_{1}-p_{2}}=0
\end{cases}$

Откуда получаем:

$\begin{cases}
\hat{p}_{1}=1/2
\\
\hat{p}_{2}=1/3
\end{cases}$
\item Найдём, какие значения должны стоять в теоретической ковариационной матрице.
Заметим, что случайная величина найти кальмаромедведя ($X$) или двурога ($Y$) есть бернулевская случайная величина с параметром $p_{1+2}=p_{1}+p_{2}$ и дисперсией $p_{1+2}\cdot(1-p_{1+2})$, но тогда:

$\Var(X + Y)=\Var(X)+\Var(Y)+2\cdot\Cov(X,Y)$

$\Cov(X,Y)=\frac{1}{2}\cdot(\Var(X+Y)-\Var(X)-\Var(Y))=\frac{1}{2}\cdot((p_{1}+p_{2})\cdot(1-p_{1}-p_{2})-p_{1}\cdot(1-p_{1})-p_{2}\cdot(1-p_{2}))=-p_{1}\cdot p_{2}$

Тогда подставляя в теоретическую ковариационную матрицу оценки параметров и домнажая всё на $1/300$, так как $\hat{p}_{1}$ и $\hat{p}_{2}$ являются средними, получим:

\[
\Var(\hat{p})=\frac{1}{n}\begin{pmatrix}
\hat{p}_{1}\cdot(1-\hat{p}_{1}) & -\hat{p}_{1}\cdot\hat{p}_{2}
\\
-\hat{p}_{1}\cdot\hat{p}_{2} & \hat{p}_{1}\cdot(1-\hat{p}_{1})
\end{pmatrix}=\frac{1}{300}\begin{pmatrix}
1/4 &-1/6
\\
-1/6 & 2/9
\end{pmatrix}
\]

\item Для начала, найдём теоретическую дисперсию $\Var(X-Y)$.

\[
\Var(X - Y)=\Var(X)+\Var(Y)-2\cdot\Cov(X,Y)=p_{1}\cdot(1-p_{1})+p_{2}\cdot(1-p_{2})+2\cdot p_{1}\cdot p_{2}
\]

Тогда подставляя оценки для $p_{1}$ и $p_{2}$ и учитывая, что это оценки среднего, получим оценку:

\[
\Var(\hat{p}_{1}-\hat{p}_{2})=1/300\cdot(1/4+2/9+2\cdot1/6)=29/(36\cdot300)
\]

\item Так как выборка достаточно велика, то статистика $\hat{p}_{1}-\hat{p}_{2}$,
являясь средним, будет иметь примерно нормальное распределение, и тогда:

$\frac{\hat{p}_{1}-\hat{p}_{2}-(p_{1}-p_{2})}{\sqrt{\Var(\hat{p}_{1}-\hat{p}_{2})}}\sim \cN(0,1)$

$\hat{p}_{1}-\hat{p}_{2}-z_{1-\frac{\alpha}{2}}\cdot\sqrt{\Var(\hat{p}_{1}-\hat{p}_{2})} \le p_{1}-p_{2} \le \hat{p}_{1}-\hat{p}_{2}-z_{\frac{\alpha}{2}}\cdot\sqrt{\Var(\hat{p}_{1}-\hat{p}_{2})}$
\end{enumerate}
\item \begin{enumerate}
\item $\hat{\alpha}=\bar{X}+\sqrt{\bar{X}+6}=10+\sqrt{10+6}=14$

\item Так как $\bar{X}$ сходится по распределению к нормальному распределению и $\hat{\alpha}=g(\bar{X})$, где $g(\bar{X})$ гладкая по $\bar{X}$ функция при $\bar{X}\ge0$, а также $\bar{X}$ сходится по вероятности к матожиданию, то можно абсолютно спокойно применить дельта-метод. Тогда:

\[
(\alpha-g(\bar{X}))\sim N(0;\sigma^{2}(g'(\E(X_{1})))^{2}/n)
\]

Но так как $\hat{\alpha}$ является состоятельной оценкой, то можно заменить $g'(\E(X_{1}))$ на $g'(\bar{X})$:
\[
g'(\bar{X})=1+\frac{1}{2\cdot\sqrt{\bar{X}+6}}=1+\frac{1}{2\cdot4}=\frac{9}{8}=1.125
\]
и тогда можно построить ассимтотический доверительный интервал:

\begin{align*}
%z_{2.5\%}\le&\frac{\hat{\alpha}-\alpha}{\sqrt{\sigma^{2}\cdot(g'(\bar{X}))^{2}/n}}\le z_{97.5\%} \\
\hat{\alpha}-z_{97.5\%}\cdot\sqrt{\sigma^{2}\cdot(g'(\bar{X}))^{2}/n}\le &\alpha \le
\hat{\alpha}-z_{2.5\%}\cdot\sqrt{\sigma^{2}\cdot(g'(\bar{X}))^{2}/n} \\
16-1.96\cdot2\cdot9/(8\cdot10)\le&\alpha\le 16+1.96\cdot2\cdot9/(8\cdot10) \\
13.559\le&\alpha\le 14.441
\end{align*}
\end{enumerate}
\item \begin{enumerate}
\item Так как не известно точно, кто сколько фотографий сделал, и так как метод оценки не указан,
то воспользуемся методом моментов для построения оценки.

\begin{align*}
N&=\E(\text{«фото Андрея»})+\E(\text{«фото Беллы»}) \\
130&=100\cdot 0.5+p\cdot100 \\
\hat{p}&=0.8
\end{align*}

Так как выборка достаточно велика, то $\frac{\hat{p}-p}{\sqrt{\hat{p}\cdot(1-\hat{p})/W}}\sim \cN(0,1)$

\begin{align*}
\hat{p}-z_{97.5\%}\sqrt{\hat{p}\cdot(1-\hat{p})/W} \le &p \le \hat{p}-z_{2.5\%}\sqrt{\hat{p}\cdot(1-\hat{p})/W} \\
0.8-1.96\cdot\sqrt{0.8\cdot0.2/100}\le &p\le0.8+1.96\cdot\sqrt{0.8\cdot0.2/100} \\
0.72\le &p\le 0.88
\end{align*}

\item Так как неизвестно, кто больше снимков сделал, то рассмотрим два случая: Андрей сделал 60 фото и Белла — 70 фото, Андрей сделал 70 фото и Белла — 60 фото. В каждом случае при помощи метода максимального правдоподобия оценим вероятность $p$, после чего сравним значения функции правдоподобия с оценёнными параметрами для каждого случая.
\begin{align*}
L(p)&=C^{60}_{100}\cdot 0.5^{60}\cdot 0.5^{40}\cdot C^{70}_{100}\cdot p^{70}\cdot(1-p)^{30} \\
\ell(p)&=const+70\ln p+30\ln(1-p) \\
\frac{\partial \ell (p)}{\partial p}&= \frac{70}{p}-\frac{30}{1-p}=0 \\
\hat{p}_1 &= 0.7
\end{align*}

Аналогично для второго случая получим оценку: $\hat{p}_2=0.6$.

Для простоты, будем сравнивать логарифмическии функции правдоподобия $\ell_1(p_1)$ и $\ell_2(p_2)$ и тогда получим:

\begin{align*}
\ell_1(p_1)&=const+70\ln 0.7+30\ln 0.3\approx const-70\cdot0.357-30\cdot 1.204=const-61.11 \\
\ell_2(p_2)&=const+60\ln 0.6+40\ln 0.4\approx const-60\cdot0.511-40\cdot0.916=const-67.3
\end{align*}

Так как $-67.3<-61.11$, то более вероятно, что $\hat{p}=0.7$

Тогда анологично предыдущему пункту получим доверительный интервал:

\begin{align*}
0.7-1.96\cdot\sqrt{0.7\cdot0.3/100}\le &p \le0.7+1.96\cdot\sqrt{0.7\cdot0.3/100} \\
0.61 \le &p \le 0.79
\end{align*}
\end{enumerate}
\end{enumerate}
