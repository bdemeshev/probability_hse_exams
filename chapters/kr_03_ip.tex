\newpage
\section{Контрольная работа 3. ИП}



\subsection[2017-2018]{\hyperref[sec:sol_kr_03_ip_2017_2018]{2017-2018}}
\label{sec:kr_03_ip_2017_2018}

дата: 2018-03-24

24 марта 2018 года — Комоедица, день пробуждения медведя.


\begin{enumerate}
  \item Медведь Михайло-Потапыч уснул в берлоге и ему снится сон про $n$-мерное пространство.
    Особенно ярко ему снится вектор $X=(X_1, X_2, \ldots, X_n)$ и вектор $e=(1, 1, 1, \ldots, 1)$.
    \begin{enumerate}
      \item Изобразите векторы $X$ и $e$ в $n$-мерном пространстве;
      \item Изобразите проекцию $X$ на $\Lin \{e\}$, обозначим её $\hat X$;
      \item Изобразите проекцию $X$ на $\Linp \{e\}$, обозначим её $\hat X^{\perp}$;
      \item Выпишите явно вектора $\hat X$ и $\hat X^{\perp}$, и найдите их длины;
      \item Сформулируйте теорему Пифагора для нарисованного прямоугольного треугольника;
      \item Изобразите на рисунке такой угол $\alpha$, что обычная $t$-статистика, используемая при построении доверительного интервала для $\mu$, имела бы вид $t = \sqrt{n-1} \cdot \ctg \alpha$.
    \end{enumerate}

  \item Исследователь Михаил предполагает, что все виды медведепришельцев встречаются равновероятно.
    Отправившись на охоту в район Малой Медведицы Михаил поймал двух лиловых кальмаромедведей,
    одного двурога медведеспинного и одного медведезавра ящероголового.

      Помогите Михаилу оценитель общее количество видов медведепришельцев с помощью метода максимального правдоподобия.


    \item Помотавшись по просторам Вселенной Михаил изменил своё мнение.
      Никто кроме кальмаромедведей, двурогов и медведезавров не попадается, однако попадаются они явно с разной вероятностью.
      Из 300 отловленных пришельцев оказалось 150 кальмаромедведей, 100 двурогов и 50 медведезавров.
      Михаил считает, что медведепришельцы встречаются независимо, $p_1$ — вероятность встретить кальмаромедведя, $p_2$ — двурога.

      \begin{enumerate}
	\item Оцените вектор $p = (p_1, p_2)$ методом максимального правдоподобия;
	\item Оцените ковариационную матрицу $\Var(\hat p)$;
	\item Оцените дисперсию $\Var(\hat p_1 - \hat p_2)$;
	\item Постройте доверительный интервал для разницы долей $p_1 - p_2$.
      \end{enumerate}

  \item Винни-Пух лично измерил количество мёда (в кг) на 100 деревьях и обнаружил, что $\bar X = 10$ и $\hat\sigma^2 = 4$.
    По мнению Кролика, состоятельная оценка для параметра $\alpha$ правильности мёда имеет вид $\hat \alpha = \bar X + \sqrt{\bar X + 6}$.

    \begin{enumerate}
      \item «Халява, сэр!» Найдите точечную оценку параметра $\alpha$;
      \item Найдите 95\%-ый доверительный интервал для $\alpha$, симметричный относительно $\hat\alpha$.
    \end{enumerate}

  \item Фотографы Андрей и Белла независимо друг от друга пытаются фотографировать кадьяков.
    Андрею удаётся сфотографировать одного кадьяка в неделю с вероятностью $0.5$, а Белле — с вероятностью $p$,
    независимо друг от друга и от прошлого.
    За 100 недель они вместе сфотографировали 130 кадьяков.

    \begin{enumerate}
      \item Оцените $p$ и постройте 95\%-ый доверительный интервал для $p$;
      \item Оцените $p$ и постройте 95\%-ый доверительный интервал для $p$, если дополнительно известно, что один фотограф опередил другого на 10 фото.
    \end{enumerate}

\end{enumerate}


\textbf{Просто красивая задачка}. Эту задачу \textbf{не нужно} решать на кр :)

Медведю Мишутке никак не удаётся заснуть в берлоге, и потому он подбрасывает правильную монетку $n$ раз.
Обозначим вероятность того, что ни разу не идёт двух решек подряд буквой $q_n$.


        \begin{enumerate}[label=\asbuk*)]
	  \item Найдите $2^8q_8$ и \textbf{назовите} это число;
          \item Найдите $\lim 2q_{n+1}/q_n$ и \textbf{назовите} это число.
	\end{enumerate}
