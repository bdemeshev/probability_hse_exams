\newpage
\thispagestyle{empty}
\section{Контрольная работа 3. ИП}


\subsection[2017-2018]{\hyperref[sec:sol_kr_03_ip_2017_2018]{2017-2018}}
\label{sec:kr_03_ip_2017_2018}

дата: 2018-03-24

24 марта 2018 года — Комоедица, день пробуждения медведя.

\begin{enumerate}
\item Медведь Михайло-Потапыч уснул в берлоге и ему снится сон про $n$-мерное пространство.
Особенно ярко ему снится вектор $X=(X_1, X_2, \ldots, X_n)$ и вектор $e=(1, 1, 1, \ldots, 1)$.
\begin{enumerate}
\item Изобразите векторы $X$ и $e$ в $n$-мерном пространстве;
\item Изобразите проекцию $X$ на $\Lin \{e\}$, обозначим её $\hat X$;
\item Изобразите проекцию $X$ на $\Linp \{e\}$, обозначим её $\hat X^{\perp}$;
\item Выпишите явно вектора $\hat X$ и $\hat X^{\perp}$, и найдите их длины;
\item Сформулируйте теорему Пифагора для нарисованного прямоугольного треугольника;
\item Изобразите на рисунке такой угол $\alpha$, что обычная $t$-статистика,
используемая при построении доверительного интервала для $\mu$, имела бы вид
$t = \sqrt{n-1} \cdot \ctg \alpha$.
\end{enumerate}

\item Исследователь Михаил предполагает, что все виды медведепришельцев встречаются
равновероятно. Отправившись на охоту в район Малой Медведицы Михаил поймал двух
лиловых кальмаромедведей, одного двурога медведеспинного и одного медведезавра
ящероголового.

Помогите Михаилу оценитель общее количество видов медведепришельцев с помощью метода
максимального правдоподобия.

\item Помотавшись по просторам Вселенной Михаил изменил своё мнение. Никто кроме
кальмаромедведей, двурогов и медведезавров не попадается, однако попадаются они
явно с разной вероятностью. Из 300 отловленных пришельцев оказалось 150 кальмаромедведей,
100 двурогов и 50 медведезавров. Михаил считает, что медведепришельцы встречаются
независимо, $p_1$ — вероятность встретить кальмаромедведя, $p_2$ — двурога.
\begin{enumerate}
\item Оцените вектор $p = (p_1, p_2)$ методом максимального правдоподобия;
\item Оцените ковариационную матрицу $\Var(\hat p)$;
\item Оцените дисперсию $\Var(\hat p_1 - \hat p_2)$;
\item Постройте доверительный интервал для разницы долей $p_1 - p_2$.
\end{enumerate}

\item Винни-Пух лично измерил количество мёда (в кг) на 100 деревьях и обнаружил, что $\bar X = 10$ и $\hat\sigma^2 = 4$.
По мнению Кролика, состоятельная оценка для параметра $\alpha$ правильности мёда имеет вид $\hat \alpha = \bar X + \sqrt{\bar X + 6}$.

\begin{enumerate}
\item «Халява, сэр!» Найдите точечную оценку параметра $\alpha$;
\item Найдите 95\%-ый доверительный интервал для $\alpha$, симметричный относительно
$\hat\alpha$.
\end{enumerate}

\item Фотографы Андрей и Белла независимо друг от друга пытаются фотографировать кадьяков.
Андрею удаётся сфотографировать одного кадьяка в неделю с вероятностью $0.5$, а Белле — с вероятностью $p$,
независимо друг от друга и от прошлого.
За 100 недель они вместе сфотографировали 130 кадьяков.

\begin{enumerate}
\item Оцените $p$ и постройте 95\%-ый доверительный интервал для $p$;
\item Оцените $p$ и постройте 95\%-ый доверительный интервал для $p$,
если дополнительно известно, что один фотограф опередил другого на 10 фото.
\end{enumerate}
\end{enumerate}

\textbf{Просто красивая задачка}. Эту задачу \textbf{не нужно} решать на кр :)

Медведю Мишутке никак не удаётся заснуть в берлоге, и потому он подбрасывает
правильную монетку $n$ раз.
Обозначим вероятность того, что ни разу не идёт двух решек подряд буквой $q_n$.

\begin{enumerate}[label=\asbuk*)]
\item Найдите $2^8q_8$ и \textbf{назовите} это число;
\item Найдите $\lim 2q_{n+1}/q_n$ и \textbf{назовите} это число.
\end{enumerate}


\newpage
\subsection[2016-2017]{\hyperref[sec:sol_kr_03_ip_2016_2017]{2016-2017}}
\label{sec:kr_03_ip_2016_2017}

Главная мораль: байесовский подход — это всего лишь формула условной вероятности.

\begin{enumerate}
\item Задача о целебных лягушках :)

У одного вида лягушек самки обладают целебными свойствами. Самцы и самки
встречаются равновероятно. Неподалёку видны аж две лягушки данного вида, но
издалека неясно кто.

Определите вероятность того, что среди этих лягушек есть хотя бы одна целебная,
в каждой из ситуаций:
\begin{enumerate}
\item Самцы квакают, самки — нет, со стороны лягушек слышно кваканье, но не разобрать,
одной лягушки или двух.
\item Самцы и самки квакают по разному, но одинаково часто. Только что послышался
отдельный квак одной из лягушек и это квак самца.
\end{enumerate}

\item Яичный бой

Саша и Маша играют в «яичный бой». Перед ними корзина яиц. В начале боя они берут
по одному яйцу и бьют их острыми концами. Каждое яйцо в корзине обладает своей «силой»,
все силы — разные. Более сильное яйцо разбивает более слабое. Внешне яйца не отличимы.
Сила яйца не убывает при ударах. Разбитое яйцо выбрасывают, побеждённый берёт новое,
а победитель продолжает играть прежним.

Какова вероятность того, что Маша победит в 11-ом раунде, если она уже победила
10 раундов подряд?

\item Классика жанра

Перед нами определение бета-распределения $Beta(\alpha, \beta)$:
\[
f(x) \propto \begin{cases}
x^{\alpha-1}(1-x)^{\beta-1}, \text{ если } x\in[0;1] \\
0, \text{ иначе.}
\end{cases}
\]

Блондинка Анжелика хочет оценить неизвестную вероятность встретить динозавра, $p$.
Она предполагает, что динозавры встречаются каждый день независимо от других с
постоянной вероятностью. Априорно Анжелика считает, что неизвестное $p$ имеет
бета-распределение $Beta(2, 3)$. За 20 дней Анжелика 5 раз видела динозавра.
Для краткости обозначим вектором $y$ все имеющиеся наблюдения. Величина $y_i$ —
результат $i$-го дня: 1, если динозавр встретился, и 0 иначе.
\begin{enumerate}
\item Чему, по-мнению Анжелики, равны априорные $\E(p)$, мода распределения $p$?
\item Найдите апостериорное распределение $f(p|y)$.
\item Найдите апостериорные ожидание $\E(p|y)$ и моду.
\item Найдите условное распределение $y_{21}$ с учётом имеющихся данных.
\end{enumerate}

\newpage
\item Рассмотрим следующий код в \verb|stan|.

  \begin{minted}[mathescape,
                 linenos,
                 numbersep=5pt,
                 frame=lines,
                 framesep=2mm]{stan}
  data {
    int<lower=1> N_x;
    int<lower=1> N_y;
    real y[N_y];
    real x[N_x];
  }
  parameters {
    real mu_x;
    real mu_y;
    real<lower=0> sigma_x;
    real<lower=0> sigma_y;
  }
  model {
    for (n_x in 1:N_x) {
      x[n_x] ~ normal(mu_x, sigma_x);
    }
    for (n_y in 1:N_y) {
      y[n_y] ~ normal(mu_y, sigma_y);
    }
    mu_x ~ normal(0, 100);
    mu_y ~ normal(0, 100);
    sigma_x ~ exponential(50);
    sigma_y ~ exponential(50);
  }
  generated quantities {
    delta = mu_x - mu_y;
    ratio = sigma_x / sigma_y;
  }
  \end{minted}

\begin{enumerate}
\item Выпишите предполагаемую модель для данных.
\item Выпишите априорное распределение.
\item Байесовский интервал для каких величин позволяет построить данный код?
\item Какие предпосылки мешают применить в данном случае классический доверительный
интервал для разности математических ожиданий, основанный на $F$-распределении?
\end{enumerate}

\item Просто красивая задача про выборку :)

Есть неизвестное количество чисел. Среди этих чисел одно число встречается строго
больше 50\% раз. Ведущий показывает числа исследователю Акану в некотором порядке.
Когда все числа закончатся, ведущий скажет «всё». Задача Акана — определить, какое
число встречается чаще всех. Проблема в том, что Акан так готовился к контрольной
по теории вероятностей, что устал. И больше 10 чисел запомнить не в состоянии.

Предложите алгоритм, которой позволит Акану определить искомое число.
\end{enumerate}
