% !TEX root = ../probability_hse_exams.tex
\newpage
\thispagestyle{empty}
\section{Контрольная работа 3. ИП}

\subsection[2022-2023]{\hyperref[sec:sol_kr_03_ip_2022_2023]{2022-2023}}
\label{sec:kr_03_ip_2022_2023}

Основную часть писали 70 минут, очно, 2023-03-20. Минимум писали на семинарах. 

\begin{enumerate}
  \item Вася тратит на обед время $Х$, которое хорошо описывается равномерным распределением на отрезке $[ \theta, 2 \theta ]$. 

  \begin{enumerate}
  \item (3) Методом моментов, используя первый момент, найдите оценку $\theta$.
  \item (3) Проверьте, будет ли эта оценка несмещённой?
  \item (4) Проверьте, будет ли эта оценка состоятельной?
  \item (10) Маша утверждает, что оценка $\tilde{\theta_2} = \min⁡\{X_1, \ldots, X_n\}$ эффективнее, чем оценка метода моментов, найденная с помощью первого момента. 
  Проверьте утверждение Маши для оценок, построенных по выборке из двух наблюдений.
  \end{enumerate}
  
  \item Служанки со всего двора используют метод максимального правдоподобия для 
  гадания про своих барышень. 
  Количества мурчаний кота в очередной день марта, $X_i$ —
  независимые случайные величины с распределением заданным формулой $\P(X_i = k) = (1-p)^{k-1}p$,
  где $p$ — вероятность того, что муж окажется богатым. 
  
  \begin{enumerate}
  \item  (8) Найдите оценку максимального правдоподобия для $p$.
  \item  (4) Вычислите информацию Фишера о $p$, содержащуюся во всей выборке.
  \item  (2) Найдите асимптотическую дисперсию оценки $p$.
  \item  (2) Найдите оценку максимального правдоподобия для асимптотической дисперсии оценки $p$.
  \item  (4) Постройте асимптотический доверительный интервал для $р$.
  \item  (3) Точечно и интервально оцените вероятность того, что муж окажется богатым, если за март было 10 дней, в которые кот мурлыкнул ровно 1 раз, 15 дней, в которые кот мурлыкнул ровно 2 раза,
  и 5 дней, в которые кот мурлыкнул \textit{не менее} 3-х раз. 
  \end{enumerate}
  
\item  Евгений опросил жителей Москвы и Санкт-Петербурга,
  какой из трёх видов отдыха они предпочитают: прогулки, чтенье, сон глубокий. 
  Из 300 опрошенных москвичей прогулки предпочитают 100 человек, чтенье — 50 и сон глубокий — 150 человек.
  Из 200 опрошенных санкт-петербуржцев прогулки предпочитают 80 человек, чтение — 70 и сон глубокий — 50 человек. 
  
  \begin{enumerate}
      \item (5) Постройте 95\% асимптотический доверительный интервал для доли москвичей, предпочитающих прогулки. 
      \item (8) Постройте 95\% асимптотический доверительный интервал для разницы  доли москвичей и доли петербуржцев, предпочитающих прогулки.
      \item (12) Постройте 95\% асимптотический доверительный интервал для разницы  доли москвичей, предпочитающих прогулки, и 
      доли москвичей, предпочитающих сон глубокий.
      \item (7) Сколько человек нужно было опросить Евгению, чтобы с вероятностью не менее 0.9 быть уверенным, что доли москвичей в выборке отличаются от их истинных значений не более, чем на 0.01? 
  \end{enumerate}
  
\end{enumerate}


Изначальный проект был приурочен к 190 лет выхода первого издания «Евгения Онегина» и имел вид:
\vspace*{0.5cm}

\begin{minipage}{0.6\textwidth}
  \begin{quote}
      И Ленский пешкою ладью \\
      Берет в рассеянье свою. 
  \end{quote}
  \begin{flushright}
      \textit{Александр Пушкин, Евгений Онегин}
  \end{flushright}
  \end{minipage}
  
  
  \begin{enumerate}
  
  
  \item В течении 10 ночей снится чудный сон Татьяне. Ей снится будто бы она идёт по снеговой поляне, 
  печальной мглой окружена, а из сугроба является большой взъерошенный медведь. 
  Размеры медведей $X_i$ независимы и имеют функцию плотности 
  \[
  f(x) = \begin{cases}    
  (\theta + x)/(\theta + 1.5), \text{ если } x \in [1;2], \\
  0,  \text{ иначе}.
  \end{cases}
  \]
  Помогите Татьяне оценить интенсивность хандры суженого $\theta$ с помощью метода моментов. 
  
  \item Служанки со всего двора используют метод максимального правдоподобия для 
  гадания про своих барышень. 
  Количества мурчаний кота в очередной день марта, $X_i$ —
  независимые случайные величины с распределением заданным формулой $\P(X_i = k) = (1-p)^{k-1}p$,
  где $p$ — вероятность того, что муж окажется богатым. 
  
  Оцените вероятность того, что муж окажется богатым, если за март было 10 дней, 
  в которые кот мурлыкнул ровно 1 раз, 15 дней, в которые кот мурлыкнул ровно 2 раза,
  и 5 дней, в которые кот мурлыкнул \textit{не менее} 3-х раз. 
  
  \item Какое минимальное количество робертов нужно сыграть игроку, чтобы 
  с вероятностью не менее 80\% быть уверенным, что доля выигранных партий в выборке 
  отличается от его истинной вероятности выиграть не более, чем на $0.01$?
  
  \item Евгений каждый свой вечер проводит у Лариных $X_i$ минут. 
  Величины $X_i$ независимы и одинаково распределены. 
  
  Евгений использует необычную оценку 
  \[
  \hat\mu = \frac{\sum_{i=1}^n X_i}{n+1}
  \]
  для неизвестного параметра $\mu = \E(X_i)$. 
  
  \begin{enumerate}
      \item Является ли эта оценка несмещённой? состоятельной?
      \item При каком условии на $\mu$ и $\sigma^2 = \Var(X_i)$ эта
      необычная оценка будет иметь среднеквадратичную ошибку $MSE = \E((\hat \mu - \mu)^2)$ меньше, чем у классической $\hat\mu=\bar X$?
  \end{enumerate}
  
  
  
  \item Евгений опросил жителей Москвы и Санкт-Петербурга,
  какой из трёх видов отдыха они предпочитают: прогулки, чтенье, сон глубокий. 
  Из 300 опрошенных москвичей прогулки предпочитают 100 человек, чтенье — 50 и сон глубокий — 150 человек.
  Из 200 опрошенных санкт-петербуржцев прогулки предпочитают 80 человек, чтение — 70 и сон глубокий — 50 человек. 
  
  \begin{enumerate}
      \item Постройте 95\% асимптотический доверительный интервал для доли москвичей, предпочитающих прогулки. 
      \item Постройте 95\% асимптотический доверительный интервал для разницы  доли москвичей и доли петербуржцев, предпочитающих прогулки.
      \item (*) Постройте 95\% асимптотический доверительный интервал для разницы  доли москвичей, предпочитающих прогулки, и 
      доли москвичей, предпочитающих сон глубокий. 
  \end{enumerate}
  
  
  \end{enumerate}
  
  \begin{flushright}
    \textit{vale}
\end{flushright}


\subsection[2018-2019]{\hyperref[sec:sol_kr_03_ip_2018_2019]{2018-2019}}
\label{sec:kr_03_ip_2018_2019}


\begin{enumerate}
\item Маша подбрасывает правильную монетку три раза. Величина $X_i$ равна единице,
если в $i$-м броске выпал орёл,и нулю иначе. Определим также суммы $S_2 = X_1 + X_2$
и $S_3 = X_1 + X_2 + X_3$.

Рассмотрим геометрию, порождаемую скалярным произведением $\langle L, R\rangle = \Cov(L, R)$.



\begin{enumerate}
  \item Какие величины из набора $X_1$, $X_2$, $X_3$, $S_2$, $S_3$ ортогональны?
  \item Приведите пример любой непостоянной случайной величины, лежащей в $S_3^{\perp}$, ортогональном дополнении к $S_3$.
  \item Выразите через $X_1$ и $S_3$ проекцию $X_1$ на $S_3$.
  \item Выразите через $X_1$ и $S_3$ проекцию $X_1$ на $S_3^{\perp}$.
\end{enumerate}

Определение :) Частной корреляцией между величинами $L$ и $R$ при фиксированной величине $M$,
$\pCorr(L, R; M)$, называется корреляция между проекциями $L$ и $R$ на подпространство $M^{\perp}$.

\begin{enumerate}[resume]
  \item Найдите частную корреляцию между $X_1$ и $X_2$ при фиксированной $S_3$.
\end{enumerate}

\item В данном упражнении храбрый Винни-Пух докажет, что нормальное распределение
обладает максимальной энтропией среди всех распределений с заданным ожиданием и дисперсией.

\begin{enumerate}
  \item Помогите медведю с опилками в голове определить, что больше, $\ln t$ или $t-1$?
  \item У Винни-Пуха есть две функции плотности, $q(x)$ и $p(x)$.
  Подставив в найденное неравенство вместо $t$ отношение плотностей докажите, что
\[
 CE_p(q) = - \int_{-\infty}^{\infty} p(x) \ln q(x) \; dx \geq  - \int_{-\infty}^{\infty} p(x) \ln p(x) \; dx = H(p)
\]
\item Помогите Винни-Пуху вспомнить формулу плотности $q(x)$ для нормального распределения.
И найдите энтропию данного распределения, $H(q)$.

\item Для произвольной случайной величины с ожиданием $\mu$, дисперсией $\sigma^2$ и плотностью $p(x)$, и для нормальной плотности $q(x)$ найдите кросс-энтропию $CE_p(q)$ и завершите доказательство.

\end{enumerate}

\item
В банке 10 независимых клиентских «окошек». В момент открытия в банк вошло 10
человек. Других клиентов банке не было. Предположим, что время обслуживания одного клиента
распределено экспоненциально с параметром $\lambda$.

Оцените параметр $\lambda$ методом максимального правдоподобия в каждой из ситуаций:
\begin{enumerate}
\item Менеджер записал время обслуживания первого клиента в каждом окошке. Первое окошко
  обслужило своего первого клиента за 10 минут, второе, своего первого, — за 20 минут; оставшуюся
  часть записей менеджер благополучно затерял.
\item Менеджер наблюдал за окошками в течение получаса и записывал время обслуживания первого
  клиента. Первое окошко обслужило своего первого клиента за 10 минут, второе, своего первого,
  — за 20 минут; остальные окошки еще обслуживали своих первых клиентов в тот момент, когда
  менеджер удалился.
\item Менеджер наблюдал за окошками в течение получаса.
 За эти полчаса два окошка успели обслужить своих первых клиентов. Остальные окошки
  ещё обслуживали своих первых клиентов в тот момент, когда менеджер удалился.
\item Менеджер наблюдал за окошками и решил записать время обслуживания первых двух клиентов.
  Первое окошко обслужило своего первого клиента за 10 минут, другое, своего первого, — за 20 минут.
  Сразу после того, как был обслужен второй клиент менеджер прекратил наблюдение.
\end{enumerate}

\item В день метеоролога, 23 марта, 23 метеоролога собрались сыграть в странную игру :)
У каждого из них есть монетка.
Монетка первого метеоролога выпадает орлом с вероятностью $1/2$, второго — $2/3$,
третьего — $3/4$, и так далее.

Метеорологи садятся за круглым столом в случайном порядке и одновременно подкидывают монетки.
Затем каждый смотрит на результаты подбрасываний двух своих соседей.
Если результаты бросков соседей совпадают между собой, то метеоролог покидает игру.
Оставшиеся в игре метеорологи повторяют подбрасывание монеток до тех пор,
пока в игре не останется один метеоролог или вообще никого.

Если в финале остался один метеоролог, то он объявляется Самым Главным Метеорологом Года.

У кого больше шансов стать Самым Главным?

\item Аня и Белла нашли неправильную монетку, что падает на орла с вероятностью $p$.
В первый день они подкинули её 100 раз, во второй — 200 раз, в третий — 400 раз.

Аня запомнила суммарное количество орлов за первые два дня — 120 орлов. А Белла —
суммарное количество за второй и третий день — 300 орлов.
\begin{enumerate}
  \item Найдите оценку $p$ с наименьшей дисперсией.
  \item Оцените дисперсию полученной оценки.
  \item Постройте 95\%-й доверительный интервал для $p$.
\end{enumerate}

\end{enumerate}





\subsection[2017-2018]{\hyperref[sec:sol_kr_03_ip_2017_2018]{2017-2018}}
\label{sec:kr_03_ip_2017_2018}

дата: 2018-03-24

24 марта 2018 года — Комоедица, день пробуждения медведя.

\begin{enumerate}
\item Медведь Михайло-Потапыч уснул в берлоге и ему снится сон про $n$-мерное пространство.
Особенно ярко ему снится вектор $X=(X_1, X_2, \ldots, X_n)$ и вектор $e=(1, 1, 1, \ldots, 1)$.
\begin{enumerate}
\item Изобразите векторы $X$ и $e$ в $n$-мерном пространстве;
\item Изобразите проекцию $X$ на $\Lin \{e\}$, обозначим её $\hat X$;
\item Изобразите проекцию $X$ на $\Linp \{e\}$, обозначим её $\hat X^{\perp}$;
\item Выпишите явно вектора $\hat X$ и $\hat X^{\perp}$, и найдите их длины;
\item Сформулируйте теорему Пифагора для нарисованного прямоугольного треугольника;
\item Изобразите на рисунке такой угол $\alpha$, что обычная $t$-статистика,
используемая при построении доверительного интервала для $\mu$, имела бы вид
$t = \sqrt{n-1} \cdot \ctg \alpha$.
\end{enumerate}

\item Исследователь Михаил предполагает, что все виды медведепришельцев встречаются
равновероятно. Отправившись на охоту в район Малой Медведицы Михаил поймал двух
лиловых кальмаромедведей, одного двурога медведеспинного и одного медведезавра
ящероголового.

Помогите Михаилу оценитель общее количество видов медведепришельцев с помощью метода
максимального правдоподобия.

\item Помотавшись по просторам Вселенной Михаил изменил своё мнение. Никто кроме
кальмаромедведей, двурогов и медведезавров не попадается, однако попадаются они
явно с разной вероятностью. Из 300 отловленных пришельцев оказалось 150 кальмаромедведей,
100 двурогов и 50 медведезавров. Михаил считает, что медведепришельцы встречаются
независимо, $p_1$ — вероятность встретить кальмаромедведя, $p_2$ — двурога.
\begin{enumerate}
\item Оцените вектор $p = (p_1, p_2)$ методом максимального правдоподобия;
\item Оцените ковариационную матрицу $\Var(\hat p)$;
\item Оцените дисперсию $\Var(\hat p_1 - \hat p_2)$;
\item Постройте доверительный интервал для разницы долей $p_1 - p_2$.
\end{enumerate}

\item Винни-Пух лично измерил количество мёда (в кг) на 100 деревьях и обнаружил, что $\bar X = 10$ и $\hat\sigma^2 = 4$.
По мнению Кролика, состоятельная оценка для параметра $\alpha$ правильности мёда имеет вид $\hat \alpha = \bar X + \sqrt{\bar X + 6}$.

\begin{enumerate}
\item «Халява, сэр!» Найдите точечную оценку параметра $\alpha$;
\item Найдите 95\%-ый доверительный интервал для $\alpha$, симметричный относительно
$\hat\alpha$.
\end{enumerate}

\item Фотографы Андрей и Белла независимо друг от друга пытаются фотографировать кадьяков.
Андрею удаётся сфотографировать одного кадьяка в неделю с вероятностью $0.5$, а Белле — с вероятностью $p$,
независимо друг от друга и от прошлого.
За 100 недель они вместе сфотографировали 130 кадьяков.

\begin{enumerate}
\item Оцените $p$ и постройте 95\%-ый доверительный интервал для $p$;
\item Оцените $p$ и постройте 95\%-ый доверительный интервал для $p$,
если дополнительно известно, что один фотограф опередил другого на 10 фото.
\end{enumerate}
\end{enumerate}

\textbf{Просто красивая задачка}. Эту задачу \textbf{не нужно} решать на кр :)

Медведю Мишутке никак не удаётся заснуть в берлоге, и потому он подбрасывает
правильную монетку $n$ раз.
Обозначим вероятность того, что ни разу не идёт двух решек подряд буквой $q_n$.

\begin{enumerate}[label=\asbuk*)]
\item Найдите $2^8q_8$ и \textbf{назовите} это число;
\item Найдите $\lim 2q_{n+1}/q_n$ и \textbf{назовите} это число.
\end{enumerate}


\newpage
\subsection[2016-2017]{\hyperref[sec:sol_kr_03_ip_2016_2017]{2016-2017}}
\label{sec:kr_03_ip_2016_2017}

Главная мораль: байесовский подход — это всего лишь формула условной вероятности.

\begin{enumerate}
\item Задача о целебных лягушках :)

У одного вида лягушек самки обладают целебными свойствами. Самцы и самки
встречаются равновероятно. Неподалёку видны аж две лягушки данного вида, но
издалека неясно кто.

Определите вероятность того, что среди этих лягушек есть хотя бы одна целебная,
в каждой из ситуаций:
\begin{enumerate}
\item Самцы квакают, самки — нет, со стороны лягушек слышно кваканье, но не разобрать,
одной лягушки или двух.
\item Самцы и самки квакают по разному, но одинаково часто. Только что послышался
отдельный квак одной из лягушек и это квак самца.
\end{enumerate}

\item Яичный бой

Саша и Маша играют в «яичный бой». Перед ними корзина яиц. В начале боя они берут
по одному яйцу и бьют их острыми концами. Каждое яйцо в корзине обладает своей «силой»,
все силы — разные. Более сильное яйцо разбивает более слабое. Внешне яйца не отличимы.
Сила яйца не убывает при ударах. Разбитое яйцо выбрасывают, побеждённый берёт новое,
а победитель продолжает играть прежним.

Какова вероятность того, что Маша победит в 11-ом раунде, если она уже победила
10 раундов подряд?

\item Классика жанра

Перед нами определение бета-распределения $Beta(\alpha, \beta)$:
\[
f(x) \propto \begin{cases}
x^{\alpha-1}(1-x)^{\beta-1}, \text{ если } x\in[0;1] \\
0, \text{ иначе.}
\end{cases}
\]

Блондинка Анжелика хочет оценить неизвестную вероятность встретить динозавра, $p$.
Она предполагает, что динозавры встречаются каждый день независимо от других с
постоянной вероятностью. Априорно Анжелика считает, что неизвестное $p$ имеет
бета-распределение $Beta(2, 3)$. За 20 дней Анжелика 5 раз видела динозавра.
Для краткости обозначим вектором $y$ все имеющиеся наблюдения. Величина $y_i$ —
результат $i$-го дня: 1, если динозавр встретился, и 0 иначе.
\begin{enumerate}
\item Чему, по-мнению Анжелики, равны априорные $\E(p)$, мода распределения $p$?
\item Найдите апостериорное распределение $f(p|y)$.
\item Найдите апостериорные ожидание $\E(p|y)$ и моду.
\item Найдите условное распределение $y_{21}$ с учётом имеющихся данных.
\end{enumerate}

\newpage
\item Рассмотрим следующий код в \verb|stan|.

  \begin{minted}[mathescape,
                 linenos,
                 numbersep=5pt,
                 frame=lines,
                 framesep=2mm]{stan}
  data {
    int<lower=1> N_x;
    int<lower=1> N_y;
    real y[N_y];
    real x[N_x];
  }
  parameters {
    real mu_x;
    real mu_y;
    real<lower=0> sigma_x;
    real<lower=0> sigma_y;
  }
  model {
    for (n_x in 1:N_x) {
      x[n_x] ~ normal(mu_x, sigma_x);
    }
    for (n_y in 1:N_y) {
      y[n_y] ~ normal(mu_y, sigma_y);
    }
    mu_x ~ normal(0, 100);
    mu_y ~ normal(0, 100);
7    sigma_y ~ exponential(50);
  }
  generated quantities {
    delta = mu_x - mu_y;
    ratio = sigma_x / sigma_y;
  }
  \end{minted}

\begin{enumerate}
\item Выпишите предполагаемую модель для данных.
\item Выпишите априорное распределение.
\item Байесовский интервал для каких величин позволяет построить данный код?
\item Какие предпосылки мешают применить в данном случае классический доверительный
интервал для разности математических ожиданий, основанный на $F$-распределении?
\end{enumerate}

\item Просто красивая задача про выборку :)

Есть неизвестное количество чисел. Среди этих чисел одно число встречается строго
больше 50\% раз. Ведущий показывает числа исследователю Акану в некотором порядке.
Когда все числа закончатся, ведущий скажет «всё». Задача Акана — определить, какое
число встречается чаще всех. Проблема в том, что Акан так готовился к контрольной
по теории вероятностей, что устал. И больше 10 чисел запомнить не в состоянии.

Предложите алгоритм, которой позволит Акану определить искомое число.
\end{enumerate}


\newpage
\subsection[2015-2016]{\hyperref[sec:sol_kr_03_ip_2015_2016]{2015-2016}}
\label{sec:kr_03_ip_2015_2016}

Правила: 3 часа, всем можно пользоваться, интернетом тоже.

Все семь задач решать вовсе не обязательно, выбирайте любые пять!
При самостоятельной работе можно всем пользоваться! :)

\begin{enumerate}

\item Случайные величины $X_1$, \ldots, $X_n$ независимо и одинаково распределены
с функцией плотности $f(x)=2ax\exp(-ax^2)$ при $x>0$.

По 100 наблюдениям известно, что $\sum X_i = 170$, $\sum X_i^2 = 350$.

\begin{enumerate}
\item Оцените параметр $a$ методом максимального правдоподобия.
\item Оцените дисперсию оценки $\hat a_{ML}$
\item Постройте 95\%-ый доверительный интервал для $a$ с помощью оценки
максимального правдоподобия
\item Оцените параметр $a$ методом моментов
\item Оцените дисперсию оценки $\hat a_{MM}$
\item Постройте 95\%-ый доверительный интервал для $a$ с помощью оценки метода моментов
\end{enumerate}

\item Для того, чтобы люди давали правдивый ответ на деликатный вопрос (скажем,
«Берёте ли Вы взятки?») при опросе используется рандомизация. Вопрос допускает
всего два ответа «да» или «нет». Перед ответом респондент подбрасывает монетку,
и только респондент видит результат подбрасывания. Если монетка выпадет «орлом»,
то респондент отвечает правду. Если «решкой», то респондент отвечает наоборот
(«да» вместо «нет» и «нет» вместо «да»).

Монетка выпадает орлом с вероятностью $0.4$. Из 500 опрошенных 300 ответили «да».

\begin{enumerate}
\item Какова вероятность того, что человек берёт взятки, если он ответил «да» в анкете?
\item Постройте оценку для доли людей берущих взятки
\item Постройте 95\%-ый доверительный интервал для доли людей берущих взятки
\end{enumerate}

\item Винни-Пух хочет измерить высоту Большого дуба, $d$. Для этого Винни-Пух три
раза в случайное время дня измерил длину тени Большого Дуба: $8.9$, $13.2$, $25.2$.

Предположим, что в дни измерений траектория движения Солнца проходила ровно через зенит :)

\begin{enumerate}
\item Найдите функцию плотности длины тени
\item Если возможно, постройте оценку метода моментов
\item Если возможно, постройте оценку метода максимального правдоподобия
\item Где живёт Винни-Пух и какого числа 2016 года он проводил измерения?
\end{enumerate}

\item Встроенный в \verb|R| набор данных \verb|morley| содержит результаты 100
опытов Майкельсона и Морли. В 1887 году они проводили измерения скорости света,
чтобы понять, зависит ли она от направления.

\begin{enumerate}
\item Постройте 95\%-ый доверительный интервал для скорости света
\item Выпишите использованные формулы и алгоритм построения интервала
\item Чётко сформулируйте все гипотезы при которых данный алгоритм даёт корректный результат
\item Накрывает ли построенный доверительный интервал фактическую скорость света?
\end{enumerate}

Полезные команды: \verb|morley|, \verb|help("morley")|, \verb|mean|, \verb|sd|,
\verb|qnorm|, \verb|pnorm|

\item  Исследователь Вениамин дрожащей от волнения рукой рисует прямоугольники
размера $a\times b$. Поскольку Вениамин очень волнуется прямоугольники де-факто
выходят со случайными сторонами $a+u_i$ и $b+v_i$. Случайные ошибки $u_i$ и $v_i$
независимы и одинаково распределены $\cN(0;1)$.

Вениамин нарисовал 400 прямоугольничков и посчитал очень аккуратно площадь каждого.
Оказалась, что средняя площадь равна $1200$ см$^2$, а выборочное стандартное
отклонение площади — $50$ см$^2$. Вениамин считает, что зная только площади
прямоугольничков невозможно оценить оценить каждую из сторон.

Если возможно, то оцените параметры $a$ и $b$ подходящим методом. Если невозможно,
то докажите.

\item На поле $D4$ шахматной доски стоит конь. Ли Седоль переставляет коня наугад,
выбирая каждый возможный ход равновероятно.

Сколько в среднем пройдет ходов прежде чем Ли Седоль снова вернёт коня на $D4$?

\item В «Киллер» играли $n$ человек. После окончания игры, когда были убиты все,
кто может быть убит, встретились два игрока (возможно убитых) и оказалось, что
один убил 5 человек, а другой — 7 человек.

Оцените $n$ подходящим методом.
\end{enumerate}


\newpage
\subsection[2013-2014]{\hyperref[sec:sol_kr_03_ip_2013_2014]{2013-2014}}
\label{sec:kr_03_ip_2013_2014}


\begin{enumerate}
\item Дед Мазай подбирает зайцев. Предположим, что длина левого уха зайца имеет
экспоненциальное распределение с плотностью $f(x)=a\exp(-ax)$ при $x\geq 0$. По
100 зайцам оказалось, что $\sum x_i=2000$.
\begin{enumerate}
\item Найдите оценку $\hat{a}$ методом моментов.
\item Оцените стандартную ошибку $se(\hat{a})$.
\item Постройте 90\%-ый доверительный интервал для неизвестного $a$.
\item На уровне значимости $\alpha=0.05$ проверьте гипотезу $H_0$: $a=15$ против
$a>15$. Найдите точное P-значение.
\end{enumerate}

\item По совету Лисы Волк опустил в прорубь хвост и поймал 100 чудо-рыб. Веса
рыбин независимы и имеют распределение Вейбулла, $f(x)=2\exp(-x^2/a^2)\cdot x/a^2$
при $x\geq 0$. Известно, что $\sum x_i^2=120$.
\begin{enumerate}
\item Найдите оценку $\hat{a}$ методом максимального правдоподобия.
\item Оцените стандартную ошибку $se(\hat{a})$.
\item Постройте 90\%-ый доверительный интервал для неизвестного $a$.
\item На уровне значимости $\alpha=0.05$ проверьте гипотезу $H_0$: $a=1$ против
$a>1$. Найдите точное P-значение.
\end{enumerate}

\item $[$R] Как известно, Фрекен-Бок пьет коньяк по утрам и иногда видит привидения.
За 110 дней имеются следующие статистические данные

\begin{tabular}{@{}lccc@{}}
\toprule
Рюмок               & 1    & 2    & 3    \\ \midrule
Дней с приведениями & $10$ & $25$ & $20$ \\
Дней без приведений & $20$ & $25$ & $10$ \\ \bottomrule
\end{tabular}

Вероятность увидеть привидение зависит от того, сколько рюмок коньяка было выпито
утром, а именно, $p=\exp(a+bx)/(1+ \exp(a+bx))$, где $x$ — количество рюмок, а $a$
и $b$ — неизвестные параметры.

\begin{enumerate}
\item Оцените неизвестные параметры с помощью максимального правдоподобия.
\item На уровне значимости $\alpha=0.05$ помощью теста отношения правдоподобия
проверьте гипотезу о том, что одновременно $a=0$ и $b=0$. В чем содержательный
смысл этой гипотезы? Найдите точное P-значение.
\end{enumerate}

%\item Иванушка-дурачок поймал 500 жар-птиц, взвесил и отпустил. Предположим, что веса жар-птиц независимы и имеют гамма-распределение с функцией плотности $f(x)=\lambda^k x^{k-1}\exp(-\lambda x)/ \Gamma(k)$ при $x\geq 0 $. Известно, что $\sum x_i=900$, a $\sum \ln x_i =200$. Логарифм гамма-функции, $\ln \Gamma(k)$, реализуется в R командой \verb|lgamma(k)|.
%\begin{enumerate}
%\item Оцените параметры гамма-распределения с помощью максимального правдоподобия.
%\item На уровне значимости $\alpha=0.05$ помощью теста отношения правдоподобия проверьте гипотезу о том, что одновременно $k=2$ и $\lambda=1$.
%\end{enumerate}

\item Кот Васька поймал 5 воробьев, взвесил и отпустил. Предположим, что веса
воробьев независимы и имеют нормальное распределение $N(\mu,\sigma^2)$. Известно,
что $\sum x_i=10$ и $\sum x_i^2=25$.
\begin{enumerate}
\item Постройте 90\% доверительный интервал для $\sigma^2$, симметричный по вероятности.
\item $[$R]  Постройте самый короткий 90\% доверительный интервал для $\sigma^2$.
\end{enumerate}

\item Задача о немецких танках. Всего выпущено неизвестное количество $n$ танков.
Для упрощения предположим, что на каждом танке написан его порядковый номер\footnote{В
реальности во время Второй мировой войны при оценке количества танков «Пантера»
выпущенных в феврале 1944 использовались номера колес. Двух подбитых танков оказалось
достаточно, чтобы оценить выпуск в 270 танков. По немецким архивам фактический объем
выпуска оказался равен 276 танков. }. В бою было подбиты 4 танка с номерами 2, 5, 7 и 12.
\begin{enumerate}
\item Найдите оценку общего выпуска танков $n$ с помощью метода максимального правдоподобия.
\item Является ли оценка максимального правдоподобия несмещенной?
\item Является ли максимум из номеров подбитых танков достаточной статистикой?
\item Является ли максимум из номеров подбитых танков полной статистикой?
\item Постройте с помощью оценки максимального правдоподобия несмещенную эффективную о
ценку неизвестного $n$.
\end{enumerate}

\item Гражданин Фёдор решает проверить, не жульничает ли напёрсточник Афанасий,
для чего предлагает Афанасию сыграть 5 партий в напёрстки. Фёдор решает, что в
каждой партии будет выбирать один из трёх напёрстков наугад, не смотря на движения
рук ведущего. Основная гипотеза: Афанасий честен, и вероятность правильно угадать
напёрсток, под которым спрятан шарик, равна $1/3$. Альтернативная гипотеза: Афанасий
каким-то образом жульничает (например, незаметно прячет шарик), так что вероятность
угадать нужный напёрсток равна $1/5$. Статистический критерий: основная гипотеза
отвергается, если Фёдор ни разу не угадает, где шарик.
\begin{enumerate}
\item Найдите уровень значимости критерия.
\item Найдите вероятность ошибки второго рода.
\end{enumerate}

\item $[$R] В службе единого окна 5 клиентских окошек. В каждое окошко стоит очередь.
Я встал в очередь к окошку номер 5 ровно в 15:00, передо мной 5 человек. Предположим,
что время обслуживания каждого клиента — независимые экспоненциальные величины с
параметром $\lambda$. Первый человек с момента моего прихода был обслужен в окошке
1 в 15:05. Второй человек с момента моего прихода был обслужен в окошке 2 в 15:10.
\begin{enumerate}
\item Оцените с помощью максимального правдоподобия параметр $\lambda$
\item Оцените, сколько мне еще стоять в очереди.
\end{enumerate}
\end{enumerate}
